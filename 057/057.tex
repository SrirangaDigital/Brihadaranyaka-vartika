
\section*{baq. a.6, bArx. 4}

\begin{artha}
\footnote[1]{hiMdina maMtarxgaLalilx karxmavAgi gAyatirxya 
pAdagaLanunx madhuvAtA eMba maMtarxda pAdavanunx BUH, BuvaH, savxH 
eMba vAyxhaqtiya oMdoMdu BAgavanunx viMgaDisi avugaLiMda karxmavAgi 
mUru gArxsavanunx BuMjisabeVku. anaMtara gAyatirxyanunx samagarxvAgi 
heVLi. madhuvAtA eMbudanUnx vAyxhaqtiyanUnx oTiTxnalilx heVLi 
elalxvanunx BakiSxsabeVku. maMtarxda viMgaDavanunx hiMdeye 
toVrisidedxVve noVDi||}gAyatirxV maMtarxdiMdalU madhumatiV maMtarxdiMdalU 
vAyxhaqtiyiMdalU oMdoMdu pAdavAgi viMgaDisikoMDu adariMda 
maMthadarxvayxdalilx oMdoMdu gArxsavanunx tinanxbeVku. hiVgeye muMdina 
gArxsagaLalUlx I vidhiyaMte anusarisabeVku||
\end{artha}

\vishaya{`tatasxvituvaRreVNayxmf' eMbudara athaR --}

\begin{artha}
mahAtamxvAda (vAyxpakavAda) savitaq deVvateya savxrUpavu nAvu 
varisuvudakekx dhAyxnisuvudakekx yoVgayxvAda (teVjasusx athavA ananx) 
adanunx dhAyxna mADuvevu eMdu (oMdu pAdada athaR) vAtAH = vAyugaLu 
madhuvinaMte suKakaravAgi biVsutatxve. (eMdu madhuvAtAQutAyate 
eMbudara athaR)
\end{artha}

\vishaya{adanenxV sapxSaTxpaDisuvudu}

\begin{artha}
udevxVgavanunx uMTumADade AnaMdakaravAgi vAyugaLu biVsutatxve. hAgeye 
nadigaLU kUDa oLeLxya amaqtarasavanunx nanagoVsakxra yAvAgalU 
sarxvisutatxve - eMdu (madhukaSxranitxsinadhxvaH eMbudara athaR)
\end{artha}

\vishaya{madhuvanunx surisuvudara Pala - matutx 
`mAdhivxVnaRHsanotxvXVSadhiV' eMbudara athaR --}

\begin{artha}
yAvudariMda nAvu saMtuSaTxrAgi AnaMdisutetxVveyoyeMbudu kaSxranitx 
eMbudara Pala. OSadhigaLU kUDa bahaLa madhuravAda rasavuLaLxvugaLAgi 
namage irali||
\end{artha}

\begin{artha}
hiVge BUHsAvxhA eMdu seVrisi mUru maMtarxvanunx ucacxrisi Ahutiya 
(oMdu tutatxnunx) bAyalilx hAkikoLaLxbeVku. hiVgeye muMdina 
maMtarxgaLalUlx mUru mUrAgi mADi ucacxrisi (eraDane matutx mUrane 
AhutigaLanunx matutx nAlakxne Ahutiyanunx savaRmaMtarxgaLiMda 
ucacxrisi bAyiyalilx hAkikoLaLxbeVku)||
\end{artha}

\vishaya{deVvasayxdhiVmahi eMbudara vAyxKAyxna matutx itara maMtarx 
vAyxKAyxna}

\begin{artha}
deVvasayx = parxkAshavuLaLx savitaqvina manoVharavAda savxrUpavanunx 
nAvu ciMtisuvevu|| (madhunakamitAyxdi maMtArxthaR) - nakatxM = 
rAtirxyu utoVSasaH = hagalugaLU saha madhu = saMtoVSavuMTu mADuvaMteyU 
duHKavanunx kaLeyutAtx irali.\\
`pAthiRvaM rajaH =' I loVkavu madhumatf = saMtoVSakaravAgirali. `dawyxH 
= duyxloVkavu, pitA = pitaqloVkavU naH = namage madhu = 
saMtoVSakaravAgali, BuvaHsAvxhA' eMdu eraDane Ahutiyanunx bAyalilx 
hAkikoLaLxbeVku||
\end{artha}

\vishaya{gAyatirxya mUrane pAdavanunx viMgaDisi athaR mADutAtxre --}

\begin{artha}
yAva savitaq deVvateyu udayisutAtx budidhxgaLanUnx 
jAcnxneVMdirxyagaLanUnx perxVreVpisuvudo, A deVvateyu namamx 
sherxVyasisxgAgi yAvAgalU perxVreVpisali||
\end{artha}

\vishaya{mUrane madhumatiV Qukakxnunx vAyxKAyxnisuvudu --}

\begin{artha}
vanasapxtiyU hAgU caMdarxnU, sUyaRnU namage suKakaranAgali gAvaH = 
kiraNagaLu mAdhimxVH = amaqtakaravAgirali, athavA gAvaH = dikukxgaLu 
suKakaravAgirali||
\end{artha}

\begin{artha}
hiMde heVLida eraDu maMtarxgaLoMdige `savxHsAhA' eMdu mUrane 
Ahutiyanunx bAyalilx hAkikoLuLxvudu. anaMtara elAlx oTATxgi 
\footnote[1]{nAlakxne Ahutiyanunx BakiSxsuvAga modalu gAyatirxyanunx 
anaMtara madhumatiV QukukxgaLanunx pUtiRyAgi heVLi `ahameVveVdaM 
savaRM BUyAsamf' eMdu heVLi. BUBuRvaHsavxHsAvxhA eMdu koneyalilx heVLi 
BakiSxsabeVkeMdu athaRvu.}gAyatirxVmaMtarxvanunx, madhumatiVmaMtarxvanunx, mUru 
vAyxhaqtigaLanunx oTuTx ucacxrisi nAlakxne Ahutiyanunx hiMdinaMte 
bAyalilx hAkikoLaLxbeVku. Adare\footnote[2]{nAlukx gArxsagaLanunx 
tegedukoMDare Ahutidarxvayxvu elalxvU mugiyuvaMte tegedukoLaLxbeVku. 
hAge modale aMdAju mADirabeVku|| anaMtara pAterxyalilx leVpisidudx 
uLididadxrU A pAterxyanunx elalx toLedu adanunx tAne maMtarxvilalxde 
sumamxne kuDiyabeVku|| anaMtara eraDu keYgaLanunx toLedukoMDu 
shudidhxgAgi Acamana mADi aginxya pashicxma BAgadalilx malagabeVkeMdu 
aBipArxyavu||}AvAga ``aha meVveVdaM 
(jagatf) savaRMBUyAsamf" eMdu heVLutAtx `BUBuRvaHsavxH sAvxhA' eMdu 
heVLi nAlakxne Ahutiyanunx BakiSxsabeVku||
\end{artha}

\vishaya{``anatxta A ca mayx" - eMbudara athaR --}

\begin{artha}
samxqqti parxsidadhxvAda Acamanavanunx koneyalilx shudidhxgoVsakxra 
mADabeVkeMdu heVLide. anaMtara aginxya hiMBAgadalilx pUvaR dikikxge 
taleyaninxTuTxkoMDu malagabeVku||
\end{artha}

\vishaya{pArxtarAditayx mupatiSaThxteV - eMbudara athaR --}

\begin{artha}
beLagina JAvadalilx anaMtara edudx BakitxyiMda AditoyxVpasAthxna 
mADabeVku (pArxtaHsaMdhAyxvaMdane mADi I upasAthxna mADabeVku) muMde 
heVLuva maMtarxdiMda yAvAgalU (udayakAladalilx) sUyaRnanunx 
upAsisabeVku||
\end{artha}

\vishaya{`\stext' eMdu upasAthxnada maMtarx, adara athaR}

\begin{artha}
`Eka puMDariVkamf' eMbalilx EkatavxveMdare pArxdhAnayxveMbudeV 
udidxSaTxvAgide. EkashabadxdiMda saMKeyx udidxSaTxvAgilalx. udA:- 
``EkaH shevxVtavAhanaH" eMbuvaMte.
\end{artha}

\vishaya{oTuTx maMtArxthaR --}

\begin{artha}
eleY AditayxneV niVnu heVge dikukxgaLige parxdhAnanAgididxVyoV hAgeye 
nAnU manuSayxralilx puruSana BoVga sAdhanagaLiMda samaqdadhxnAgi 
ninanx anugarxhadiMda parxdhAnanAguvenu||
\end{artha}

\begin{artha}
(Aditayxna upasAthxnada naMtara horage) yAva sathxLadiMda hoVgidadxno 
hAgeye baMdu aginxya hiMBAgadalilx kuLitu parxyatanxpUvaRkavAgi 
samAdhAnavAgidudx (EkacitatxnAgi) `taMheYtamf' eMba 
vaMshamaMtarxvanunx BakitxyiMda japisabeVku||
\end{artha}

\section*{baq. a.6, bArx. 3, kaMDike 12}

\stext

\stext

\stext

\stext

\stext

\stext

\vishaya{ta\c  heYtavf itAyxdi maMtarxda vAyxKAyxna}

\begin{artha}
AmeVle (pArxNoVpAsaneya naMtara) adeV I manadhx kamaRvanunx 
parxyatanxpUvaRka heVLidadxnunx udAdxlakana maga budidhxvaMtanAda 
(AruNiyu) yAjacnxvalakxyXnige heVLi punaH AshacxyaRpaTuTx 
maMthakamaRvanunx udedxVshisi heVLidanu||
\end{artha}

\vishaya{Enu heVLidaneMdare --}

\begin{artha}
maMthadarxvayxvananx (BakaSxNakAkxgi saMsakxrisidadxnunx) yAru oNagida 
moVTumarada meVlU celulxvaro, adara shAKegaLu huTiTxye huTuTxvavu, 
elegaLU ciguruvavu||
\end{artha}

%% shloka footnotes
\begin{artha}
\footnote[1]{hiMdeye eMdare - `audumabxreV camaseV kaMseVvA' eMbuvalilx}\\
\footnote[2]{maMthadarxvayxvanunx atitxmarada pAterxyalilx hAki mosaru 
modalAda mUru darxvayxgaLanunx (mosaru, madhu, tupapx ivugaLanunx) 
seVrisi oMdu mathana mADuva kaDagoVliniMda kaDedu madhayxdalilxrisi 
auduMbarada surxvadiMda nitAyxjAyxhutigaLanunx koTuTx anaMtara 
AhutigaLalilx saMpAtavanunx mADabeVkeMdu heVLidAdxgide.}\\
I vacanakekx yAva athaRvo adeV athaRvu muMdina vAkayxgaLalUlx iruvudu. 
`caturawdumabxroBavati' - eMba kaMDikeya athaRvanunx hiMdeye 
vAyxKAyxnisidAdxgide. suKavAgi tiLiyuvudariMda tAnAgiye 
tiLidukoLaLxbahudu (adariMda punaH vAyxKAyxnisuvudilalx)||
\end{artha}

\section*{baq. a.6, bArx. 3, kaMDike 13}

\footnote[3]{nAlukx auduMbarada yahocnxVpayoVgi vasutxgaLu - surxva, 
camasa, idhamx (samitutx). eraDu mathana mADuva kaDagoVlugaLu I nAlukx 
atitxmaradiMda mADidAdxgirabeVkeMdathaR.}\stext

\begin{center}
itishirxV baqhadAraNayxkoVpaniSadfBASayx vAtiRkadalilx Arane 
adhAyxyadalilx mUraneV bArxhamxNavu
\end{center}

\section*{(baq.6--4--1, 2)\\ baq. a.6, bArx. 4, kaMDike 1}

\stext

\stext

\begin{center}
baqhadAraNayxka nAlakxne bArxhamxNa\\
||dakiSxNAmUtaRyeVnamaH||
\end{center}

\begin{artha}
`ESAmf' eMdu AraMBisi `reVtaH' eMbuva payaRMtara puruSabiVjada 
sutxtiyu vivakiSxtavAgide. `sahaparxjApatiriVkASxMcakerxV' eMbuvalilx 
(yAralilx iTaTx reVtasusx puruSatavxvanunx hoMduvudo) A 
AdhAraBUtavAdudanunx saqSiTxsuvaneMdu AloVcisidaneMdathaRvu||
\end{artha}

\vishaya{adanenxV muMde vAyxKAyxnisuvudu --}

\begin{artha}
yAralilx seVcane mADida puruSaviVyaRvu puruSatavxvanunx hoMduvado A 
bageya AdhAra vasutxvanunx (sitxrXVyanunx) saqSiTxsuveneMdu 
IshavxranAda parxjApati barxhamxnu AloVcisidanu||
\end{artha}

\vishaya{`sasitxrXya \c sasaqjeV'}

%% shloka footnote
\begin{artha}
\footnote[1]{`patishacx patinxVcABavatAmf' eMbuvalilx heVLida 
shatarUpA eMbuva patinxyanunx saqSiTxsidaneMdu athaR||}\\
\footnote[2]{gArxmayxdhamaRveMdare kAmashAsotxrXVkatxvAgiruvaMte 
pashukamaR.}\\
adakekx yoVgayxvAda madhukAMDadalilx heVLidaMtiruva sitxrXVyanunx 
saqSiTxsidanu. saqSiTxmADi anaMtara A sitxrXVyanunx 
gArxmayxdhamaRdiMda (meYthunadiMda) keLaBAgadalilx seVvisidanu||
\end{artha}

\vishaya{`tasAmxtf sitxrXya madha upAsiVta' eMbudara athaR}

\begin{artha}
parxjApatiyu makakxLa utapxtitxgAgi hiMde parxyatanxpUvaRka 
sitxrXVyara keLaBAgavanunx seVvisidanu. adariMda putarxkAmiyAdavanU 
adanunx seVvisabeVku||
\end{artha}

\vishaya{`sa EtaM pArxcnAcxM...... itAyxdi maMtarxda tAtapxyaR'}

%% shloka footnote
\begin{artha}
\footnote[1]{vAjapeVya yAgakekx samAnavAgi I meYthuna kamaRvu 
iruvudeMdu BAvisabeVku. yAgadalilx soVmalateyanunx kuTiTx rasavanunx 
tegeyalu udadxvAda kalalxnunx upayoVgisuvuduMTu. hAgeye I pAshava 
kamaRdalilx A kalilxna sAthxnadalilx tananx jananeVMdirxyavu ideyeMta 
BAvisabeVku, kaThinavAgiruvudariMda hAgeyeV BAvisabeVku.}\\
IvAga soVmABiSava rUpavanunx kalipxsikoLaLxlu heVLuvudu. I tananx 
shishanxvanunx soVmarasavanunx hiMDalu upayoVgisuva kalilxnaMte 
gaTiTxyAgiruvaMte tuMbikoMDanu,
\end{artha}

\vishaya{parxjApatiyu muMdeVnu mADidanu? --}

\begin{artha}
A shishanxvanunx QujuvAgi diVGaRvAgi mADikoMDu heVge soVmarasavanunx 
tegeyuva kalulx iruvudo hAgeye mADikoMDu adanunx sitxrXVyalilx 
(shAsitxrXVya) parxyatanxdiMda oMdAgi seVrisidanu||
\end{artha}

\vishaya{`teVneYnA maBayxsaqjata' eMbudara vAyxKAyxna --}

\begin{artha}
hAgeye iruvaMte BAvisida, kalilxniMda sitxrXVyanunx matetxmatetx 
saMbaMdha mADidanu. I meYthuna kamaRvanunx vAjapeVya yAgaveMdu BAvisi 
upAsane mADuvudu tananx puruSAthaRkAkxgi||
\end{artha}

\vishaya{upAsaneya karxmavanunx shurxtiyu heVLuvudu}

\section*{baq. a.6, bArx. 4, kaMDike 3}

\stext

\begin{artha}
yAva sitxrXVyalilx puruSanu tananx jananeVMdirxyavanunx oLapaDisuvano, 
A sitxrXVya guhayx sAthxnave veVdiyeMdu BAvisabeVku. adu eraDu toDegaLa 
meVliruvaMte BAvisabeVku. adara meVle huTiTxruva roVmagaLe 
daBeRgaLeMdU, adara camaRvanenx etitxna camaRveMdU, eraDu BAgadalUlx 
oLage aDagiruva eraDu mAMsapiMDagaLanunx karxmavAgi soVmarasavanunx 
hiMDuva PalakagaLanAnxgiyU nAvu BAvisabeVku. yoVniya oLagiruva 
parxdeVshavanunx uriyuva aginxyeMdU BAvisabeVku. hiMde heVLida camaRvu 
yoVnideVshada sutatxlU iruvudanunx toVriside. adaroLage iruva 
muSakxgaLeMdare vaqSaNagaLeMdu (mAMsagarxMthigaLeMdu) tiLiyabeVku. I 
riVtiyAda muSakxgaLalilx soVmarasavanunx hiMDuva PalakagaLeMdu 
BAvaneyanunx mADabeVku|| 
\end{artha}

\vishaya{I riVtiyAgi BAvisi mADuva meYthuna kaqtayxvanunx vAjapeVya 
yAgaveMba utatxma BAvane mADuvavanige baruva Pala --}

\begin{artha}
vAjapeVya yAgavanunx mADuva yajamAnanige yAva loVkavu 
parxmANAnusAravAgi siguvudo, matutx vAjapeVyadiMda eSuTx loVkavanunx 
vAyxpisuvano aSeTx loVkavanunx I vAjapeVyoVpAsakanU vAyxpisuvanu||
\end{artha}

\vishaya{meYthuna kamaRda aMgagaLalilx vAjapeVyAMgagaLanunx AroVpi 
BAvisidarU, A kamaRvanunx vAjapeVyaveMdu heVge BAvisuvudu? adara 
sAdaqshayxvilalxvalalx? eMdare --}

\begin{artha}
vAjapeVya yAgadalilx elAlx ananxvanunx bayasuvavanige hadineVLu 
ananxgaLanunx parxjApati deVvatege koDuvaMte sheVKarisalapxDuvavu. 
(`sapatxdasha pArxjApatAyxnf pashUnAlaBeVta' eMdu) A vAjapeVya 
yAgavanunx kataRvayxveMdu vidhisuvudu.
\end{artha}

\begin{artha}
parxkaqta ananxrasavAgiruva reVtasasxnenx ananxveMba AhutiyeMdu yAva 
meYthuna vAjapeVyadalilx BAvisabeVkeMdu iruvudo, A meYthunaveMba 
kamaRdalilx vAjapeVya yAgaveMdu ananxvu eraDaralUlx 
samAnavAgiruvudariMda BAvisabahudu. idanunx vAjapeVyada Palavanunx 
bayasuvavarige heVLiruvudu||
\end{artha}

\vishaya{I upAsanege adhika PalavU iruvudu --}

\begin{artha}
yAva upAsakanu hiMde heVLidaMte sitxrXVya taLaBAgavanunx (vAjapeVya 
BAvaneyiMda) seVvisuvano, avanu sitxrXVyara elalx puNayxgaLanunx 
tananx kaDege eLedukoLuLxvanu|| \stext I riVtiyAgi tiLiyade iruva 
(vAjapeVya BAvaneyilalxdavanu) mADida puNayxvanunx sitxrXVyaru 
eLadukoLuLxvaru||
\end{artha}

\section*{baq. a.6, bArx. 4, kaMDike 4}

\stext

\begin{artha}
aMdare I kamaRvu vAjapeVya yAgaveMdu BAvisi tiLidavanAgi UTa 
mADutatxlU, kuDiyutatxlU sitxrXVgamana mADutatxlU iruvAga anusaMdhAna 
mADutAtx reVtasisxna rUpada AhutigaLanunx elAlx pArxNiyU yAvAgalU 
hoVma mADuvudeMdu kANutatxliruva AruNiyu (A goVtarxdavanu) I meYthuna 
rahasayxda tatavxvanunx heVLidanu||
\end{artha}

\vishaya{AruNiyu Enu heVLidaneMdare? --}

\begin{artha}
jAti mAtarxvAgidudx bArxhamxNaru sitxrXVyariMda apaharisalapxTaTx - 
tamamx shuBalABavuLaLxvarAgi heVLida I vidhiyanunx ariyadavarAgi 
meYthunadalilx AsakatxrAgidudx sAyuvaru||
\end{artha}

\begin{artha}
idara parxyoVjanavu bahaLavAgiruvudariMda idU bahuvAgide. 
savxpanxdalilx athavA ecacxravAgidAdxga kAmiyAda ivanige hecAcxgi 
athavA kamimxyAgi reVtasusx saKxlane hoMdidarU A reVtasasxnunx 
keYyiMda sapxshiRsi anaMtarave aBimaMtirxsabeVku. hAgU punaH 
tegedukoLaLxbeVku (adakekx beVkAda maMtarxvidu)
\end{artha}

\section*{baq. a.6, bArx. 4, kaMDike 5}

\stext

yanemxVdayx itAyxdi maMtarxda athaR - yAva nananx reVtasusx I dina 
keLakekx (nelada meVle) bididxto OSadhigaLalUlx hoVgi seVrito, hAgU 
tananx upAdAna kAraNavAda jaladalUlx hoVgi seVrito, adanunx punaH nAnu 
sivxVkarisuvenu||

\begin{artha}
yAvudanunx ililx aBimashaRnakekx heVLideyo punaH tegedukoLuLxvudakUkx 
adeV maMtarxveMdu tiLiyabeVku. tirugi nananxlilxge A reVtasusx baMdu 
seVrali. punaHsetxVjaH eMbalilx teVjasesxMdare jAcnxnaveMdu 
heVLalapxDuvudu (adaraMte jAcnxnavU tirugi baMdu seVrali.)
\end{artha}

\vishaya{`punaBaRgaH' eMbudara athaR}

\begin{artha}
BagaH eMdare sawBAgayxvu (punaH nananxnanx seVrali) hAgeye aginxyu 
(reVtasisxna rUpadalilx horage hoVdadudx nananxlilxge barali) matutx 
uLida deVvategaLU nananxnunx baMdu seVrali. hAgeye samasatx 
deVvategaLa sAthxnagaLu parxkAshakavAgiruvudariMda aginxgaLeMdu ililx 
opipxde. A aginx modalAda deVvategaLu punaH (BUmiyalilx bididxruva 
nananx reVtasasxnunx) savxsAthxnakekx hoMdisali|| (eMdu maMtArxthaRvu)
\end{artha}

\vishaya{`anAmikAMguSAThxBAyxM' itAyxdi maMtarxda athaR}

\begin{artha}
hebebxraLu matutx anAmikA eMba beraLugaLiMda A tananx reVtasasxnunx 
tegedukoMDu satxnagaLanunx, hububxgaLanunx adariMda leVpisabeVku. 
alalxde satxnagaLa naDuve leVpisabeVku\footnote[1]{shirxVmaMtha 
kamaRvanunx mADi patinxyu QutukAlavanunx barxhamxcayaRdalelx idudx 
niriVkiSxsabeVku. A kAlavu odaguva muMceye atirAgadiMda I puruSana 
ecacxrinalolx savxpanxdalolx reVtasusx saKxlaneyanunx hoMdidare AvAga 
`yanemxVadayxreVtaH.......' I maMtarxvanunx heVLutAtx reVtasasxnunx 
japisabeVku. yAva maMtarxdiMda aBimaMtirxsididxto, adeV maMtarxdiMda 
aMguSaThx anAmikeyeMba eraDu beraLugaLiMda A reVtasasxnunx 
tegedukoLaLxbeVku. anaMtara `\stext' eMdu heVLutAtx tegedukoMDu 
patinxya satxnagaLa meVlU hububxgaLa meVlU satxnagaLa naDumadhayx 
BAgadalUlx leVpisabeVku eMdu oTuTx tAtapxyaR.}||
\end{artha}

\section*{baq. a.6, bArx. 4, kaMDike 6}

\stext

\begin{artha}
reVtasisxna tananx upAdAna kAraNavAda niVrinalilx reVtaH seVcane 
mADuva puruSanu parxmAdadiMda tananxnunx (tananx CAyeyanunx) noVDidedx 
Adare I `mayi teVja iMdirxyavf.......' eMba maMtarxdiMda A niVranunx 
aBimaMtirxsabeVku\footnote[2]{hAgeye kadAcitf reVtasasxnunx seVcisuva 
puruSanu reVtasisxgU mUlakAraNavAda jaladalilx tananx CAyeyanunx 
parxmAdadiMda noVDidare hiVge noVDida pApakAkxgi `\stext' eMdu I 
niVranunx muTiTx japisabeVku. AvAga noVDida pApakekx aparAdhakekx oMdu 
pArxyashicxtatxvAguvudu.}||
\end{artha}

\vishaya{maMtArxthaR --}

\begin{artha}
nananxlilx teVjasusx eMdare vijAcnxnavu. adU reVtasesxMdu ililx 
heVLalapxDuvudu. utatxma putarx saMpatitxge kAraNavAdadxriMda A 
reVtasusx nananxlilx irali eMdu tiLidu maMtarxvanunx japisabeVku. 
hiVgeye teVjaH shabadxdaMte muMdiruva (`iMdirxyaM, yashoVdarxviNaM 
sukaqtamf' eMbuva) payARMtara reVtasesxMdeV athaR mADikoLaLxbeVku||
\end{artha}

\section*{avataraNike}
reVtaHseVcane mADuva puruSanige deVshakAlagaLanunx miVri parxmAdadiMda 
reVtaHsaKxlaneyAdadadxkekx pArxyashicxtatxvanunx Ivarege heVLidAdxyitu.

\vishaya{inunx muMde QutukAladalilx BAyARgamanavanunx heVLalu 
`shirxVhaRvA ESA' itAyxdi maMtarxdiMda BUmikeyanunx raciside. 
adaralilx `maloVdAvxsAH' eMbudara athaRvanunx heVLutAtxre --}

\begin{artha}
sitxrXVyara naDuve I patinxyu lakiSxmXye AgiruvaLu. ivaLige Qutu 
kaLedu nAlakxne dinadalilx maladiMda kUDida niVreyu meVlakekx 
tegeyalapxDuvudu. adariMda A patinxyanunx maloVdAvxsaLeMdu A meYthuna 
kamaR mADuvavaru heVLuvaru.
\end{artha}

\vishaya{I sitxrXVyu lakiSxmXyeMdu heVge heVLidudx?}

\begin{artha}
guNasaMpananxnAda putarxPalavuLaLx puSapxvatiyAgiruva kAraNadiMda 
malavadAvxsavuLaLx (QutusAnxtaLAda)vaLanunx sirige 
kAraNaLAgiruvudariMda shirxV (lakiSxmX)yeMdu shurxti heVLide|| 
\end{artha}

\vishaya{`tasAmxtf' itAyxdi maMtarxda athaR --}

\begin{artha}
QutuvAgi nAlakxne dinadalilx sAnxna mADidadx A patinxyanunx 
upamaMtirxsabeVku. ililx upamaMtarxNaveMdare tananx kaDege tirugalu 
vAkf parxyatanx mADuvudu||
\end{artha}

\begin{artha}
perxVmadiMda karedAgalU I patige Ikeyu avakAshavanunx oMdu veVLe 
koDadidadxre vasatxrX, ABaraNa muMtAda BoVgavasutxgaLiMda tananx 
vashakekx baruvaMte mADikoLaLxbeVku.
\end{artha}

\section*{baq. a.6, bArx. 4, kaMDike 7}

\stext

\begin{artha}
AdarU heVLidarU patinxyu avakAshavanunx koDuvudilalxvAdare Akeyanunx 
balAtakxrisi vasha mADikoLaLxbeVku. kupitanAgi shApavanunx koDalu 
Akeyanunx atikarxmisi saMgama mADabeVku. koVpadiMda `\stext' itAyxdi 
maMtarxdiMda shapisabeVku. patishApadiMda Akeyu putarxhiVnaLAgi 
shiVGarxdalelx tananx vashavAguvaLu||
\end{artha}

\begin{artha}
ninanxnunx shapisuveneMdu heVLi patiyu avaLanunx tananx vasha 
mADikoLaLxbeVku. AvAga Akeyu shApa BayadiMda avakAshavanunx koDuvaLu. 
AvAga anukUlavAgi tAnu naDeyabeVku||
\end{artha}

\vishaya{anukUlavAda AcaraNeyeMdare Enu? --}

\begin{artha}
anaMtara shApada BayadiMda patige iSaTxvAdudanunx AdaradiMda Akeyu 
koDuvaLu. AvAga A oLeLxya patiyu I maMtarxdiMda shApavanunx 
hiMtirugisabeVku||
\end{artha}

\vishaya{`\stext' itAyxdi maMtarxda athaR --}

\begin{artha}
puruSananunx devxVSisuva heMDatiyanunx patiyu kAmisuvudAdare 
`nananxnunx Ikeyu kAmisali' eMdu avaLigAgi I vidhiyanunx AcarisabeVku||
\end{artha}

\section*{baq. a.6, bArx. 4, kaMDike 9}

\stext

\vishaya{muMde heVLuva kamARdhikAravu elalxrigU iruvudilalx - hAgAdare 
yArige?}

\begin{artha}
hiMde heVLida maMthakamaRvanunx Acarisi niyamavanunx AcarisidavanAgi 
muMde heVLuva kamaRgaLalilx kataRvayxvAdadedxlAlx Acarisatakakxdudx||
\end{artha}

\vishaya{adeVneMdare - `tasAyx mathaRmf' itAyxdi maMtArxthaR --}

\begin{artha}
sitxrXVyoVniyalilx tananx liMgavanunx oLapaDisi muKavanunx muKadoDane 
seVrisi upasethxyanunx sapxshiRsi muMde heVLida maMtarxvanunx 
japisabeVku||
\end{artha}

\vishaya{vashiVkaraNa mADuva maMtArxthaR - `aknAgx daknAgxtf' itAyxdi 
maMtarxda modalaneya padada athaR --}

\begin{artha}
eleY reVtasesx? nananx AyAya aMgadiMda niVnu huTuTxve heVgeMdare? - 
nAnu tiMda AhArada pariNAmavAda \footnote[1]{}rasadiMda rakatx itAyxdi 
karxmadiMda shukarxvAgi pariNami hoMdi idariMda niVnu huTuTxve.
\end{artha}

\vishaya{haqdayAdadhi jAyaseV - itAyxdi maMtarxda athaR}

\begin{artha}
haqdayada mUlaka shukarxvanunx parxvahisuva (harisuva) nADiya mUlaka 
niVnu huTuTxve. niVnu aMgagaLa kaSAya rasavAgiruve. auSadhadiMda 
leVpisida bANadiMda hoDeyalapxTaTx heNuNx jiMkeyaMte. nananx 
perxVmavanunx taDedidadxriMda I nananx patinxyanunx nananx vashakekx 
baruvaMte mADu||
\end{artha}

\section*{baq. a.6, bArx. 4, kaMDike 10}

\stext

\begin{artha}
I maMtarxda athaR --
\end{artha}

\begin{artha}
A patiyu yAva BAyeRyanunx kAmisuvano, Akeyu gaBaRvanunx dharisadirali 
eMdu (aBipArxyapaTaTxre muMde heVLuvaMte AcarisabeVku) (hiVge Eke 
aBipArxyapaDuvaneMdare) gaBaRvanunx dharisuvalilx rUpavu nAshavAguvudu 
matutx yawvavxnakUkx hAniyAguvudeMdu I kAraNadiMda hiVge avanu 
bayasuvanu. AvAga avaLa yoVniyalilx tananx liMgavanunx irisi 
muKadoDane muKavanunx seVrisi - anaMtara
\end{artha}

\begin{artha}
modalu pArxNayx = reVcakavanunx mADi keLakekx biDabeVku. anaMtara 
adaralilx biTaTx reVtasasxnunx pArxNavAyxpAradiMda niyamAnusAra 
adhoVmuKavAgi biDuva apAnavAyxpAradiMda adeV dAvxradiMda punaH 
tegedukoLuLxveneMdu aBimAna mADikoLaLxbeVku. AvAga A reVtasusx 
nAshavAguvudu. ide I kamaRda Palavu||
\end{artha}

\vishaya{I riVtiyAgi BAvisi I pAshada kamaRvanunx mADuvavanu `\stext' 
itAyxdi maMtarxdiMda patinxyanunx aBimaMtirxsabeVku.}

%% shloka footnote
\begin{artha}
\footnote[1]{meYthuna mADuva kAladalilx modalu tananx liMgada mUlaka 
BAyeRya yoVniyalilx pArxNavAyuvanunx biTuTx adeV dAvxradiMdale nAnu 
punaH nananx reVtasasxnunx hiMde tegedukoMDidedxVneMdu aBimAna 
mADikoLaLxbeVkeMdu idara athaR.}\\
`\stext' itAyxdi maMtarxvanunx heVLi A patinxyanunx sapxshiRsabeVku. 
maMtArxthaR:- iMdirxyadiMdaleV reVtasisxniMdale ninanx reVtasasxnunx 
nAnu tegedukoLuLxvenu. idariMda patiyu aBimaMtirxsidadx A sitxrXVyu 
reVtasisxlalxdavaLAguvaLu.
\end{artha}

\section*{baq. a.6, bArx. 4, kaMDike 11}

\stext

\vishaya{idara athaR --}

\begin{artha}
patiyu yAva sitxrXVyanunx gaBaRvanunx Ikeyu dharisali eMdu 
apeVkeSxpaDuvano, avanu Akeyanunx udedxVshisi (avaLalilx) 
parxyatanxpUvaRka \footnote[2]{tananx jananeVMdirxyada mUlaka BAyeRya 
jananeVMdirxyadiMda reVtasasxnunx eLedukoMDu putorxVtapxtitxyAgalu 
samathaRvAyiteMdu tiLidu tananx reVtasisxnoDane kUDisi 
adaralilxrisabeVku. ide vAtiRkada apAna shabadxda athaR. 
pArxNavAyxpAraveMdare hiVge mADi anaMtara reVtasesxVcana 
mADabeVkeMbude, adU saha muMde heVLida maMtarxdiMda}apAnavAyxpAravanunx mADi (tananx 
viVyaRdoDane Akeya shoVNatavu kUDuvaMte mADi) maMtarxdiMda 
pArxNavAyxpAravanunx mADabeVku. A maMtarxvidu ``\stext" eMbudu 
maMtarxdiMda nAnu irisuvenu||
\end{artha}

\vishaya{athayasayx jAyAyeY - itAyxdi maMtarxda tAtapxyaR --}

\begin{artha}
\footnote[1]{vivAhavAda naMtaraveMdathaR}anaMtara I \footnote[2]{I 
pashukamaRvanunx heVLuva saMdaBaRdalelx upapatiyAda jArapuruSananunx 
nAshamADabeVkeMdu aBilASepaDuva patiyu I aBicArikaveMba shaturxmAraNa 
kamaRvanunx mADaleMdu, jArapuruSana mAraNakekx idu upAyaveMdu 
tiLisuvudakAkxgi shurxtiyu boVdhiside. shurxtiyalilx 
`sheyxVneVnABicaranf yajeVta' eMbudAgi tiLisida sheyxVnayAgadaMte 
idoMdu sAdhana. Adare mADaleVbeVkeMdu idanunx shAsatxrXvu vidhisilalx. 
EkeMdare - `mAhiMsAyxtf savARBUtAni' eMdu yAva pArxNiyanunx 
vadhisabAradeMdu niSeVdhisiruvudariMda pApakaravAda I hiMseyanunx 
vidhisuva udedxVshavanunx shurxtiyu iTuTxkoMDilalx. 
shaturxmAraNaveVnoV idariMda sididhxsuvudu. hAgeye adara pApadiMda 
narakavU sidadhxvAguvudu. narakavu baMdarU ciMteyilalx. 
shaturxnAshavAdare sAkeMdu taqpitxpaDuva hiMsABilASiyAdavanu idanunx 
tiLiyalu shurxtiyu boVdhiside eMdu tAtapxyaR||}parxsaMgadalilx aBicArikaveMba 
kamaRvanunx heVLuvudu. idoMdu sheyxVnayAgadaMte sAdhanaveMdu (shaturx 
hiMsege sAdhanaveMdu) tiLisuvudakAkxgi. Adare idu kataRvayxveMdu 
shurxtiyiMda vidhisalapxDuvudilalx||
\end{artha}

\section*{baq. a.6, bArx. 4, kaMDike 12}

\stext\footnote[1]{iSaTxM = shawrxtakamaRgaLu} \footnote[2]{sukaqtaM = 
sAmxtaRkamaRgaLu} \footnote[3]{AshA = pArxthaRne} 
\footnote[4]{parAkAshaH = vAcA yAvudanunx mADutetxVneMdu heVLi kamaRNA 
adanunx mADade idudx adanunx niriVkeSx mADuvudeV}

\vishaya{I maMtarxda vAyxKAyxna}

\begin{artha}
anaMtara yAva gaqhasathxna patinxge yArAdarU jAra puruSanobabxnu 
elilxyAdarU idadxre avananunx patiyu roVSadiMda devxVSisuvudAdare 
AvAga I parxtikirxyeyanunx AraMBisabeVku||
\end{artha}

\vishaya{``divxSAyxtf" eMbudu EtakAkxgi?}

\begin{artha}
devxVSa mADada manasusxLaLxvanige I kamaRvu sididhxyanunx 
hoMduvudilalx. adariMda adhikAravanunx tiLisalu `divxSAyxtf' eMdu 
visheVSaNavanunx koTiTxde||
\end{artha}

\vishaya{`AmapAterxV' eMdu Etakekx tegedukoMDide?}

\begin{artha}
`AmapAtarx'dalilx aginxyanunx sAthxpisi eMdu heVLidadxriMda I 
aBicArika kamaRkekx AmapAtarxda yoVgayxteyanenx tiLiside. (aMdare 
hasiyAda maDakeyeV I kamaRkekx yoVgayxveMdu tiLiside.) adu heVgeMdare, 
jArapuruSananunx siVLi hAkuvude I kamaRda PalavAdadxriMda adakekx 
takakxdAda pAterx hasiyAdare siVLuvudeMba athaR saMbaMdhavu baruvudu. 
(adariMda adeV yoVgayx)
\end{artha}

\vishaya{visheVSaNada tAtapxyaR vaNaRneya upasaMhAra --}

\begin{artha}
heVge hasiyAda maNiNxna pAterxyu niVrinalilx kUDale karagi hoVguvudo, 
hAgeye namamxlilxge baMda jAra puruSanu nananx shaturxvAgiralu 
shiVGarxvAgi siVLihoVgali|| 
\end{artha}

\vishaya{`aginxmf' aMdare yAva aginxyanunx ililx tegedukoLaLxbeVku?}

\begin{artha}
`aginxmf' eMdu EkavacanadiMdalU ulilxKayx itAyxdi gamaka liMgadiMdalU 
`Avasathayx' veMba aginxyanunx nideRVshiside, Adare terxVtAginxgaLige 
ililx garxhaNavilalx||
\end{artha}

\vishaya{`parxtiloVmamf' itAyxdi vacanada athaR}

\begin{artha}
kamaRvu parxtikUlavAgiruvudariMda sharaveMba daBeRyanunx 
parxtiloVmavAgi parisatxraNa hAki sAvadhAnavAgi I vidAvxnf 
(pArxNoVpAdakanAgi maMthakamaRvanunx Acarisidavanu) roVSadiMda 
kUDidudx sharaveMba taqNada kaDiDxgaLanunx tupapxdalilx nenesi A 
aginxyalilx hoVma mADabeVku. `mama' itAyxdi maMtarxdiMda jArapuruSana 
doVSavanunx parxkaTapaDisi tavxreyiMda hoVma mADabeVku||
\end{artha}

\vishaya{maMtarx vAyxKAyxna --}

\begin{artha}
nananx savxtAtxgiruva sitxrXVyeMba aginxyalilx yawvavxna modalAda 
kAraNadiMda uriyutitxruvAga adaralilx reVtasesxMba Ahutiyanunx 
yAvudariMda hoVma mADideyo idu ninanx atikarxmaNavAyitu. I 
aparAdhadiMda badukirabeVkeMdu bayasuva ninanx pArxNa apAnagaLanunx 
nAnu seLedukoLuLxveneMdu maMtArxthaR, AdadeV eMbudara muMde PaTf eMba 
pada parxyoVgadiMda sharaveMba taqNada kaDiDxgaLanunx 
(viloVmavAgiruvaMte mADi tupapxdiMda leVpisi) hoVma mADabeVku||
\end{artha}

\vishaya{hAgeyeV itara maMtarxgaLiMdalU hoVma mADabeVku. avugaLa athaR 
saMkeSxVpa --}

\begin{artha}
eleY kAmuka? ninanx hAgU putarxranUnx pashugaLanunx Igale 
sivxVkarisuve - eMdu maMtArxthaR. ililx iSaTxveMdare shawrxtakamaR, 
sukaqtaM eMdare sAmxtaRkamaR||
\end{artha}

\begin{artha}
shawrxta matutx sAmxtaR rUpavAda yAvudeV oMdu puNayxkamaRvanunx hiMde 
mADidadxrU adelalxvanunx nAnu tegedukoLuLxveneMdu (maMtArxthaRvanunx 
tiLidu) koVpadiMda Ahutiyanunx aginxyalilx hAkabeVku||
\end{artha}

\begin{artha}
maMtarxdalilxruva AshA eMbudu pArxthaRne. parAkAshaH eMbudu 
parxtiVkaSxNe. maMtarxvu AdadeV eMbuva payaRMtara, elalx 
maMtarxgaLalUlx ideV riVtiyAgi tiLiyabeVku||
\end{artha}

\vishaya{parxtiVkaSxNaveMdare Enu? --}

\begin{artha}
bAyimAtiniMda mAtarx parxtijecnx mADidadxnunx kirxyeyiMda naDeyisade 
iruvudeV parxtiVkaSxNa. hiVge iciCxsuvudeV parAkAsha eMbudu.
\end{artha}

\vishaya{aBicArakakamaRda Palavanunx nirUpisuvudu --}

\begin{artha}
anaMtara `savA ESa nirinidxrXyaH' itAyxdi vAkayxdiMda Itanige hiMde 
heVLida kamaRda Palavanunx, vacanadiMda heVLida shaturxvige 
niriMdirxya modalAda riVtiyalilx heVLuvudu||
\end{artha}

\vishaya{`tasAmxditAyxdi' vAkayxda avataraNike --}

\begin{artha}
shorxVtirxyanAda vidAvxMsana BAyeRyoDane mADuva meYthunaveMba kamaRvu 
samasatx puruSAthaR lABakUkx loVpavanunx uMTumADuvudeMdu 
vaNiRsidAdxyitu.
\end{artha}

\begin{artha}
idariMda shorxVtirxyana patinxyoDane hAsayxvanunx saha mADalu 
bayasabAradu, meYthunavanAnxdaro visheVSa riVtiyalilx hiMde heVLida 
anathaRvanunx tapipxsikoLaLxlu savaRthA bayasabAradu.
\end{artha}

\vishaya{atheVtAyxdi maMtarxda avataraNike --}

\begin{artha}
Ivarege aBicArika kamaRvanunx pashukamaRda parxsaMgadalilx heVLidudx, 
aSeTx. muKayxvAgi udedxVshisi heVLilalx. inunx muMde putarx saMtAna 
paDeyalu yAvudanunx AraMBisi heVLideyo A kamaRdalilx 
(dhamaRkalApavanunx) visatxrisuvudu||
\end{artha}

\section*{baq. a.6, bArx. 4, kaMDike 13}

\stext

\begin{artha}
maMthakamaRda vidhAnavanunx tiLida yAva gaqhasathxna BAyeRyu yAvAga 
Qutu dhamaRvanunx hoMduvaLo, AvAga avana BAyeRyu mUru rAtirxgaLu 
kaMcina pAterxyiMda (niVranunx kuDiyabAradu) pAnavanunx mADabAradu. 
hAgU UTavanunx adaralilx mADabAradu.
\end{artha}

\vishaya{`ahatavAsAH' idara athaR --}

%% shloka footnote
\begin{artha}
\footnote[1]{}\\
matutx I mUru dinagaLalUlx avaLu QutudoVSadiMda kaluSitavAda 
vasatxrXvuLaLxvaLeMdalalx, AvAgalU shudadhxvAda 
aMtaHkaraNavuLaLxvaLAgirabeVku. Ikeyanunx (sAnxna mADidavaLanenxV 
sAnxna mADadavaLanunx) vaqSalanAgali vaqSaliyAgali yAva riVtiyalUlx 
muTaTxbAradu||
\end{artha}

\begin{artha}
beVre yArobabx pApiyidadxrU Atanu sapxshaR, saMBASaNe muMtAdavugaLiMda 
varxtasathxLAda A shorxVtirxya patinxyanunx keDisabAradu. EkeMdare? 
satavxtarx jananaveMbuva iSaTxPalavu laBisuvudakAkxgi keDisabAradu||
\end{artha}

\vishaya{`sAtirxrAtArxnetxV' eMbudara athaR --}

\begin{artha}
A patinxyu mUru rAtirxgaLu kaLeda meVle (nAlakxne dina) sAnxnavanunx 
mADi shudadhx vasatxrXvanunx uTuTx shuciyAgiruvaLu. AvAga patiyu 
caruvanunx pAkamADalu avaLiMda batatxvanunx kuTiTxsabeVku||
\end{artha}

\section*{baq. a.6, bArx. 4, kaMDike 14}

\stext

\begin{artha}
I maMtarxdalilx shukalx eMdare haLadi baNaNxdavaneMdu tiLiyabeVku. 
athavA balarAmanaMte shukalx (shudadhx) eMdAdarU tiLiyabahudu. 
uLidadadxnunx nAveV suKavAgi vAyxKAyxna mADabahudAdadxriMda tAnAgiye 
tiLidukoLaLxbahudu||
\end{artha}

\vishaya{kiSxVrawdanamitAyxdi maMtArxthaR --}

\begin{artha}
patiyu kiSxVrAnanxvanunx A patinxyiMdale pAkamADisi daMpatigaLibabxrU 
tupapxdoMdige ananxvanUnx BuMjisabeVku. idariMda satupxtarxnanunx 
utApxdisalu A daMpatigaLu savxtaMtarxrAgabalalxru||
\end{artha}

\section*{oTuTx 14ne kaMDikeya tAtapxyARthaR}
yAva gaqhasathxnu tanage beLaLxge iruva maganu huTaTxleMdu bayasuvano 
avanu oMdu veVdavanunx paThisabeVkeMdU matutx nUru vaSaRgaLeMba 
pUNARyusasxnunx hoMdabeVkeMdu bayasuvano avanu tananx heMDatiyiMdale 
kiSxVrAnanxvanunx beVyisi tupapxvanunx seVrisi daMpatigaLibabxrU UTa 
mADabeVku. adariMda aMtaha satupxtarxnanunx paDeyalu samathaRrAguvaru.

\section*{baq. a.6, bArx. 4, kaMDike 15}

\stext

\section*{tAtapxyARthaR}
yAva gaqhasathxnu nanage haLadi baNaNxda maganu athavA hoMbaNaNxda 
maganu huTaTxleMdu bayasuvano avanu eraDu veVdagaLanunx 
paThisabeVkeMdU, matutx pUNARyusasxnunx hoMdaleMdu bayasuvano avanu 
patinxyiMda mosarananxvanunx mADisi tupapxdoMdige ibabxrU 
BuMjisabeVku. adariMda avaribabxru iMtaha putarxnanunx janisalu 
samathaRrAguvaru||

\begin{artha}
hiMde heVLida putorxVtApxdane mADalu hiMde heVLida caruhoVmAdi 
kamaRdiMdale A daMpatigaLu samathaRrAguvaru. athavA shiVGarxvAgi 
putarxnanunx utApxdisalu samathaRrAguvaru||
\end{artha}

\vishaya{veVdAdhayxyanakekx anadhikAriyAda magaLige adara 
jAcnxnavanunx bayasuvudu heVge? `duhitA meV paMDitA' eMbudu heVge 
saMgata? eMdare}

\begin{artha}
`duhitA paMDitA' eMbuvalilx sitxrXVyarige yoVgayxvAda kamaRgaLa 
viSayadalilx aMdare gaqhakaqtAyxdigaLalilx beVkAda pAMDitayxvanenx 
ililx heVLide. Adare veVdAthaRda viSayadalilx heVLidadxlalx||
\end{artha}

\vishaya{vijigiVta itAyxdi maMtarxBAgavanunx vAyxKAyxnisuvudu --}

\begin{artha}
vi shabadxdiMda atayxMta eMdathaRvu gArxhayx. I jagatitxnalilx bahaLa 
hogaLisikoMDavaneMdu `vijigiVtaH' eMbudara athaR. `samitiMgamaH' 
eMbalilx samitiyeMdare vidAvxMsara saBe. adakekx yoVgayxnAdavane 
samitiMgamaH eMbudara athaR||
\end{artha}

\section*{baq. a.6, bArx. 4, kaMDike 18}

\stext

\vishaya{`savARnf veVdAnf' eMbudakekx athaRBeVda --}

\begin{artha}
`tirxVnf veVdAnf' eMdu hiMde parxsutxtavAda mUru saMKeyxyalilx 
savaRshabadx parxyoVgavanunx mADiruvudariMda (ililx 
punarukitxyilalxdaMte) savaRshabadxthaR jAcnxnavAgalu nAlukx 
veVdagaLanunx savaRshabadxdiMda tegedukoLaLxbeVku||
\end{artha}

\vishaya{`mAMsawdanaM' eMba shabAdxthaR --}

\begin{artha}
mAMsa mishirxtavAda akikxyanunx beVyisidalilx A ananxvanunx 
mAMsawdanaveMdu heVLuvaru. `aukeSxVNa' eMbuvalilx ukASx eMdare 
viVyaRvanunx seVcane mADalu shakatxvAda (gaMDu goVvu) vaqSaBa. ade 
savxlapx doDaDxdAgidadxre QuSaBaveMdu `ASaRBeVNa' eMbuvalilx 
heVLalapxTiTxde||
\end{artha}

\vishaya{aukeSxVNa eMbudAgi mAMsakekx visheVSaNavAgiruvadariMda 
goVmAMsavu ililx BoVjayxveMdu athaRvAdiVtu. adu 
parxsidadhxvalalxvaSeTx? eMdare --}

\begin{artha}
loVkaparxsididhxyilalxvAdadxriMda mAMsawdana eMbuvalilx mAMsaveMdare 
kaqSaNxmaqga, athavA ADu ivugaLa mAMsaveMdeV garxhisabeVku. adanenx 
Iga mADuvaru||
\end{artha}

\vishaya{hAgAdare I mAMsakAkxgi kaqSaNxmaqga ADu ivugaLanunx 
vadhisabeVkAguvudu. idu yukatxve? eMdare --}

\begin{artha}
parxyatanxdiMda saMpAdisidare Agabahudu. athavA koMDukoMDAdarU 
mAMsavanunx tarabeVku. hiMseyeV niSidadhxvAgiruvudariMda savxyaM 
pashugaLanunx kolalxbAradu.
\end{artha}

\vishaya{`atha ya iceCxVtf' eMdu aneVkasala maMtarxgaLalilx atha eMdu 
parxyoVgisuvudara udedxVshaveVnu?}

\begin{artha}
atha shabadxvu vikalApxthaRdalilxde. hiMde heVLida kAmAyx 
kamaRgaLalilx iSaTxbaMdaMte yAvudAdaroMdu pakaSxvanunx avalaMbisi 
BoVjana niyamavanunx heVLiruvudu||
\end{artha}

\vishaya{hiMdina dina virxVhigaLa avaGAta saMsAkxra, mArane dina 
iSaTxvAda ananxvanunx BuMjisuvudeMdu vayxvasethxyilalxveMdu 
boVdhisuvaru --}

\begin{artha}
sUyoRVdayavAdoDane alilxMda AraMBisi sAnxnAdi sakala kamaRgaLanunx 
Acarisi hiMde heVLida avaGAta saMsAkxravanunx yatanxdiMda mADi 
(iSaTxvAda ananxvanunx BuMjisabeVku)
\end{artha}

\section*{17, 18 kaMDikegaLa oTuTx tAtapxyAR --}
yAvanu nanage paMDitaLAgiyU pUNARyuSayxdiMda kUDiyU iruva magaLu 
huTaTxleMdu iciCxsuvano avanu eLaLxnanxvanunx mADi tupapxvanunx 
seVrisi BuMjisabeVku. athavA vidavxtf saBege hoVgalu samathaRnU 
susaMsakxqqtavAda athaRvatAtxda manoVharavAda mAtanADuvavanU 
parxsidadhxnU Agiruva veVda catuSaTxyavanunx tiLidu pUNARyuvAda maganu 
tanage huTaTxleMdu bayasuvano avanu hiMde heVLida mAMsa misharxvAda 
ananxvanunx BuMjisabeVku. AvAga I daMpatigaLu iMtaha makakxLanunx 
paDeyalu samathaRrAguvaru.

\section*{baq. a.6, bArx. 4, kaMDike 19}

\stext

\vishaya{idaralilx `sAthxliVpAkAvaqtA' eMbudara athaR --}

\begin{artha}
sAthxliVpAkadalilx yAva kirxyeyu ideyo adu Avaqtf eMdu 
heVLalapxDuvudu. sAthxliVpAkada vidhiyaMte Ajayxvanunx saMsakxrisi 
hAgeye hoVma mADabeVku||
\end{artha}

\begin{artha}
AjayxveMdu heVLidudx beVre caru modalAdavugaLigU upalakaSxNa. (sUcaka) 
athavA Adishabadxkekx loVpaviruvudariMda AjAyxdi eMdeV tiLiyabeVku||
\end{artha}

\vishaya{sAthxliVpAkasayx itAyxdi maMtarxda athaR}

\begin{artha}
AjAyxdi darxvayx saMsAkxravanunx mADida naMtara `aginxyeVsAvxhA' 
itAyxdi sAthxliVpAkada maMtarxdiMda matetxmatetx AvApa mADi AvApa 
mADida sAthxnadalilx nitayxvAda AhutigaLanunx AdaradiMda hoVmavanunx 
mADi (caru hoVmavanunx mADabeVku)||
\end{artha}

\vishaya{kataRvayxvAda Ahuti saMKeyxyanunx heVLutAtxre --}

\begin{artha}
aginx muMtAda deVvategaLanunx modalu mADi koDuva yAva AhutigaLu iveyo, 
avu mUru. sivxSaTxkaqdodhxVma payaRMtara mADi AkamaRvanunx mADi 
adariMda sAthxliyiMda caruvanunx meVlakekx tegedukoMDu samAdhAna 
citatxvuLaLxvanAgi iSATxthaRvanunx anusarisi tupapxdiMda adanunx 
saMsakxrisikoMDu BuMjisabeVku||
\end{artha}

\vishaya{pArxshayx itAyxdi maMtArxthaR}

\begin{artha}
hoVma mADi uLida caruvanunx tAnu BuMjisi uciCxSaTxvAda caruvanunx 
patinxge patiyu koDabeVku.
\end{artha}

\section*{baq. a.6, bArx. 4, kaMDike 19}

\stext

\begin{artha}
parxyatanxdiMda eraDu keYgaLanunx toLedukoMDu shudidhxgAgi 
sAmxtARcamanavanunx mADabeVkeMdu tiLiyuvudu, heVge? pANigaLanunx 
parxkASxLana mADabeVkeMdu heVLidadxriMdale athaRsAmathaRyxdiMdale 
shudidhxgAgi Acamana mADabeVkeMbudu tiLiyuvudu|| 
\end{artha}

\vishaya{`udapAtarxM pUrayitAvx' itAyxdi maMtarxda athaR --}

\begin{artha}
jalapAterxyanunx anaMta tegedukoMDu adaralilx niVriniMda heMDatige 
muMde heVLuva maMtarxdiMda saMtAna lABakAkxgi matetxmatetx 
seVcisabeVku||
\end{artha}

\vishaya{A maMtarx `\stext' itAyxdiyAgide. adara athaR --}

\begin{artha}
vishAvxvasu eMba hesariniMda ililx gaMdhavaRnanunx saMboVdhiside. eleY 
gaMdhavaR, vishAvxvasuH namamx heMDatiyanunx biTuTx meVlakekx edudx 
beVrekaDege hoVgu.
\end{artha}

\begin{artha}
parxpUvARyxmf eMbuvalilx taruNiyAda nAriyu heVLalapxTiTxde. eleY 
vishAvxvasuve? patiyoDane kirxVDisuva parxpUviRVM = puSaTxLAda beVre 
sitxrXVya hatitxra shiVGarx hoVgu. hiMde heVLida kamaRvanunx I 
satiyalilx satupxtarxna utapxtitxgAgi satapxtiyu avashayx 
AcarisabeVku. parxpUvARyxmf eMbuva shabadx sAmathaRyxdiMda aMdare 
(tAruNayxvanunx sUcisuva shabadx baladiMda) taruNiyAda satiyalilx 
QutukAlavu baMdAgalelAlx AcarisabeVku||
\end{artha}

\begin{artha}
nAnAdaro I nananx heMDatiyanunx seVrabeVkeMdiruvenu. adakekx avakAsha 
mADikoDu eMdu gaMdhavaRnanunx horage horaDisi Ikeyanunx hoMduvanu.
\end{artha}

\begin{artha}
iSaTxvAda gaBARdhAnakAkxgi patiyu heMDatiyanunx AlaMgisuvanu, heVge? 
eMdare? `amoVhamasimxsAtarxyXmf' eMba maMtarxvanunx heVLutAtx nAvu 
ibabxrU deVvatA savxrUpavAgidedxVve. BAvisutAtx enunxtAtx 
AlaMgisuvanu||
\end{artha}

\section*{baq. a.6, bArx. 4, kaMDike 20}

\stext

\vishaya{atheYnAmaBipadayxteV - amoV\s ha masimx eMba maMtarxda athaR}
I patinxyanunx aBimaMtirxsi kiSxVrAnanx modalAda ananxvanunx AyAya 
putarxnanunx kAmaneyaMte BuMjisi patiyu AlaMgisuvudu. patiyu seVruva 
kAladalilx nAnu patipArxNavAgidedxVne. A niVnu vAkAkxgididx, heVge? 
vAkukx pArxNAdhiVnavAgidadxriMda niVneV vAkukx, nAneV 
pArxNavAgidedxVne. alalxde nAnu sAmaveVdavAgidedxVne. niVneV 
QukAkxgididxVye, nAnu aMtarikaSx loVka, niVnu paqthiviVloVka. A 
nAvibabxrU saha I namamx udayxmavanunx AraMBisoVNa reVtasasxnunx nAvu 
seVrisoVNa. EtakAkxgi satfputarxna lABakAkxgi eMdu aBipArxya||

\begin{artha}
eleY deVviye? nAvibabxrU I udayxmavanunx mADoVNa. satupxtarxna 
janamxvu sididhxsalu niVnU matutx nAnU seVri yoVniyalilx reVtasasxnunx 
dharisuva.
\end{artha}

\begin{artha}
reVtasasxnunx seVcane mADida Palavanunx shurxtiye gaMDumagananunx 
hoMduvudakAkxgi eMdu heVLide. `puMseputArxya vitatxyeV' eMbalilx 
alalxde `amoV\s ha masimx' eMba maMtorxVcAcxraNeyAda naMtara 
BAyeRyanunx udedxVshisi `vicihiVthAMdAyxnA paqthiviV' eMbudanunx 
heVLabeVku||
\end{artha}

\section*{baq. a.6, bArx. 4, kaMDike 21}

\stext

\vishaya{I maMtarxda vAyxKAyxna AraMBavAgide --}

\begin{artha}
`vijihiVthAmf' itAyxdi maMtarxdiMda patinxya toDegaLanunx biDisuvudu. 
toDegaLanunx udedxVshisi A maMtirxsideyeMdu `vijihiVthA' eMbuvalilx 
tiLiyabeVkAdadudx.
\end{artha}

\vishaya{`vihApayati' eMbudara vuyxtapxtitx, matutx athaR}

\begin{artha}
`vihApayati' eMbudu vi upasagaRviruva ja{hA}ti eMba Ni jf parxtayxyAMta, 
gatayxdhaRda dhAtuvina rUpavu. mUru Pala anuloVmavAgi (taleyiMda 
kAlinavarege) heMDatiyanunx keYyiyxMda maMtarxdiMda anumAjaRne 
mADabeVku. maMtarxveVneMdare? `viSuNxyoRVniM kalapayatu' eMdu 
ucacxrisutAtx anumAjaRne mADabeVku||
\end{artha}

\vishaya{`viSuNxyoRVniM kalapxyatu' eMbalilx kalapxyatu eMbudara athaR 
--}

\begin{artha}
kalapxneyeMdare samathaRne shakitxyuMTu mADuvudu eMdathaR. 
tavxSaTxqqbarxhamxnu nananx magana rUpagaLu. parxtiyoMdu avayavagaLU 
aMdavAgiruvaMte saqSiTxsali||
\end{artha}

\vishaya{`gaBaRMdheVhi si niVvAliV' eMbuvalilx sinivAliV padada 
athaRvanunx heVLuvaru --}

\begin{artha}
amAvAseyxya ahasisxna deVvateye siniVvAli eMdu heVLalapxDuvudu. 
paqthuSuTxke eMbudU adeV deVvate. kAraNaveVneMdare? hecucx 
sutxtiyuLaLx deVvate idu, adariMda
\end{artha}

\begin{artha}
ninage kamalada hAravanunx hAkikoMDiruva ashivx deVvategaLu 
gaBaRvanunx cenAnxgi uMTumADali. ililx sUyaR caMdarxreV ashivx 
deVvategaLeMdu tiLiyabeVku|| 
\end{artha}

\vishaya{parxsidadhxvAda ashivxniV deVvategaLu Eke AgabAradu? eMdare --}

\begin{artha}
tananx kiraNamAleyuLaLxvarAgi A sUyaRcaMdarxreV jagatitxnalilx 
parxsidadhxve Agiruvaru. (adariMda sUyaRcaMdarxreV puSakxra sarxjaw 
ashivxnaw eMdu heVLalapxDuvaru)
\end{artha}

\section*{baq. a.6, bArx. 4, kaMDike 22}

\stext

\vishaya{hiraNamxyiV araNiV eMbudara athaR --}

\begin{artha}
hiraNayxM eMdare amaqta savxrUpavAda joyxVti. tanamxyavAda araNigaLu 
aMdavAgiratakakxvu. A eraDu araNigaLiMda ashivx deVvategaLu hiMde 
amaqtavanunx mathana mADiruvaru||
\end{artha}

\begin{artha}
ashivx deVvategaLu eMtaha gaBaRvanunx parxyatanxdiMda mathana 
mADiruvaru aMtaha rUpavuLaLx gaBaRvanenx nAvu irisoVNa. EtakAkxgi? 
hatatxne tiMgaLalilx adu janisuvudakAkxgi||
\end{artha}

\vishaya{yathA itAyxdi maMtArxthaR --}

\begin{artha}
paqthiviyu heVge aginxye gaBaRvAgivuLaLxdodx, matutx aMtarikaSx 
loVkavu sUyaRneMba iMdarxniMda gaBaRvuLaLxdodx, dikukxgaLige vAyuve 
heVge gaBaRvAgidudx calana kamaRvanunx mADuvudo, hAgeye --
\end{artha}

\vishaya{asaw eMbudakekx eraDathaRgaLu --}

\begin{artha}
tananx hesaranunx ucacxrisi ninage hAgeye gaBaRvanunx irisuvenu eMdu 
`gaBaRM dadhAmiteV' eMdu maMtarxvanunx ucacxrisabeVku, athavA A 
maMtarxvanunx patiyu ucacxrisutAtx heMDatiya hesaranAnxdarU 
garxhisabeVku||
\end{artha}

\section*{baq. a.6, bArx. 4, kaMDike 23}

\stext

\vishaya{I maMtarxda athaR --}

\begin{artha}
hatutx mAsagaLu kaLeda naMtara parxsavisuva BAyeRyanunx `yathA vAyuH' 
itAyxdi maMtarxdiMda melalxne niVriniMda porxVkiSxsabeVku. vAyuvu 
koLavanunx calisuvaMte patiyu calisuvaMte mADuvanu (gaBaRkekx 
upahatiyilalxdeyiruvaMte ADisuvanu)
\end{artha}

\vishaya{ide daqSATxMtadalilx beVkAda aMshavanunx I muMde tiLisutAtxre 
--}

\begin{artha}
heVge gALiyu koLavanunx elelxDe ADisutAtx idadxrU hAniyanunxMTu 
mADuvudilalxvo hAgeye BAyeRge hAniyanunxMTu mADuvudilalx. ninanx 
gaBaRvu suKavAgi calisali||
\end{artha}

\begin{artha}
ninanxnunx suKapaDisutAtx gaBaRciVladoMdige gaBaRvu (shishuvu) horage 
barali varxja eMdare gaBaRda mAgaR. adu hiMde saqSiTxkAladalilx agaRla 
= agaLiyiMda (paDeyiMda) kUDiyeV saqSiTxsalapxTiTxde||
\end{artha}

\vishaya{agaRla matutx avarA eMbudanunx vAyxKAyxnisuvudu --}

\begin{artha}
gaBaRda ciVlave (jarAyu) taDeyAguvudu, eleY iMdarx? niVnu shiVGarxvAgi 
agaRlavanunx oDedu hAku. (dAri mADikoDu), gaBaRvu horage horaTa meVle 
anaMtara yoVniyiMda yAvudu horage baruvudo, A mAMsada mudedxyu 
gaBaRkekx samAnavAgidudx avarA eMdu heVLalapxDuvudu. (adanunx horage 
horaDisu) ililx iMdarx eMdare pArxNavAyu, adanenxV patiyu 
pArxthiRsuvudu||
\end{artha}

(\textbf{soVSayxnitxV mitAyxdi kaMDikeya tAtapxyARthaR-} patiyu 
parxsavisuva heMDatiyanunx niVriniMda parxsavakAladalilx 
suKaparxsavavAgalu I muMde heVLida maMtarxdiMda porxVkiSxsabeVku 
(`yathA vAyuH' itAyxdi maMtarxdiMda) heVge vAyuvu koLavanunx 
nAshapaDisade ADisuvudo hAgeye ninanx gaBaRvu ninage toMdare mADade 
calisali. gaBaRda ciVladiMdoDagUDi gaBaRvu Icege barali, pArxNa eMba 
iMdarxna mAgaRvu saqSiTxkAladalelx jarAyuveMbuva ciVladiMda 
sututxvareyalapxTiTxruvudu. eleY iMdarx? niVnu A taDeyanunx BeVdisi A 
mAgaRdiMda gaBaRdoMdige horage bA. gaBaRvu horage baMda meVle baruva 
mAMsa piMDavanunx horage horaDisu. hiVge patiyu pArxthiRsabeVku||)

\section*{baq. a.6, bArx. 4, kaMDike 24}

\stext

\vishaya{I maMtarxda athaR -- (putarx jananavAda meVle jAtakamaRvanunx 
vidhisutAtxre)}

\begin{artha}
kumAranu huTiTxda meVle anaMtara patiyu A kumArananunx tananx toDeya 
meVle hatitxsikoMDu avasathayxveMba aginxyanunx sAthxpisikoMDu hiMde 
vAyxKAyxnadiMda parxsidadhxvAda athaRvuLaLx `kaMseVpaqSadAjayxmf' eMdu 
heVLida darxvayxvanunx hoVma mADabeVku||
\end{artha}

\vishaya{`paqSavAjayxmf' eMdare --}

\begin{artha}
tupapx matutx mosaru eraDanunx misharx mADidalilx adanunx 
paqSadAjayxveMdu heVLuvudu. adanunx oMdu mADi anaMtara kaMcina 
pAterxyalilx muMde irisikoMDu maMtarxdiMda `paqSadAjayxsoyxVpaGArxtamf 
juhoVti' eMdaMte punaHpunaH savxlapxsavxlapx tegedukoMDu misharxvAda A 
jAyxhutigaLanunx hoVma mADabeVku. ililx `upaGAtamf' eMbuvalilx Namulf 
parxtayxyadiMda punaHpunaH eMdathaRvu laBisuvudu||
\end{artha}

\begin{artha}
yatheVcaCxvAgi I nananx gaqhadalilx sahasarx manuSayxranunx yAvAgalU 
poVSisuvaMte nAnu Aguvenu. sahasarx eMba saMKAyxvAcaka padadiMda 
bahutavxvanunx heVLuvudu. sahasarx padavu bahutavxvanenx boVdhisuvadu||
\end{artha}

\begin{artha}
I riVtiyAgi vaqdidhx hoMdutAtx niVnu putarxsaMtatiyanunx ariyabeVku. 
upasaMdi eMdare saMtati. idaralilx niVnu pashu muMtAdavugaLa 
saMtatiyanunx uMTumADuvanAgu||
\end{artha}

\vishaya{`mayi pArxNA \c satxyXyi' eMba maMtarxda vAyxKAyxnavu 
anavashayxveMdu heVLutAtxre --}

\begin{artha}
sapxSATxthaRvAgideyeMba kAraNadiMda ililx vAyxKAyxnavanunx 
apeVkiSxsuvudilalx. `mayipArxNAnf' eMba garxMthavanunx tAnAgiye 
tiLiyabahudu||
\end{artha}

\textbf{24ne kaMDikeyalilxruva maMtarxgaLa athaR}\\
I nananx gaqhadalilx putarxrUpadiMda vaqdidhx hoMdutAtx nAnu sAvirAru 
manuSayxranunx poVSisuvenu. aneVka manuSayxranunx poVSisuvanaMte 
AgabeVku. I nananx putarxnige saMtatiyalilx parxje athavA pashugaLa 
saMpatutx viciCxnanxvAgadirali|| taMdeyAda nananxlilx yAva pArxNagaLu 
irutatxveyo, avugaLanunx putarxnAda ninanxlilx manasA apiRsuvenu|| 
nAnu kamaRdalilx atireVkavanonx athavA nUyxnateyanonx mADiruvudu 
idadxre adelalxvanunx tiLida sAkiSxBUtanAda yajecnxVshavxranu cenAnxgi 
yAga mADidaMteye aMdare adhikavAgi mADiradaMteyU nUyxnavAgiradeyU 
samavAgiruvaMte mADali||

\section*{baq. a.6, bArx. 4, kaMDike 25}

\stext

\begin{artha}
anaMtara I kumArana balagiviyanunx tananx bAyiya hatitxra irisikoMDu 
vAkf vAkf vAkf eMdu mUrusala heVLabeVku. vAkf eMdare tarxyiV aMdare 
mUru veVdagaLeMdu heVLalapxDuvudu. I tarxyiVrUpavAda vAkukx kiviya 
mUlaka oLage maguvina sherxVyasisxgAgi parxveVshisali. `yasetxsatxnaH' 
- eMba maMtarxdiMda ililx sarasavxtiVdeVviyu 
hogaLalapxDuvaLu\footnote[1]{uLida maMtArxthaR - anaMtara mosaru, 
jeVnutupapx, tupapx I mUranUnx seVrisi hiraNayxdiMda maguvige 
pArxshana mADisabeVku. adakekx I nAlukx maMtarxvanunx heVLabeVku - (1) 
BUH setxVdadAmi, (2) BuvasetxV dadhAmi, (3) savxsetxVdadhAmi, (4) 
BUBuRvaHsavxH savaRM tavxyidadhAmi eMdu koneyalilx heVLi pArxshana 
mADisabeVku. ililx 120ne vAtiRkadiMda Acege naDuve 
vAtiRkavididxrabeVku. 120kekx maMtArxthaRvu pUNaRvAguvudilalxveMbude I 
riVti nAvu Uhisalu kAraNa.} - (putarxnige satxnayxpAnada mUlaka) 
udAravAda guNasaMpatutx uMTAgaleMdu heVLide. maMtarxdalilx shashayaH 
eMdare sashaya eMdu tiLiyabeVku. shaya eMdare kamaRda PalaveMdu 
heVLalapxDuvudu.
\end{artha}

\section*{baq. a.6, bArx. 4, kaMDike 26, 27}

\stext

\stext

\begin{artha}
satxnavu meVlakekx baruva edeya BAgaveV guhA. adanenxV shaya eMdu 
shurxtiyu parxtipAdiside. mayoVBUH eMdare savaRpArxNigaLanunx 
poVSisuva ananxrUpavAda vasutxveMdu heVLalapxDuvudu.
\end{artha}

\vishaya{yoVratanxdhA itAyxdi maMtArxthaR}

\begin{artha}
ratanxdhA = eMdare sherxVSaThxvAda hAlanunx utApxdane mADuvudu matutx 
hAlige AdhAravU deVviya satxnavu AdhAravAgiruvudo matutx vasutxveMba 
dhanavanunx vaqSiTx muMtAdavugaLiMda huTuTxvudanunx hoMdiyU iruvudo, 
alalxde hecucx maMgaLakaravAgiruvudariMda sudatarxveMdU 
heVLalapxDuvudo, matutx yAva satxnadiMda nAvu beVDuva deVvAdi 
pArxNigaLanunx elalxvanunx niVnu poVSisuveyo eleY sarasavxtiV deVviye? 
nananx putarxnige pAnakAkxgi nananx heMDatiya satxnadalilx hiMde 
heVLida guNavuLaLx samasatx iSATxthaRvanunx koDuva A ninanx 
satxnavanunx irisu||
\end{artha}

\section*{baq. a.6, bArx. 4, kaMDike 28}

\footnote[1]{satxnayxpAnavanunx mADisida meVle I meVlina `ilAsi' eMba 
maMtarxdiMda patiyu aBimaMtirxsabeVku - idara athaR - niVnu sutxtige 
pAtarxLAgiruve. meYtArxvaruNi vasiSaThxra patinx aruMdhatiye Agididx. 
viVrapuruSanAda nananxnunx nimitatxmADi niVnu putarxnanunx janisiruve. 
A niVnu bahu putarxvuLaLxvaLAgu. niVne namamxnunx bahuputarx 
saMpananxranAnxgi mADiruve. alalxde putarxnu taMdeyanUnx tAtananunx 
miVrisidavanu. AshacxyaRvidu alalxde siriyiMdalU kiVtiRyiMdalU 
barxhamxvacaRsisxniMdalU paramAvadhiyanunx hoMdiruvanu. yAru I 
riVtiyAgi putarx saMpatitxyanunx paDeda bArxhamxNanige putarxnAgi 
huTuTxvano A taMdeyU saha tananx taMde tAta ivaranenxlAlx 
miVrisiruvanu eMdathaR||}\stext

\begin{artha}
patiyu tAnu ilAsi eMba maMtarxdiMda putarxna tAyiyanunx AdaradiMda 
aBimaMtirxsabeVku. EtakAkxgi? eMdare utatxmavAda kamaR (parxshasatx 
putarxpAlaneyeMba) kamaRkAkxgi aBimaMtirxsabeVku||
\end{artha}

\begin{center}
ililxge baqhadAraNayxkoVpaniSadf BASayx vAtiRkadalilx Arane 
adhAyxyadalilx nAlakxne bArxhamxNavu pUNaRgoMDide.\\
||shirxVdakiSxNAmUtaRyeVnamaH||\\
||namaH shirxVshaMkaraguraveV namaH||
\end{center}

