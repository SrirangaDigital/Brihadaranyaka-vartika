
\section*{baq. a.6, bArx. 4}

\begin{artha}
\footnote[1]{hiMdina maMtarxgaLalilx karxmavAgi gAyatirxya 
pAdagaLanunx madhuvAtA eMba maMtarxda pAdavanunx BUH, BuvaH, savxH 
eMba vAyxhaqtiya oMdoMdu BAgavanunx viMgaDisi avugaLiMda karxmavAgi 
mUru gArxsavanunx BuMjisabeVku. anaMtara gAyatirxyanunx samagarxvAgi 
heVLi. madhuvAtA eMbudanUnx vAyxhaqtiyanUnx oTiTxnalilx heVLi 
elalxvanunx BakiSxsabeVku. maMtarxda viMgaDavanunx hiMdeye 
toVrisidedxVve noVDi||}gAyatirxV maMtarxdiMdalU madhumatiV maMtarxdiMdalU 
vAyxhaqtiyiMdalU oMdoMdu pAdavAgi viMgaDisikoMDu adariMda 
maMthadarxvayxdalilx oMdoMdu gArxsavanunx tinanxbeVku. hiVgeye muMdina 
gArxsagaLalUlx I vidhiyaMte anusarisabeVku||
\end{artha}

\vishaya{`tatasxvituvaRreVNayxmf' eMbudara athaR --}

\begin{artha}
mahAtamxvAda (vAyxpakavAda) savitaq deVvateya savxrUpavu nAvu 
varisuvudakekx dhAyxnisuvudakekx yoVgayxvAda (teVjasusx athavA ananx) 
adanunx dhAyxna mADuvevu eMdu (oMdu pAdada athaR) vAtAH = vAyugaLu 
madhuvinaMte suKakaravAgi biVsutatxve. (eMdu madhuvAtAQutAyate 
eMbudara athaR)
\end{artha}

\vishaya{adanenxX sapxSaTxpaDisuvudu}

\begin{artha}
udevxVgavanunx uMTumADade AnaMdakaravAgi vAyugaLu biVsutatxve. hAgeye 
nadigaLU kUDa oLeLxya amaqtarasavanunx nanagoVsakxra yAvAgalU 
sarxvisutatxve - eMdu (madhukaSxranitxsinadhxvaH eMbudara athaR)
\end{artha}

\vishaya{madhuvanunx surisuvudara Pala - matutx 
`mAdhivxVnaRHsanotxvXVSadhiV' eMbudara athaR --}

\begin{artha}
yAvudariMda nAvu saMtuSaTxrAgi AnaMdisutetxVveyoyeMbudu kaSxranitx 
eMbudara Pala. OSadhigaLU kUDa bahaLa madhuravAda rasavuLaLxvugaLAgi 
namage irali||
\end{artha}

\begin{artha}
hiVge BUHsAvxhA eMdu seVrisi mUru maMtarxvanunx ucacxrisi Ahutiya 
(oMdu tutatxnunx) bAyalilx hAkikoLaLxbeVku. hiVgeye muMdina 
maMtarxgaLalUlx mUru mUrAgi mADi ucacxrisi (eraDane matutx mUrane 
AhutigaLanunx matutx nAlakxne Ahutiyanunx savaRmaMtarxgaLiMda 
ucacxrisi bAyiyalilx hAkikoLaLxbeVku)||
\end{artha}

\vishaya{deVvasayxdhiVmahi eMbudara vAyxKAyxna matutx itara maMtarx 
vAyxKAyxna}

\begin{artha}
deVvasayx = parxkAshavuLaLx savitaqvina manoVharavAda savxrUpavanunx 
nAvu ciMtisuvevu|| (madhunakamitAyxdi maMtArxthaR) - nakatxM = 
rAtirxyu utoVSasaH = hagalugaLU saha madhu = saMtoVSavuMTu mADuvaMteyU 
duHKavanunx kaLeyutAtx irali.\\
`pAthiRvaM rajaH =' I loVkavu madhumatf = saMtoVSakaravAgirali. `dawyxH 
= duyxloVkavu, pitA = pitaqloVkavU naH = namage madhu = 
saMtoVSakaravAgali, BuvaHsAvxhA' eMdu eraDane Ahutiyanunx bAyalilx 
hAkikoLaLxbeVku||
\end{artha}

\vishaya{gAyatirxya mUrane pAdavanunx viMgaDisi athaR mADutAtxre --}

\begin{artha}
yAva savitaq deVvateyu udayisutAtx budidhxgaLanUnx 
jAcnxneVMdirxyagaLanUnx perxVreVpisuvudo, A deVvateyu namamx 
sherxVyasisxgAgi yAvAgalU perxVreVpisali||
\end{artha}

\vishaya{mUrane madhumatiV Qukakxnunx vAyxKAyxnisuvudu --}

\begin{artha}
vanasapxtiyU hAgU caMdarxnU, sUyaRnU namage suKakaranAgali gAvaH = 
kiraNagaLu mAdhimxVH = amaqtakaravAgirali, athavA gAvaH = dikukxgaLu 
suKakaravAgirali||
\end{artha}

\begin{artha}
hiMde heVLida eraDu maMtarxgaLoMdige `savxHsAhA' eMdu mUrane 
Ahutiyanunx bAyalilx hAkikoLuLxvudu. anaMtara elAlx oTATxgi 
\footnote[1]{nAlakxne Ahutiyanunx BakiSxsuvAga modalu gAyatirxyanunx 
anaMtara madhumatiV QukukxgaLanunx pUtiRyAgi heVLi `ahameVveVdaM 
savaRM BUyAsamf' eMdu heVLi. BUBuRvaHsavxHsAvxhA eMdu koneyalilx heVLi 
BakiSxsabeVkeMdu athaRvu.}gAyatirxVmaMtarxvanunx, madhumatiVmaMtarxvanunx, mUru 
vAyxhaqtigaLanunx oTuTx ucacxrisi nAlakxne Ahutiyanunx hiMdinaMte 
bAyalilx hAkikoLaLxbeVku. Adare\footnote[2]{nAlukx gArxsagaLanunx 
tegedukoMDare Ahutidarxvayxvu elalxvU mugiyuvaMte tegedukoLaLxbeVku. 
hAge modale aMdAju mADirabeVku|| anaMtara pAterxyalilx leVpisidudx 
uLididadxrU A pAterxyanunx elalx toLedu adanunx tAne maMtarxvilalxde 
sumamxne kuDiyabeVku|| anaMtara eraDu keYgaLanunx toLedukoMDu 
shudidhxgAgi Acamana mADi aginxya pashicxma BAgadalilx malagabeVkeMdu 
aBipArxyavu||}AvAga ``aha meVveVdaM 
(jagatf) savaRMBUyAsamf" eMdu heVLutAtx `BUBuRvaHsavxH sAvxhA' eMdu 
heVLi nAlakxne Ahutiyanunx BakiSxsabeVku||
\end{artha}

\vishaya{``anatxta A ca mayx" - eMbudara athaR --}

\begin{artha}
samxqqti parxsidadhxvAda Acamanavanunx koneyalilx shudidhxgoVsakxra 
mADabeVkeMdu heVLide. anaMtara aginxya hiMBAgadalilx pUvaR dikikxge 
taleyaninxTuTxkoMDu malagabeVku||
\end{artha}

\vishaya{pArxtarAditayx mupatiSaThxteV - eMbudara athaR --}

\begin{artha}
beLagina JAvadalilx anaMtara edudx BakitxyiMda AditoyxVpasAthxna 
mADabeVku (pArxtaHsaMdhAyxvaMdane mADi I upAsAthxna mADabeVku) muMde 
heVLuva maMtarxdiMda yAvAgalU (udayakAladalilx) sUyaRnanunx 
upAsisabeVku||
\end{artha}

\vishaya{`\stext' eMdu upasAthxnada maMtarx, adara athaR}

\begin{artha}
`Eka puMDariVkamf' eMbalilx EkatavxveMdare pArxdhAnayxveMbudeV 
udidxSaTxvAgide. EkashabadxdiMda saMKeyx udidxSaTxvAgilalx. udA:- 
``EkaH shevxVtavAhanaH" eMbuvaMte.
\end{artha}

\vishaya{oTuTx maMtArxthaR --}

\begin{artha}
eleY AditayxneV niVnu heVge dikukxgaLige parxdhAnanAgididxVyoV hAgeye 
nAnU manuSayxralilx puruSana BoVga sAdhanagaLiMda samaqdadhxnAgi 
ninanx anugarxhadiMda parxdhAnanAguvenu||
\end{artha}

\begin{artha}
(Aditayxna upasAthxnada naMtara horage) yAva sathxLadiMda hoVgidadxno 
hAgeye baMdu aginxya hiMBAgadalilx kuLitu parxyatanxpUvaRkavAgi 
samAdhAnavAgidudx (EkacitatxnAgi) `taMheYtamf' eMba 
vaMshamaMtarxvanunx BakitxyiMda japisabeVku||
\end{artha}

\section*{baq. a.6, bArx. 3, kaMDike 12}

\stext

\stext

\stext

\stext

\stext

\stext

\vishaya{ta\c  heYtavf itAyxdi maMtarxda vAyxKAyxna}

\begin{artha}
AmeVle (pArxNoVpAsaneya naMtara) adeV I manadhx kamaRvanunx 
parxyatanxpUvaRka heVLidadxnunx udAdxlakana maga budidhxvaMtanAda 
(AruNiyu) yAjacnxvalakxyXnige heVLi punaH AshacxyaRpaTuTx 
maMthakamaRvanunx udedxVshisi heVLidanu||
\end{artha}

\vishaya{Enu heVLidaneMdare --}

\begin{artha}
maMthadarxvayxvananx (BakaSxNakAkxgi saMsakxrisidadxnunx) yAru oNagida 
moVTumarada meVlU celulxvaro, adara shAKegaLu huTiTxye huTuTxvavu, 
elegaLU ciguruvavu||
\end{artha}

%% shloka footnotes
\begin{artha}
\footnote[1]{hiMdeye eMdare - `audumabxreV camaseV kaMseVvA' eMbuvalilx}\\
\footnote[2]{maMthadarxvayxvanunx atitxmarada pAterxyalilx hAki mosaru 
modalAda mUru darxvayxgaLanunx (mosaru, madhu, tupapx ivugaLanunx) 
seVrisi oMdu mathana mADuva kaDagoVliniMda kaDedu madhayxdalilxrisi 
auduMbarada surxvadiMda nitAyxjAyxhutigaLanunx koTuTx anaMtara 
AhutigaLalilx saMpAtavanunx mADabeVkeMdu heVLidAdxgide.}\\
I vacanakekx yAva athaRvo adeV athaRvu muMdina vAkayxgaLalUlx iruvudu. 
`caturawdumabxroBavati' - eMba kaMDikeya athaRvanunx hiMdeye 
vAyxKAyxnisidAdxgide. suKavAgi tiLiyuvudariMda tAnAgiye 
tiLidukoLaLxbahudu (adariMda punaH vAyxKAyxnisuvudilalx)||
\end{artha}

\section*{baq. a.6, bArx. 3, kaMDike 13}

\footnote[3]{nAlukx auduMbarada yahocnxVpayoVgi vasutxgaLu - surxva, 
camasa, idhamx (samitutx). eraDu mathana mADuva kaDagoVlugaLu I nAlukx 
atitxmaradiMda mADidAdxgirabeVkeMdathaR.}\stext

\begin{center}
itishirxV baqhadAraNayxkoVpaniSadfBASayx vAtiRkadalilx Arane 
adhAyxyadalilx mUraneV bArxhamxNavu
\end{center}

\section*{baq.6--4--1, 2)\\ baq. a.6, bArx. 4, kaMDike 1}

\stext

\stext

\begin{center}
baqhadAraNayxka nAlakxne bArxhamxNa\\
||dakiSxNAmUtaRyeVnamaH||
\end{center}

\begin{artha}
`ESAmf' eMdu AraMBisi `reVtaH' eMbuva payaRMtara puruSabiVjada 
sutxtiyu vivakiSxtavAgide. `sahaparxjApatiriVkASxMcakerxV' eMbuvalilx 
(yAralilx iTaTx reVtasutx puruSatavxvanunx hoMduvudo) A 
AdhAraBUtavAdudanunx saqSiTxsuvaneMdu AloVcisidaneMdathaRvu||
\end{artha}

\vishaya{adanenxV muMde vAyxKAyxnisuvudu --}

\begin{artha}
yAralilx seVcane mADida puruSaviVyaRvu puruSatavxvanunx hoMduvado A 
bageya AdhAra vasutxvanunx (sitxrXVyanunx) saqSiTxsuveneMdu 
IshavxranAda parxjApati barxhamxnu AloVcisidanu||
\end{artha}

\vishaya{`sasitxrXya \c sasaqjeV'}

%% shloka footnote
\begin{artha}
\footnote[1]{`patishacx patinxVcABavatAmf' eMbuvalilx heVLida 
shatarUpA eMbuva patinxyanunx saqSiTxsidaneMdu athaR||}\\
\footnote[2]{gArxmayxdhamaRveMdare kAmashAsotxrXVkatxvAgiruvaMte 
pashukamaR.}\\
adakekx yoVgayxvADa madhukAMDadalilx heVLidaMtiruva sitxrXVyanunx 
saqSiTxsidanu. saqSiTxmADi anaMtara A sitxrXVyanunx 
gArxmayxdhamaRdiMda (meYthunadiMda) keLaBAgadalilx seVvisidanu||
\end{artha}

\vishaya{`tasAmxtf sitxrXya madha upAsiVta' eMbudara athaR}

\begin{artha}
parxjApatiyu makakxLa utapxtitxgAgi hiMde parxyatanxpUvaRka 
sitxrXVyara keLaBAgavanunx seVvisidanu. adariMda putarxkAmiyAdavanU 
adanunx seVvisabeVku||
\end{artha}

\vishaya{`sa EtaM pArxcnAcxM...... itAyxdi maMtarxda tAtapxyaR'}

%% shloka footnote
\begin{artha}
\footnote[1]{vAjapeVya yAgakekx samAnavAgi I meYthuna kamaRvu 
iruvudeMdu BAvisabeVku. yAgadalilx soVmalateyanunx kuTiTx rasavanunx 
tegeyalu udadxvAda kalalxnunx upayoVgisuvuduMTu. hAgeye I pAshava 
kamaRdalilx A kalilxna sAthxnadalilx tananx jananeVMdirxyavu ideyeMta 
BAvisabeVku, kaThinavAgiruvudariMda hAgeyeV BAvisabeVku.}\\
IvAga soVmABiSava rUpavanunx kalipxsikoLaLxlu heVLuvudu. I tananx 
shishanxvanunx soVmarasavanunx hiMDalu upayoVgisuva kalilxnaMte 
gaTiTxyAgiruvaMte tuMbikoMDanu,
\end{artha}

\vishaya{parxjApatiyu muMdeVnu mADidanu? --}

\begin{artha}
A shishanxvanunx QujuvAgi diVGaRvAgi mADikoMDu heVge soVmarasavanunx 
tegeyuva kalulx iruvudo hAgeye mADikoMDu adanunx sitxrXVyalilx 
(shAsitxrXVya) parxyatanxdiMda oMdAgi seVrisidanu||
\end{artha}

\vishaya{`teVneYnA maBayxsaqjata' eMbudara vAyxKAyxna --}

\begin{artha}
hAgeye iruvaMte BAvisida, kalilxniMda sitxrXVyanunx matetxmatetx 
saMbaMdha mADidanu. I meYthuna kamaRvanunx vAjapeVya yAgaveMdu BAvisi 
upAsane mADuvudu tananx puruSAthaRkAkxgi||
\end{artha}

\vishaya{upAsaneya karxmavanunx shurxtiyu heVLuvudu}

\section*{baq. a.6, bArx. 4, kaMDike 3}

\stext

\begin{artha}
yAva sitxrXVyalilx puruSanu tananx jananeVMdirxyavanunx oLapaDisuvano, 
A sitxrXya guhayx sAthxnave veVdiyeMdu BAvisabeVku. adu eraDu toDegaLa 
meVliruvaMte BAvisabeVku. adara meVle huTiTxruva roVmagaLe 
daBeRgaLeMdU, adara camaRvanenx etitxna camaRveMdU, eraDu BAgadalUlx 
oLage aDagiruva eraDu mAMsapiMDagaLanunx karxmavAgi soVmarasavanunx 
hiMDuva PalakagaLanAnxgiyU nAvu BAvisabeVku. yoVniya oLagiruva 
parxdeVshavanunx uriyuva aginxyeMdU BAvisabeVku. hiMde heVLida camaRvu 
yoVnideVshada sutatxlU iruvudanunx toVriside. adaroLage iruva 
muSakxgaLeMdare vaqSaNagaLeMdu (mAMsagarxMthigaLeMdu) tiLiyabeVku. I 
riVtiyAda muSakxgaLalilx soVmarasavanunx hiMDuva PalakagaLeMdu 
BAvaneyanunx mADabeVku|| 
\end{artha}

\vishaya{}

\begin{artha}

\end{artha}

\begin{artha}

\end{artha}

\begin{artha}

\end{artha}

\begin{artha}

\end{artha}

\begin{artha}

\end{artha}

\begin{artha}

\end{artha}

\begin{artha}

\end{artha}

\begin{artha}

\end{artha}

\begin{artha}

\end{artha}

\begin{artha}

\end{artha}

\begin{artha}

\end{artha}

\begin{artha}

\end{artha}

\begin{artha}

\end{artha}

\begin{artha}

\end{artha}

\begin{artha}

\end{artha}

\begin{artha}

\end{artha}

\begin{artha}

\end{artha}

\begin{artha}

\end{artha}

\begin{artha}

\end{artha}

\begin{artha}

\end{artha}

\begin{artha}

\end{artha}

\begin{artha}

\end{artha}

\begin{artha}

\end{artha}

\begin{artha}

\end{artha}

\begin{artha}

\end{artha}

\begin{artha}

\end{artha}

\begin{artha}

\end{artha}

\begin{artha}

\end{artha}

\begin{artha}

\end{artha}

\begin{artha}

\end{artha}

\begin{artha}

\end{artha}

\begin{artha}

\end{artha}

\begin{artha}

\end{artha}

\begin{artha}

\end{artha}

\begin{artha}

\end{artha}

\begin{artha}

\end{artha}

\begin{artha}

\end{artha}
