%%%% From 016.tex

%~ \begin{center}
%~ \section*{sureVshavxravAtiRka}
%~ {\Large\textbf{adhAyxya - 3,  (1neV ashavxla bArxhamxNa)}}
%~ \medskip
%~ \end{center}
%~ \section*{mUrane adhAyxya (1neV ashavxla bArxhamxNa)}
\chapter{bArxhamxNa - 1 : ashavxla bArxhamxNa}
\vishaya{1. yAjacnxvalakxyXkAMDa vAyxKAyxna parxtijecnx}

\begin{shl}
samApotxV madhukANADxthoVR yAjacnxvalikxVyakANaDxgaH |\\
ataH paraM parxyatenxVna shurxtAyx vAyxKAyxyateV suPxTamf \hfill || 1 || 
\end{shl}

\begin{artha}
madhukAMDada athaRvu pUNaRvAyitu, IvAga yAjacnxvalakxyX kAMDadalilxruva viSayavanunx shurxtiyu parxyatanxdiMd sapxSaTxvAgi vAyxKAyxnisuvudu.
\end{artha}

\vishaya{2. munikAMDavu AmeVle baralu kAraNa EneMdare ---}

\begin{shl}
yAjacnxvalikxVyameVtasAmxnamxdhukANADxdananatxramf |\\
kANaDxM vicAraBUyiSaThxmadhunA\s \s raBayxteV paramf \hfill || 2 ||
\end{shl}

\begin{artha}
I madhukAMDavAda meVle yAjacnxvalakxyX munikAMDavu bahuvicAradiMda tuMbideyAdadxriMda Iga muMde AraMBisalapxTiTxde.
\end{artha}

\vishaya{3. I eraDu kAMDagaLalilx punarukitx shaMkA parihAra}

\vishaya{shaMke --- }

\begin{shl}
nanu pUvaRM ya ukotxV\s thaRH sa EveVhApi BaNayxteV |\\
punarukatxM na ca nAyxyayxmeVkaparxvacanasathxyoVH \hfill || 3 || 
\end{shl}

\begin{artha}
modalu heVLida viSayavanenxV ililxyU heVLide, alalxve ? oMdeV parxvacana shAKeyalilx iruva kAMDagaLalilx punarukitx irabAradu, (nAyxyavalalx).
\end{artha}

\vishaya{samAdhAna ---}

\begin{shl}
EkAthaRtevxV\s pi ca sati neYvAsayx punarukatxtA |\\
yAjacnxvalikxVyakANaDxsayx huyxpapatitxparxdhAnataH \hfill || 4 ||
\end{shl}

\begin{artha}
oMdeV viSayavuLaLxdAdxdarU idu punarukatxvAgilalx, yAjacnxvalakxyXra kAMDavu yukitxparxdhAnavAgiruvudariMda A doVSavilalx.
\end{artha}

\begin{shl}
AgamoVkitxparx dhAnatavxM madhukANaDxsayx vaNiRtamf |\\
AgamAthaRvishudadhxyXthaRM yukitxVratarx parxvakaSxyXti \hfill || 5 ||
\end{shl}

\begin{artha}
madhukAMDakekx Agama vacanagaLe muKayxveMdu vaNiRsalapxTiTxde, Agamada athaRshoVdhanegAgi yukitxgaLanunx ililx muMde heVLuvudu.
\end{artha}

\vishaya{4. Agamada sAvxtaMtarxyXvu yukitx heVLuvudariMda hoVgadu}

\begin{shl}
na cA\s \s gamasayx sAvxtanatxrXyaM yukutxyXketxVrapanudayxteV |\\
athARnatxratAvxduyxkitxVnAM parxmANeVBoyxV yatasatxtaH \hfill || 6 ||
\end{shl}

\begin{artha}
Agamada sAvxtaMtarxyXvanunx yukitxgaLanunx heVLi alalxgaLeyuvudilalx(dhakekx Aguvudilalx) parxmANagaLigU yukitxgaLigU \footnote[1]{yukitxgaLu parxtayxkASxdigaLaMte parxmANagaLalalx. keVvala saMBAvanArUpavAdaveMdu heVLuvaru || nAyxyashAsatxrXdalUlx}athaRBeVdaviruvudariMda (veVdada sAvxtaMtarxyXvanunx yukitxgaLiMda haLeyuvudakekx Aguvudilalx).
\end{artha}

\vishaya{adu heVge ? eMdare ---}

\begin{shl}
savaRparxmANasheVSatavxM yukitxVnAmupavaNiRtamf |\\
shAsArxnatxreVSavxpi tathA nAtaH sAvxtanatxrXyXKaNaDxnamf \hfill || 7 ||
\end{shl}

\begin{artha}
yukitxgaLu sakala parxmANagaLigU aMgavAgiveyeMdu vaNiRside, \footnote[2]{``avijAcnxtatatevxV\s theVR kAraNoVpapatitxtaH tatavxjAcnxnAthaRmUhasatxkaRH '' eMdu takaRlakaSxNavanunx heVLiruvaru, nAyxyaBASayxrarU saha - `tatavxviSayABayxnujAcnx lakaSxNa dUhAtf BAvitAtf parxsanAnx dana taraM parxmANA sAmathAyxR tatxtavx jAcnxna mupapadayxta iti' eMdU nAyxyavAtiRkakArarU ``parxmANAni punaH parxvataRmAnAni takaRvivikatx mathaRM yathABUta madhigamayanitxH" eMdU beVre kelavarU saha ``tayA parxmANoVpapatAtxyX itikataRvayxtayA parxmANa viSaya maBayxnujAnanAtxyX vishoVdhiteV viSayeV parxmANa maparxtuyxM haM parxvataRta iti" - eMdU heVLiruvudanunx parAmashiRsidalilx, Uha, takaR yukitxyeMdu heVLuvudelalxvU parxmANavalalxveMdU parxmANadiMda baruva tatavxjAcnxnakekx sahAyakavAgiveyeMdU tiLidubaruvudu.}beVre shAsatxrXgaLalUlx hiVgeye heVLide. idariMdaleV \footnote[3]{yukitxgaLu parxmANakekx aMgavAgidadxre yukitxyanunx apeVkiSxsuva Agamakekx nirapeVkaSxteyeMbuva sAvxtaMtarxyXvu tapipx hoVguvudilalx.}sAvxtaMtarxyXvanunx kaLeyuvudilalx.
\end{artha}

\vishaya{yukitxyeMdareVnu ? adariMda Agama sAvxtaMtarxyXkekx BaMgavilalxve ?}

\vishaya{5. yukitxgU AgamakUkx iruva viSaya BeVda}

\begin{shl}
padAthaRvipayA ceVyaM yukitxsatxkoVR\s BidhiVyateV |\\
vAkAyxthaRsAtxvXgamAdeVva nirapeVkaSxmatoV vacaH \hfill || 8 ||
\end{shl}

\begin{artha}
yukitxyu padAthaR viSayadalilx iruvudu, idanunx takaRveMdu heVLuvudu, vAkAyxthaRvAdaro AgamadiMdaleV tiLiyuvudu, adariMda shurxtiya vacanavu nirapeVkaSx, savxtaMtarx parxmANaveV sari.
\end{artha}

\vishaya{6. takaR athavA yukitxge pwruSeVya vAkayxgaLalilx pArxmuKayxte.}

\begin{shl}
pwruSeVyavacaHsevxVva yuketxVH pArxdhAnayxmiSayxteV |\\
anaroVkwtx tu tAtapxyaRM yuketxVrathoVR na yukitxtaH \hfill || 9 ||
\end{shl}

\begin{artha}
pwruSeVya vAkayxgaLalelxV yukitxge pArxdhAnayxviruvudu. apwruSeVya vacanadalAlxdaroV tAtapxyaRvu yukitxyiMda tiLiyuvudu, aSeTx. adara athaRvu yukitxyiMda tiLiyuvudalalx.
\end{artha}

\begin{shl}
parxtayxkASxdayxtivatiRtAvxduyxkatxyXpeVkASx na vidayxteV |\\
AgamAtheVR, yatheYvaM sAyxduyxkatxyXtheVR nA\s \s gameVkaSxNamf \hfill || 10 || 
\end{shl}

\begin{artha}
AgamAthaRda viSayadalilx parxtayxkASxdiparxmANagaLanunx miVriruvadariMda yukitxya apeVkeSx iruvudilalx. idu heVgoV hAgeye yukitxyiMda tiLiyuva viSayadalUlx Agamavanunx niriVkiSxsuvudilalx.
\end{artha}

\vishaya{Agamakekx yukitxya apeVkeSxyilalxvAdare EtakAkxgi adanunx heVLuvudu eMdare ? ---}

\begin{shl}
puMsavxBAvAnuroVdheVna yukitxveVRdeV\s BidhiVyateV |\\
Agamasayx parxvaqtitxsutx meVyavaqtatxvayxpeVkaSxyA \hfill || 11 ||
\end{shl}

\begin{artha}
puruSara savxBAvavanunx anusarisi veVdagaLalilx yukitxyanunx heVLuvadaSeTx, Agamavu mAtarx tananx parxmeVyada savxBAvavanunx anusarisi parxvatiRsuvudu.
\end{artha}

\vishaya{7. madhukAMDadalUlx yukitxvaNaRneyu samAnaveMba shaMkA parihAra ---}

\begin{shl}
nanUkAtx madhukANeDxV\s pi dunudxBAyxduyxpapatatxyaH |\\
AgameYkaparxdhAnatavxM kathaM taseyxVti BaNayxteV \hfill || 12 || 
\end{shl}

\begin{artha}
shaMke --- madhukAMDadalUlx duMduBi muMtAda (daqSATxMtagaLiMda) yukitxgaLanunx heVLideyalalxve ? adariMda A kAMDavu Agama mAtarx parxdhAnavAgiruvudeMdu heVge ? heVLuvudu ?
\end{artha}

\vishaya{samAdhAna ---}

\begin{shl}
neYSa doVSoV yatasatxtarx yukitxVnAmaparxdhAnatA |\\
pArxdhAnayxM yAjacnxvalikxVyeV yukitxVnAmaBidhiVyateV \hfill || 13 || 
\end{shl}

\begin{artha}
ideVnU doVSavalalx, A madhukAMDadalilx yukitxgaLidadxrU avugaLu aparxdhAnavAgive, yAjacnxvalakxyXra kAMDadalilx mAtarx yukitxgaLigeV pArxdhAnayxviruvudu.
\end{artha}

\begin{shl}
yukAtxyXgamw hi saMBUya karasAthxmalakAdivatf |\\
susUkaSxmXmapi sadavxsutx shakwtx jAcnxpayituM yataH \hfill || 14 ||
\end{shl}

\begin{artha}
yukitx - matutx AgamagaLu seVri keYyalilxruva nelilxkAyi - muMtAdavugaLaMte atayxMta sUkaSxmXvAda sadavxsutxvanunx tiLisuvudakekx shakatxvAgive.
\end{artha}

\begin{shl}
yukatxyoV\s toV\s BidhiVyanetxV pUvaRpakASxdisaMsharxyAH |\\
yAjacnxvalikxVya EtasimxnAkxNeDxV jalopxVkitxvatamxRnA \hfill || 15 ||
\end{shl}

\begin{artha}
adariMda I yAjacnxvalakxra kAMDadalilx pUvaRpakASxdigaLanunx avalaMbisi jalapxvAdada mAgaRdalilx yukitxgaLanunx heVLuvudu.	
\end{artha}

\vishaya{I bArxhamxNagaLa saMbaMdhavanunx heVLutAtxre ---}

\begin{shl}
udigxVthaparxmuKA yeV\s thAR madhukANeDxV puroVditAH |\\
teVSAmeVva vishudadhxyXthaRM vicAraH kirxyateV\s dhunA \hfill || 16 ||
\end{shl}

\begin{artha}
madhukAMDadalilx modalu heVLida yAva udigxVtha modalAda viSayagaLiveyoV, avugaLa shoVdhanegAgiyeV I vicAravanunx mADuvudu.
\end{artha}

\begin{shl}
dashaRnasAyxsayx teVnAta EkAvAkayxtavxmiSayxteV |\\
tatheYvAnayxpadAtheVRSu tadadxvXyoVrapi kANaDxyoVH \hfill || 17 ||
\end{shl}

\begin{artha}
I \footnote[1]{I bArxhamxNavu udigxVtha bArxhamxNadalilx heVLida vAgAdi savxrUpagaLa niNaRyakAkxgi baMdiruvudariMda I bArxhamxNadalilxna dashaRnavu alilxya upAsanege aMgavAgiruvadariMda I hiMdu muMdina bArxhamxNagaLige EkavakayxteyideyeMdathaRvu |}dashaRnakekx A bArxhamxNadoDane EkavAkayxteyu saMmatavAgide, \footnote[2]{I modalina bArxhamxNakekx mAtarx pUvaRsaMbaMdhavalalx, matetxVneMdare - beVre padAthaRgaLigU I saMbaMdhavideyeMdathaR.}hAgeye beVre bArxhamxNadalilx heVLida padAthaRgaLalilx (takaRrUpavAda athaRgaLalilx) A eraDu kAMDagaLigU EkavAkayxteyu iruvudu.
\end{artha}

%~ \section*{baq. a. 3. bArx. 1, kaMDike 3.}
\kandike{kaMDike 3}

\begin{kandikeshl}
yAjacnxvalekxyXVti hoVvAca yadidaM savaRM maqtuyxnApatxM savaRM maqtuyxnABipananxM keVna yajamAnoV maqtoyxVrApitxmatimucayxta iti hoVtarxtivxRjAginxnA vAcA vAgevxY yajacnxsayx hoVtA tadeyxVyaM vAkosxV\s yamaginxH sa hoVtA sa mukitxH sAtimukitxH || 3 ||
\end{kandikeshl}

\vishaya{8. `yadidaM savaRM' itAyxdi padAthaR vAyxKAyxnada AraMBa}

\begin{shl}
yadidaM kamaRNoV\s sheVSasAdhanaM maqtuyxnA\s \s pulxtamf |\\
sAvxmimananxrXtivxRgagAnxyXdi tavxBipananxM vashiVkaqtamf \hfill || 18 ||
\end{shl}

\begin{artha}
yAvudu I kamaRgaLa samasatx sAdhanagaLu iveyo, avelalxvU maqtuyxviniMda vAyxpatxvAgive, samasatx sAdhanagaLeMdare yajamAna, maMtarx, yativxkf, aginx muMtAdavu, ivu aBipananxvAgive vashiVkaqtavAgive.\footnote[3]{A kamaRsAdhanagaLoMdige kamaRvanunx mADuva yajamAnanu yAvudariMda maqtuyxsaMbaMdhadiMda biDugaDe hoMduvanu mukatxnAguvanu ? eMdu sheVSavAda vAkAyxthaRvanunx I saMdaBaRdalilx iTuTxkoLaLxbeVku. hiVgeMdu AnaMdagirigaLu (AnaM - TiVkA)}
\end{artha}

\vishaya{maqtuyx eMdare Enu ?}

\begin{shl}
pariceCxVdakaqdajAcnxnaM sAsaknagxM maqtuyxsaMjicnxtamf |\\
keVnAyaM yajamAnoV\s toV maqtoyxVrApetxVviRmucayxteV \hfill || 19 ||
\end{shl}

\begin{artha}
BeVdavanunxMTu mADuva ajAcnxna, AsaMgaveMbuva rAgadoVSadiMda kUDidudx maqtuyx eMdu (I saMdaBaRdalilx) hesaruLaLxdAdxgide, (keVna... itAyxdi vAkayxda athaR) I yajamAnanu yAva upAyadiMda I maqtuyxvina saMbaMdhadiMda vimukatxnAgutAtxne ? (eMdu parxshenx).
\end{artha}

\vishaya{I parxshenx heVge saMgata}

\begin{shl}
yanamxtayxRsAdhanaM sAdhayxM matayxRM tadapi jAyateV |\\
sAdhanAnumitaM sAdhayxM mukitxH keVnAta ucayxteV \hfill || 20 ||
\end{shl}

\begin{artha}
yAvudu nashavxravAda sAdhanavo, adara PalavAda sAdhayxvU nashavxraveV Aguvudu, I riVtiyAgi sAdhanadiMda sAdhayxvanUnx adara savxrUpavanUnx Uhisuva kAraNa \footnote[1]{sAdhana sahitavAgi PalasameVtavAgi I kamaRgaLu nAshavAguvudariMda namage tAyxjayxvAgive, adariMda iMtaha nashavxravAda kamaRsathxlavanenxlAlx vAyxpisiruva matutx barxhamxkaSxtArxdi BeVdakekx kAraNavAda BArxMtijAcnxna, rAgAdidoVSagaLeMba maqtuyxvanunx yAvudariMda dATuvudeMdu parxshenxya vivaraNe.}mukitxyu yAvudariMda AguvudeMdu parxshenxyanunx mADide.
\end{artha}

\vishaya{parihAra ---}

\begin{shl}
hoVtarxtivxRjA\s ginxnA vAcA maqtoyxVrApetxVviRmucayxteV |\\
iti parxshanxparxtivacoV yAjacnxvalokxyXV\s puyxvAca tamf \hfill || 21 ||
\end{shl}

\begin{artha}
aginxrUpanAda hoVtaq eMba QutivxkikxniMdalU, vAkikxniMdalU maqtuyx saMbaMdhadiMda mukatxnAguvaneMdu yAjacnxvalakxyXru A ashavxlaveMbuvananunx kuritu I parxshenxge utatxravanunx heVLidaru.
\end{artha}

\begin{shl}
parxvakatx yAjacnxvalokxyXV\s tarx taM paqcaCxnatxyXshavxlAdayaH |\\
sarAjakeV samAjeV\s yaM vicAraH kirxyateV mahAnf \hfill || 22 ||
\end{shl}

\begin{artha}
I janaka mahArAjaniMda kUDida I saBeyalilx yAjacnxvalakxyXru utatxra heVLuvavaru, avaranunx kuritu ashavxla modalAdavaru parxshenx mADutAtxre. adariMda (tatavxniNaRyakAkxgi) I mahAvicAravanunx ililx mADide.
\end{artha}

\vishaya{9. `vAgf veYyajacnxsayx' itAyxdi vAkayxda athaR ---}

\begin{shl}
yajamAnasayx yeVyaM vAgoGxVtA ceYtadavxyaM yadA |\\
adhideYvAtamxnA veVtitx sa hoVteYvaMvidhoV BaveVtf \hfill || 23 ||
\end{shl}

\begin{shl}
ananatxvigarxhaH soV\s ginxhoVRtA maqtoyxVyaRthoVditAtf |\\
yajamAnasayx mukitxH sAyxdatimukitxsatxtheYva ca \hfill || 24 ||
\end{shl}

\begin{shl}
AsurAtAsxdhanAdedxYvapArxpitxmuRkitxrihoVcayxteV |\\
sAdhAyxdapAyxsurAdedxYvasAdhAyxpitxratimukitxtA \hfill || 25 ||
\end{shl}

\begin{artha}
yajamAnana yAva I vAkukx iruvudo matutx hoVtaq eMba Qutivxkukx I eraDanUnx adhideYvateya rUpadiMda hoVtaqvu yAvAtanu tiLiyuvano, dhAyxna mADuvano, avanu I riVtiyAguvanu (aMdare yajamAnanigU tanagU I bageya maqtuyxnAshavAguvudakekx kAraNanAguvanu).
\end{artha}

\begin{artha}
aneVka rUpavAgiruva A aginxyu hoVtaqvU hiMde heVLida (BeVdajAcnxna rAgAdidoVSa sahitavAgidadx) I maqtuyxviniMda yajamAnanige \footnote[1]{`samukitxH sahoVtA\s ginxH mukitxH' eMba BASayxdaMte A hoVtaq rUpavanunx aginxyu mukitxyeMdare aginx savxrUpadiMda noVDuvudeV mukitxge sAdhanaveMdare, oTiTxnalilx yajamAnana vAkukx, hoVtaq eMba eraDu sAdhanagaLanunx aginxrUpadalilx yAvAga dhAyxnadiMda noVDuvano ava AgaleV sAvxBAvikavAda viSayAsakitxyeMbuva maqtuyxviniMdalU AdhAyxtimxka matutx AdhiBwtikavAda pariciCxnanx rUpadiMda biDugaDe hoMduvaneMdathaR adariMda A hoVtaqvu aginxrUpadiMda kaMDalilx yajamAnana maqtuyxveMba pariciCxnanx budidhxyiMda mukitxyAgalu sAdhanavAguvanu.}mukitxyAguvudu, hAgu \footnote[2]{I yAva mukitxyideyoV adeV ati mukitxge sAdhanavAguvudu, aMdare eraDu sAdhanagaLanunx pariciCxnanxvAdavugaLanunx adhideVvatArUpaveMba apariciCxnanxvAda vAyxpakavAda aginxrUpadiMda noVDuvude mukitx, adeV adhideVvatAdaqSitxya adhAyxtamx adhiBUtagaLeMba pariceCxVda viSayadalilxruva AsakitxrAgaveMbuva maqtuyxvanunx miVri adhideYvavAda aginx savxrUpavu laBisuvudeMba PalarUpa BAvaveV ati mukitx, kAyaRkAraNa BAvadiMda aupacArikavAgi mukitxyeV ati mukitxyeMdu heVLide.}ati mukitxyAguvudu. (25) mukitx padAthaRveVneMdare - asura sAdhanadiMda (vidAyx vihiVnavAda keVvala kamaRdiMda, athavA pApamisharxvAda kamaRdiMda) \footnote[1]{sUtArxtamx savxrUpalABakekx kAraNavAda jAcnxnakamaR samucacxya lABaveV mukitxyeMdathaR, uLida athaRvanunx idaraMte AnaMdagirigaLa TiVkeyelilxruvaMte meVle anuvAdadalilx (\quad) kaMsadiMda sUcisiruvevu.}deYva BAvavu baruvudeV mukitxyu, Asura sAdhanaPalavAda (savxgARdigaLigiMtalU) deYvasAdhanaPalavAda sUtAmxtamx lABavu ati mukitxyeMdu tiLiyabeVku.
\end{artha}

\vishaya{10. kelavaru mADida beVre vAyxKAyxna}

\begin{shl}
yathoVkotxVpAsanAdeVva keYvalayxM ceVdivxvakaSxyXteV |\\
barxhamxvidAyx kimatheVRyaM nAnayxnumxketxVH PalaM tataH \hfill || 26 ||
\end{shl}

\begin{artha}
hiMde heVLida \footnote[2]{kelavaru kamaR samucicxtavAda sUtorxVpAsaneyiMda neVra moVkaSxveV AguvudeMdu I athaRdalelx `sAtimukitxH' eMbudanunx vAyxKAyxnisuvaru. adu sariyalalxveMdu vAtiRkakAraru nirAkarisidAdxre. AvAga barxhamxvideyxyu vayxthaRvAguvudu, adakokxMdu beVre sathxlavidadxre AvAga I vAyxKAyxnavu sari hoVgutitxtutx, barxhamxvideyxge moVkaSxvanunx biTuTx beVroMdu PalavideyeMdu elilxyU yArU heVLuvudilalxveMdathaR.}upAsaneyiMdale moVkaSxvu baruvudeMdu kelavaru heVLuvaru. hAgeye ililx aBipArxyavidadxre I barxhamxvideyxyu Etakekx ? adu kamaRkekx aMgavAgidudx mukitxPalavuLaLxdedxMdare - alalx. mukitxge beVreyAgi matotxMdu Palavu barxhamxvideyxyiMda baruvudilalx.
\end{artha}

%~ \section*{baq. a. 3 - bArx. 1. kaMDike 4}
\kandike{kaMDike 4}

\begin{kandikeshl}
yAjacnxvalekxyXVti hoVvAca yadidaM savaRmahoVrAtArxBAyxmApatxM savaRmahoVrAtArxBAyxmaBipananxM keVna yajamAnoV\s hoVrAtarxyoVrApitxmatimucayxta itayxdhavxyuRNwQtivxRjA cakuSxSAditeyxVna cakuSxveYR yajacnxsAyxdhavxyuRsatxdayxdidaM cakuSxH soV\s sAvAditayxH soV\s dhavxyuRH sa mukitxH sAtimukitxH || 4 ||
\end{kandikeshl}

\begin{shl}
AsurAtakxmaRNoV maqtoyxVratimukitxrihoVditA |\\
kAlAtakxmARtireVkeVNa maqtoyxVmukitxrihoVcayxteV \hfill || 27 ||
\end{shl}

\begin{artha}
AsurakamaRveMdu maqtuyxviniMda ati mukitxyanunx ililx heVLadAdxyitu. kamARtirikatxvAda kAlaveMba maqtuyxviniMda mukitxyu heVLalapxDuvadu.
\end{artha}

\vishaya{11. kAla, maqtuyxveMbudanunx nirUpisuvudu.}

\begin{shl}
parxyoVgasamavAyeyxVva darxvayxkatArxRdisAdhanamf |\\
tatapxrXyoVgAvasAneV ca savaRM tadapavaqjayxteV \hfill || 28 ||
\end{shl}

\begin{shl}
parxyoVgAvasiteVsUtxdhavxRM kAlaH pArxkacx parxyoVgataH |\\
kaSxpayanavxtaRteVjasarxM savaRM tatakxmaRsAdhanamf \hfill || 29 || 
\end{shl}

\begin{shl}
tasAmxtakxmARtireVkeVNa kAloV maqtuyxH parxtiVyatAmf |\\
tatoV\s pi mukitxvaRkatxveyxVtayxta AraBayxteV paraH \hfill || 30 ||
\end{shl}

\begin{artha}
parxyoVgadalilx (kamARnuSAThxnadalilxyeV) iruva darxvayx-kaqtaqR muMtAda sAdhanavu elalxvU A parxyoVgavu munidoDane nashisuvudu. parxyoVgavu mugida meVlu, parxyoVgakekx modalU kamaR matutx adara sAdhanagaLanunx kaLeyutAtx yAvAgalU kAlavu idedxV iruvudu. adariMda kamARtirikatxvAda kAlaveV maqtuyxveMdu tiLiyabeVku, adariMdalU mukitxyanunx heVLabeVkAgideyeMdu muMdina garxMthavu AraMBavAguvudu.
\end{artha}

\begin{shl}
kAlashacx divxvidhaH porxVkatx EkoV\s hoVrAtarxlakaSxNaH |\\
tithAyxdilakaSxNashAcxnayxsAtxBAyxM mukitxrihoVcayxteV \hfill || 31 ||
\end{shl}

\begin{artha}
kAlavU saha eraDu bageyAgiruvudu, oMdu ahoVrAtarx rUpavAdudu, tithi muMtAda rUpavAdadudx matotxMdu beVre, ivugaLiMda mukitxyanunx heVLuvudu.
\end{artha}

\vishaya{modalane parxshenxya tAtapxyaR}

\begin{shl}
upasAthxpayataH kamaR tathA kaSxpayatoV yataH |\\
ahoVrAterxV tatoV maqtuyxsAtxBAyxM mukitxH kutoV BaveVtf \hfill || 32 ||
\end{shl}

\begin{artha}
kamaRvanunx odagisuvudu matutx kaLeyuvudU ahoVratirxgaLeV AgiruvudariMda avugaLiMda maqtuyxvu saMBAvitavAgide, adariMda biDugaDeyAguvudu heVge ?
\end{artha}

\vishaya{parihArada tAtapxyaR}

\begin{shl}
adhAyxtamxM cakuSxradhavxyuRradhiyajacnxM davxyaM raviH |\\
sAkASxdananatxdeVhoV\s yamiti dhAyxyanivxmucayxteV \hfill || 33 ||
\end{shl}

\begin{artha}
yajamAnana AdhAyxtamx (shariVradoLagina) kaNuNx, yajacnxdoLagiruva adhavxyuRveMba QutivxjanU, I eraDU apariciCxnanxvAda sUyaRne, eMdu ivanu dhAyxna mADutAtx ahoVrAtirxgaLeMba maqtuyxviniMda biDugaDeyanunx hoMdutAtxne.
\end{artha}

\vishaya{12. cakuSxyaRjacnxsayx itAyxdi vAkayxda tAtapxyaR ---}

\begin{shl}
AtAmxvayava EvAyaM sUyaRshacxkuSxmaRmAMshumAnf |\\
adhavxyuRrahameVveVti sAkASxtakxqqtAvx vimucayxteV \hfill || 34 ||
\end{shl}

\begin{artha}
nananx (adhavxyuRvina) saMbaMdhavAda yAva AtamxniruvanoV A yajamAnana avayavaveV IcakuSxriMdirxya matutx adhavxyuRvAda nAnU (I eraDU) kiraNagaLuLaLx sUyaRneV eMdu dhAyxnadiMda neVra parxtayxkaSxmADidalilx (adhavxyuR matutx yajamAnanu) ahoVrAtarxgaLeMba maqtuyxviniMda mukatxrAgutAtxre.
\end{artha}

\vishaya{``yadidaM savaRM pUvaRpakaSx'' - itAyxdiyAgi mADida parxshenxge ``barxhamxNA QutivxjA manasAcaMderxVNa'' eMba mAdhayxMdina shurxtiyiMda utatxravanunx heVLutAtxre ---}

\begin{shl}
kalAvaqdidhxkaSxyABAyxM tu pakaSxyoVruBayoVsatxthA |\\
canadxrXH kateVRha tatApxrXpAtxyX pakASxBAyxM viparxmucayxteV \hfill || 35 ||
\end{shl}

\begin{artha}
shukalx pakaSx, kaqSaNxpakaSxgaLeraDaralilx kalegaLu vaqdidhxyAguvudU kaSxyisuvudU kANuvudariMda A mUlaka caMdarxnu pakaSxgaLanunx nimiRsuvanu, \footnote[1]{caMdarxnalilx vaqdidhx kaSxyagaLu iruvudariMda shukalxpakaSx, kaqSaNxpakaSxgaLalU avugaLu kANuvudariMda I sAdaqshayxdiMda caMdarxnu pakaSx nimARpakaneMdu heVLuvudu, barxhamxneMba QutivxjananUnx, manasasxnUnx caMdarxrUpaveMdu dhAyxnisuvudariMda pakaSxgaLa rUpadalilxruva maqtuyxviniMda yajamAnanU barxhamxnU biDugaDe hoMduvadeMdathaR.}caMdarxnanunx hoMduvudariMda eraDu pakaSxgaLiMda Itanu mukatxnAguvanu.
\end{artha}

\vishaya{13. `udAgxtArx QutivxjA vAyunA pArxNeVna' eMba kANavx shurxtiya tAtapxyaR ---}

\begin{shl}
hArxsavaqdodhxyXVyaRtaH katAR vAyushacxnadxrXmasasatxtaH |\\
vAyuneYvoVpasaMhAraH pArxNoVdAgxtorxVrayaM kaqtaH \hfill || 36 ||
\end{shl}

\begin{artha}
caMdarxna hArxsakUkx vaqdidhxgU vAyu veV kataRnAdadxriMda vAyuvanenxV iTuTx pArxNavAyu matutx udAgxtaqveMba Qutivxjara viSayadalilx \footnote[2]{sUtArxtamxnAda vAyuvu tananx avayavavAda caMdarxna vaqdidhxkaSxyAdigaLige kataRnAdanu, yajamAnana pArxNavAyu, matutx udAgxtaqveMba Qutivxja I eraDanunx vAyurUpaveMdu upasaMhAra mADiruvudu kANavx shurxtiyalilx kANutatxde, adariMda vAyurUpadalilx udAgxtaq, pArxNa, ivugaLanunx dhAyxnisuvudariMda udAgxtaqvU yajamAnanU shukalxpakaSx, kaqSaNxpakaSxgaLeMba maqtuyxviniMda mukatxrAgutAtxre -- eMdu tAtapxyaR.}upasaMhAra mADide.
\end{artha}

\vishaya{ideVnu ? virudadhxvAgi eraDu bageyalilx dhAyxnavanunx heVLide ? heVge anuSAThxna mADuvudu ? eMdare --- utatxra ---}

\begin{shl}
manoV\s dhAyxtamxM yadasAyxBUdabxrXhAmx ceYvAdhiyajacnxgaH |\\
tasAyxdhideVvatA canadxrX iti mAdhayxMdinashurxtiH \hfill || 37 ||
\end{shl}

\begin{artha}
shariVradoLagina manasusx AdhAyxtamx, adhiyajacnxdalilxruva barxhamxnU I eraDakUkx adhideVvate (aBimAni deVvateyu) caMdarxneMdu mAdhayxMdina shurxtiyu heVLiruvudu\footnote[1]{idariMda shAKegaLu beVreyAgidadxdadxriMda AyAya shAKeyavaru uditeV hoVma | anuditeV hoVma | eMbuvaMte vikalapxvAgi yAvudAdaroMdanunx dhAyxnamADabahudeMdu tAtapxyaR.}.
\end{artha}

\begin{shl}
kamaRtaH kAlatoV maqtoyxVmuRkotxV\s yaM savxgaRmeVSayxti |\\
taM parxyAsayxti keVnAyamAkarxmeVNeVti paqcaCxyXteV \hfill || 38 ||
\end{shl}

\begin{artha}
kamaRdiMdalU kAladiMdalU I bageya maqtuyxviniMda biDugaDe hoMdida Itanu savxgaRvanunx hoMduvanaSeTx, ivanu yAva AkarxmadiMda A savxgaRkekx hoVguvuneMdu `keVna-AkarxmeVNa' eMdu parxshenx mADide.
\end{artha}

\vishaya{Akarxma shabadxda athaRveVnu ? matutx parxshAnxthaRda vivaraveVnu ? ---}

\begin{shl}
savxgaRloVkagatAvatarx sAdhanaM paqcaCxyXteV yataH |\\
anatxrikaSxmanAdhAraM gatw heVtuniSeVdhanamf \hfill || 39 ||
\end{shl}

\begin{artha}
savxgaRloVkakekx hoVgalu sAdhanaveVneMdu parxshinxside. EkeMdare ? aMtarikaSxvu AdhAravilalxdaMte ide. adariMda hoVguvudakekx sAdhanaveV ilalxveMdu toVruvudu, adariMda parxshenx mADide.
\end{artha}

\begin{artha}
BataqR parxpaMcaru AraMbaNa AkarxmaNa shabadxgaLige shariVraveMba athaRvanunx koTiTxruvudariMda deVhavanunx tegedukoLuLxva viSayadalilx I parxshenxyeMdu vAyxKAyxnisiruvaru - adanunx ivaru nirAkarisutAtxre ---
\end{artha}

\vishaya{14. BaqtaqR parxpaMcada vAyxKAyxna nirAkaraNe}

\begin{shl}
na tu deVhagarxheV parxshanxH sati ganatxri paqcaCxyXteV |\\
agAnxyXdidaqSiTxBishAcxsayx deVhaH pArxkapxrXtipAditaH \hfill || 40 ||
\end{shl}

\begin{artha}
idu deVha garxhaNada viSayadalilx baMda parxshenxyalalx, EkeMdare ? - hoVguvavanidadxre A parxshenx mADatakakxdudx, matutx `soV\s ginxrabhavatf' itAyxdiyAgi cakuSxrAdi iMdirxyagaLanunx aginx muMtAda BAvanegaLiMda baruva deVhavanunx hiMdeye (udigxVtha bArxhamxNadalelx) parxtipAdiside. adariMdalU I parxshenxyu deVhagarxhaNadadxlalx.
\end{artha}

\vishaya{15. barxhamxNA itAyxdi parihAra vAkayxda tAtapxyaR ---}

\begin{shl}
manoV\s dhiyajacnxM barxhemxYva barxhAmx canodxrXV\s dhideYvatamf |\\
canedxrXVNa manasA loVkamavaSaTxmeBxVna yAsayxti \hfill || 41 || 
\end{shl}

\begin{artha}
yajamAnana manasusx adhAyxtamx, barxhamxneMba Qutivxkf adhiyajacnx (yajacnxdalilxruvadu) caMdarxnu adhideYva (deVvAMsha) ivugaLalilx caMdarxrUpadalilx dhAyxnisida manasusx barxhamxveMbuva avalaMbaneyiMda (yajamAnanU barxhamxneMba QutivxjanU) barxhamxloVkakekx hoVguvaru.
\end{artha}

\begin{shl}
adhAyxtamxM pArxNa Eva sAyxdudAgxtA yoV\s dhiyajacnxgaH |\\
sa vAyuriti pATheV sAyxdAvxyXKAyx mAdhayxMdineV tivxyamf \hfill || 42 ||
\end{shl}

\begin{artha}
yajamAnana pArxNavAyuveV adhAyxtamx, udAgxtaq eMbuva QutivxjaneV adhiyajacnxdalilxruvanu \footnote[1]{`udAgxtarxtivxRjA vAyunA pArxNeVna' eMba mAdhayxMdina pAThadaMte yajamAnana pArxNavAyuvanunx udAgxtaqveMba Qutivxjananunx vAyuveMba daqSiTxyiMda dhAyxna mADidavanu barxhamxloVkavanunx seVrutAtxneMdu athaR.}`sa vAyuH' eMba mAdhayxMdina shAKeya pAThadalilx I vAyxKAyxnavu Aguvudu.
\end{artha}

\vishaya{16. I upAsanegaLanunx yajamAna mADuvade ? Qutivxjaru mADuvade ? eMbudara niNaRya}

\begin{shl}
yajamAnashurxteVratarx pUvoVRdigxVtheYkavAkayxtaH |\\
yajamAnoV japasatxtarx jAcnxnaM neVtayxvasiVyatAmf \hfill || 43 ||
\end{shl}

\begin{artha}
udigxVtha bArxhamxNadalilx yajamAnaneV japakekx sAvxmi, (avaneVkataqR, jAcnxnavAdaroV yajamAnanige saMbaMdhisidadxrU avanu mADatakakxdadxlalx, adakekx Qutivxjanu mADatakakxdedxMdu heVLidadxriMda) ililxyU yajamAnaneV jAcnxnakekx sAvxmi, adanunx mADuvavanu QutivxjaneV, horatu sAvxmiyalalxveMdu tiLiyabeVku, yajamAna shabadxvu iruvudariMdalU udigxVtha vAkayxdoDane EkavAkayxteyu iruvudariMda hiVgeyeV tiLiyabeVku.
\end{artha}

\begin{shl}
saMBAvayxteV na yatatxtarx yajamAnasayx mAnataH |\\
asutx kAmaM tadudAgxtunaRtu tadayxjamAnagamf \hfill || 44 ||
\end{shl}

\begin{artha}
yAvudu A upAsanegaLanunx yajamAnanige parxmANadiMda saMBAvitavAgilalxvo, alilx A jAcnxnavu udAgxtaqvige irali, Adare yAva udigxVthAdi jAcnxnavu yajamAnanalilx iruvudo. adu QutivxjanadAgilalx.
\end{artha}

\begin{shl}
itiVtuyxkatxparAmashoVR hayxtideVshAthaR ucayxteV |\\
itoV\s nayxtArxtimoVkASx yeV teV\s peyxVvamiti viVkaSxyXtAmf \hfill || 45 ||
\end{shl}

\begin{artha}
hiVgeMdu heVLida athaRvaneVnx itiyeMbudu parAmashaR mADide adu idakUkx beVre sathxLadalilx ati deVsha mADuvudakekx heVLide `ati moVkASxH' - aMdare (heVge vAgAdigaLu aginx muMtAda rUpadalilx dhAyxnisalapxTaTxvugaLAgi maqtuyxviniMda ati mukitxge kAraNavAgiveyoV, hAgeye anukatxvAda tavxgiMdirxya modalAdavugaLu vAyu muMtAda rUpadalilx kaMDiruva ivu atimoVkaSxkekx kAraNavAgideyeMdeV tiLiyabeVku.
\end{artha}

\vishaya{ati deVshavaneVnx visatxrisutAtxre ---}

\begin{shl}
adhideYvAtamxnA teVSAM daqSaTxyaH sAdhanAtamxnAmf |\\
atimoVkASxH suyxH savaRtarx yathoVkAtxdeVva lakaSxNAtf \hfill || 46 ||
\end{shl}

\begin{artha}
sAdhanarUpavAda vAgAdigaLanunx adhideYva, aginx muMtAda deVvatA savxrUpadiMda noVDuvalilx A daqSiTxgaLu maqtuyxvanunx atikarxmisi biDugaDe hoMdalu kAraNavAguvavu, hAgeye tavxgiMdirxya modalAdavugaLanunx hiMde heVLida lakaSxNadaMte vAyu muMtAda deVvatA savxrUpadiMda kaMDalilx A daqSiTxgaLu (dhAyxnagaLu) ati moVkaSxvAguvavu (maqtuyxvanunx dATi moVcanege kAraNavAguvavu).
\end{artha}

\vishaya{17. atha saMpadaH eMbalilx saMpatf padada athaR}

\begin{shl}
PalavatakxmaRNAM kAvxpi kiMcitAsxmAnayxsaMsharxyAtf |\\
saMpatitxmahaRtAM saMpadalipxVyaHkamaRsUcayxteV \hfill || 47 ||
\end{shl}

\begin{artha}
\footnote[1]{yathAshakitxyAgi aginxhoVtArxdi alapxkamaRgaLanunx AcarisuvudariMda ashavxmeVdhAdi yAgagaLe naninxMda AcarisalapxDutatxveyeMdu dhAyxnisuvudeV saMpatutx.}PalavuLaLxkamaRgaLa (ashavxmeVdhAdi mahAkamaRgaLa) kamaRveMbuva alapxsAdaqshayxda avalaMbaneyiMda saNaNxkamaRgaLalilx (aginxhoVtarx modalAda kamaRgaLalilx) saMpAdaneyu (aBinanxvAgi mahAkamaRgaLeMdu dhAyxna mADuvudeV) saMpatf eMbudAgi tiLiyabeVku.
\end{artha}

\vishaya{`PalaseyxYva vA' eMba BASayxda aBipArxya ---}

\begin{shl}
yadi vA tataPxlaseyxYva kiMcitAsxmAnayxvatamxRnA |\\
saMpAdanaM BaveVtasxMpadaginxhoVtArxdikamaRNi \hfill || 48 ||
\end{shl}

\begin{artha}
athavA adara \footnote[2]{athavA vihitavAda adhayxyanavu athaRjAcnxna-anusAThxnAdigaLa mUlaka Palavanunx koDuvudariMda sAthaRkaveMbudu shAsatxrXsaMmatavAgide, Adare ashavxmeVdha-rAjasUya muMtAda mahAyAgagaLalilx bArxhamxNarigU veYshayxrigU adhikAraviruvudilalx, kaSxtirxyarige mAtarx adhikAravide, adariMda ashavxmeVdhAdi kAMDAdhayxyana mADuvudu bArxhamxNarigU veYshayxrigU vayxthaRvAguvudeMba shaMkeyanunx mADabAradu. ashavxmeVdhAdi mahAyAgavanunx dhAyxnisuvadariMdale adara Palavu sididhxsuvudeMdu heVLabeVkAgide. (2) PalaveMdare kamaRgaLiMda baruva deVvaloVkAdi lABa, avugaLalilxruva ujavxlate muMtAda sAdaqshayxvu kamaRdalilx upayoVgisuva AjAyxdi AhutigaLalilx iruvadariMda PalABeVdavanunx ciMtisuvudeV saMpatetxMdu matotxMdu athaRvu.}Palada sAdaqshayxvu savxlapxvAgi iruva mAgaRdiMda aginxhoVtArxkamaRgaLalilx (ashavxmeVdha modalAda mahAkamaRgaLanunx oMdAgi dhAyxnisuvudeV) saMpatutx.
\end{artha}

\begin{shl}
saMpadA ceVtaPxlapArxpitxrashavxmeVdhAdikamaRNAmf |\\
tarxyANAmapi vaNARnAM tatApxThaH PalavAnaBxveVtf \hfill || 49 ||
\end{shl}

\begin{artha}
ashavxmeVdha modalAda kamaRgaLa Palavu saMpatitxniMda (aBeVdadhAyxnadiMdaleV) laBisuvudAdare mUru vaNaRdavarigU (bArxhamxNa-kaSxtirxya-veYshayxrigU) ashavxmeVdhAdikAMDanunx adhayxyana mADuvudu saPalavAguvudu. alapxdhAyxnadiMda ashavxmeVdhAdi mahAyAgagaLa Palavu heVge ? baruvudu ? `- eMdare ---
\end{artha}

\begin{shl}
nAtiBAroV\s sitx noV budedhxVH shAsatxrXM ceVtatxtapxraM BaveVtf |\\
viduSAM sherxVyaseV\s toV\s dhAvx na kavxcitapxrXtihanayxteV \hfill || 50 ||
\end{shl}

\begin{artha}
atha saMpada:- itAyxdi shAsatxrXvu A tAtapxyaRvuLaLxdeV Adare namamx budidhxge  aMdare shAsAtxrXthaRvanunx tiLiyuvudakekx BAra athavA AyAsaveVnU hecAcxgi Aguvudilalx, vidAvxMsara sherxVyasisxgAgi heVLuva mAgaRvu elilxyU (bArxhamxNAdigaLu ashavxmeVdhakAMDapATha mADuva viSayadalUlx) kuMThitavAguvudilalx.
\end{artha}

%~ \section*{baq. - a. 3 - kaMDike 7}
\kandike{kaMDike 7}

\begin{kandikeshl}
yAjacnxvalekxyXVti hoVvAca katiBirayamadayxgiBxRhoVRtAsimxnayxjecnxV kariSayxtiVti tisaqBiriti katamAsAtxsitxsarx iti puroVnuvAkAyx ca yAjAyx ca shaseyxYva taqtiVyA kiM tABijaRyatiVti yatikxcnecxVdaM pArxNaBaqditi || 7 ||
\end{kandikeshl}

\vishaya{vAtiRka}

\begin{shl}
tisaqBiriti saMKAyxthaRparxshanxniNaRyamabarxviVtf |\\
puroVnuvAkAyxduyxkAtxyX tu saMKeyxVyAthaRviniNaRyamf \hfill || 51 ||
\end{shl}

\begin{artha}
`tisaqBi' eMdu utatxravAkayxdalilx saMKeyx eSeTxMba parxshenxge takakx niNaRyavanunx heVLide. puroVnuvAkAyx, yAjAyx, shasAyx eMbuva vacanadiMda, A mUru yAvudeMbudakekx parigaNisida viSayagaLa niNaRyavanunx heVLiruvudu.
\end{artha}

\begin{shl}
terxYloVkayxsaMKAyxsAmAnAyxtAsxyXtasxvaRpArxNaBaqjajxBaH |\\
savaRpArxNaBaqtAM yasAmxtirxrXSevxVveYteVSu saMBavaH \hfill || 52 ||
\end{shl}

\begin{artha}
mUru loVkagaLa saMKeyxyu samAnavAgiruvudariMda elAlx pArxNigaLa  jayavu saMpatitxniMda Aguvudu, heVgeMdare samasatx pArxNigaLU I mUru loVkagaLaleVlx seVrikoMDiruva kAraNadiMda (loVkoVpAsaneyiMda savaRpArxNigaLa jayavu yukatxveV Agide).
\end{artha}

\begin{shl}
ujajxvXlatAvxdisAmAnAyxdedxVvaloVkAdisaMpadaH |\\
PalasaMpada EveYtA nAtarx kamaR vivakaSxyXteV \hfill || 53 ||
\end{shl}

\begin{artha}
deVvaloVkAdi saMpaMtutxgaLu ujavxlavAgiruvudeV modalAda sAdaqshayxdiMda Pala saMpatutxgaLeV ivu, ililx kamaRvu vivakiSxtavalalx.
\end{artha}

\vishaya{Pala sAdaqshayxda vivaraNe ---}

\begin{shl}
diVpitxnARdoV\s dhaHshayanamAjayxmAMsapayoVmaBxsAmf |\\
deVvaloVkAdisaMpatAsxyXdidxVpitxmatAtxvXdisaMBavAtf \hfill || 54 ||
\end{shl}

\begin{artha}
hoLapu tupapxdalUlx, shabadxvu mAMsadalUlx, BUmiya taLadalilx nelasuvudeMbudu kiSxVra, niVru ivugaLalUlx kANuvudu. \footnote[1]{ajayx muMtAda AhutigaLalilx parxkAshisuva deVvaloVka saMpatatxnUnx hoVma mADida mAMsa muMtAda AhutigaLalilx caTapaTa eMba shabadxvu AguvudariMda pitaqloVkada saMpatatxnUnx soVmarasa kiSxVra muMtAdavugaLalilx AhutigaLu nelada meVle hoVgi nilulxvadariMda manuSayxloVkasaMpatatxnUnx, `I AhutigaLanunx koDuvAga ivu deVvaloVka, pitaqloVka manuSayxloVkagaLAgi pariNamisutatxveyeMdu BAvisuvudariMda. adariMda loVkajayavu sididhxsuvudeMdu tAtapxyaR, idaralilx pitaqloVkadalilx shabadxveMdare I loVkakekx saMbaMdhisida saMyaminiV paTaTxNadalilx yamaniMda yAtane paDutitxruva jiVvara `biDu biDu' hAyf kAyf eMba AtaRnAda kUgATavu keVLibaruvudariMda pitaq, loVkadalilx shabadxvu ideyeMdu tiLidu adara sAmayxvanunx mAMsAdi AhutigaLalilx garxhisikoLaLxbeVku. I bagegx idara mUlaBASayxveV AdhAra (3-1-8).}I diVpitx muMtAdavu deVvaloVkAdigaLalUlx saMBasuvudariMda deVvaloVkAdi saMpatf (dhAyxna)vu (ajAyxdi AhutigaLalilx) nAyxyavAgide.
\end{artha}

\vishaya{18. `ananatxM veYmanaH' - eMbudara athaR ---}

\begin{shl}
vaqtAtxyXnanAtxyXnamxnoV\s nanatxM shabAdxdiVnAmananatxtaH |\\
vishevxVdeVvAnamxnoVvaqtitxVH saMpAdAyx\s \s nanatxyXsAmayxtaH \hfill || 55 || 
\end{shl}

\begin{shl}
ananatxmeVva teVnAsw samayxgAjxcnXneVna vinadxti |\\
loVkaM yathoVkatxdaqSiTxH sanayxjamAnaH PalaM savxyamf \hfill || 56 ||
\end{shl}

\begin{artha}
shabAdxdi viSayagaLu anaMtavAgi bahaLavAgiruvudariMda manoVvaqtitxgaLU anaMtavAgiruvavu, adariMda manasUsx anaMtaveMdu heVLi adara manoVvaqtitx	gaLalilx anaMta guNasAmAyxdiMda vishevxVdeVva deVvatABAvaneyanunx mADabeVku adariMda I utatxma daqSiTxyiMda Itanu yajamAnanu tAneV anaMtaloVkavanunx paDeyuvanu.
\end{artha}

\vishaya{``sotxVtirxyAH sotxVSayxti'' --- eMbalilx sotxVtirxya shabAdxthaR ---}

\begin{shl}
tisarxH puroVnuvAkAyxdAyx QucaH pUvaRmudiVritAH |\\
yAsAtx EvAtarx vijecnxVyAH sotxVtirxyA api nAparAH \hfill || 57 ||
\end{shl}

\begin{artha}
yAva puroVnuvAkAyx muMtAda mUru Qukf maMtarxgaLu hiMde heVLalapxTaTxvo A QkukxgaLaneVnx ililx sotxVtirxyaveMdU karedide, ivu beVreyalalx.
\end{artha}

\begin{artha}
`katamA sAtx yA adhAyxtamxmf' --- idara parxshAnxthaRvanunx biDisuvadu ---
\end{artha}

\begin{shl}
giVtayasatxvXdhiyajacnxM tA adhAyxtamxM kAsutx tA iti |\\
pArxNApAnavAyxnarUpA adhAyxtamxM tAH parxcakaSxteV \hfill || 58 ||
\end{shl}

\begin{artha}
adhiyajacnxda giVtegaLu avu yAvuvu ? matutx adhAyxtamx niVtigaLu yAvuvu? eMba parxshenx. utatxra - pArxNa, apAna, vAyxna eMbavugaLeV adhAyxtamxvAda niVtigaLeMdu heVLuvaru.
\end{artha}

\begin{shl}
jayoV BUrAdiloVkAnAM saMKAyxditAvxdisAmayxtaH |\\
ashavxloV\s puyxpareVmeV\s tha sovxVkatxparxshanxviniNaRyAtf \hfill || 59 || 
\end{shl}

\begin{artha}
BU muMtAda loVkagaLigU saMKeyx modalAda sAdaqshayx A loVkagaLa jayavu AguvudeMdu tAnu keVLida parxshenxya niNaRyavu baMdiruvudariMda ashavxlanu sumamxnAdanu.
\end{artha}

\vishaya{iti shirxV baqhadAraNayxkoVpaniSadf BASayx vAtiRkadalilx modalaneya bArxhamxNa ashavxla bArxhamxNavu samApitxgoMDide.}
