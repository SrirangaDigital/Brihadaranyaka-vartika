\section*{baq 4 a- ide bArx-vA (963 - 1106)}

\section*{baq. 4-3 neV bArxhamxNa vAtiRka- 93 riMda 1109 varege}

\vishaya{muMde heVLuva nAyxyavanunx saMkeSxpisi nenapukoDuvaru-}

\begin{shl}
na jAgarxdedxVshagaH savxponxV jAgarxtAsxmagarxyXBAvataH | \\
adhAyxtAmxdayxthaRviraheV yataH savxpAnxnapxrXpashayxti \hfill ||  963 ||  
\end{shl}

\begin{artha}
savxpanxvu jAgara sathxLadalilx (biVLuvudalalx) athavA iruvudilalx EkeMdare? jAgarakAlada sAmagirxyu iruvudilalxvAdadxriMda, shariVra iMdirxyAdi viSayagaLu ilalxdiruvAga I savxpanxgaLanunx noVDuvanu.
\end{artha}

\vishaya{pUvAR para saMbaMdha}

\begin{shl}
savxyaMjoyxVtiSaTxvXmasoyxVkatxM savxpenxV cAkArakAtamxtA | \\
maqtuyxrUpAtayxyashecxYvaM parxtiVcaH pUvaRvAkayxtaH \hfill||  964 ||  
\end{shl}

\begin{artha}
I Atamxnige savxyaM joyxVti savxrUpavanunx heVLidAdxyitu matutx savxpanxdalilx kAraka rUpanAgilalxveMdU maqtuyxvina rUpagaLanunx dATiruvaneMdU IriVtiyAgi hiMdina vAkayxdiMda (`atArxya puruSaH savxyaM joyxVtiH' `na tatarx rathA na rathayoVgAH' `atikArx mati maqtoVrUpANi' I riVtiyAda vAkayxdiMda) heVLidAdxyitu.
\end{artha}

\vishaya{parxshenxya tAtapxyaR}

\footnotetext[1]{`maqtoyxV rUpANi atikArxmati' eMdu heVLidadxriMda kAmakamaR modalAda maqtuyxvu Atamxna savxBAvave? athavA alalxve? eMdu shaMkisi savxBAvaveV eMdAdare mokaSxveV baruvudilalxveMdu tiLida rAjanu adivxtiVya niNaRyakAkxgi `ata UdhavxRMvimoVkASxyeY bUrxhiVti' eMdu punaH parxshinxsidaneMdu aBipArxya.}
\begin{shl}
\footnotemark[1]kAmAdayaH savxBAvoV\s sayx maqtayxvoV\s tha na vA BaveVtf | \\
itAyxshaknakxyX naqpoV doVSaM yAjacnxvalakxyXmapaqcaCxta\hfill ||  965 ||  
\end{shl}

%%%%%%shloka footnote[1]
\begin{artha}
kAmAdi maqtuyxgaLu Atamxna savxBAvavo? athavA alalxvo? eMdu shaMkisi rAjanu yAjacnxvalakxyXranunx kuritu doVSavanunx parxshinxsidanu.
\end{artha}

\begin{shl}
paqSaTxM vasutx suniNiVRtaM mayeVtiparxkaqtoVkitxtaH | \\
nivivaqtusxM naqpoV viparxM BUyoV\s paqcaCxdivxmukatxyeV\hfill ||  166 ||  
\end{shl}

\begin{artha}
rAjanu `shatama AtAmx' eMdu hiMde parxshinxsida vasutxvu naninxMda `yAliyamf' itAyxdi parxkaqta vacanadiMda niNaRyisalapxTiTxdeyeMdu hiMdakekx hoVgalu bayasida bArxhamxNananunx (muniyanunx) kuritu punaH moVkaSxkAkxgi parxshinxsidanu.
\end{artha}

\vishaya{budidhx modalAdavugaLige beVre yAda Atamxnu savxyaM joyxVtiyeMdu heVLidadxriMdaleV mukitxyu sididhxsalu adakAkxgi punaH parxshinxsuvudu yukatxvalalx, alalxveV? eMdu shaMkisidare utatxra-}

\begin{shl}
savxyaMjuyxtiSaTxvXmuditaM muketxVraknagxmamanayxta\hfill||  967 | \\
na cAknagxniNaRyoVketxyXYva niNaRyaM manayxteV\s knigxnaH | \\
yatoV\s toV moVkaSxmudidxshayx yAjacnxvalakxyXmapaqcaCxta \hfill||  968 ||  
\end{shl}

\begin{artha}
yAvudariMda hiMde heVLida savxyaM joyxVti savxrUpavu mukitxge aMgaveMdu tiLididadxnoV, aMga niNaRyavanunx heVLidadxriMdaleV aMgiya (parxdhAna) niNaRyavu AyiteMdu tiLiyalilalxvo, adariMda moVkaSxvaneVnx udedxVshisi yAjacnxvalakxyXranunx parxshinxsidanu.
\end{artha}

\vishaya{Iga utatxra vAkayxda tAtapxyaRvanunx heVLuvaru-}

\begin{shl}
AsaknagxparxviveVkAthaRM sa vA itAyxdi BaNayxteV | \\
kAmAdiparxviveVkeV hi moVkaSxyoVgayxtavxmAtamxnaH \hfill||  969 ||  
\end{shl}

\begin{artha}
AsaMga sAvxBAvikavAda viSaya saMga idara viveVcanegAgi `sa vA ESa Etasimxnf saMparxsAdeV.......' itAyxdi maMtarxvu heVLalapxTiTxde, EkeMdare kAmAdigaLanunx viveVcane mADidare Atamxnige moVkaSxkekx beVkAda ahaRteyu baruvudu.
\end{artha}

\begin{shl}
kAmAdAyxtamxguNashecxVtAsxyXdanimoVRkaSxH parxsajayxteV | \\
na hi guNayxvinAsheVna tadugxNasayx nirAkaqtiH\hfill ||  970 ||  
\end{shl}

\begin{artha}
kAmAdigaLu AtamxguNaveV Agidadxre moVkaSxvu bArade hoVguvudu. EkeMdare, \footnote{niNaRyA GATa-eMdare nidiRSaTx viSayadalilx tananx niNaRyada daqDhate samasatx parxshenxgaLanunx niNaRya mADidadxriMda koTaTx dAnavalalx} guNavuLaLx vasutx nAshavAgade adara vAsatxvika guNavu nashisuvudilalx.
\end{artha}

\vishaya{maqtuyxvina rUpagaLanunx atikarxmisiruvudeMbudu modaleV tiLidide, muMdina vAkayxveVke? eMdare-}

\begin{shl}
rUpANeyxVvAyamAtAmx\s tarx maqtoyxVH savxpenxV\s tivataRteV | \\
na tu maqtuyxM yataH savxpenxV moVdatArxsAdi daqshayxteV \hfill||  971 ||  
\end{shl}

\begin{artha}
I Atamxnu savxpanxdalilx maqtuyxvina rUpagaLaneVnx dATiruvanu aSeTx. Adare maqtuyxvanunx dATiruvudilalx, EkeMdare? haSaR BayAdigaLU savxpanxdalilx kANutatxve (adariMda)
\end{artha}

\vishaya{parxshenxyanunx anavxyisuvudu-}

\begin{shl}
moVdatArxsAdikoV maqtuyxH savxBAvoV na yathA\s \s tamxnaH | \\
tathA\s ta UdhavxRM parxbUrxhi BagavanemxV vimukatxyeV \hfill||  972 ||  
\end{shl}

\begin{artha}
haSaR, Baya modalAdavugaLeV maqtayx, idu Atamxna savxBAvavalalxveMbudu heVgo hAgeye `ata UdhavxRM bUrxhi.....' pujayxreV ililxMda muMdakekx moVkaSxkAkxgi beVkAda viSayavanunx heVLi eMdu parxshinxside.
\end{artha}

\vishaya{parxshenxyalilx bAkiyidadxre sahasarx suvaNaRdAna mADuvudu heVge saMgata?}

\begin{shl}
sahasarxdAnaM tUkatxsayx savxyaMjoyxVtiSaTxvXvasutxnaH | \\
niNaRyAGATavijacnxpetxyXY nAsheVSaparxshanxniNaRyAtf \hfill||  973 ||  
\end{shl}

\begin{artha}
sahasarx dAnavAdarU hiMde heVLida savxyaM joyxVti vasutxvina niNaRyavu daqDhavAgide eMbudanunx tiLisidadxkekx, idu
\end{artha}

\vishaya{`sa E Sa saMparxsAdaH' padagaLige AvashayxkavAda padagaLanunx adhAyxhAra mADiyoVjisuvudu.}

\begin{shl}
savxyaMjoyxVtiH sa vA ESa yaH purA parxtipAditaH | \\
savxpanxBUmAvasaknogxV\s sw \footnotemark[1]saMparxsAdeV parxsiVdati \hfill||  974 ||  
\end{shl}
\footnotetext[1]{saMparxsAda eMdare suSupitx.}

%%%%%shloka footnote[1]
\begin{artha}
yAva I Atamxnu hiMde savxyaM joyxVtiyeMdu parxtipAdisalapxTiTxruvano avaneV ivanu savxpanx dasheyalilx asaMganAgi suSupitxyalilx parxsananxvAgiruvanu.
\end{artha}

\vishaya{savxyaM joyxVtiyAda Atamxnu parxsananxnAgiralu kAraNavanunx daqSATxMtadiMda heVLuvaru-}

\footnotetext[2]{kataka eMdare cilalx biVja, adara puDiyeMdu ililx athaR. niVrige I puDiyanunx hAkidare koLeyanunx tegedukoMDu I puDiyu taLadalilx nilulxvudu. meVle nimaRlavAda tiLi niVru nilulxvudu, hAgeye samasatx saMsAra vAsanegaLu dUravAdalilx savxsathxteyu uMTAguvudu idadx sithxti toVruvudu.}
\begin{shl}
apAM \footnotemark[2]katakasaMpakARdayxthA\s tayxnatxparxsananxtA | \\
apAsAtxsheVSasaMsAraBAvanasyxvamAtamxnaH \hfill||  975 ||  
\end{shl}
				
\begin{shl}
sAvxsathxyXM parxsananxteYtasimxnusxSupetxV BavatiVtayxtaH | \\
saMparxsAdamimaM pArxhuH suSupatxM tadivxdoV janAH \hfill||  976 ||  
\end{shl}

%%%%%shloka footnote[2]
\begin{artha}
niVrige cilalx biVjada puDiyanunx hAkuvudariMda adu bahaLa nimaRlavAguvudu adaraMte samasatx saMsAra vAsanegaLu dUravAgiruva Atamxnige savxsathxteyu (savxrUpasithxtiyu) yAvuduMTo adeV I suSupitxyalilx parxsananxteyeMbuvudu uMTAguvudu, adariMda adanunx tiLida janaru (pArxjacnxru) I suSupitxyanunx saMparxsAdaveMdu heVLuvaru.
\end{artha}

\vishaya{saMpArxsAdaveMbudakekx suSupitxyeMbathaR mADalu beVre kAraNavU ide-}

\begin{shl}
mAtArxdAnaM yadakaroVdAtAmx savxpanxriraMsayA | \\
tatakxSXyeV sa nirAsaknagxH sAvxtamxneyxVva parxsiVdati \hfill||  977 ||  
\end{shl}

\begin{artha}
Atamxnu yAva jAgara vAsanegaLanunx savxpanxdalilx ramisalu iciCxsi tegedu koMDiruvano adu nAshavAgalu Atanu saMga shUnayxnAgi tananx savxrUpadalelxV niMtu shAMtanAgiruvanu.
\end{artha}

\vishaya{`parxsiVdati' eMdare sAku, `samf' eMba upasagaR veVtakekx eMdare-}

\begin{shl}
jAgarxdavxyXpeVkaSxyA savxpenxV kiMciceCxVSaH parxsiVdati | \\
tasAyxpayxsatxmayAtApxrXjecnxV samiti sAyxdivxsheVSaNamf \hfill||  978 ||  
\end{shl}

\begin{artha}
jAgara dashegiMta savxpanxdalilx savxlapx vAsaneyuLidu savxlapx parxsananxnAguvanu, A vAsaneyU hoVgiruvudariMda pArxjacnxnalilx `samf'eMba visheVSaNavu.
\end{artha}

\begin{shl}
keVvalAjAcnxnamAtArxdhiriha parxtayxknf vayxvasithxtaH | \\
kAraNAtAmx yatasatxsAmxjAjxgarxtasxvXpAnxKayxkAyaRkaqtf \hfill||  979 ||  
\end{shl}

\begin{artha}
keVvala\footnote{keVvala eMdare sUthxla kAyaRgaLalxvu ilalxde eMdathaR. mAtarx eMdare vAsaneyU ilalxde eMdathaR. iMtaha mUlAjAcnxnavidudx A upAdhiyiMda ivanu kAraNaneMdu heVLiruvudu, ililx mUlAjAcnxnavu suSupitxyalilxdeyeMdu heVLidudx avaqtatxvAgi sAkiSx mAtarxkekx gotAtxgidudx edadx meVle suPxTavAgi `na jAnAmi' eMdu toVruvudu. idakekx kAraNa AvAga vayxMjakavAda aMtaHkaraNavilalxdiruvudu, sAkiSxge aMtaHkaraNa sahAyavu baMdoDane ajAcnxnavu idadudx suPxTavAgi toVruvudu adu jAgaradalilx `nakeMcida veVdiSaM suKa mahamasAvxpasxmf' eMdu baruva ajAcnxnAnuBavaveV idakekx sAkiSx, sAkiSx ceVtanavu itara vasutxvanunx toVrisalu itara sahAyavanunx apeVkiSxsadidadxrU suPxTa parxtayxkaSxvAguvudakekx apeVkiSxsuvudu, ceYtanayxvu savaRsamAnayxvAgidadxrU iMdirxya sahAyadiMda gaMdhAdi visheVSa jAcnxnavanunx sapxSaTxvAgi mADi koLaLxlu GArxNAdi iMdirxyagaLanunx bayasuvaMte, `gaMdhAya GArxNaM' eMba shurxtiyu idakUkx sAkiSxyAgide, adaraMte ililxyU tiLiya beVku. vAtiRkakArarige suSupitx ajAcnxna biVjavu avayxkatxvAgi asuPxTa parxtayxkaSxvAgideyeMbudu samamxtave, neYSakxmayxR sididhxyalUlx (   ) puTadalilx noVDabahudu. `suSupAtxKayxM tamoV\s jAcnxnaM biVjaM savxpanx parxboVdhayoVH' eMbuva upadeVshasAhasirxV vacanavu idakekx AdhAra. biVjavilalxdidadxre savxpanx jAgaragaLu huTaTxlu avakAshaveV iruvudilalx.} ajAcnxnavoMdeV upAdhiyAgiruva parxtayxgAtamxnu I suSupitxyalilxdudx kAraNAtamxnAgiruvanu, adariMda jAgara savxpanxgaLeMba kAyaRvanunx punaH mADuvanu.
\end{artha}

\begin{shl}
kathaM keVna karxmeVNAtarx parxtAyxgAtAmx parxsiVdati | \\
iteyxVtadadhunA\s \s caSeTxVratevxVtAyxdigirA shurxtiH \hfill||  980 ||  
\end{shl}

\begin{artha}
heVge? yAva karxmadiMda' I parxtayxgAtamxnu ililx parxsananxnAgiruvanu hiVgeMba viSayavanunx Iga `ratAvx caritA' itAyxdi shurxtiyu heVLuvudu.
\end{artha}

\section*{baq-4-bArx-3. 15}

\begin{shl}
sa vA ESa EtasimxnasxmapxrXsAdeV ratAvx caritAvx daqSeTxvXYva puNayxM ca pApaM ca ||punaH parxtinAyxyaM parxtiyoVnAyxdarxvati savxpAnxyeYva sa yatatxtarx kicnicxtapxshayxtayxnanAvxgatasetxVna BavatayxsaknogxV hayxyaM puruSa iteyxVvameVveYtadAyxjacnxvalakxyX soV\s haM BagavateV sahasarxM dadAmayxta UdhavxRM vimoVkASxyeYva bUrxhiVti || 15 ||
\end{shl}

\vishaya{I shurxtiya padAthaRvanunx toVrisuvaru}

\begin{shl}
savxpanxBUmAvayaM ratAvx kirxVDAM kaqtAvx\s \s tamxmAyayA | \\
caritAvx ca vihaqtAyx\s \s tAmx vAsanAmAtarxsAdhanaH \hfill||  981 ||  
\end{shl}
				
\begin{shl}
daqSeTxvXYveVteyxVvashabedxVna kArakatavxM nivAyaRteV | \\
puNayxpApaPalaM ceVha puNayxpApagiroVcayxteV \hfill||  982 ||  
\end{shl}

\begin{artha}
vAsanA mAtarx sAdhanavuLaLxvavAgi I Atamxnu savxpanxsathxLadalilx ramisi kirxVDeyanunx mADi tananx mAyeyiMdale saMcarisi viharisi (pArxjacnxnalilx parxsananxnAgiruvanu) `daqSATxvXEva eMbudariMda kArakatavxvu nivArisalapxTiTxde, aMdare savxpanxdalilx noVDidedxV horatu mADidadxlalxveMdu tiLiya beVku, puNayx pApagaLa PalavU saha ililx puNayxpApa eMba mAtiniMda heVLalapxDuvudu.
\end{artha}

\begin{shl}
asayx jAgarxdavasAthxyAminidxrXyAthARdisaMBavAtf | \\
kirxyA nivaRtayxRteV sA ca suKAdiPaladA\s \s tamxnaH \hfill||  983 ||  
\end{shl}

\begin{artha}
I Atamxnige jAgarAvasethxyalilx iMdirxya viSaya iveV modalAdavugaLu saMBavisuvudariMda kirxyeyu huTuTxvudu, A kirxyeyu suKa modalAda Palavanunx koDuvudu.
\end{artha}

\begin{shl}
savxpenxV tu kArakABAvAnanx kirxyAsididhxrAtamxnaH | \\
cidABaM vAsanAmAtarxM parxthateV\s toV\s vikAriNaH \hfill||  984 ||  
\end{shl}

\begin{artha}
savxpanxdalilx kArakagaLu, ilalxvAdadxriMda Atamxnige kirxyeyu huTuTxvudilalx, adariMda vikAravAgada Atamxnige ceYtanayxdiMda toVruva vAsanA rUpavoMdeV parxsidadhxvAgiruvudu.
\end{artha}

\vishaya{EvakArada athaRvanunx upasaMharisutAtxre-}

\begin{shl}
yatoV\s tarx na kirxyA tasAmxdavadadherxV shurxtiH savxyamf | \\
daqSeTxvXYvA\s \s tAmx na kaqtevxVti parxtayxknf tasAmxnanx kArakaH \hfill||  985 ||  
\end{shl}

\begin{artha}
yAvudariMda I savxpanxdalilx kirxyeyu ilalxvo adariMdaleV shurxtiyu tAne `daqSeTxvXYva AtAmx, na kaqtAvx' eMdu (noVDidedxV horatu mADalilalx)veMdu (EvakAradiMda) avadhAraNa mADide, adariMda parxtayxgAtamxnu kArakanalalx (kataRnalalx).
\end{artha}

\begin{shl}
puNayxpApaPalaM ceVha puNayxpApABidhaM matamf | \\
kAyaRM kAraNavatatxsAmxdupacArAtasxmiVritamf \hfill||  986 || 
\end{shl}

\begin{artha}
puNayxpApa shabadxgaLa athaRvu puNayxpApagaLa PalaveMdu iSaTxvAgide adariMda kAyaRvanunx kAraNadaMte aupacArikavAgi heVLide.
\end{artha}

\begin{shl}
savxpanxkamaRvuyxparameV tataH pArxjecnxV savxyaMparxBaH | \\
parxsiVdati \footnotemark[1]paroV deVvaH parxtiVceyxVva\footnotemark[2] pareV padeV \hfill||  987 ||  
\end{shl}
\footnotetext[1]{`paroV deVvaH' eMdu jiVvAtamxnalilx vayxvaharisidudx paravasutxvigiMta jiVvanu beVre ilalxveMdu tiLisuvudakekx, hAgU}
\footnotetext[2]{EvakAradiMda kUDida `parxtiVceyxVva'eMba padavu suSupitx dasheyalilx paramAtamx noDane Atamxnu idAdxneMbudanunx heVLalu baMdide.}

%%%%%%shloka footnote[1, 2]
\begin{artha}
savxpanxkamaRgaLu niMtu hoVdare anaMtara nijavAgi para deVvateyAda (jiVvanu)
savxyaM parxkAshanAgi pArxjacnxnAda parxtayxgAtamxve Agiruva parama padaviyalilx (viSuNx savxrUpadalelxV) parxsananxnAgiruvanu.
\end{artha}

\vishaya{`punaH parxtinAyxyaM parxtiyoV nAyxdarxvati' eMbalilx punaH eMba padada athaR-}

\begin{shl}
purA\s payxsakaqdAtAmx\s yaM sAthxnAtAsxthXnAnatxraM gataH | \\
yatoV\s taH punariteyxVvaM saMparxsAdApitxtoV\s BidhA\hfill ||  988 ||  
\end{shl}

\begin{artha}
hiMdeyU I Atamxnu aneVkasala saMparxsAdalABa sAthxnavanunx biTuTx beVre savxpanx sAthxnavanunx hoMdidadxnu adariMda punaH eMdu shabadxvanunx heVLiruvudu.
\end{artha}

\begin{shl}
AnuloVmayxM suSupAtxpitxreVvamuketxVna vatamxRnA | \\
taduvxyXtAthxnaM parxtiloVmayxM \footnotemark[2]savxpAnxdisAthxnasaMcaraH \hfill||   989 ||  
\end{shl}
\footnotetext[2]{`parxti nAyxyamf' eMdare parxti=vipariVtavAgi hiMde heVLidakekx badalAgi nAyxya=savxpAnxdi sAthxnadalilx saMcarisuvudu eMdathaR.}

%%%%%%%shloka footnote[2]
\begin{artha}
I riVti hiMde\footnote[1]{EvaM eMdare hiMde heVLida mAgaR jAgaradiMda savxpanxkekx baruvudu anaMtara suSupitxge baruvudeMba anu loVmakarxmadiMda.} heVLida mAgaRdalilx suSupitxyanunx hoMduvudeV AnuloVmayx alilxMda ELuvudu aMdare adakekx vipariVtavAgi savxpanx modalAda sAthxnadalilx saMcarisuvudu.
\end{artha}

\vishaya{parxti shabAdxthaRveVneMdare-}

\begin{shl}
pArxtiloVmAyxthaR EvAtaH parxtishabodxV\s yamiSayxteV | \\
vipariVtA\s \s gatiH savxpenxV \footnotemark[3]parxtisAthxnaM tatheYtayxjaH \hfill||  990 ||  
\end{shl}
\footnotetext[3]{parxtiyoVni. Adarxvati eMbudara athaRvanunx saMkeSxVpiside I vAtiRkada koneyapAdadalilx, heVge suSupitxyanunx hoMduvanoV hAge savxpAnxdi BoVga sAthxnavanunx kuritu BoVgakAkxgi baruvaneMdathaR.}
%%%%%%shloka footnote[3]
\begin{artha}
parxtiloVmayx vipariVta eMba athaRdalelxV EvakAravide idariMda I parxtishabadxvidu, suSupitxyiMda vipariVtavAgi (savxpanx jAgaragaLige) baruvudu hAgeye Atamxnu parxti sAthxnakekx hAgeye baruvanu.
\end{artha}

\vishaya{`parxtinAyxyamf' eMbudara avayavAthaR}

\begin{shl}
yathAtheVR parxtishabodxV\s yaM nishacxyAtheVR niritayxyamf | \\
ayanaM gamanaM ca sAyxdAyoV dhAtoVriNoV Gacni \hfill||  991 ||  
\end{shl}

\begin{artha}
I parxti shabadxvu yathAthaRdalilxde (aMdare yAva riVti anuloVma karxmadiMda suSupitxge baruvudo hAgeye adakekx vipariVtavAda parxtiloVma karxmadiMda AgamanaveMbudu udidxSaTxvAgidadxlilx) parxyoVgisalapxTiTxde. ni eMbudu nishacxya veMbathaRdalilxde. Aya eMdare ayana aMdare gamana. iNfgatw\footnote[1]{iNf dhAtuvige BAvAthaRdalilx Gacnf parxtayxyavanunx hacacxlAgi GakAra cnakAragaLige itf saMjecnx baMdu loVpavAgalu akAra mAtarx uLiyuvudu i+a eMdiralu ikArakekx vaqdidhx ai eMdAgi AmeVle `EcoV yavAyAvaH' eMbudariMda Ayf eMba AdeVshavAgi Aya eMba rUpavusidadhxvAgide.}eMba dhAtuvige Gacna parxtayxyavu baMdare Aya eMdAgutatxde.
\end{artha}

\vishaya{mAdhayxMdina pATha `savA ESa Etasimxnf savxpAnxnetxV ratAvx' eMdu upakarxma......`parxtiyoVnAyxdarxvati budAdhxnAtxyeYva' eMdu pATha idaraMte yoVni shabAdxthaRvanunx heVLutAtxre-}

\begin{shl}
shorxVtArxdikaraNAnayxtarx mAtArxdAnasayx kAraNamf | \\
yoVnishabAdxBilapayxM sAyxjAjxgarxdedxVhasamAsharxyamf \hfill||  992 ||  
\end{shl}

\begin{artha}
shorxVtarx modalAda iMdirxyagaLu rUpAdi viSayagaLa vAsanA sivxVkArakekx kAraNavAdadudx yoVni shabadxdiMda heVLalapxDuvudu, jAgarxtitxna deVhakekx AsharxyavAgiruvudu.
\end{artha}

\footnotetext[2]{yoVni shabAdxthaRvanunx anusarisi `parxti' eMba padakU' Adarxvati' eMba padakUkx athaRvanunx heVLalu I vAtiRka baMdide jAgarAvasethxyalilx yAva Atamxnige rUpAdi viSaya BoVgada vAsanAsivxVkArakekx kAraNavAgi cakuSxrAdi iMdirxyagaLidadxvo adeV sAvxsharxyavAda kAraNavanunx avalaMbisi savxpanxdalilxdadxvo adeV sAvxsharxyavAda kAraNavanunx avalaMbisi savxpanxdalilxdadxvo. A puruSanu adara kamaRvu kaSxyisidoDane jAgarAvasethxge beVkAda kAraNavu udaBxvisidoDane jAgaravanunx punaH paDeyuvaneMdathaR.}
\begin{shl}
\footnotemark[2]mAtArxdAnasayx yA yasayx yoVnirAsiVtupxrA\s \s tamxnaH | \\
tAmeVvAyaM punayoVRniM savxpanx AtAmx parxpadayxteV \hfill||  993 ||  
\end{shl}

%%%%%\shloka
\begin{artha}
yAva Atamxnige rUpAdi BoVgavAsaneya sivxVkArakekx kAraNavAgi yAvudu hiMde ididxto, adeV yoVniyanunx (kAraNavanunx) savxpanxdalilx Atamxnu maraLi paDeyutAtxne.
\end{artha}

\vishaya{`parxtiyoVnAyxdarxvati savxpAnxyeYva' eMba kANavx pAThada athaR-}

\footnotetext[3]{hiMde jAgarAvasethxyalilx savxpanx saqSiTxyu parxsutxvAgidAdxga vAsanegaLanunx heVge tegedu koMDididxto hAgeye A vAsanegaLanunx sivxkarisi savxpanxdalilxdudx adara kamaRvu kaSxyisalu anaMtaraveV suSupitxyanunx hoMduvanu, savxpanx kamaRvu aBivayxkatxvAdalilx biVja rUpavAgidadx vAsaneyu aBivayxkatxvAguvudariMda adariMdada numiRsalapxTaTx deVhavanunx vAsanA rUpavAda iMdirxyagaLa maneyanUnx AMdare savxpanxkekx kAraNavAda (yoVni shabAdxthaR) vanunx paDeyuvaneMdathaR.}
\begin{shl}
\footnotemark[3]yathA\s \s dAnaM kaqtaM pUvaRM mAtArxNAM savxpanxsajaRneV | \\
savxpanx AdAya tA mAtArxH sAvxpinxVM yoVniM parxpadayxteV \hfill||  994 ||  
\end{shl}

%%%%%%\shloka 
\vishaya{kANavx matutx mAdhayxMdina pAThagaLeraDanunx anusarisi `\stext'eMbudara athaRvanunx heVLuvaru-}

\begin{shl}
parxtinAyxyagirA cAsayx yathAvatAmxRBidhiVyateV | \\
parxtiyoVniraveVNeYva yathAsAthxnamihoVcayxteV \hfill||  995 ||  
\end{shl}

\begin{artha}
parxtinAyxyaveMba padadiMda Itanige \footnotemark idadx mAgaRveV heVLalapxTiTxde, hAgU parxtiyoVni' eMba padadiMda \footnotemark hiMde idadx sAthxnaveV heVLalapxTiTxde.
\end{artha}

\vishaya{mAdhayxMdina PAThAnusAravAgi maMtarxda vAkAyxthaRvanunx upa saMharisuvudu-}

\footnotetext[1]{I vAtiRkadalilx baruva mAtArx padakekx vAsanegaLeMdu athaR, jAgarAvasethxyalilx cakuSxriMdirxya modalAda yAva yAva rUpAdi viSaya vAsanegaLu huTiTx koMDidadxvo AvAsanegaLanunx sivxVkarisi I Atamxnu savxpanxvanunx nimiRsikoMDidadxnoV AyAya ecacxrina deVhada iMdirxyagaLalilx A rUpAdi viSaya vAsanegaLa shakitxgaLanunx Atamx ceYtanayx BAsavenunxva biDibiDi jAcnxnagaLoMdige Itanu savxpanxsAthxnadiMda jAgarakekx hoVguvAga dharisuvaneMdathaR.}
\begin{shl}
yasAmxdayxsAmxdupAdAya \footnotemark[1]mAtArxH savxpanxM cakAra saH | \\
tatarx taterxYva tA dhatetxV mAtArxH sAvxtAmxMshuBiH saha \hfill||  996 ||  
\end{shl}

%%%%%%shloka footnote[1]
\begin{artha}
yAva yAvadariMda rUpAdi vAsanegaLanunx sivxVkarisi A Atamxnu savxpanxvanunx nimiRsidano AyAya sathxLadalelx idadx vAsanegaLanunx tananx Atamxna aMshagaLiMda (ceYtanAyxBAsagaLiMda) kUDi tegedu koLuLxvanu.
\end{artha}

\vishaya{kANavxpAThadaMte maMtarx vAkAthaRvanunx upasaMharisuvudu.}

\footnotetext[2]{yAva kAma-kamARdi mAgaRdiMda jAgaradiMda savxpanxkUkx savxpanxda kamaRvu kaSxyisidoDane alilxMda suSupitx sAthxnakUkx hoVgidadxno, adeV mAgaRdiMda savxpanx kamaRgaLu edudx koMDAga adariMda perxVreVpisalapxTaTxvanAgi kAraNadalilxruva vAsanegaLanunx sivxVkarisi suSupitxyiMda savxpanxkekx baruvanu, ililx kamaRvAyu samiVrita:'eMbuva visheVSaNadiMdale savxpanxkekx hoVguva mAgaR kamARdigaLeMdu sUcisiruvaru.}
\begin{shl}
\footnotemark[2]yeVneYva vatamxRnA yAtaH punasetxVneYva vatamxRnA | \\
yAti mAtArxH samAdAya kamaRvAyusamiVritaH \hfill||  997 ||  
\end{shl}

%%%%%shloka footnote[2]
\begin{artha}
yAva (kAmakamARdi) mAgaRdiMda (jAgaradiMda savxpanxkUkx alilxMda suSupitxgU) baMdiruvano, adeV mAgaRdiMda kamaRveMba vAyuviniMda perxVreVpisalapxTaTxvanAgi vAsanegaLanunx sivxVkarisi suSupitxyiMda savxpanxkekx hoVguvanu.
\end{artha}

\section*{ananAvxgata itAyxdi vAkayxkekx BASAyxnusAri vAyxKAyxna.}

\vishaya{`sayatatxtarxkiMcitf pashayxti' eMbudara athaR-}

\begin{shl}
sa AtAmx puNayxpApoVtathxM PalaM nAnAparxBeVdakamf | \\
tatarx savxpanxvidhw tiSaThxnapxshayxtiVha samiVkaSxteV \hfill||  998 || 
\end{shl}

\begin{artha}
A Atamxnu puNayxpApagaLiMda huTuTxva aneVka BeVdavuLaLx PalagaLanunx I savxpanx vidhiyalilx niMtu noVDuvanu parxtayxkaSxvAgi kANuvanu.
\end{artha}

 \vishaya{`ananAvxgaraH seVtxnaBavati' eMbudara athaR-}

\begin{shl}
neYnaM sAthxnAnatxraM pArxpatxM gArxmAdAgxrXmAnatxraM yathA | \\
shuBAshuBaM yadadArxkiSxVcatAsxkASxdanugacaCxti \hfill||  999 ||  
\end{shl}

\begin{artha}
oMdu gArxmadiMda matotxMdu gArxmakekx baMdavananunx (oMdu gArxmadalilx mADida shuBAshuBagaLu A kataqRvanunx heVge hiMbAlisuvavo) hAgeye savxpanxdalilx shuBAshuBa kamaRvanunx yAvudanunx noVDidadxno adu beVre sAthxnavanunx (jAgaravanunx) hoMdiruva Itananunx hiMbAlisuvudilalx.
\end{artha}

\vishaya{`pashayxti' eMdu heVLidadxriMda noVDuvavanige shuBAshuBagaLa saMpakaRviruvude? deVvatA dashaRnadiMda puNayx baruvaMte pApi dashaRnadiMda pApavu baruvaMte barabalalxde? eMdu keVLidare-}

\footnotetext[1]{shuBAshuBa kamaRgaLAgali adara PalavAgali savxpanxdalilx kaMDare Atamxnu adakekx sAkiSxye horatu avugaLu sAkiSxyalilx nijavAgi aMTuvudilalx, noVDalapxDuva daqshayx vasutxgaLa guNa doVSagaLu sAkiSx ceVtanadalilx seVruvudilalx idaralilx huTuTxvudU ilalx nijavAgi sAkiSx ceVtanada dashaRna savxBAvavanunx pashayxti noVDutAtxne eMdu heVLide dashaRna kirxyeyU vAsatxvavalalx niviRkAravasutxvige adu baruvudilalxveMdu tAtapxyaR-}
\begin{shl}
shuBAshuBakirxyeYveVha\footnotemark[1] pashayxtiVtayxBidhiVyateV | \\
na tu yadAvxsatxvaM vaqtatxM pashayxtiVtayxkirxyAtamxnaH \hfill||  1000 ||  
\end{shl}

%%%%%%shloka footnote[1]
\begin{artha}
A shuBAshuBa kamaRgaLeV ililx pashayxti eMba padadiMda heVLalapxTiTxve. Adare kirxyA shUnayxnAda Atamxnige pashayxti'eMbudu vAsatxvikavAda vAyxpAravalalx.
\end{artha}

\section*{ananAvxga itAyxdi vAkayxkekx BASAyxnusAri vAyxKAyxna}

\vishaya{`asaknogxV hayxyaM puruSaH' eMbudu EtakekxV eMdare}

\footnotetext[1]{`teVna ananAvxgataH' eMdare Atamxnu puNayxpApagaLiMda kUDiruvudilalx eMdathaR, ideV parxtijAcnxthaR, idakekx yukitx asaMga eMbudu yAvudariMda Itanu savxBAvavAgi savaRsaMbaMdhavilalxdavano adariMda pApa puNayxgaLiMdalU kUDiruvudilalx eMdu athaRvu sididhxsuvudu}
\begin{shl}
\footnotemark[1]teVnAnanAvxgata itiparxtijAcnxtAthaRsidadhxyeV | \\
asaknogxV hiVti heVtUkitxH parxtijAcnxthoVR\s thavA BaveVtf \hfill||  1001 || 
\end{shl}

%%%%%shloka footnote[1]
\begin{artha}
`ananAvxgata setxVna' (A kamaRdiMda aMTikoMDiruvudilalx)eMdu parxtijecnx mADida viSayavanunx sAdhisuvudakekx `asaknogxVhi'  eMbudu heVtu vAkayxvAgide athavA ide parxtijAcnxthaRvuLaLxdAdxguvudu.
\end{artha}

\vishaya{`asaknoVhi' eMbudu parxtijecnxyAdare `ananAvxgataH' eMbudu adakekx heVtu vAkayxvAguvudeMdu-tiLisutAtxre-}

\footnotetext[2]{asaMga vAkayxveV parxtijAcnx vAkayxvAguva pakaSxdalilx `Atamxnu savaR saMbaMdhashUnayxnu' heVge? eMdare athavA adakeVkxnu kAraNaveMdare kAraNavanunx boVdhisalu `ananAvxgata setxVna' eMbudu baMdide, yAvudariMda savxpanxda shuBAshuBagaLiMda jAgaradalilxruva Atamxnu kUDikoMDiruvudilalxvo adariMda Itanu asaMga'eMdu tiLiyuvudu.}
\begin{shl}
\footnotemark[2]yadeYvaM pUvaRmeVva sAyxtatxdA heVtuvacaH suPxTamf | \\
teVnAnanAvxgata iti pArxkalxqqteVneYva kamaRNA \hfill||  1002 ||  
\end{shl}

%%%%%shloka footnote[2]
\begin{artha}
yAvAga I riVti (asaMga vAkayxvu parxtijAcnx vAkayxvAguvudo) AvAga hiMdinadedx (ananAvxgataH eMbudeV) heVtu vAkayxvAguvudu, (jAgarakikxMta) hiMde savxpanxdalilx mADida kamaRdiMda I Atamxnu saMbaMdhisiruvudilalx eMbudu sapxSaTxvAdadudx.
\end{artha}

\vishaya{I pakaSxkekx gamakaveVnu? eMdare-}

\begin{shl}
sAsaknagxtavxM samAshaknakxyX parxvaqtetxYSA yataH shurxtiH | \\
ananAvxgata itayxsayx heVtutavxM higiroVcayxteV \hfill||  1003 ||  
\end{shl}

\begin{artha}
Atamxnalilx kAmAdi doVSa saMpakaRviruvude? eMdu shaMkisi (adanunx pariharisalu) I `sanA ESa Etasimxnf saMparxsAdaratAvx....'itAyxdi shurxtiyu parxvatiRside, `hi' eMba padadiMda `ananAvxgataH' eMbudakekx heVtatavxvu sapxSaTxpaTiTxde.
\end{artha}

\section*{ananAvxgata itAyxdi vAkayxkekx BASAyxnusAri vAyxKAyxna}

\vishaya{inunx muMde `ananAvxgataH' eMbudara vAkAyxthaRvanunx visAtxravAgi tiLisuvaru-}

\begin{shl}
na kamaR kuruteV savxpenxV kArakANAmasaMBavAtf | \\
PalamAtarxmAyaM tatarx BuknekxV \footnotemark[1]sAvxBAsavatamxRnA \hfill||  1004 ||  
\end{shl}
\footnotetext[1]{sAvxBAsavatamxnA eMdare' ceYtanAyx BAsada mUlaka eMdathaR. idu budidhxyalilx parxtibiMbisiruva ceYtanayx, aMdare aMtaHkaraNa vaqtitxjAcnxna. idara mUlakave BoVgavAguvudeMdu vAtiRka tAtapxyaR, yAvudoMdu sAdhanagaLilalxdavanAdarU savxpanx Barxmeyanunx kANuva I Atamxnige vAsaneye kirxyA kAraka rUpadalilx tALuvudu.}

%%%%%shloka footnote[1]
\begin{artha}
Atamxnu savxpanxdalilx kamaRvanunx mADuvudilalx, EkeMdare? kArakagaLeV saMBavisuvudeV ilalx, alilx ivanunx tananx ABAsa mAgaRdalilx Palavanunx anuBavisutAtxne.
\end{artha}

\vishaya{savxpanxdalilx kirxyeyilalxveMbudanunx vijAtiVya daqSATxMtadiMda tiLisuvudu-}

\begin{shl}
na hi niSapxdayxteV savxpenxV kirxyA jAgariteV yathA | \\
katArxRdikArakasAthxneV nAtoV\s nevxVtayxkirxyAtamxnaH \hfill||  1005 ||  
\end{shl}

\begin{artha}
kataqR modalAdakArakagaLige sAthxnavAda ecacxrinalilx kirxyeyu heVge naDeyuvudo hAge savxpanxdalilx, kirxyeyu naDeyuvudilalx, idariMda kirxyAshUnayxnAda Atamxnige kirxyeyu saMbaMdhisuvudilalx.
\end{artha}

\begin{shl}
yadi sAthxnAnatxraM pArxpatxmanivxyAtasxvXpanxjA kirxyA | \\
shAsArxramoBxV vaqtheYva sAyxtasxvXpenxV koV nAparAdhayxti \hfill||  1006 ||  
\end{shl}

\begin{artha}
savxpanxdalilx huTiTxda kirxyeyu (pApa puNAyxdikirxyegaLu) beVre sAthxnavanunx (jAgarasAthxnavanunx) hoMdida Atamxnalilx seVrikoLuLxvudAdare \footnote{yAva doVSavU ilalxdavaneV adhAyxtamx shAsatxrXpaThaNakekx adhikAriyAguvudu, savxpanxdalilx barxhamxhatAyx viVrahatAyxdi aneVka bageya pApa kaqtayxgaLanunx mADidaMte kANuvudariMda avugaLu ecacxrAda meVlU Itanige leVpisuvudAdare nidoRSiyAda adhikAriyeV dulaRBavAguvanu, I riVtiyAgi savxpanxvanunx kANada puruSaneV dulaRBa, hiVgiruvAga elalxralUlx savxpanx doVSagaLa saMGaTanege avakAshavAguvudariMda adhikAriyeV sikukxvudilalx, adariMda shAsAtxrXraMBaveV vayxthaRvAguvudeMdu tAtapxyaR, adariMda savxpanx kirxyeyU jAgarAvasethxge baMda Atamxnalilx saMpakiRsuvudilalxveMdu tAtapxyaR.} shAsAtxrXraMBaveV vayxthaRvAguvudu, EkeMdare! savxpanxdalilx yAru tAne tapupx mADuvudilalx.
\end{artha}

\vishaya{savxpanxdalilx Atamxnu EnU mADuvudilalxvAdare tAnu mADideneMdu kataqRtavx budidhxyu heVge? baruvudu:-}

\begin{shl}
apeVtasAdhanasAyxpi pashayxtaH savxpanxviBarxmamf | \\
biBatiR vAsaneYvAsayx kirxyAkArakarUpatAmf \hfill||  1007 ||  
\end{shl}

\begin{artha}
yAvudoMdu sAdhanavilalxdavanAdarU savxpanxBarxmeyanunx kANuva I Atamxnige vAsaneye kirxyAkAraka rUpavanunx tALuvudu.
\end{artha}

\section*{ananAvxgata itAyxdi vAkayxkekx BASAyxnusAri vAyxKAyxna}

\vishaya{kataqRitAyxdi budidhxyu savxpanxdalilx vAsanA mAtarxvAdalilx PalitAMshavidu}

\begin{shl}
PalasaMBoVgamAtarxM ca yasAmxtasxvXpenxV samiVkaSxyXteV | \\
PalAnatxrAramaBxvidhinARtaH sAyxdakirxyAtamxnaH \hfill||  1008 ||  
\end{shl}

\begin{artha}
PalAnuBava mAtarxveV savxpanxdalilx yAva kAraNadiMda noVDalapxTiTxdeyo, A kAraNadiMda kirxyA shUnayxnAda Atamxnige toVrikeyanunx biTuTx beVre Palavu huTuTxvudeMba kirxyeyu iruvudilalx.
\end{artha}

\vishaya{kirxyAkAraka jAcnxnavAguvaMte PalajAcnxnavU ati sapxSaTxvAgilalxve? eMdare utatxra-}

\begin{shl}
na karoVti yataH savxpenxV kirxyAmiva samiVkaSxteV | \\
parxkirxyAPalasaMyoVgoV nAtoV boVdheV\s sayx viVkaSxyXteV \hfill||  1009 ||  
\end{shl}

\begin{artha}
yAva kAraNadiMda savxpanxdalilx kamaRvanunx Itanu nijavAgi mADuvudilalxvo keVvala kirxyeyaMte noVDutAtxneyo adariMda kamaRPala saMbaMdhavU Itanige ecacxrinalilx kANuvudilalx.\footnote{savxpanxdalilx nijavAgi kirxye ididxdadxre jAgaradalUlx adara Palavu ivanalilx kANabeVkAgitutx, adu kANuvudilalx, udA:-savxpanxdalilx BoVjana, sAnxna modalAda kirxyeyanunx mADidaMte kANuvevu, Adare edadx meVle namamxlilx sAnxna mADida Pala nimaRlate, meY odedxyAdadudx enU kANuvudilalx hAgU BoVjanada Pala hasivu aDagi taqpitx yAguvudu idU ecacxrAda meVle namage kANuvudilalx, adariMda kirxye nijavAgilalx I riVtiyAgi ecacxrinalilx Palavu kANadadxriMda PalABAvadiMda kirxyeyilalxveMba kirxyABAvavanunx savxpanxdalilx takiRsa beVkeMdu yukitxyanunx toVrisidaMte Ayitu.}
\end{artha}

\vishaya{savxpanxdalilx kANuva kirxyAdigaLu asatayx eMbudakekx beVreyukitxyide-}

\begin{shl}
taqpatxH savxpenxV\s tha saMbudadhxH kuSxtapxriVtaH parxbudhayxteV \hfill||  1010 || \\
yasAmxdanaqtameVveVdaM yatikxMcidiha viVkaSxyXteV | \\
parxtayxknaknxkArakasatxsAmxlilxpayxteV na kirxyAPaleYH \hfill||  1011 ||
\end{shl}

\begin{artha}
savxpanxdalilx taqpatxnAgidadxvanu anaMtara edadxvanu hasiviniMda kUDiye toVri baruvanu, adariMda I savxpanxdalilx yAvudoMdu kaMDidadxrU idu asatayxve adariMda parxtayxgAtamxnu kataRnalalx kamaR PalagaLiMda nijavAgi kUDiyU ilalx.
\end{artha}

\section*{ananAvxgata itAyxdi vAkayxkekx BASayxnusAri vAyxKAyxna}

\vishaya{patayxgAtamxnalilx savxpanxda kirxye modalAdavu nijavAgilalxvAdare savxpanxdalilx Aguva reVta saKxlaneya nimitatxvAgi pArxyashicxtatxvanunx Eke vidhisiruvudu? eMdare- utatxra-}

\footnotetext[1]{`savxpenxV ceVMdirxya dwbaRlAyxtf sitxrXyaM daqSATxvXkaSxreVdayxvi |' `pArxyashicxtarxM tu tasoyxVkatxM pArxNAyAmAsutx SoVDasha ||' ililx savxpanx saKxlana nimitatxvAgi hadinAru pArxNayAma mADa beVkeMdu ukatxvAgide}
\begin{shl}
savxpanxsakxnanxnimitatxM tu \footnotemark[1]pArxyashicxtatxM yaducayxteV | \\
sateyxVnidxrXyavikAratAvxtatxcAcxpi na nirAsharxyamf \hfill||  1012 ||  
\end{shl}

%%%%%shloka footnote[1]
\begin{artha}
savxpanx saKxlana nimitatxvAgi pArxyashicxtatxvu yAvudu heVLalapxTiTxdeyo adu satayxvAda iMdirxya vikAravAdadxriMda heVLalapxTiTxde, adeVnU nirAdhAravAdadadxlalx.
\end{artha}

\vishaya{pArxyashicxtatx vidhige idU oMdu kAraNavide-}

\begin{shl}
savxpanxsakxnanxM yathA savxpenxV yathA boVdheV\s pi viVkaSxyXteV | \\
AtamxceYtanayxvatatxsAmxtApxrXyashicxtatxM tadudaBxvamf \hfill||  1013 || 
\end{shl}

\begin{artha}
savxpanxdalilx heVge savxpanxsaKxlaneyAdadudx kANuvudo hAgeye ecacxrAda meVlU AtamxceYtanayxvu kANuvaMte kANuvudu adariMda pArxyashicxtatxvu A nimitatxvAgi huTiTxkoMDitu.
\end{artha}

\vishaya{BASayxkArara vAyxKAyxnada upasaMhAra-}

\begin{shl}
EkA tAvadiyaM vAyxKAyx yathoVketxYSoVpavaNiRtA || 
\end{shl}

\begin{artha}
idoMdu vAyxKAyxnavu hiMde heVLidaMte visatxrisapxTiTxruvudu.
\end{artha}

\vishaya{BataqR parxpaMca vAyxKAyxnavu I muMde baMdide}

\begin{shl}
vAyxcakaSxteV\s nayxthA ceVdaM vAkeyxVmatadayxthoVditamf ||  1014 || \\
\end{shl}

\begin{artha}
I vAkayxvanunx hiMde heVLidaMte beVre riVtiyAgi (BataqRparxpaMcaru) vAyxKAyxnisuvaru-
\end{artha}

\vishaya{BataqRparxMcara ananAvxgata maMtarx vAyxKAyxna}

\vishaya{ivara matadalilx ananAvxgata vAkayxda tAtapxyaR-}

\begin{shl}
ananAvxgatavAkeyxVna viveVkaH kamaRNaH kaqtaH | \\
asaknagxvacasA tavxsayx kAmanimoVRka ucayxteV \hfill ||  1015 ||
\end{shl}

\begin{artha}
ananAvxgata vAkayxdiMda kamaR viveVcaneyu mADalapxTiTxde, asaMgavAkayxdiMda Itanige kAma viveVcaneyu heVLalapxDuvudu.
\end{artha}

\vishaya{asaMga vAkayxdiMda pariharisuva shaMkeyidu-}

\footnotetext[1]{`yathAhuH jAgarita deVsheV\s puyxpacayApacayeV na kirxyeYva heVtuH kiMtahiRkAmoVheVtuH' eMdu BataqRparxpaMcara BASayx vacanavanunx AnaMda giriyalilx koTiTxdAdxre. athaR jAgara sathxLadalUlx kamaRgaLa vaqdidhx hArxsagaLige kirxyeyeV (parxvaqtitxyeV) kAraNavalalx. matetxVneMdare kAmaveV kAraNa eMdu. I athaRdalilx avara vacanadaMte `nanu itAyxdi vAtiRkadalilx anuvAda mADideyeMdu tiLiyabeVku.}
\begin{shl}
nanu \footnotemark[1]neYva kirxyA sAkASxdAdhxnavaqdidhxV parxtiVSayxteV ||\\  
api jAgaritasAthxneV kAmoV heVtuyaRtasatxyoVH \hfill||  1016 ||
\end{shl}

%%%%%%shloka footnote[1]
\begin{artha}
kamaRda hArxsa vaqdidhxgaLige neVra kirxyeya parxvaqtitxyu kAraNaveMbudu samamxtavalalx, EkeMdare? jAgara sathxLadalUlx avugaLige kAraNa kAmaveMbudu ide alalxve?
\end{artha}

\vishaya{kAmavilalxdavanigU parxvaqtitxyu kaMDide yAdadxriMda kAmavu parxvaqtitxge heVge kAraNa? veMdare-}

\begin{shl}
akAmasayx kirxyA yasAmxninxSaPxleYva samiVkaSxyXteV | \\
sa yathAkAma itAyxdi tathA coVdhavxRM parxvakaSxyXteV | \\
itayxsayx parihArAthaRmasaknogxV hiVti BaNayxteV \hfill||  1017 ||  
\end{shl}

\begin{artha}
kAmavilalxdavanige kirxyeyu parxvaqtitxyu niSaPxla veMdu yAva kAraNadiMda kaMDideyo, (adariMda kirxyeyu PalavatAtxgiruvalilx kAmadiMdaleV Agiruvudu) idakekx parxmANavAgi `sa yathA kAmaH' itAyxdi shurxtiyU meVle udAharisalapxDuvudu- hiVgeMba shaMkeyanunx pariharisuvudakAkxgi ``asaknogxVhayxyaM puruSaH'' eMbudAgi heVLalapxTiTxde.
\end{artha}

\vishaya{Atamxnu asaMganeMbudu heVge?eMdare AvAkayxda tAtapxyaRvanunx heVLutAtxre-}

\begin{shl}
yamAsaknagxmiha savxpenxV kAmaM tavxmanupashayxsi | \\
savxpatoV na BaveVtAkxma iteyxVtadaBidhiVyateV \hfill||  1018 ||  
\end{shl}
				
\begin{shl}
asaknogxV hiVti vAkeyxVna tadavxtoV\s saMBavAdiha | \\
garxsatxmatarx manaH kAmi vAsanAmAtarxsheVSataH \hfill||  1019 ||  
\end{shl}

\begin{artha}
yAva kAmavanunx asaMga eMbuvadanunx I savxpanxdalilx niVnu kANuveyo, AkAmavu malagidavanige iruvudeV ilalxveMbudAgi `asaknogxVhi eMba vAkayxdiMda' heVLalapxDuvudu, Eke ilalx? eMdare kAmavuLaLx (manasusx) I savxpanxdalilx ilalxvAdadxriMda, manasusx iruvAga Ekilalx? eMdare vAsanA mAtarxvAgi uLisiruvudariMda kAmavuLaLx manasusx (kAraNa vasutxviniMda) nuMgalapxTiTxde.
\end{artha}

\begin{shl}
yatoV\s toV\s saknagx EvAyaM puruSaH savxpanxBUmigaH | \\
bahiH kulAyAdituyxkatxM pApamxnoV vijahAti ca \hfill||  1020 ||  
\end{shl}

\begin{artha}
yAvudariMda alilx manasusx ilalxvo, adariMda I puruSanu savxpanx sathxLadalilxdadxvanu asaMganeV (savaRsaMga shUnayxneV) ``ba hiH kulAyAdamaqtashacxritAvx''eMdU (4-3-11) pApamxno vija{hA}ti' eMdU (4-3-8) I viSayavanunx udAhariside.
\end{artha}

\vishaya{savxpAnxvasethxyalilxruva Atamxnige shariVra kamaR ivugaLa saMbaMdhavilalxdidadxrU kAmaveVke ilalx? -aMdare-}

\begin{shl}
deVhAsharxyAdaqteV kAmAH sanitx nAnayxtarx kutarxcitf | \\
savxpanxparxpacnacxH savoVR\s pi vAsanAmAtarxmeVva tu \hfill||  1021 ||  
\end{shl}

\begin{artha}
deVhada Asharxyavilalxde kAmaRgaLu beVre elilxyU iruvudilalx, matutx elAlx savxpanx parxpaMcavU kUDa \footnote[1]{aMdare elalxvU mitheyxye asatayxveV, eMdu BAva. idu parxtijAcnxvAkayx. muMdina vAtiRkadalilx heVtuvanunx heVLiruvudu, ililx kANuva gajaturagAdi parxpaMcavu asatayxve ekeMdare, avelalx jAcnxna mAtarx aMdare tiLiyuvudaSeTx viSayaveVnU nijavAgilalx.}vAsanA rUpaveV.
\end{artha}

\vishaya{elalxvU vAsanA mAtarxve eMbudakekx kAraNa-}

\begin{shl}
vAyxcakASxNoV yataH savxpanxM jAcnxnamAtarxM parxBASateV \hfill||  1022 | \\
jAneV putArxyutaM jAtaM jAneV pitaramAgatamf | \\
tathAceVvagirA yukAtxnasxvXpAnxnAvxyXcakaSxteV janAH \hfill||  1023 ||  
\end{shl}

\begin{artha}
yAva kAraNadiMda savxpanxvanunx heVLuvavanu jAcnxnavanunx mAtarx heVLuvano heVgeMdare? hatutx sAvira makakxLu Adadadxnunx tiLididedxVne matutx taMdeyu baMdadadxnunx tiLididedxVne'eMdu heVLuvanu. alalxde janaru iva shabadxdiMda [aMte] kUDisi savxpanx vasutxgaLanunx\footnote[2]{idakekx loVka parxsidadhxvAda anuBavavanunx udAharisiruvaru janaru savxpanxvanunx kaMDu heVLuvAga hiVge heVLuvaru ``asitxnoV GaTikaqtA dhAvanitxVva daqSATx mayA ashAvx api dhAvanatx iva daqSATxH putarxshataM jAtamiva daqSaTxmf''eMdu iva shabadxdoMdige heVLuvaru. athaR AnegaLu oTATxgi ODutitxruvaMte nanage kaMDavu hAgeye kuduregaLu ODutitxruvaMte nanage kaMDavu eMdu, idariMda ODutitxruvaMte kaMDavu nijavAgi ODuvudalalxveMdu asatayxveMdu tiLidubarutatxde. I parxsididhxyiMdalU elalxvU vAsanA mAtarx, satayxvAgilalx.} heVLuvaru.
\end{artha}

\footnotetext[3]{``yatheYva rathAdiVnA masatAM nimARNaM bAhAyxnAM tathA shariVreVMdirxyANAM cAnatxrANAmiti'' eMdu BatayxR parxpaMcara BASayxvAkayxdaMte ililx anuvAda mADide.}
\begin{shl}
\footnotemark[3]rathAdeVrasatoV yadavxsatxvXpenxV nimARNamiVkaSxyXteV | \\
shariVreVnidxrXyakAmAdeVrasatasatxdavxdeVva tu \hfill||  1024 ||  
\end{shl}

%%%%%shloka footnote[3]
\begin{artha}
ilalxdiruva rathAdigaLa nimARNavu savxpanxdalilx heVge kANuvudo hAgeye ilalxdiruva shariVra iMdirxya kAma modalAdavugaLa nimARNavu kANuvudu.
\end{artha}

\begin{shl}
ratheVnidxrXyAdivatatxsAmxtAkxmoV\s payxtarx samiVkaSxyXtAmf | \\
savxpenxV\s toV\s yaM nirAsaknagxH pumAnaBuyxpagamayxtAmf \hfill||  1025 ||  
\end{shl}

\begin{artha}
AdudariMda ratha iMdirxya modalAdavugaLaMte ililx kAmavU savxpanx nimARNavAdadxriMda (asatayxveMdu) tiLiyabeVku adariMda savxpanxdalilx puruSanu AsaMga shUnayx kAmavilalxdavaneMdu opapxbeVku.
\end{artha}

\vishaya{asaMga padakekx beVre athavU ide.}

\begin{shl}
kwTasathxyXM yadi vA puMsoV\s saknagxshabedxVna BaNayxteV | \\
kirxyA\footnotemark[1]shelxVSamaqteV yasAmxnAnxdhayxkaSxPalasaMgatiH \hfill||  1026 ||  
\end{shl}
\footnotetext[1]{kirxyA sheVSamf eMdu pAThAnatxra}

%%%%%%\shloka footnote[1]
\begin{artha}
athavA puruSana niviRkArateye asaMga shabadxdiMda heVLalapxTiTxde kirxyA saMdhavilalxde adara sAkiSxge PalasaMBaMdhavu baruvudilalx.
\end{artha}

\vishaya{AkeSxVpa adu heVge niviRkAra?}

\begin{shl}
nanu katAR pumAneVSa sa hi kateVRtivAkayxtaH | \\
neYvaM savxpenxV yataH puMsaH kataqRtavxM BAvanAkaqtamf \hfill||  1027 ||  
\end{shl}

\begin{artha}
I puruSanu kataRnalalxve? `sahikatAR eMdu vAkayxdiMda tiLidu barutatxde eMdu akeVSxpavu baMdare samAdhAna- hiVgalalx kAraNaveVneMdare? puruSanige kataqRtavxvu vAsaneyiMda kalipxtavAdadudx. matutx-
\end{artha}

\begin{shl}
kataqRkAyARvaBAsitAvxjAjxgarxtAkxleV\s pi cA\s \s tamxnaH | \\
kataqRtavxM na savxtaH savxpenxV kimu vakatxvayxmiSayxteV \hfill||  1028 ||  
\end{shl}

\begin{artha}
ecacxra kAladalUlx Atamxnu kataqR matutx kAyaRgaLige sAkiSxyAgiruvudariMda Itanige kataqRtavxvu sAvxBAvikavalalx aMda meVle savxpanxdalilx heVLabeVkAdadudx Enide?
\end{artha}

\begin{shl}
avidAyxsorxVtaseYvAsayx kirxyAkArakatA\s \s tamxnaH | \\
tasathxceYtanayxbimebxVna BuknekxV\s sw kamaRNaH Palamf \hfill||  1029 ||  
\end{shl}

\begin{artha}
avideyxyaparxvAhadiMdale I Atamxnige kirxyA kAraka rUpagaLu, A avideyxyalilxruva ceYtanayx parxtibiMbadiMda kAraka rUpagaLu, A avideyxyalilxruva ceYtanayx parxbiMbadiMda Itanu kamaRda Palavanunx anuBavisuvanu.
\end{artha}

\vishaya{hAgAdare `sahikatAR' eMdu heVLidudx heVge? aMdare-}

\begin{shl}
katArxRdisAkiSxNoV\s sAyxBUtakxtaqRtavxM yatupxrA\s \s tamxnaH | \\
sa hi kateVRti taseyxYva vacanaM sAyxtupxnaHshurxtiH \hfill||  1030 ||  
\end{shl}

\begin{artha}
hiMde ecacxrinalilx kataqR modalAdavugaLa sAkiSxyAda Atamxnige kataqRtavxvu Enidadxto adanenx `sahikatAR'eMdu shurxtiyu punaH ucacxriside (anuvadiside).
\end{artha}

\vishaya{athARpatitx parxmANadiMda vAsatxva kataqRtavxveMdu shakisuvudu}

\footnotetext[1]{`BAvanA'eMdare I saMdaBaRdalilx vAsanA rUpavAda kAmaveMdu BAvanA rUpakAmasayx eMdu 1039neV vAtiRkadalilxde}
\begin{shl}
nanavxseyxYva tu tatakxmaR \footnotemark[1]BAvanApeVkaSxyA\s Kilamf | \\
parxvataRteV kathaM tasayx kataqRtavxM vinivAyaRteV\hfill ||  1031 ||
\end{shl}

%%%%%shloka footnote[1]
\begin{artha}
A elAlx kamaRvU ivanige BAvanApeVkeSxyiMda (kAmada apeVkeSxyiMda) Palavanunx koDalu parxvatiRsuvudu, hiVgiruvAga avanige kataqRtavxvanunx heVge? nirAkarisuvudu?
\end{artha}

\vishaya{samAdhAna-}

\footnotetext[2]{``BAvanA hi kamaRguNa Eva satiV puruSaM BAvayati'' eMdu BataqRparxpaMca BASayx vacana (AnaM-TiVkA)}
\begin{shl}
kAmaM parxvataRtAM kamaR na ca doVSoV\s tarx kashacxna | \\
\footnotemark[2]BAvaneVyaM yataH kamaRguNa Eva satiV sadA | \\
savxsAkiSxNaM BAvayanitxV na tavxsAvAtamxnoV guNaH \hfill||  1032 ||  
\end{shl}

%%%%%%shloka footnote[2]
\begin{artha}
kamaRvu Palavanunx koDalu yatheVSaTxvAgi parxvatiRsali, idaralilx oMdU doVSavilalx, I BAvaneyu vAsanArUpavAdakAmavU kamaRkekx anukUlavAgiye irutAtx yAvAgalU tananx sAkiSxyanunx saMsakxrisutAtx iruvudu. Adare Atamxna guNavalalx.
\end{artha}

\begin{shl}
puSaNxganadhxH puTasothxV\s pi puSapxseyxYva yathA tathA | \\
budAdhxyXdikArakasathxsayx BAvanA kamaRNoV guNaH | \\
\footnotemark[1]AtamxnasUtxcArAtAsxyXnanx tu sAkASxdugxNoV BaveVtf \hfill||  1033 || 
\end{shl}
\footnotetext[1]{``BAvanAmaracnajxteV vijAcnxnAtamx vijAcnxneV kataqRtovxVpacAraH'' eMdu BataqRparxpaMcara ukitxyide.}

%%%%%%\shloka footnote[1]
\begin{artha}
hUvina vAsaneyu saMpuTadalilxdadxrU adu hUvinadedxV eMbudu heVgoV hAgeye budidhx muMtAda kArakagaLalilxruva kamaRda guNave BAvaneyeMbudu, adu Atamxnige aupacArikavAgi guNaveMdAguvudu Adare Atamxnalilxruva neVra iruva guNavAgalAradu.
\end{artha}

\vishaya{BAvaneyu kamaRda guNavAguvudAdarU heVge? kamaRvu niguRNavalalxve? eMdare-adu keVvala upakaraNa mAtarxvenunxtAtxre-}

\footnotetext[2]{`tAmeVvAmaracnajxsAM kataqRtevxVnAknegxV kuruteV kamoVRtapxtitx BoVgakAlayoVH''}
\begin{shl}
\footnotemark[2]utapxtwtx BoVgakAleV ca kamAR\s \s dAyeYva BAvanAmf | \\
sAvxtAmxnaM laBateV yasAmxtAtxM vinA tadanathaRkamf \hfill||  1034 ||  
\end{shl}

%%%%%%sholoka footnote[2]
\begin{artha}
yAvudariMda kamaRvu tAnu huTuTxvAgalU Pala BoVgavanunx koDuva kAladalUlx BAvaneyanunx upakaraNavAgi sivxVkarisiye tananx savxrUpavanunx hoMduvudo, adariMda A BAvaneyu ilalxde hoVdare AkamaRvu niSaPxlavAguvudu.
\end{artha}

\vishaya{kamaR Atamxna guNavAgali eMdare-}

\footnotetext[3]{``kamaRNoVpayxtayxnatxM parxyoVgA shirxtatAvxtf budAdhxyXdi saMnikaSaRpakaSxteYva, na kadAcanA\s tamxguNAtA |"eMdu BataqRparxpaMcara ukitxyaMte vAtiRkadalilxde}
\footnotetext[4]{parxyoVga eMdare atisUkaSxmXvAda budidhx muMtAda kAraka samudAyavu, ``budAdhxyXdikArakajAta matisUkaSxmX parxyoVga shabAdxthaRH'' eMdu (AnaM- 1037 vAtiRkada vAyxKAyxnadalilx hiVge athaR mADide, budidhx modalAdavugaLa sUthxla rUpagaLu kaSxyisidarU sUkaSxmX rUpavAda budAdhxyXdigaLanenxV maraNa avasethxgaLalilx kamaRvu asharxyisikoMDiruvudu, Atamxnanunx Asharxyisiruvudilalx eMdu aBipArxyavu)}
\begin{shl}
\footnotemark[3]kamaRNoV\s pi tathA\s tayxnatxM \footnotemark[4]parxyoVgeYkasamAsharxyAtf | \\
budAdhxyXdayxthARBisaMbanadhxpakaSxteYvAvasiVyateV | \\
na tAvxtamxguNatA tasayx kamaRNaH sAyxtakxdAcana \hfill||  1036 ||  
\end{shl}

%%%%%%sholoka footnote[3, 4]
\begin{artha}
kamaRvU saha bahaLavAgi parxyoVgavoMdanenx avalaMbisuvudariMda budidhx muMtAda vasutxgaLa saMbaMdhavuLaLxdedxMba pakaSxkekx seVruvudeMdu niNaRyisalapxTiTxde. Adare A kamaRvu Atamx guNaveMdu eMdigU AgalAradu.
\end{artha}

\begin{shl}
parxyoVgApeVkaSxNeVvA\s \s setxV kamaR budAdhxyXdisaMkaSxyeV\hfill ||  1037 | \\
sAvxtamxtaH parxtilABAya na tAvxtAmxnamapeVkaSxteV | \\
saBAvanamataH kamaR parxtiVcaH parxvivicayxteV \hfill||  1038 ||  
\end{shl}

\begin{artha}
kamaRvu parxyoVgavanenxV apeVkiSxsiruvudu, budidhx muMtAdavugaLu kaSxyisidalilx tananx savxrUpa lABakekx aMdare tananx huTiTxgAgi Atamxnanunx beVDuvudilalx. adariMda BAvanA sahitavAda kamaRvu parxtayxgAtamxniMda beVpaRTiTxde.
\end{artha}

\footnotetext[2]{kamaRguNasayx eMbudanenx kamaRkekx upakaraNavAdadedxMdathaR.}
\footnotetext[3]{ililx kamaR eMdare aMtaHkaraNa.}
\begin{shl}
tasAmxtf \footnotemark[2]kamaRguNaseyxYva \footnotemark[3]kamaRsathxseyxYva savaRdA | \\
sAkASxyXtAmx BAvanArUpakAmasayx savxpanxsadamxni \hfill||  1039 ||  
\end{shl}

%%%%%%shloka footnote[2, 3]
\begin{artha}
AdadxriMda kamaRkekx upakaraNavAdadxgiye iruva matutx yAvAgalU aMtaHkaraNadalelx iruva BAvanA rUpavAda kAmakekx savxpanxsAthxnadalilx Atamxnu sAkiSxyAgiruvanu.
\end{artha}

\begin{shl}
Asaknagx AtamxnashecxVtAsxyXtAsxkASxcecxYtanayxvadugxNaH | \\
na pasheyxVdAtamxvatAkxmaM kamaRsathxM savxparxboVdhayoVH \hfill||  1040 ||  
\end{shl}

\begin{artha}
Atamxnige kAmaveMbudu ceYtanayxdaMte oMdu guNavAgi neVra ididxdedxV Adare aMtaHkaraNadalilxruva kAmavanunx AtamxnaMte savxpanx jAgaragaLalilx noVDutitxralilalx.
\end{artha}

\begin{shl}
EvaM cAvasithxteV pakeSxV shurxtAyx\s kAri samacnajxsamf | \\
parxsidadhxvadupAdAnamasaknogxV hiVti sAdaraNf \hfill||  1041 ||  
\end{shl}

\begin{artha}
hiVge pakaSxvu irutitxralu (Atamxnu samasatx kAmagaLiMda nimuRkatxneMbudakekx beVkAda) sidadhxvAdaMte iruva heVtuvanunx `asaknoVgxhi' eMdu AdaradiMda kUDi shurxtiyu sivxVkarisiruvudu samaMjasavAguvudu.
\end{artha}

\begin{shl}
na kamoVRpacayoV\s toV\s tarx savxpenxV jAgarxdavxdiSayxteV | \\
savARsaknagxvinimoVRkAdAvxsanAmAtarxsheVSataH \hfill||  1042 ||  
\end{shl}

\begin{artha}
idariMda jAgarAvasethxyalilxruvaMte savxpanxdalilx kamaRvaqdidhxyu iruvudilalx EkeMdare? samasatx kAmavU biTuTx hoVgiruvadariMdalU vAsanA mAtarxveV uLidirudariMdalU kamaRvaqdidhxyu iruvudilalx.\footnote[1]{``yasAmxcAcxsaknogxV\s toV na kamoVRpacayaH savxpenx'' eMdu BataqRparxpaMcara ukitx (AnaM-TiVkA)}
\end{artha}

\vishaya{savxpanxdalilx kAma kamaRgaLu ilalxvAdare avugaLanunx noVDuvudu heVge? aMdare-}

\begin{shl}
nAyaM kamaRmayoV yasAmxnanx ca kAmamayaH savxtaH | \\
parxtayxgajAcnxnatasatxsAmxtatxnamxyatavxmiveVkaSxyXteV \hfill||  1043 ||  
\end{shl}

\begin{artha}
yAvudariMda Itanu kamaRmayanU alalx matutx savxtaH kAmamayanU alalxvoV, adariMda parxtayxgAtamxna ajAcnxnadiMda Atamxnalilx kamaRmayatavx matutx kAmamayatavxvU iruvaMte kANuvudu.
\end{artha}

\begin{shl}
nAnAsAdhanasaMbanAdhxtakxtaqRtavxM jagatiVkaSxyXteV | \\
\footnotemark[2]ayaM tavxsaknogxV yeVnAtoV na karoVti na lipayxteV \hfill||  1044 ||   
\end{shl}
\footnotetext[2]{ililx asaMgatavx eMba heVtuviniMda kataqRtavx BoVkatxqqtavxgaLanunx shudAdhxtamxnalilx sAdhiside. AtAmx kataqRtAvxdi rahitaH, asaMgatAvxtf' eMba takaRvu ililx gArxhayx asaMganAdadxriMda kataqRtavx BoVkatxqqtavxgaLeraDU ilalxveMdathaR.}

%%%%%shloka footnote[2]
\begin{artha}
aneVka sAdhanagaLa saMbaMdhadiMda jagatitxnalilx kataqRtavx kANuvudu. ItanAdaroV asaMgane, yAva asaMgatavx heVtuviniMdaleV Itanu EnoMdu mADuvudilalxveMtalU yAvudariMdalU leVpisilapxDuvudilalxveMtalU sididhxsuvudu.
\end{artha}

\vishaya{asaMga eMbudakekx matotxMdu heVtuvanunx heVLavaru-}

\footnotetext[1]{`neVti neVti' eMba shurxtiyU `asUthxla manaNu'....itAyxdi shurxtiyU saha heVLuvudeMdu athaR. kAyaRkaraNaveMba mUtaR vasutxgaLoDane matotxMdu mUtaRvasutxvige saMbaMdhavAguvudu. I saMbaMdhavu kirxyege kAraNaveMbudu kaMDide. shariVra iMdirxyagaLiruva Atamxnu ceYtarx muMtAda jiVvanu GaTAdi mUtaRvasutxvanunx keYyayxlilx hiDidu jala muMtAda vasutxgaLanunx taruvanu I jalAnayanAdi kirxyege ceYtarx deVvadatatx itAyxdi mUtARtamxnigU mUtaR GaTAdi vasutxgaLigU saMbaMdhaveV kAraNa I mUtaR vasutx athavA AkAravilalxda shudadhx parxtayxgAtamxnu kamaRvanunx mADuvudeMbudu elilxyU kaMDilalx, EkeMdare `asaMga puruSa, Itanige savxBAvavAgiye elilxyU saMbaMdhavilalx, adariMda kirxyA kataqRtavxveMbudu heVgU hoMduvudilalx. kAyaRkaraNa saMGAta saMbaMdhavu vAsatxvikavAgi I Atamxnige ilalx. adariMda asaMga. saMbaMdha shUnayx. Ada kAraNa kataqRtavxvU avanigilalxveMdu aBipArxya.}
\begin{shl}
\footnotemark[1]mUtARmUtARdayxpahunxtAyx hAyxtamxnasatxtatxvXmabarxviVtf | \\
nAtoV\s nAtAmxBisaMbanadhxH kUTasethxYkAtamxyXheVtutaH \hfill||  1045 ||  
\end{shl}

%%%%%%shloka footnote[1]
\begin{artha}
mUtaR matutx amUtaRvasutxgaLa savxrUpavalalxveMdu niSeVdhisidadxriMdalU Atamxna tatavxvanunx (shurxtiyu) heVLiruvudu, adariMda Atamxnige anAtamx vasutxvina saMbaMdha viruvudilalx EkeMdare! niviRkAravAgiyU EkAtamx savxrUpavAgiyU iruvudariMda (eMdu tiLiyabeVku)
\end{artha}

\vishaya{`ratAvx caritAvx itAyxdi shurxtiya meVle yathA shurxtavAda shaMke}

\begin{shl}
suSupetxV nanavxsaMboVdhAdarxtAvxdi kathamucayxteV | \\
na tivxtayxpi tathA vAkayxM na bAhayxM ceVti vakaSxyXti \hfill||  1046 ||  
\end{shl}

\begin{artha}
suSupitxyalilx divxtiVya vasutxvina jAcnxnaveV ilalxdadxriMda `ratAvx itAyxdi mAtanunx heVge? heVLidudx. `natu davxtiVya masitx' `na bAhayxM kiMcana veVda' eMba shurxti vAkayxgaLU kUDi (eraDane vasutxveV ilalxveMdu muMde heVLuvavu).
\end{artha}

\vishaya{naDuve I anupapatitxyU ilalxveMdu shaMkisuvudu-}

\begin{shl}
savxpenxV ratAvxdayxtha mataM neYvamapuyxpapadayxteV | \\
ananAvxgatagiVH kupeyxVtasxvXpenxV\s nevxVteyxVva tatakxqqtamf \hfill||  1047 ||  
\end{shl}

\begin{artha}
savxpanxdalelxV ramaNa muMtAdavugaLu heVLalapxTiTxveyeMdu shurxtiya aBipArxyavAgiruvudaSeTx, eMdu keVLidare pUvaRvAdiya utatxra-neYvamf eMdu, hiVgU saha hoMdibaruvudilalx, hAgidadxre sa yatatxtarx itAyxdi ananAvxgataH eMba vAkayxvu virudadhxvAguvudu, avanu mADidudx (savxpanxdalilx) noVDuvavanige baMdu seVriyeV seVruvudu.                         
\end{artha}

\begin{shl}
yathA jAgarxtakxqqtaM kamaR katARramanugacaCxti | \\
jAgarxtasxthXM savxpanxkameYRvaM savxpenxV samadhigacaCxti \hfill||  1048 ||  
\end{shl}

\vishaya{heVge savxpanxdalilx mADida kamaRvu beVre avasethxyalilxruva Atamxnige seVruvudilalxvo hAgeyeV savxpanxdalilxruva noVDuva Atamxnige seVruvudilalx eMdare-} 

\begin{artha}
heVge jAgaravasethxyalilx mADidakamaRvu jAgaradalilxruva kataRnige seVruvudo hAgeye savxpanxda kamaRvU savxpanxdalilxruva Atamxnige seVruvudu.
\end{artha}

\vishaya{I riVti mADida AkeSxVpakekx kelavara parihAra}

\begin{shl}
iteyxVvaM coVditeV keVcitapxrihAraM parxcakaSxteV | \\
padaceCxVdeVna kushalA \footnotemark[1]aratevxVteyxVvamAdinA \hfill||  1049 ||  
\end{shl}
\footnotetext[1]{aratAvx eMdu padaceCxVda mADidadxriMda suSupitxyalilx kirxVDisalilalx hAgeye saMcarisalU, noVDalU ilalxveMdeV athaRvu laBisi suSupitxyalilx jAcnxnavilalxdiruvAga ramaNwdigaLu heVge saMBavisuvavu? eMdu hiMde parxshinxsadadxkekx idu parihAravAguvudeMdu ivara aBipArxya.}

%%%%%shloka footnote[1]
\begin{artha}
I riVtiyAgi parxshenxyu baralu kelavaru parihAravanunx hiVge heVLuvaru heVge? `saMparxsAdeV\s ratAvx\s caritAvx' eMbalilx `aratAvx' `acaritAvx' `adaqSATxvX' itAyxdi riVtiyalilx padaceCxVda mADi kelavu kushalaru heVLuvaru.
\end{artha}

\vishaya{idu sariyalalxveMdu dUSisuvudu}
 
\footnotetext[2]{Adare I utatxravu sariyalalx sayatatxtarx kiMcitf pashayxti....itAyxdi vAkayx sheVSadalilx pashayxti eMdu dashaRnavanunx heVLidadxriMda noVDadaye eMdu heVLidadxkekx I vAkayxsheVSavu virudadhxvAgutatxde, vAtiRkadalilx `asaknogxVketxVH ' eMba pAThakikxMta `asaknogxVtetxVH ' eMba pAThavu sariyAgide; vAkayxsheVSavu saMgatavAguvudilalxvAdadxriMda eMbathaRvu parxkaqtoVcitavAguvudu shurxtiyalilxruva `tatarx' eMbudakekx `savxpAnxyeYva' eMdu heVLida savxpanxvanunx parAmashaRmADi savxpanxdalilx kirxVDisi, naDedu, noVDi eMdu athaRmADidare vAkayxsheVSa viroVdhavilalxveMdu yadayxpi heVLabahudu, Adare `savxpAnxyeYva' eMbudanunx parAmashaR mADalu idakekx bahudUravAguvudu, saninxhitavAgiyU parxtayxkaSxvAgiyU iruvudaneVnx tacaCxbadxvu parAmashaR mADuvudeV pArxyavAgiruvudu, oMdu veVLe parAmashaR mADidarU muMde savxpenxVratAvx eMdu savxpanxvanunx heVLuvudu vayxthaRvAguvudu, jAgarxtitxnalilx asaMgatavxvanunx parxtipAdisuvudakAkxgi adanunx heVLideyeMbudU sariyalalx `parxtiyoVnAyxdarxvati budAdhxnAtxyeYva' eMdu budAdhxnatx padadiMda jAgarasAthxnavanunx heVLuvudariMda vayxthaRvAguvudu, matutx tatarx eMba tatf padadiMda suSupitxyanunx parAmashaRmADuvudaMtU sariyalalx, Adasheyalilx `pashayxti' eMdu noVDutAtxneMdu BeVda daqSiTxyanunx heVLidaMtAgi anuBava viroVdhavAgiyU vAyxKAyxnavAguvudu suSupitxyalilx yAva BeVdadaqSiTxyU ilalxveMbudu yukatxvAdudu hiVge koneyalilx vAkayxsheVSada viroVdhaveV aratAvx acaritAvx adaqSATxvX itAyxdi padaceCxdavanunx opupxvavarige baruvudeMdu Ashaya.} 
 \begin{shl}
\footnotemark[2]sa yatatxterxVtayxMsaknogxVketxVneVRdamapuyxtatxraM BaveVtf | \\
daqSiTxM tatArxpi cAdhayxsayx parihAraM parxcakaSxteV \hfill||  1050 ||  
\end{shl}
 
%%%%%%shloka footnote[2]
\begin{artha}
``sa yatatxtarx kiMcitapxshayxtayx nanAvxgata setxVna Bavati'' ``asaknogxVhayxyaM puruSaH''eMdu vAkayx sheVSavu asaMgatavAguvudariMda I utatxravU sari hoVguvudilalx \footnote{beVre kelavaru savxpanxdaMte suSupitxyalUlx daqSiTxyanunx AroVpisi nacnf padada anavxyavilalxde `sayatatxtarx' eMbuva tatf padadiMda suSupitxyanunx parAmashaRmADi vAyxKAyxnisuvudariMda vAkayx sheVSavU sari hoVguvudeMdu tiLidu parihAra heVLuvaru, AvAga `A Atamxnu A suSupitxyalilx EnoMdanunx noVDuvano, adariMda Itanu aMTikoMDiruvudilalxveMdu' vAkayx sheVSada athaR mADabeVku.}matotxbabxru savxpanxdaMte suSupitxyalUlx jAcnxnavanunx AroVpisi (nacnf padakekx anavxyavilalxde padavanunx biDisutAtx) parihAravanunx heVLuvaru.
\end{artha}

\section*{vAtiRka}

\vishaya{ramaNAdigaLanunx AroVpisuvudakekx nimitatxveVneMdu keVLidare adara PalAnuBavaveV nimitatxvenunxtAtxre-}

\footnotetext[1]{`AnanadxmayoVhAyxnanadxBukf' eMdu mAMDUkayxdalilx Agama parxmANavU adanunx anusarisida gwDapArada meVlina ukitxyU ideyeMdathaR, ideV AnaMdAnuBavaveV suSupitxyalilx ramaNAdigaLa AroVpakekx nimitatxvAguvudu.}
\begin{shl}
\footnotemark[1]asitx BoVgaH suSupetxV\s pi tathAcA\s \s nanadxBuknaBxtaH | \\
\footnotemark[2]aishavxroV vA\s tarx BoVgoV\s sitx savaRsAthxnABimAnataH \hfill||  1051 ||  
\end{shl}
\footnotetext[2]{alalxde avidAyxbiVjadiMda kUDida parxtayxgAtamxneV Ishavxra jagatAkxraNa, I kAraNa savxrUpadalilxruva jiVvAtamxnu Ishavxrana AnaMdAnuBavavanunx hoMduvanu, ``EtasemxYvA\s nanadxsAyxnAyxni BUtAni mAtArx mUpajiVvanitx'' eMdu beVre shurxtiyalUlx Ishavxrana savxrUpAnaMda avalavaleVshavanenx elAlx pArxNigaLU anuBavisutatxveyeMdu heVLide, adariMdalU ramaNwdi boVgavanunx AroVpisi heVLiruvudeMdu aBipArxya, Ishavxrana BoVgavu jiVvanige kAraNa rUpadiMda heVLuvudakUkx kAraNavide. Ishavxranu savaRkAraNanAdadxriMda taninxMda saqSiTxyAguva mUru avasethxgaLU tananxdeMdu tiLidiruvanu, suSupitxyalilx kAraNarUpadiMda iruva jiVvanu avana BoVgakekx BAgiyAguvaneMdathaR.}

%%%%%shloka footnote[1, 2]
\begin{artha}
suSupitxyalUlx BoVgavu ide, ``AnanadxBukf tathApArxjacnxH'' eMbudAgi (suKAnuBavavanunx) opipxruvuduMTu, matutx Ishavxrana BoVgavAdarU I suSupitxyalilxde enanx beVku. elAlx sAthxnagaLalUlx (aMdare avasAthxtarxyadalilx) (Ishavxranige) aBimAnaviruvudariMda (hiVge heVLabeVku).
\end{artha}

\vishaya{dashaRnAdi vaqtitxgaLanunx suSupitxyalUlx AroVpisidare itarAvasethxgaLigU idakUkx visheVSavilalxveMdAgutatxde eMdare-}

\begin{shl}
kAyaRBUmigatoV hAyxtAmx suSupatxM parxsamiVkaSxyXteV | \\
kAyaRsayx kAraNavAyxpetxVnaR suSupatxparoVkaSxtA \hfill||  1052 ||  
\end{shl}

%%%%%shloka footnote[1]
\begin{artha}
\footnote{ililx suSupitx eMdare Atamxna ajAcnxna. biVjarUpavAda ajAcnxna, I kAraNAjAcnxnaveMba suSupitxyiMda edadxvanu kAyaRvAda jAgaradalolxV savxpanxdalolxV iruvanu. I kAyaR BUmiyalilxruva Atamxnu A savxpAnxdi sAthxnagaLalilx Aguva ramaNa (kirxVDe) saMcAra. vasutxdashaRna itAyxdi kAyaR heVtuviniMda kAraNavAda suSupitxyanunx takiRsuvanu ideV parxsamiVkaSxteV eMbudara athaR. kAyaRvu kAraNadiMda vAyxpatxvAgi ideVdx iruvudu, adariMda `savA ESa Etasimxnf savxpenxV ratAvx caritAvx daqSeTxYva..... adarxvati' eMdu heVLidudx. Adare takaRdiMda suSupitxyitetxMdu UhisuvudariMda adu paroVkaSxveMdu meVlemxle tiLiyuvudu sariyalalx anayxthA `sayatatxtarx yatikxMcitf pashayxti' eMdu dashaRnaveMda parxtayxkASxnuBavavaneVnx heVLide shurxtige viroVdha baruvudu adariMda suSupitxyeMbuva AtAmxjAcnxnavu paroVkaSxvalalx, adu aparoVkaSx, I ajAcnxnaveMba kAraNa rUpadalilx budidhxyu Aga alilxruvudu, edadx meVle `na kiMcida veVdiSamf' nAneVnU ariyalilalx parAmashaRvAguvudariMda vAkayxsheVSada viroVdhaveVnU ilalx eMdu tAtapxyaR.}kAyaRsAthxnadalilx (savxpanx jAgaragaLalilx) iruva Atamxnu suSupitxyanunx noVDuvanu EkeMdare kAyaRvu kAraNadiMda vAyxpatxvAgiruvudariMda (kAraNavAgiruva suSupitxyanunx takiRsuvanu) AdarU suSupitxyu paroVkaSxvalalx.
\end{artha}

\vishaya{takaRdiMda suSupitxyanunx UhisidarU adu beVre avasethxgaLige samAnaveMdAguvudu tapapxlilalxveMba sidAdhxMtavanunx heVLutAtxre -}

\footnotetext[2]{ramaNAdi vaqtitxgaLiMda suSupitxyanunx takiRsuvudAdarU adakUkx itarAvasethxgaLigU BeVdaveV ilalxveMdAguvudu. I doVSavu baruvudariMda sidAdhxMta ililx mADi heVLidAdxdare. savxpanxdaleVlx ramaNAdi kirxyegaLanunx heVLidare alilx mADida pApa puNayxgaLu savxpanxdalilxruva Atamxnige aMTuvudilalxve? aMTuvudAdare `ananAvxgata:'eMba viroVdhaveV baMdiVtu eMdare- adakAkxgi savxpanxyeVraBita: `eMba vAtiRkada utatxrAdhaRvu baMdide' ililx pArxjacnx eMdare suSupitxsAthxna, adara hiMdu muMdina avasethxgaLAda savxpanxgaLalilx pApa puNayxgaLu noVDuva atamxnalilx seVruvudilalx, savxpanxdiMda suSupitxge hoVguvanu, punaH alilxMda savxpanxkekx baruvanu, hiVge suSupitxya hiMdoMdu savxpanx muMdoMdu savxpanxvU saMBAvikavAgive I savxpanxgaLeraDaneVnx vAtiRkadalilx heVLide, I suSupitxya hiMde idadx savxpanxda kamaRdiMdalU muMde Aguva savxpanxdalilxruva Atamxnu aMTikoLuvudilalx, hAgeye muMde Aguva savxpanxda kamaRdiMdalU hiMdina savxpanxdalilxruva Atamxnu aMTikoLuLxvudilalx, hAgU savxpanxdalilxruva shuBAshuBagaLiMda suSupitxyalUlx jAgaradalUlx aMTikoLuLxvudilalx, adariMda ajAcnxnadiMda shuBAshuBagaLiMda kUDiruvaMte Atamxnalilx toVridarU vAsatxvikavAgi avasAthxtarxyadalUlx adu ilalxvAdadxriMda `anAnAvxgata' eMba vAkayxvu sidAdhxMtadalilx virudadhxvAgilalx'' ideV ``suSupetx nanavxsaMboVdhAtf ratAvxdikatha mucayxteV'' eMdu hiMde shaMkisida viSaya muKayx parihAraveMdu tiLiyabeVku.}
\begin{shl}
\footnotemark[2]yadi vA savxpanx EvAsutx ratAvxdi yadudiVritamf | \\
savxpanxyoVraBitaH pArxjAcnxdananAvxgatatA\s payxtaH  \hfill||  1053 ||  
\end{shl}

%%%%%shloka footnote[2]
\begin{artha}
athavA hiMde heVLida rati muMtAdavu savxpanxdaleVlx AguvaveMdu iTuTxkoLaLx bahudu pArxjAcnxtamxna sUtatxlU Aguva savxpanxgaLalilx (pApa puNAyxdi kirxyeya) saMbaMdha viruvudilalx.
\end{artha}

\begin{artha}
``savA ESa Etasimxnf saMparxsAdeV ratAvx caritAvx....''itAyxdi vAkayxdalilx 1046 neV vAtiRkadalilx hiMde mADida parxshenxge aMdare suSupitxyalilx rati, saMcAra, pApa puNayx dashaRna itAyxdigaLu heVge? saMgata? eMdu yathAshurxtavAgi keVLida parxshenxge matotxMdu samAdhAnavanunx Iga heVLuvaru-
\end{artha}

\vishaya{matotxMdu samAdhAna-}

\begin{shl}
tirxdhA tirxdhA vA kalxqqpetxVH sAyxdeVkeYkaseyxVtayxdoVSatA | \\
jAgarxtasxvXpanxsuSupAtxnAM jAgarxtasxvXpanxsuSupatxtaH \hfill||  1054 ||  
\end{shl}

\begin{artha}
jAgara, savxpanx, suSupitx ivugaLalilx oMdoMdanunx jAgarxtf, savxpanx, suSupitxyeMdu mUru mUrAgi kalipxsuvudariMda (suSupitxyalilx ramaNAdikirxyeyu idadxre doVSavilalx).
\end{artha}

\vishaya{adu heVge mUru mUrAgide? eMdare modalu jAgara mUru bage}

\begin{shl}
kAyARkAyARdi yatAsxkASxtapxrXmANAtapxrXsamiVkaSxyateV | \\
jAgarxjAjxgaritaM tAdaqkatxtasxvXponxV yanamxqqSeVkaSxNamf \hfill||  1055 ||  
\end{shl}
				
\begin{shl}
kAmAdiviSayAsaketxVnaR kiMcidivxvinakitx yatf | \\
jAgarxtusxSupatxM tAdaqkAsxyXninxviRveVkasavxBAvataH \hfill||  1056 ||  
\end{shl}

\begin{artha}
kAyaR, akAyaR modalAdavanunx neVra parxmANadiMda noVDuvudeMbudu yAvudo adeV jAgarxjAjxgaritaveMbuvudu, asatayxvAgiye kANuvudeMbudu yAvuduMTo adeV jAgarxtasxvXpanxveMdU. kAmAdi viSayAsakitxyiMda EnoMdanunx viveVcisi tiLiyade iruvudeMbudu yAvudo adeV jAgarxtusxSupitxyeMdU heVLalapxDuvudu'' ekeMdare? viveVka vilalxda savxBAvavAdadxriMda suSupitxyeMdu idanunx heVLuvudu.
\end{artha}

\vishaya{savxpanx mUru bage hAgU suSupitxyU mUru bage-}

\begin{shl}
supotxV\s pi kamaR kuruteV naraH savxpenxV parxboVdhavatf | \\
savxpanxjAgarxtatxthArUpaM savxpanxH savxpAnxtamxkoV\s tarx yaH \hfill||  1057 ||  
\end{shl}
				
\begin{shl}
daqSATxvX\s pi yatasxmAKAyxtuM parxbudodhxV neYva shakunxyAtf | \\
tAdaqkasxvXpanxsuSupatxM sAyxtusxSupatxM ca tirxdhoVcayxteV \hfill||  1058 ||  
\end{shl}
				
\begin{shl}
suSupatxjAgarxnUmxDhaH sAyxcACxnotxV\s sw savxpanx ucayxteV | \\
aikAtamxyXtatAtxvXsaMboVdhaH suSupatxH pArxjacnx ucayxteV \hfill||  1059 ||  
\end{shl}

\begin{artha}
naranu malagidavanU saha kelasavanunx ecacxrinalilx mADuvaMte savxpanxdalilx mADuvuduMTu adeV savxpanx jAgarxtutx, savxpanxdalilx savxpanxvAdaMtiruva anuBavaveV savxpanxveMbudu, noVDidadxrU yAvudanunx ecacxrAdavanu heVLalu asamathaRnAguvano adu savxpanxsuSupitxyenunxvudu hiVgeye suSupitxyU saha mUru bageyAgi heVLalapxDuvudu. mUDhanAgidadxre suSupatx jAgarxtetxMdU, shAMtanAgidadxre suSupitx savxpanxveMdU EkAtamx savxrUpaveMba tatavxvu tiLiyadiruvudeV suSupatx pArxjacnx (suSupitx)yeMdu heVLalapxDuvudu.
\end{artha}

\vishaya{Iga I bageya suSupitxge parxkaqtoVpayoVgavanunx heVLuvaru-}

\begin{shl}
tirxvidhatAvxtusxSupatxsavx savaRM ratAvxdi yujayxteV | \\
daqSATxnatxH saMparxsAdoV vA savxpanxjAgarxdavasathxyoVH \hfill||  1060 ||  
\end{shl}

\begin{artha}
suSupitxyu mUru bage yAdadxriMda ratAvx caritAvx itAyxdiyAgi heVLidedxlalxvU yukatx vAguvudu, suSupitxyanunx savxpanx jAgarAvasethxgaLige \footnote{heVge suSupitxyalilx Atamxnu kAyaRkaraNa saMGAtadiMda biDalapxTuTx savxparxkAsha ceYtanayx rUpanAgiruvaMte kamARdigaLiMda biDalapxTaTxMte itarAvasethxgaLalUlx iruvaneMdu BAva.}daqSATxnatxvAgi heVLide.
\end{artha}

\vishaya{daqSATxMta dASATxRMtikagaLalilx hoVlikeyanunx sapxSaTxpaDisuvudu-}

\begin{shl}
yathA ratAvxdayxsaMBAvayxM saMparxsAdeV tatheYva ca | \\
itaratArxpi vijecnxVyaM \footnotemark[1]tatarx vasatxvXsamiVkaSxNAtf \hfill||  1061 ||  
\end{shl}
\footnotetext[1]{`tatavx' eMdu pAThAMtaravide. avAga tatavxveMbuva vasutxvu kANadiruvudariMda eraDu avasethxgaLalUlx ratAyxdigaLu saMBavisuvudilalx veMdathaR.}

%%%%%shloka footnote[1]
\begin{artha}
suSupitxyalilx heVge ramaNAdi kirxyeyu asaMBAvitavo hAgeye beVre avasethxgaLalUlx asaMBAvitaveMdu tiLiyabeVku EkeMdare; adaralUlx nijavAda vasutxkANadiruvudariMda hiVge tiLiyabeVku.
\end{artha}

\vishaya{parxkaraNa viSayavanunx upasaMharisuvudu-}

\begin{shl}
yata EvamataH kamaRkAmadoVSavivajiRtaH | \\
puruSoV\s yaM savxtaH sidodhxV \footnotemark[2]yathoVketxVneYva vatamxRnA\hfill ||  1062 ||  
\end{shl}
\footnotetext[2]{kAma saMga eraDU payARya shabadxgaLAdadxriMda kAmavu savxpanx parihAradalilxdeyeMdu heVLidadxriMdale asaMga saMgavilalxdavaneMdu heVLuvudakekx Aguvudilalx, adariMda heVtuvu asidadhx veMdathaR.}

%%%%%shloka footnote[2]
\begin{artha}
I riVtiyAgi (aMdare kamaRda upakaraNa kAma, adu parxyoVgada mUlaka kArakavanunx AsharxyisuvudeMdu) hiMde heVLida riVtiyalilx I puruSanu yAvakAraNadiMda kamaR kAmagaLeMba doVSa shUnayxnAgiruvano, adariMda savxtaH kataRnalalx eMdu sidadhxnAguvanu.\end{artha}

\vishaya{`asaknogxVhi' eMdu heVtuvu Atamxnalilx pApAdi kataqRtavxvu seVradiruvudakekxMdu heVLideyaSiTx, adu sididhxsuvudilalxveMdu AkeSxVpa-}

\begin{shl}
asaknagxtAvxdakateVRti nanavxsidodhxV\s yamucayxteV | \\
savxpanx kAmasayx daqSaTxtAvxdayxtarx kAmamiti shurxteVH \hfill||  1063 ||  
\end{shl}

\begin{artha}
asaMganAdadxriMda kataRnalalx eMdu heVLida heVtuvu asidadhx= ilalx eMdu heVLalapxDuvudu adu heVge? eMdare- savxpanxdalilx kAmavu kaMDiruvudariMdalU saIyateV\s maqtoV yatarxkAmamf eMdu shurxtiyiMdalU kaMDidadxriMdalU asaMtavx heVtuvu sididhxsilalx.
\end{artha}

\vishaya{parihAra}

\begin{shl}
itayxsayx parihArAthaRM sa vA itAyxdikaM vacaH | \\
punaH savxpanxsamAramoBxV gatatAvxtikxmitiVyaRteV | \\
\footnotemark[1]budAdhxnAtxdeVyaRthA tadavxnAnxtaH sAyxtupxnarukatxtA \hfill||  1064 ||  
\end{shl}
\footnotetext[1]{athaRBeVda tAtapxyaRBeVda viruvudariMda punarukitxyilalxveMdu BAva, jAgara suSupitxgaLalilx athaRBeVdaviruvudariMda heVge punarukitxyilalxvo, hAgeye savxpanxvanunx aneVka sala heVLuvudu punarukitxyalalx, `sahisavxponxV BUtAvx' eMbudAgi kAyaRkaraNa saMGAtakekxMda Atamxnu beVreyeMdu tiLisalu Atamxna savxpanxvanunx heVLide `sayatatxtarxparxsavxpiti' eMbalilx savxpanxvanunx heVLidudx avanu savxyaM joyxVtiyeMbudanunx tiLisuvudakekx `saMparxsAdeVratAvx itAyxdi maMtarxdalilx kamaRviveVcanegAgi adanunx heVLide, ililx kAma viveVcanegAgi heVLide eMdu parxyoVjana BeVdadiMdalo tAtapxyaRBeVdadiMdaloV punarukitx doVSavilalxveMdu BAva.}

%%%%%shloka footnote[1]
\begin{artha}
 I AkeSxpavanunx pariharisalu `savA ESa Etasimx ratAvx caritAvx.....' itAyxdi vacanavu baMdide, `sahi savxponxV BUtAvx' eMba vacanakekx idu punarukatx gatavAdadxriMda punaH savxpanx vicAravanunx heVLuvudeVtakekx? eMdare budAdhxnAtxdi viSayadalilx heVge punarukitx yilalxvo hAgeye idariMda punarukitxyilalx.
\end{artha}

\vishaya{savxpanx sAthxnadiMda suSupitxge hoVgiruvudanunx adhikAMshavanunx heVLalu anuvadisuvaru-}

\begin{shl}
nishAyAM saMparxvaqtAtxyAM saMhaqtAkaSxsayx nidarxyA | \\
avasAthxM teYjasiVM BukatxvXM sAvxpoV Bavati deVhinaH \hfill||  1065 ||  
\end{shl}

\begin{artha}
rAtirxyu baMdAga niderxyiMda elAlx iMdirxyagaLanunx upasaMharisi koMDiruva jiVvanige savxpAnxvasethxyanunx anuBavisidanaMtara suSupitxyu baruvudu.
\end{artha}

\begin{shl}
sa AnanadxH paroV jecnxVyaH suKaduHKavivajiRtaH | \\
satataM savaRBUtAnAM tiSaThxteyxVSa puriVtati \hfill||  1066 ||  
\end{shl}

\begin{artha}
A suSupitxye suKaduHKavilalxda paramAnaMdaveMdu tiLiya beVku. I AnaMdavu samasatx pArxNigaLa \footnote[1]{puriVtatf eMdare haqdaya koVshavanunx sututxvareda mAMsaKaMDaveMdu BASayxkAraru sUtarxkAraru sUtarx BASayxdalilx heVLidAdxre, ``puriVtaditi haqdaya pariveVSaTxna mucayxteV'' (sU. BA. a. ) puriVtatf eMdare oMdu nADiyeMdu tAkiRkaru heVLuvaru, namage AyuveRVdadalilx heVLidaMte athaRniNaRya mADiruvudu sariyAdadudx. adaralilx jiVvAtamxna upAdhiyu aMtaHkaraNavu layavAdare shudadhxvAda barxhamx savxrUpavu I puriVtatitxna mUlaka jiVvanige laBisuvudu idu savaRvAyxpakaveMdu tiLisalu samasatx pArxNigaLa saMbaMdhavanunx tiLisidAdxre satataM eMbudariMda itarAvasethxgaLalUlx ideV tatavxBUtavAda AnaMda vasutx iruvudeMdu tiLiside. adu ililx sAMsArikavAda suKaduHKagaLiMda biDalapxTiTxruvudariMda utakxqqSaTxvAdadedxMdu para eMbudariMda sUciside. A savxrUpAnaMdave idu eMdu atayxMta saninxhitaveMdu tiLisalu ESa: eMdide A paramAnaMdavu ideV parxtayxgAtamxne: eMdathaR.}puriVtatitxnalilx yAvAgalU iruvudu.
\end{artha}

\footnotetext[2]{sAvxpanxveMdu hiMde heVLida suSupitxyanunx hoMdidavanu mukatxnaMte beVre avasethxyanunx hoMdalAranaSeTx? eMdu parxshinxsidalilx utatxravidu, I saSupitxyalilx kirxVDisi aMdare hiMde heVLida paramAnaMdavanunx niviRkalapxka nitayx ceYtanayx rUpadiMda anuBavisi kelavukAlaveMdare-beVre avasethxge beVkAda kamaRgaLu ililx aBivayxkatxvAgadiruva kAladalilxdudx paramAnaMdavanunx anuBavisi naMtara yadaqceCxyiMda akasAmxtAtxgi I avasethxyanunx tirasakxrisuvanu, ililx kaMcitf kAlamf = eMdare savxpanx jAgaragaLige kAraNavAda kamaRgaLu aBivayxkitx hoMdadiruva kAladalilx eMdathaR.}
\footnotetext[3]{yadaqceCxyeMdare niyAmakavilalxde yadAkadAcitf kAyaRvu udayisuvudeMdu heVLuvudu Adare ililx A athaRvilalx savxpanxkekx kAraNavAda AviBaRvisida kamaRveMdathaR, I kamaRvu udaBxvisidalilx matotxMdu avasethxyanunx savxpAnxvasethxyanunx hoMduvaneMdathaR suSupatxyXvasethxyanunx mucicx matotxMdu avasethxyanunx hoMdalu muKayx kAraNa kamaR adu udayisabeVku, paripAkadasheyanunx hoMdabeVku, AgaleV savxpanx biVLuvudu, athavA jAgarAvasethx kANuvudu.}
\begin{shl}
tatarx ratAvx yathAkAmaM \footnotemark[2]kaMcitAkxlaM \footnotemark[3]yadaqcaCxyA | \\
tAmavasAthxM tirashacxkarx AyiyAsusatxtoV\s parAmf \hfill||  1067 ||  
\end{shl}

%%%%%shloka footnote[2, 3]
\begin{artha}
alilx (sAvxpanxdalilx) tananx icACxnusAra kelakAla kirxVDisi A suSupitx avasethxyanunx kamARnusAra tirasakxrisuvanu EkeMdare? adakUkx nikaqSaTxvAda savxpAnxvasethxyanunx hoMdalu bayasidavanAgi tirasakxrisuvanu.
\end{artha}

\vishaya{Atamxnu savxpanxkekx hoVgi Enu mADuvanu?}

\begin{shl}
BAvanAvigarxhoV BUtAvx tatarx savxpanxriraMsayA | \\
ucAcxvacAni vasUtxni \footnotemark[4]BAvanAtaH karoVti saH \hfill||  1068 ||  
\end{shl}
\footnotetext[4]{BAvaneyiMda nimiRsuvaneMdu heVLidadxriMda adu mitheyxyeMdu tiLiya beVku, vAsanA nimiRtavAda gajaturagAdigaLu asatayxveMbudakekx matotxMdu kAraNavide.}

%%%%%shloka footnote[4]
\begin{artha}
savxpanxdalilx jAgarada vAsanA nimiRtavAda shariVravuLaLxvanAgi avanu alilx kirxVDisaliceCxyiMda aneVka bageyAda vasutxgaLanunx vAsaneyiMdaleV nimiRsuvanu.
\end{artha}

\footnotetext[5]{savxpanxdalilx noVDuva Atamxnu dUraviruvadanunx noVDuvanu ihakAkxgi deVhavanunx biTuTx horage hoVgi tiLiyuvudAdare pArxNaviyoVgavAguvudu, AvAga maraNave Aguvudu, AdakAraNa horage hoVgade bahudUraviruvudanunx noVDuvudariMda adu mitheyxyeMdu tiLiyabeVku, alalxde aneVka ahoVrAtirxgaLalilx noVDabeVkAda matutx aneVka divasagaLalilx saqSiTxsa beVkAda rathAdigaLu savxpanxdalilx kaSxNa mAtarxdalelxV saqSiTxyAdaMte kANuvudu, jAgarxtitxnalilxruva rathAdigaLige yoVgayxvAda kAla, deVsha, yAvudoMdu sAmagirxyU savxpanxdalilx deVhadoLage iruvudilalx, iruva saMBavavU ilalx, AdarU kANuvavu adariMda mitheyxyeMdu tiLiyabeVku `mAyA mAtarxM tu kAtenxRsXnAnaBivayxkatx savxrUpatAvxtf' eMba barxhamXsUtarxdalilx savxpanx mithayxyeMbudAgi niNaRyiside.}
\begin{shl}
gatw pArxNaviyoVgaH sAyxtf \footnotemark[5]yathAkAlasayx ceVkaSxNamf | \\
tadasaMBAvayxdeVshAdw dashaRnAtatxnamxqqSAtamxtA \hfill||  1069 ||  
\end{shl}

%%%%%shloka footnote[5]
\begin{artha}
Atanu (savxpanxdalilx) horage hoVgidadxre pArxNa viyoVgaveV (sAveV) uMTAdiVtu, yoVgayx kAlavuLaLx vasutxvina dashaRnavu Aguvudu, asaMBAvitavAda (saMBavisadiruva) deVsha modalAda sathxLadalilx savxpanx vasutxvanunx kANuvudariMda adu mitheyxyeMdu tiLiyabeVku.
\end{artha}

\begin{artha}
savxpanx viSayadalilx eSuTx heVLa beVkAdadudx ideyo avelalxvanunx I modaleV ( ) heVLidAdxgide. adariMda I Atamxnige savxpanxdalilx kAmada gaMdhavU ilalx.
\end{artha}

\section*{(baq - 4 - 3 - 17)}
 
 \begin{shl}
sa vA ESa EtasimxnubxdAdhxnetxV ratAvx caritAvx daqSeTxvXYva puNayxM ca pApaM ca punaH parxtinAyxyaM parxtiyoVnAyxdarxvati savxpAnxnAtxyeYva || 17 ||
\end{shl}
 
\vishaya{I meVlina shurxtiya tAtapxyaR.}

\begin{shl}
asaknagxtAvxdasaMbanodhxV yatheVha savxpanxkamaRBiH | \\
jAgarxtayxpi na katAR\s yameVtasAmxdeVva kAraNAtf \hfill||  1071 ||  
\end{shl}

\begin{artha}
I savxpanxdalilx Atamxnu asaMganAdadxriMda savxpanxkamaRgaLa (kirxyegaLa) saMbaMdhaviruvudilalxvo hAgeye jAgarxtitxnalUlx Itanu ideV kAraNadiMda kataRnalalxveMdu tiLiyabeVku.
\end{artha}

\begin{shl}
asajAjxgarxdasethxVyaM jAgarxdUrxpatavxkAraNAtf | \\
savxpenxV jAgarxdavasAthxvatatxthAcA\s \s dayxnatxvatatxvXtaH \hfill||  1072 ||  
\end{shl}

\begin{artha}
satatxlalxdudx\footnote{ililx eraDu takaRgaLive jAgarAvasethxyu mitheyxyeMbudakekx jAgarada savxrUpavAdadxriMda eMbudu heVtu yukitx, daqSATxMta savxpanxdaleVlx kANuva jAgarAvasethx matotxMdu utapxtitxvinAshavuLaLxdadxriMda eMbudu heVtu.} I jAgarAvasethx, jAgarxtitxna savxrUpavAda kAraNadiMdaleV hiVgeMdu takiRsuvudu. savxpanxdalilx kANuva jAgarAvasethxyaMte, hAgU utapxtitxvinAshagaLuLaLxvugaLAdadxriMdalU mitheyx satatxlalx
\end{artha}

\begin{artha}
AdA vanetxV ca yanAnxsitx vataRmAneV\s pi tatatxthA |\\
vitatheYH sadaqshAH sanotxV\s vitathA ivalakiSxtAH ||
eMdu gwDapAdaru heVLidaMte I yukitxyide yAvudu huTuTxva modalU huTiTxnAshavAda meVlU iruvudilalxvo adu  iruva kAladalUlx ilalxveMde heVLabeVku. hiVge asatAtxgidadxrU padAthaRgaLu mithAyxvasutxvige samAnavAgidadxrU satayx vasutxvinaMte toVrive aSeTx eMdu kArikeya athaR. parxkaqta jAgarxdavasethxyu adaralilx kANuvavasutxgaLU saha Adi aMtayxviruvavAdadxriMda hiMdeyU muMdeyU ilalxda asadf vasutxvAgi madhayxkAladalilx satayxvAgiruvaMte toVruvavAdadxriMda asatetxMdu tiVmARnavu hiMdumuMdu ilalxda vasutxvige madhayxkAladalilx satetxyu heVge baruvudu? asatutx satAtxguvudilalx, satutx asatAtxguvudilalx asatutx asateVtx, satutx satetxV, ``nAnaroV vidayxteV BAvoV nABAvoV vidayxte sataH''eMdu giVtAthaRvanunx ililx samxrisabeVku.
\end{artha}

\vishaya{mitheyxyeMbudakekx sAdhakavAgi matotxMdu takaR-}

\begin{shl}
ajAcnxnoVtathxmidaM jAgarxjajxDarUpasamanavxyAtf | \\
tathA\s peVkASxtamxkatAvxcacx maqgataqSoNxVdakAdivatf \hfill||  1073 ||  
\end{shl}

\begin{artha}
I jAgarxtutx ajAcnxnadiMda huTiTxkoMDide, jaDa dhamaRvu adaralilx seVriruvudariMdalU sApeVkaSxvAda savxrUpavuLaLxdadxriMdalU maruBUmiyalilx bidadx bisalu JaLadalilx kANisuva niVrinaMte ajAcnxnadiMda huTiTxkoMDadudx (adariMda mitheyx).
\end{artha}

\begin{shl}
nirAsaknagxH pumAneVSa savxpanxbudAdhxnatxyoVrapi | \\
saMparxsAdeV kimaknAgxyaM yatarx kiMcinanx viVkaSxyXteV \hfill||  1074 ||  
\end{shl}

\begin{artha}
I puruSanu savxpanx matutx jAgaragaLalilx AsaMgavilalxdavanu=kAmavilalxdavaneMdu tiLida meVle suSupitxyalUlx puruSanu AsaMga= kAmAvilalxdavaneMdu heVLa beVkAdadudx Enide? ilalx EkeMdare yAva dasheyalilx oMdu vasutxvU kANisuvudilalxvo: (A suSupitx dasheyalilx kAmavelilxMda barabeVku?)
\end{artha}

\begin{artha}
IvaregU avasAthxtarxyadalUlx asaMganAdadxriMda Atamxnu kataRnalalx eMdu heVLidAdxyitu, adaralilx Iga jAgarAvasethxyalilx kataqRtavxvilalxveMbudanunx AkeSxVpisuvudu-
\end{artha}

\vishaya{AkeSxVpa}

\begin{shl}
nanu jAgarxdavasAthxyAM daqSeTxvXYveVtuyxcayxteV kathamf | \\
parxtayxkaSxkataqRtA tatarx yata Atamxna IkaSxyXteV \hfill||  1075 ||  
\end{shl}

\begin{artha}
jAgarAvasethxyalilx (pApa puNayxgaLanunx) noVDiyeV eMdu heVLidudx heVge? EkeMdare! Atamxnalilx parxtayxkaSxvAgi kataqRtavxveV kANuvudu adariMda (noVDidudx mAtarxveV eMbudu heVge?)
\end{artha}

\vishaya{parihAra}

\begin{shl}
neYvaM katArxRdisAkiSxtAvxtatxcicxdABAsatasatxthA | \\
kataqRtavxmAtamxnoV boVdha upacArAdayxtasatxtaHsa \hfill||  1076 ||  
\end{shl}

\begin{artha}
I riVtiyalalx, kataqRmodalAdavugaLige sAkiSxyAdadxriMda ceYtanAyxBAsada mUlaka Atamxnige kataqRtavxvu yAvudariMda hiVgideyo adariMda upacAradiMda gwNavaqtitxyiMda kataqRtavx jAcnxnaveMdu tiLiyabeVku.
\end{artha}

\vishaya{kataqRtavxvu aupAdhikaveV, savxtaH kataqRtavxviruvudilalxveMbudakekx parxmANavanunx toVrisuvaru -}

\begin{shl}
AtamxneYvAyamiti ca dhAyxyatiVveVti ca shurxteVH | \\
nA\s \s tamxnaH kArakatavxM sAyxtasxvARvasAthxtilakniGxnaH \hfill||  1077 || 
\end{shl}

\begin{artha}
``AtamxneYvAyaM joyxVtiSA\s \s setxV'' ``dhAyxyatiVva leVlAyatiVva'' eMba shurxtiyiMda Atamxnige elAlx avasethxgaLanunx miVriruvavanige kArakatavxvu iruvudeV ilalxveMdu (tiLidide).
\end{artha}

\vishaya{I viSayakekx giVtAsamxqqtiyanunx Adharisalu horaTidAdxre-}

\begin{shl}
veVdAtAmx\s pi tathAcA\s \s ha parxpanAnxya kiriVTineV | \\
saMsAraheVtunibiDadhAvxnotxVcicxciCxtasxyA\s \s darAtf \hfill||  1078 || 
\end{shl}
				
\begin{shl}
anAditAvxninxguRNatAvxtapxramAtAmx\s yamavayxyaH | \\
shariVrasothxV\s pi kwnetxVya na karoVti na lipayxteV \hfill||  1079 ||  
\end{shl}

\begin{artha}
veVda puruSanAda BagavanatxnU sharaNAgatanAda ajuRnanige saMsArakekx kAraNavAda daTaTxvAda ajAcnxnavanunx nAshagoLisaliceCxyiMda Adara pUvaRkavAgi hAgeye heVLiruvanu.
\end{artha}

\begin{artha}
adeVneMdare eleY ajuRna keVLu anAdiyAgidadxriMdalU niguRNanAdadxriMdalU I avayxyanAda paramAtamxnu shariVradoLage idadxrU (kamaRvanunx) mADuvudilalx. adara PaladiMda leVpisalapxDuvudilalx. (kataRnU alalx, BoVkatxnU alalx) ||
\end{artha}

\vishaya{mahAmatasxyXvAkayxkekx pUvARpara saMbaMdhavanunx heVLuvaru-}

\begin{shl}
savxpanxbaqdAdhxnatxvAkAyxBAyxM maqtoyxVrAsaknagxlakaSxNAtf | \\
vivikatxteYvaM yatenxVna niNiVRtA parxtAyxgAtamxnaH \hfill||  1080 ||  
\end{shl}

\begin{artha}
savxpanxvAkayx budAdhxMta vAkayxgaLiMda kAmarUpavAda maqtuyxviniMda parxtayxgAtamxnige parxyatanx pUvaRkavAgi vivikatxteyanunx (beVpaRTiTxruvudanunx) niNaRyiside.
\end{artha}

\vishaya{hAgU kamaRdiMda vivikatxneMdU adariMdale niNaRyiside- enunxtAtxre}

\begin{shl}
anoyxVnayxparihAreVNa savxpanxjAgarxdavasathxyoVH | \\
kAmakamARtamxkAsaknagxvivikatxtavxmihoVditamf \hfill||  1081 ||  
\end{shl}

\begin{artha}
savxpanx jAgarxtf eMba eraDu avasethxgaLalUlx parasapxra biTiTxruvudariMda Atamxnalilx kAma matutx kamaRrUpavAda AsaMgadiMda beVpaRTiTxruvudanunx ililx heVLidAdxgide.
\end{artha}

\begin{shl}
yathoVditAthaR EtasimxnadxqqSATxnotxV\s payxdhunoVcayxteV | \\
nirAsaknagxtavxsidadhxyXthaRM parxtayxgAtAmxKayxvasutxnaH \hfill ||  1082 ||  
\end{shl}

\begin{artha}
I meVle heVLida viSayadalilx Iga daqSATxMtavanunx heVLuvudu EtakAkxgi? eMdare parxtayxgAtamxveMba vasutxvige (kAma kamaRgaLeMba) AsaMgavilalxveMbudanunx tiLiyapaDisuvudakAkxgi.
\end{artha}

\section*{baq-a4-bArx3-18ne kaMDide}

\begin{shl}
tadayxthA mahAmatasxyX uBeV kUleV anusacnacxrati pUvaRM cAparaM ceYvameVvAyaM puruSa EtAvuBAvanAtxvanusacnacxrati savxpAnxnatxM ca budAdhxnatxM ca || 18 ||
\end{shl}

\vishaya{shAMkara BASayxdaMte mahAmatasxyX vAkayxda vAyxKAyxna-}

\vishaya{mahAmatasxyX: `eMbalilx mahatetxMba visheVSaNada parxyoVjana-}

\begin{shl}
sAvxtanatxrXyXparxtipatatxyXthaRM mahAniti visheVSaNamf | \\
nAdeVyasorxVtasA\s hAyoVR matosxyXV daqSATxnatx ucayxteV \hfill||  1083 ||  
\end{shl}
				
\begin{shl}
yathA sinodhxVmaRhAmatasxyX uBeV kUleV mithaH paqthakf | \\
anukarxmeVNa saMcaranamxtosxyXV\s nayxH kUlatoV mataH \hfill||  1084 ||  
\end{shl}

\begin{artha}
mahAnf eMba visheVSaNavu matasxyXda sAvxtaMtarxyXvanunx parxkAsha paDisuvudakAkxgide. nadiya parxvAhadiMda alAlxDisuvudakekx Agada matasxyXvu asaMga parxtayxgAtamxnige daqSATxMtavAgi heVLalapxTiTxde. mahAmatasxyXvu nadiya eraDu tiVragaLigU anoyxVnayx anukarxmavAgi (paratiVradiMda pUvaRkUkx pUvaRtiVradiM paratiVrakUkx) hiVge saMcarisutAtx idudx eraDu tiVragaLiMdalU heVge beVpaRDuvudo, hAgeye.
\end{artha}

\vishaya{saMgaviruva matasxyXvu asaMga puruSanige heVge daqSATxMtavAdiVtu?-}

\begin{shl}
kUlABAyx karxmasaMbanAdhxdavxyXtireVkAcacx kUlataH | \\
sorxVtoVmadheyxV davxyApArxpetxVrasaknogxV\s yaM timiyaRthA \hfill||  1085 ||  
\end{shl}
				
\begin{shl}
jAgarxtasxvXpanxkulAyABAyxM karxmasaMbanadhxtasatxthA | \\
vayxtireVkoV davxyApArxpetxVrasaknogxV\s yaM pumAnapi \hfill||  1086 || 
\end{shl}

\begin{artha}
eraDu tiVragaLoMdige karxmavAgi saMbaMdhavAgudariMdalU, tiVragaLige beVpaRTiTxruvudariMdalU parxvAhada madhayxdalilx eraDu tiVragaLanunx tAnu hoMdilalxvAdadxriMdalU matasxyXvu heVge asaMgavAgiye (saMbaMdhavilalxde ideyoV) hAgeye jAgara savxpanxgaLa eraDu gUDugaLoDane karxmavAgi saMbaMdha viruvudariMdalU, hAgeye I jAgarasavxpanxgaLiMda beVpaRTiTxruvudariMdalU I eraDanUnx (sUthxla sUkaSxmX deVhagaLanUnx) (naDuve suSupitxyalilx) hoMdilalxvAdudariMdalU I puruSanu asaMganu.
\end{artha}

\begin{artha}
deVhAtamxvAdiyu jAgara matutx savxpanxda deVhagaLanunx biTuTx beVreyAgi matotxMdu AtamxviruvudeMdu opipxlalxvAdadxriMda adakekx asaMgatavxvanunx aMgiVkarisuvudilalxvalalx? eMdu shaMkisidare adakekx utatxra.
\end{artha}

\begin{shl}
yugapatAsxyXdadxvXyoVBoVRgoV jAgarxtasxvXpanxkulAyayoVH | \\
BUtamAtArxvisheVSatAvxtAtxBAyxmAtAmx na ceVtapxqqthakf \hfill||  1087 ||  
\end{shl}

\begin{artha}
A jAgarxtf savxpanxgaLa gUDugaLiMda Atamxnu beVpaRTiTxlalxvAdare paMcaBUtagaLa sUkaSxmXgaLu eraDu shariVragaLigU samAnavAdadxriMda EkakAladalilx jAgarxtf savxpanxgaLa gUDugaLeraDaralUlx BoVgavu Aga beVkAgi baruvudu.
\end{artha}

\vishaya{BataqRparxpaMcaru heVLidaMte mahAmatasxyX vAkayxda pUvARpara saMbaMdha-}

\begin{shl}
saMbanadhxM kuvaRteV keVcidanayxtheYva mahAdhiyaH | \\
mahAmatAsxyXdivAkayxsayx sa yathoVkatxsatxthoVcayxteV \hfill||  1088 ||  
\end{shl}

\begin{artha}
kelavaru mahAbudidhxvaMtaru mahA matAsxyXdi vAkayxkekx beVre riVtiyalilx pUvARpara saMbaMdhavanunx mADiruvaru, heVge adanunx heVLiruvaro hAge adanunx muMde heVLuvudu
\end{artha}

\footnotetext[1]{vAtiRkadalilx budidhx modalAdavanunx parxyoVgaveMdu karedide kAraNaveVneMdare parxyoVgakekx kAraNavAdadxriMda parxyoVgaveMdu `lAMgalaM jiVvanamf' eMdu heVLuvaMte gwNavAgi karedide, I budAdhxyXdigaLaneVnx AsharxyisideyAdadxriMda kamaRvu AtamxnalilxlalxveMdathaR. BataqRparxpaMcara BASayx vacanavanunx AnaMdagirigaLu ililx udAharisidAdxre- yathAhuH- ``asayxhi puruSasayx bAhayxM kamaR budAdhxyXdiparxyoVgAshirxtamf'' eMdu hAgU `tataH kamaRNoV BAvanA sameVtayx Ena mAsakxnadxti tayoV viRveVkoV vAyxKAyxta iti' eMdU udAharisidAdxre-}
\begin{shl}
\footnotemark[1]bAhayxM kamARsayx budAdhxyXdiparxyoVgAshirxtameVva hi | \\
puMsaH kila tatoV\s BeyxVtayx BAvanA\s \s ponxVti deVhinamf \hfill||  1089 ||  
\end{shl}
				
\begin{shl}
tasAyx viveVkA vAyxKAyxtoV yathoVkatxvacasA\s \s tamxnaH | \\
\footnotemark[2]avidAyx tAvxtamxvijAcnxnasaMshirxteYva na sA\s nayxtaH \hfill||  1090 ||  
\end{shl}
\footnotetext[2]{kAmakamaRgaLa viveVkavanunx hiMde mADikoTiTxruvudanunx anuvadisi muMde heVLuva saMbaMdhakekx upayukatxvAgiruva avideyxya savxBAvavanunx adara kAyaRvanunx heVLalu (vAtiRkakAraru) BataqRparxpaMcaru heVLiruvaMte IvAtiRkadalilx avidAyxvicAravanenxtitxdAdxre.}

%%%%%shloka footnote[1, 2]
\begin{artha}
puruSana BataqRparxpaMcada avidAyx vaNaRne horagina kamaRvu budidhx modalAda parxyoVga (kAraka samudAya)vanunx avalaMbisiye ideyaSeTx. (kamaRdiMda huTiTxda) vAsaneyu budidhx modalAda sathxLadiMda Atamxnige edurAgi baMdu seVruvudu. A vAsanA viveVkavanunx Atamxnige hiMde heVLida asaMgoVhi eMba vacanadiMda vAyxKAyxnisidAdxyitu, Iga muMde heVLuva saMbaMdhakekx upayukatxvAda avideyxya savxBAvavanunx heVLiruvaru eneMdare- avideyxyAdaro Atamxna vijAcnxnavanunx AsharxyisikoMDeV iruvudu, beVroMdanunx Asharxyisi koMDilalx.
\end{artha}

\begin{artha}
I avideyxyu Atamx ceYtanayxvanunx AsharxyisikoMDeV iruvudu, adanunx biTuTx beVre kaDe ilalx, beVre AsharxyavU ilalx adu savxtaMtarxvAgiyU ilalx hAgidadxre savxrUpakekx BaMga baruvudu. adariMda Atamxna vijAcnxnavanunx AsharxyisideyeMdu heVLide.
\end{artha}

\vishaya{vAtiRka}

\vishaya{Atamxnalilxruva avideyxya eraDu kAyaRgaLu-}

\footnotetext[1]{`utatxMca tadeVva vijAcnxnaM vikaqtayx vipariVta garxhAya parxkalapxyati' eMdu BataqRparxpaMcara BASayxvacana (AnaM-TiVkA)}
\begin{shl}
yatatxdivxjAcnxnamAtimxVyaM \footnotemark[1]tadivxkaqtAyxvatiSaThxteV | \\
mithAyxjAcnxnagarxhAyAsAvavidAyx parxtayxgAtamxnaH \hfill||  1091 || 
\end{shl}

%%%%%%\shloka footnote[1]
\begin{artha}
yAvudu Atamxna vijAcnxnavideyo adanunx vikAragoLisi BArxMti jAcnxnavanunxMTu mADalu I parxtayxgAtamxna avideyxyu nilulxvudu.
\end{artha}

\vishaya{avideyxyiMda Ada ceYtanayxvikAra savxrUpa-}

\begin{shl}
avidAyxviSasaMdaSaTxM tajAjxcnXnaM paramAtamxnaH | \\
pAraMpayeVRNa budAdhxyXdw sUthxliVBAvaM nigacaCxti \hfill||  1092 ||  
\end{shl}

\begin{artha}
avideyxyeMba doVSadiMda aMTigoMDiruva paramAtamxna AjAcnxnavu (ceYtanayxvu) \footnote[1]{A Atamx ceYtanayxvu avidAyx doVSadiMda kUDi budidhx modalAdavugaLa mUlaka shabAdxdi viSayagaLige adhiVnavAguvudu, ideV sUthxliV BAvavanunx hoMduvudeMbudakekx athaR.}paraMpareyiMda budidhx muMtAda sathxLadalilx sUthxla rUpavanunx hoMduvudu.
\end{artha}

\vishaya{Iga avideyxyiMda huTiTxda mithAyxjAcnxnada savxrUpa-}

\footnotetext[2]{I avideyxyu budidhx iMdirxya itAyxdigaLa mUlaka shabadx modalAda rUpavanunx hoMdi ceYtanayxkekx GaTAdi viSayagaLanunx parxkAshagoLisalu anukUlisuvudu. A viSaya parxkAshavu mithAyxjAcnxnaveMdathaR ``ukatxMhi sa vijAcnxna visheVSoV budAdhxyXdi paraMparayA'' ``sUthxliV BUteV bAhayxH parxkAshoV vayxvahArAya kalapxteV'' eMdu BataqRparxpaMca ukitxyanunx (AnaM-TiVke)yalilx udAhariside, I riVtiyAgi avideyx matutx ceYtanayx eraDU parasapxra seVrikoMDu AtamxniMda horage nAnAmuKavAgi horaTu baruvudu.}
\begin{shl}
\footnotemark[2]sUthxliVBUtA bahiH seVyaM parxkAshatAvxya kalapxteV | \\
EvaM bahiravideyxVyaM niSAkxrXmatAyxtamxnaH paqthakf \hfill||  1093 ||  
\end{shl}

%%%%%shloka footnote[2]
\begin{artha}
avideyxya savxBAva vaNaRneya upasaMhAra. I avideyxyu (budidhx modalAdavugaLa mUlaka) sUthxlavAgidadxdudx ceYtanayxkekx viSayavanunx parxkAshapaDisuvudakekx anukUlisuvudu, IriVtiyAgi avideyxyu horage (nAnA muKavAgi) AtamxniMda horaTu baruvudu.
\end{artha}

\begin{shl}
savxpenxV viveVkoV vAyxKAyxtaH kamaRNoV\s sAyx\s \s tamxnoV\s dhunA | \\
avidAyxparxviveVkoV\s yaM vakaSxyXteV\s sAyxta utatxramf \hfill||  1094 ||  
\end{shl}

\begin{artha}
savxpanxdalilx kamaRda viveVkavanunx I Atamxnige vaNiRsidAdxyitu. Iga avideyxya viveVkavanunx Itanige muMde heVLuvudu, adariMda muMdina garxMthavu\footnote{``ya AsaknogxVkaqtaH saMsagaRH teVna viveVkoV vAyxKAyxtaH'' ``yasutx avidAyxkaqtaH saMsagaRH teVna viveVkoV vAyxKAyxtavayxH'' eMdu (anaM-TiVke) (BataqR....BASayx)} baMdiruvudu yukatx.
\end{artha}

\vishaya{kamaRviveVkaveV sAku avidAyx viveVkaveVtakekx?}

\begin{shl}
kamaRNA nimiRtaM loVkamAtAmx pashayxtayxvidayxyA | \\
tayoVviRveVkAduBayoVH sAvxBAvayxM parxtipadayxteV \hfill||  1095 ||  
\end{shl}

\begin{artha}
kamaRdiMda nimiRsalapxTaTx loVkavanunx Atamxnu avideyxyiMda kANuvanu, A kamaR matutx avideyx eraDanunx viveVcisuvudariMda Atamxnu savxrUpa sithxtiyanunx paDeyuvanu\footnote{eraDanunx viveVka mADikoMDAgaleV Atamxnige mukitx ilalxvAdare keVvala kamaR viveVkadiMda AgalAradu. avideyxyaviveVkavanunx mADikoLaLxdidadxre adariMda anathaRveV udayisuvudeMdu tAtapxyaR.}.
\end{artha}

\vishaya{hAgAdare eraDara viveVcaneyeV sAku, kAma viveVcane Etakekx? eMdare-}

\begin{shl}
rUpAdiviSayAsaknagx karaNeYshacxkuSxrAdiBiH | \\
anuparxvishayx BoVkAtxraM racnajxyitAvx\s vatiSaThxteV \hfill||  1096 ||  
\end{shl}

\begin{artha}
rUpAdi viSayagaLa Asakitxyu (kAmavu) cakuSx modalAda iMdirxyagaLa mUlaka BoVkatxnAda Atamxnanunx parxveVshisi raMjanegoLisi\footnote{kAma viveVkavU Avashayxka. iMdirxyagaLa mUlakavAgi viSaya kAmaneyu huTiTx BoVgapaDuva Atamxnanunx raMjisuvudu, anurakatxnanAnxgi mADuvudu, karxmeVNa adu anathaRvanunxMTu mADuvudu, adariMda kAma viveVkavU beku.} nilulxvudu.
\end{artha}

\begin{shl}
vijAcnxnaM pwruSaM shudadhxM teVnA\s \s saknegxVna dUSitamf | \\
tameVvA\s \s saqtayx nigaRmayx bAhayxtoV vayxvatiSaThxteV \hfill||  1097 ||  
\end{shl}

\begin{artha}
puruSana vijAcnxnavu shudadhxvAgidudx AkAmadiMda keTuTx adeV (viSayAdhiVnavAda) kAmavaneVnx anusarisi AtamxniMda horage baMdu nilulxvudu.
\end{artha}

\vishaya{avideyxyiMda viSayAdhiVnavAgideyeMdu heVLidadxnunx biTuTx kAmadiMda viSayAdhiVnaveMdu heVLidedxVke?-aMdare}

\begin{shl}
AsaknAgxvidayxyoVreVvamanoyxVnAyxsharxyatoVditA | \\
aBAvAdanayoVrAtAmx sAvxtamxsathxH saMparxsiVdati \hfill||  1098 ||  
\end{shl}

\begin{artha}
kAma\footnote{avideyxya mUlaka kAmavu Atamxnalilx saMbaMdhisuvudu aMdare ajAcnxniyAdAga kAmiyAguvaneMdathaR, kAmavidadxvanige kAmavu anathaRvanunxMTu mADuvudu, hiVge averaDU parasapxra guNa pArxdhAnayxgaLiMda aMgAMgiBAvadiMda oTuTx seVruvudariMda pUvARparaviroVdhaveVnU ilalx,} matutx avideyxgaLige I riVtiyAgi parasapxra avalaMbaneyideyeMdu heVLidAdxgide: I eraDU ilalxvAdadxriMda Atamxnu savxpanx rUpadalilx niMtu suparxsananxnAguvanu, (shAMtanAguvanu).
\end{artha}

\begin{shl}
parxsidadhxmeVtalolxVkeV\s pi yadi roVgAdisaMgatiH \\
na Bavatayxtha jalapxnitx savxsothxV\s yamiti loVkikAH \hfill ||  1099 || 
\end{shl}

\begin{artha}
idu loVkadalUlx parxsidadhxvAgiruvudu adeVneMdare roVgAdigaLa saMbaMdhavu yAvAga iruvudilalxvo kUDale lwkikaru Itanu savxsathxnAgidAdxneMdu heVLuvaru.
\end{artha}

\vishaya{kAma. avideyx ivugaLa viveVkadiMda Aguva Pala-}

\begin{shl}
nirAsaknagxsayx viduSasatxsAmxnumxkitxM parxtiVmaheV | \\
\footnotemark[2]sAyxdanayxtaraveYkaleyxV na veVteyxVtadanishicxtamf \hfill||  1100 ||  
\end{shl}
\footnotemark[2]{oMdu biTuTx hoVdarU mukitxyu nishicxtavalalxveMdu tiLidiruvAga eraDanUnx viveVcisilalxvAdare mukitxyu nishicxtavalalxveMdu parxteyxVka heVLabeVkAgilalx.}

%%%%%shloka footnote[2]
\begin{artha}
AdadxriMda kAmavilalxda jAcnxnige mukitxyanunx nAvu \footnote[3]{tiLiyutetxVve eMdare veVda matutx upaniSatutxgaLa adhayxyanadiMda tiLiyutetxVveMdathaR.}tiLiyutetxVve. kAmakamaRgaLalilx yAvudu oMdilalxvAdarU mukitxyAguvudeMbudu nishicxtavalalx.
\end{artha}

\vishaya{muMdina garxMthadalilx avideyxya viveVkaveV kANuvudilalxvalalx? eMdare}

\begin{shl}
tadayxthA sheyxVna itayxsAmxdAyxvatisxtXrXVbArxhamxNAditi | \\
avidAyxparxviveVkAthaRmeVvameVtAvaducayxteV \hfill||  1101 ||  
\end{shl}

\begin{artha}
`tadayxthAsimxnf nAnxkAsheV sheyxVnoV vA supaNoVR vA' itAyxdi vAkayxdiMda AraMBisi sitxrXVbArxhamxNa meYterxVyiV bArxhamxNa Kara payaRMtaviruva vAkayxvu iSuTx avideyxya viveVkakAkxgi parxtayxgAtamxnanunx avideyxyiMda beVpaRDisuvadakAkxgi heVLalapxDuvudu.
\end{artha}

\vishaya{sheyxVna vAkayxdalilx heVge avidAyx viveVkavu moVkaSxkAkxgi ukatxvAgide? eMdare-}

\begin{shl}
savxpanxjAgarxtayxcAreV\s simxnAnxtAmx sheyxVnaH pataninxva | \\
parishArxnatxH suSupAtxKayxM niVDaM dhAvatayxthA\s \s tamxnaH \hfill||  1102 ||  
\end{shl}

\begin{artha}
AkAshadalilx hArADutAtx idudx baLalida giDagavu vishArxMtigAgi tananx gUDige heVge ODi baruvudoV hAgeye Atamxnu savxpanx jAgarAvasethxgaLa saMcArada nimitatxvAgi baLali (vishArxMtigAgi) \footnote{ajAcnxtavAda parabarxhamxveV suSupatx eMbudara athaR. ideV jiVvAtAmxdige vishArxMti sAthxna. tAnu yAva maneyiMda horaTu horage savxpanx jAgaragaLalilx baMdidudx saMcarisidanoV adeV manege baMdu seVra beVkAdadudx. pUNaR vishArxMtiyu doreyuvudu barxhamxnalelxV horatu beVre sathxLadalilx alalx. kAyaRvu kAraNadalilx layavAguvudu nAyxya. kAraNabarxhamxdalilx kAyaRvAda jiVvavu layavAguvudu sari. adakAkxgi biVja BUtavAda ajAcnxnavidudx adU avayxkatxvAgidAdxga adara upAdhiyiMda kUDida barxhamx ajAcnxtabarxhamx ideV suSupitx sAthxna.} suSupitxyeMbuva tananx gUDige baMdu seVruvanu.
\end{artha}

\vishaya{savxpanx jAgara dashegaLanunx Atamxnu biTuTx sharxmaparihArakAkxgi suSupitxge hoVguvaneMdeV toVruvudu avidAyx viveVka toVrilalx? eMdare?}

\begin{shl}
kamARvidAyxvinimuRkatx EtasimxnenxVva lakaSxyXteV | \\
savaRshoVkAtigaH savxsathxH sitxmitaH sAvxtamxni sithxtaH\hfill ||  1103 ||  
\end{shl}

%%%%%shloka footnote[2]
\begin{artha}
\footnote{ililx shoVkashabadxdiMda kAmavu garxhisalapxDuvudu `tiVNoVRhitadA savARnf shoVkAnf haqdayasayx Bavati' eMbudara athaRvanunx `savaRshoVkAtigaH' eMdu vAtiRkadalilx anuvAdamADide. vasutxtaH shoVkashabadxvu duHKaveMba athaRvaneVnx muKayxvAgi koDuvudu AdarU ililx iSaTxviSayadalilx huTuTxva kAmaneyaneVnx garxhisabeVku ekeMdare? ``ananAvxgataM puNeyxVna ananAvxgataM pApeVna'' eMbudAgi pApa puNayxgaLiMda Atamxnu saMbaMdhisade aMTikoLaLxde iruvaneMdu hiMde heVLidadxkekx kAraNavanunx `tiVNoVRhitadA savARshoVkAnf' eMdu heVLide I shabadxdiMda kAraNavanunx sUciside, yathAshurxtavAgi shoVkavanunx biTiTxdAdxneMdu heVLidare pApa puNayxgaLu aMTilalxde iruvaneMbudakekx idu sAdhakavAguvudilalx. iSaTxviSayavAda dhanakanakAdigaLalilx huTuTxva apeVkeVSxyeV kAmaveMdU adu iSaTx vasutxgaLu keY biTuTx hoVdAga shoVkavAgi pariNamisuvudu. adariMda shoVkakekx mUlavAda kAmavaneVnx shoVka eMdu gwNavAgi `AyuGaqRtaM' eMdu kareyuvaMte karedide, loVkadalilx iSaTxvAdadudx baradeV idAdxgalU baMdu seVri keY tapipx hoVdAgalU adariMda puruSanu duHKa paDuvanu adariMda shoVka, rati, kAma I mUrU payARya shabadxgaLu `na kaMcana kAmaM kAmayateV\s ticadhxnAdxH' eMba parxkaraNadalilxruvudariMda shoVka shabadxkekx kAmaveMdu athaR mADabeVku. kAmaveV kamaRkekxkAraNa kAmavilalxvanAdare yAva kamaRvanUnx mADuvudilalx, Aga pApa puNayxgaLanunx mADade iruvaneMba athaRvu sari hoVguvudu.}
I suSupitxyalelxV kamaR matutx avideyxgaLiMda biDalapxTuTx savaRshoVkagaLanunx (kAmagaLanUnx) miVri savxsathxvAgi nishacxlavAgi savxsavxrUpadalilxruvanu.
\end{artha}

\section*{idara BASayxveV idakekx AdhAra (baq-4-3-22)}

\vishaya{paramatoVpa saMhAra}

\begin{shl}
iti vAyxcakaSxteV keVcinamxhAmasAtxyXdikAM shurxtimf | \\
tananxyAyayxmathavA\s nAyxyayxM yatAnxnAnxyXyeYH pariVkaSxyXtAmf \hfill ||  1104 ||  
\end{shl}

\begin{artha}
hiVgeMdu kelavaru mahAmatAsxyXdi shurxti vAkayxvanunx vAyxKAyxna mADutAtxre adu nAyxya yukatxvo? athavA nAyxyayukatxvalalxvo? eMbuvanunx nAyxyagaLiMda yatanx pUvaRka pariVkiSxsa beVkAdadu.
\end{artha}

\vishaya{hiMde `bAhayxM kamARsayx' itAyxdiyAgi heVLidadxralilx vikalapxdiMda parxshinxsi nirAkarisuvudu.}

\begin{shl}
AtamxvasatxvXtireVkeVNa nAsitx vasatxvXnatxraM yadi | \\
bAhAyxnatxHparxviBAgoV\s yaM kimAshirxtayx parxkalapxyXteV \hfill||  1105 ||  
\end{shl}

\begin{artha}
Atamxvasutxvige beVreyAgi vasutx ilalxveMdAdare bAhayx aMtaraveMba viBAgavu yAvudanunx avalaMbisi mADalapxTiTxde?
\end{artha}

\vishaya{Atamxnige beVreyAgi vasutx iruvudAdare adu vAsatxvave? alalxve? vAsatxva eMdare-}

\begin{shl}
viSayeVnidxrXyAdi yadavxsutx nAvidAyxvAyxtireVki tatf | \\
vasatxvXnatxrasayx sadABxva aikAtamxyXM bAdhayxteV yataH \hfill||  1106 ||  
\end{shl}

\begin{artha}
viSaya-iMdirxya modalAda vasutx yAvuduMTo adu ajAcnxnakekx beVreyAgilalx, EkeMdare! beVre vasutxvAgidadxre EkAtamxsavxrUpavu \footnote{beVre vasutx bAdhitaveV eMdu heVLuvalilx aneVka shurxtigaLu samxtigaLU loVka nAyxya ivu viroVdhisuvavu, adariMda Atamxnige BinanxvAda vasutx vAsatxvavAgilalx, vAsatxvavalalxveMdu heVLidare hiMde heVLida viBAga kalapxneyu pArxmANikavAgudilalx, ajAcnxna mUlakavAgi huTiTxda ajAcnxna kAyaRvelalxvU AtamxnalelxV layavAgi biDuveneMdu tAtapxyaR.}bAdhitavAguvudu (suLALxguvudu).
\end{artha}

\section*{(baq. 4-3-18)} vAtiRka 1206

