\begin{center}
|| shirxV dakiSxNAmUtaRyeVnamaH ||
\end{center}

\section*{baq. adhAyxya 3 -- bArxhamxNa 5}

\begin{artha}%%% 166
modalu mUru bArxhamxNagaLiMda kamaRveMba parxyoVjakadoMdige saMsAra baMdhanavanunx elalxvanunx heVLidAdxyitu matutx saMsAriyu (vasutxtaH deVhAdigaLige beVreyAgi idAdxneMdu) asitxtavxvanunx BinanxteyanUnx nAlakxne bArxhamxNadiMda niNaRya mADide.
\end{artha}

\vishaya{muMdina bArxhamxNa tAtapxyaRveVnu? eMdare-}

\begin{artha}
saMsArada anathaRvu uMTAguvudakekx kAraNavAdadudx nAshavAgalu Iga saMnAyxsavoMdeV sahAyavAgiruva Atamxtatavx jAcnxnavanunx heVLuvudu||
\end{artha}

\vishaya{jAcnxnavanenxV heVLuvadanunx biTuTx sAMnAyxsavanunx Eke? heVLuvudu? eMdare-}

\begin{artha}
tatavx jAcnxnavu huTiTxdavanige saMnAyxsavu lakaSxNavAgeyAdadxriMda AjAcnxnavu huTuTxvudakUkx sAdhanavAdadudx saMnAyxsave eMdu ililx heVLuvudu||
\end{artha}

\vishaya{saMnAyxsavu jAcnxna sAdhanaveMbudakekx parxmANa-}

\begin{artha}%%% 167
\textbf{BataqRparxpaMcaru aBipArxyapaDuva pUvARpara saMbaMdhavu pUvaR pakaSx}
\end{artha}

\begin{artha}
I uSasatx parxshenxyanenx kahoVLanU keVLuvanu, oMdAvatiR niNaRya mADida adeV parxshenxyanenxV punaH keVLuvudu EtakAkxgi?
\end{artha}

\begin{artha}
parxshenx keVLuvavaru beVreyAdadadxriMda punaH parxshinxsuvudu doVSavalalxveMdare hAgalalx, utatxra vacanavu beVreyAdagiruvudariMda punarukitxdoVSavanunx pArxshinxkaraBeVdavu pariharisalAradu.

oMdeV viSayadalilx horaTa parxshanxvAkayxkekx oMdeV utatxra vAkayxviruvudu saMmataveMbudu parxsidadhxvAgeide. 
\end{artha}

\begin{artha}
punarukitxyiruvudu oMdu shAKeyalilx elilxyU nAyxyavalalx. I shurxtiyalilx loVkasidadhxvAda (mAtu) athavA jalapxvu iruvudilalx. hAgidadxre pArxshinxka BeVdavu irabahudAgididxtu||
\end{artha}

\vishaya{sidAdhMta}

\begin{artha}
adariMda ivu samAna shabadxvuLaLxvugaLAgiyU BinAnxthaRvanunx heVLuvadAgiyU iruva parxshenxgaLu kahoVLa matutx uSasatxna bAyiyiMda horaTaveNdu niNaRyavu.
\end{artha}%%% 168

\vishaya{hAgAdare avugaLalilx BinAnxthaRgaLu yAvavu? eMdare}

\begin{artha}
uSasatxnu jiVvAtamxnanunx parxshinxsidanu, parama pada (paramAtamx)veMbudanu parxshinxsalilalx, kahoVlanU kUDa paramAtamxnanunx parxshinxsidanu, saMsAri jjiVvananunx parxshinxsalilalx||
\end{artha}

\vishaya{jiVva paramAtamxrige Ekatavxvu iSATxvAgidadxre parxshenxgaLige BinAnxthaRgaLu heVge? saMgata? eMdare-}

\begin{artha}
samAnadeVshadalilx eraDanUnx heVLiruvudariMda EkAthaRveMbudAgi niNaRyisuvudu, Adare parxshenx vAkayxgaLalilx parasapxra saMkiVNaRvAda guNagaLanunx heVLidAdxga hoMdibaruvaMte vayxvasethxyU iruvudu||
\end{artha}

\begin{artha}%%% 169
\footnote[1]{`\stext ' yadeVva sAkASxda paroVkASxdf \stext eMdu heVLida savARMtaratavxvu kahoVla vAkayxdalilx paramAtamxdhamaRvAgide. AdarU jijVvadhaRvAdadadhxriMda - hiMdina aSatxsatx parxshenxyalilxdudxdAgi kaMDu baMdiruvudariMda jiVvAtamxnige seVruvudu. oMdeV parxkaraNadalilx eraDukUkx savARMtaratavxvanunx heVLidadxriMda paramAtamxnalilxruvaMte heVLidudx jiVvAtamxnalUlx sidadhxveV Aguvudu||} yAvudu (kahoVla vAkayxdalilx) paramAtamxna dhamaRveMdu savARMtaratama rUpavanunx hiMde heVLididxto adu Atamxnalilx heVLidudx adara EkadeVshavanunx heVLidadxriMda jiVvAtamxnalilx seVruvudu
\end{artha}

\begin{artha}
\footnote[2]{paramAtamx dhamaRdalilx toVrisida karxmavanunx jiVvadhamaRdalUlx toVrisuvaru. yadayxpi shoVkamoVhAdi shUnayxte, savxparxjAshatavx ivugaLu joyxVtibArxRhamxNadalilx jiVvadhamaRvAgi ukatxvAgive. AdarU I sathxLadalilx heVLidudx paramAtamxnige seVruvudu. paramAtamxna savxBAvavavAgiye avu iruvudariMda avanige saMbaMdhisive. I eraDu dhamaRgaLu jiVvAtamxnalUlx ideyeMdu heVLutitxruvAga paramAtamxnigU avu sidadhxvAguvavu eMdu BataqR parxpaMcara aBipArxyavu I 11.12ne vAtiRkagaLu sUkaSxmxvAgi vayxkatxpaTiTxde. (AnaM-TiVkeyalilx aBipArxyavanunx tiLisalu I riVti anuvAda mADiudx kANutatxde- 11neV TiVkA `\stext '|| 12neV vAtiRkadalilx ``\stext || eMdu (AnaM-TiVkA)} shoVka moVhAdigaLanunx miVriruvudU matutx parxkAsha savxrUpavU jiVvAtamxnalilx iveyeMdu heVLidadxrU paramAtamxnalilx seVruvudu.
\end{artha}

\vishaya{paMcamAdhAyxya sheVSavu kahoVla parxkaraNadoDane EkavAkayxvAgide- eMdu heVLutAtxre}%%% 170

\begin{artha}
I aidaneV adhAyxyadalilx uLida BAgadiMda paramAtamxniNaRyavAguvudu Arane adhAyxyadalilx jiVvAtamx niNaRyavu niNARvAguvudu||
\end{artha}

\begin{artha}
jiVvAtamx paramAtamx iveraDara udAharaNiyu nAyxyamAgaRda avalaMbaneyiMda punarukitxyilalxdaMte tiLidu baMdide
\end{artha}

\vishaya{paramatada upasaMhAra idara nirAkaraNe}

\begin{artha}
hiVgeMdu kelavaru (BataqR parxpaMcaru) I hiMde heVLida I parxshenxgaLanunx vAyxKAyxnisidAdxre-

Adare I vAyxKAyxnavu sariyAgilalxveMbudanunx heVge heVLabeVko hAge I muMde heVLuvudu-
\end{artha}

\begin{artha}
I eraDu parxshenxgaLigU `\stext ' eMbudAgi utatxra vAkayxviruvudariMda jiVvaparamAtamxru ililx AtamxgaLeMbudu vivakiSxtavAgilalx|| 

matutx utatxra vAkayxvu oMdeV rUpavAgidunx savxrasavAgi anuBavakekx baMdiruvudu `\stext ' eMbudu ililx utatxravAkayx, adariMda vAkayxgaLu BinAnxthaRvuLaLxdedxMbudu heVge?
\end{artha}%%% 171

\begin{artha}
matutx oMdeV shariVradalilx ibabxru Atamxru (iruvudu saMBAvitavalalx) saMBavisuvudilalx. AtamxjAcnxnavU saha oMdeV viSayadAdxgide, hAgU AtamxnobabxneyeMbudakekx shurxtivAkayxgaLU ive||
\end{artha}

\begin{artha}
yAva itara shariVragaLalilx veYmatayxvu iruvudo, A elAlx shariVragaLU parxtivAdiya shariVradaMte shariVratavx dhamaRvu samAnavAgivuLaLxdadxriMda oMdeV AtamxvuLaLxvugaLeMdu takiRsabahudu||
\end{artha}

\begin{artha}
sAkASxtf itAyxdiyAgi savARMtaraveMbuva payaRMtara viruva visheVSaNagaLige oMdeV mubAyxthaRvAgiruvariMda parxshanxvAkayxdalilx eraDu AtamxgaLu tAtapxyaRvAgiruvudilalx||
\end{artha}

\begin{artha}%%% 172
gwNAthaRvilalxde muKAyxthaRvA avayxvadhAnAdi visheVSaNagaLige aneVka Atamxru visheVSayxvAgiveyeMbudu nAyxyavU AgalAradu EkeMdare adu asaMBavaveMba kAraNadiMda. nAyxyavalalx||
\end{artha}

\begin{artha}
oMdu veVLe AjiVva paramAtamxrige BeVdaveV iruvudAdare upaniSatatxnunx AraMBisuvudeV vayxthaRvAguvudu matutx moVkASxvU bArade hoVguvudu||  
\end{artha}

\begin{artha}
(saqSiTxmADi) parxveVshisidavanAda jiVvAtamxnige vAyxpakatevxvu iSaTxvAgilalxvAdare apUNaR jAcnxna viSayadalilx niSeVdha mADiruvu (aMdare BeVda daqSiTxyanunx niMdisiruvudu) yukatxvAguvudilalx||
\end{artha}

\vishaya{AtamxBeVdavu vAsatxvavAgilalxdaMte anAtamxBeVvU vAsatxvavalalx-}

\begin{artha}
AdariMda vasutx savxBAvada anusAravAgi (paramAthaRvAgi) elilxyU BeVdaviruvudilalx, vasutx savxrUpavanunx tiLiyadiruvadariMda savxpanx matutx iMdarx jAladaMte BeVdavu toVruvudu (asatayx)||
\end{artha}

\begin{artha}
adariMda oMdeV parxshenx. eraDu parxshenxgaLu saMmatabalalx. AdarU yAvudoV oMdu visheVSavanunx iTuTxkoMDu keVLida parxshenxyanenxV punaH keVLide aSeTx||
\end{artha}%%% 173

\begin{artha}
hiMde parxshanxvAkayx utatxra vAkayxgaLalilx upadeVshada viSayavu nidhaRrisalapxTiTxde. Iga adara nijasavxrUpanunx ililx parxshinxside||
\end{artha}

\vishaya{modalane parxshenxya viSayavanunx sapxSaTxpaDisuvudu-}

\begin{artha}
shariVradiMda AraMBisi budidhx payaRMtaraviruva vasutxviniMta beVreyAgi sAkiSxyu parxtayxkaSxvAgi parxtayxgAtamxneMdu tiLiyabeVkAdududx. ideV modalina parxshAnxthaRda saMkeSxVpavu||
\end{artha}

\begin{artha}
hiMde heVLida Atamxvasutxvina viSayadalelx. hasivu muMtAda (sakalasaMsAravaMnenx atikarxmisuva savxrUpavanunx eraDane parxshenxyiMda tatavxvAgi heVLuvudu, EtakekxMdare! parxtayxgAtamxna ajAcnxnavanunx nivaqtitxgoLisuvudakekx||
\end{artha}

\begin{artha}
athavA parxshenxyidadxMte hiMde utatxravanunx sAriyAgi heVLiralilalxveMdu tiLidu punaH muniyananx kuritu keVLida viSayavanenx keVLiruvaneMdu heVLabahudu.
\end{artha}

\begin{artha}%%% 174
`sAkASxtf ' itAyxdiyAgi sAkASxtf itAyxdi lakaSxNavanunx parxshinxsidudx heVge saMBavisudo hAge utatxra vacanadalilx  savxlapxvU kANuvudilalx||
\end{artha}

\begin{artha}
vidAvxMsanAda kahoVlanu meVle heVLida asaMtuSitxYiMda uSasatx parxshenxyanenx punaH tatAvxthaR niNaRyakAkxgi muniyanunx keVLidanu||
\end{artha}

\vishaya{asaMtuSiTxge kArANaveVnu?}

\begin{artha}
`ESaAtAmx' eMba utatxravU yaHpArxNeVna... eMbuva matotxMdu utatxravU ide, Adare aSuTx mAtarx heVLidadxriMda I parxshAnxthaRvu niNaRyavAgalAradu||
\end{artha}

\vishaya{oMdeV Atamxtatavxdalilx eraDu parxshenxgaLeMbudara meVle AkeSxVpa-}

\begin{artha}
adivxtiVyAnAda Atamxnalilx oMdeV kAladalilx\footnote[1]{jamanx maraNa, jarA, (mupupx) vAyxdhi, kuSxtf, puvAse, ivugaru AruteregaLu}  Aru teregaLanunx dATiruva rUpavU duHKasaMbaMdhavU iralAravu, heVge beLakU katatxle ivugaLu oMdeV kaDe oMdeV kAladalilx iruvudilalxvo. hAgeye alalxve? samAdhAna-
\end{artha}

\begin{artha}%%% 175
ideVnu doVSavalalx, Atamxtatavxda ajAcnxna matutx vidhAyxjAcnxnaveMba BarxmadeMba nimitatxdiMda Atamxnige AkAshadalilx kalipxsida kapipxnaMte saMsAra saMbaMdavu kalipxtavAgide||
\end{artha}

\begin{artha}
hiMdeVyeV doDaDx vicAra mADidAga I viSayavanunx vicAra. mADidAdxgide. hAgU virudadhxvAda shurxtigaLa viSayavanunx viveVcane mADuva parxsaMgadalUlx (madhu bArxhamxNada koneyalilx) vicAra mADiye ide||
\end{artha}

\begin{artha}
oMdu hagagxdalilx rajujxtavx- matutx sapaR rUpagaLeraDU yAva riVtiyalilx iruvavo. AtamxnalUlx hAgeye tananxdeV Ada oMdu rUpavu adara ajAcnxnadiMda matotxMdu rUpavU saMbavisabahudu||

parxkaqta mutakx rUpavU badadhx rUpavU parasapxra virudadhxvAda dhamaRgaLAdarU oMDeV kaDe oMdeV kAladalilx saMBavisabahudu, Adare niVlaguNavu utapxlatavx jAtiyA nijavAgi kamaldalilx militavAguvaMte ivu militavAgudalalx||

I riVtiyAgi virudadhx dhamaRgaLAdarU yAvudoMdu doVSavU ilalx punaH parxshenx-

I adevxYtAtamxnalilx nAmArUpAdigaLu iruvudu doVSaveV alalxve? eMdu punaH parxshinxside. adariMda Atamxnu sadivxtiVyanAguvalilx-
\end{artha}%%% 176

\begin{artha}
`neVha nAnAsitx kiMcana' veMbavacanadiMdalU EkameVvAdivxtiVyaM bahamx eMba vacanadiMdalU viroVdhavu baruvudu.
\end{artha}

\begin{artha}
hiVgalalx, adanunx maqtitxna daqSATxMta muMtAda yukitxgaLiMda parihAra mADide yAdadxriMda viroVdhavilalx||
hiVgalalx, adanunx maqtitxna daqSATxMta muMtAda yukitxgaLiMda parihAra mADide yAdadxriMda viroVdhavilalx||
\end{artha}

\begin{artha}
satatxtavxvanunx garxhisiduruvAga adu vipariVtavAgiruvaMte kANutatxde.  vasatx idadxMte garxhisuvAga (kANuvAga) idadx hAgeye kANuvudu||
\end{artha}

\begin{artha}
I viSayadalilx parxshenxyuMTAgi heVLabeVkAda utatxravanunx hiMdeye aneVka sala bahaLa cenAnxgi heVLidAdxgide. adelalxvanunx ililx adaravuLaLxvaru samxrisabeVku||
\end{artha}

\vishaya{adeVnu samAdhAna mADide? eMdare-}

\begin{artha}%%% 177
tatavxvicAra mADadiruvAgale sidadhxvAda ajAcnxnadiMda hALAda citatxvuLaLxvarige adivxtiVyavAdarU I parabarxhamxvu divxtiVya vasutxvuLaLxdaMte asatayxvAgi toVruvudu||

heVge nimaRlavAda AkAshavanunx timiraveMba kaNiNxna doVSadiMda piVDitarAda janaru\footnote[1]{timiraveMba neVtarx doVSadiMda mAnavanige kadAcatf AkAshadalilx keMpubiLipu, niVli, haLadi, itAyxdi baNaNxgaLiMda kUDiruvaMte kANuvudu} citarx baNaNxda kalegaLiMda saMkiVNaRvAgiruvaMte kANuvaro, hAgeye I nimaRlavAda niviRkAravAda parabarxhamxvanunx ajAcnxnadiMda kalamxSadidanunx hoMdidAMtAgi BeVdarUpavuLaLxdAgi parxkAshisuvudu||
\end{artha}

\begin{artha}
A parabarxhamxda savxrUpavAda ceYtanayxvu oMdeV agidadxrU aneVka bageyAgi beVpaRDutatxde. loVkAriSaTxkAladalilx samudarxda niVru beMkiya keMDagaLu kUDiruvaMte kANisuvudu heVge hAge.
\end{artha}

\vishaya{jAcnxnavu oMdarU anAdiyAgi rUDha mUlavAda ajAcnxnavu hAgeye irabahudaSeTx? eMdare}%%% 178

\begin{artha}
Atamxna ajAcnxnavu bAdhitavAguvudu. Adare idu yAvudakUkx bAdhaka (nAshaka)vAguvudilalx. hAgU parxtayxgAtamxna tatavxjAcnxnavu eMdigU bAdhitavAguvudalalx, adu matotxMdakekx bAdhaka (nAshaSa)vAguvudu||
\end{artha}

\begin{artha}
ajAcnxna matutx jAcnxnagaLu tananx savxBAvavanunx biDadaye seVruvavu hiVgeMdu manasisxnalilx iTuTxkoMDu samasatx yoVgeVshavxrarigU IshavxranAda BagavaMtanu ajAcnxnavanunx hoVgalADisalu tananxnunx sharaNa hoMdida ajuRnanigAgi avugaLa bAdhayxbAdhaka rUpavu cuyxtihoMduvudilalxveMbudanunx parxkaTa paDisutAtx I riVti upadeVshisiruvanu -(BagavadigxVte 2neV adhAyxya)
\end{artha}

%%% shloka footnote
\begin{artha}
\footnote[1]{ajacnxrAda elAlx janarigU Atamxtatavxvu katatxleyaMte ajAcnxtavAgiruvu idu rAtirx adu jAcnxnige ecacxrinaMte tiLideyiruvudariMda hagalu. namage hagalAdadudx gUbegaLige rAtirxyeMbuvaMte, pArxNigaLu, yAva avideyxyaMba rAtirxyalilx malagidadxrU ecacxrAgiruvavaraMte vayxvarisuvaro. avarige adu hagalinaMte Agide, adare I hagalu tatavxvanunx noVDuva yoVgige rAtirxyeV. aMdare ajAcnxnigaLige kAlavaMte kANuvudilalxveMdathaR, aMdare jAcnxnavidadxre ajAcnxna viruvudilalx, nAshavAguvudu ajAcnxnavidadxre jAcnxnavu udayisuvudilalx ilalx.} samasatx pArxNigaLigU yAvudu rAtirxyoV, adeV rAtirxyalilx jAcnxniyu ecacxravAgirutAtxneV (aMdare itanige adu hagalu) yAva (kAladalilx) rAtirxyalilx pArxNigaLu ecacxravAgirutatxdeyoV adu (Ahagalu) jAcnxniyAda musige rAtirxyAguvudu||
\end{artha}

\begin{artha}%%% 179
jAcnxna matutx ajAcnxnagaLanunx iTuTxkoMDu Atamxnalilx shAsitxrXVyavAda matutx lwkikxvAda vayxvahAragaLu huTuTxvavu. adariMda idaralilx virodhavanunx shaMkisuvahAgilalx||
\end{artha}

\vishaya{jAcnxniya vayxvahAravu ajAcnxnadiMda heVge Aguvudu?}

\begin{artha}
AtamxjAcnxnavu huTiTxdadxvarige Atamxna AjAcnxna adariMda huTiTxdudx tatavxjAcnxnaveMba beMkiyiMda nitayxvu suDalapxDutitxdadxrU huTuTxvadu (aMdare bAdhitavAgidadxrU saMsAkxra rUpadalilx anuvatiRsuvudu) 
\end{artha}

\begin{artha}
EkAtamx vasutxvina nija savxrUpavanunx tAtapxyaRvAgiTuTx idara muMdina garxMthavu baMdiruvudu. 
\end{artha}

\vishaya{parxshenxyAvudeMdare akahoVla parxshenxya vAyxKAyxnAraMBavu yadeVva itAyxdiyAgi AraMBisi sAkASxtf itAyxdiyAgi mADiruva vAkayx ``\stext " eMbudeV parxshenx}

\begin{artha}
yAjacnxvalakxyXrU saha ``\stext " eMdu hiMdinaMte utatxrisidAdxre. yAvakAraNadiMda ililxyeV (I AtamxnalelxV) sAkASxtf itAyxdi lakaSxNavu saMBavisuvado. adariMda||
\end{artha}

\stext

\begin{artha}%%% 180
samasatx niNaRyada utatxravanunx bayasida kahoVlanU saha hiMdinaMte ``\stext" eMdu keVLidanu. \stext itAyxdiyAgi shudadhx barxhamxdalilx budidhxyiTaTxvArAgi avananunx kuritu muniyu utatxrisiruvanu||
\end{artha}

\vishaya{A maMtarxda athaR saMkeSxVpa-}

\begin{artha}
yAvudu hasivu bAyArike modalAda elAlx saMsAra rUpagaLanunx tananx savxBAvakekx viroVdhavAgidadxriMda nija savxrUpadalelx miVridudx savxmahimeyalilx niMtiruvudo. adeV nijavAda Atamxtatavxvu||
\end{artha}

\vishaya{inunx muMde maMtarxda parxtipada vAyxKAna mADuvudu-}

\begin{artha}
yAva Atamxnu hasivu bAyArikegaLanunx adara kAraNavu ilalxvAdadxriMda yAvAgalU nijasavxBAvadalilx miVriruvano avaneV parxtayxkaSxvAda parabarxhamxveMdu tiLiyabeVku||
\end{artha}

\begin{artha}
viSayAsakitxge biVjavAda shoVkadiMda\footnote[1]{ililx shoVkaveMdare shoVkakUkx mUlavAda kAmaveMdathaR, gwNAthaR, kAyaR padavanunx kAraNadalilx parxyoVga mADide} (kAmadiMda) A hiMde heVLida hasivu, bAyArikegaLu huTuTxvavu, avugaLiMda sAvirAru beVre icACx visheVSagaLU huTuTxvavu||
\end{artha}

\vishaya{anavxya karxmavanunx tiLisuvudu}%%% 181

\begin{artha}
`\stext ' eMdu anavxya, atoyxVti eMbudu kirxyAthaRvuLaLxdAdxdare Atamxnu sakirxyanAguvudilalxve? eMdare ilalx. ililx kirxyAthaRvu vivakiSxtavalalx. 
\end{artha}

\vishaya{hAgAdare kirxyApadavanunx parxyoVgisidudx Etakekx?}

\begin{artha}
uSaNxteyu sheYtayxvanunx heVge vasutx savxBAvadiMdale yAvAgalu biTiTxruvudo, hAgeye (Atamxnu sabxBAvavAgiye hasivu modalAda saMsAridhamaRgaLanunx biTiTxruvanu) `\stext ' eMdu heVLuvaMte ililx parxmANavanunx anusarisi (muKAyxthaRdalilx parxyoVgaveMdu heVLuvudu) hoMduvudilalx. 
\end{artha}

\begin{artha}
hasivu bAyArike I eraDanunx mAtarx Atamxnu miVriruvudalalx matetxVneMdare avugaLigU yAva shoVkavu biVjavAgiruvado adanUnx miVriye iruvanu||
\end{artha}

\vishaya{ililx shoVka padakekx EnathaR?}

\begin{artha}%%% 181
ililx shoVkaveMdare arati mAtarx, adu citatxda avayxvasethx, shoVkavanunx moVhavanunx miVridavanu, moVhaveMdare citatxvu vipariVtavAgiruvike, aMdare budidhxyu horagina viSayagaLanunx apaharisuvudu||
\end{artha}

\begin{artha}
savxtaH nAnA rUpavalalxda vasutxvinalilx ajAcnxnadiMda yAva aneVka Atamx BAvane mADuvado adeV midhAyxjAcnxna, anAtamxvAgi toVruvadu ideV moVhaveMdu heVLalapxDuvudu|
\end{artha}

\vishaya{ashanAyApipAseV eMdu samAsa padaveVke}

\begin{artha}
savaRvananx nuMguvaMtaha pArxNa vasutxvina dhamaRvAdadadxriMda hasivu bAyArikegaLanunx samAsa padadiMda heVLiruvudu athavA shoVkada kAyaRvAdudariMdalU eraDanunx hAge heVLiruvudu||
\end{artha}

\begin{artha}
shoVka moVha shabadxgaLige samAsavilalxde nideRVsha mADiruvudu EkeMdare? Binanx pariNAma nimitatxdiMda, heVge? BinanxveMdare:- aratiya kAyaRparxvaqtitx, moVhadakAyaR anathaRda hoDeta||
\end{artha}

\begin{artha}
mupupx eMbudu deVhada pariNAma (vikAra) adu meYyalilx sukukx, talehaNaNxgi beLaLxhAguvudu, maqtuyx eMdare deVhada vicaCxVda=viyoVga, sUthxlashariVra liMgashariVragaLa viyoVgaveMdathaR
\end{artha}

\vishaya{ashanAyA itAyxdigaLu yAvudara dhamaRveMdare-}%%% 182

\begin{artha}
modalina hasivu bAyArike ivugaLu pArxNada dhamaR, shoVka moVhagaLu manasisxnalilxruva dhamaRgaLu, mupupx, maraNagaLu deVhada damaRgaLeMdu niNaRyavu (Atamxna dhamaRvalalx)||
\end{artha}

\begin{artha}
adariMda hasivu modalAdavugaLa saMbaMdhavu Atamxnige vAsatxvavAgi ilalx? EkeMdare: Atamxnu savcaRsaMga shUnayxnAdadxriMDa adu vAsatxvavalalx, vAsatxvaveV Agidadxre adariMda biDugaDeyeV AgadeV hoVguvudu||
\end{artha}

\begin{artha}
yAvudariMda I riVti asaMga puruSanAgiruvano adariMda parxtayxgAtamxna nija savxrUpavanunx tiLiyade iruvudariMda hasivu muMtAda dhamaR saMbaMdhaviruvudu, adariMda tatavxjAcnxnadiMda adunAshavAguvudu||
\end{artha}

\vishaya{ashanAyAdidhamaRgaLa atayxyaveMdare}

\begin{artha}
yAvudu tananx kAraNada seVrikoLuLxvudo adeV kAyaRgaLa layaveMdAguvudu, idu atayxMtavalalxdadxriMda I layavanunx garxhisuvudilalx||
\end{artha}

\vishaya{hAgAdare eMtaha atayxyavanunx garxhisabeVku-}%%% 183

\begin{artha}
A hasivu bAyArikegaLigU kAraNavAda ajAcnxnavu matutx kAma ivugaLa aBAvaveV savxmahimeyiMda vasutxvige yAvudu parxsidadhxvAgideye, adanenx atayxyaveMdu moVkaSx shAsatxrXdalilx moVkavanunx tiLidavaru heVLutAtxre||
\end{artha}

%%%  shloka footnote
\begin{artha}
\footnote[1]{saMvagaR eMdare pArxNa ``\stext " eMdu shurxti} adariMda ashanAyA pipAsA shabadxgaLiMda avugaLigU kAraNavAdadadxnenx garxhisabeVku, hasivubAyArikegaLanunx miVridedxMdu heVLidadxriMda adakakAraNavAda pArxNa vasutxvanunx miVriruvudedAyitAdadadxriMda adara kAyaRvanunx miVriruvudeMbudu niNaRyavAguvudu||
\end{artha}

\vishaya{ashanAyA padadiMda adarakAraNavanunx garxhisalu vAkayxda upakarxmavU sahAyakavAgide enunxvaru-}

\begin{artha}
``\stext " eMdu upakarxmisi I maqtuyxveMbu yAvudeMdu keVLidAga shurxtiyu idara lakaSxNavanunx `\stext " eMdu heVLiruvadu, EtakAkxgi eMdare-pArxNaveMba maqtayxvanunx ililx tiLiyuvadakAkxgi||
\end{artha}

\vishaya{}%%% 184
