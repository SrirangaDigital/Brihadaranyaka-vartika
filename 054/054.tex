
\section*{baq. a.5, bArx. 14, kaMDike 1}

\stext

\vishaya{hiMdu muMdina bArxhamxNagaLa saMbaMdhavu}

\begin{artha}
haqdaya, satayx itAyxdiyAgi aneVka upAdhigaLiMda kUDida barxhamxna 
upAsaneyanunx bahuPalagaloMdige heVLidAdxyitu. (Iga gAyatirxV 
upAdhiyiMda kUDida barxhomxVpaneyanunx heVLalu horaTide)
\end{artha}

\begin{artha}
Iga samasatx upAdhigaLanunx tanonxLage seVrisikoLuLxva 
gAyatirxVmaMtarxvanunx upAdhiyAgi mADikoMDu parabarxhomxVpAsaneyanunx 
Iga heVLabeVkeMdu muMdina shurxtiyu horaTiruvadu||
\end{artha}

\vishaya{itareV CaMdasusxgaLanunx Eke? biTuTxdudx?}

\begin{artha} 
mahatAtxda Palavu laBisalu gAyatirxge sAmathaRyxviruvudeMbudu 
shurxtiyalilx alalxlilx parxsidadhxvAgide. adariMda gAyatirxya 
upAsaneye sherxVSaThxveMdu tiLiyabeVku||
\end{artha}

\begin{artha} 
matutx mahatAtxda PalalABavu bArxhamxNanige Aguvudu, beVreyavarige 
alalx. kAraNa bArxhamxNanige gAyatirxyeV muKayxsAdhanavu adariMdalU 
gAyatirxyeV elalxdakUkx sherxVSaThxvAgide||
\end{artha}

\begin{artha} 
`BUmiranatxrikaSxmf' itAyxdi vAkayxdiMda aSATxkaSxravuLaLx gAyatirxyu 
mUruloVka savxrUpaveMdu heVLalapxDuvudu. gAyatirxge eMTu 
akaSxravuLaLxdudx. modalane pAdavuveMbudu tiLide ide.
\end{artha}

\begin{artha} 
gAyatirxya modalane pAdavu aSATxkaSxragaLa sAmayxviruvudariMda 
saMKAyxsAdaqshayxda nimitatxvAgi \footnote[1]{``\stext" eMdu 
bArxhamxNana eraDane janamxvu gAyatirxV nimitatxvAgide. adariMda 
gAyatirxye parxdhAna. ``\stext" eMdu utatxmavAda moVkaSxveMba 
puruSAthaR lABavanunx bArxhamxNanige heVLi toVrisiruvudu. alalxde 
elalx CaMdasusxgaLigU gAyatirxVCaMdasusx parxdhAna. idanunx 
parxyoVgisuva janara pArxNavanunx rakiSxsuvudariMda `gayatArxNAtf' 
gAyatirxyeMdu muMdeyu tiLisuvudu. I rakaSxNA sAmathaRyxvu mikakx 
CaMdasusxgaLige ilalx. alalxde elalx CaMdasusxgaLigU gAyatirxye 
pArxNavAgide. savaR CaMdasusxgaLigU pArxNAveV Atamx. EkeMdare elalx 
CaMdasusxgaLU pArxNadiMdale adara baladiMdale horaTubarabeVku. 
pArxNavu vAgAdi iMdirxyagaLige rakaSxka. kaSxtatArxNAtf kaSxtarxveMdU 
hiMde adanunx karedide. adariMda anuSAThxtaqgaLa vAgAdi 
iMdirxyagaLanunx rakiSxsuvudariMda gAyatirxyu pArxNaveMdu heVLide. 
adariMda iMtaha gAyatirxV CaMdasisxna upAdhiyiMda soVpAdhika 
barxhomxVpAsaneyanunx muMde heVLiruvudeMdu tAtapxyaR. bArxhamxNana 
bArxhamxNayxvu gAyatirxV mUlakavAgide. adariMdalU gAyatirxya 
savxrUpavanunx heVLabeVkAgide. gAyatirxyiMda punaH janamxvetitxda 
bArxhamxNanu niraMkushanAgi taDeyilalxde moVkaSxveMba parama 
puruSAthaR sAdhaneyanunx mADikoLaLxlu adhikAriyAgiruvanu.}terxYloVkayxvU 
oMdAgabahudu||
\end{artha}

\vishaya{matotxMdu kAraNavU ide --}

%% shloka footnote
\begin{artha} 
\footnote[2]{terxYloVkayx eMba padadalilx tatAra, reVPa, aisavxra, 
lakAra, OkAra, kakAra, yakAra eMdu eMTakaSxragaLu ive. hAgeye 
`tatasxvituvaRreVNayxmf' eMba parxthama pAdadalilx yakAravanunx 
seVrisikoMDare eMTakaSxragaLAguvavu. I riVti akaSxra sAmayxvideyeMde, 
1ne pAdavanunx terxYloVkayxveMdu BAvisabeVku||}\\
I terxYloVkayxveMdu heVLidudx aSATxkaSxra sAmayxdiMda. gAyatirxya 
modalane pAdakekx virATf eMba hesariniMda (BUmiritAyxdi maMtarxdiMdalU 
kUDi aSATxkaSxra sAmayxvanunx heVLidadxriMdalU) terxYloVkayx 
savxrUpanAda virATf puruSana Ekatavxvanu I pAdavu hoMduvudu||
\end{artha}

\begin{artha} 
I riVtiyAgi gAyatirxya modalane pAdavu mUruloVkagaLeV. 
kAraNaveVneMdare? shabAdxthaRgaLa saMbaMdhavu atayxMtavAgi (aMdare 
BeVdavilalxde tAdAtamxyXve) iruvudu, upAsakanu adariMda virATf 
puruSana aikayxvanunx hoMduvanu. (adariMda virATf puruSana rUpadalilx 
I pAdavanunx dhAyxnisabeVku)||
\end{artha}

\vishaya{saMkiSxpatxvAgi heVLida Palavanunx visatxrisutAtxre --}

%% shloka footnote
\begin{artha} 
\footnote[1]{aBidhAnaveMdare vAcakavAda shabadx. adu ililx gAyatirxya 
modalaneya pAda. aBidheVyaveMdare athaR, adu virATf puruSa, 
terxYloVkayx savxrUpanu. gAyatirxya AdayxpAdavu terxYloVkayx 
savxrUpanAda virATf puruSa savxrUpaveMdu dhAyxnamADalu adara PalavAgi 
virATf puruSana aikayxvanunx hoMduvaneMdathaR||}\\
hiMde heVLida gAyatirxya modalane pAdaveMba shabadxda mUlaka virATf 
eMbuva athaRvanunx oMdAgi anusaMdhAna mADidadxriMda yAvanu tirxloVka 
savxrUpavanunx hoMdiruvano, avanu heVLidaMte upAsaneya aBAyxsadiMda I 
upAsakanu virATf puruSaneV Aguvanu||
\end{artha}

\vishaya{shurxtuyxkatxvAda Palavanunx Iga heVLuvaru --}

\begin{artha} 
mUruloVkagaLalUlx puruSana BoVgakAkxgi ililx eSuTx sAdhanagaLu iveyoV 
adelalxvanUnx I modalaneya upAsaneyiMda PalavAgi shiVGarxdalelx 
sAvxdhiVnapaDisikoLuLxvanu||
\end{artha}

\vishaya{hiMdina kaMDikeyalilx udidxSaTxvAdadudx Enu?}

\begin{artha} 
BUmi modalAda rUpadalilx heVLidadxriMda gAyatirxya parxthama pAda 
savxrUpadalilx rUparAshiyanunx saMgarxhisidaMtAyitu. hAgU `QucuH 
yajUMSi' itAyxdi vacanadiMda (gAyatirxya divxtiVyapAda rUpadalilx) 
nAmarAshiyanunx saMgarxhisidaMtAyitu||
\end{artha}

\vishaya{rUparAshiyanunx upasaMharisida naMtaraveV nAmarAshiyanunx 
upasaMharisuva udedxVshaveVnu? eMdare --}

%% shloka footnote
\begin{artha} 
\footnote[1]{pUvoRVtatxra kaMDikegaLa tAtapxyaR saMkeSxVpavidu||}\\
athaR matutx shabadxgaLeMba eraDu rAshigaLU nitayxveMbudU matutx 
parasapxra oMdAgi saMbaMdhisigoMDiruvavu eMbudu vayxvasethxyAgive. 
parxkaqta gAyatirx mUlaka averaDanUnx heVLida hAgAyitu||
\end{artha}

\vishaya{divxtiVyapAdavanunx mUru veVdagaLa rUpadalilx dhAyxnisidare 
baruva Palavanunx heVLalu AraMBisidAdxre --}

\section*{baq. a.5, bArx. 14, kaMDike 2}

\stext

%% shloka footnote
\begin{artha} 
\footnote[2]{2ne kaMDikeyalilx gAyatirxya divxtiVyapAdavanunx 
veVdatarxyarUpadalilx dhAyxnisabeVku. QucoV, yajUMSi, sAmAni - ivu 
oTuTx eMTakaSxravuLaLxvu. ivugaLa sAdaqshayx gAyatirxV 
divxtiVyapAdakekx ive. I mUru veVdagaLiMda eSuTx PalagaLu 
pashuputArxdigaLu Aguvavo, avelalxvU I divxtiVyapAdada upAsaneyiMda 
laBisuvaveMdu tAtapxyaR.}\\
QugevxVda modalAda shabadxgaLu (veVdarAshiyU) eSaTxnunx (eSuTx 
Palavanunx) vAyxpisiruvuvo, avelalxvanunx sAdhakanu heVLida 
upAsaneyiMda hoMduvanu||
\end{artha}

\section*{baq. a.5, bArx. 14, kaMDike 3}

\stext

\begin{artha} 
pUvaRvAkayxkUkx muMdina vAkayxgaLigU vAyxKAyxnavu samAnaveV Agide. 
adariMda akaSxragaLige punaH vAyxKAyxnavanunx mADuvudilalx||
\end{artha}

\begin{artha} 
anaMtara Iga ``pArxNoVpAnoVdAyxna" itAyxdi maMtarxdiMda kamaRrAshiyu 
heVLalapxTiTxde. pArxNa eMdare hiMdina nAmarUpagaLige vidhaqtiH 
vidhArakaveMdu madhukAMDadalelx heVLalapxTiTxde. 
pArxNAtamxvanunx gAyatirxya mUraneV pAdavAgi mADikoMDu yatanxdiMda 
(niraMtara dhAyxnavAgi) BAvanemADi saMsAkxravanunx hoMdidavanu 
pArxNAtamxne Aguvanu\footnote[1]{pArxNAdivAyugaLelalxvU pArxNave eMdu 
sapAtxnanx parakaraNadalilx heVLide. `\stext' pArxNavanunx hogaLide.}||
\end{artha}

\vishaya{mUru vAkayxgaLa athaRvanunx upasaMharisuvaru --}

\begin{artha} 
athaRvanunx tirxloVkaveMba vacanadiMda upasaMhAra mADideyeMdu `Quca' 
itAyxdi vAkayxdalilx heVLide. adaraMteye shabadxda upasaMhAravU 
heVLalapxTiTxde. `pArxNa' itAyxdi vAkayxdiMda kamaRda upasaMhAravu 
ukatxvAgide. nAmArUpa, kamaR eMdu iSeTxV vasutx jagatitxnalilxruvudu. 
adu gAyatirxya pAdagaLanunx avalaMbiside||
\end{artha}

\vishaya{`athAsAyx EtadeVva turiVyaM dashaRtamf' I vAkayxda tAtapxyaR 
--}

\begin{artha} 
yAva athaRkekx vAcakavAda I gAyatirxyu tirxpadAtamxkavAgiruvado A 
nAlakxneV pAdavu athaRrUpadalilxruvudu. adanunx heVLalu muMdina 
\footnote[1]{``turiVyaM dashaRtaM padaM paroVrajA ya ESa tapati" 
eMbuva shurxtiyu ililx gArxhayx.}shurxtiyu upakarxmisalapxTiTxde||
\end{artha}

\vishaya{`yadevxY catuthaRmf' itAyxdi vAkayxdalilx punarukitx 
shaMkeyanunx parihAra mADutAtx vAyxKAyxnisutAtxre --}

\begin{artha} 
`yadevxYcatuthaRmf' eMba vacanadiMda shurxtiyu tuyaR muMtAda 
padasamUhada athaRvanunx tAneV tAtapxyaRdiMda heVLuvudu||
\end{artha}

\vishaya{``savaR muheyxYveYSa rajaH" eMbalilx rajasesxMbudara 
vAyxKAyxna --}

%% shloka footnote
\begin{artha} 
\footnote[1]{`suKAnushAyiVrAgaH' eMdu heVLide. `\stext' eMba BArata 
vacanavu kAma athavA rAgaveMbudu samasatx kAyaRgaLalilx 
parxvaqtitxyuMTumADuvudeMdu heVLide. saMgadiMda kAmavu 
huTuTxvudeMbudanunx `saMgAtasxMjAyateVkAmaH' eMba giVteyalilx heVLiye 
ide. BoVgavasutxgaLanenx ciMtisuvavanige avugaLalilx Asakitx 
huTuTxvudu adariMda kAmavu huTuTxvadeMdu adara athaR.}\\
yAvudu parxvaqtitxge kAraNavo adu rAga. ideV rajasesxMdu tiLiyabeVku, 
kAraNaveVneMdare? raMjisuvudariMda raja eMdu heVLide. raMjane, kAma, 
AsaMga ivu Atamxnige savARnathaRvanunxMTumADuvadu||
\end{artha}

\vishaya{`upayuRyxpari tapateyxVva' eMbudara athaR --}

\begin{artha} 
A elAlx loVkavanunx meVle Akarxmisi niMtu (I sUyaRmaMDaladalilxruva 
puruSanu) tapisuvanu. `tapati' eMbudu kirxyApada, 
\footnote[2]{`loVkArajAMsuyxcayxnetx' eMba shurxtiyaMte rajaH 
shabadxdiMda loVkagaLanunx BASayxdalilx heVLidaMte 
tegedukoLaLxbahudeMdu pakASxMtaravanunx `vA' eMba vAtiRkada padadiMda 
sUcitavAgide.}athavA 
rajashashxbadxdiMda samasatx (pArxNigaLiMda kUDida) loVkagaLe 
heVLalapxDutatxve||
\end{artha}

\begin{artha} 
ati uSaNxrashimxyuLaLx I sUyaRnu meVlemxle A loVkagaLanunx 
savARdhipatayxrUpavAda siriyiMdalU kiVtiRyiMdalU tapisuvanu. ivanaMte 
upAsakanU saha guNagaLiMda elalxranunx miVrisi meVlAgidudx siriyiMdalU 
ujajxvXlavAda KAyxtiyiMdalU shoVBisutAtx shaturxgaLanunx tapisuvanu. 
(avarige tApavuMTumADuvanu)||
\end{artha}

\vishaya{`upayuRpari' eMdu viVpesxge Enu parxyoVjanaveMdu AkeSxVpisi 
pariharisuvudu --}

\begin{artha} 
(gAyatirxya nAlakxne pAdavu AditayxmaMDaladalilxruva sUtArxtamx 
savxrUpavu. adanunx ahamf eMdu dhAyxnisuvavanige savARdhipatayxveMba 
Palavu laBisuvudeMdu heVLidAdxyitu. IvAga viVpAsx padada 
AkeSxVpavidu)-- savARdhipatayxveMbuva iSATxthaRvu savaR eMbudAgi 
tegedukoMDidadxriMdaleV labadhxvAgiruvudariMda `upayuRpari' eMdu 
viVpesxyu EtakAkxgi heVLalapxTiTxde? - eMdu AkeSxVpa. parihAra - idu 
doVSavalalx, savaRshabadxdiMda samasatx loVkagaLanunx 
garxhisabeVkAdadudx. adariMda yAva samasatx loVkagaLa meVle sUyaRnu 
meVlapxTiTxruvano, (adariMda viVpesxyu 
Avashayxka)\footnote[1]{viVpesxyu ilalxvAdare savARdhipatayxveMbudu 
ilalxvAguvudeMdu tAtapxyaR||}||
\end{artha}

\vishaya{viVpAsxthaRvanunx iTaTxlilx savARdhipatayxvu sididhxsuvudeMdu 
heVLuvaru --}

\begin{artha} 
sUyaRna meVliruva loVkagaLu yAvuduMTo avelalxvanunx ililx 
tegedukoLaLxbeVku. idu sididhxsuvudu heVge? eMdu (shurxtige 
AkAMkeSxyiruvudu). adariMda viVpesxyu parxyoVgisalapxTiTxde||
\end{artha}

\vishaya{`neYSA gAyatirxV' itAyxdi maMtarxda tAtapxyaR --}

\begin{artha} 
yAva gAyatirxyanunx jagatitxna ruPaveMdu hiMde 
\footnote[2]{gAyatirxya modalane pAdavu tirxloVkAtamxka, eraDane 
pAdavu mUru veVdagaLa savxrUpavu, mUrane pAdavu samasatx 
pArxNasavxrUpaveMdu yatanxdiMda hiMde shurxtiyu tiLiside.}yatanxdiMda vAyxKAyxnisididxto adu mUtARmUtaR vasutxgaLa 
sAravAda sUyaRnalilx tirxpAdiyAgi niMtiruvadu||
\end{artha}

\section*{baq. a.5, bArx. 14, kaMDike 4}

\stext

\begin{artha} 
`tadevxY tatasxteyxV parxtiSiThxtamf' - eMbuvalilx A I nAlakxnepAdavu 
satayxvAda adhAyxtamxdalilx neleside. parxkAsharUpavAda kaNiNxnalilx 
samasatxrUpavu nelesiruvadu||
\end{artha}

\vishaya{satayxveMdare Enu?}

\begin{artha} 
A satayxveMbudu yAvudeMdare satayxveMbudu kaNuNx eMdu heVLalapxDuvudu, 
kaNuNx heVge? satayxveMdare A satayxteyanunx heVLuvudu||
\end{artha}

\begin{artha} 
`tasAmxtf' itAyxdi vAkayxdiMda kaNiNxna satayxteyanunx vimashiRside, 
nAvu keVLidudx asatayxvAgiyU kaMDide. Adare nAvu noVDidudx suLeLxMdu 
anuBavadalilx kANuvudilalx. EkeMdare? kaNiNxniMda noVDidudx 
visheVSadalilx payARvasAnagoLuLxvudu, adariMda (kaNiNxniMda Aguva) 
dashaRnavu suLeLxMdu Aguvudilalx||
\end{artha}

\vishaya{kaNuNx satayxvAdare parxkaqta PalitAMshaveVnu? aMdare --}

\begin{artha} 
ideV riVtiyAgi nAlakxne pAdavu I satayxvAda yAvAgalU parxkAshavoMde 
savxBAvavAgiruva I kaNiNxnalilx sAkASxtf neVrA nelesiruvadu||
\end{artha}

\vishaya{`tadevxYtatf satayxM' - itAyxdi vAkayxda athaR --}

\begin{artha} 
parxkAsharUpavAda I kaNiNxnalilx elAlx rUpavU niMtiruvudu, 
balavenisida pArxNavasutxvinalUlx kaNiNxna parxtiSeThxyanunx 
(Asharxyavanunx) toVrisuvudu||
\end{artha}

\vishaya{Aditayxnu cakuSxriMdirxya mUlakavAgi pArxNavasutxvinalilx 
nilulxvudeMdu toVrisidAdxyitu. Iga pArxNavanunx iMdarxneMdU 
Aditayxnanunx aginxyeMdU upAsanegoVsakxra viMgaDisi heVLuvudu --}

\begin{artha} 
A pArxNa, Aditayx eMbavu iMdarx, aginx eMbudAgi sidadhxvAgive. 
avugaLalilx aginxyu parxkAshavuMTumADuvadu, eMdarx eMbudeV parxNa. adu 
hiMde heVLida nAmarUpagaLige vidhAraka. (aMdare avugaLanunx 
hiDiyabalalxdu)
\end{artha}

\vishaya{aginxyeV parxkAshaka eMbudu heVge?}

\begin{artha} 
jagatitxnalilx eSuTx parxkAshavideyo, adelalxvU aginxyeMdu ililx 
heVLalapxTiTxde. hAgU elAlx kaDeyalulx iruva calaneyu pArxNaveMdU 
calanAtamxkavAda pArxNavu (baladeVvateyAda) iMdarxneMdu (dhAyxnakAkxgi 
heVLiruvudu||)
\end{artha}

\vishaya{`EvamevxVSA gAyatirxV' - itAyxdi vAkayxda tAtapxyaR --}

\begin{artha} 
hiVgeye mUruloVkagaLu, mUruveVdagaLu, pArxNAdi mUruvasutxgaLu 
adhAyxtamxdalilx (I shariVradalilx) hiMde heVLidaMte (`neyxSA 
gAyatirxV' itAyxdiyAgi heVLida mAgaRdaMte) parxtiSiThxtavAgive||
\end{artha}

\begin{artha} 
pArxNa, AditayxribabxrU gAyatirxyalilx hiMde heVLida mAgaRdalilx 
seVrikoMDiruvaru. gAyatirxVrUpanAda IshavxraniMda beVre pArxNagaLige 
jiVvanavu Aguvudu. adariMda `sAheYSA' eMba shurxtiyu heVLida 
athaRvanunx tiLisuvudakAkxgiye baMdiruvudu. gaya eMdare pArxNagaLu 
(elAlx iMdirxyagaLu) avugaLanunx rakiSxsutatxdeyAdadxriMda 
gAyatirxyeMdu idu heVLalapxTiTxruvadu||
\end{artha}

\vishaya{`sayAmeVva' itAyxdi vAkayxda tAtapxyARthaR --}

\begin{artha} 
upanayana kAladalilx A AcAyaRnu pAdavAgiyU, athaR QukAkxgiyU, pUNaR 
QukAkxgiyU yAva savitaq deVvateyiMda kUDida I maMtarxrUpavAda 
gAyatirxyanunx vaTuvige upadeVshisuvano, A QukUkx ideV AgiruvudeMdu 
tiLiyabeVku. adanunx parxyatanxdiMda vAyxKAyxnisidAdxyitu. A AcAyaRnu 
yArige upadeVshisuvano avana pArxNagaLanunx (pArxNa 
iMdirxyAdigaLanunx) I gAyatirxyu takiSxsuvudu||
\end{artha}

\section*{upasaMhAra}

\begin{artha} 
I riVtiyAgi tiLida (I riVti upAsane mADida) A AcAyaRnu AdaradiMda kUDi 
yArige upadeVshisuvano. A vaTuvina pArxNagaLanunx I gAyatirxyu 
rakiSxsuvadu. adaralilx saMshayavilalx||
\end{artha}

\vishaya{tAmitAyxdi vAkayxda tAtapxyaR --}

\begin{artha} 
mANavakana upanayana samayadalilx veVdavAdigaLige CaMdasisxna 
viSayadalilx I vivAdavide. adara niNaRyakAkxgi I muMdina shurxtiyu 
baMdide.
\end{artha}

\section*{baq. a.5, bArx. 14, kaMDike 5}

\stext

\begin{artha} 
`tAM heYtAmf' eMba vAkayxdiMda sAvitirxVmaMtarxda viSayadalilx 
anuSaTxpf Canadxsasxnunx aMgiVkarisi toVrisabeVkAdadu, EtakAkxgi? 
eMdare? pUvaRpakaSxvu sididhxsuvudakAkxgi (ililx toVrisabeVku||)
\end{artha}

\vishaya{meVlina maMtarxda pUvARdhaRvanunx viMgaDisi anavxyisuvaru --}

%% shloka footnote
\begin{artha} 
\footnote[1]{`\stext' \\ eMbuva anuSaTxpf CaMdasusxLaLx QugevxVda 
maMtarx, idu savitaq deVvateyuLaLxdAdxdadxriMda sAvitirxyeMdu 
heVLutAtxre.}\\
A sAvitirxVmaMtarxvanunx kelavu AcAyaRru upaniVtanAda vaTuvige 
yatanxdiMda nAyxyavanunx avalaMbisi anuSaTxpf CaMdasisxnadeMdu 
boVdhisuvaru||
\end{artha}

\vishaya{pUvARpakaSx adeVnu nAyxya eMdare --}

\begin{artha} 
I \footnote[1]{`vAgAvx anuSaTxpf' eMba shurxtiyanunx Adharisi I 
AcAyaRru I pakaSxvanunx etitxruvaru. vAkukx sarasavxtiye eMbudu 
parxsidadhxveV AgideyAdadxriMda I anuSaTxpf Qukakxnunx heVLabeVkeMdu 
heVLutAtxre.}vAkukx anuSaTxpf CaMdasisxnadu. adariMda 
shariVradoLagiruva vAkukx sAkASxtf sarasavxtiye. adariMda 
upanayanavAgidadx vaTuvige adanenx heVLikoDabeVku. adu biTuTx 
beVreyilalxveMdu heVLuvaru||
\end{artha}

\vishaya{sidAdhxMta `na tathA' itAyxdiyAgi sidAdhxMta pakaSxvu}

%% shloka footnote
\begin{artha} 
\footnote[2]{idu tapupx, gAyatirxV CaMdasisxna maMtarx - 
`tatasxvituvaRreVNayxmf' itAyxdi maMtarxveV sAvitirxV savitarx deVvatA 
parxtipAdaka maMtarx. ideV vaTuvige upadeVshisabeVkAda maMtarxvu||}\\
I AcAyaRriMda heVge heVLalapxTiTxto hAgeye tiLidavanu heVLikoDabAradu. 
matetxVneMdare? gAyatirxyanenx (gAyatirxV CaMdasisxna maMtarxvanenx) 
sAvitirxyeMdu \footnote[2]{pArxNavanunx gAyatirxyeMdu heVLide. 
pArxNavanunx heVLida meVle vAkakxnunx sarasavxtiyanunx beVre 
iMdirxyagaLeMba pArxNavanunx elalxvanunx AcAyaRnunx mANavakanige 
opipxsida hAgAguvadu. adariMda gAyatirxyanenx sAvitirxyeMdu 
tiLiyabeVku||}samasatx PalavU laBisuvadariMda 
heVLikoDabeVku|| 
\end{artha}

\vishaya{savaRPalapArxpitx heVge?}

\begin{artha} 
hiMde heVLida gAyatirxyalelx elAlx jagatutx nelaside. adanunx 
heVLuvalilx elalxvanunx heVLidaMtAguvudu. yAvudu puruSAthaRkekx 
sAdhanavAgideyo (adelalxvU gAyatirxyalilxde)
\end{artha}

\vishaya{yadi havA itAyxdi vAkayxda tAtapxyaR --}

\begin{artha} 
vijAcnxnarUpanAda puruSanige (jiVvAtamxnige) yAvAgalU I elAlx 
jagatutx. sAmAnayx visheVSavuLaLxvelalxvU savxBAvavAgiye AtamaveV 
Agiruvadu||
\end{artha}

\vishaya{jiVvanige samaSiTx vayxSiTxrUpagaLu elilxMda? baMdavu.}

\begin{artha} 
sAdhAraNa vasutxgaLanunx asAdhAraNa vasutxgaLanUnx elalxvanunx 
sivxVkarisade pArxNige yAvudoMdu kirxyeyU naDeyuvudilalx\footnote[1]{}||
\end{artha}

\vishaya{upAsaneya naMtara samaSiTx vayxSiTxrUpagaLu baMdarU modalu 
heVge? iruvavu? eMdare --}

%% shloka footnote
\begin{artha} 
\footnote[2]{idanunx madhubArxhamxNadalilx noVDabahudu. samaSiTx 
vayxSiTx savxrUpavu EkAneVkarUpavu dhAyxnada baladiMda ciMtisida 
rUpavU oMdu riVtiyAgi modalu toVruvudu. sAkASxtAkxradiMda muMde 
vayxkatxvAguvAga adu ati sapxSaTxvAguvudu. mUtiRya sAkASxtAkxrakekx 
modalu dhAyxnakAladalilx mUtiRya AkAravu hoLeyuvaMte. modale idu 
itetxMdu aBipArxyavu.}\\
aBivayxkitxyAguva muMceyU savxBAvavAgi I rUpavu iruvudu. 
aBivayxkitxyAda meVle samaSiTx vayxSiTxrUpagaLu avu sAkASxtf 
toVruvavu||
\end{artha}

\begin{artha} 
I\footnote[1]{samasatx ceVtanagaLigU savxBAvavAgiye samaSiTx 
vayxSiTxrUpavu ideyeMbudanunx hiMde heVLida nAyxyadaMte sidadhxvAgalu 
gAyatirxV upAsakanige parxmANadiMda hArxsavaqdidhxgaLu 
baruvudilalxvAdadxriMda aneVka parxtigarxha mADidarU doVSavilalxveMdu 
aBipArxya.} riVtiyAda I mahimeyu sidadhxvAgiralu hiMde heVLida 
mAgaRdalilx hAni athavA vaqdidhxyu parxmANadiMda saMBavisuvaMtilalx||
\end{artha}

\vishaya{sAmAnayxvAgi gAyatirxV upAsakanige Palavanunx heVLi 
visheVSavAgiyU heVLalu AraMBisidAdxre --}

\begin{artha} 
ideV athaR (Atamxnige savxtaH vaqdidhx hAnigaLilalxveMbudanunx) 
aMgiVkarisi `sa ya' itAyxdi vAkayxdiMda doDaDx parxtigarxha 
mADuvudariMdalU I riVtiyAgi gAyatirxV upAsane mADidavanu doVSavanunx 
hoMduvudilalx. 
\end{artha}

\section*{baq. a.5, bArx. 14, kaMDike 6}

\stext

\begin{artha} 
`sa ya' itAyxdi vAkayxdiMda upAsaneya sutxtiye ideMdu heVLalapxTiTxde. 
kAraNaveVneMdare? aMtaha parxtigarxhaveMbudu ihadalilx elilxyU 
(yAralUlx) saMBavisuvudilalx||
\end{artha}

\vishaya{beVre tAtapxyaRvU ide --}

\begin{artha} 
\footnote[1]{nAnu vidAvxMsanu, idara baladiMda nanage baruva 
doVSavelalx BasamxvAguvudeMba duraBimAnadiMda duSaTx 
parxtigarxhavanunx mADabahudu. Itanu vidAvxMsaneMbuva bahu 
gawravadiMda avanige aneVka dAnagaLanunx koDabahudu. Adare I 
parxtigarxha doVSavu barxhamxsAkASxtAkxravilalxde hoVguvudilalx. 
sakala doVSagaLanunx pariharisuvudu jAcnxnavoMde beVre yAvudU ilalx. 
adara nepadiMda parxtigarxhavanunx niMdisuvudu. mUru loVkadaSuTx 
aparimita darxvayx parxtigarxhadiMda baruva suKaBoVgavu gAyatirxVpAda 
mAtarxvanunx upAsane mADida Palakekx samAnavalalx. adariMda 
parxtigarxhavu sherxVyasakxravalalx. parxtigarxhavu namamx 
puNayxvanenx kaLeyuvudu||}athavA ililx parxtigarxhavanunx niMdisuvudu tAnu 
vidAvxnf eMba aBimAnadiMda tanage asatf parxtigarxhavU 
saMBavisuvudariMda adu samasatx puNayxgaLanUnx kaLeyutatxdeyAda kAraNa 
adanunx niSeVdhisuvudakAkxgi parxtigarxhavanunx niMdisuvudu||
\end{artha}

\begin{artha} 
gAyatirxge heVLida pAdagaLalUlx mADuva jAcnxnavanunx (upAsaneyanunx) 
gAyatirxV upAsakanAdavanu pArxNisAmAnayxnunx bahuvAgi parxtigarxha 
mADidudx saha keDisuvudakekx samathaRvAgilalx||
\end{artha}

\vishaya{`kuta' itAyxdi vAkayxda tAtapxyaR}

\begin{artha} 
I riVtiyAgi dAnakoDuvavanU ilalx. aMtaha parxtigarxhavU ilalx. 
tegedukoLuLxvavanU ilalxveMdeV `kuta' itAyxdi vAkayxdiMda 
heVLalapxDuvadu||
\end{artha}

\vishaya{hAgAdare `saya imAnf' itAyxdi vAkAyxthaRvu heVge yukatx?}

\begin{artha} 
I dAnAdigaLanunx opipxkoMDarU saha upAsakanige yAva doVSavU 
saMBavisuvudilalx. adariMda `saya imA' itAyxdi shurxtiyu heVLiruvadu||
\end{artha}

\vishaya{I shurxtiyanunx Iga vAyxKAyxna mADuvudu --}

\begin{artha} 
yAvanAdarobabxnu I upAsakanu taqSeNxya upadarxvadiMda puruSana BoVga 
sAdhana vasutxgaLiMda tuMbiruva I loVkagaLanunx parxtigarxha mADali. 
avaniMda gAyatirxya oMdanepAda mAtarx parijAcnxna mADikoMDa Palave 
anuBava mADalapxTaTxMte Aguvudu. uLida gAyatirxV dashaRnada Palavanunx 
nAshagoLisuvudakekx A parxtigarxhavu kAraNavAguvudilalx||  
\end{artha}

\vishaya{`\stext' itAyxdi maMtarxda vAyxKAyxnavu --}

\begin{artha} 
I tarxyiV videyayu eSiTxdeyoV aSaTxnUnx parxtigarxha mADuvavanige 
eraDane pAdada jAcnxnakekx mAtarx hAniyAguvudu. Adare savaRtarx 
hAniyAgadu||
\end{artha}

\begin{artha} 
saMBavisadidadxrU aMtaha tarxyiVvidAyx saMbaMdha parxtigarxhavanunx 
kalipxdidalilx AgalU parxtigarxhavu mUrane pAdavanunx paDeyuvudilalx||
\end{artha}

\vishaya{`\stext' eMbudara tAtapxyaR --}

\begin{artha} 
barxhAmxMDadoLage vibAga hoMdiruva vasutxgaLanunx hiMde heVLidaMte 
parxtigarxha mADuvudariMda samaSiTx shariVravanunx dharisiruva anaMta 
savxrUpanAda gAyatirxV upAsakanige yAva hAniyU ilalx||
\end{artha}

\vishaya{anaMta savxrUpakekx nAshavilalxveMbudakekx inonxMdu kAraNa}

\begin{artha} 
elAlx jagatitxnalUlx parxtiyoMdu parxmANadalUlx pariciCxnanxdiMda 
aMdare alapxteyiMda vinAshiyAda viSayavuLaLx (upAsaneyu) kaMDide, 
Adare anaMta savxrUpakekx (catuthaRpAda jAcnxna Palakekx) elilxyU 
nAshaviruvudu kaMDilalx||
\end{artha}

\vishaya{barxhamxjAcnxnada PalakikxMta beVreyAdadudx anaMtaveMbudu 
heVge?}

\begin{artha} 
hagalUrAtirx I upAsakanu kaNuNx matutx sUyaRnalilxruva anaMta 
AtamxvAda pArxNavanunx (sUtArxtamxnanunx) upAsaneyiMda AtamxveMdeV 
BAvisi tAdAtamxyXvanunx hoMdidadxnu. adariMda avanige anaMtatavxvu 
(apariciCxnanxteyu) iruvudariMda yAvudariMdalU nAshaviruvudilalx. 
aMtavuLaLxdudx loVkadalilx kaSxyisuvudu aMtavilalxdudx yAvudariMdalU 
kaSxyisuvudilalx||
\end{artha}

\begin{artha} 
kaSxyaveMba shabadxdiMda duHKaveV tuMbiruva sAthxnagaLalilx Avaqtitx 
hiMdakekx baruvikeyeMdathaR. keYvalayxve koneyAgiruvudariMda samaSiTx 
(Eka) rUpavanunx dharisuvavanige I Avaqtitxyu iruvudilalx||
\end{artha}

\section*{baq. a.5, bArx. 14, kaMDike 7}

\stext

\begin{artha} 
hiMde heVLida athaRvanunx tiLidavanu sharxdedhxyiMda kUDi `gAyatirx 
asi' eMba maMtarxdiMda gAyatirxya upasAthxnavanunx yAvAgalU mADabeVku||
\end{artha}

\begin{artha} 
EkapadiV divxpadiV tirxpadiV eMdu modalina pAdagaLanUnx ciMtisabeVku. 
nAlakxneV pAdadiMda catuSATxtf eMdu ciMtisabeVku||
\end{artha}

\begin{artha} 
`apadasi' eMba padadiMda gAyatirxya AnaMtayxvanunx 
(vAyxpakatavxvanunx) heVLide. avayxyaLU akaSxyaLU niVneV Agididx. 
ninage aMtaveMbudu kANuvudilalx. `namasetx\char'263sutx tuhiyAya' eMba 
\end{artha}

\begin{artha} 
ukitxyiMda turiVyAthaR pArxdhAnayxvanunx hiMde heVLida mUru pAdagaLa 
apArxdhAnayxvanunx heVLide. `asaw' eMba padavu ililx
\end{artha}

\begin{artha} 
shaturxvina nAmavanunx garxhisuvudakAkxgi, adaH eMbudu Palashurxti 
idanunx upAsane mADuvavanige. hAgeye ahamf eMbudu||
\end{artha}

\vishaya{`\stext' eMba maMtarxgaLanunx anavxyisutAtxre --}

\begin{artha} 
yAva iSATxthaRvanunx shaturxvu bayasiruvano, adanunx I shaturxvu 
paDeyadirali. athavA avanu apeVkiSxsidudx ivanige vaqdidhx 
hoMdadirali. athavA shaturxvu apeVkiSxsidadxnunx nAneV hoMduvenu||
\end{artha}

\vishaya{oMdeV upasAthxnakekx eraDu virudadhxvAda PalagaLu heVge? 
eMdare --}

%% shloka footnote
\begin{artha} 
\footnote[1]{aBidhAna eMdare mUru maMtarxgaLu - 1. asaw, 2. ado, 3. 
mApArxpatf - eMdu ililx iTuTxkoLaLxbeVku. icACxnusAravAgi vikalapxvAgi 
iTuTxkoLaLxbahudu.}\\
aBicArakAkxgi aMdare shaturxvige toMdare mADuvudakAkxgi I 
upasAthxnavanunx heVLida naMtara mUru maMtarxgaLalilx EkadeVshavanunx 
heVLidadxriMda Pala viSayadalilx vikalapxvu (eraDAgiruvudu)||
\end{artha}

\vishaya{tanagAgi upasAthxna mADuvalilx I Palavu --}

\begin{artha} 
sAvxthaRkAkxgi `aha mada' eMdu upasAthxna mADidalilx maMtorxVkatxvAda 
Palavu Agiye Aguvudu. adaH= I parxyoVjanavanunx deVvi=eleY deVviye! 
ninanx anugarxhadiMda pArxpunxyAM= paDeyuvenu||
\end{artha}

\section*{sArAMsha}

\begin{artha} 
iSeTxV tiLiyabeVkAdadedx? athavA beVre oMdeVnAdarU uLidideye? eMdu 
keVLidalilx utatxra - beVre oMdu tiLiyabeVkAdadUdx ide. adilalxde 
pUNaRvAguvudilalx||
\end{artha}

\begin{artha} 
videyxyanunx pUNaRvAgi cenAnxgi aBAyxsa mADidadxlilx upAsakanige ade 
Palavanunx koDalu shakatxvAgiruvudu. idakekx vipariVtavAgiruvalilx adu 
anathaRkekx kAraNavAguvudeMbudanunx muMde parxtipAdisuvudu||
\end{artha}

\begin{artha} 
gAyatirxyu savARtamxkavAdadxriMda aginxyanunx sivxVkarisidaMtAgide. 
AdarU gAyatirxya muKaveMdu aginxyu tiLidilalxvAdadxriMda adanunx 
vidhisuvudakAkxgi muMdina shurxtiyu baMdide\\
\stext
\end{artha}

\vishaya{aginxyu gAyatirxya muKavAgidadxre `hasitxV BUtoV vahasi' eMdu 
heVLidudx heVge? --}

\begin{artha} 
muKajAcnxnavilalxdiruvudariMda anathaR Palavanunx heVLiruvudu. A 
gAyatirxge aginxyeV muKa eMdu heVLidadxriMda gAyatirxV daqSiTxyalilx 
pUNaRteyanunx heVLida hAgAyitu. gAyatirxV dashaRnakekx pUNaRPalavu 
ililx hiMde heVLida aginx muKajAcnxnadiMda AguvudeMdu yadi itAyxdi 
vacanadiMda Iga heVLuvudu||
\end{artha}

\vishaya{`\stext' itAyxdi Palashurxtiyanunx yoVcisutAtxre --}

\begin{artha} 
yAru aginxye muKavAgiruva I gAyatirxyanunx tiLiyuvano, upAsane 
mADuvano avanu aginxyeV Aguvanu. adariMda beMkiyu kaTiTxgeyanunx 
suTuTx urisuvaMte elalx parxtigarxha doVSavanunx suTuTx urisuvanu||
\end{artha}

\vishaya{`api bahivxva pApaM' - itAyxdi shurxtiyalilx iva shabadxda 
athaR --}

\begin{artha} 
suDuva beMkige iMdhanavu bahaLavAgiruvudilalx. Adare beMkiyu 
dAhayxvAda (iMdhana) saMbaMdhadiMda hecucxvudariMda samasatx 
dAhayxvasutxgaLigU kaSxyaveV Aguvudu. idariMda iva eMba padavu 
(bahuvAgilalx, bahuvAgiruvaMte eMba athaRdalilxruvudu)
\end{artha}

\section*{baq. a.5, bArx. 14, kaMDike 14}

\begin{artha} 
gAyatirxV upAsakanU ideV riVtiyAgi beMkiyu iMdhanavanunx saMpUNaR 
suDuvaMte samasatx pApavanunx tiMduhAki (nAshagoLisi) pApa 
leVpavilalxde shudadhxnAguvanu||
\end{artha}

\vishaya{`pUtoV\char'263 jaroV\char'263 maqtaH saMBavati' - eMbudara 
athaR --}

%% shloka footnote
\begin{artha} 
\footnote[1]{hiMde pApa saMbaMdhavilalxdadxriMda shudadhxnAguvuneMdu 
heVLide. Iga pApaPala saMbaMdhavilalxdadxriMda pUtateyanunx 
heVLideyAdadxriMda punarukitxyilalx. Atamxnige yAva pariNAmavU 
ilalxvAdadxriMda ajara, amaqta, eMdu heVLide. 
sUthxladeVhavilalxdariMdalU amaqta, ajara, amara eMdu tiLiyabeVku. 
gAyatirxV upAsakanu I gAyatirxyeMbudanunx parxjApati sUtArxtamx eMbuva 
samaSiTx pArxNarUpadalilx upAsane mADidadxriMda pArxNa mAtarx 
savxrUpavuLaLxvanAgiruvanu. aginxmuKavAgiruva gAyatirxyanunx upAsane 
mADiruvudariMda Itanu aginxyaMte savaRpApavanunx dahisikoMDu muMdedU 
pApavanunx mADade shudadhxnAda parxjApatibarxhamx 
savxrUpavuLaLxvanAguvaneMdu tAtapxyaR||}\\
inonxMdaroDane seVradiruva savxBAvavuLaLxdAdxdadxriMdaleV Atamxnu 
pavitarxnAgiruvanu, pariNAmavilalxdavanu. adariMda mupipxlalxdavanu, 
matutx amaqtanu, sUthxla shariVra saMbaMdhavilalxdadxriMda amaqtanu. 
pArxNamAtarx savxBAvavAgiruvavanAdadxriMda sUthxlashariVrarahitanu||
\end{artha}

aidane adhAyxyadalilx hadinAlakxne bArxhamxNavu 
\footnote[2]{vAtiRka garxMthadaMte 16ne bArxhamxNaveMta baredide. 
vAtiRkadalilx bArxhamxNa viBajane BASayxdalilxruva viBajanegiMta 
vayxtAyxsagoMDide. BASayxdaMte nAvu ililx 14ne bArxhamxNaveMta 
baredidedxVve. idanunx paMDitaru shoVdhisabeVkAdududx vAtiRkadaMte 
hadinAru bArxhamxNagaLeMdu viBAga mADikoLaLxbeVku.\\ 
\begin{tabular}{crccl}bArxhamxNa & 1 & vAtiRka & sholxVka & 1--130 
varege\\
" & 2 & " & " & 1--3 \qquad "\\
" & 3 & " & " & 1--5 \qquad "\\ 
" & 4 & " & " & 1--5 \qquad "\\
" & 5 & " & " & 1--14 \quad \,"\\ 
" & 6 & " & " & 1--6 \qquad "\\
" & 7 & " & " & 1--3 \qquad "\\
" & 8 & " & " & 1--4 \qquad "\\
" & 9 & " & " & 1--3 \qquad "\\
" & 10 & " & " & 1--7 \qquad "\\
" & 11 & " & " & 1--4 \qquad "\\
" & 12 & " & " & 1--6 \qquad "\\
" & 13 & " & " & 1--4 \qquad "\\
" & 14 & " & " & 1--15 \quad \,"\\
" & 15 & " & " & 1--9 \qquad "\\
" & 16 & " & " & 1--83 \quad \,"\end{tabular}}(hadinArane bArxhamxNavu) mugidide.

\eject

\section*{baq. a.5, bArx. 15}

\vishaya{aidane adhAyxya hadineYdane bArxhamxNavu (hadineVLane 
bArxhamxNa) samucacxyavanunx Acarisuvavanige savitaq matutx aginx I 
deVvategaLa upasAthxnavanunx muMde heVLalu AraMBiside --}

\begin{artha} 
hiMde heVLida upAsaneya aBAyxsadiMda saMsAkxravanunx hoMdida 
kamARnuSAThxna kataqRvige upasAthxnavanunx heVLabeVkeMdu muMdina 
shurxtiyu baMdide.\\
\stext\\
(jAcnxnakamaR samucacxyavanunx anuSAThxna mADuvavanu aMtayxkAladalilx 
Aditayxnanunx pArxthiRsuvudeV I bArxhamxNada muKayx tAtapxyaR--1)
\end{artha}

\vishaya{upasAthxnavanunx heVLalu Aditayx deVvatA parxsaMgavanunx 
mADide}

\begin{artha} 

\end{artha}

\begin{artha} 

\end{artha}

\begin{artha} 

\end{artha}

\begin{artha} 

\end{artha}

\begin{artha} 

\end{artha}

\begin{artha} 

\end{artha}

\begin{artha} 

\end{artha}

\begin{artha} 

\end{artha}

\begin{artha} 

\end{artha}

\begin{artha} 

\end{artha}

\begin{artha} 

\end{artha}

\begin{artha} 

\end{artha}


