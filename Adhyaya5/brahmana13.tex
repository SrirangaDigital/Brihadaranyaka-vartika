%%%% From 053.tex
\chapter{bArxhamxNa - 13}
%~ \section*{bArxhamxNa 13}

\begin{shl}
ukAthxdiguNapUgeVna vishiSaTxsAyxnarUpiNaH  | \\
upAsanamatheVdAniVM barxhamxNoV BaNayxteV paramf \hfill ||  1 || 
\end{shl}

%~ \section*{baq. a. 5, bArx. 13, kaMDike 1}
\kandike{kaMDike 1}
\begin{kandikeshl}
ukathxM pArxNoV vA ukathxM pArxNoV hiVdaM savaRmutAthxpayatuyxdAdhxsAmxdukathxvidivxVrasitxSaThxtuyxkathxsayx sAyujayxM saloVkatAM jayati ya EvaM veVda || 1 ||
\end{kandikeshl}

\begin{artha}
bArxhamxNada tAtapxyaR :- ukathx modalAda guNa samudAyadiMda kUDiruva pArxNa savxrUpavuLaLx barxhamxna upAsaneyanunx inunx muMde heVLuvadu.
\end{artha}

\vishaya{shurxtiya pada vAyxKAyxna \mdash }

\begin{shl}
\footnote{ukathx veMbudu oMdu shasatxrX, shasatxrX eMdare deVvateya guNagaLanunx hogaLuva QuknamxMtarx, gAnavilalxde deVvatAguNavanunx hogaLuvadu, sotxVtarxveMba sAmagAna mADida QukikxniMda deVvatAguNakiVtaRna mADuvadu, idu miVmAMsAshAsatxrXdalilx heVLuva sotxVtarx, shasatxrXgaLa pAriBASika padagaLa athaR.}ukathxM pArxNaH kutoV yasAmxtApxrXNeV sati carAcaramf | \\
\footnote{pArxNavu idadxre puNayxkamaRvo pApakamaRvo jagatatxnunx saqSiTxsuvudu, athavA jiVvanu kataRnAgidudx saqSiTxsuvanu.}utAthxpayati kameVRdaM katAR vA nAnayxthA tataH \hfill ||  2 || 
\end{shl}

%%%%%footnotes in shloka
\begin{artha}
ukathxveMba shasatxrXvu pArxNa, adu heVge? pArxNavu idadxre carAcarAtamxkavAda I jagatatxnunx kamaRvAgali, kataRnAgali saqSiTxsuvudu, beVre vidhadalilx saqSiTxsuvudilalx, adariMda ukathxveMbudu pArxNaveMdu (tiLiyabeVku).
\end{artha}

\begin{shl}
viVrashocxVkAthxtamxvitupxtarx upAsiVnasayx jAyateV | \\
daqSaTxM PalamadaqSaTxM tu huyxkathxseyxYkAtamxyXmashunxteV \hfill ||  3 || 
\end{shl}

\begin{artha}
viVranu = ukathx shasatxrXda savxrUpavanunx tiLidavanAda putarxnu upAsakanige huTuTxvanu, idoMdu parxtayxkaSx Pala, adaqSaTxvAda PalaveMdare ukathx shasatxrXdoDane EkAtamxsavxrUpavu (sAyujayx) idanunx Itanu hoMdutAtxne.
\end{artha}

%~ \section*{baq. 5, bArx. 13, kaMDike 2}
\kandike{kaMDike 2}
\begin{kandikeshl}
yajuH pArxNoV veY yajuH pArxNeV hiVmAni savARNi BUtAni yujayxnetxV yujayxnetxV hAsemxY savARNi BUtAni sherxYSAThxyXya yajuSaH sAyujayxM saloVkatAM jayati ya EvaM veVda || 2 ||
\end{kandikeshl}

\vishaya{I meVlina maMtarx vAyxKAyxna \mdash }

\begin{shl}
yajushacx pArxNa EveVti sadoVpAsiVta yatanxtaH | \\
kutoV yajuSaTxvXM taseyxVti yukitxleVshoV\s BidhiVyateV \hfill ||  4 || 
\end{shl}

\begin{shl}
yujayxnetxV savaRBUtAni saMhanayxnetxV parasapxramf | \\
pArxNeV sati yajusatxsAmxtaPxloVkitxH pUvaRvatatxthA \hfill ||  5 || 
\end{shl}

\begin{artha}
yajuveRVdavu pArxNave eMdu parxyatanxdiMda yAvAgalU upAsane mADabeVku, pArxNavu yajasusx heVge eMbudakekx saNaNxyukitxyanunx heVLide. elAlx pArxNigaLU yujayxnetxV aMdare parasapxra hoMdikoLuLxtatxveyAdadxriMdalU pArxNavu idadxre, yajususx horaDuvudu, adariMda PalashurxtiyU hiMdinaMteyeMdu tiLiyabeVku.
\end{artha}

\begin{shl}
sAmApi pArxNa EveVti kathaM taditi BaNayxteV | \\
samayxcnicx savaRBUtAni pArxNeV sati yatasatxtaH \hfill ||  6 || 
\end{shl}

\begin{artha}
sAmaveMde pArxNaveMdu upAsane mADabeVku, pArxNavu sAmaveVdavAdadudx heVge? eMbudanunx heVLuvudu, yAva kAraNadiMda elAlx pArxNigaLU pArxNavu idadxre sAmayxvanunx hoMdutatxve.
\end{artha}

%~ \section*{baq. 5, bArx. 13, kaMDike 3-4}
\kandike{kaMDike 3, 4}
\begin{kandikeshl}
sAma pArxNoV veY sAma pArxNeV hiVmAni savARNi BUtAni samayxcnicx samayxcnicx hAsemxY savARNi BUtAni sherxYSAThxyXya kalapxnetxV sAmanxH sAyujayxM saloVkatAM jayati ya EvaM veVda || 3 ||
\end{kandikeshl}

\begin{kandikeshl}
kaSxtatxrXM pArxNoV veY kaSxtatxrXM pArxNoV hi veY kaSxtatxrXM tArxyateV heYnaM pArxNaH kaSxNitoVH parx kaSxtatxrXmatarxpArxponxVti kaSxtatxrXsayx sAyujayxM saloVkatAM jayati ya EvaM veVda || 4 ||
\end{kandikeshl}

\vishaya{vAtiRka}

\begin{shl}
kaSxtarxrXM ca pArxNa EveVti cinatxyeVtasxtataM haqdA | \\
tArxyateV heVti yukutxyXkitxH kaSxtarxrXtavxsayx parxsidadhxyeV \hfill ||  7 || 
\end{shl}

\begin{artha}
kaSxtarxveMde pArxNavanunx haqdayadalilx yAvAgalU ciMtisabeVku, kaSxtarxveMdu heVge eMbudanunx tiLisalu `tArxyateV ha' eMbudu yukitxya vacana (pArxNavu) kaSxtarxveMbudanunx muMde yukitxyiMda vivarisuvaru \mdash 
\end{artha}

\begin{shl}
kaSxNitoVhiR kaSxtAtApxrXNasAtxrXyateV na tu taM vinA | \\
kaSxtarxM pArxNasatxtoV jecnxVyoV yathoVkatxnAyxyagwravAtf \hfill ||  8 || 
\end{shl}

\begin{artha}
pArxNavAyuvu shasAtxrXdiGAtadiMda hiMsisalapxTaTx gAyadiMda (deVhavanunx) (punaH mAMsAdi dhAtugaLiMda tuMbi) rakiSxsuvadu, pArxNavilalxde rakiSxsalu sAdhayxvilalx, adariMda pArxNavu kaSxtarxveMdu hiMde heVLida nAyxyada baladiMda hiVge tiLiyabeVku.
\end{artha}

\vishaya{parx. kaSxtarxmf' eMbudara athaR \mdash }

\begin{shl}
pArxponxVtayxtarxmasw kaSxtarxrXmatArxtaqkamaniVshavxramf | \\
sa Eva tArxtA savaRsayx hayxtarxM kaSxtarxrXmatoV BaveVtf \hfill ||  9 ||
\end{shl}


\begin{artha}
`parxkaSxtarx matarxmf' I upAsakanu matotxMdu rakaSxkanilalxde, beVre obabx niyAmakanU ilalxde iruva pArxNaveMba atarx, kaSxtarxvanunx hoMduvanu. \footnote{ililx pArxNavu keSxVtarx eMdu shasAtxrXGAtadiMda gAyagoMDa shariVravanunx tAnu idAdxga auSadhAdi upacAragaLiMda rakatxmAMsAdigaLa beLavaNige koDuva mUlaka rakiSxsuvudu. adariMda aupacArikavAgi kaSxtarx eMdu kaSxtAtftArxyata iti kaSxtarxM eMdu vuyxtapxtitxyiMda tiLiside. hAgeye atarx eMdu `natArxyateV aneyxVnakeVnacitf' eMba vuyxtapxtitxyiMda baruva avayavAthaRvanenx iTuTx atarx eMdu heVLida, adara athaR pArxNavAyuvanunx biTuTx beVre yAvudariMdalU deVhavu rakiSxsalapxDuvudilalxveMdathaR.}avaneV elalxvanunx rakiSxsuvanu adariMda atarx, kaSxtarxveMdAguvanu. (I pArxNadeVvateya sAyujayxvanunx sAloVkayxvanunx hoMduvanu).
\end{artha}

\begin{center}
{\bf ililxge baqhadAraNayxkoVpaniSadABxSayxda vAtiRkadalilx}

\smallskip
{\bf aidaneV adhAyxyada harimUrane bArxhamxNavu}

\smallskip
{\bf pUNaRgoMDide}
\end{center}
