%%%% From 054.tex
\chapter{bArxhamxNa - 14}
%~ \section*{baq. a.5, bArx. 14, kaMDike 1}
%~ \section*{bArxhamxNa 14}
\kandike{kaMDike 1}
\begin{kandikeshl}
BUmiranatxrikaSxM dwyxritayxSATxvakaSxrANayxSATxkaSxraM ha vA EkaM gAyaterxyXY padameVtadu heYvAsAyx Etatasx yAvadeVSu tirxSu loVkeVSu tAvadadhxjayati yoV\s sAyx EtadeVvaM padaM veVda || 1 ||
\end{kandikeshl}

\vishaya{hiMdu muMdina bArxhamxNagaLa saMbaMdhavu}

\begin{shl}
haqdayaM satayxmitAyxdibahUpAdhisamAsharxyamf | \\
upAsanamatikArxnatxM barxhamxNaH Palavadabxhu \hfill ||  1 || 
\end{shl}

\begin{artha}
haqdaya, satayx itAyxdiyAgi aneVka upAdhigaLiMda kUDida barxhamxna 
upAsaneyanunx bahuPalagaLoMdige heVLidAdxyitu. (Iga gAyatirxV 
upAdhiyiMda kUDida barxhomxVpaneyanunx heVLalu horaTide)
\end{artha}

\begin{shl}
savoVRpAdhuyxpasaMhArigAyaturxyXpAdhikaM parxBoVH  | \\
adhunoVpAsanaM vAcayxmitayxthaRH para AgamaH \hfill ||  2 || 
\end{shl}

\begin{artha}
Iga samasatx upAdhigaLanunx tanonxLage seVrisikoLuLxva 
gAyatirxVmaMtarxvanunx upAdhiyAgi mADikoMDu parabarxhomxVpAsaneyanunx 
Iga heVLabeVkeMdu muMdina shurxtiyu horaTiruvadu.
\end{artha}

\vishaya{itareV CaMdasusxgaLanunx Eke? biTuTxdudx?}

\begin{shl}
udAraPalasaMpArxpAtxveYshavxyaRM tatarx tatarx hi   | \\
shurxtaM shurxtw hi gAyatArxyXsatxdupAsitxvaRrA tataH \hfill ||  3 || 
\end{shl}

\begin{artha} 
mahatAtxda Palavu laBisalu gAyatirxge sAmathaRyxviruvudeMbudu 
shurxtiyalilx alalxlilx parxsidadhxvAgide. adariMda gAyatirxya 
upAsaneye sherxVSaThxveMdu tiLiyabeVku.
\end{artha}

\begin{shl}
udAraPalasaMbanodhxV bArxhamxNaseyxYva nAnayxtaH | \\
bArxhamxNasayx ca gAyatirxV kAraNaM teVna sA varA \hfill ||  4 || 
\end{shl}

\begin{artha} 
matutx mahatAtxda PalalABavu bArxhamxNanige Aguvudu, beVreyavarige 
alalx. kAraNa bArxhamxNanige gAyatirxyeV muKayxsAdhanavu adariMdalU 
gAyatirxyeV elalxdakUkx sherxVSaThxvAgide.
\end{artha}

\begin{shl}
BUmiritAyxdivAkeyxVna tirxloVkayxSATxkaSxroVcayxteV | \\
aSATxkaSxraM padaM sidadhxM gAyatArxyXH parxthamaM ca yatf \hfill ||  5 || 
\end{shl}

\begin{artha} 
`BUmiranatxrikaSxmf' itAyxdi vAkayxdiMda aSATxkaSxravuLaLx gAyatirxyu 
mUruloVka savxrUpaveMdu heVLalapxDuvudu. gAyatirxge eMTu 
akaSxravuLaLxdudx. modalane pAdavuveMbudu tiLide ide.
\end{artha}

\begin{shl}
aSATxkaSxratavxsAmAnAyxdAgxyatArxyXH parxthamaM padamf  | \\
tacecxVdaM ceYkatAmeVti saMKAyxsAmAnayxheVtutaH \hfill ||  6 || 
\end{shl}

\begin{artha} 
gAyatirxya modalane pAdavu aSATxkaSxragaLa sAmayxviruvudariMda 
saMKAyxsAdaqshayxda nimitatxvAgi \footnote{``gAyatArxyXbArxhamxNa masaqjata, tirxSuTxBArAjanayxM, jagatAyxveYshayxM" eMdu 
bArxhamxNana eraDane janamxvu gAyatirxV nimitatxvAgide. adariMda 
gAyatirxye parxdhAna. ``bArxhamxNA vuyxbAdhxya bArxhamxNA aBivadanitx, sa bArxhamxNoV vipApoVvirajoV\s vicikitosxV bArxhamx Bavati" eMdu utatxmavAda moVkaSxveMba 
puruSAthaR lABavanunx bArxhamxNanige heVLi toVrisiruvudu. alalxde 
elalx CaMdasusxgaLigU gAyatirxVCaMdasusx parxdhAna. idanunx 
parxyoVgisuva janara pArxNavanunx rakiSxsuvudariMda `gayatArxNAtf' 
gAyatirxyeMdu muMdeyu tiLisuvudu. I rakaSxNA sAmathaRyxvu mikakx 
CaMdasusxgaLige ilalx. alalxde elalx CaMdasusxgaLigU gAyatirxye 
pArxNavAgide. savaR CaMdasusxgaLigU pArxNaveV Atamx. EkeMdare elalx 
CaMdasusxgaLU pArxNadiMdale adara baladiMdale horaTubarabeVku. 
pArxNavu vAgAdi iMdirxyagaLige rakaSxka. kaSxtatArxNAtf kaSxtarxveMdU 
hiMde adanunx karedide. adariMda anuSAThxtaqgaLa vAgAdi 
iMdirxyagaLanunx rakiSxsuvudariMda gAyatirxyu pArxNaveMdu heVLide. 
adariMda iMtaha gAyatirxV CaMdasisxna upAdhiyiMda soVpAdhika 
barxhomxVpAsaneyanunx muMde heVLiruvudeMdu tAtapxyaR. bArxhamxNana 
bArxhamxNayxvu gAyatirxV mUlakavAgide. adariMdalU gAyatirxya 
savxrUpavanunx heVLabeVkAgide. gAyatirxyiMda punaH janamxvetitxda 
bArxhamxNanu niraMkushanAgi taDeyilalxde moVkaSxveMba parama 
puruSAthaR sAdhaneyanunx mADikoLaLxlu adhikAriyAgiruvanu.}terxYloVkayxvU 
oMdAgabahudu.
\end{artha}

\vishaya{matotxMdu kAraNavU ide \ndash }

\begin{shl}
\footnote{terxYloVkayx eMba padadalilx taKAra, reVPa, aisavxra, 
lakAra, OkAra, kakAra, yakAra hAgU `a'savxra eMdu eMTakaSxragaLu ive. hAgeye 
`tatasxvituvaRreVNayxmf' eMba parxthama pAdadalilx yakAravanunx 
seVrisikoMDare eMTakaSxragaLAguvavu. I riVti akaSxra sAmayxvideyeMde, 
1neV pAdavanunx terxYloVkayxveMdu BAvisabeVku.}terxYloVkAyxBihitiriyamaSATxkaSxrasamatavxtaH | \\
gAyatArxyXdayxpAdasamatoV virAjA\s \s temxYkayxmashunxteV \hfill ||  7 || 
\end{shl}

%% shloka footnote
\begin{artha} 
I terxYloVkayxveMdu heVLidudx aSATxkaSxra sAmayxdiMda. gAyatirxya 
modalane pAdakekx virATf eMba hesariniMda (BUmiritAyxdi maMtarxdiMdalU 
kUDi aSATxkaSxra sAmayxvanunx heVLidadxriMdalU) terxYloVkayx 
savxrUpanAda virATf puruSana Ekatavxvanu I pAdavu hoMduvudu.
\end{artha}

\begin{shl}
EvaM padaM tarxyoV loVkA aBidhAnABisaMgateVH | \\
atayxnatxmaBidheVyasayx virAjeYkayxM parxpadayxteV \hfill ||  8 || 
\end{shl}

\begin{artha} 
I riVtiyAgi gAyatirxya modalane pAdavu mUruloVkagaLeV. 
kAraNaveVneMdare? shabAdxthaRgaLa saMbaMdhavu atayxMtavAgi aMdare 
BeVdavilalxde tAdAtamxyXveV Agi iruvudu, upAsakanu adariMda virATf 
puruSana aikayxvanunx hoMduvanu. (adariMda virATf puruSana rUpadalilx 
I pAdavanunx dhAyxnisabeVku).
\end{artha}

\vishaya{saMkiSxpatxvAgi heVLida Palavanunx visatxrisutAtxre \ndash }

\begin{shl}
\footnote{aBidhAnaveMdare vAcakavAda shabadx. adu ililx gAyatirxya 
modalaneya pAda. aBidheVyaveMdare athaR, adu virATf puruSa, 
terxYloVkayx savxrUpanu. gAyatirxya AdayxpAdavu terxYloVkayx 
savxrUpanAda virATf puruSa savxrUpaveMdu dhAyxnamADalu adara PalavAgi 
virATf puruSana aikayxvanunx hoMduvaneMdathaR.}ukAtxBidhAnadAvxreVNa saMpanonxV yasitxrXloVkatAmf | \\
yathoVkotxVpAsanABAyxsAdivxrADeVva BaveVdasw \hfill ||  9 || 
\end{shl}

%% shloka footnote
\begin{artha} 
hiMde heVLida gAyatirxya modalane pAdaveMba shabadxda mUlaka virATf 
eMbuva athaRvanunx oMdAgi anusaMdhAna mADidadxriMda yAvanu tirxloVka 
savxrUpavanunx hoMdiruvano, avanu heVLidaMte upAsaneya aBAyxsadiMda I 
upAsakanu virATf puruSaneV Aguvanu.
\end{artha}

\vishaya{shurxtuyxkatxvAda Palavanunx Iga heVLuvaru \ndash }

\begin{shl}
tirxSu loVkeVSu yAvatAsxyXtupxMBoVgAyeVha sAdhanamf  | \\
tAvatasxvaRM jayatAyxshu parxthamoVpAsitxtaH Palamf \hfill ||  10 ||
\end{shl}

\begin{artha} 
mUruloVkagaLalUlx puruSana BoVgakAkxgi ililx eSuTx sAdhanagaLu iveyoV 
adelalxvanUnx I modalaneya upAsaneyiMda PalavAgi shiVGarxdalelx 
sAvxdhiVnapaDisikoLuLxvanu.
\end{artha}

\vishaya{hiMdina kaMDikeyalilx udidxSaTxvAdadudx Enu?}

\begin{shl}
rUparAshirasheVSeVNa BUmAyxduyxkotxyXVpasaMhaqtaH | \\
Quca itAyxdinA jecnxVyA nAmarAshuyxpasaMhaqtiH \hfill ||  11 || 
\end{shl}

\begin{artha} 
BUmi modalAda rUpadalilx heVLidadxriMda gAyatirxya parxthama pAda 
savxrUpadalilx rUparAshiyanunx saMgarxhisidaMtAyitu. hAgU `QucaH 
yajUMSi' itAyxdi vacanadiMda (gAyatirxya divxtiVyapAda rUpadalilx) 
nAmarAshiyanunx saMgarxhisidaMtAyitu.
\end{artha}

\vishaya{rUparAshiyanunx upasaMharisida naMtaraveV nAmarAshiyanunx 
upasaMharisuva udedxVshaveVnu? eMdare \ndash }


\begin{shl}
\footnote{pUvoRVtatxra kaMDikegaLa tAtapxyaR saMkeSxVpavidu.}aBidheVyABidhAnAKayxrAshiV nitwyx vayxvasithxtw | \\
parasapxrABisaMbanwdhx gAyatirxVvatamxRnoVditw \hfill ||  12 ||
\end{shl}

%% shloka footnote
\begin{artha} 
athaR matutx shabadxgaLeMba eraDu rAshigaLU nitayxveMbudU matutx 
parasapxra oMdAgi saMbaMdhisigoMDiruvavu eMbudu vayxvasethxyAgive. 
parxkaqta gAyatirx mUlaka averaDanUnx heVLida hAgAyitu.
\end{artha}

\vishaya{divxtiVyapAdavanunx mUru veVdagaLa rUpadalilx dhAyxnisidare 
baruva Palavanunx heVLalu AraMBisidAdxre \ndash }


\begin{shl}
\footnote{2ne kaMDikeyalilx gAyatirxya divxtiVyapAdavanunx 
veVdatarxyarUpadalilx dhAyxnisabeVku. QucoV, yajUMSi, sAmAni - ivu 
oTuTx eMTakaSxravuLaLxvu. ivugaLa sAdaqshayx gAyatirxV 
divxtiVyapAdakekx ive. I mUru veVdagaLiMda eSuTx PalagaLu 
pashuputArxdigaLu Aguvavo, avelalxvU I divxtiVyapAdada upAsaneyiMda 
laBisuvaveMdu tAtapxyaR.}QugAdeVraBidhAnasayx yAvatiV vAyxpitxriSayxteV | \\
tAvatasxvaRmavAponxVti yathoVkotxVpAsanAnanxraH \hfill ||  13 || 
\end{shl}

%~ \section*{baq. a.5, bArx. 14, kaMDike 2}
\kandike{kaMDike 2}
\begin{kandikeshl}
QucoV yajUMSi sAmAniVtayxSATxvakaSxrANayxSATxkaSxraM ha vA EkaM gAyaterxyXY padameVtadu heYvAsAyx Etatasx yAvatiVyaM tarxyiV vidAyx tAvadadhx jayati yoV\s sAyx EtadeVvaM padaM veVda || 2 ||
\end{kandikeshl}

%% shloka footnote
\begin{artha} 
QugevxVda modalAda shabadxgaLu (veVdarAshiyU) eSaTxnunx (eSuTx 
Palavanunx) vAyxpisiruvuvo, avelalxvanunx sAdhakanu heVLida 
upAsaneyiMda hoMduvanu.
\end{artha}

%~ \section*{baq. a.5, bArx. 14, kaMDike 3}
\kandike{kaMDike 3}
\begin{kandikeshl}
pArxNoV\s pAnoV vAyxna itayxSATxvakaSxrANayxSATxkaSxraM ha vA EkaM gAyaterxyXY padameVtadu heYvAsAyx Etatasx yAvadidaM pArxNi tAvadadhx jayati yoV\s sAyx EtadeVvaM padaM veVdAthAsAyx EtadeVva turiVyaM dashaRtaM padaM paroV rajA ya ESatapati yadevxY catuthaRM tatutxriVyaM dashaRtaM padamiti dadaqsha iva heyxVSa paroVrajA iti savaRmu heyxVveYSa raja upayuRpari tapateyxV vaM heYva shirxyA yashasA tapati yoV\s sAyx EtadeVvaM padaM veVda || 3 ||
\end{kandikeshl}

\vishaya{vAtiRka}

\begin{shl}
vAyxKAyx tuleyxYva pUveVRNa vAkeyxVnoVtatxravAkayxyoVH | \\
akaSxrANAmiha jecnxVyA nAtoV vAyxKAyxyateV punaH \hfill ||  14 || 
\end{shl}

\begin{artha} 
pUvaRvAkayxkUkx muMdina vAkayxgaLigU vAyxKAyxnavu samAnaveV Agide. 
adariMda akaSxragaLige punaH vAyxKAyxnavanunx mADuvudilalx.
\end{artha}

\begin{shl}
kamaRrAshiratheVdAniVM pArxNa itAyxdinoVcayxteV | \\
vidhaqtiH pUvaRyoVH pArxNoV madhukANeDxV yathoVditaH \hfill ||  15 || 
\end{shl}

\begin{shl}
pArxNAtAmxnaM yathoVketxVna pAdeVnA\s \s pAdayx yatanxtaH | \\
pArxNAtemxYva BavatAyxshu sadA tadABxvaBAvitaH \hfill ||  16 || 
\end{shl}

\begin{artha} 
anaMtara Iga ``pArxNoVpAnoVvAyxna" itAyxdi maMtarxdiMda kamaRrAshiyu 
heVLalapxTiTxde. pArxNa eMdare hiMdina nAmarUpagaLige vidhaqtiH 
vidhArakaveMdu madhukAMDadaleVlx heVLalapxTiTxde. 
pArxNAtamxvanunx gAyatirxya mUraneV pAdavAgi mADikoMDu yatanxdiMda 
(niraMtara dhAyxnavAgi) BAvanemADi saMsAkxravanunx hoMdidavanu 
pArxNAtamxne Aguvanu\footnote{pArxNAdivAyugaLelalxvU pArxNaveV eMdu 
sapAtxnanx parxkaraNadalilx heVLide. `pANoV\s pAnoVvAyxna udAnaH samAnoV\s na iteyxVtatasxvaRM pArxNa Eva' pArxNavanunx hogaLide.}.
\end{artha}

\vishaya{mUru vAkayxgaLa athaRvanunx upasaMharisuvaru \ndash }

\begin{shl}
aBidheVyoVpasaMhArasirxloVkiVvacasoVditaH | \\
Quca itAyxdinA tadavxdaBidhAnoVpasaMhaqtiH \hfill ||  17 || 
\end{shl}

\begin{shl}
kamaRNoV\s puyxpasaMhAraH pArxNa itAyxdinA tathA | \\
EtAvadavxsutx jagati gAyatirxVpAdasaMsharxyamf \hfill ||  18 || 
\end{shl}

\begin{artha} 
athaRvanunx tirxloVkaveMba vacanadiMda upasaMhAra mADideyeMdu `Quca' 
itAyxdi vAkayxdalilx heVLide. adaraMteye shabadxda upasaMhAravU 
heVLalapxTiTxde. `pArxNa' itAyxdi vAkayxdiMda kamaRda upasaMhAravu 
ukatxvAgide. nAmarUpa, kamaR eMdu iSeTxV vasutx jagatitxnalilxruvudu. 
adu gAyatirxya pAdagaLanunx avalaMbiside.
\end{artha}

\vishaya{`athAsAyx EtadeVva turiVyaM dashaRtamf' I vAkayxda tAtapxyaR 
\ndash }

\begin{shl}
aBidheVyasayx gAyatirxV yaseyxVyaM tirxpadA matA | \\
tasAyxBidhitasxyeVdAniVM pArxrabedhxYSoVtatxrA shurxtiH \hfill ||  19 || 
\end{shl}

\begin{artha} 
yAva athaRkekx vAcakavAda I gAyatirxyu tirxpadAtamxkavAgiruvado A 
nAlakxneV pAdavu athaRrUpadalilxruvudu. adanunx heVLalu muMdina 
\footnote{``turiVyaM dashaRtaM padaM paroVrajA ya ESa tapati" 
eMbuva shurxtiyu ililx gArxhayx.}shurxtiyu upakarxmisalapxTiTxde.
\end{artha}

\vishaya{`yadevxY catuthaRmf' itAyxdi vAkayxdalilx punarukitx 
shaMkeyanunx parihAra mADutAtx vAyxKAyxnisutAtxre \ndash }

\begin{shl}
vAyxcaSeTxV savxyameVvAthaRM tuyARdipadasaMhateVH | \\
yadevxY catuthaRmitAyxdivacasA yatanxtaH shurxtiH \hfill ||  20 || 
\end{shl}

\begin{artha} 
`yadevxYcatuthaRmf' eMba vacanadiMda shurxtiyu tuyaR muMtAda 
padasamUhada athaRvanunx tAneV tAtapxyaRdiMda heVLuvudu.
\end{artha}

\vishaya{``savaR muheyxYveYSa rajaH" eMbalilx rajasesxMbudara 
vAyxKAyxna \ndash }

\begin{shl}
rAgoV raja iti jecnxVyaH parxvaqtetxVH kAraNaM tu yatf  | \\
\footnote{`suKAnushAyiVrAgaH' eMdu heVLide. `yadayxdidhx kuruteV kamaR tatatxtf kAmasayxceVSiTxtamf' eMba BArata 
vacanavu kAma athavA rAgaveMbudu samasatx kAyaRgaLalilx 
parxvaqtitxyuMTumADuvudeMdu heVLide. saMgadiMda kAmavu 
huTuTxvudeMbudanunx `saMgAtasxMjAyateVkAmaH' eMba giVteyalilx heVLiye 
ide. BoVgavasutxgaLaneVnx ciMtisuvavanige avugaLalilx Asakitx 
huTuTxvudu adariMda kAmavu huTuTxvadeMdu adara athaR.}racnajxnaM kAma AsaknagxH savARnathaRkaqdAtamxnaH \hfill ||  21 || 
\end{shl}

%% shloka footnote
\begin{artha} 
yAvudu parxvaqtitxge kAraNavo adu rAga. ideV rajasesxMdu tiLiyabeVku, 
kAraNaveVneMdare? raMjisuvudariMda raja eMdu heVLide. raMjane, kAma, 
AsaMga ivu Atamxnige savARnathaRvanunxMTumADuvadu.
\end{artha}

\vishaya{`upayuRyxpari tapateyxVva' eMbudara athaR \ndash }

\begin{shl}
taM karxmitovxVpari sithxtAvx tapatiVti kirxyApadamf | \\
BUtAni vA rajoVvAcA BaNayxnetxV\s tArxKilAni tu \hfill ||  22 || 
\end{shl}

\begin{artha} 
A elAlx loVkavanunx meVle Akarxmisi niMtu (I sUyaRmaMDaladalilxruva 
puruSanu) tapisuvanu. `tapati' eMbudu kirxyApada, 
\footnote{`loVkArajAMsuyxcayxnetx' eMba shurxtiyaMte rajaH 
shabadxdiMda loVkagaLanunx BASayxdalilx heVLidaMte 
tegedukoLaLxbahudeMdu pakASxMtaravanunx `vA' eMba vAtiRkada padadiMda 
sUcitavAgide.}athavA 
rajashashxbadxdiMda samasatx (pArxNigaLiMda kUDida) loVkagaLeV heVLalapxDutatxve.
\end{artha}

\begin{shl}
upayuRpari tAneyxVSa tapatuyxlabxNarashimxvAnf | \\
shirxVyashoVBAyxM yatheYvAyaM tapateyxVvamupAsakaH \hfill ||  23 ||
\end{shl}

\begin{shl}
guNeYH savARnatikarxmayx sithxtovxVpari shirxyA javxlanf | \\
yashasA ca sudiVpetxVna vidAvxMsatxpati vidivxSaH \hfill ||  24 || 
\end{shl}

\begin{artha} 
ati uSaNxrashimxyuLaLx I sUyaRnu meVleVmxle A loVkagaLanunx 
savARdhipatayxrUpavAda siriyiMdalU kiVtiRyiMdalU tapisuvanu. ivanaMte 
upAsakanU saha guNagaLiMda elalxranunx miVrisi meVlAgidudx siriyiMdalU 
ujavxlavAda KAyxtiyiMdalU shoVBisutAtx shaturxgaLanunx tapisuvanu. 
(avarige tApavuMTumADuvanu).
\end{artha}

\vishaya{`upayuRpari' eMdu viVpesxge Enu parxyoVjanaveMdu AkeSxVpisi 
pariharisuvudu}

\begin{shl}
kAmitAthaRsayx labadhxtAvxtasxvaRgarxhaNamAtarxtaH | \\
upayuRpariVti viVpAsx kimathaRmiti BaNayxteV \hfill ||  25 || 
\end{shl}

\begin{shl}
neYva doVSoV yatoV yeVSAM sUyaRH sAyxdupari sithxtaH | \\
loVkAnAM savaRshabedxVna teVSAmeVva garxhoV BaveVtf \hfill ||  26 || 
\end{shl}

\begin{artha} 
(gAyatirxya nAlakxne pAdavu AditayxmaMDaladalilxruva sUtArxtamx 
savxrUpavu. adanunx ahamf eMdu dhAyxnisuvavanige savARdhipatayxveMba 
Palavu laBisuvudeMdu heVLidAdxyitu. IvAga viVpAsx padada 
AkeSxVpavidu)\ndash  savARdhipatayxveMbuva iSATxthaRvu savaR eMbudAgi 
tegedukoMDidadxriMdaleV labadhxvAgiruvudariMda `upayuRpari' eMdu 
viVpesxyu EtakAkxgi heVLalapxTiTxde? - eMdu AkeSxVpa. parihAra - idu 
doVSavalalx, savaRshabadxdiMda samasatx loVkagaLanunx 
garxhisabeVkAdadudx. adariMda yAva samasatx loVkagaLa meVle sUyaRnu 
meVlapxTiTxruvano, (adariMda viVpesxyu 
Avashayxka)\footnote{viVpesxyu ilalxvAdare savARdhipatayxveMbudu 
ilalxvAguvudeMdu tAtapxyaR.}.
\end{artha}

\vishaya{viVpAsxthaRvanunx iTaTxlilx savARdhipatayxvu sididhxsuvudeMdu 
heVLuvaru \ndash }

\begin{shl}
parAcnocxV yeV raveVloVRkAsetxVSAmapi parigarxhaH | \\
kathaM nu nAma sidadhxH sAyxdatoV viVpAsx parxyujayxteV \hfill ||  27 || 
\end{shl}

\begin{artha} 
sUyaRna meVliruva loVkagaLu yAvuduMTo avelalxvanunx ililx 
tegedukoLaLxbeVku. idu sididhxsuvudu heVge? eMdu (shurxtige 
AkAMkeSxyiruvudu). adariMda viVpesxyu parxyoVgisalapxTiTxde.
\end{artha}

\vishaya{`neYSA gAyatirxV' itAyxdi maMtarxda tAtapxyaR \ndash }

\begin{shl}
vAyxKAyxtA yA purA yatAnxdAgxyatirxV jagadAtimxkA | \\
mUtARmUtaRraseV BAnw tirxpAdeVSA parxtiSiThxtA \hfill ||  28 || 
\end{shl}

\begin{artha} 
yAva gAyatirxyanunx jagatitxna rUpaveMdu hiMde 
\footnote{gAyatirxya modalane pAdavu tirxloVkAtamxka, eraDane 
pAdavu mUru veVdagaLa savxrUpavu, mUrane pAdavu samasatx 
pArxNasavxrUpaveMdu yatanxdiMda hiMde shurxtiyu tiLiside.}yatanxdiMda  vAyxKAyxnisididxto adu mUtARmUtaR vasutxgaLa 
sAravAda sUyaRnalilx tirxpAdiyAgi niMtiruvadu.
\end{artha}

%~ \section*{baq. a.5, bArx. 14, kaMDike 4}
\kandike{kaMDike 4}
\begin{kandikeshl}
seYSA gAyaterxyXVtasimxMsutxriVyeV dashaRteV padeV paroVrajasi parxtiSiThxtA tadevxY tatasxteyxV parxtiSiThxtaM cakuSxveYR satayxM cakuSxhiR veY satayxM tasAmxdayxdidAniVM dwvx vivadamAnAveVyAtAmahamadashaRmahamashwrxSamiti ya EvaM bUrxyAdahamadashaRmiti tasAmx Eva sharxdadxdhAyxma tadevxY tatasxtayxM baleV parxtiSiThxtaM pArxNoV veY balaM tatApxrXNeV parxtiSiThxtaM tasAmxdAhubaRlaM satAyxdoVgiVya iteyxVvaMveVSA gAyatarxyXdhAyxtamxM parxtiSiThxtA sA heYSA gAyaMsatxterxV pArxNA veY gayAsatxtApxrXNAMsatxterxV tadayxdagxyAMsatxterxV tasAmxdAgxyatirxV nAma sa yAmeVvAmUM sAvitirxVmanAvxheYSeYva sA sa yasAmx anAvxha tasayx pArxNAMsAtxrXyateV || 4 ||
\end{kandikeshl}

\vishaya{vAtiRka}

\begin{shl}
tadapiVdaM padaM tuyaRM sateyxV\s dhAyxtemxV parxtiSiThxtamf | \\
akiSxNi parxkAsharUpeV hi savaRM rUpaM parxtiSiThxtamf \hfill ||  29 ||
\end{shl}

\begin{artha} 
`tadevxY tatasxteyxV parxtiSiThxtamf' - eMbuvalilx A I nAlakxnepAdavu 
satayxvAda adhAyxtamxdalilx neleside. parxkAsharUpavAda kaNiNxnalilx 
samasatxrUpavu nelesiruvadu.
\end{artha}

\vishaya{satayxveMdare Enu?}

\begin{shl}
kiM punasatxtasxtayxmiti satayxM cakuSxritiVyaRteV | \\
cakuSxSaH satayxtA kasAmxditi ceVducayxteV tathA \hfill ||  30 || 
\end{shl}

\begin{artha} 
A satayxveMbudu yAvudeMdare satayxveMbudu kaNuNx eMdu heVLalapxDuvudu, 
kaNuNx heVge? satayxveMdare A satayxteyanunx heVLuvudu.
\end{artha}

\begin{shl}
tasAmxditAyxdivAkeyxVna satayxtA\s koSxNXV viBAvayxteV | \\
daqSaTxM maqSA\s pi sharxvaNaM na tu daqSiTxmaqRSeVkaSxyXteV | \\
tasAyx visheVSaniSaThxtAvxnamxqqSAtavxM noVpapadayxteV \hfill ||  31 || 
\end{shl}

\begin{artha} 
`tasAmxtf' itAyxdi vAkayxdiMda kaNiNxna satayxteyanunx vimashiRside, 
nAvu keVLidudx asatayxvAgiyU kaMDide. Adare nAvu noVDidudx suLeLxMdu 
anuBavadalilx kANuvudilalx. EkeMdare? kaNiNxniMda noVDidudx 
visheVSadalilx payaRvasAnagoLuLxvudu, adariMda (kaNiNxniMda Aguva) 
dashaRnavu suLeLxMdu Aguvudilalx.
\end{artha}

\vishaya{kaNuNx satayxvAdare parxkaqta PalitAMshaveVnu? aMdare \ndash }

\begin{shl}
EvaM turiVyameVtasimxnasxteyxV cakuSxSi savaRdA | \\
parxkAsheYkasavxBAveV hi sAkASxdeVva parxtiSiThxtamf \hfill ||  32 || 
\end{shl}

\begin{artha} 
ideV riVtiyAgi nAlakxne pAdavu I satayxvAda yAvAgalU parxkAshavoMdeV
savxBAvavAgiruva I kaNiNxnalilx sAkASxtf neVra nelesiruvadu.
\end{artha}

\vishaya{`tadevxYtatf satayxM' - itAyxdi vAkayxda athaR \ndash }

\begin{shl}
akiSxNX parxkAsharUpeV\s simxnasxvaRM rUpaM parxtiSiThxtamf | \\
baleV pArxNeV parxtiSAThx ca cakuSxSoV\s pi parxdashayxRteV \hfill ||  33 || 
\end{shl}

\begin{artha} 
parxkAsharUpavAda I kaNiNxnalilx elAlx rUpavU niMtiruvudu, 
balavenisida pArxNavasutxvinalUlx kaNiNxna parxtiSeThxyanunx 
(Asharxyavanunx) toVrisuvudu.
\end{artha}

\vishaya{Aditayxnu cakuSxriMdirxya mUlakavAgi pArxNavasutxvinalilx 
nilulxvudeMdu toVrisidAdxyitu. Iga pArxNavanunx iMdarxneMdU 
Aditayxnanunx aginxyeMdU upAsanegoVsakxra viMgaDisi heVLuvudu \ndash }

\begin{shl}
inAdxrXginxV tAvimw sidAdhxvaginxsatxtarx parxkAshakaH | \\
inodxrXV vidharaNaH pArxNaH pUvaRyoVnARmarUpayoVH \hfill ||  34 || 
\end{shl}

\begin{artha} 
A pArxNa, Aditayx eMbavu iMdarx, aginx eMbudAgi sidadhxvAgive. 
avugaLalilx aginxyu parxkAshavuMTumADuvadu, iMdarx eMbudeV pArxNa. adu 
hiMde heVLida nAmarUpagaLige vidhAraka. (aMdare avugaLanunx 
hiDiyabalalxdu)
\end{artha}

\vishaya{aginxyeV parxkAshaka eMbudu heVge?}

\begin{shl}
yAvAnapxrXkAshoV jagati sa savoVR\s ginxrihoVcayxteV  | \\
parisapxnadxshacx savaRtarx pArxNa inadxrXsatxthoVcayxteV \hfill ||  35 || 
\end{shl}

\begin{artha} 
jagatitxnalilx eSuTx parxkAshavideyo, adelalxvU aginxyeMdu ililx 
heVLalapxTiTxde. hAgU elAlx kaDeyalUlx iruva calaneyu pArxNaveMdU 
calanAtamxkavAda pArxNavu (baladeVvateyAda) iMdarxneMdu (dhAyxnakAkxgi 
heVLiruvudu).
\end{artha}

\vishaya{`EvamevxVSA gAyatirxV' - itAyxdi vAkayxda tAtapxyaR \ndash }

\begin{shl}
tirxloVkiV ca tirxveVdiV ca pArxNAditarxyameVva ca | \\
parxtiSiThxtamihAdhAyxtamx EvamuketxVna vatamxRnA \hfill ||  36 || 
\end{shl}

\begin{artha} 
hiVgeye mUruloVkagaLu, mUruveVdagaLu, pArxNAdi mUruvasutxgaLu 
adhAyxtamxdalilx (I shariVradalilx) hiMde heVLidaMte (`neyxSA 
gAyatirxV' itAyxdiyAgi heVLida mAgaRdaMte) parxtiSiThxtavAgive.
\end{artha}

\begin{shl}
pArxNAditwyx hi gAyatirxVsakAtxvuketxVna vatamxRnA | \\
saMjiVvanaM tadaneyxVSAM pArxNAnAM BavatiVshavxrAtf \hfill ||  37 || 
\end{shl}

\begin{shl}
sA heYSeVti shurxtiratoV yathoVkAtxthARvabudadhxyeV  | \\
pArxNAnagxyAnayxtasatxterxV gAyatirxVyamatoV matA \hfill ||  38 || 
\end{shl}

\begin{artha} 
pArxNa, AditayxribabxrU gAyatirxyalilx hiMde heVLida mAgaRdalilx 
seVrikoMDiruvaru. gAyatirxVrUpanAda IshavxraniMda beVre pArxNagaLige 
jiVvanavu Aguvudu. adariMda `sAheYSA' eMba shurxtiyu heVLida 
athaRvanunx tiLisuvudakAkxgiye baMdiruvudu. gaya eMdare pArxNagaLu 
(elAlx iMdirxyagaLu) avugaLanunx rakiSxsutatxdeyAdadxriMda 
gAyatirxyeMdu idu heVLalapxTiTxruvadu.
\end{artha}

\vishaya{`sayAmeVva' itAyxdi vAkayxda tAtapxyARthaR \ndash }

\begin{shl}
upaniVtw sa AcAyaRH pacaCxshAcxdhaRcaRshasatxthA | \\
sAvitirxVM yAmamUM pArxha vaTaveV manatxrXlakaSxNAmf \hfill ||  39 ||
\end{shl}

\begin{shl}
sA\s peyxVSeYva ca vijecnxVyA vAyxKAyxtA yA parxyatanxtaH | \\
yasemxY pArxha sa AcAyaRsatxsayx pArxNAnavatayxsw \hfill ||  40 || 
\end{shl}

\begin{artha} 
upanayana kAladalilx A AcAyaRnu pAdavAgiyU, adhaR QukAkxgiyU, pUNaR 
QukAkxgiyU yAva savitaq deVvateyiMda kUDida I maMtarxrUpavAda 
gAyatirxyanunx vaTuvige upadeVshisuvano, A QukUkx ideV AgiruvudeMdu 
tiLiyabeVku. adanunx parxyatanxdiMda vAyxKAyxnisidAdxyitu. A AcAyaRnu 
yArige upadeVshisuvano avana pArxNagaLanunx (pArxNa 
iMdirxyAdigaLanunx) I gAyatirxyu rakiSxsuvudu.
\end{artha}

\vishaya{upasaMhAra}

\begin{shl}
EvaMvitasxnasx AcAyoVR yasAmx anAvxha sAdaraH | \\
tArxyateV tasayx gAyatirxV vaToVH pArxNAnanx saMshayaH \hfill ||  41 || 
\end{shl}

\begin{artha} 
I riVtiyAgi tiLida (I riVti upAsane mADida) A AcAyaRnu AdaradiMda kUDi 
yArige upadeVshisuvano. A vaTuvina pArxNagaLanunx I gAyatirxyu 
rakiSxsuvadu. adaralilx saMshayavilalx.
\end{artha}

\vishaya{tAmitAyxdi vAkayxda tAtapxyaR \ndash }

\begin{shl}
mANavakasoyxVpanayanasamayeV veVdavAdinAmf | \\
CanadxH parxti vivAdoV\s yaM taninxNiVRtw parA shurxtiH \hfill ||  42 || 
\end{shl}

\begin{artha} 
mANavakana upanayana samayadalilx veVdavAdigaLige CaMdasisxna 
viSayadalilx I vivAdavide. adara niNaRyakAkxgi I muMdina shurxtiyu 
baMdide.
\end{artha}

%~ \section*{baq. a.5, bArx. 14, kaMDike 5}
\kandike{kaMDike 5}
\begin{kandikeshl}
tAM heYtAmeVkeV sAvitirxVmanuSuThxBamanAvxhuvARganuSuTxbeVtadAvxcamanubUrxma iti na tathA kuyARdAgxyatirxVmeVva sAvitirxVmanubUrxyAdayxdi ha vA apeyxVvaMvidabxhivxva parxtigaqhANxti na heYva tadAgxyatArxyX Ekacnacxna padaM parxti || 5 ||
\end{kandikeshl}

\vishaya{vAtiRka}

\begin{shl}
tAM heYtAmiti vAkeyxVna sAvitirxVM parxtidashayxRtAmf | \\
anuSuTxpaCxnadxsaM bUrxyAtUpxvaRpakaSxparxsidadhxyeV \hfill ||  43 || 
\end{shl}

\begin{artha} 
`tAM heYtAmf' eMba vAkayxdiMda sAvitirxVmaMtarxda viSayadalilx 
anuSaTxpf Canadxsasxnunx aMgiVkarisi toVrisabeVkAdadu, EtakAkxgi eMdare? pUvaRpakaSxvu sididhxsuvudakAkxgi (ililx toVrisabeVku).
\end{artha}

\vishaya{meVlina maMtarxda pUvARdhaRvanunx viMgaDisi anavxyisuvaru \ndash }


\begin{shl}
tAmeVtAmeVka AcAyAR upaniVtAya yatanxtaH | \\
\footnote{`tatasxvitu vaqRNiV maheV vayaM deVvasayx BoVjanamf |\\
sherxVSaThxM savaRdhAtamaMturaM Bagasayx dhiVmahiH ||' \\ eMbuva anuSaTxpf CaMdasusxLaLx QugevxVda 
maMtarx, idu savitaq deVvateyuLaLxdAdxdadxriMda sAvitirxyeMdu 
heVLutAtxre.}anuSuTxpaCxnadxsaM pArxhuH sAvitirxVM nAyxyasaMsharxyAtf \hfill ||  44 || 
\end{shl}

%% shloka footnote
\begin{artha} 
A sAvitirxVmaMtarxvanunx kelavu AcAyaRru upaniVtanAda vaTuvige 
yatanxdiMda nAyxyavanunx avalaMbisi anuSaTxpf CaMdasisxnadeMdu 
boVdhisuvaru.
\end{artha}

\vishaya{pUvaRpakaSx adeVnu nAyxya eMdare \ndash }

\begin{shl}
vAganuSuTxbayxtaH sAkASxdAvxkacx sAkASxtasxrasavxtiV | \\
upaniVtAya seYvAtoV vakatxvAyx na tatoV\s parA \hfill ||  45 || 
\end{shl}

\begin{artha} 
I \footnote{`vAgAvx anuSaTxpf' eMba shurxtiyanunx Adharisi I 
AcAyaRru I pakaSxvanunx etitxruvaru. vAkukx sarasavxtiye eMbudu 
parxsidadhxveV AgideyAdadxriMda I anuSaTxpf Qukakxnunx heVLabeVkeMdu 
heVLutAtxre.}vAkukx anuSaTxpf CaMdasisxnadu. adariMda 
shariVradoLagiruva vAkukx sAkASxtf sarasavxtiye. adariMda 
upanayanavAgidadx vaTuvige adanenx heVLikoDabeVku. adu biTuTx 
beVreyilalxveMdu heVLuvaru.
\end{artha}

\vishaya{sidAdhxMta `na tathA' itAyxdiyAgi sidAdhxMta pakaSxvu}

\begin{shl}
yatheYtadukatxmAcAyeYRH kuyARdivxdAvxnanx\footnote{idu tapupx, gAyatirxV CaMdasisxna maMtarx - 
`tatasxvituvaRreVNayxmf' itAyxdi maMtarxveV sAvitirxV savitarx deVvatA 
parxtipAdaka maMtarx. ideV vaTuvige upadeVshisabeVkAda maMtarxvu.} tatatxthA | \\
gAyatirxVmeVva sAvitirxVM bUrxyAtasxvaRPalApitxtaH \hfill ||  46 || 
\end{shl}

%% shloka footnote
\begin{artha} 
I AcAyaRriMda heVge heVLalapxTiTxto hAgeye tiLidavanu heVLikoDabAradu. 
matetxVneMdare? gAyatirxyanenx (gAyatirxV CaMdasisxna maMtarxvanenx) 
sAvitirxyeMdu \footnote{pArxNavanunx gAyatirxyeMdu heVLide. 
pArxNavanunx heVLida meVle vAkakxnunx sarasavxtiyanunx beVre 
iMdirxyagaLeMba pArxNavanunx elalxvanunx AcAyaRnunx mANavakanige 
opipxsida hAgAguvadu. adariMda gAyatirxyanenx sAvitirxyeMdu 
tiLiyabeVku.}samasatx PalavU laBisuvadariMda 
heVLikoDabeVku.
\end{artha}

\vishaya{savaRPalapArxpitx heVge?}

\begin{shl}
yathoVkAtxyAM hi gAyatArxyXM kaqtasxrXM jagadupAhitamf | \\
tadukwtx savaRmukatxM sAyxdayxtupxmathARya sAdhanamf \hfill ||  47 || 
\end{shl}

\begin{artha} 
hiMde heVLida gAyatirxyalelx elAlx jagatutx nelaside. adanunx 
heVLuvalilx elalxvanunx heVLidaMtAguvudu. yAvudu puruSAthaRkekx 
sAdhanavAgideyo (adelalxvU gAyatirxyalilxde)
\end{artha}

\vishaya{yadi havA itAyxdi vAkayxda tAtapxyaR \ndash }

\begin{shl}
vijAcnxnapuruSaseyxVdaM savxBAvAdeVva savaRdA | \\
AtemxYva hi jagatakxqqtasxnXM sAdhAraNavisheVSavatf \hfill ||  48 ||
\end{shl}

\begin{artha} 
vijAcnxnarUpanAda puruSanige (jiVvAtamxnige) yAvAgalU I elAlx 
jagatutx. sAmAnayx visheVSavuLaLxvelalxvU savxBAvavAgiye AtamxveV 
Agiruvudu.
\end{artha}

\vishaya{jiVvanige samaSiTx vayxSiTxrUpagaLu elilxMda? baMdavu.}

\begin{shl}
sAdhAraNAni vasUtxni tathA\s sAdhAraNAnayxpi | \\
nAnupAdAya kaqsAnxni janotxVH kAcitikxrXyeVSayxteV \hfill ||  49 || 
\end{shl}

\begin{artha} 
sAdhAraNa vasutxgaLanunx asAdhAraNa vasutxgaLanUnx elalxvanunx 
sivxVkarisade pArxNige yAvudoMdu kirxyeyU naDeyuvudilalx\footnote{}.
\end{artha}

\vishaya{upAsaneya naMtara samaSiTx vayxSiTxrUpagaLu baMdarU modalu 
heVge? iruvavu? eMdare \ndash }


\begin{shl}
\footnote{idanunx madhubArxhamxNadalilx noVDabahudu. samaSiTx 
vayxSiTx savxrUpavu EkAneVkarUpavu dhAyxnada baladiMda ciMtisida 
rUpavU oMdu riVtiyAgi modalu toVruvudu. sAkASxtAkxradiMda muMde 
vayxkatxvAguvAga adu ati sapxSaTxvAguvudu. mUtiRya sAkASxtAkxrakekx 
modalu dhAyxnakAladalilx mUtiRya AkAravu hoLeyuvaMte. modaleV idu 
itetxMdu aBipArxyavu.}aBivayxketxVH purA\s peyxVtadUrxpamAsiVtasxvXBAvataH | \\
aBivayxkwtx tu tatAsxkASxtasxmaSiTxvayxSiTxlakaSxNamf \hfill ||  50 || 
\end{shl}

%% shloka footnote
\begin{artha} 
aBivayxkitxyAguva muMceyU savxBAvavAgi I rUpavu iruvudu. 
aBivayxkitxyAda meVle samaSiTx vayxSiTxrUpagaLu avu sAkASxtf 
toVruvavu.
\end{artha}

\begin{shl}
EvaM sidedhxV mahimanxyXsimxnayxthoVketxVneYva vatamxRnA | \\
kaniVyasAtx vivaqdidhxvAR neYva saMBAvayxteV miteVH \hfill ||  51 || 
\end{shl}

\begin{artha} 
I\footnote{samasatx ceVtanagaLigU savxBAvavAgiye samaSiTx 
vayxSiTxrUpavu ideyeMbudanunx hiMde heVLida nAyxyadaMte sidadhxvAgalu 
gAyatirxV upAsakanige parxmANadiMda hArxsavaqdidhxgaLu 
baruvudilalxvAdadxriMda aneVka parxtigarxha mADidarU doVSavilalxveMdu 
aBipArxya.} riVtiyAda I mahimeyu sidadhxvAgiralu hiMde heVLida 
mAgaRdalilx hAni athavA vaqdidhxyu parxmANadiMda saMBavisuvaMtilalx.
\end{artha}

\vishaya{sAmAnayxvAgi gAyatirxV upAsakanige Palavanunx heVLi 
visheVSavAgiyU heVLalu AraMBisidAdxre \ndash }

\begin{shl}
UriVkaqteyxVmameVvAthaRM sa ya itAyxdinoVcayxteV | \\
mahAparxtigarxheVNApi neYvaMvidodxVSamaqcaCxti \hfill ||  52 || 
\end{shl}

\begin{artha} 
ideV athaR (Atamxnige savxtaH vaqdidhx hAnigaLilalxveMbudanunx) 
aMgiVkarisi `sa ya' itAyxdi vAkayxdiMda doDaDx parxtigarxha 
mADuvudariMdalU I riVtiyAgi gAyatirxV upAsane mADidavanu doVSavanunx 
hoMduvudilalx. 
\end{artha}

%~ \section*{baq. a.5, bArx. 14, kaMDike 6}
\kandike{kaMDike 6}
\begin{kandikeshl}
sa ya imAMsitxrXVMlolxVkAnUpxNARnapxrXtigaqhiNxVyAtosxV\s sAyx EtatapxrXthamaM padamApunxyAdatha yAvatiVyaM tarxyiV vidAyx yasAtxvatapxrXtigaqhiNxVyAtosxV\s sAyx EtadidxvXtiVyaM padamApunxyAdatha yAvadidaM pArxNi yasAtxvatapxrXtigaqhiNxVyAtosxV\s sAyx EtatatxqqtiVyaM padamApunxyAdathAsAyx EtadeVva turiVyaM dashaRtaM padaM paroVrajA ya ESa tapati neYva keVnacanApayxM kuta u EtAvatapxrXtigaqhiNxVyAtf || 6 ||
\end{kandikeshl}

\begin{shl}
dashaRnasayx sutxtiriyaM sa ya itAyxdinoVcayxteV | \\
tAdaqkapxrXtigarxhaseyxVha na kavxcitasxMBavoV yataH \hfill ||  53 || 
\end{shl}

\begin{artha} 
`sa ya' itAyxdi vAkayxdiMda upAsaneya sutxtiye ideMdu heVLalapxTiTxde. 
kAraNaveVneMdare? aMtaha parxtigarxhaveMbudu ihadalilx elilxyU 
(yAralUlx) saMBavisuvudilalx.
\end{artha}

\vishaya{beVre tAtapxyaRvU ide \ndash }

\begin{shl}
parxtigarxhasayx ninAdx vA vidavxnAmxnAtapxrXsakitxtaH | \\
niHsheVSapuNayxmoVSitAvxninxSeVdhAthARya kutasxyXteV \hfill ||  54 || 
\end{shl}

\begin{artha} 
\footnote{nAnu vidAvxMsanu, idara baladiMda nanage baruva 
doVSavelalx BasamxvAguvudeMba duraBimAnadiMda duSaTx 
parxtigarxhavanunx mADabahudu. Itanu vidAvxMsaneMbuva bahu 
gawravadiMda avanige aneVka dAnagaLanunx koDabahudu. Adare I 
parxtigarxha doVSavu barxhamxsAkASxtAkxravilalxde hoVguvudilalx. 
sakala doVSagaLanunx pariharisuvudu jAcnxnavoMde beVre yAvudU ilalx. 
adara nepadiMda parxtigarxhavanunx niMdisuvudu. mUru loVkadaSuTx 
aparimita darxvayx parxtigarxhadiMda baruva suKaBoVgavu gAyatirxVpAda 
mAtarxvanunx upAsane mADida Palakekx samAnavalalx. adariMda 
parxtigarxhavu sherxVyasakxravalalx. parxtigarxhavu namamx 
puNayxvanenx kaLeyuvudu.}athavA ililx parxtigarxhavanunx niMdisuvudu tAnu 
vidAvxnf eMba aBimAnadiMda tanage asatf parxtigarxhavU 
saMBavisuvudariMda adu samasatx puNayxgaLanUnx kaLeyutatxdeyAda kAraNa 
adanunx niSeVdhisuvudakAkxgi parxtigarxhavanunx niMdisuvudu.
\end{artha}

\begin{shl}
ukatxpAdeVSavxpi jAcnxnaM neYvAlamapabAdhitumf | \\
gAyatirxVveVdinoV janotxVrapi BUyAnapxrXtigarxhaH \hfill ||  55 || 
\end{shl}

\begin{artha} 
gAyatirxge heVLida pAdagaLalUlx mADuva jAcnxnavanunx (upAsaneyanunx) 
gAyatirxV upAsakanAdavanu pArxNisAmAnayxnu bahuvAgi parxtigarxha 
mADidudx saha keDisuvudakekx samathaRvAgilalx.
\end{artha}

\vishaya{`kuta' itAyxdi vAkayxda tAtapxyaR}

\begin{shl}
neYvaM dAtA jagatayxsimxnanx ca tAdaqkapxrXtigarxhaH | \\
garxhiVtA veVha neYveVti kuta itAyxdinoVcayxteV \hfill ||  56 ||
\end{shl}

\begin{artha} 
I riVtiyAgi dAnakoDuvavanU ilalx. aMtaha parxtigarxhavU ilalx. 
tegedukoLuLxvavanU ilalxveMdeV `kuta' itAyxdi vAkayxdiMda 
heVLalapxDuvadu.
\end{artha}

\vishaya{hAgAdare `saya imAnf' itAyxdi vAkAyxthaRvu heVge yukatx?}

\begin{shl}
apayxBuyxpagameV ceYSAM neYva doVSasayx saMBavaH | \\
viduSoV\s satiVtayxtaH pArxha sa ya itAyxdinA shurxtiH \hfill ||  57 || 
\end{shl}

\begin{artha} 
I dAnAdigaLanunx opipxkoMDarU saha upAsakanige yAva doVSavU 
saMBavisuvudilalx. adariMda `saya imA' itAyxdi shurxtiyu heVLiruvadu.
\end{artha}

\vishaya{I shurxtiyanunx Iga vAyxKAyxna mADuvudu \ndash }

\begin{shl}
sa yaH kashicxdimAlolxVkAnUpxNARnupxMBoVgasAdhaneYH | \\
vidAvxnasxMparxtigaqhiNxVyAtakxthacitAkxmaMpulxteVH \hfill ||  58 || 
\end{shl}

\begin{shl}
AdayxpAdaparijAcnxnamAtarxseyxYva parxtigarxhaH  | \\
kaSxyAyAlaM na sheVSasayx gAyatirxVdashaRnasayx saH \hfill ||  59 || 
\end{shl}

\begin{artha} 
yAvanAdarobabxnu I upAsakanu taqSeNxya upadarxvadiMda puruSana BoVga 
sAdhana vasutxgaLiMda tuMbiruva I loVkagaLanunx parxtigarxha mADali. 
avaniMda gAyatirxya oMdanepAda mAtarx parijAcnxna mADikoMDa Palave 
anuBava mADalapxTaTxMte Aguvudu. uLida gAyatirxV dashaRnada Palavanunx 
nAshagoLisuvudakekx A parxtigarxhavu kAraNavAguvudilalx.
\end{artha}

\vishaya{`atha yAvatiVyaM tarxyiV vidAyx yaHsAtxvatf parxtigaqNiVhxyAtf' itAyxdi maMtarxda vAyxKAyxnavu \ndash }

\begin{shl}
yAvatiVyaM tarxyiV vidAyx tAvanatxmapi gaqhaNxtaH | \\
divxtiVyapadavijAcnxnakaSxtireVva na savaRtaH \hfill ||  60 || 
\end{shl}

\begin{artha} 
I tarxyiV videyxyu eSiTxdeyoV aSaTxnUnx parxtigarxha mADuvavanige 
eraDane pAdada jAcnxnakekx mAtarx hAniyAguvudu. Adare savaRtarx 
hAniyAgadu.
\end{artha}

\begin{shl}
asaMBaveV\s pi kalepxyXVta yadi tAdaqkapxrXtigarxhaH | \\
terxYvidayxlakaSxNaH soV\s pi taqtiVyaM nA\s \s punxyAtapxdamf \hfill ||  61 || 
\end{shl}

\begin{artha} 
saMBavisadidadxrU aMtaha tarxyiVvidAyx saMbaMdha parxtigarxhavanunx 
kalipxsidalilx AgalU parxtigarxhavu mUrane pAdavanunx paDeyuvudilalx.
\end{artha}

\vishaya{`AthAsAyx EtadeVva turiVyaMdashaRtaM padamf' eMbudara tAtapxyaR \ndash }

\begin{shl}
anatxraNaDxviBakatxsayx yathoVketxYH sAyxtapxrXtigarxheYH | \\
kaSxyoV nAnanatxrUpasayx sAyxtasxmaSiTxvapuBaqRtaH \hfill||  62 || 
\end{shl}

\begin{artha} 
barxhAmxMDadoLage viBAga hoMdiruva vasutxgaLanunx hiMde heVLidaMte 
parxtigarxha mADuvudariMda samaSiTx shariVravanunx dharisiruva anaMta 
savxrUpanAda gAyatirxV upAsakanige yAva hAniyU ilalx.
\end{artha}

\vishaya{anaMta savxrUpakekx nAshavilalxveMbudakekx inonxMdu kAraNa}

\begin{shl}
pariciCxnenxVna savaRtarx parxtimAnaM jagatayxpi | \\
anatxvadivxSayaM daqSaTxM na tavxnanatxsayx kutarxcitf \hfill||  63 || 
\end{shl}

\begin{artha} 
elAlx jagatitxnalUlx parxtiyoMdu parxmANadalUlx pariciCxnanxdiMda 
aMdare alapxteyiMda vinAshiyAda viSayavuLaLx (upAsaneyu) kaMDide, 
Adare anaMta savxrUpakekx (catuthaRpAda jAcnxna Palakekx) elilxyU 
nAshaviruvudu kaMDilalx.
\end{artha}

\vishaya{barxhamxjAcnxnada PalakikxMta beVreyAdadudx anaMtaveMbudu heVge?}

\begin{shl}
ayaM cAnanatxmAtAmxnamakiSxsUyaRvayxvasithxtamf  | \\
agAdupAsanAtApxrXNamAtamxtevxVna divAnishamf \hfill||  64 || 
\end{shl}

\begin{shl}
saMBAvayxteV kaSxyasatxsayx na kutashicxdananatxtaH | \\
anatxvAnikxSXVyateV loVkeV na tavxnanatxH kutashacxna \hfill||  65 || 
\end{shl}

\begin{artha} 
hagalUrAtirx I upAsakanu kaNuNx matutx sUyaRnalilxruva anaMta 
AtamxvAda pArxNavanunx (sUtArxtamxnanunx) upAsaneyiMda AtamxveMdeV 
BAvisi tAdAtamxyXvanunx hoMdidadxnu. adariMda avanige anaMtatavxvu 
(apariciCxnanxteyu) iruvudariMda yAvudariMdalU nAshaviruvudilalx. 
aMtavuLaLxdudx loVkadalilx kaSxyisuvudu aMtavilalxdudx yAvudariMdalU 
kaSxyisuvudilalx.
\end{artha}

\begin{shl}
AvaqtitxH kaSxyashabedxVna duHKapArxyAsu BUmiSu | \\
keYvalAyxvasiteVnARsw sAyxtasxmaSiTxvapuBaqRtaH \hfill||  66 || 
\end{shl}

\begin{artha} 
kaSxyaveMba shabadxdiMda duHKaveV tuMbiruva sAthxnagaLalilx Avaqtitx 
hiMdakekx baruvikeyeMdathaR. keYvalayxve koneyAgiruvudariMda samaSiTx 
(Eka) rUpavanunx dharisuvavanige I Avaqtitxyu iruvudilalx.
\end{artha}

%~ \section*{baq. a.5, bArx. 14, kaMDike 7}
\kandike{kaMDike 7}
\begin{kandikeshl}
tasAyx upasAthxnaM gAyatarxyXseyxVkapadiV divxpadiV tirxpadiV catuSapxdayxpadasi na hi padayxseV ||namasetxV turiVyAya dashaRtAya padAya paroVrajaseV\s sAvadoV mA pArxpaditi yaM divxSAyxdasAvasemxY kAmoV mA samaqdidhxVti vA na heYvAsemxY sa kAmaH samaqdhayxteV yasAmx EvamupatiSaThxteV\s hamadaH pArxpamiti vA || 7 ||
\end{kandikeshl}

\begin{shl}
upasAthxnaM yathoVkAtxyA gAyatArxyXH sharxdadhxyA\s nivxtaH | \\
gAyatarxyXsiVtimanetxrXVNa kuyARdukAtxthaRvitasxdA \hfill||  67 || 
\end{shl}

\begin{artha} 
hiMde heVLida athaRvanunx tiLidavanu sharxdedhxyiMda kUDi `gAyatirx 
asi' eMba maMtarxdiMda gAyatirxya upasAthxnavanunx yAvAgalU mADabeVku.
\end{artha}

\begin{shl}
EkadivxtirxpadayxsiVti pUvARnApxdAnivxcinatxyeVtf  | \\
catuSApxtatxvXM ca tuyeVRNa yathoVketxVna vicinatxyeVtf \hfill||  68 || 
\end{shl}

\begin{artha} 
EkapadiV divxpadiV tirxpadiV eMdu modalina pAdagaLanUnx ciMtisabeVku. 
nAlakxneV pAdadiMda catuSATxtf eMdu ciMtisabeVku.
\end{artha}

\begin{shl}
apadasiVtayxpi girA tasAyx AnanatxyXmucayxteV  | \\
avayxyA cAkaSxyA\s siVti na hayxnatxsetxV\s dhigamayxteV\hfill ||  69 || 
\end{shl}

\begin{artha} 
`apadasi' eMba padadiMda gAyatirxya AnaMtayxvanunx 
(vAyxpakatavxvanunx) heVLide. avayxyaLU akaSxyaLU niVneV Agididx. 
ninage aMtaveMbudu kANuvudilalx. `namaseVtx\s sutx turiVyAya' eMba 
\end{artha}

\begin{shl}
namasetxV\s sutx turiVyAyeVtuyxkAtxyX tuyaRparxdhAnatAmf | \\
guNaBAvaM yathoVkatxsayx vidAyxtApxdatarxyasayx tu \hfill||  70 || 
\end{shl}

\begin{artha} 
ukitxyiMda turiVyAthaR pArxdhAnayxvanunx hiMde heVLida mUru pAdagaLa 
apArxdhAnayxvanunx heVLide. `asaw' eMba padavu ililx
\end{artha}

\begin{shl}
shaturxnAmagaqhiVtayxthaRmasAvitipadaM tivxha | \\
PaloVkitxrada iteyxVtadupAsiturahaM tathA \hfill||  71 || 
\end{shl}

\begin{artha} 
shaturxvina nAmavanunx garxhisuvudakAkxgi, adaH eMbudu Palashurxti 
idanunx upAsane mADuvavanige. hAgeye ahamf eMbudu.
\end{artha}

\vishaya{`asAvadoVmApArxpatf kAmoV\s semxY mA samaqdidhx aha madaH pArxpayf' eMba maMtarxgaLanunx anavxyisutAtxre \ndash }

\begin{shl}
shaturxNA kAmitoV yoV\s thoVR mA pArxpatatxmasAviti | \\
mA samaqdadhxyXthavA soV\s semxY taM vA\s haM pArxpunxyAmiti \hfill||  72 || 
\end{shl}

\begin{artha} 
yAva iSATxthaRvanunx shaturxvu bayasiruvano, adanunx I shaturxvu 
paDeyadirali. athavA avanu apeVkiSxsidudx ivanige vaqdidhx 
hoMdadirali. athavA shaturxvu apeVkiSxsidadxnunx nAneV hoMduvenu.
\end{artha}

\vishaya{oMdeV upasAthxnakekx eraDu virudadhxvAda PalagaLu heVge? eMdare \ndash }

\begin{shl}
aBicArAthaRmeVtasimxnunxpasAthxneV yathoVditeV | \\
\footnote{aBidhAna eMdare mUru maMtarxgaLu - 1. asaw, 2. ado, 3. 
mApArxpatf - eMdu ililx iTuTxkoLaLxbeVku. icACxnusAravAgi vikalapxvAgi 
iTuTxkoLaLxbahudu.}aBidhAneYkadeVshoVkAtxyX vikalapxH PalagoVcaraH \hfill||  73 || 
\end{shl}

%% shloka footnote
\begin{artha} 
aBicArakAkxgi aMdare shaturxvige toMdare mADuvudakAkxgi I 
upasAthxnavanunx heVLida naMtara mUru maMtarxgaLalilx EkadeVshavanunx 
heVLidadxriMda Pala viSayadalilx vikalapxvu (eraDAgiruvudu).
\end{artha}

\vishaya{tanagAgi upasAthxna mADuvalilx I Palavu \ndash }

\begin{shl}
sAvxtheVR tavxhamada iti huyxpasAthxneY PalaM BaveVtf | \\
adaH parxyoVjanaM deVvi pArxpunxyAM tavxtapxrXsAdataH \hfill||  74 || 
\end{shl}

\begin{artha} 
sAvxthaRkAkxgi `aha mada' eMdu upasAthxna mADidalilx maMtorxVkatxvAda 
Palavu Agiye Aguvudu. adaH= I parxyoVjanavanunx deVvi=eleY deVviye! 
ninanx anugarxhadiMda pArxpunxyAM= paDeyuvenu.
\end{artha}

\vishaya{sArAMsha}

\begin{shl}
EtAvadeVva kiM jecnxVyaM kiMvA\s nayxdapi shiSayxteV | \\
asatxyXnayxdapi vijecnxVyaM tadivxnA\s kaqtasxnXtA yataH \hfill||  75 || 
\end{shl}

\begin{artha} 
iSeTxV tiLiyabeVkAdadedx? athavA beVre oMdeVnAdarU uLidideye? eMdu 
keVLidalilx utatxra - beVre oMdu tiLiyabeVkAdadUdx ide. adilalxde 
pUNaRvAguvudilalx.
\end{artha}

\begin{shl}
kAtesxnXyXVRna vidAyx savxBayxsAtx PalAyAlamupAsituH | \\
vipayaRyeVNAnathARya tadeVtatapxrXtipAdayxteV \hfill||  76 ||
\end{shl}

\begin{artha} 
videyxyanunx pUNaRvAgi cenAnxgi aBAyxsa mADidadxlilx upAsakanige ade 
Palavanunx koDalu shakatxvAgiruvudu. idakekx vipariVtavAgiruvalilx adu 
anathaRkekx kAraNavAguvudeMbudanunx muMde parxtipAdisuvudu.
\end{artha}

\begin{shl}
savARtamxkatAvxdAgxyatArxyX agenxVrapi parigarxhaH | \\
tanumxKatevxVnAsaMsidedhxVsAtxdatheyxVRnoVtatxrA shurxtiH \hfill||  77 || 
\end{shl}

\begin{artha} 
gAyatirxyu savARtamxkavAdadxriMda aginxyanunx sivxVkarisidaMtAgide. 
AdarU gAyatirxya muKaveMdu aginxyu tiLidilalxvAdadxriMda adanunx 
vidhisuvudakAkxgi muMdina shurxtiyu baMdide\\
\begin{shl}
Etadadhx veY tajajxnakoV veYdeVhoV buDilamAshavxtarAshivxmuvAca yanunx hoV tadAgxyatirxVvidfbUrxthA atha kathaM hasitxVBUtoV vahasiVti muKaM hayxsAyxH samArxNanx vidAcnacxkAreVti hoVvAca tasAyx aginxreVva muKaM yadi ha vA api bahivxvAgAnxvaBAyxdadhati savaRmeVva tatasxnadxhateyxVvaM heYveYvaMvidayxdayxpi bahivxva pApaM kuruteV savaRmeVva tatasxMpAsxya shudadhxH pUtoV\s jaroV\s maqtaH samaBxvati || 8 ||
\end{shl}
\end{artha}

\vishaya{aginxyu gAyatirxya muKavAgidadxre `hasitxV BUtoV vahasi' eMdu 
heVLidudx heVge? \ndash }

\begin{shl}
muKavijAcnxnavirahAdanathaRPalakiVtaRnamf | \\
aginxreVva muKaM tasAyx ituyxkAtxyX kaqtasxnXtoVcayxteV \hfill||  78 || 
\end{shl}

\begin{shl}
gAyatirxVdashaRnaseyxVha PalakAtasxRnXyXM tatheYva ca | \\
yathoVkAtxginxmuKajAcnxnAdayxdiVtAyxdigirA\s dhunA \hfill||  79 || 
\end{shl}

\begin{artha} 
muKajAcnxnavilalxdiruvudariMda anathaR Palavanunx heVLiruvudu. A 
gAyatirxge aginxyeV muKa eMdu heVLidadxriMda gAyatirxV daqSiTxyalilx 
pUNaRteyanunx heVLida hAgAyitu. gAyatirxV dashaRnakekx pUNaRPalavu 
ililx hiMde heVLida aginx muKajAcnxnadiMda AguvudeMdu yadi itAyxdi 
vacanadiMda Iga heVLuvudu.
\end{artha}

\vishaya{`yadihavA api bahivxvAgwnx aBAyxsadhati savaRmeVva tatasxMdahateyxVva' itAyxdi Palashurxtiyanunx yoVcisutAtxre \ndash }

\begin{shl}
yoV veVdAginxmuKAmeVtAM gAyatirxVmaginxreVva saH | \\
aginxrinadhxnavatasxvaRM daheVdivxdAvxnapxrXtigarxhamf \hfill||  80 || 
\end{shl}

\begin{artha} 
yAru aginxyeV muKavAgiruva I gAyatirxyanunx tiLiyuvano, upAsane 
mADuvano avanu aginxyeV Aguvanu. adariMda beMkiyu kaTiTxgeyanunx 
suTuTx urisuvaMte elalx parxtigarxha doVSavanunx suTuTx urisuvanu.
\end{artha}

\vishaya{`api bahivxva pApaM' - itAyxdi shurxtiyalilx iva shabadxda 
athaR \ndash }

\begin{shl}
dAhakasayx na bahavxsitx dAhayxyoVgABivaqdidhxtaH | \\
kaSxyashAcxsheVSadAhayxsayx yasAmxtatxsAmxdiveVtigiVH \hfill||  81 || 
\end{shl}

\begin{artha} 
suDuva beMkige iMdhanavu bahaLavAgiruvudilalx. Adare beMkiyu 
dAhayxvAda (iMdhana) saMbaMdhadiMda hecucxvudariMda samasatx 
dAhayxvasutxgaLigU kaSxyaveV Aguvudu. idariMda iva eMba padavu 
(bahuvAgilalx, bahuvAgiruvaMte eMba athaRdalilxruvudu)
\end{artha}

\section*{baq. a.5, bArx. 14, kaMDike 14}

\begin{shl}
EvaMvideVvameVva sAyxtApxpaM vahinxvadinadhxnamf | \\
savaRM saMBakaSxyX tatApxpaM shudadhxH pApaviyoVgataH \hfill||  82 ||
\end{shl}

\begin{artha} 
gAyatirxV upAsakanU ideV riVtiyAgi beMkiyu iMdhanavanunx saMpUNaR 
suDuvaMte samasatx pApavanunx tiMduhAki (nAshagoLisi) pApa 
leVpavilalxde shudadhxnAguvanu.
\end{artha}

\vishaya{`pUtoV\s jaroV\s maqtaH saMBavati' - eMbudara athaR \ndash }


\begin{shl}
\footnote{hiMde pApa saMbaMdhavilalxdadxriMda shudadhxnAguvuneMdu 
heVLide. Iga pApaPala saMbaMdhavilalxdadxriMda pUtateyanunx 
heVLideyAdadxriMda punarukitxyilalx. Atamxnige yAva pariNAmavU 
ilalxvAdadxriMda ajara, amaqta, eMdu heVLide. 
sUthxladeVhavilalxdariMdalU amaqta, ajara, amara eMdu tiLiyabeVku. 
gAyatirxV upAsakanu I gAyatirxyeMbudanunx parxjApati sUtArxtamx eMbuva 
samaSiTx pArxNarUpadalilx upAsane mADidadxriMda pArxNa mAtarx 
savxrUpavuLaLxvanAgiruvanu. aginxmuKavAgiruva gAyatirxyanunx upAsane 
mADiruvudariMda Itanu aginxyaMte savaRpApavanunx dahisikoMDu muMdeMdU 
pApavanunx mADade shudadhxnAda parxjApatibarxhamx 
savxrUpavuLaLxvanAguvaneMdu tAtapxyaR.}pUtoV\s saMsagiRdhamaRtAvxdajaroV\s pariNAmavAnf | \\
amaqtoV\s sUthxladeVhatAvxtApxrXNamAtarxsavxBAvataH \hfill||  83 || 
\end{shl}

%% shloka footnote
\begin{artha} 
inonxMdaroDane seVradiruva savxBAvavuLaLxdAdxdadxriMdaleV Atamxnu 
pavitarxnAgiruvanu, pariNAmavilalxdavanu. adariMda mupipxlalxdavanu, 
matutx amaqtanu, sUthxla shariVra saMbaMdhavilalxdadxriMda amaqtanu. 
pArxNamAtarx savxBAvavAgiruvavanAdadxriMda sUthxlashariVrarahitanu.
\end{artha}

aidane adhAyxyadalilx hadinAlakxne bArxhamxNavu 
\footnote{vAtiRka garxMthadaMte 16ne bArxhamxNaveMta baredide. 
vAtiRkadalilx bArxhamxNa viBajane BASayxdalilxruva viBajanegiMta 
vayxtAyxsagoMDide. BASayxdaMte nAvu ililx 14ne bArxhamxNaveMta 
baredidedxVve. idanunx paMDitaru shoVdhisabeVkAdududx vAtiRkadaMte 
hadinAru bArxhamxNagaLeMdu viBAga mADikoLaLxbeVku.\\ 
\begin{tabular}{crccl}bArxhamxNa & 1 & vAtiRka & sholxVka & 1\ndash 130 
varege\\
" & 2 & " & " & 1\ndash 3 \qquad "\\
" & 3 & " & " & 1\ndash 5 \qquad "\\ 
" & 4 & " & " & 1\ndash 5 \qquad "\\
" & 5 & " & " & 1\ndash 14 \quad \,"\\ 
" & 6 & " & " & 1\ndash 6 \qquad "\\
" & 7 & " & " & 1\ndash 3 \qquad "\\
" & 8 & " & " & 1\ndash 4 \qquad "\\
" & 9 & " & " & 1\ndash 3 \qquad "\\
" & 10 & " & " & 1\ndash 7 \qquad "\\
" & 11 & " & " & 1\ndash 4 \qquad "\\
" & 12 & " & " & 1\ndash 6 \qquad "\\
" & 13 & " & " & 1\ndash 4 \qquad "\\
" & 14 & " & " & 1\ndash 15 \quad \,"\\
" & 15 & " & " & 1\ndash 9 \qquad "\\
" & 16 & " & " & 1\ndash 83 \quad \,"\end{tabular}}(hadinArane bArxhamxNavu) mugidide.

\eject
