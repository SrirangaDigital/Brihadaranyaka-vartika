%%%% From 053.tex
\chapter{bArxhamxNa - 12}
%~ \section*{bArxhamxNa 12}

\vishaya{punaH parxkaqta barxhomxVpAsanegaLa visatxraNe}

\begin{shl}
ananxM barxhemxVti nivaRkitx pureVvAnayxdupAsanamf | \\
udAraPalasidadhxyXthaRM tasayx nAyxyeVna niNaRyaH \hfill ||  1 || 
\end{shl}

\begin{artha}
ananxM barxhamx eMbudAgi hiMdinaMte barxhamxna beVre oMdu upAsaneyanunx udAravAda Palavu sididhxsuvudakAkxgi heVLuvudu, adanunx nAyxyadiMda niNaRya mADiruvudu.
\end{artha}

%~ \section*{baq. 5, bArx. 12, kaMDike 1}
\kandike{kaMDike 1}
\begin{kandikeshl}
ananxM barxhemxVteyxVka Ahusatxnanx tathA pUyati vA ananxmaqteV pArxNAtApxrXNoV barxhemxVteyxVka Ahusatxnanx tathA shuSayxti veY pArxNa QuteV\s nAnxdeVteV ha tevxVva deVvateV EkadhABUyaM BUtAvx paramatAM gacaCxtasatxdadhx sAmxha pArxtaqdaH pitaraM kiMsivxdeVveYvaM viduSeV sAdhu kuyARM kimeVvAsAmx asAdhu kuyARmiti sa ha sAmxha pANinA mA pArxtaqda kasetxvXVnayoVreVkadhABUyaM BUtAvx paramatAM gacaCxtiVti tasAmx u heYtaduvAca viVtayxnanxM veY vayxnenxV hiVmAni savARNi BUtAni viSATxni ramiti pArxNoV veY raM pArxNeV hiVmAni savARNi BUtAni ramanetxV savARNi ha vA asimxnUBxtAni vishanitx savARNi BUtAni ramanetxV ya EvaM veVda || 1 ||
\end{kandikeshl}

\vishaya{vAtiRka}

\begin{shl}
ananxmeVva paraM barxhemxVteyxVka AhuviRpashicxtaH | \\
tatatxthA nAvaganatxvayxM kilxdayxtayxnanxmasuM vinA \hfill ||  2 ||
\end{shl}

\begin{shl}
pArxNoV barxhemxVti cApayxneyxV tacacx gArxhayxM pureVva na | \\
anAnxdaqteV yataH pArxNaH shoVSamAshu nigacaCxti \hfill ||  3 || 
\end{shl}

\begin{artha}
kelavu vidAvxMsaru `ananxveV parabarxhamx'veMdu heVLuvaru, adanunx hAge tiLiya takakxdadxlalx, EkeMdare ? ananx pArxNavilalxde dugaRMdhavAguvudu adariMda pArxNaveV barxhamxveMdu beVre vidAvxMsaru heVLuvaru, hiMdinaMte adU namage gArxhayxvalalx, kAraNaveVneMdare ? ananxvilalxde pArxNavu shiVGarxvAgi shoVSaNeyanunx hoMduvudu.
\end{artha}

\begin{shl}
barxhamxtavxM nAnayoVyaRsAmxdeVkeYkasayx na yukitxmatf | \\
saMBUya deVvateV tasAmxdabxrXhamxtavxM saMnigacaCxtaH \hfill ||  4 || 
\end{shl}

\begin{artha}
yAvudariMda I ananxpArxNagaLalilx oMdoMdakUkx barxhamx savxrUpavu iruvudeMbudu yukatxyukatxvAgilalxvo adariMda eraDu deVvategaLU seVri barxhamxsavxrUpavanunx tALuvavu.
\end{artha}

\begin{shl}
tadeVtatasxMparxdhAyAR\s \s ha pArxtaqdaH pitaraM kila | \\
kiMsivxditAyxdi saMhaqSaTxsatxdodxVSasAyxsamiVkaSxNAtf \hfill ||  5 || 
\end{shl}

\begin{artha}
adeV idu eMdu aMdare ananx-pArxNagaLeMba eraDu upAdhigaLiMda kUDidudx barxhamxveMdu nidhaRrisi pArxtaqdaneMbuvanu tananx taMdeyanunx kuritu heVLidanu, kAraNaveVneMdare :- guNahiVnavAda barxhamxvasutxvanunx upAsane mADuvalilx doVSavanunx kANadiruvudariMda saMtoVSapaTuTx `kiM sivxdeVva'... itAyxdiyAgi heVLidanu.
\end{artha}

\begin{shl}
kiMsivxdeVvaMvideV sAdhu kaqtasxnXsAdhAvxpitxheVtutaH | \\
kuyARmasAdhu vA tasemxY savARsAdhunirAkaqtoVH \hfill ||  6 || 
\end{shl}

\begin{artha}
nAnu heVge barxhamxvanunx kalipxsikoMDiruveno, hAge tiLida jAcnxnige samasatx shuBavU laBisuvudakekx kAraNavAdadxriMda yAva oLeLxya pUjeyanunx nAnu mADali ? hiVgeye samasatx pApagaLanunx nivArisuvudariMda aMtaha jAcnxnigAgi yAva pApavanunx mADali\footnote{ananx pArxNagaLanunx oTuTxgUDi barxhamxveMdu tiLidavanu pApAcaraNeyiMda keTuTx hoVguvudU ilalx, puNAyxcaraNeyiMda hecicxsalapxDuvudU ilalx. aMdare savaRriVtiyalUlx kaqtAthaRnAguvaneMdu putarxnu tAnu tiLida barxhamxviSayadalilx kaMDa mahimeyanunx opipxsidanu.} ?
\end{artha}

\vishaya{``sahasAmx ha pANinA'' itAyxdi maMtarxda vAyxKAyxna \mdash }

\footnotetext{guNagaLilalxda ananx, pArxNagaLalilx parxteyxVkavAgi oMdoMdakUkx, yoVgayxteyu ilalxvAdadxriMda barxhamxtavxvu eraDu kUDidAga I barxhamxsavxrUpavu iruvudeMdu tiLiyalu yArU samathaRrAgilalxveMdu tAtapxyaR.}
\begin{shl}
pANinA meYvamituyxkAtxyX taM pitA parxtayxSeVdhayatf | \\
\footnotemark{}saMBUya paramatavxM kaH savxtoVshakitxravApunxyAtf \hfill ||  7 || 
\end{shl}

%%%%footnote in shloka
\begin{artha}
taMdeyu hiVge heVLutatxlidadx putarxnanunx keYyiMda `hiVgelalx heVLabeVDa eMdu nivArisidanu. I eraDanunx seVrisi yAru tAne savxtaH shakatxnalalxdavanu barxhamxvasutxveMdu tiLidAru ? (yArU samathaRrilalx).
\end{artha}

\vishaya{savxtaH ashakatxnAdavanige beVreyobabx ashakatxna yoVgavidadxrU shakitx baralAradeMbudakekx daqSATxMta \mdash }

\begin{shl}
savxtoV\s shakitxmatoVloVRkeV shakitxyoVRgeV\s pi neVkaSxyXteV | \\
jAtayxnadhxyoVnaR yoVgeV\s pi shakitxV rUpavadiVkaSxNeV \hfill ||  8 || 
\end{shl}

\begin{artha}
loVkadalilx savxtaH tanage shakitxyilalxdavarige ashakatxrAdavara saMbaMdhavAdarU shakitxyu baruvudu kaMDilalx, udA:- huTuTx kuruDaru ibabxru seVridarU rUpavuLaLx vasutxvanunx noVDuva sAmathayxRvu avaralilx iruvudilalx.
\end{artha}

\begin{shl}
tasAmxcaCxkitxmatoVreVva gArxsAMshAnAM yathA tathA | \\
taqpitxshakitxrihApi sAyxtApxrXNAnAnxdAyxtamxnoVyuRtw \hfill ||  9 || 
\end{shl}

\begin{artha}
AdudariMda (shakitxyuLaLx vasutxgaLige) ananxda tututxgaLa aMshagaLalilx taqpitxyanunxMTu mADuva shakitxyu heVge iruvudo hAge pArxNa, ananx modalAda savxrUpagaLu shakitxyuLaLxvugaLAgidudx seVridalilx barxhamxtavx shakitxyu uMTAguvudu.
\end{artha}

\vishaya{hAgAdare ivugaLige shakitx heVge baruvudu ? eMdare \mdash }

\begin{shl}
na ceVtapxramatAM yAtaHkathaM paramatoVcayxtAmf | \\
parxteyxVkaM shakitxmatotxvXVkitxviVRtAyxdivacasoVcayxteV \hfill ||  10 || 
\end{shl}

\begin{artha}
ananx pArxNagaLu paramatavxvanunx (barxhamxrUpavanunx) hoMdilalxvAdare paramatavxvu heVge ? baruvudeMbudanunx heVLabeVku. - utatxra - vi' itAyxdi vacanadiMda parxteyxVka shakitxyiruvudeMdu heVLuvudu.
\end{artha}

\vishaya{`ananxMveY vi' eMba vAkayxda tAtapxyaR \mdash }

\begin{shl}
vAyxKAyxnAya tu viVtayxsayx tavxnanxM veY viVti BaNayxteV | \\
anenxV viSATxni sanetxyXVva BUtatavxM yAnitx yeVna hi \hfill ||  11 ||
\end{shl}

\begin{shl}
ananxviSATxni savARNi BUtAniVtayxnanxmucayxteV  | \\
viVti teVna sadA pArxNoV ramiteyxVvamihoVcayxteV \hfill ||  12 || 
\end{shl}

\begin{artha}
vi' eMbudara vAyxKAyxnakAkxgi `ananxM veY vi' eMdu heVLalapxTiTxde, yAvudariMda samasatx BUtagaLU ananxvanunx AsharxyisikoMDivevo adariMda `ananx viSAtxni savARNi BUtAni' BUtarUpavanunx hoMdu eMba vutapxtitxyiMda vi' eMba ananxvu samasatx BUtagaLigU AsharxyaveMdu heVLalapxDutatxde hAgeye I vAkayxdalilx ramf' eMdu yAvAgalU pArxNavAyuveMdu heVLalapxDuvudu.
\end{artha}

\begin{shl}
pArxNeV sati ramanetxV hi loVkeV daqSATxni savaRdA  | \\
BUtAni ramiti porxVkatxsatxsAmxtApxrXNoV\s pi sUriBiH \hfill ||  13 || 
\end{shl}

\begin{artha}
pArxNavu idadxre elalx pArxNigaLU kirxVDisuvavu eMbudu yAvAgalU kaMDive, adariMda pArxNavU vidAvxMsariMda pArxNavAyuveMdu heVLalapxTiTxde.
\end{artha}

\begin{shl}
savxtoV guNavatoVreVvamananxpArxNAtamxnoVBaRveVtf  | \\
saMBUya paramatavxM hi PaloVkitxH seYva yoVditA \hfill ||  14 || 
\end{shl}

\begin{artha}
I riVtiyAgi savxtaH guNavuLaLx ananx pArxNa savxrUpagaLige oTiTxnalilx paramatavxvu baruvudu, \footnote{BASayxdalilxruvaMte oTiTxnalilx siguva tAtapxyaRveVneMdare \mdash  `vi' eMbudu ananx, ramf eMbudu pArxNaveMdu taMde heVLidudx ananxvidadxre I elalx pArxNigaLU adanunx avalaMbisikoMDirutatxve adariMda ananxvanunx `vi' eMdu heVLidudx, pArxNaveV ramf' eMbudu. heVge ? eMdare, balavanunx avalaMbisi samasatx pArxNigaLU ATavADutatxve. balakekx AsharxyavAdadudx pArxNavu, oTiTxnalilx savaRBUtAsharxyaveMba guNavuLaLxdudx ananx, savaR BUtagaLa kirxVDAsAthxnavAdadedxMba guNavuLaLxdudx pArxNa, loVkadalUlx yArobabxrU maneyilalxde Asharxyavilalxde kirxVDisuvudilalx, matutx maneyidadxrU pArxNavilalxdavanu, dubaRlanAdavanu kirxVDisuvudilalx, yAvAga maneyidudx pArxNavU idudx balavuLaLxvanAgiruvano AvAga nAnu kaqtAthaRnAdeneMdu tiLidavanAgi kirxVDisuvanu. idu ``yuvAsAyxtasAdhuyuvAdhAyxyakaH'' itAyxdi shurxtiyiMda tiLiyuvudu. ananx guNa jAcnxnadiMda upAsakanalilx elalx pArxNigaLU baMdu Asharxyisutatxve, hAgU pArxNaguNa jAcnxnadiMda kirxVDisutatxve ideV Palavanunx vAtiRkadalUlx vivariside, ananx pArxNa eraDara savxrUpaveV Ada parxjApati barxhamxdalilx Asharxya paDedu ramisutatxveyeMdu visheVSAthaRvanunx toVrisidAdxre.}`svARNi havA asimxnf' itAyxdi yAvudanunx heVLideyo, adeV parxkaqta upAsakanige baruva Pala vacanavu.
\end{artha}

\vishaya{Palashurxtiyanunx vivarisuvudu.}

\begin{shl}
vishanitx savaRBUtAni hayxnanxBUteV parxjApatw  | \\
ramanetxV pArxNaBUteV ca PalameVtadupAsituH \hfill ||  15 || 
\end{shl}

\begin{artha}
samasatx BUtagaLu ananxrUpavAda parxjApati barxhamxnalilx parxveVshisutatxve. I riVtiyAgi pArxNarUpavAda parxjApatiyalelx kirxVDisutatxveyeMbudu parxsidadhxvAgide. ideV upAsakanige baruva Palavu.
\end{artha}

\begin{center}
{\bf ililxge shirxVbaqdadAraNayxkoVpaniSadf BASayxda}
\smallskip

{\bf vAtiRkadalilx hanenxraDane bArxhamxNavu pUNaRgoMDide}
\end{center}
