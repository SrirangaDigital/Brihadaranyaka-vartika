%%%% From 053.tex
\chapter{bArxhamxNa - 8}
%~ \section*{bArxhamxNa 8}

\begin{shl}
taseyxYva barxhamxNoV\s thAnayxdupAsanamihoVcayxteV | \\
vAgedhxVnUpAdhiyoVgeVna PalAya mahateV\s dhunA \hfill ||  1 || 
\end{shl}

\begin{artha}
adeV barxhamxna upAsaneyanunx ililx vAgfdheVnu eMba upAdhi saMbaMdhadiMda mahatAtxda PalakAkxgi Iga shurxtiyu heVLuvudu.
\end{artha}

%~ \section*{baq. 5, bArx. 8, kaMDike. 1}
\kandike{kaMDike 1}

\begin{kandikeshl}
vAcaM dheVnumupAsiVta tasAyxshacxtAvxraH satxnAH sAvxhAkAroV vaSaTAkxroV hanatxkAraH savxdhAkArasatxseyxY dwvx satxnw deVvA upajiVvanitx sAvxhAkAraM ca vaSaTAkxraM ca hanatxkAraM manuSAyxH savxdhAkAraM pitarasatxsAyxH pArxNa QuSaBoV manoV vatasxH || 1 ||
\end{kandikeshl}

\vishaya{vAtiRka}

\begin{shl}
vAgitayxtarx tarxyiV gArxhAyx na sAthxnakaraNAdayaH | \\
sAvxhAkArAdi nAnayxtarx tarxyAyxH saMBAvayxteV kacxcitf \hfill ||  2 || 
\end{shl}

\begin{artha}
ililx vAkf eMdare mUru veVdagaLeMdu garxhisabeVku, Adare vAkf sAthxnavAgali adaralilxruva iMdirxyAdigaLAgali gArxhayxvalalx, EkeMdare sAvxhAkArAdigaLu veVdatarxyavanunx biTuTx beVre elilxyU saMBavisuvudilalxvaSeTx.
\end{artha}

\begin{shl}
tarxyiVyaM dhayateV savaRM sAvxhAkArAdiBiH satxneYH | \\
dheVnumiteyxVva tAM teVna sadoVpAsiVta BAratiVmf \hfill ||  3 || 
\end{shl}

\begin{artha}
I mUru veVdagaLu sAvxhAkArAdi satxnagaLiMda (molegaLiMda) elalxvanunx (puruSAthaRgaLanunx) surisutatxve. adariMda I vAkakxnunx dheVnuveMde yAvAgalU dhAyxnisabeVku.
\end{artha}

\vishaya{yAva yAvudariMda I vAgedhxVnuvu surisuvadu? eMdare \mdash }

\begin{shl}
keVBayxH kaSxrati sA dheVnuriteyxVtadadhunoVcayxteV | \\
taseyxY dwvx satxnAvituyxkAtxyX kathayxnetxV ca satxnAsatxthA \hfill ||  4 ||
\end{shl}

\begin{artha}
yAva molegaLiMda A dheVnuvu surisuvudeMbudanunx Iga heVLuvudu, A dheVnuvige eraDu molegaLeMdu heVLidadxriMda nAlukx molegaLanunx heVLidaMte tiLiyabeVku.
\end{artha}

\vishaya{`tasAyxHpArxNa QuSaBaH' - eMbudara athaR \mdash }

\begin{shl}
QuSaBoV\s sAyxsatxthA jecnxVyaH pArxNasatxsAmxtapxrXsUyateV | \\
apArxNasayx na vAgasitx vAgapuyxcAcxyaRteV balAtf \hfill ||  5 || 
\end{shl}

\begin{artha}
(loVkadalilxruva dheVnuvige vaqSaBaviruvaMte) I vAgfdheVnuvige pArxNaveV vaqSaBa, EkeMdare adariMda shabadxvu huTuTxvadu, pArxNavilalxdavanige vAkukx iruvudilalx, pArxNada baladiMdale vAkukx ucacxrisalapxDuvadu.
\end{artha}

\vishaya{`manoVvatasxH' eMbudara athaR \mdash }

\begin{shl}
vAkapxrXsarxvaNaheVtutAvxdedhxVnAvx vatosxV manoV BaveVtf | \\
yadayxdAdhxyXyati manasA tatatxdAvxcA parxBASateV \hfill ||  6 || 
\end{shl}

\begin{artha}
shabadxvanunx surisuvudakekx kAraNavAdadxriMda dheVnuvige manasesxV karu Aguvudu, yAva yAvudanunx manasisxniMda ciMtisuvano adanunx vAgiMdirxyadiMda mAtanADuvanu.
\end{artha}

\begin{shl}
yathoVkotxVpAsanaPalamukatxbarxhAmxtamxrUpatA | \\
taM yathA yathA ituyxketxVranideVRsheV\s pi gamayxteV \hfill ||  7 || 
\end{shl}

\begin{artha}
hiMde heVLida upAsaneya Palavu meVle heVLida barxhamxsavxrUpa lABaveV, ``taM yathA yathA upAsateV tadeVva Bavati'' eMbudAgi shurxtiyu heVLiruvudariMda ililx nideRVsha mADade idadxrU I PalaveMdu toVrutatxde.
\end{artha}

\vishaya{idara sArAMsha}

\begin{artha}
hiMde heVLida barxhamxda upAsaneyanunx matotxMdu karxmadalilx heVLiruvudu - parabarxhamxdiMdale huTiTxda mUru veVdagaLeMba vAkf eMbudeV dheVnu, hasu, hasuvige nAlukx molegaLiruvaMte sAvxhA, vwSaTf, haMta, savxdhA' eMbive molegaLu, eraDu molegaLanunx sAvxhA, vwSaTfgaLanunx deVvategaLu karugaLaMte idudx upayoVgisuvaru, manuSayxru haMta' eMdu veYshavxdeVvadalilx upayoVgisuvaru, pitaqgaLu savxdhA eMbudanunx upajiVvisuvaru, I nAlukx molegaLiruva I dheVnuvige pArxNaveV vaqSaBa, manasesxV karu, I riVtiyAgi BAvisi vAgf dheVnuvanunx upAsane mADabeVku, adariMda veVdatarxyaveV upAdhiyAgi iruva soVpAdhika barxhamx savxrUpavanunx upAsakanu paDeyuvanu.
\end{artha}

\begin{center}
{\bf ililxge baqhadAraNayxkoVpaniSadf BASayxvAtiRkadalilx}
\smallskip

{\bf aidane adhAyxyadalilx eMTane bArxhamxNavu}

{\bf pUNaRgoMDide}
\end{center}
