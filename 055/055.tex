%~ \begin{center}
%~ \section*{sureVshavxravAtiRka}
%~ {\Large\textbf{adhAyxya - 6,   bArxhamxNa - 1}}
%~ \medskip
%~ \end{center}

\chapter{bArxhamxNa - 1}

\begin{shl}
samApatxH sapatxmoV\s dhAyxyaH pArxpAtxvasara ucayxteV | \\
aSaTxmaH KilakANeDxV\s simxnUpxvaRkANeDxVSavxnukitxtaH \hfill|| 1 || 
\end{shl}

\begin{artha} 
\footnote[1]{OMkAra, damAdi sAdhana, barxhamx, barxhemxVtara 
upAsanegaLu, adara PalagaLu, adakAkxgi heVLida mAgaRgati, AditAyxdi 
deVvateya upasAthxna - ivugaLanunx heVLidAdxyitu. Iga idaralilx 
barxhamxBinanxvAda vasUtxpAsanegaLanunx parxdhAnavAgi heVLuvudu. adara 
PalagaLanunx shirxVmaMthAdi kamaRvanunx hiMde heVLade iruvudanunx heVLalu 
eMTaneV adhAyxyavu baMdide. barxhomxVpAsaneya vicAravu muKayxvAgi 
mugida naMtara barxhemxVtara upAsaneya vicAravanunx pArxdhAnavAgi 
mADalu beVre adhAyxyavu baMdide. parxdhAna vicAravanunx mADade 
aparxdhAna vicAravanunx mADuvaMtilalx, saMBavisuvudU ilalx.}ELaneV adhAyxyavu mugiyitu. Iga (avakAshavu 
laBisiruvudariMda) pUvaRkAMDagaLalilx heVLadeyiruvudariMda Iga 
avakAshavu odagidadxriMda I KilakAMDadalilx eMTaneV adhAyxyavanunx 
heVLuvudu.
\end{artha}

\vishaya{Iga modalane bArxhamxNakekx avAMtara saMbaMdhavanunx heVLalu 
AraMBiside {\rm --}}

\begin{shl}
gAyatArxyXH pArxNaBAvoVkitxH kasAmxdedhxVtoVH puroVditA | \\
na tu vAgAdiBAvoV\s sAyxsatxtarx heVturihoVcayxteV \hfill|| 2 || 
\end{shl}

\begin{artha} 
gAyatirxyanunx pArxNaveV eMdu yAva kAraNadiMda hiMde heVLidudx? AdarU 
I gAyatirxyu vAgAdi savxrUpaveMdu heVLalilalx. idakekx 
kAraNaveVneMbudanunx I bArxhamxNadalilx heVLuvudu.
\end{artha}

\footnotetext[2]{hiMde heVLida pArxNoVpAsaneyu anayxsheVSavAgidadxre 
ukAthxdi upAsanegaLige sheVSavo (aMgavo) athavA gAyatirxV upAsanege 
aMgavo athavA maMthanakamaRkekx aMgavo eMdu vikalapxmADi 
utatxriside. vAgAdiiMdirxyagaLu idadxrU avugaLanunx biTuTx ukAthxdiguNa vishiSaTxvAda pArxNoVpAsaneyaneVnx heVLide. adakekx kAraNavAgi jeyxVSaThxtAvxdiguNavanunx heVLide. I upAsaneyu ukAthxdi upAsanege 
aMgavAgilalx. hiMdina garxMthavAda naMtara I garxMthavu EtakAkxgi 
baMdideyeMbudanunx hoMdisalu jeyxVSaThxtAvxdigaLanunx vaNiRside. adu 
pArxNavoMde upAsayxveMbudakekx kAraNa. guNaBeVdaviruvudariMdalU 
PalaBeVdaviruvudariMdalU pArxNoVpAsaneyu ukothxVpAsanege 
aMgavAguvudilalxveMdu tAtapxyaR.}
\begin{shl}
\footnotemark[2]jeyxVSaThxH sherxVSoThxV yataH pArxNoV na tu vAgAdayasatxtaH | \\
pArxNAtamxBAva EvoVkatx AnanatxyARthaRmeVva tu \hfill|| 3 || 
\end{shl}

%% shloka footnote
\begin{artha} 
yAvudariMda pArxNaveV jeyxVSaThx matutx sherxVSaThxvo adariMda matutx 
vAgAdi iMdirxyagaLu sherxVSaThxvU jeyxVSaThxvU alalxvo adariMda 
gAyatirxyu pArxNasavxrUpaveMdeV heVLide. matutx garxMthada AnaMtara 
saMbaMdhavanunx heVLuvudakAkxgiyU heVLalapxTiTxde.
\end{artha}

\section*{baq. a.6, bArx. 1, kaMDike 1}

\begin{shl}
OM yoV ha veY jeyxVSaThxM ca sherxVSaThxM ca veVda jeyxVSaThxshacx sherxVSaThxshacx sAvxnAM Bavati pArxNoV veY jeyxVSaThxshacx sherxVSaThxshacx jeyxVSaThxshacx sherxVSaThxshacx sAvxnAM Bavatayxpi ca yeVSAM buBUSati ya EvaM veVda || 1 ||
\end{shl}

\begin{shl}
yoV ha veY vasiSAThxM veVda vasiSaThxH sAvxnAM Bavati vAgevxY vasiSAThx vasiSaThxH sAvxnAM Bavatayxpi ca yeVSAM buBUSati ya EvaM veVda || 2 ||
\end{shl}

\begin{shl}
upAsatxyXnatxrameVveYtataPxlavatutx vivakiSxtamf | \\
na tUkatxsheVSateYtasAyx BinonxVpAsitxtavxkAraNAtf \hfill|| 4 || 
\end{shl}

\begin{artha} 
idoMdu PalavuLaLx beVre upAsaneyeMbudeV vivakiSxta. Adare hiMde 
heVLida gAyatirxV upAsanege idu aMgavAgilalx. idu beVre 
upAsaneyAgiruvudeV kAraNa.
\end{artha}

\vishaya{shaMke}

\begin{shl}
manathxkamaRNi yeV manAtxrXH pacnacx jeyxVSAThxdayaH shurxtAH | \\
pArxNAtamxveVdinasetxVSAM parxyoVgoV\s torxVpavaNayxRteV \hfill|| 5 || 
\end{shl}

\begin{artha} 
manathxnakamaRdalilx jeyxVSAThxdi maMtarxgaLu aidu. yAvudu 
shurxtavAgiveyo avugaLa parxyoVgavanunx pArxNAtamxvanunx 
upAsisuvavanige ililx muMde vaNiRsideyeMdu heVLabahudaSeTx? (eMdu 
shaMke).
\end{artha}

\vishaya{parihAra {\rm --}}

\begin{shl}
paqthagAvx PalanideVRshAdoyxV ha vA iti pacnacxdhA | \\
pArxNavidAyx paqthaknamxnAthxnamxnathxsutx mahimAthiRnaH \hfill|| 6 || 
\end{shl}

\begin{artha} 
`yoV havA' eMdu aidu bageyAgi beVre Palavanunx nideRVshisidadxriMda 
pArxNavideyxyu maMthakamaRkikxMta beVreyAgide. (adakekx aMgavilalx) 
maMthanakamaRvu mahimeyanunx bayasuvavanige heVLide.
\end{artha}

\vishaya{mahatatxvX Palavu pArxNavideyxge ilalxve? eMdare {\rm --}}

\begin{shl}
PaleV\s nayxsimxnanxnidiRSeTxV vAkayxsheVSagataM Palamf | \\
tasimxnasxti hi sadABxvAdAvxgAdiVnAM na taM vinA \hfill|| 7 || 
\end{shl}

\begin{artha} 
beVre Palavanunx nideRVsha mADadiruvAga vAkayxsheVSadalilxruva 
Palavanunx garxhisabeVkAguvudu. (beVre Palavu nideRVshisalapxTaTxlilx 
Palavu iruvudariMda vAkayxsheVSadalilxna (mahatatxvXPalavu 
gArxhayxvalalx) pArxNavu idadxre vAgAdi iMdirxyagaLu iruvudariMdalU 
adilalxde ivu ilalxdiruvudariMdalU (vAgAdi iMdirxyagaLige 
sherxVSaThxteyu ilalx).)
\end{artha}

\vishaya{I viSayakekx shAsatxrXsaMvAdavU irutatxde {\rm --}}

\begin{shl}
shAserxVNoVkAtx shariVreV\s simxnavxqqtitxH pArxNasayx jiVvanamf | \\
pUvaRmAvishati pArxNoV deVhaM pashAcxcacx mucnacxti \hfill|| 8 || 
\end{shl}

\begin{artha} 
shAsatxrXvu heVLuvudu EneMdare? I shariVradalilx pArxNada vaqtitxyeV 
jiVvana. (gaBaRdalilx) pArxNavAyuveV modalu shariVravanunx 
parxveVshisuvudu. anaMtaravU pArxNaveV biDuvudu.
\end{artha}

\begin{shl}
jeyxVSaThxH sherxVSaThxshacx saveVRSAM pArxNAnAmAsharxyoV hi saH | \\
sherxVSaThxtA vakaSxyXmANeVna garxnethxVnAsayx viBAvayxteV \hfill|| 9 || 
\end{shl}

\begin{artha} 
elalx vAgAdi pArxNagaLigU A pArxNaveV jeyxVSaThx matutx sherxVSaThx. 
muMde heVLuva garxMthadiMda idakekx sherxVSaThxteyu gotAtxguvudu.
\end{artha}

\vishaya{`yoVhaveY vasiSAThxM' itAyxdi maMtarxda athaR {\rm --}}

\footnotetext[1]{yAru vAkayxvanunx vasiSaThxveMba guNavuLaLxdAdxgiruvaMte upAsane mADuvaro, alalxde jAcnxtigaLalilx hecicxnavanAgi vasiSaThxnAgalu bayasuvaro avaru upAsanege takakxMte vasiSaThxneV Aguvanu. vAkukx vasiSaThxveMbudakekx vAsayati atishayeVna eMba vuyxtapxtitxyiMda vAgimxgaLAdavaranunx hecicxna haNavaMtaranAnxgi mADabalalxdu. adariMda vasiSaThx matutx vasatx iti vasiSAThx eMba vuyxtapxtitxyiMda inonxbabxranunx mucucxvudu. vAkikxniMda matotxbabx vAcAligaLanunx tirasakxrisuvanu. adariMda `vasa nivAse, vasa AcACxdane' eMbuva eraDu dhAtugaLa rUpave ililx vasiSaThxveMba pAdavu. I athaRdalilx vasiSaThxveMba guNagaLiMda kUDi vAkakxnUnx upAsisuvanu, vAgimxyAgi matotxbabxranunx aDagisi ati dhanavaMtanAgi bALuvanu eMdu I maMtarxda athaR. idu BASayxdalilxruva athaR. cakuSxrAdi itara iMdirxyagaLige upahatiyAguvaMte vAkikxge upahatiyilalx. eraDu loVkagaLalUlx idara parxvaqtitxyuMTu. tananx deVhadoLagiruva pArxNavanunx (niroVdhisuva) niyamisuvaMte itara deVhadalilxruva (pArxNa) itara iMdirxyagaLanunx niyamisabalalxdu vAkukx udA:- obabx yajamAnanu manemaMdiyelAlx janaranunx IkaDe noVDikoLiLx, idanunx keVLiri, idanunx tininxri, eMdu vAcA niyamisabalalxnu. hAgeye rAjanu Baqtayxranunx hoVgu, noVDibA, ililx noVDiko, ililx keVLu, hiVge hoVgu, bA eMdu matotxbabxnanunx niyamisuvanu hiVgeye vAkukx elalxra parxvaqtitxgU kAraNavAdadxriMda adu akuMThitavAgiruvudeMdu. itara iMdirxyagaLa parxvaqtitxgU vAkekxV kAraNavAdadxriMda adu adhiraveMdu upAsayxvAgide.}
\begin{shl}
\footnotemark[1]asimxMlolxVkeV parasimxMshacx vAgeVva na vihanayxteV | \\
deVhAnatxrasAthxnApxrXNAMshacx niyuknatx\s sA tatoV\s dhikA \hfill|| 10 || 
\end{shl}

%% shloka footnote
\begin{artha} 
ihaloVkadalUlx paraloVkadalUlx vAkekxMbudeV kuMThitavAguvudilalx. 
beVre deVhadalilxruva pArxNagaLanUnx kUDa niyoVgisuvudu. adariMda adu 
itara pArxNagaLigiMta hecicxnadu.
\end{artha}

\vishaya{baq. a.6, bArx. 1, kaMDike 3}

\begin{shl}
yoV ha veY parxtiSAThxM veVda parxtitiSaThxti sameV parxtitiSaThxti dugeVR cakuSxveYR parxtiSAThx cakuSxSA hi sameV ca dugeVR ca parxtitiSaThxti parxtitiSaThxti sameV parxtitiSaThxti dugeVR ya EvaM veVda || 3 ||
\end{shl}

\vishaya{I meVlina maMtarxda athaR {\rm --}}

\begin{shl}
shurxtAnamxtAtatxthoVkAtxdAvx parxmANaM daqSiTxrAtamxnaH | \\
cakuSxH parxtiSAThx jAcnxnAnAmAtAmx tatarx parxtiSiThxtaH \hfill|| 11 || 
\end{shl}

\begin{artha} 
\footnote[2]{kaNuNx eMbuva cakuSxriMdirxyavu parxtiSAThx nijavAda 
Asharxya. tAnu nilulxvudakekx kAraNa heVge? kaNiNxniMda samaBUmiyalilx 
noVDi nilulxvanu dugaRmavAda pavaRtAdigaLalUlx noVDi nilulxvanu. 
hAgeyeV suBikaSxvAda kAladalUlx duBiRkASxdi kAladalUlx kaNiNxniMda 
noVDi tiLidu nilulxvanu. adariMda tAnu nilulxvudakekx muKayxvAda 
kAraNa cakuSx adu parxtiSAThx Asharxya. I guNaviruvaMte 
cakuSxriMdirxyavanunx upAsane mADidavanu samadeVshadalUlx, 
viSamaparxdeVshadalUlx, samaviSama kAlagaLalUlx noVDi nilalxbalalxnu. 
idu BASayxdaMte athaR. vAtiRkadalilx visheVSAthaRvide. nAvu oMdu 
viSayavanunx keVLi tiLiyuvudakikxMtalU, matutx takiRsuvudakikxMtalU 
KacitavAgi niNaRya koDuvudu daqSiTx. kaNiNxniMda noVDidadxnunx 
naMbabahudu. adariMda itara jAcnxnagaLigU ideV Asharxya. koneya 
niNaRyakekx kAraNa, `nahidaqSeTxV anupapananxM nAma' noVDidadxralilx 
anupapatitxyilalx. saMshayakekx eDeyilalx. adariMda cakuSx Asharxya. 
`yoV\char'263 yaMdakiSxNeV\char'263 kaSxnf puruSaH' eMdu 
balagaNiNxnalilx I Atamxnu visheVSavAgi nelesuvanu. AsAthxnadalilxdudx 
noVDuvaneMdu visheVSavAgi heVLiyU ide. adariMda shorxVtarx itAyxdi 
iMdirxyagaLigU adhikavAdadudx cakuSx, adu upAsayxveMdathaR.}keVLidadxkUkx (tAnu tiLididadxkUkx) takiRsidadxkUkx 
hAgeyeV heVLiruvudakUkx hecicxna parxmANaveMdare tanage kaNaNxlilx 
noVDuvudeV. kAraNaveVneMdare elAlx jAcnxnagaLigU Asharxya 
cakuSxriMdirxya.
\end{artha}

\vishaya{`yoVhaveY saMpada' - idara athaR {\rm --}}

\begin{shl}
vAgiGx saMpadayxteV shorxVtArxdashurxtaM na hi BASateV | \\
savxvaqtetxVH paravaqtetxVshacx saMpacoCxrXVterxV parxtiSiThxtA \hfill|| 12 || 
\end{shl}

\begin{artha} 
vAkukx shorxVterxVMdirxyadiMda pUNaRvAguvudu. keVLade idadxdadxnunx 
tAnu mAtanADuvudilalxvaSeTxV. tananx vAyxpArada PalavAgi baruva 
saMpatutx matotxbabxna vAyxpArada PalavAda saMpatutx saha 
shorxVterxVMdirxyadalelxV niMtiruvudu.
\end{artha}

\vishaya{baq. a.6, bArx. 1, kaMDike 4}

\footnotetext[1]{I meVlina maMtArxthaR - shorxVterxVMdirxya saMpatutx 
heVgeMdare, idu idadxre elAlx veVdagaLu keYgUDuvavu. kivi idadxvane 
veVdavanunx keVLi adhayxyana mADuvudu. veVdadalilx vidhisida 
kamaRgaLige adhiVnavAgi elAlx BoVgasaMpatutxgaLU baruvuvu. adariMda 
shorxVterxVMdirxyavu saMpatetxMdu heVLide. yAva sAdhakanu saMpatetxMba 
guNavuLaLxdAdxgi I shorxVtarxvanunx dhAyxnisuvano avanige I Palavu 
baruvudu yAva BoVgavanunx bayasuvano adeV kAma. adeV BoVgavu ivanige 
odaguvudu.}
\begin{shl}
\footnotemark[1]yoV ha veY samapxdaM veVda saM hAsemxY padayxteV yaM kAmaM kAmayateV shorxVtarxM veY samapxcoCxrXVterxV hiVmeV saveVR veVdA aBisamapxnAnxH saM hAsemxY padayxteV yaM kAmaM kAmayateV ya EvaM veVda || 4 ||
\end{shl}

\vishaya{baq. a.6, bArx. 1, kaMDike 5}

\footnotetext[1]{yAvanu manaseVsx iMdirxyagaLigU viSayagaLigU AsharxyaveMdu 
tiLidu dhAyxnisuvano, avanu tananx jAcnxtigaLigU itararigU 
AsharxyanAguvanu. manasisxnalilx nelesida viSayagaLu Atamxnige 
BoVgayoVgayxvAgutatxve. iMdirxyagaLu manasisxna saMkalapxkekx 
takakxMte naDeyutatxve. manasisxdadxre iMdirxyagaLu horage ODADutatxve 
ilalxvAdare hiMtirugutatxve. adariMda manaseVsx Ayatana Asharxya. 
adariMda adu namage upAsayx.}
\begin{shl}
\footnotemark[1]yoV ha vA AyatanaM veVdAyatanaM sAvxnAM BavatAyxyatanaM janAnAM manoV vA AyatanamAyatanaM sAvxnAM BavatAyxyatanaM janAnAM ya EvaM veVda || 5 ||
\end{shl}

%% maMtarx footnote

\vishaya{I maMtarxda athaR {\rm --}}

\begin{shl}
mana AyatanaM tatarx vAgAdiVnAM hi vaqtatxyaH | \\
sithxtAsatxtUpxviRkAshecxYva dhAyxyataH sAdhanaM hi tatf \hfill|| 13 || 
\end{shl}

\begin{artha} 
iMdirxyagaLigU viSayagaLigU manasesxV Asharxya. adaralilx elAlx vAgAdi 
iMdirxyagaLa vaqtitxgaLU iruvavu. manaHpUvaRkavAgiyeV I vaqtitxgaLu 
adaralilx nelesive. dhAyxna mADuvavanige manasesxV sAdhana.
\end{artha}

\section*{baq. a.6, bArx. 1, kaMDike 6}

\footnotetext[2]{I maMtarxda athaR - yAvanu parxjAti reVtasusx eMbuva 
jananeVMdirxyavanunx parxjAsaMtatige kAraNaveMdu tiLidu adanunx 
upAsisuvano avanu parxjeyiMdalU pashugaLiMdalU saMpananxnAguvanu. 
adariMda adu upAsayxveMdathaR.}
\begin{shl}
\footnotemark[2]yoV ha veY parxjAtiM veVda parxjAyateV ha parxjayA pashuBiV reVtoV veY parxjAtiH parxjAyateV ha parxjayA pashuBiyaR EvaM veVda ||6||
\end{shl}

\vishaya{I maMtarxda tAtapxyaR {\rm --}}

\begin{shl}
upasethxVnidxrXyaM parxjAtiH sAyxtatxsayx janemxYkaheVtutaH | \\
na hi reVtoV vinA janamx pArxNinoV\s tarx samiVkaSxyXteV \hfill|| 14 || 
\end{shl}

\begin{artha} 
parxjAti eMdare upasethxVMdirxya. adu janamxvoMdakekx kAraNa. reVtasusx ilalxde yAva pArxNigU huTeTxMbudeV ilalx. idu ihadalilx hiVge kANuvudilalx.
\end{artha}

\vishaya{`teVha' itAyxdi maMtarxda tAtapxyaR {\rm --}}

\begin{shl}
vaqtitxVnAM pArxNapUvaRtAvxdavxyXpadeVshAcacx tatakxqqtAtf | \\
pArxNAnAM parxthamaH pArxNaH sa hayxtAtx\s nanxM hi tasayx tatf \hfill|| 15 || 
\end{shl}

\begin{artha}
samasatx iMdirxya vAyxpAragaLu pArxNavAyuvina mUlakavAgiruvudariMdalU pArxNa nimitatxvAgiyeV pArxNaveMdu karediruvudariMdalU elAlx pArxNagaLigU (iMdirxyagaLigU) I pArxNavAyuvu modalaneyadu, (jeyxVSaThx, sherxVSaThx), A pArxNaveV BoVkatx. adakekx itara pArxNAdigaLu BoVgayxvaSeTx.
\end{artha}

\vishaya{baq. a.6, bArx. 1, kaMDike 7}

\begin{shl}
teV heVmeV pArxNA ahaMsherxVyaseV vivadamAnA barxhamx jagumxsatxdodhxVcuH koV noV vasiSaThx iti tadodhxVvAca yasimxnavx utAkxrXnatx idaM shariVraM pApiVyoV manayxteV sa voV vasiSaThx iti || 7 ||
\end{shl}

\vishaya{I maMtarxda vAyxKAyxna {\rm --}}

\begin{shl}
teV pArxNAH savxguNeVruketxYvaRsiSaThxtAvxdilakaSxNeYH | \\
sherxVyAnasamxyXhameVveVti vivadanatxH parasapxramf \hfill|| 16 || 
\end{shl}

\begin{shl}
niNaRyAthARya teV barxhamx jagumxrinadxrXM parxjApatimf | \\
koV noV vasiSaThx iti taM paparxcuCxniRNaRyAthiRnaH \hfill|| 17 || 
\end{shl}

\begin{artha}
A pArxNagaLu (iMdirxyagaLu) hiMde heVLida vasiSaThxtavx modalAda 
tamamx guNagaLiMda nAneV sherxVSaThxneMdu anoyxVnayx jagaLavADutAtx 
idara niNaRyakAkxgi barxhamx eMbuva IshavxranAda parxjApatiya hatitxra 
hoVdaru. niNaRyavanunx bayasida I pArxNagaLu namamx peYki yAru 
vasiSaThx? (hecicxna vAgimx?) eMdu keVLidavu.
\end{artha}

\vishaya{adakekx barxhamxnu heVLida utatxra {\rm --}}

\begin{shl}
yasimxnfva iti vAkeyxVna vasiSaThxtavxsayx lakaSxNamf | \\
pArxNeVBayxH pArxbarxviVdabxrXhamx pakaSxpAtaBayAtikxla \hfill|| 18 || 
\end{shl}


\begin{artha}
`yasimxnf vaH' eMba vAkayxdiMda parxjApatiyu pakaSxpAtavAguva 
BayadiMda vasiSaThxtavxda lakaSxNavanunx pArxNagaLige modalu 
heVLidanu.
\end{artha}

\begin{shl}
jAnananxpi vasiSAThxdiguNavatatxvXM yathAthaRtaH | \\
tathA\s pi nAvadadabxrXhamx sAvxnuBUtayxvabudadhxyeV \hfill|| 19 || 
\end{shl}

\begin{artha}
matutx vasiSAThxdi guNagaLu yathAthaRvAgi iruvudanunx tiLidavanAdarU 
parxjApatiyu heVLalilalx. EkeMdare? tamamx anuBavadiMdaleV 
tiLiyabeVkeMbudakAkxgi.
\end{artha}

\vishaya{adeVneMdare:-}

\begin{shl}
utAkxrXnetxV\s nayxtameV yasimxnapxrXteyxVkamapasapaRNeV | \\
pApiVyoV manayxteV loVkoV vasiSoThxV vaH samiVkaSxyXtAmf \hfill|| 20 || 
\end{shl}

\begin{artha}
eleY vAgAdigaLirA, nimamx peYki obabxru parxteyxVkavAgi obobxbabxrU 
shariVravanunx biTuTx edudx Acege saridalilx I shariVravanunx janaru 
pApiSaThxveMdu (atayxMta doVSayukatx asapxqqshayx)veMdu tiLiyuvaro, 
AvAga nimamxlilx obabxnu vasiSaThx eMdu Aguvanu, noVDiri eMdu 
heVLidanu.
\end{artha}

\vishaya{baq. a.6, bArx. 1, kaMDike 8}

\begin{shl}
vAgoGxVcacxkArxma sA saMvatasxraM porxVSAyxgatoyxVvAca kathamashakata madaqteV jiVvitumiti teV hoVcuyaRthAkalA avadanotxV vAcA pArxNanatxH pArxNeVna pashayxnatxshacxkuSxSA shaqNavxnatxH shorxVterxVNa vidAvxMsoV manasA parxjAyamAnA reVtaseYvamajiVviSemxVti parxviveVsha ha vAkf || 8 ||
\end{shl}

\vishaya{I shurxtigaLa tAtapxyaR {\rm --}}

\begin{shl}
anavxyavayxtireVkABAyxM vasiSaThxtAvxvabudadhxyeV | \\
upAsAyxthaRpariVkASxyeY parxvaqtetxYSA parA shurxtiH \hfill|| 21 || 
\end{shl}

\begin{artha}
anavxyavayxtireVkagaLiMda vasiSaThx eMbudanunx tiLiyalu matutx upAsayx vasutxvanunx pariVkiSxsalu I muMdina shurxtiyu baMdiruvudu.
\end{artha}

\vishaya{`sA saMvatasxraM porxVSayx' itAyxdi maMtarxdalilx toVruvaMte 
oMdu vaSaR payARMtara shariVravanunx biTuTx vAgiMdirxyAdigaLu horage 
hoVgidadxveMba athaRvu vivakiSxtave? eMba shaMkeyanunx pariharisuvudu 
{\rm --}}

\begin{shl}
saMhatAnAM kirxyAsidedhxVH karaNAnAM paqthakapxqqthakf | \\
neYkeYkasayx kirxyAsididhxH shibikoVdAvxhavatatxtaH \hfill|| 22 || 
\end{shl}

\begin{artha}
nAlukx janaru seVri palalxkikxyanunx horuvaMte malinavAda 
iMdirxyagaLalelxV kirxyAsididhxyAguvudariMda oMdoMdakekx kirxyeya 
niSapxtitxyAgalAradeMdu tiLiyabeVku. (adariMda vaSaRkAla parxvAsavu 
vivakiSxtavalalx)
\end{artha}

\begin{shl}
pArxNapArxdhAnayxsidadAdhxyXthaRM shurxtAyx\s \s KAyxyikacaCxdamxnA | \\
anavxyavayxtireVkABAyxM nAyxyoV lwkika ucayxteV \hfill|| 23 || 
\end{shl}

\begin{artha}
pArxNa vasutxvige pArxdhAnayxvu sididhxsalu shurxtiyu katheya nepadalilx anavxya vayxtireVkagaLiMda kUDida lawkika nAyxyavanunx heVLiruvudu.
\end{artha}

\vishaya{lawkika nAyxyavanunx I muMde toVrisuvaru {\rm --}}

\begin{shl}
yathA mUkA vinA vAcA yathA\s nAdhxshacxkuSxSA vinA | \\
itAyxdivacasA pArxNeV sati jiVvanamucayxteV \hfill|| 24 || 
\end{shl}

\begin{artha}
heVge bAyiyilalxdidadxvaru mUkarAguvaro, hAgeye kaNiNxlalxdavaru kuruDarAguvaro itAyxdi vacanadiMda pArxNavAyuvu idadxreV jiVvana (baduku) eMdu heVLiruvudu.
\end{artha}

\begin{shl}
utAkxrXnwtx ca parxveVsheV ca hayxlaM dahaH savxkamaRNeV | \\
vAgAdiVnAM na pAtoV\s sayx nApi coVtAthxnamiVkaSxyXteV \hfill|| 25 || 
\end{shl}

\begin{artha}
vAgAdi iMdirxyagaLu shariVravanunx biTuTx meVlakekx horaTAgalU punaH parxveVshisidAgalU (I aMtaradalilx) deVhaveV tananx kAyaRvanunx naDesalu samathaRvAgididxtu. avu biTuTx edAdxga deVhakekx patanavAgalilalx. avu baMdu punaH parxveVshisidAga shariVravu edudxkoMDidUdx kANisalilalx.
\end{artha}

\begin{artha}
\textbf{8-9-10-11-12 maMtarxgaLa sArAthaR {\rm --}}
(parxjApatiyu heVLida naMtara vAgAdi iMdirxyagaLeMba pArxNagaLu tamamx shakitxyanunx pariVkiSxsikoLaLxlu karxmavAgi vAgiMdirxya, cakuSxriMdirxya, shorxVterxVMdirxya, manasusx - matutx jananeVMdirxya - ivugaLu oMdoMdAgi edudx oMdu vaSaR shariVravanunx biTuTx dUra hoVgiralu deVhavu tananx uLida kAyaRvanunx naDesutAtx jiVvisididxtu. vAgiMdirxya horaTAga mAtanADade idadxrU itara iMdirxyagaLiMda dashaRna, sharxvaNAdigaLanunx mADutAtx muKayxvAgi pArxNavAyuviniMda usirADutAtx jiVvisididxtu. hiVgeye kaNuNx horaTidAdxga kuruDAdarU itareVMdirxyagaLiMda pArxNavAyuviniMda vayxvaharisutAtx jiVvisididxtu. shorxVterxVMdirxyavu horaTAga kivuDAdarU itara vayxvahAravanunx sAgisididxtu. hAgeyeV manasUsx Acege hoVdAga mUDharAgidudx tiLiyadavarAgi itara vayxvahAravanunx mADutAtx sAdhisitu. jananeVMdirxyavu horaTAga napuMsakavAgi idudx jiVvisididxtu. oTiTxnalilx yAva jAcnxneVMdirxya kameRVMdirxyagaLU citatxvU ilalxdidadxrU pArxNavAyuviniMdale shariVravu jiVvisididxtu. adariMda itareVMdirxyagaLige pArxdhAnayxvilalxveMdu niNaRyavAyitu.)
\end{artha}

\begin{shl}
atha ha pArxNa utakxrXmiSayxnayxthA mahAsuhayaH seYnadhxvaH paDivxVshashaknUkxnasxMvaqheVdeVvaM heYveVmAnApxrXNAnasxMvavahaR teV hoVcumAR Bagava utakxrXmiVnaR veY shakASxyXmasatxvXdaqteV jiVvitumiti tasoyxV meV baliM kuruteVti tatheVti || 13 ||
\end{shl}

\vishaya{I maMtarxda tAtapxyaR {\rm --}}

\begin{shl}
utAkxrXnetxV pArxNa EvAsAmxcaCxriVraM patati dhurxvamf | \\
utitxSaThxti parxviSeTxV ca pArxNaH sherxVyAMsatxtoV\s nayxtaH \hfill|| 26 || 
\end{shl}

\begin{artha}
pArxNaveV shariVravanunx biTeTxdadxre shariVravu nishacxyavAgi 
biVLuvudu, adeV pArxNavu shariVradoLage parxveVshisidalilx shariVravu 
edudx nilulxvudu. adariMda beVre elalxdakUkx pArxNaveV sherxVSaThx.
\end{artha}

\begin{artha}
(\textbf{maMtArxthaR} {\rm --} anaMtara pArxNavu shariVravanunx biTeTxVLalu 
yatinxsuvAga AgaleV tananx tananx sAthxnadiMda vAgAdi iMdirxyagaLu 
calisidavu. adu heVgeMdare, kudureya savAranu pariVkASxthaRvAgi oMdu 
siMdhu deVshadalilx huTiTxdadx lakaSxNayukatxvAda doDaDx kudureyanunx 
Eralu A kudureyu tananx kAlugaLanunx kaTiTxhAkidadx gUTagaLanunx 
kitutxkoMDu oMdeVsala meVlakekx hAruvudo, hAgeye calisidavu. avugaLu 
calisuvaMte pArxNavu mADuvudu. AvAga A vAgAdi pArxNagaLu heVLidavu 
EneMdare? - pUjayx pArxNave? niVnu shariVravanunx biTuTx horaDabeVDa, 
niVnilalxde nAvu badukuvudakekx Aguvudilalx, eMdu. idanunx keVLi 
pArxNavu heVLitu, niVvugaLu nananx sherxVSaThxteyanunx tiLidideVdx 
Adare nanage kapapxkANikeyanunx opipxsi eMdu.)
\end{artha}

\vishaya{I meVlina maMtarxdalilx `teVhoVcuH' - itAyxdi vAkayxda 
tAtapxyaRvanunx heVLutAtxre {\rm --}}

\begin{shl}
mAmaqteV jiVvituM yUyaM yadayxshakAtxH sathx savaRdA | \\
parxdhAnaM tahiR mAM vitatx BavanatxshAcxparAdhinaH \hfill|| 27 || 
\end{shl}

\begin{shl}
karaM baliM parxdhAnAya datatx vAgAdayoV\s cirAtf | \\
ituyxkAtxsetxV tatheVtUyxcuH savaRsavxM dadateV parxBoVH \hfill|| 28 || 
\end{shl}

\begin{artha}
eleY, vAgAdi iMdirxyagaLirA niVvu shiVGarxvAgi parxdhAnavAda 
pArxNadeVvanige kapapxkANikeyanunx koDiri eMdu heVLalu hAgeyeV AgaleMdu 
heVLidavu. parxBuvige samasatxvanunx salilxsidavu.
\end{artha}

\vishaya{`sAheVtAyxdi' maMtarxda tAtapxyaR {\rm --}}

\vishaya{baq. a.6, bArx. 1, kaMDike 14}

\begin{shl}
sA ha vAguvAca yadAvx ahaM vasiSAThxsimx tavxM tadavxsiSoThxV\s siVti yadAvx ahaM parxtiSAThxsimx tavxM tatapxrXtiSoThxV\s siVti cakuSxyaRdAvx ahaM samapxdasimx tavxM tatasxmapxdasiVti shorxVtarxM yadAvx ahamAyatanamasimx tavxM tadAyatanamasiVti manoV yadAvx ahaM parxjAtirasimx tavxM tatapxrXjAtirasiVti reVtasatxsoyxV  ||
\end{shl}

\begin{artha}
athaR (A vAkukx modalu karavanunx koDalu parxvatiRsi hiVge heVLitu - 
yadAvx nAneV vasiSaThxLAgidedxVne. nananx vasiSaThxtavxvu ninanxdeV 
Agide. A vasiSaThxtavx guNadiMda niVnu vasiSaThxnAgiruve. hiVgeye 
itara iMdirxyagaLu karxmavAgi parxtiSAThx, saMpatutx, Asharxya, 
parxjanana itAyxdi guNavuLaLxvugaLeMdu heVLikoLuLxtAtx avelalxvU 
ninanxdeV AgiveyeMdu pArxNakekx opipxsidavu - idAda naMtara utatxmavAda 
karavanunx niVvu koTiTxdidxVri. I guNagaLiMda kUDida pArxNadeVvanige 
ananx yAvudu? baTeTx yAvudu? tegedukoMDu baninx eMdu pArxNavu heVLitu.)
\end{artha}

\begin{shl}
yadidaM kicnAcxshavxBayx A kaqmiBayx A kiVTapataknegxVBayxsatxtetxV\s nanxmApoV vAsa iti na ha vA asAyxnananxM jagadhxM Bavati nAnananxM parxtigaqhiVtaM ya EvameVtadanasAyxnanxM veVda tadivxdAvxMsaH shorxVtirxyA ashiSayxnatx AcAmanatxyXshitAvxcAmanetxyXVtameVva tadanamanaganxM kuvaRnotxV manayxnetxV || 14 ||
\end{shl}

\begin{shl}
tavxdavxsiSaThxtayeYvAhaM vAgavxsiSeThxVtuyxdAharatf | \\
ituyxkAtxvX\s neyxV\s pi savaRsavxM daduvARgAdayaH surAH \hfill|| 29 || 
\end{shl}

\begin{artha}
pArxNadeVvaneV! ninanxlilxruva vasiSaThxteyeMbuva guNadiMdaleV nAnU 
vasiSaThxnAgiruveneMdu vAkukx heVLitu. hiVgeyeV itara iMdirxyAdigaLU 
heVLidavu anaMtara vAgAdi deVvategaLu tamamx savaRsavxvanunx apiRsidavu.
\end{artha}

\vishaya{eraDu parxshenxgaLa vAyxKAyxna}

\begin{shl}
kimananxM meV buBukoSxVH sAyxdAvxsoV vA meV kimiVyaRtAmf | \\
iti pArxNavacaH shurxtAvx parxtUyxcuH karaNAni tamf \hfill|| 30 || 
\end{shl}

\begin{shl}
A shavxBoyxV yadidaM kiMcidA kaqmiBayxshacx lakaSxyXteV | \\
ananxM tadaBxvataH savaRM vayaM tavxceCxVSaBoVginaH \hfill|| 31 || 
\end{shl}

\begin{artha}
nanage ananx AhAra yAvudu? nanage hasivu Aguvudu, nanage baTeTx 
yAvudu? eMdu pArxNavu heVLida mAtanunx keVLi iMdirxyagaLu A 
pArxNavanunx kuritu nAyigaLiMda AraMBisi kirxmigaLa payaRMtara 
yAvudoMdu kaMDideyo, adelalx ananxvU ninage irali. nAvu niVnu UTa mADi 
uLida sheVSavanunx BuMjisuvavaru.
\end{artha}

\vishaya{ililx oMdu shaMke}

\begin{shl}
jiVvaH pArxNoV\s tarx saMsAriV BoVketxVnidxrXyamanaHparaH | \\
\footnotemark[1]pArxNoV heyxVtAni savARNi BavatiVti ca liknagxtaH \hfill|| 32 || 
\end{shl}
\footnotetext[1]{`AtemxVMdirxya manoVyukatxM BoVketxVtAyxhu maRniVSiNaH' eMba shurxtiyaMte jiVvaneV BoVkAtx. parxkaqta elAlx ananxvU pArxNakekx eMdu heVLidadxriMda pArxNavU BoVkatxqqveMdu heVLuvudu yukatx. saveRVMdirxyagaLu pArxNavaneVnx AsharxyisiruvudariMda iMdirxyasAvxmitavxvu jiVvanigiMta beVre kaDeyilalxvAdadxriMda pArxNaveMbudu jiVvave eMdu aBipArxya.}

%% shloka footnote
\begin{artha}
meVle heVLida maMtarxdalilx pArxNaveMdare iMdirxya manasusxgaLalilx AsakatxnAgi BoVkatx eMbuva saMsAri jiVvaneMdu heVLabeVkaSeTx. idakekx `pArxNoVheyxVtAni savARNi Bavati' eMba shurxtiyu gamakavAgide. adariMdalU (jiVvane pArxNa eMbudara athaR)
\end{artha}

\vishaya{muKayx pArxNakUkx iMdirxyagaLa sAvxmitavxvirabAradeVke? eMdare {\rm --}}

\footnotetext[2]{iMdirxyagaLu huTuTxvAgalU iruvAgalU layavAguvAgalU 
jiVvAdhiVnavAgiruvudariMda iMdirxyagaLige jiVvaveV sAvxmi. 
`tamutAkxrXmanatx pArxNamanUtAkxrXmati' eMba shurxtiyU parxmANavAgide. 
elalxvU pArxNakekx ananxveMdu upAsane mADuvavanige pArxNatAdAtamxyXveV 
baruvudu. upAsakanu pArxNaveV AgiruvudariMda adakekx heVLida 
ananxvelalxvU upAsakanige seVruvudu.}
\begin{shl}
\footnotemark[2]tanamxyA hiVtareV pArxNAsatxnUmxlAsatxninxbanadhxnAH \hfill|| 33 | \\
EvaMvideV hi nAnananxM kiMcidasitxVti dashaRnAtf | \\
PalaM sAyxtApxrXNasAyujayxM savaRM tasayx hi BoVjanamf\hfill|| 34 || 
\end{shl}

%% shloka footnote
\begin{artha}
itara pArxNagaLu muKayxpArxNa savxrUpagaLeV. A pArxNada mUlakavAgiyU pArxNavAyu nimitatxvAgiyU iruvavu. I riVtiyAgi tiLidavanige ananxvalalxdudx yAvudoMdU ilalxveMdu kaMDiruvudariMda pArxNasAyujayxveV PalavAguvudu. avanige elalxvU UTaveV Aguvudu.
\end{artha}

\vishaya{hAgAdare upAsakanu ananxvAgirali eMdare {\rm --}}

\begin{shl}
BavatayxtAtx sa savaRsayx nAnanxM Bavati kasayxcitf | \\
keVvaleV\s vasithxteV\s tatxqqtevxV maqtuyxnA\s pi na giVyaRteV \hfill|| 35 || 
\end{shl}

\begin{artha}
upAsakanu elalxvanunx tinunxva (BoVkatx)nAguvanu. Adare yArigU ananxvAguvudilalx. keVvala BoVkatxqqtavxveV irutitxralu maqtuyxviniMdalU avanu nuMgalapxDuvudilalx.
\end{artha}

\vishaya{pUvaRpakaSx shaMke {\rm --}}

\begin{shl}
adiBxH paridadhateyxVnamashiSayxnatx iti shurxteVH | \\
vidadhAti hayxpAM pAnamapUvaRM shAsatxrXlakaSxNamf \hfill|| 36 || 
\end{shl}

\begin{artha}
UTa mADalapxDuva ananxvanunx UTamADuva pArxNavanunx `adiBxHparidadhAti' eMbudAgiyU `Ena mashiSayxnatx' eMbudAgiyU iruvudariMda apUvaRvAda jalapAnavanunx vidhisuvudu.
\end{artha}

\begin{shl}
shAsarxM muKAyxthaRmeVva sAyxdivxdhayxthaRM yadi kalapxyXteV | \\
yukatxH PalAnuSaknogxV\s sayx vAsoVlABoV vidhiyaRdi \hfill|| 37 || 
\end{shl}

\begin{artha}
shAsatxrX vidhigAgi baMdideyeMdu kalipxsidare muKAyxthaRvuLaLxdedxV Aguvudu. (AvAga Acamanakekx vidhiyu sididhxsuvudu) AvAga pArxNakekx vasatxrXlABavu ide. PalasaMbaMdhavU yukatxvAguvudu.
\end{artha}

\begin{shl}
EkeV ca shAKinoV vayxkatxM vidhirUpamadhiVyateV | \\
manetxrXVNa pArxshanaM cApAmeVkeVSAM shAKinAM matamf \hfill|| 38 || 
\end{shl}

\begin{artha}
kelave shAKeyavaru (CaMdoVgaru) `ashiSayxnAnxcAmeVtf ashitAvxcA cAmeVtf' eMdu vidhisavxrUpavanunx sapxSaTxvAgi adhayxyana mADuvaru. I maMtarxdiMda jalapArxshanavu kelavu shAKeyavarige saMmatavAgide.
\end{artha}

\footnotetext[1]{ililx pUvoRVtatxra pakaSxgaLa sArAMshavanunx vAcakara sawkayaRkAkxgi koTiTxruvevu. `kiMmeVananxM kiMmeVvAsaH' eMdu pArxNavu keVLidadxkekx `yadidaMkiMca AshavxBayxH AkaqmiBayxH satxtetxV\char'263 nanx mApoVvAna iti' - ililx samasatx pArxNigaLU tinunxva ananx AhAravu nAyi, kirxmi payaRMtaviruvudelalxvU pArxNakekx ananxveMdU meVle kuDiyuva niVre vasatxrXveMdu heVLide. hiVgeMdu upAsakanu ciMtisabeVkeMdu heVLiruvudu vasutxsithxti.\\
CAMdoVgayxdalilx `tasAmxdAvx EtadashiSayxnatxH purasAtxcocxVpariSAvx cAcxdidhxH paridadhati' (CAM-5-2-2) hAgeyeV vAjasaneVyadalilx `tadivxdAvxMsaH shorxVtirxyA ashiSayxnatx AcAmanatxyXshitAvx cAcAmanitx EtameVva tadana managanxMkuvaRnotx manayxnetxV ' ide. eraDu shAKegaLalUlx `tasAmxdeVvaMvidashiSayxnAnxcAmeVtf ashitAvx cAcAmeVtf EtameVva tadanamanaganxM kuruteV' iti ililx AcamanavU, Acamana mADida niVrinalilx (anaganxtA ciMtane) vasatxrXBAvane mADuvudU saha pArxNakekx saMbaMdhisidaMte toVrutatxde. ililx eraDU vidhisalapxDuvudeV? athavA AcamanavoMdeV vihitavo? athavA anaganxtA ciMtanavoMdeV vihitavo? eMdu saMshaya baMdare pUvaRpakaSxvidu. eraDU tiLiyuvudariMda eraDU apUvaRvAgiruvudariMdalU vihitave. athavA AcamanavoMdeV vihitavAgide. AcAmeVtf - eMdu vidhiviBakitxyu sapxSaTxvAgide. anaganxtA kiVtaRna keVvala sutxtigAgi eMdu pUvaRpakaSxvu sidAdhxMta-samxqqtiyiMda shudidhxgAgi kataRvayxveMdu vihitavAda sakala kamARMgavAda Acamanavu modaleV parxsidadhx. adanunx anuvAda mADi anaganxtA dhAyxnave vihitavAguvudeMdu sidAdhxMta. `divxjoVnitayxmupasapxqqsheVtf' eMba samxqqti savaRsAmAnayx viSayavAdadudx. elalxrU shudidhxgAgi Acamana mADabeVku eMbudu sAmAnayxvidhi pArxNavidAyx parxkaraNadalilx paThitavAda shurxtiyu pArxNoVpAsaneya aMgavAgide. Adare pArxNavideyxge aMgavAgiyU apUvaRvAgiyU Acamanavu vidhisalapxDuvudilalx. puruSasAmAnayxranunx udedxVshisi vihitavAda AcamanaveV ililxyU toVruvudu. adariMda eraDakUkx vidhiyilalx. adariMda BoVjanakekx modalU AmeVlU mADabeVkAda Acamanavanunx anuvAda mADi AcamanayoVgayxvAda jaladalilx `EtameVva tadana managanxMshuvaRnotxV manayxnetx' eMba vAkayxdiMda anaganxtavx (vasatxrXsavxrUpa)vanunx ciMtisabeVkeMdu pArxNavidAyx saMbaMdhavAgi hosadAgi vidhisalapxDuvudu. Acamanavu vidhiyoVgayxvalalxdadxriMda adara sutxtiyu vivakiSxtavalalx. vasatxrXciMtaneyeV oMdu beVre kirxye. shudidhxgAgi heVLida Acamanavu beVre oMdu kirxye. vasatxrX ciMtaneyu pArxNavAyuvige hodudxkoLuLxvudakAkxgi vihitavAgide. adariMda Acamanavu shudidhxgAgiyU AcACxdanakAkxgiyU ideyeMbudU sariyalalx matutx `yadidaMkiMca' - eMba shurxtiyalilx elAlx vidhavAda ananxvanunx tinanxbeVkeMdu vihitavAgilalx. vidhisuvudakekx sAdhayxvU alalx, hAgeMdu heVLuva vAkayxvilalx, yArigU shakayxvU alalx. elalxvU pArxNadeVvanige ananxveMdu ananxdaqSiTxyanunx vidhiside. adaroMdige Acamanada jaladalilx vasatxrXciMtaneyU vihitavAgide. adariMda apUvaRvAgi jaladalilx AcamaniVya jaladalilx vasatxrXkalapxneyU vihitave horatu jAcnxtavAda Acamanavu punaH vidhiyoVgayxvalalxveMdu sidAdhxMtavu.\\`kAyARKAyxnAdapUvaRmf' eMba barxhamxsUtarxda () sidAdhxMtavidu|| alalxde pArxNoVpAsakanige savARnanx BakaSxNavu vihitavU alalx. tinanxbArada beLuLxLiLx, IruLiLx itAyxdi aBoVjayx BoVjanavanunx mADabeVkAgibaruvudu. aBoVjayx BoVjanada doVSakekx BAgiyAgabeVkAdiVtu. pArxNApAyAdi Apatitxnalilx elilx ananxvanunx tinanxbahudeMdu shAsatxrXvidadxrU Apatitxlalxda kAladalilx savARnanxvanunx tinunxvudakekx shAsatxrX parxmANaveV ilalx. `savARnAnxnumatiH pArxNAtayxyeV tadadxshaRnAtf' eMdu barxhamxsUtarxdalilx savARnanxvanunx tinanxlu anumati koTiTxruvudu pArxNavu hoVguva saMdaBaRvidadxlilx mAtarxveMdu niNaRyavanunx veVdavAyxsaru heVLiruvaru. vishAvxmitarxru shunaka mAMsavanunx tiMdareMbuva katheyu avaru ApatAkxladalilx tiMdareMdu heVLidadxriMda Apatitxlalxdavanige I udAharaNeyu salalxdu. visheVSa vicAravanunx barxhamxsUtarx BASayxdaleVlx tiLiyabahudu.}
\begin{shl}
\footnotemark[1]ananxdashaRnavacecxYkeV vAsoVdaqSiTxM parxcakaSxteV | \\
pArxyatAyxthaRmapAM pAneV daqSeTxVH parxkaraNAdiha | \\
savARBakaSxyXparxsakitxH sAyxdayxdi kamaR vidhitasxyXteV \hfill|| 39 || 
\end{shl}

%% shloka footnote
\begin{artha}
kelavaru ananxdashaRnadaMte (upAsaneyaMte) vasatxrXdaqSiTxyanUnx 
heVLuvaru. shudidhxgAgi jalavanunx pAna mADuvAga adaralilx 
vasatxrXdaqSiTxyanunx parxkaraNadiMda heVLabeVkAgide. 
pArxNoVpAsakanige savARnanx BakaSxNavanunx vidhisuva udedxVshavidadxre 
samasatx aBakaSxyXBakaSxNa parxsaMga baMdiVtu.
\end{artha}

\begin{center}
ililxge shirxVbaqhadAraNayxka BASayx vAtiRkadalilx\\
AraneV adhAyxyadalilx oMdaneV bArxhamxNavu mugidide.\\
|| shirxVdakiSxNAmUtaRyeV namaH ||
\end{center}


\centerline{\vishaya{adhAyxya - 6,   bArxhamxNa - 2}}

%~ \begin{center}
%~ eraDaneV bArxhamxNa
%~ \end{center}

\begin{shl}
yananx saMBAvitaM pUvaRM vasutxpArxdhAnayxheVtutaH | \\
vakutxM tadatarx vakatxvayxM KilakANADxdhikArataH \hfill|| 1 || 
\end{shl}

\begin{artha}
hiMde (jAcnxnakAMDadalilx) yAvudanunx vasutxpArxdhAnayxda 
nimitatxdiMda heVLuvudakekx saMBavisade ididxto adanunx ililx 
KilakAMDa parxkaraNavAdadxriMda heVLabeVkAgide.
\end{artha}

\begin{shl}
sapatxmAvasitAvukatxM mAgaRpArxthaRnamaginxtaH | \\
supatheVti shurxtaM tatarx shurxtAyx mAgaRvisheVSaNamf \hfill|| 2 || 
\end{shl}

\begin{artha}
ELane adhAyxyada koneyalilx aginxyiMda mAgaRpArxthaRne mADiruvudanunx 
heVLide. `agenxV naya supathA rAyeV' eMdu shurxtiyiMdaleV mAgaRvisheVSaNayukatxvAgi sapxSaTxpaTiTxde.
\end{artha}

\vishaya{alilx supathA eMba visheVSaNaveVtakekx? eMdare {\rm --}}

\footnotetext[1]{`saMBavavayxBicArABAyxM sAyxdivxsheVSaNamathaRvatf' eMba 
nAyxyadaMte `niVlamutapxlamf' itAyxdi sathxLadalilx 
rakotxVtapxlavanunx vAyxvaqtitxmADalu niVla eMba visheVSaNavanunx 
koTiTxruvudu. idu sAthaRka, niVlaveMba visheVSaNavu ilalxvAdare 
visheVSayxvu aMdare utapxlavu rakatxvaNaRvAgiyU irabahudeMdu 
athaRvAdiVtu. adakAkxgi I visheVSaNavu sAthaRkavAguvudu. adaraMte 
`supathA' eMdu visheVSaNavu vayxBicAravu parxsakatxvAgalu 
sAthaRkavAgide.}
\begin{shl}
\footnotemark[1]saMBaveV vayxBicAreV ca visheVSaNavisheVSayxyoVH | \\
daqSaTxM visheVSaNaM loVkeV yatheVhApi tatheVkaSxyXtAmf \hfill|| 3 || 
\end{shl}

%% shloka footnote
\begin{artha}
visheVSaNa visheVSayxgaLalilx visheVSaNavu saMBavisuvudidadxrU 
vayxBicAravu (visheVSaNAthaRvu kelavu kaDe ilalxdiruvudU) 
parxsakatxvAgidadxlUlx visheVSaNavanunx parxyoVgisuvudu loVkadalilx 
kaMDide. idu heVgo ililxyU hAgeyeV iruvudeMdu kANuvudu.
\end{artha}

\vishaya{hAgAdare supathA eMbudu vAyxvataRka visheVSaNavAgali? eMdare {\rm --}}

\begin{shl}
supatheVti tatoV yukatxM saMBaveV BUyasAM pathAmf | \\
visheVSaNamatoV vAcAyxH panAthxnaH kamaRheVtavaH \hfill|| 4 || 
\end{shl}

\begin{artha}
aneVka mAgaRgaLu saMBavisuvAga supathA eMba visheVSaNavu yukatxvAgide. 
idariMda kamaRnimitatxvAda mAgaRgaLanunx heVLabeVkAgide. (adakAkxgi I 
bArxhamxNavu baMdide)
\end{artha}

\vishaya{hAgAdare yAva mAgaRgaLu?}

\begin{shl}
dakiSxNoVdagadhoVmAgAR vihitaparxtiSidadhxyoVH | \\
vipAkAH kamaRNoVvARcAyxsatxdevxYrAgayxparxsidadhxyeV \hfill|| 5 || 
\end{shl}

\begin{artha}
vihita kamaRkUkx niSidadhx kamaRkUkx Aguva paripAkagaLenisida 
dakiSxNamAgaR, utatxramAgaR, adhoVgati eMbuvugaLanunx adaralilx 
veYrAgayxvu sididhxsalu avashayx heVLabeVkAgide.
\end{artha}

\vishaya{veYrAgayxveVtakekx? eMdare {\rm --}}

\begin{shl}
nAvirakatxsayx niHsheVSasAMsArikapumathaRtaH | \\
parxvaqtitxmuRkatxyeV tasAmxcuCxrXtAyx yatAnxtatxducayxteV \hfill|| 6 || 
\end{shl}

\begin{artha}
(virakatxnAgadiruva) samasatx saMsAradalilxruva puruSAthaRdalilx 
virakitxyanunx hoMdadavanige mukitxgAgi parxvaqtitxyeV huTuTxvudilalx.
adariMda shurxtiyu parxyatanxpUvaRka A veYrAgayxvanunx heVLuvudu.
\end{artha}

\vishaya{kamaRvipAkadalilx veYrAgayxvu heVge? baMdiVtu? kamaRdiMdaleV sakala puruSAthaRvalalxve? eMdare {\rm --}}

\begin{shl}
shakunxvanitx na kamARNi savaRkAmasamApanamf | \\
niSeVdudhxM vA\s KilAnathARMsatxtaPxlasAyxtiPalugxtaH \hfill|| 7 || 
\end{shl}

\begin{artha}
kamaRgaLu samasatx kAmanegaLanunx pUNaRgoLisalu samathaRvAgilalx. 
athavA elalx anathaRgaLanunx nivArisalu samathaRvAgilalx. avugaLa 
Palavu bahaLa kuSxdarxvAgiruvudariMda samathaRvAgilalx.
\end{artha}

\vishaya{kamaRgaLige mukitxyu Palavalalx {\rm --}}

\begin{shl}
na kamaR kAraNaM muketxVnARginxdaRhajavxrApanutf | \\
kamaRBoVyx janamx niyataM janamx ceVninxvaqRtiH kutaH \hfill|| 8 || 
\end{shl}

\begin{artha}
mukitxge kamaRvu kAraNavAguvudilalx, aginxyu dAha, javxravanunx 
hoVgalADisuvudilalx. kamaRgaLiMda janamxvu kaDADxyavAgi baruvudu. 
janamxvidadxre mukitxyu elilxMda barabeVku?
\end{artha}

\vishaya{AtamxsavxrUpavAda moVkaSx viSayadalilx kamaRvu niSaPxla eMbudakekx shurxtiyeV parxmANa-}

\begin{shl}
na kamaRNA kaniVyasAtx mahatatxvXM cAnatxrAtamxnaH | \\
iti bAhumivoVdadhxqqtayx veVdAnetxYGoVRSaNA kaqtA \hfill|| 9  || 
\end{shl}

\begin{artha}
aMtarAtamxnige kamaRdiMda alapxteyU mahatatxvXvU baruvudilalx; hiVgeMdu veVdAMtagaLu keYyanunx meVlakekxtitx heVLuvaMte GoVSaNe mADiruvavu.
\end{artha}

\vishaya{adeV viSayakekx yAjacnxvalakxyXra samxqqtiyU parxmANavide {\rm --}}

\begin{shl}
na tatarx dakiSxNA yanitx vidayxyeYva tadApayxteV | \\
iti shirxVyAjacnxvalekxyXVna mukatxkaNaThxmudAhaqtamf \hfill || 10 || 
\end{shl}

\begin{artha}
`natatarx dakiSxNAyanitx vidayxyeYva ta dApayxteV' eMdu shirxV yAjacnxvalakxyXru mukatx kaMThadiMda heVLidAdxre.
\end{artha}

\begin{shl}
atoV mukitxM pariVcaCxdiBxrutapxtAtxyXdiviroVdhiniVmf | \\
tayxkAtxvX kamARNayxtheYkAtamxyXjAcnxnaM savARtamxnA\s \s sharxyeVtf \hfill|| 11 || 
\end{shl}

\begin{artha}
adariMda janAmxdigaLige viroVdhiyAda mukitxyanunx apeVkiSxsuvavaru 
kamaRgaLanunx biTuTx EkAtamx savxrUpajAcnxnavanunx savaRvidhadalUlx 
AsharxyisabeVku.
\end{artha}

\vishaya{kamaRvu mukitxya parxtibaMdhakavanunx kaLeyuvudilalx - EkeMdare {\rm --}}

\begin{shl}
tamoVnatxrAyatoV muketxVnARnatxrAyoV\s paroV\s sitx hi | \\
tamoVhatinaR kamaRBoyxV jAcnxnAtAsx vayxcnajxkatavxtaH \hfill|| 12 || 
\end{shl}

\begin{artha}
mukitxge ajAcnxnaveV parxtibaMdhaka. adanunx biTuTx matotxMdu parxtibaMdhakavilalx. ajAcnxnada nivaqtitxyu kamaRgaLiMda Aguvudalalx. jAcnxnadiMdaleV Aguvudu. EkeMdare jAcnxnavu parxkAshakavAdudariMda (adariMdaleV nivaqtitxyu).
\end{artha}

\vishaya{pUvoRVtatxra saMbaMdha viveVcane - `supathA' eMba 
visheVSaNadiMda tiLida aneVka mAgaRgaLanunx nirUpisuvudakAkxgiyU, 
jAcnxnaveV mukitxge kAraNa. kamaRgaLu baMdhanakekx kAraNaveMbudariMda 
adakekx saMbaMdhapaDuvudilalxveMdu heVLuvudakAkxgiyU I bArxhamxNavu 
baMdideyeMdu heVLidAdxyitu. matotxMdu bageyalilx pUvaRgarxMtha 
saMbaMdhavanunx heVLalu aginxhoVtarx parxkaraNadalilx heVLida 
viSayavanunx  anuvAdisutAtxre {\rm --}}

\footnotetext[1]{aginxhoVtarx parxkaraNadalilx niVnu I 
aginxhoVtArxhutigaLa apUvaRveMba sUkaSxmX vasutxgaLa 1. utAkxMtiyanunx 
2. gatiyanunx 3. parxtiSeThxyanunx 4. taqpitxyanunx 5. 
punarAvaqtitxyanunx 6. loVkAMtarakekx edudx niMtiruvudanunx 
tiLidididxVyA? tiLidilalx. adakAkxgi parxtivacanavu `teVvAEteV 
AhutiVhuteV utAkxrXmataH' itAyxdiyAgi baMdide. adaralilx AhutigaLa 
kAyaRvanunx heVLide. aginxhoVtarxdalilx sAyaMkAla pArxtaHkAlagaLalilx 
koDalapxDuva eraDu AhutigaLu ive. avugaLa sUkASxmXMshavAda apUvaRvu 
aMtarikASxdi rUpadalilxruva loVkadalilx tirugi ELuvudariMda Aru 
bageyAgi jagatitxna rUpadalilx pariNAmavAguvudeMdu shurxtiyu apUvaR 
vikAravanunx heVLuvudeMdu tAtapxyaR:}
\begin{shl}
\footnotemark[1]na tu tavxmeVtayoVveVRtethxVtuyxtAkxrXnAtxyXdisavxlakaSxNamf | \\
SaDavxdhaM pariNAmAthaRmaginxhoVtArxhutiVhayoVH \hfill|| 13 || 
\end{shl}

%% shloka footnote
\begin{artha}
Aru bageyAgi (utAkxrXMti modalAda rUpadalilx) aginxhoVtarxda AhutigaLa 
apUvaR pariNAmavAda (jagatatxnunx) niVnu tiLidilalxveMdu 
AkeSxVpayukatxvAda parxshenxyanunx (janakanu yAjacnxvalakxyX 
muniyanunx kuritu) `natu tavxmeVtayoVveRVtathx' eMba vAkayxdiMda 
mADidadxnu.
\end{artha}

\footnotetext[2]{I Ahutiya pariNAmaveV Ahuti lakaSxNavuLaLx kamaRda Pala. 
kataRqvilalxde avananunx avalaMbisade I kamaRvu savxtaMtarxvAgi 
utAkxrXMti modalAda kAyARraMBavanunx mADalAradu. kamaRda kAyARraMBaveV 
kataRqvigAgi, kamaRvu sAdhanavanunx Asharxyiside. alilx A 
parxkaraNadalilx aginxhoVtarxvanunx sutxtisalu Aru bageyanunx 
aginxhoVtarxda kAyaRveMdeV heVLide. parxkaqta I vidAyxparxkaraNadalilx 
adeV kataRqvige baruva PalaveMdu upadeVshisuvudu. kamaRPala 
vijAcnxnavu apeVkiSxtavAgiruvudariMda Aru bageyAgiyU tiLiside. A 
mUlaka paMcAginx videyxyu ililx utatxramAgaRpArxpitxge sAdhanaveMdu 
tiLisuva udedxVsha mADide. elAlx saMsAragatiyanunx upasaMharisidaMte 
Aguvudu. adariMda sakala kamaRvipAkavanunx upasaMharisidaMte Aguvudu. 
adakAkxgi I muMdina KilakAMDada bArxhamxNavu baMdideyeMdu BASayxdaMte 
vivarisida viSayavanunx gamanisabeVku.}
\begin{shl}
\footnotemark[2]itiparxshanxparxtivacasetxV vA itAyxdikaM jagw | \\
loVkaM parxtuyxtithxtaM yAvadaginxhoVtArxdudxtiVhayoVH \hfill|| 14 || 
\end{shl}

%% shloka footnote
\begin{artha}
I riVtiyAgi mADida parxshenxge utatxravAgi `teVvA' itAyxdi 
vacanavanunx hADiruvanu elilxyavarege? eMdare aginxhoVtArxhutigaLu 
paraloVkadalilx tananx Asharxyada utithxtige kAraNavAda 
pariNAmapayaRMtaravAgi.
\end{artha}

\begin{shl}
apUvaRpariNAmoV\s yamaginxhoVtArxKayxkamaRNaH | \\
utAkxrXnAtxyXdigireVhoVkatx AloVkoVtAthxnavAkayxtaH ||  15 ||  
\end{shl}

\vishaya{\mdash athaR bareyabeVku\mdash }

\begin{shl}
AhutoyxVraginxhoVtarxsayx hayxnatxrikASxdiBeVdataH | \\
A loVkoVtAthxnataH shurxtAyx ukAtx\s pUvaRsayx vikirxyA ||  16 ||  
\end{shl}

\vishaya{\mdash athaR bareyabeVku\mdash }

\begin{shl}
tadeVvoVkatxmihA\s \s lamavxyX tadaginxVkaSxNasidadhxyeV | \\
shevxVtakeVturiti garxnathxH para AraBayxteV\s dhunA \hfill|| 17 || 
\end{shl}

\begin{artha}
IvAga adeV jagatatxnenxV hiMde heVLidadxnunx avalaMbisi I KilakAMDadalilx A jagatitxnalilx iDabeVkAda aginxdaqSiTxyu sididhxsuvudakAkxgi `shevxVtakeVtu haRvA AruNiVyaH' itAyxdi muMdina garxMthavanunx AraMBisiruvudu.
\end{artha}

\centerline{\vishaya{baq. 6 - 2 - 1 kaMDike}}

\begin{shl}
shevxVtakeVtuhaR vA AruNeVyaH pacnAcxlAnAM pariSadamAjagAma sa AjagAma jeYvaliM parxvAhaNaM paricArayamANaM tamudiVkASxyXBuyxvAda kumArA3 iti sa BoV3 iti parxtishushArxvAnushiSoTxV\s navxsi piterxVtoyxVmiti hoVvAca || 1 ||
\end{shl}

\vishaya{I AKAyxyikeya tAtapxyaR}

\begin{shl}
pacnAcxginxvidAyx yatenxVna vaqdedhxVnAbArxhamxNAdapi | \\
hitAvx dhanaM ca mAnaM ca labedhxVtuyxkitxH sutxtidhiRyaH \hfill|| 18 || 
\end{shl}

\begin{artha}
bArxhamxNanalalxdavaniMdalU vaqdadhxnAdavanU saha dhana, aBimAna idanenxlAlx biTuTx parxyatanxdiMda paMcAginxvideyxyanunx saMpAdisikoMDaneMdu heVLidudx videyxya sutxtigAgi.
\end{artha}

\vishaya{duyxloVkAdigaLu aginxyalalxdadxriMda avugaLalilx aginxshabadx parxyoVgavu heVge sAdhu? eMdare}

\begin{shl}
pariNAmoV hi pAkeVna pAkashacx na vinA\s ginxnA | \\
dashaRnAtapxriNAmasayx pakAtxHsavaRtarx pAvakaH \hfill|| 19 || 
\end{shl}

\begin{artha}
yAvude oMdu pariNAmavu pAkadiMda Aguvudu. A pAkavU aginxyilalxde 
Aguvudilalx. \footnote[1]{duyxloVka, pajaRnayx muMtAdavugaLa 
adhiVnavAgidudx deVhavu pariNAmagoLuLxvudeMdu `iti tu paMcamAyx mAhutAvApaH puruSa vacasoV Bavanitx' itAyxdi 
shurxtiyalilx I viSayavu tiLiyuvudu. aginxyeV 
paripAkavanunxMTumADuvudeMba niyamadaMte ivugaLalilx aginxshabadx 
parxyoVgavanunx mADiruvudu yukatx.}I deVha rUpavAda pariNAma kaMDiruvudariMda 
pAkamADuvudu elalx sathxLadalUlx aginxyeV.
\end{artha}

\centerline{\vishaya{avataraNike}}

\begin{shl}
samApAtxsheVSavidayxM hi samAvatayxR pitA sutamf | \\
samApAtxsheVSavidoyxV\s siVteyxVvamAhoVtasxsajaR ca \hfill|| 20 || 
\end{shl}

\begin{artha}
taMdeyu samasatxvidAyxBAyxsavanunx pUNaRgoLisi gurukuladiMda 
hiMdakekx baMdiruva magananunx kuritu, niVnu elAlx videyxgaLanunx 
pUNaRgoLisiruveyeMdu heVLidanu, matutx biTuTxkoTaTxnu. 
\end{artha}

\vishaya{taMde gawtamanu putarxnanunx biDalu udedxVshaveVneMdare {\rm --}}

\begin{shl}
nikaSoVpalasaMsethxVSu veVdavitusx pariVkaSxyXtAmf | \\
videyxVyaM yatakxtoV vatasx darxDhimenxV macuCxrXtasayx ca \hfill|| 21 || 
\end{shl}

\begin{artha}
eleY magane? oregalilxna sAthxnadalilxruva veVdajacnxralilx ninanx 
videyxyanunx parxyatanxdiMda pariVkiSxsikoLaLxbeVku. EkeMdare 
naninxMda sharxvaNa mADida shAsatxrXda dADhaRyxkAkxgi.
\end{artha}

\vishaya{itare vidavxtf pariSatutxgaLidadxrU paMcAladeVshada pariSatitxge Itanu Eke baMdidudx?}

\begin{shl}
parxsidAdhx\s tiVva vidavxtAtx pacnAcxlabArxhamxNeVSu hi | \\
tAmeVva pariSadaM tasAmxdAjagAma tavxrAnivxtaH \hfill|| 22 || 
\end{shl}

\begin{artha}
paMcAla deVshada bArxhamxNarugaLalilx vidavxtutx bahaLavAgi ididxteMdu 
parxsidadhxvAgididxtu. adariMda adeV pariSatitxge tavxreyiMda kUDi 
baMdanu.
\end{artha}

\vishaya{putarxnu ililxge baruva udedxVshaveVnu?}

\begin{shl}
pacnAcxlabArxhamxNAcnijxtAvx vidoyxVtakxSeYRkaheVtutaH | \\
rAjAnamapi jeVSAyxmiVtAyxjagAma naqpaM tataH \hfill|| 23 || 
\end{shl}

\begin{artha}
paMcAla bArxhamxNaranunx jayisi videyxya hirimeya nimitatxvAgi 
rAjananunx gelulxtetxVneMdu udedxVshisi rAjanalilxge baMdanu.
\end{artha}

\begin{artha}
\textbf{1{\rm --}2ne kaMDikegaLa sArAMsha-kathAraMBa}\mdash AruNiya maganAda 
shevxVtakeVtu eMbuvanu gurukuladalilx sakalavideyxgaLanunx pUNaRvAgi 
kalitu tananx videyxyanunx saBegaLalilx parxkAshagoLisi yashasasxnunx 
gaLisalu taMdeya AjecnxyaMte pAMcAla deVshada bArxhamxNavidAvxMsara 
pariSatitxge baMdanu. avaranunx gedudx rAjananunx jayisutetxVneMdu 
rAjana saBegU baMdanu. parxvAhaNa eMba jiVvalana putarxneV pAMcAla 
deVshada rAjanAgidadxnu. seVvakariMda seVveyanunx keYgoLuLxtitxruvAga 
I rAjana hatitxra shevxVtakeVtuvu baruvAgaleV Itana vidAyxBimAna 
gavaRviruvudanunx modaleV keVLi tiLididadx rAjanu ivananunx 
viniVtananAnxgi mADabeVkeMdu tiLidu avananunx baMda kUDale avana 
aBipArxyavanunx Uhisi hiVge heVLidanu {\rm --}\\
kumAra, eMdu karedu hedarisidanu. shevxVtakeVtuvU saha koVpisikoMDu 
`BoVH' eMdu kaSxtirxyanige anucitavAgidadxrU hiVge saMboVdhisidanu. 
taMdeyiMda AjAcnxpisalapxTaTxvanAgi baMdiruveyeVnu? eMdu rAjanu 
parxtuyxtatxravitatxnu. idakekx shevxVtakeVtuvu hawdu eMdanu. ninage 
saMshayavidadxre keVLu eMdanu. adakekx rAjanu muMde heVLuvaMte Aru 
parxshenxgaLanunx hAkidanu. yAvudoMdu parxshenxgU utatxravanunx koDalu 
tiLiyade, nanage gotitxlalxveMdeV shevxVtakeVtuvu heVLidanu.
\end{artha}

\begin{shl}
taM jigiVSuM samAyAnatxmunAmxgeVR saMsithxtaM divxjamf | \\
sanAmxgaRparxtipatatxyXthaRM rAjoVvAca savxshAsatxrXtaH \hfill|| 24 || 
\end{shl}

\begin{artha}
jayisalu iciCxsi barutatxliruva mAgaRvanenxV biTiTxruva A 
bArxhamxNananunx kuritu sanAmxgaRda tiLivaLike uMTumADalu tananx 
shAsAtxrXnusAravAgi rAjanu hiVge heVLidanu.
\end{artha}

\begin{shl}
AmanatxrXyAmAsa ca taM kumAra! iti bAlavatf | \\
parxtishushArxva soV\s puyxkotxV BoV ituyxkAtxyX guruM yathA \hfill|| 25 || 
\end{shl}

\begin{artha}
avananunx huDugananunx kareyuvaMte kumAra eMdu rAjanu karedanu. AtanU 
I riVtiyAgi kareyalapxTaTxvanAgi BoVH eMdu guruvanunx heVLidaMte 
rAjananunx kuritu heVLidanu.
\end{artha}

\vishaya{`anushiSoTxVnu' eMba vAkayxvanunx vAyxKAyxnakekx tegedidAdxre {\rm --}}

\begin{shl}
dapoVRtesxVkasamAveVshAnAnxnushiSoTxV\s yamAdarAtf | \\
piterxVti jAtasaMdeVhaH payaRpaqcaCxdatoV naqpaH \hfill|| 26 || 
\end{shl}

\begin{artha}
dapaRvu hececxdudx kANuvudariMda Itanu taMdeyiMda shikiSxsalapxTiTxruvane? athavA ilalxve? eMdu saMshayavu huTiTx rAjanu I parxshenxyanunx mADidanu.
\end{artha}

\begin{shl}
anushiSoTxV\s si kiM pitArx utAhoV neVti BaNayxtAmf | \\
nAnushiSaTxsayx jagati vaqtatxmiVdaqkasxmiVkaSxyXteV \hfill|| 27 || 
\end{shl}

\begin{artha}
taMdeyiMda shikiSxsalapxTaTxvane? athavA ilalxve eMbudanunx heVLabeVku. AjAcnxpisalapxTaTxvanige jagatitxnalilx iMtaha naDavaLike kANutitxlalx.
\end{artha}

\vishaya{OmitAyxdi vAkayxda athaR {\rm --}}

\begin{shl}
bADhaM pitArx\s nushiSoTxV\s simx kiM na pashayxsi majajxyamf | \\
tavxtapxNiDxteVSu saveVRSu paqcaCx mAM yadi shaknakxseV \hfill|| 28 || 
\end{shl}

\begin{artha}
hawdu, taMdeyiMda nAnu vidAyxBAyxsa mADiruvenu nananx jayavanunx niVnu 
kANalilalxveVnu? ninanx elalx paMDitaralUlx keVLu. nananx meVle niVnu 
anumAnapaTiTxdadxre.
\end{artha}

\begin{shl}
EvaM rAjocnxV yathoVkotxVkAtxyX hayxBuyxpeVteV\s nushAsaneV | \\
shevxVtakeVtumathApArxkiSxVtapxcnacx parxshAnxnakxrXmAnanxqqpaH \hfill|| 29 || 
\end{shl}

\begin{artha}
I riVtiyAgi rAjanu heVLida mAtiniMda taMde mADida anushAsanavanunx 
shevxVtakeVtuvu aMgiVkarisalu rAjanu anaMtaraveV I aidu 
parxshenxgaLanunx shevxVtakeVtuvanunx kuritu keVLidanu.
\end{artha}

\section*{baq. 6 - 2 - 2 kaMDike}

\begin{shl}
veVtathx yatheVmAH parxjAH parxyatoyxV viparxtipadayxnAtx 3 iti neVti hoVvAca (3) veVtothxV yatheVmaM loVkaM punarApadayxnAtx 3 iti neVti heYvoVvAca (4) veVtothxV yathAsw loVka EvaM bahuBiH punaH punaH parxyadiBxnaR samUpxyaRtA3 iti neVti heYvoVvAca veVtothxV yatithAyxmAhutAyxM hutAyAmApaH puruSavAcoV BUtAvx samutAthxya vadanitxV3 iti neVti heYvoVvAca (5) veVtothxV deVvayAnasayx vA pathaH parxtipadaM pitaqyANasayx vA yatakxqqtAvx deVvayAnaM vA panAthxnaM parxtipadayxnetxV pitaqyANaM vApi hi na QuSeVvaRcaH shurxtaM devxV saqtiV ashaqNavaM pitaqRNAmahaM deVvAnAmuta matAyxRnAM tABAyxmidaM vishavxmeVjatasxmeVti yadanatxrA pitaraM mAtaraM ceVti nAhamata Ekacnacxna veVdeVti hoVvAca || 2 ||
\end{shl}

\vishaya{modalaneV parxshenxya vAyxKAyxna {\rm --}}

\begin{shl}
anushiSoTxV\s si ceVdUbxrXhi tuleyxV\s pi maraNeV parxjAH | \\
yathA viparxtipadayxneVtx BinanxvatamxRparxBeVdataH \hfill|| 30 || 
\end{shl}

\begin{artha}
elAlx pArxNigaLigU maraNavu samAnavAgidadxrU mAnavaru sAyuvavaru 
Binanx mAgaRgaLa parxBeVdadiMda heVge? beVrebeVreyAgi hoVguvaro (hAge 
niVnu tiLidididxVyA?)
\end{artha}

\vishaya{adaneVnx punaH vivarisuvudu {\rm --}}

\begin{shl}
yeVna kamaRvisheVSeVNa samAnAyAM maqtw parxjAH | \\
anAyx aneyxVna saMyAnitx yathA\s neyxVnAparAsatxthA \hfill|| 31 || 
\end{shl}

\begin{artha}
janaralilx maraNavu samAnavAgidadxrU obabxru beVre mAgaRdiMdalU 
matotxbabxru matotxMdu mAgaRdiMdalU yAva kamaRvisheVSadiMda janaru 
heVge? hoVguvaro, hAge tiLidididxVyA?
\end{artha}

\vishaya{I parxshenxge shevxVtakeVtuvina utatxra {\rm --}}

\begin{shl}
tavxyoVkatxM na viveVdAhaM nAnushiSiTxrihAsitx meV | \\
veVtethxVha tA yathA BUya vataRnetxV parxjA iti  \hfill|| 32 || 
\end{shl}

\begin{artha}
niVnu heVLidadxnunx nAnu tiLidilalx. nanage I viSayadalilx 
shikaSxNavilalx. (eraDane parxshenx) satutxhoVda janaru 
(loVkAMtarakekx hoVdavaru) heVge ililxge punaH baruvaro? hAge niVnu 
tiLidididxVyA?
\end{artha}

\vishaya{idara vivaraNe}

\begin{shl}
yathA yeVna maqtAH satoyxV heVtunA\s neVna ca parxjAH | \\
taM veVtathx sivxnanx veVtuyxkotxV neVti hoVvAca taM punaH \hfill|| 33 || 
\end{shl}

\begin{artha}
janaru maqtapaTaTxvaru yAvudariMda hiMdakekx ililxge baruvaro adanunx 
niVnu tiLidididxVyA? athavA ilalxveMdu parxshinxsidAga shevxVtakeVtuvu 
ilalx, tiLidilalxveMdu punaH utatxrisidanu.
\end{artha}

\vishaya{2ne parxshenx - idara vAyxKAyxna {\rm --}}

\begin{shl}
parxyadiBxrasakaqdUBxteYmaRhadiBxbaRhuBiH sadA | \\
neYvAsw pUyaRteV loVkoV yathA veVtathx tathA\s tarx kimf \hfill|| 34 || 
\end{shl}

\begin{artha}
aneVkAvatiR janaru huTiTx bahumaMdi doDaDxvarAgidadxrU avariMda 
yAvAgalU I paraloVkavu BatiRyAgalilalx. idu heVgeyeMbudanunx niVnu 
tiLidididxVyA?
\end{artha}

\vishaya{utatxra}

\begin{shl}
neVti hoVvAca paqSaTxH sanArxjA paparxcaCx taM punaH | \\
hutAyAmAhutw veVtathx yatithAyxM puruSABidhAH \hfill|| 35 || 
\end{shl}

\begin{artha}
parxshinxsida naMtara Atanu nAnu tiLidilalxveMdu heVLidanu. rAjanu 
avananunx kuritu matetx parxshinxsidanu - yAva saMKeyxya (eSaTxneya) 
Ahutiyalilx hoVma mADidadxralilx jalavu puruSa shabadxvuLaLxdAdxgi 
Aguvudu?
\end{artha}

\vishaya{punaH I parxshenxyanunx vivarisuvudu {\rm --}}

\begin{shl}
Apa Eva samutAthxya puruSAkaqtayoV hutAH | \\
parxvadanitx yathA veVtathx tathA\s \s shu parxtipadayxtAmf \hfill|| 36 || 
\end{shl}

\begin{artha}
hoVma mADidadx jalaveV meVlakekx edudx puruSAkAravuLaLxdAdxguvudu eMdu heVge heVLuvaro adanunx tiLidididxVyA? hAgidadxre shiVGarxvAgi parxtipAdisu.
\end{artha}

\begin{shl}
rAjAnaM neVti hoVvAca nAhaM veVdimx tavxyoVditamf | \\
payaRpaqcaCxdatoV rAjA shAnatxdapaRM divxjaM punaH \hfill|| 37 || 
\end{shl}

\begin{shl}
pathasatxvXM deVvayAnasayx pitaqyANasayx vA\s cnajxsA | \\
veVtathx parxtipadaM kiMvA na veVtisxVtayxBidhiVyatAmf \hfill|| 38 || 
\end{shl}

\begin{artha}
adakekx shevxVtakeVtuvu rAjananunx kuritu niVnu heVLidadxnunx nAnu 
aritilalx eMdanu. anaMtara rAjanu dapaRvu iLiduhoVgidadx I 
bArxhamxNananunx kuritu matetx parxshenx mADidanu. 2. niVnu deVvayAna 
athavA pitaqyANada parxtipatatxnunx tiLidididxVyA? athavA tiLidilalxvo 
Enu heVLu eMdu parxshinxsidanu.
\end{artha}

\vishaya{parxtipatf eMdare {\rm --}}

\footnotetext[1]{yAva kamaRvanunx aMdare aginxhoVtarx muMtAda 
vishiSaTxtama kamaRvanunx mADi, athavA upAsaneyanonx mADi, deVvayAna 
mAgaRvanonx pitaqyANa mAgaRvanonx I janaru hoMduvaro A kamaRveV 
parxtipatf eMdu heVLalapxDuvudu. parxtipadayxte = pArxpayxteV yayA 
aginxhoVtArxdi kirxyayA upAsanA kirxyayA sA parxtipatf eMba 
vuyxtapxtitxyiMda ililx mAgaRdavxyavanunx hoMdisuva kamaR, athavA 
upAsaneyeMdathaR.}
\begin{shl}
parxtipadavxcanasAyxthaRM \footnotemark[1]yatakxqqtevxVtAyxdarAcuCxrXtiH | \\
vAyxcaSeTxV deVvayAnAdiparxtipatwtx \footnotemark[2]kirxyeYva sA \hfill|| 39 || 
\end{shl}
\footnotetext[2]{}

%% shloka footnote
\begin{artha}
parxtipatetxMba padakekx shurxtiyu `yatakxqqtAvx' eMbudAgi AdaradiMda 
vAyxKAyxnisuvudu. adeVneMdare - deVvayAna pitaqyANa mAgaRvanunx 
paDeyalu beVkAda kamaRveV \textbf{parxtipatf} eMdu heVLalapxDuvudu.
\end{artha}

\vishaya{mAgaRdavxyavU pArxmANikavenunxtAtxre {\rm --}}

\begin{shl}
sAvxBUyxha iti mA shaknikxVyaRtoV mAgaRdavxyeV\s pi naH | \\
QuSeVmaRnatxrXsayx sharxvaNamasitx tacacx viBAvayxteV \hfill|| 40 || 
\end{shl}

\begin{artha}
mAgaRgaLalilx eraDaralUlx tananx UheyeMdu ililx niVnu shaMkisabeVDa. 
namage I athaRvanunx parxkAshapaDisuva maMtarxvu shurxtavAgide. 
adanunx parxmANavAgi BAvisidedxVve.
\end{artha}

\vishaya{QugevxVda maMtarx}

\begin{shl}
``devxV saqtiV ashaqNavaM pitqRNAmf............. mAtaraMca"
\end{shl}

\vishaya{I maMtarxda vAyxKAyxna {\rm --}}

\begin{shl}
devxV saqtiV ashaqNavaM sAkASxtasxMbanidhxnwyx divwkasAmf | \\
pitaqRNAM cApi matAyxRnAM mArwgx tAvadhikArataH \hfill|| 41 || 
\end{shl}

\begin{shl}
tABAyxM savaRmidaM gacaCxdayxthAkamaR yathAshurxtamf | \\
sameVti madheyxV BoVgAya roVdasoyxVH kamaRNoV jagatf \hfill|| 42 || 
\end{shl}

\begin{artha}
deVvategaLigU pitaqgaLigU neVrA saMbaMdhisida eraDu mAgaRgaLu iveyeMdu 
keVLidedxVne. avugaLu manuSayxnige tamamx (kamaR, matutx upAsanegaLa) 
adhikArAnusAravAgi laBayxvAda mAgaRgaLu. I eraDu mAgaRgaLiMda I 
jagatetxlalxvU vAyxpatxvAgi naDeyutitxruvudu. tananx kamaRkekx 
yoVgayxvAgiyU upAsanegU yoVgayxvAgiyU iruvaMte dAyxvApaqthivigaLa 
naDuve kamaRda BoVgakAkxgi A mAgaRgaLoMdige seVriruvudu.
\end{artha}

\begin{shl}
tavxdukAtxtapxrXshanxgaNatoV na veVdemxyXVkamapiVritamf | \\
parxshanxM mA mAmataH pArxkiSxVrituyxkAtxvX\s vAkishxrA hayxBUtf \hfill|| 43 || 
\end{shl}

\begin{artha}
rAjane? niVnu keVLida parxshenxgaLa samudAyadalilx oMdanUnx nAnu 
tiLidilalx. adariMda niVnu idakUkx meVle parxshenxyanunx nananxlilx 
keVLabeVDa. eMdu heVLi shevxVtakeVtuvu taleyanunx tagigxsikoMDu 
niMtanu.
\end{artha}

\vishaya{`adheYnaMvasayxtAyx'... itAyxdi vacanada athaR}

\begin{shl}
nidhURtAsheVSakaluSaM shAnatxdapaRM samiVkaSxyX tatf | \\
vasatAyx\s \s manatxrXyAMcakarx uSayxtAmiti pAthiRvaH \hfill|| 44 || 
\end{shl}

\begin{artha}
samasatx kalamxSavU hoVgi, dapaRvU shAMtavAgiruva A 
shevxVtakeVtuvanunx rAjanu tananxlilx vAsamADalu AhAvxnisidanu. 
ililxyeV vAsa mADu eMdanu.
\end{artha}

\vishaya{baq. 6 - 2 - 3 kaMDike}

\begin{shl}
atheYnaM vasatoyxVpamanatxrXyAcnacxkerxV\s nAdaqtayx vasatiM kumAraH parxdudArxva sa AjagAma pitaraM taM hoVvAceVti vAva kila noV BavAnupxrAnushiSATxnavoVca iti kathaM sumeVdha iti pacnacx mA parxshAnxnArxjanayxbanudhxrapArxkiSxVtatxtoV neYkacnacxna veVdeVti katameV ta itiVma iti ha parxtiVkAnuyxdAjahAra ||3||
\end{shl}

\vishaya{`anAdaqtayx\mdash itivAva' idara vAyxKAyxna {\rm --}}

\begin{shl}
hirxVtoV roVSAcacx tadAvxkayxM vasatayxthaRmudiVritamf | \\
anAdaqtayx parxdudArxva yatArx\s \s setxV gwtamaH pitA \hfill|| 45 || 
\end{shl}

\begin{artha}
nAcikepaTaTx shevxVtakeVtuvu koVpadiMdalU vAsamADalu heVLida mAtanunx 
tirasakxrisi elilx taMde gawtamanu iruvano alilxge ODi hoVdanu.
\end{artha}

\begin{shl}
pArxpAyxtha pitaraM roVSAtAsxBayxsUyaM nirAha saH | \\
iti \footnotemark[1]vAveVti vacanaM pUvoVRtatxraviroVdhataH \hfill|| 46 || 
\end{shl}
\footnotetext[1]{I shurxtiya aBipArxyavanunx koDalu horaTa I eraDu 
vAtiRkagaLa Ashayavidu - rAjana AmaMtarxNavanunx tirasakxrisi baMdu 
taMdeya hatitxra hiVge heVLidanu. `niVvu hiMde gurukuladiMda hiMdakekx 
baMdu nananxnunx kuritu' elAlx videyxgaLalUlx sushikiSxtanAgididxVye 
eMdu heVLididxVrA, alalxve? idu Iga nananx BAgakekx suLALxgide. 
EkeMdare? nAnu tamimxMda pUNaRvAgi elAlx videyxgaLalUlx shikaSxNa 
hoMdidedxVneMdu naMbi rAjanidadxlilxge hoVde. Atanu Aru 
parxshenxgaLanunx hAkidanu. nAnu yAvudoMdakUkx utatxra koDalu 
sAdhayxvAgalilalx. tamimxMda sakala vidAyxshikaSxNavu Agidadxre rAjana 
parxshenxge utatxrakoDalu ajAcnxnaveVke? avana parxshenxyu tiLiyade 
idadxkAraNa, nimamx mAtu suLALxyitu.}

%% shloka footnote
\begin{artha}
anaMtara koVpadiMda taMdeyanunx hoVgi kuritu asUyeyiMda kUDiruvaMte 
(beYgaLiMda kUDiruvaMte) Atanu heVLidanu. samAvataRne kAladalilx 
`niVnu elAlx videyxgaLalUlx sushikiSxtanAgiruve' eMdu hiMdumuMdina 
mAtugaLige viroVdhaviruvaMte niVvu heVLididxVralalxve? eMdu heVLidanu.
\end{artha}

\vishaya{iti eMbudara athaR}

\footnotetext[2]{pitaqshikaSxNavu suLuLx eMdu rAjana parxshanx vAkayxdiMda 
nanage tiLiyitu. idu takaRdiMda, rAjana parxshenxyeV takaR heVtu, 
nanage A shikaSxNavAgilalxveMbudu modalu parxtayxkaSxvAgiralilalx, Iga 
tiLiyitu. adariMdaleV `anumAnAdidhx tadagxtiH' eMdu heVLiruvadu.}
\begin{shl}
itiVtuyxkatxparAmashoVR \footnotemark[3]vAkoVvAkayxM naqpeVritamf | \\
\footnotemark[2]apArxtayxkASxyXtikxleVtuyxkitxranumAnAdidhx tadagxtiH \hfill|| 47 || 
\end{shl}
\footnotetext[3]{sholxVkadalilx vAkoVvAkayx eMbudakekx 
takaRshAsatxrXveMdathaRvidadxrU parxkaqta adakekx samAnavAda rAjanu 
heVLida parxshanxrAshiyanunx garxhisabeVku. takaRshAsatxrXveMbathaRvu 
parxkaqtakekx hoMdadu. I parxshenxvAkayxgaLe heVtuvAgidudx adariMda 
pitaqvAkayxvu niVnu elAlx videyxgaLanunx kalitu 
pUNaRgoLisididxyeMbuva mAtu adakekx virudadhxvAgideyeMbudu samxraNege 
baruvudeMdu Ashaya. rAjanu keVLida parxshenxgaLu tiLiyade 
ajAcnxnavidadxdudx kaMDubaMdadadxriMda taMde heVLida mAtu asatayx, 
suLeLxMdu tiLidu baMditeMdu heVLidudx sari.}

%% shloka footnote
\begin{artha}
`iti' eMbudu hiMde heVLidadxnunx parAmashiRsuvudakekx rAjanu heVLida 
parxshanx samudAyaveMba heVtuvAkayx. kila eMba mAtu taMdeya 
shikaSxNavu (pUNaRvAgi) parxtayxkaSxvAgalilalxvAdadxriMda idanunx
sUcisuvudakAkxgi, takaRdiMdaleV alalxde adu tiLidiruvudu.
\end{artha}

\begin{shl}
naqpoVkatxyXBiBavAlilxknAgxdavxcnicxtoV\s simxVti liknagxyXteV | \\
yathAvadanushiSaTxsayx nABiBUtiyaRtoV\s nayxtaH \hfill|| 48 || 
\end{shl}

\begin{artha}
rAjanu heVLida vacanagaLiMda Ada tirasAkxraveMba heVtuviniMda 
nAnu taMdeyiMda moVsahoVdeneMbudu UhisalapxTiTxde. EkeMdare? 
nijavAgiyeV shikiSxtanAgiruvavanige matotxbabxriMda tirasAkxravu 
AgalAradaSeTx.
\end{artha}

\vishaya{adu heVge taMdeyiMda vaMcitanAde? eMdare}

\begin{shl}
taM mAmananushiSeyxYva kimituyxkatxM tavxyA purA | \\
anushiSoTxV\s si puterxVti vacnicxtoV\s simxVtayxtoV matiH \hfill|| 49 || 
\end{shl}

\begin{artha}
A nananxnunx pUNaRvAgi vidAyxBAyxsa mADisade niVvu hiMde`putarxne? 
niVnu vidAyxBAyxsa mADidavanAgididxV', eMdu heVLidadxriMda nimimxMda 
Iga moVsa hoVdeneMdu tiLivaLike baMdide.
\end{artha}

\begin{shl}
kathaM tavxM nAnushiSoTxV\s si bUrxhi tatAkxraNaM mama | \\
pacnacx mAmitayxtoV\s voVcadayxthA hayxnanushAsanamf \hfill|| 50 || 
\end{shl}

\begin{artha}
niVnu heVge shikiSxtanAgilalx? nanage adara kAraNavanunx heVLu eMdu 
taMdeyu heVLutitxralu `paMca mAparxshAnxnf rAjanayx banudhx 
rapArxkiSxVtf' eMdu rAjanu nananxnunx aidu parxshenxgaLanunx 
keVLidanu. (adanunx tiLidilalxveMdanu).
\end{artha}

\begin{shl}
parxshAnxsetxV katameV vatasx yAMsatxvXM na jAcnxtavAnasi | \\
parxshanxparxtiVkAnavadatapxqqSaTxH pitArx samAsataH \hfill|| 51 || 
\end{shl}

\begin{artha}
magu! A parxshenxgaLu yAvuvu? yAvudanunx niVnu tiLiyade hoVdeyo? 
adanunx heVLu eMdu taMdeyu punaH keVLalu saMkeSxVpavAgi parxshenxgaLa 
EkadeVshagaLanunx heVLidanu.
\end{artha}

\vishaya{baq. 6 - 2 - 4 kaMDike}

\footnotetext[1]{I kaMDikeya sArAMsha - taMdeyAda AruNiyu kupitanAda 
magananunx samAdhAnapaDisalu I riVti heVLidanu. eleY vatasx niVnu 
namamxnunx hAge tiLiyabeVku nAnu tiLida vijAcnxnavelalxvanunx ninage 
heVLikoTiTxruveneMdeV tiLidiko. ninagiMtalU pirxVtipAtarxnu nanage 
beVre yAridAdxre? yArigAgi nAnu jAcnxnavanunx bacicxDabeVku? nAnU kUDa 
rAjanu keVLida parxshenxyanunx tiLidavanalalx. adariMda bA hoVgoVNa. 
ililxMda hoVgi rAjanalelxV barxhamxcayaR varxtadalelxV idudx 
jAcnxnakAkxgi vAsisoVNa eMdu taMdeyu shevxVtakeVtuvige heVLidanu {\rm --}}
\begin{shl}
\footnotemark[1]sa hoVvAca tathA nasatxvXM tAta jAniVthA yathA yadahaM kicnacx veVda savaRmahaM tatutxBayxmavoVcaM perxVhi tu tatarx parxtiVtayx barxhamxcayaRM vatAsxyXva iti BavAneVva gacaCxtivxti sa AjagAma gwtamoV yatarx parxvAhaNasayx jeYvaleVrAsa tasAmx AsanamAhaqtoyxVdakamAhArayAcnacxkArAtha hAsAmx aGayxRM cakAra taM hoVvAca varaM BagavateV gwtamAya dadamx iti ||4||
\end{shl}

\begin{shl}
bArxhamxNajAcnxnatoV\s nayxtarx vidAyxM paparxcaCx BUmipaH | \\
na hi bArxhamxNavijAcnxneV kiMcidasitx tavxyA\s gatamf \hfill|| 52 || 
\end{shl}

\begin{artha}
rAjanu bArxhamxNanu tiLida videyxge beVreyeV Ada viSayadalilx 
videyxyanunx parxshinxsidanu. bArxhamxNanige tiLida vijAcnxnaveV 
Agidadxre nininxMda tiLiyada aMshavu oMdU irutitxralilalx.
\end{artha}

\begin{shl}
iteyxVtadadhxqqdayeV kaqtAvx tathA na iti soV\s vadatf | \\
mA shaknikxSAThxsatxtoV mAM tavxM noVkatxM savaRM mameVti hi \hfill|| 53 || 
\end{shl} 

\begin{artha}
I riVtiyAgi aBipArxyavanunx haqdayadalilxTuTxkoMDu saH = A taMdeyu A 
parxkAradalilx namamxnunx hAge tiLiyabeVku eMdu maganige heVLidanu. 
EneMdare? niVnu nanage elalxvanunx heVLikoDalilalxveMdu nananx meVle 
shaMkepaDabeVDa eMdu. (nanage tiLididedxlAlx videyxyanunx 
heVLikoTiTxruveneMdeV tiLi).
\end{artha}

\begin{shl}
perxVhi tatarx gamiSAyxvasatxdivxdAyxlabadhxsidadhx\\
barxhamxcayaRM ca vatAsxyXva AvAM tatarx naqpeV gatw \hfill|| 54 || 
\end{shl}

\begin{artha}
bA, alilxge hoVgoVNa. A videyxya lABavu Agalu barxhamxcayaRdalilx 
nAvibabxrU rAjanalelxV vAsamADoVNa.
\end{artha}

\vishaya{BavAneVva itAyxdi shurxtiya athaR}

\begin{shl}
yAtu tatarx BavAneVva nAhaM taM ganutxmutasxheV | \\
ituyxkatxH sUnunA hirxVtaH savxyameVva jagAma tamf \hfill|| 55 || 
\end{shl}

\begin{shl}
sasaMBarxmaH sa coVtAthxya tasAmx AsanamAharatf | \\
apa AhArayAMcakarx aGayxRpAdAyxthaRsidadhxyeV \hfill|| 56 || 
\end{shl}

\begin{shl}
satakxqqtayx ca yathAshAsarxM rAjA\s tha tamuvAca ha | \\
varaM kAmaM parxyacACxmoV yaH kAmoV vAcniCxtasatxvXyA \hfill|| 57 || 
\end{shl}

\begin{shl}
QuSirAhAtha rAjAnaM kAmitAthaRsayx sidadhxyeV | \\
parxtijAcnxtoV varasAtxvadaBxvatA\s pArxthiRtoV\s pi sanf \hfill|| 58 || 
\end{shl}

\begin{artha}
niVneV alilxge hoVgu. nAnu avana hatitxra hoVgalu iciCxsuvudilalx eMdu 
putarxnu heVLida meVle nAcikepaTuTx tAneV A rAjana hatitxra hoVdanu. 
rAjanU kUDa saMBarxmadiMda toreyiMda meVlakekxdudx Itanige Asanavanunx 
koTaTxnu. niVranunx tarisikoTaTxnu, EtakAkxgi? aGaRyx, 
pAdAyxdigaLanunx apiRsalu. anaMtara shAsatxrXdaMte satAkxravanunx mADi 
avanu A AruNiyanunx kuritu heVLidanu. niVvu yAva varavanunx 
bayasuviro, adanunx nAvu yatheVSaTxvAgi koDuvevu eMdu heVLalu rAjanige 
bArxhamxNanu hiVge utatxrisidanu - niVnu iSATxthaRsididhxgAgi 
naninxMda pArxthiRsalapxDadidadxrU varavanunx (beVDidadxnunx) 
koDutetxVneMdu parxtijecnx mADiruve.
\end{artha}

\vishaya{baq. a.6, bArx. 1, kaMDike 2}

\begin{shl}
sa hoVvAca parxtijAcnxtoV ma ESa varoV yAM tu kumArasAyxnetxV vAcamaBASathAsAtxM meV bUrxhiVti ||5||
\end{shl}

\begin{shl}
sa hoVvAca deYveVSu veY gwtama tadavxreVSu mAnuSANAM bUrxhiVti ||6||
\end{shl}

\begin{shl}
asutx satayxparxtijocnxV\s tarx parxtijAcnxtaM tavxyeVha yatf | \\
deVhi parxshAnxtimxkAM vAcaM yAM matUsxnoVraBASathAH \hfill|| 59 || 
\end{shl}

\begin{shl}
rAjA\s pi tamuvAcAtha deYveVSivxti paraM vacaH | \\
deYveVSavxyaM varaH sidodhxV mAnupANAM varaM vaqNu \hfill|| 60 || 
\end{shl}

\begin{shl}
na hi mAnuSatoV deYvaH pArxthaRniVyoV vijAnatA | \\
mAnuSasUtxcitoV dAtumAdAtuM mAnuSAdavxraH \hfill|| 61 || 
\end{shl}

\begin{artha}
adakekx badalAgi bArxhamxNanu utatxravitatxnu - niVnu 
satayxparxtijecnxyuLaLxvanAgu. niVne yAvudanunx modale parxtijecnx 
mADididxVyo, adanenx koDu. nananx maganige parxshanxrUpavAda yAva 
mAtanunx ADideyo, adanenx koDu, (adeV nananx vara). rAjanu 
bArxhamxNanige heVLidanu, `niVnu yAvudanunx beVDuveyo adu deYva 
varagaLalilx seVride, mAnuSa varagaLalilx oMdu varavanunx keVLu' 
eMdanu. manuSayxriMda deYva varavanunx tiLidavaru pArxthiRsuvudalalx, 
mAnuSa varavanenx koDuvudu ucita. manuSayxriMda sivxVkarisuvudakUkx 
adeV yoVgayxvAda vara - eMdanu.
\end{artha}

\vishaya{baq. a.6, bArx. 2, kaMDike 7}

\begin{shl}
sa hoVvAca vijAcnxyateV hAsitx hiraNayxsAyxpAtatxM goVashAvxnAM dAsiVnAM parxvArANAM paridAnasayx mA noV BavAnabxhoVrananatxsAyxpayaRnatxsAyxBayxvadAnoyxV BUditi sa veY gwtama tiVtheVRneVcACxsA ituyxpeYmayxhaM Bavanatxmiti vAcA ha semxYva pUvaR upayanitx sa hoVpAyanakiVtoyxVRvAsa || 7 ||
\end{shl}

\begin{shl}
mamApayxsetxyXVva tatasxvaRM yadayxdidxtasxsi mAnuSamf | \\
vijAcnxyateV mayeYvA\s \s dw BavatA\s pi parxmAnatxrAtf \hfill|| 62 || 
\end{shl}

\begin{artha}
gawtamanu nanagU adelAlx idedxV ide. yAva yAva mAnuSa vasutxgaLanunx 
koDalu iciCxsideyo, nanage modaleV gotitxde, ninagU beVre parxmANadiMda 
tiLide ide.
\end{artha}

\begin{shl}
na ca tatApxrXthaRniVyaM meV BUri yadivxdayxteV mama | \\
tasAmxdedxYvoV varoV mahayxM diVyatAM nAsatxyXsw mama \hfill|| 63 || 
\end{shl}

\begin{artha}
nanage yAvudu hecAcxgi ideyo, adanunx nAnu beVDatakakxdadxlalx. adariMda deYva varavaneVnx nanagAgi koDabeVkAdadudx, idu nanage iruvudilalx.
\end{artha}

\vishaya{adanunx koDuvudakekx Aguvudilalx nanage loVBa Ase ide - eMdare {\rm --}}

\begin{shl}
bahoVrananAtxpayaRnatxdeYvavitatxsayx loVBataH | \\
mA BUraBayxvadAnayxsatxvXM dAtA BUtevxVha naH parxti \hfill|| 64 || 
\end{shl}

\begin{artha}
bahaLavAgiyU anaMtavAgiyU koneye ilalxdaSuTx irada deYvavitatxda loVBadiMda niVnu nanenxduru ililx dAtaqvAgidudx dAniyAgade irabeVDa. (koDade irabeVDa eMdu gawtamanu heVLidanu)
\end{artha}

\vishaya{`saveYgawtama' itAyxdi vAkayxda tAtapxyaR {\rm --}}

\begin{shl}
deYvaM varaM na saMdAtuM parxtAyxKAyxtuM ca taM divxjamf | \\
iti duHKitavxmApananxsitxVtheVRneVceCxVtuyxvAca tamf \hfill|| 65 || 
\end{shl}

\begin{artha}
deYva varavanunx koDuvudakUkx tirasakxrisuvudakUkx rAjanu 
shakatxnAgalilalx. A bArxhamxNananunx tirasakxrisuvudakUkx Agade 
duHKakekx oLagAgi avananunx kuritu `\footnote[1]{tiVthaR eMdare 
nAyxya, shAsatxrXvihitavAda mAgaRveMdathaR.}tiVtheRVneVcACx' eMdu heVLidanu.
\end{artha}

\vishaya{hAgAdare rAjanu bArxhamxNananunx tirasakxrisuvavanaMte Eke heVLidudx? eMdare {\rm --}}

\footnotetext[2]{`nAtiVtheVRna SaDf bAheyxVna AcAyoVR\s SiRtoV\s pi tasemxY nadadAyxtf' eMdu manAvxdi vAkayxveMdu AnaMdagirigaLu ililx udAharisidAdxre. shiSAyxdi Aru janaralilx obabxru tiVthaR (satApxtarx) venisuvaru. avariMda pArxthiRsalapxTaTx guruvu shiSayxvaqtitxyiMda tananx hatitxra baMdavanige avashayxvAgi videyxyanunx dAna mADabeVku. Aru janaralilx seVradiruvavane atiVthaR, avaniMda pArxthiRsalapxTaTxrU avanige videyxyanunx dAna mADabAradeMdu heVLide. adanunx anusarisi rAjanU kUDa tiVthaRmAgaRdiMdale naninxMda niVnu videyxyanunx paDe eMdu rAjanu ililx heVLidanu.}
\begin{shl}
tiVtheVRna vidAyx deVyeVti \footnotemark[2]nAtiVtheVRneVti cA\s \s gamaH | \\
yatoV\s tasitxVthaRsaqteyxYva matotxV vidAyxM tavxmApunxhi \hfill|| 66 || 
\end{shl}

%% shloka footnote
\begin{artha}
nAyxyadiMda videyxyanunx dAna mADabeVku. nAyxyavilalxde dAna mADabAradeMba Agamavu iruvudu. adariMda nAyxyamAgaRdiMdaleV naninxMda niVnu videyxyanunx paDe eMdanu.
\end{artha}

\vishaya{`upeYmiVti' eMba vAkAyxthaR {\rm --}}

\begin{shl}
shAsArxthaRM sAmxritaH soV\s tha rAjocnxVpeYmiVtayxthoVcivAnf | \\
\footnotemark[3]vAceYva hayxvarAnUpxvaR upayanitx yatasatxtaH \hfill|| 67 || 
\end{shl}
\footnotetext[3]{bArxhamxNanu videyxgAgi rAjana hatitxra barabahude? eMdu 
keVLidare utatxravidu. `vAceYva hayxvarAnf pUveRV' eMdu. udAdxlaka 
modalAda bArxhamxNareV ashavxpatiyeMba rAjanalilx videyxgoVsakxra 
AsharxyaNa mADiruvuduMTu. adariMda modalu huTiTxda bArxhamxNaru 
modaligarAdarU ApatAkxladalilx anaMtara baMda kaSxtirxyaranunx 
vidAyxjaRnegAgi shiSayxvaqtitxyiMda Asharxyisabahudu. hAgeye 
kaSxtirxyarU Apatitxnalilx veYshayxranunx Asharxyisabahudu. Adare 
guruvina pAdoVpasapaRNa guru shushUrxSAdigaLiMda Asharxyisuva 
padadhxtiyilalx. keVvala `nAnu videyxgAgi tamamxnunx avalaMbisi 
baMdidedxVne' eMdu vAcA heVLi upasatitx mADabeVkAdudu hiVgeMdu 
`sa veY gwtama tiVtheVRnecACxdA itUyxpeYmayxhaM Bavanatxmiti vAcA ha semxYva pUravxupayanitx sahoVpAyanakiVtAyxR upAsa' eMba shurxtiya athaRvu, matutx I 67, 68ne vAtiRkagaLa tAtapxyaR.}

\begin{shl}
sa hoVpAyanakiVteyxYRva barxhamxcayaRmuvAsa ha | \\
upeYmiVti hi saMkiVteVRnARnAyxtikxMcicacxkAra saH \hfill|| 68 ||  
\end{shl}

%% shloka footnote
\begin{artha}
rAjanu shAsAtxrXthaRvanunx nenapige taMdukoTaTxmeVle `nAnu shiSayxnAgi ninanxnunx paDeyuve eMdu' bAyi mAtinaleVlx bArxhamxNanu heVLidanu. EkeMdare modalina bArxhamxNaru kaniSaThxranunx (kaSxtirxyAdi AcAyaRpuruSaranunx) vAcA videyxgAgi hoMduvaru aSeTx. (pAdoVpasapaRNa shushUrxSAdigaLiMdalU alalx)

Atanu upAyana kiVtaRnadiMdale barxhamxcayaRvanunx avalaMbisi idadxnu. adariMda `upeYmi' eMdu heVLuvudanunx biTuTx avanu beVre oMdanunx mADalilalx.
\end{artha}

\begin{shl}
sAparAdhaM savxmAtAmxnaM rAjA pariharananxtha | \\
kaSxmayAmAsa tamaqSiM tathA na itivAkayxtaH \hfill|| 69 || 
\end{shl}

\begin{artha}
rAjanu AnaMtara tananxnunx aparAdhiyeMdu tiLidu adanunx pariharisalu yatinxsi A QuSiyanunx `tathAnaH' eMba vAkayxdiMda kaSxmisuvaMte mADidanu.
\end{artha}

\vishaya{A vAkayxda vAyxKAyxna {\rm --}}

\vishaya{baq. a.6, bArx. 2, kaMDike 8}

\begin{shl}
sa hoVvAca tathA nasatxvXM gwtama mAparAdhAsatxva ca pitAmahA yatheVyaM videyxVtaH pUvaRM na kasimxMshacxna bArxhamxNa uvAsa tAM tavxhaM tuBayxM vakASxyXmi koV hi tevxYvaM burxvanatxmahaRti parxtAyxKAyxtumiti || 8 ||
\end{shl}

\begin{shl}
mAnoV\s parAdhinoV maMsAthxsatxva pUveVR pitAmahAH | \\
nAmanayxnatx yathA tadavxdaBxvAnapayxparAdhinaH \hfill|| 70 || 
\end{shl}

\begin{artha}
namamxnunx niVvu aparAdhiyeMdu tiLiyabeVDi. nimamx pUviRkarAda 
pitAmaharu. nananx pUvaRpitAmaharanunx aparAdhigaLeMdu tiLidiralilalx. 
niVvu avaraMte tiLiyabeVDiri eMdu (rAjanu kaSxme keVLidanu).
\end{artha}

\begin{shl}
tavxtasxMparxdAnataH pUvaRM videyxVyaM hi kadAcana | \\
noVvAsa bArxhamxNeV sAdhivxV sAkASxdapi baqhasapxtw \hfill|| 71 || 
\end{shl}

\begin{artha}
ninage I vidAyxdAna mADuva muMce, I utatxmavAda videyxyu yAvatUtx 
bArxhamxNanalilx iralilalx, sAkASxtf baqhasapxtiyalUlx iralilalx||
\end{artha}

\begin{shl}
EvaM gupAtxmapi tu tAM vakASxyXmeyxVvAhamacnajxsA | \\
parxtAyxKAyxtuM samathaRH koV burxvanatxM bArxhamxNaM naqpaH \hfill|| 72 || 
\end{shl}

\begin{artha}
I riVtiyAgi rakiSxsalapxTiTxdadxrU A videyxyanunx nAnu neVrA heVLiye 
koDuvenu. yAva rAjanu I riVti heVLuva bArxhamxNananunx tirasakxrisalu 
samathaRnAguvanu?
\end{artha}

\vishaya{baq. a.6, bArx. 2, kaMDike 8, 9}

\begin{shl}
sa hoVvAca tathA nasatxvXM gwtama mAparAdhAsatxva ca pitAmahA yatheVyaM videyxVtaH pUvaRM na kasimxMshacxna bArxhamxNa uvAsa tAM tavxhaM tuBayxM vakASxyXmi koV hi tevxYvaM burxvanatxmahaRti parxtAyxKAyxtumiti || 8 ||
\end{shl}

\begin{shl}
asw veY loVkoV\s ginxrwgxtama tasAyxditayx Eva samidarxshamxyoV dhUmoV\s haraciRdiRshoV\s knAgxrA avAnatxradishoV visuPxliknAgxsatxsimxnenxVtasimxnanxgwnx deVvAH sharxdAdhxM juhavxti tasAyx AhuteyxY soVmoV rAjA samaBxvati || 9 ||
\end{shl}

\vishaya{nAlakxne parxshenxyaneVnx modalu niNaRya mADalu Enu kAraNaveMdu parxshinxsuvudu {\rm --}}

\begin{shl}
asAviti karxmoV\s BeVdi kasAmxdedhxVtoVritiVyaRtAmf | \\
parxshanxseyxVha catuthaRsayx pArxdhAnAyxdiBxdayxteV karxmaH \hfill|| 73 || 
\end{shl}

\begin{artha}
`asawveY' itAyxdiyAgi heVLida karxmavu yAva kAraNadiMda beVpaRTiTxteMbudanunx heVLabeVku. ililx nAlakxne parxshenxge samAdhAna pArxdhAnayxviruvudariMda karxmavu beVpaRTiTxde (eMbudeV utatxra).
\end{artha}

\vishaya{pArxdhAnayxviruvudariMda nAlakxneya parxshenxyanunx modalina niNaRyakekx tegedukoMDideyeMbudanunx samapiRsuvudu {\rm --}}

\begin{shl}
utapxtetxVsatxdadhiVnatAvxjajxnAmxyatAtx sithxtisatxthA | \\
sithxtayxpAyeV parxyANaM ca shurxtAyx\s BeVdi karxmasatxtaH \hfill|| 74 || 
\end{shl}

\begin{artha}
mAnavara janamxvu nAlakxne parxshenxya niNaRya nimitatxvAgiruvudariMda (pArxdhAnayxvidudx) I parxshenxge parxthama sAthxnaviralu, anaMtara janAmxdhiVnavAda sithxtiyU hAgU sithxtige apAyavuMTAdAga parxyANavU saMBavisuvudu. adariMda shurxtiyalilx pAThakarxmavu beVpaRTiTxtu.
\end{artha}

\vishaya{inunx `a sw veY loVkoVginxH'... itAyxdi maMtarxda 
vAyxKAyxnavu {\rm --}}

\begin{shl}
dUratoV\s muSayx loVkasayx sAyxdasAviti giVriyamf | \\
samidUdhxmAdiBiV rUpeYloVRkayxteV loVkagiVrapi \hfill|| 75 || 
\end{shl}

\begin{shl}
veYshabadxH samxraNAya sAyxdaginxsatxtapxriNAmataH | \\
yata AhavaniVyoV\s ginxduyxRloVkAtamxtayA sithxtaH \hfill|| 76 || 
\end{shl}

\begin{artha}
shurxtiya `asw' eMba padavu dUradalilxruva paraloVkavanunx heVLuvudu. 
samitutx, dhUma modalAda rUpagaLiMda kANisuvudariMda loVkaveMba padavU 
ide. veY shabadxvu samxraNAthaRkavAgide. avugaLa pariNAmadiMda 
aginxyeMdu heVLalapxTitxde, kAraNaveVneMdare? AhavaniVyAginxyu 
duyxloVka rUpadalilxruvudu.
\end{artha}

\vishaya{adu heVgeMdare:-}

\begin{shl}
apUvaRpariNAmoV\s yamaginxhoVtArxKayxkamaRNaH | \\
duyxloVkoVpakarxmoV jecnxVyoV yAvatupxruSasaMBavaH \hfill|| 77 || 
\end{shl}

\begin{artha}
aginxhoVtarxveMba kamaRda apUvaR pariNAmavidu. adu duyxloVkadiMda 
AraMBisi puruSanalilx huTuTxva payaRMtaraveMdu tiLiyabeVku.
\end{artha}

\vishaya{aginxhoVtarx parxkaraNavanunx yoVcisidarU duyxloVkAdigaLu aginx eMdu heVLuvudu yukatx {\rm --}}

\begin{shl}
AhutoyxVraginxhoVtarxsayx yA viBUtiH puroVditA | \\
seYveVha daqSiTxvidhayxthaRM shurxtAyx vAyxKAyxyateV\s cnajxsA \hfill|| 78 || 
\end{shl}

\begin{artha}
matutx aginxhoVtarxda AhutigaLa viBUtiyanunx hiMde yAvudanunx heVLididxto, adaneVnx ililx duyxloVkadalilx oMdAgi kANabeVkeMba upAsanA daqSiTxyanunx vidhisalu shurxtiyu sAkASxtf vAyxKAyxnisiruvudu.
\end{artha}

\begin{shl}
adhAyxtemxV cAdhiyajecnxV ca hayxdhiloVkAdhideYvayoVH | \\
shurxtirAhavaniVyAdeVvAyxRcaSeTxV visatxqqtiM suPxTAmf \hfill|| 79 || 
\end{shl}

\begin{artha}
shurxtiyu AhavaniVyAdi visAtxravanunx \footnote[1]{`natevxVveYtayoVH satxvXmutAkxrXnitxM nagatiM na parxtiSAThxM nataqpitxM na punarAvaqtitxM na loVkaM parxtuyxtAthxyinaM veVtethxti' eMdu aginxhoVtarx parxkaraNadalilx Aru parxshenxgaLa niNaRyakAkxgi utatxra vacanavu baMdide. `teVvA EteV AhutiV huteV (satwyx) utAkxrXmataH teV anatxrikaSxM parxvishataH teV anatxrikaSx mAhavaniV yaMkuvARteV, vAyuM samidhaM mariVciVreVva shukArx mAhutiM | teV anatxrikaSxM tapaRyataH | teV tata utAkxrXmataH || teV diva mAvishataH | teV diva mAhavaniVyaM kuvARteV AditayxM samidhamf' itAyxdiyAgi heVLidudx kaMDide. yajamAnanu maraNa hoMduvAga avana aginxhoVtarxda AhutigaLeraDU sAdhana padAthaRgaLoMdige biTuTx ELuvavu. heVge badukiruvAga yAva sAdhanagaLiMda kUDiyeV yAva AhutigaLu, kUDalapxTaTxvo aMdare AhavaniVyAginx, samitutx, dhUma, aMgAra (keMDa) kiDigaLU matutx Ahuti darxvayxveMba sAdhanagaLiMda kUDidaveMdu tiLididadxvo hAgeye I loVkavanunx biTuTx paraloVkakekx ededxVLuvavu. aginx, samitutx, dhUma, aMgAra, kiDigaLu, Ahuti darxvayx kiSxVrAdidarxvayx eMba rUpadalelx saqSiTxya Adiyalilx avAyxkaqtadasheyalilxyU sUkaSxmXvAda matotxMdu rUpadiMda (parabarxhamx rUpadalilx) irutatxve. I riVtiyAgidudx aginxhoVtarxveMba kamaRvu sAdhanasahitavAgi apUvaR savxrUpavAgi vayxvasithxtavAgidudx punaH saqSiTxkAladalilx aMtarikASxdigaLige AhavaniVyAginx muMtAda rUpadalilx pariNAmagoLisuvudu. idu heVgoV hAgeye IgalU aginxhoVtarxveMba kamaRvu pariNAmavanunx hoMduvudu. I riVtiyAgi aginxhoVtarxda AhutigaLa apUvaR pariNAmavAgi jagatetxlalxvU iruvudeMdu AhutigaLa sutxtigAgiye heVLide. adariMda utApxrXpitx muMtAda Aru padAthaRgaLanunx kamaRparxkaraNadalilx niNaRyiside. parxkaqta I bArxhamxNadalilx kataRqqvige kamaRveVneMbudanunx heVLalu aMtarikaSx loVkadiMda AraMBisi, yoVSitf payaRMtara iruva 5 padAthaRgaLalilx paMcAginx daqSiTxyanunx kataRvayxveMdu heVLi, adu utatxra mAgaRpArxpitxge sAdhanaveMdU, adeV vishiSaTxvAda kamaRPalada BoVgakAkxtgi vidhisalapxDuvudeMdU, tiLisutAtx ililx modalu duyxloVkAginx muMtAda daqSiTxyanunx heVLalu shurxtiyu AraMBisiruvudu.}adhAyxtamx, adhiyajacnx, adhiloVka, adhideYva ivugaLalilx sapxSaTxvAgi heVLiruvudu. (adariMdalU duyxloVkAdigaLu aginxyeMdu tiLiyabahudu)
\end{artha}

\begin{shl}
loVka AhavaniVyoV\s ginxrasAviti vicinatxyeVtf | \\
AditAyxdiSavxpi tathA samidAdisamiVkaSxNamf \hfill|| 80 || 
\end{shl}

\begin{artha}
I paraloVkave AhavaniVyAginxyeMdu ciMtisabeVku. hAgeyeV AditAyxdigaLalUlx samitutx modalAda daqSiTxyanunx iDabeVku.
\end{artha}

\begin{shl}
saminadhxnAtasxmidABxnU rashamxyoV dhUma itayxpi | \\
samininxgaRmasAmAnAyxdaciRrahasatxtheYva ca \hfill|| 81 || 
\end{shl}

\begin{artha}
cenAnxgi parxkAshagoLisuvudariMda Aditayxnu samitetxMdU, samitutxgaLiMda horage baruva hoVlikeyiMda sUyaRna kiraNagaLeV dhUmaveMdU, hagaleV jAvxleyeMdU ciMtisabeVku.
\end{artha}

\begin{shl}
shAnatxtAvxcacx dishoV\s knAgxrAH samididhx pariNAmataH | \\
aciRraknAgxraBAvasayx yatheYvaM BAnuheVtukAH | \\
rashamxyashacx dishashecxYtA AditayxsamidAsharxyAH \hfill|| 82 || 
\end{shl}

\begin{shl}
avAnatxradishasatxdavxdivxkiSxpatxtevxYkaheVtutaH | \\
visuPxliknAgx iti jecnxVyAsatxsimxnanxgwnx yathoVditeV \hfill|| 83 || 
\end{shl}

\begin{artha}
dikukxgaLeV aMgAragaLu (keMDagaLu) kAraNa shAMtavAgiruvudariMda aciR (jAvxle)yeMbudu aMgAraveMbuva dikukxgaLadudx. heVge kiraNagaLu dikukxgaLu Aditayx samitatxnunx Asharxyisiruvavo hAgeyeV Aditayx samititxna pariNAmavAdadxriMda dikukxgaLeV aMgAragaLeMdu heVLalapxTiTxve.

madhayxvatiRyAda mUledikukxgaLu vikiSxpatxvAgiruva nimitatxdiMda visuPxliMgagaLeM(kiDigaLeM)dU tiLiyabeVku. A aMtarikaSxveMba aginxyaleVlx deVvategaLu hoVma mADuvaru.
\end{artha}

\vishaya{hoVma mADuva deVvategaLu yAru? eMdare {\rm --}}

\footnotetext[1]{aSaTxvasugaLu, EkAdasharudarxru, dAvxdasha Aditayxru 
iMdarx, parxjApati, oTuTx 33 deVvategaLu, adhAyxtamx, adhideYva 
modalAda BeVdadiMda BinenxYsi AyAya BAvavanunx hoMdi yajamAnanu 
irutitxralu avana kamaR, upAsanegaLige anusAravAgi AyAya 
kalipxtAginxgaLalilx hoVma mADuva hoVtaqgaLAgiruvaru. parxkaqta ililx 
AdhAyxtimxkavAda pArxNagaLu (vAgAdigaLu) yajamAnanige 
saMbaMdhapaTiTxruvugaLeV deVvategaLeMdu heVLalapxTiTxve. avugaLe 
hoVtaqgaLAgi AdhideYvikavAgi pariNamisi iMdArxdi deVvategaLAgiruvaru.}
\begin{shl}
\footnotemark[1]tarxyasitxrXMshacacx yeV deVvAH suyxsetxV\s dhAyxtAmxdiBUmigAH | \\
hoVtArasatxtarx tatarx suyxH kamaRjAcnxnAnuroVdhataH \hfill|| 84 || 
\end{shl}

%% shloka footnote
\begin{artha}
yAva mUvatutx mUru deVvategaLu sidadhxvAgiruvaro avareV adhAyxtamx 
modalAda sAthxnagaLalilx hoVtaqgaLAgi viMgaDisalapxTiTxve. alalxlilx 
kamaRjAcnxnAnusAravAgi iruvavu.
\end{artha}

\begin{shl}
QutivxgUrxpeVNa teV hAyxsanayxthA pArxkaqtakamaRNi | \\
hoVtAraH pariNAmeVSu tatheYvoVtatxraBUmiSu \hfill|| 85 || 
\end{shl}

\begin{artha}
hiVge joyxVtiSoTxVmAdi pArxkaqta kamaRgaLalilx QutivxkAkxgi 
hoVtaqgaLAgiruvaro hAgeye kamaRda apUvaR pariNAmavAda muMdina 
duyxloVkAdi BUmigaLalilx A yajamAna \footnote[1]{hiMde heVLidaMte 
yajamAnana pArxNaveMbuva iMdirxyagaLu I vayxvahAradalilx 
aginxhoVtarxda hoVtaqgaLAgi AdhAyxtimxkavenisidudx AdhideYvika 
rUpadalilx pariNamisi iMdArxdi deVvategaLAgi avareV duyxloVkAginxyalilx 
hoVma mADuva hoVtaqgaLAdaru. avaru aginxhoVtarxdx Palavanunx anuBava 
mADalu aginxhoVtarxvaneVnx hoVmADidaru. avare aginxhoVtarxda PalavAgi 
pariNamisuvAga PalaBoVgigaLAguvudariMda alalxlilx hoVtaqsAthxnavanunx 
hoMduvaru. duyxloVka, pajaRnayx, paqthiviV, puruSa, yoVSitf eMba 
aginxsAthxnagaLige yoVgayxvAda rUpadalilx pariNAmavanunx hoMdi 
deVvareMdu heVLalapxDuvaru. ivareV hoVtaqgaLu.}pArxNa deVvategaLu 
hoVtaqgaLAgiruvaru.
\end{artha}

\vishaya{`sharxdAdhxM juhavxti' eMdu sharxdedhxyu hoVma darxvayxveMdu heVLidudx heVge yukatx? {\rm --}}

\begin{shl}
AhutoyxVH pariNAmoV\s yamUgarxRsoV\s pUvaRmitayxpi | \\
tasayx sharxdedhxYkaheVtutAvxcaCxrXdAdhx nAmeVti kiVtayxRteV \hfill|| 86 || 
\end{shl}

\begin{artha}
AhutigaLa pariNAmave idu baliSaThxvAda rasa, matutx apUvaRveMdU heVLalapxDuvudu. adu sharxdedhxyeMba oMdeV kAraNadiMda huTiTxdeyAdadxriMda \footnote[2]{idaralilx aginxhoVtarx kamaRdalilx upayukatxvAda payoVdarxvayxvanunx AhavaniVyadalilx hoVma mADidAga aginxyu adanunx BakiSxsuvudu, aginxyu sivxVkarisidudx kaNiNxge kANadaSuTx sUkaSxmX rUpadalilx pariNAma hoMdi kataRqqvAda yajamAnanoDane paraloVkakekx dhUmAdi mAgaR karxmadalilx horaTu aMtarikaSxdalilx seVri duyxloVkavanunx parxveVshisuvudu. I riVtiyAgi AviyAgi sUkaSxmXvAgi pariNAma hoMdida (kiSxVradarxvayxda Ahuti pariNAmavAda) sUkaSxmXvAda jalavanunx kataRqqsameVtavAgidadxdadxnunx sharxdAdhx eMdu shabadxdiMda shurxtiyalilx karedide. soVmaloVkadalilx (caMdarx loVkadalilx) yajamAnanige divayxshariVravanunx uMTumADalu duyxloVkadalilx adu parxveVshisidAdxga adu hoVma mADalapxDuvudeMdu shurxtiyalilx gawNavAgi heVLalapxTiTxde. `sharxdAdhxM juhavxti' eMdu, A sUkaSxmXvAda payoVdarxvayxveV duyxloVkavanunx parxveVshisi caMdarxmaMDaladalilx kataRqqvina shariVravanunx huTiTxsuvudu. idanenxV `deVvAH sharxdAdhxM juhavxti tasAyx AhuteyxYsoVmoVrAjA saMBavati' eMdu heVLiruvudu.}sharxdAdhx eMba hesariniMda kareyalapxTiTxde.
\end{artha}

\vishaya{utatxravAkayxvanunx vAyxKAyxnisuvaru {\rm --}}

\begin{shl}
tasAyxshAcxpAyxhuteVH soVmoV rAjA saMBavatiVti ca | \\
tasAyxBivaqdidhxH saMBUtinaR tavxBUtajaniyaRtaH \hfill|| 87 || 
\end{shl}

\begin{artha}
A sharxdAdhx eMba AhutiyiMda soVmaveV rAjanAguvanu. (pitaqgaLigU bArxhamxNarigU) rAjanAguvanu (caMdarxnAguvanu)|| adara utapxtitx heVgeMdare? idadx vasutxvina aBivaqdidhxye saMBUti (utapxtitx) yeMdu tiLiyabeVku. EkeMdare? ilalxda vasutxvige utapxtitxyu saMBavisuvudeV ilalx.
\end{artha}

\vishaya{I riVtiyAgi modalane payARyada vAkAyxthaRvanunx saMkeSxVpisi upasaMharisuvudu {\rm --}}

\begin{shl}
dwyxraginxH samidAditayxH sharxdAdhx tasimxnihx hUyateV | \\
sUyeVR samidhi diVpAtxyAM sharxdAdhxM juhavxti deVvatAH \hfill|| 88 || 
\end{shl}

\begin{artha}
duyxloVkaveV aginx, AditayxneV samitutx, adaralilx hoVma mADalapxDuva 
darxvayxveMdare sharxdedhx (sUkaSxmXdarxvadarxvayx), sUyaRsamitutx 
uriyutitxralu A aginxyalilx deVvategaLu sharxdedhxyanunx hoVma mADuvaru.
\end{artha}

\vishaya{sharxdAdhxshabadxvanunx kamaRda sUkaSxmXvasutxvinalilx heVge parxyoVgisidudx?}

\footnotetext[1]{BawtikavAda parxpaMcakekx paMcaBUta sUkaSxmXgaLU avashayxvAgi upAdAna kAraNavAgiruvavu. parxkaqta parxdhAnavAgi tegedukoMDidudx AhutidarxvayxkiSxVra, adara sUkaSxmXvAda jalAMshavanunx ililx parxdhAnavAgiyeV heVLide vinaha itara BUtagaLanunx biTiTxlalx. I sUkaSxmXjalavu itara BUta sUkaSxmX vasutxgaLoMdige Bawtika parxpaMcakekx poVSakavAgiyU AsharxyavAgiyU iruvudariMda sharxdAdhx shabadxda athaRvAgiruvavu.}
\begin{shl}
sharxyateVH sharxdadxdhAteVvAR sharxdedhxVtAyxhuviRpashicxtaH \hfill|| 89 || \\
\footnotemark[1]sharxyaNAdAdhxraNAcAcx\s \s paH sharxdAdhxhAvxH kAraNAtimxkAH | \\
BUtAvx\s \s pa iti liknAgxcacx ApaH sharxdAdhxBidhAsatxtaH \hfill|| 90 || 
\end{shl}

%% shloka footnote
\begin{artha}
`sharxyatiVti sharxdAdhx, sharxdadhxdhAtiVti sharxdAdhx' eMdu vidAvxMsaru sharxdAdhxshabadxda vuyxtapxtitxyeMdu karediruvaru. Bawtika parxpaMcakekx AsharxyavAdadxriMdalU poVSakavAdadxriMdalU I sUkaSxmXvAda jalavu sharxdedhxyeMdu hesarAgiruvavu. kAraNarUpavAgiruvadariMdalU `ApaH puruSavAcoVBUtAvx samutAthxyavadanitx' eMba upakarxmaliMgadiMdalU apf sharxdAdhx shabadxdiMda heVLalapxTiTxde.
\end{artha}

\vishaya{sUkaSxmXvAda apf aBivaqdidhxyAguva bage}

\footnotetext[1]{kaqSaNxpakaSxdalilx caMdarxnu kaSxyisutAtx 
amAvAseyxyalilx sUyaRnoLage seVruvanu. adaralilxruva niVriniMda 
shukalxpakaSxdalilx karxmavAgi aBivaqdidhx hoMduvanu. I vaqdidhxyanenx 
hiMde caMdarxna saMBUti (utapxtitx) yeMdu heVLidudx \\`vivasAvxnaMshuBiH sitxVkeSxNXYH adAyajagatoV jalamf |\\
soVmeV mucnacxtayxtheVnudx shacx vAyunADiVmayeYdivxRjaH ||'\\ sUyaRnu 
tiVkaSxNXvAda kiraNagaLiMda jagatitxna niVranunx caMdarxnalilx 
biDuvanu. anaMtara caMdarxnU kUDa vAyunADigaLa rUpadalilx niVranunx 
eLadukoMDu biDuvaneMdu I samxqqtiya athaR. TiVkAkAraru (AnaM-TiVkA) 
udAharisidAdxre. I sholxVkavu meVlina viSayakekx heVge 
hoMdikeyAguvudo? namage tiLiyadu. sholxVkada BAvAthaRvU namage 
tiLiyadAgide. vidAvxMsaru shoVdhisabeVkAgide.}
\begin{shl}
\footnotemark[1]AkaqSaTxM rashimxBisotxVyamAditeyxV parxtitiSaThxti | \\
tasAmxdAditayxgaH soVmaH kiSxVNa ApAyxyateV punaH \hfill|| 91 || 
\end{shl}

%% shloka footnote
\begin{artha}
sUyaRna kiraNagaLiMda meVlakekx AkaSiRsalapxTaTx niVru Aditayxnalilx 
nilulxvudu. adariMda Aditayxnalilxruva caMdarxnu kiSxVNanAgi punaH 
vaqdidhxgoLuLxvanu.
\end{artha}

\vishaya{caMdarxnu teYjasaneMdu yAru heVLuvaro avara matakekx AhutiyiMda caMdarxnu huTuTxvaneMbudu virudadhxvalalxve? eMdare}

\begin{shl}
pariNAmoV hayxpAM soVmaH shiVtAMshusetxVna soV\s mamxyaH | \\
sharxdAdhxhuteVhiR soVmasayx saMBavaH shAsarx ucayxteV \hfill|| 92 || 
\end{shl}

\begin{artha}
shiVtakiraNavuLaLx caMdarxnu jalada pariNAma. adariMda avanu 
jalamayanu. adariMda shAsatxrXdalilx shudadhxvAda AhutiyiMda caMdarxna 
utapxtitxyanunx heVLide.
\end{artha}

\vishaya{hAgAdare manuSayxloVkAginxyalilx caMdarxnu aMgAravAdadudx heVge? {\rm --}}

\begin{shl}
aknAgxrAshacxnadxrXmAsatxsimxnuhxteV\s gwnx soVmasaMBavaH | \\
soVmacanadxrXmasoVreVvaM BeVdaH shAsetxrXVNa dashiRtaH \hfill|| 93 || 
\end{shl}

\begin{shl}
canadxrXmA maNaDxlaM savxcaCxM canadxrXkeVNa mitoV hi saH | \\
soVmasutx maNaDxleV shevxVtoV vadhaRteV harxsateV ca yaH \hfill|| 94 || 
\end{shl}

\begin{artha}
A duyxloVkAginxyalilx hoVma mADida meVle soVmana (caMdarxna) 
utapxtitxyu Agabahudu. caMdarxnu (manuSayx loVkAginxyalilx) 
aMgAravAgabahudu. I riVtiyAgi soVma matutx caMdarxra BeVdavanunx 
toVriside\footnote[2]{caMdarxnu kAraka, adara Pala soVma eMbudeV BeVda, 
`caMdarxmA aknAgxrAH tasAyx AhuteyxY soVmoVrAjA Bavati' eMba shAsatxrXveV I BeVdakekx parxmANa.}.
\end{artha}

\begin{artha}
\footnote[3]{caMdarxmaMDaladalilx caMdarxmA eMba shabadx parxyoVgakekx kAraNa `caMdarxkeVNa mitaH' eMbudu. aMdare aBarxkakekx samAnavAdadadxriMda caMdarx eMdu heVLuvudu. maMDaladoLagiruva puruSanalilx (Atamxnalilx) soVma shabadxvanunx parxyoVgisuvudakekx nimitatx vaqdidhxhArxsagaLanunx hoMdutatxlilxruvudeV. heVge soVmalateyu shukalxpakaSxdalilx vaqdidhxyanunx kaqSaNxpakaSxdalilx kaSxyavanunx hoMduvado hAgeyeV caMdarxmaMDalada puruSanu hoMduvanu. adariMda soVma eMdu gawNavAgi I puruSananunx kareyuvudu. Adare vaqdidhxkaSxyagaLu caMdarxmaMDaladedxV AgiveyeMdu heVLalAgadu. idakekx niyAmakaveVnU ilalx. `Eva meVnAMsatxtarxBakaSxyanitx' eMdu shurxtiyu caMdarxnanenxV BoVgayxveMdu heVLideyAdadxriMda keVvala maMDalaveV BoVgayxvAguvudilalx. caMdarxnanunx paDeda kamiRgaLanunx deVvategaLu upaBuMjisuvaru. Baqtayxranunx parxBugaLu tamamx BoVgakAkxgi seVvArUpadalilx upayoVgisikoMDu anuBavisuvaro hAgeye deVvategaLu caMdarxnalilx caMdarxna shariVradaMtiruva shariVravanunx hoMdiruva kamiRgaLanunx deVvategaLu seVvA mUlaka anuBavisuvareMdu `EvameVnAM.....' eMba meVlina shurxtiya athaR. adariMda deVvategaLige BoVgayxvAgiruvudu maMDaladoLagina puruSaneMde heVLuvudu yukatxveMdu TiVkAkArara aBipArxyavide.}caMdarxneMbudu savxcaCxvAda maMDala. aBarxkakekx (biMkakekx) samAnavAgide. maMDaladalilxruva puruSanu beLaLxge vaqdidhx hoMduvanu, matutx kaSxyavanunx hoMduvanu yAroV avaneV soVmanu.
\end{artha}

\begin{shl}
canadxrXmAH para AditAyxdavARkosxVmaH shurxteVmaRtaH | \\
AditAyxcacxnadxrXmitAyxha neYteV saMvatasxraM tathA \hfill|| 95 || 
\end{shl}

\begin{artha}
matutx caMdarxnu Aditayxna naMtaraveMdU, Aditayxna Icege soVmaveMdU shurxtiya aBipArxyavide. \footnote[1]{deVvayAna mAgaRvanunx tiLisuvAga `teVciRSamaBisaMBavanitx aciRSoV\s haH...........saMvatasxrAdAditayxmf, AditAyxcacxdarxmasaM caMdarxmasoVviduyxtamf' eMdu heVLide. hAgU pitaqyANa mAgaRvanunx heVLuvAga `teV dhUmamaBisaMBavanitx.......AkAshAcacxMdarxmasa meVSa soVmoV rAjA' eMdu caMdarxna naMtara soVmavanunx heVLide. adariMda caMdarx soVmagaLige savxlapx BeVdavideyeMdu vAtiRkada Ashaya. `neYteV saMvatasxra maBipArxpunxvanitx' eMdu saMvatasxra deVvateya hatitxra I dakiSxNa mAgaRdalilxruva kamiRgaLu baruvudilalxveMdu heVLiyU ide.}``AditAyxcacxMdarxmasamf" eMdu CAMdoVgayxdalilxde.
\end{artha}

\vishaya{upasaMhAra {\rm --}}

\begin{shl}
soVmacanadxrXmasoVsatxsAmxdeBxVdaH samavagamayxteV | \\
\footnotemark[1]deVshABeVdAdaBinwnx tAveVSa soVma iti shurxteVH \hfill|| 96 || 
\end{shl}
\footnotetext[1]{savxrUpaBeVdavidadxre deVshavoMdAdadxriMda oMdu eMdu 
heVLuvudu. parxkaqta `pitaqloVkAcacxnadxrXmeVSasoVmoV rAjA' eMba aBeVdashurxti, soVmavu 
caMdarxmaMDaladalelxV iruvudariMda averaDanUnx oMdAgi heVLide.}

%% shloka footnote
\begin{artha}
adariMda soVma matutx caMdarxgaLige BeVdavu tiLiyuvadu. sathxLavu 
oMdAgiruvudariMda averaDU oMdeV, `ESa soVmaH' eMba shurxtiyu parxmANa.
\end{artha}

\vishaya{hAgAdare averaDU Binanxvenunxva shurxtiyu heVge sari? eMdare}

\footnotetext[2]{ililx dhamaRBeVdaveMdare = savxrUpaBeVda.}
\begin{shl}
Binwnx ca \footnotemark[2]dhamaRBeVdeVna tasAmxduBayathA shurxtiH | \\
\end{shl}

\begin{artha}
alalxde dhamaRBeVdadiMda avugaLu BinanxvU Agive. adariMda eraDu bageyalUlx shurxtiyu saMgatavAgide. 
\end{artha}

\vishaya{eraDaneV payARya vAyxKAyxna {\rm --}}

\vishaya{baq. a.6, bArx. 2, kaMDike 10}

\begin{shl}
pajaRnoyxV vA aginxrwgxtama tasayx saMvatasxra Eva samidaBArxNi dhUmoV viduyxdaciRrashaniraknAgxrA hArxdunayoV visuPxliknAgxsatxsimxnenxVtasimxnanxgwnx deVvAH soVmaM rAjAnaM juhavxti tasAyx AhuteyxY vaqSiTxH samaBxvati || 10 ||
\end{shl}

\begin{shl}
pajaRnoyxV\s ginxriti jecnxVyaH samitasxMvatasxraH samxqqtaH \hfill|| 97 || 
\end{shl}

\begin{shl}
saMvatasxreV samidedhxV hi pajaRnayxsayx sameVdhanAtf | \\
dhUmoV\s BArxNiVti sAdaqshAyxdivxduyxdaciRsatxtheYva ca \hfill|| 98 || 
\end{shl}

\begin{shl}
shAnitxvatuRlatoV\s knAgxrAH pajaRnayxshiKinoV\s shaniH | \\
vikiSxpatxtevxYkasAmAnAyxtusxPXliknAgxH satxnayitanxvaH \hfill|| 99 || 
\end{shl}

\begin{shl}
soVmaM juhavxti tatArxgwnx deVvAshAcxtorxVditAH purA | \\
vaqSeTxVshacx saMBavoV\s payxsAmxtf loVkeV\s simxnf sApi hUyateV \hfill|| 100 || 
\end{shl}

\begin{artha}
pajaRnayxveV aginxyeMdU, saMvatasxraveV samitetxMdU tiLiyabeVku. 
saMvatasxravu uriyalapxDalu pajaRnayxvU uriyuvudariMda hiVge 
kalipxside. meVGagaLeV dhUma. sAdaqshayxviruvudariMda viduyxtetxV 
miMceV jAvxle pajaRnAyxginxya aMgAragaLeMdare ashani, siDilu, 
upashAMti vatuRlagaLiMda guDugina shabadxgaLeV kiDigaLeMdU elelxDe 
haraDiruva sAdaqshayxdiMda heVLalapxDuvavu. I aginxyalilx hiMde 
heVLida deVvategaLu modalu hoVmavanunx mADuvaru. adariMda vaqSiTxya 
utapxtitx. adanenxV I BUloVkadalUlx hoVma mADalapxDuvudu.
\end{artha}
