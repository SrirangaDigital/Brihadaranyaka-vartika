\section*{baq 5 a 3 ne bArx - vA (1107 - 1255)}

\section*{baq a - 4 - bArx 3 - kaMDike 18}

\section*{vAtiRka 1102 riMda} 

\begin{artha}
`puMsaH kila tatoV\s BeyxVtayx' eMdu (1089)ne vAtiRkadalilx Atamxnalilx kAmavu baMdu seVruvudeMdu heVLidudx viSayada mUlakavAgiyo? athavA avideyxya mUlakavAgiyo? idaralilx modalina pakaSxvu sariyallx.
\end{artha}

\begin{shl}
AsaknagxsAyx\s \s gatiratoV viSayeVnidxrXyavatamxRnA || \\
nAyxyABAvAdayuketxYva nApayxvidAyx tamAnayeVtf \hfill || 1107 ||  
\end{shl}

\begin{artha}
idariMda viSaya matutx iMdirxyagaLa mAgaRdiMda kAmavu Atamxnalilxge baruvudeMbudu yukatxvalalx. EkeMdare; nAyxyavilalx. matutx avideyxyU adanunx taralAradu.
\end{artha}

\vishaya{adakekx kAraNaveVneMdare-}

\begin{shl}
mithAyxjAcnxnamaqteV nAnayxtAkxyaRM kiMcidapiVSayxteV ||  \\
avidAyxyA yatoV nAta AsaknAgxdAyxhaqtiH savxtaH \hfill || 1108 ||  
\end{shl}
				
\begin{shl}
AtAmxvidAyxparxsidedhxyXYva havidAyx\s pAyxtamxnoV yataH || \\
na savxtaH paratoV vA\s toV vasutxtaH parxtayxgAtamxni \hfill || 1109 ||  
\end{shl}

\begin{artha}
BArxMti jAcnxnavalalxde avideyxge beVroMdu kAyaRvanunx savxlapxvU opupxvudilalx adariMda kAmAdigaLanunx savxtaH Atamxnalilx oyuyxvudakekx Aguvudilalx (avidAyxtAvxtamx vijAcnxna saMshirxtA eMdu (1089)ralilx heVLididxtu) adakekx utatxra AtAmxvideyxye aparxsidadhxvAguvudariMda sariyalalx Atamxnige savxtaH athavA inonxMdariMdaloV vAsatxvavAgi avideyxyU iruvudilalxvAdadxriMda sariyalalx.
\end{artha}

\vishaya{``yatatxdivxjAcnxna mAtimxVyamf'' eMdu (1091)ralilx heVLidadxkekx utatxra-}

\begin{shl}
niHsheVSavikirxyAvagaRparxtiSeVdhashurxteVsatxthA || \\
AtamxjAcnxnaM vikaqtAyx\s \s setxV\s videyxVteyxVtacacx duBaRNamf \hfill || 1110 ||  
\end{shl}

\begin{artha}
samasatx vikAragaLa samudAyavanenxV Atamxnalilx nirAkarisuva shurxtiyu (`na jAyateV mirxyateV vA itAyxdi shurxtiyu) iruvudariMda
\end{artha}

\section*{baq-a4-3-18}

\begin{artha}
AtamxjAcnxnavanunx avideyxyu vikAragoLisiruvudeMdu heVLuvudakekx baruvudilalx.
\end{artha}

\vishaya{hAgAdare sidAdhxMtadalilx parxtayxgAtamxnalilx avideyxyu heVge iruvudeMdu heVLutitxVri? eMdare-}

\begin{shl}
parxtayxkicxdABAvidAyx\s toV hayxvicAritasididhxkA || \\
sidAdhxyateV parxtiVciVyaM pArxkasxmayxgAjxcnXnajanamxnaH \hfill || 1111 || 
\end{shl}

\begin{artha}
idariMda (Atamxnu niviRkAranAdadxriMda) parxtayxkf ceYtanayxda aBAsavAgi \footnote{avideyxyu anivARcayx cidABAsadiMda kUDidedxMdu heVLuvudelalx tatatxvXjAcnxnakekx modalu, parxtayxgAtamxnalilx adu mithAyxBAsavAgi suLALxgi aparoVkaSxvAgi toVruvudeMbudU saha anuBava matutx yukitxyiMda adariMda sidAdhxMtadalilx parxshenxge avakAshavilalx.}tatatxvX vicAravanunx mADadiruva kAladalilx sididhxsuva avideyxyu tatatxvX jAcnxnavu huTuTxva muMce parxtayxgAtamxnalilx idadxMtiruvudu.
\end{artha}

\vishaya{(avideyxya viveVkakAkxgi muMdina garxMthaveMdu heVLidadxkekx badalutatxra)}

\begin{shl}
jAgarxtasxvXpanxviveVkoV\s toV vAkAyxthaRparxtipatatxyeV || \\
vAkAyxthaRparxtipatetxyXYva parxtayxgajAcnxnanihunxtiH \hfill || 1112 ||  
\end{shl}

\begin{artha}
\footnote{avasAthxtarxyadalUlx karxmavAgiyU karxmavilalxdeyU saMcarisidadxriMda jAgarAdigaLalilx kAmakamaRgaLu biTiTxruvudariMda parishudadhxnAda jiVvapadAthaR (tavxMpadAthaRvu) vAkAyxthaRvAda barxheYkayxvu toVruvudakekx beVkAgide, padAthaRjAcnxnavilalxde vAkAyxthaR- jAcnxnavAguvudilalx adariMda uMTAda jAcnxnadiMda ajAcnxnavu nAshavAguvudu, adariMda, tavxMpadAthaR shudidhxyAgi Aguva vAkAyxthaR jAcnxnavilalxda ajAcnxnanAshavAguvudariMda parxteyxVka avidAyx viveVkavu beVkilalx, yadayxpi avideyx adara kAyaRgaLiMda viveVcisalapxTaTx shudadhxtavxMpadAthaRjAcnxna vAkAyxthaRjAcnxnakekx upayukatxvAgide. AdarU avidAyx viveVkavu sheyxVnavAkayxdalilx mADalapxTiTxruvaMte toVruvudilalx, idakekx badalu ajAcnxtavAda barxhamxveMba kAraNa vasutxveMba suSupitxyanenxV jiVvanu paDediruvudu kANutatxdeyeMdu tAtapxyARthaR.}adariMda jAgara matutx savxpanxda viveVkavu vAkAyxthaRvu tiLiyuvudakAkxgiyeV iruvudu, vAkAyxthaRda jAcnxnadiMdaleV parxtayxgAtamxna ajAcnxna nivaqtitxyAguvudu.
\end{artha}

\section*{vAtiRka}

\begin{artha}
avasAthxtarxyadalilx baMda garxMtha saMdaBaRvu inonxMdakekx sheVSa (aMga)vAguvudilalx adariMda sAvxthaRdalelxV = tananx athaRdalelxV payARvasAna hoMduvudalalxve? eMdare-
\end{artha}

\begin{shl}
jAgarxtasxvXpanxsuSupetxVSu saMcAroV\s yaM parxmAnatxrAtf ||  \\
sidodhxV yasAmxdatasatxsimxnAvxkayxM sAyxdanuvAdakamf \hfill || 1113 ||  
\end{shl}

\begin{artha}
jAgara savxpanx suSupitxgaLalilx I saMcAravAgiruvudeMbudu beVre parxmANadiMda (parxtisaMdhAnadiMda) sididhxsuvudu adariMda A viSayadalilx vAkayxvu anuvAdakavAguvudu (vidhAyakavalalx).
\end{artha}

\begin{artha}
idaraMte mahAvAkayxvU jAgarAdi vAkayxdaMte sAvxthaRdalilx tAtapxyaRvilalxdudx adU parxmANAMtaradiMda tiLiyalu yoVgayxvaSeTx? eMdu keVLidare-
\end{artha}

\vishaya{barxhAmxtamxveV jiVva eMba vAkAyxthaRvu beVre parxmANadiMda tiLidilalx eMdu toVrisutAtxre-}

\begin{shl}
karxmasaMcAriNasatxsayx jAgarxtasxvXpanxsuSupitxSu || \\
barxhamxtavxM nAnayxtoV\s jAcnxyi vAkayxM tatapxrXtipatitxkaqtf \hfill || 1114 ||  
\end{shl}

\vishaya{\mdash  athaR bareyabeVku\mdash  }

\begin{shl}
AtamxnoV barxhamxtA sAkASxdabxrXhamxNoV\s pAyxtamxtA savxtaH ||  \\
tatatxvXmasAyxdivAkayxsavx viSayoV\s yamihoVditaH \hfill || 1115 || 
\end{shl}

\begin{artha}
Atamxnige neVra barxhamxsavxrUpavU, barxhamxnige savxta: AtamxrUpavU ililx- I shAsatxrXdalilx tatatxvXmasi muMtAda vAkayxkekx viSayaveMdu ukatxvAgide.
\end{artha}

\begin{artha}
Ivarege mahAmatasxyXvAkayxdiMda AraMBisi meYterxVyiV bArxhamxNa(5) payaRnatxraviruva vAkayxkekx BataqRparxpaMcaru heVLuva saMbaMdhavanunx nirAkarisidAdxyitu, muMde adara mahAmatasxyXvAkayxda vAyxKAyxnavanunx nirAkarisuvudu-
\end{artha}

\begin{shl}
mahAmatAsxyXKayxdaqSATxnatxH savxpanxjAgarxdavasathxyoVH || \\
vAyxKAyxtoV\s payxnayxthA tavxneyxV daqSATxnetxV vAyxcacakiSxreV \hfill || 1116 ||  
\end{shl}

\begin{artha}
mahAmatasxyXveMba daqSATxMtavanunx savxpanx jAgaragaLalilx kAmakamaRgaLa viveVkakAkxgi vAyxKAyxnisidAdxyitu. beVre kelavaru (BataqRparxpaMcaru) I daqSATxMtavanunx beVre riVtiyalilx vAyxKAyxnisuvaru.
\end{artha}

\section*{BataqRparxpaMcara vAyxKAyxna}

\vishaya{modalu vicAravanunx mADalu horaTiTxdAdxre-}

\vishaya{saMshaya}

\begin{shl}
vijAcnxnaM parxsutxtaM tasimxnasxMdeVhoV naH parxjAyateV || \\
kimeVtadwBxtikaM jAcnxnaM BUteVBoyxV\s nayxsayx cA\s \s tamxnaH \hfill || 1117 ||  
\end{shl}

\begin{artha}
hiMde parxsAtxpisida vijAcnxna yAvuduMTo adaralilx namage I saMdeVhavu barutatxde, ideVnu BwtikavAda jAcnxnavo? athavA BUtagaLiMda (mahABUtagaLiMda) BinenxYsida Atamxna jAcnxnavo? eMdu.
\end{artha}

\vishaya{pUvaRpakaSx-(deVhAtamxvAdiyu pUvaRpakaSx)}

\begin{shl}
vijAcnxnaM BwtikaM tAvadUBxtasaMsagaRjanamxnaH || \\
yatheYva bAhayxnayanaparxkAshAthARBisaMgateVH \hfill || 1118 ||  
\end{shl}
				
\begin{shl}
viSayajAcnxnajanemxYvamAtamxnoV BUtasaMgateVH || \\
budidhxVnidxrXyAdisaMsagARdaBivayxkitxH samiVkaSxyXteV \hfill || 1119 || 
\end{shl}
				
\begin{shl}
saMvijAjxcnXnavisheVSasayx tathA sati na kiMcana || \\
vayxtirikatxsayx vijAcnxtuviRjAcnxneV\s sitx parxyoVjanamf \hfill || 1120 ||  
\end{shl}

\begin{artha}
I vijAcnxnavu BwtikavAdadudx, kAraNa? paMcaBUtagaLa saMpakaRdiMda huTiTxdadxriMda
\end{artha}

\section*{baq - a4 - 3 - 18}

\vishaya{BataqRparxpaMcaravAyxKAyxna}

\begin{artha}
heVge bAhayxdaqSiTxyiruvavanige parxkAshakUkx viSayakUkx saMbaMdhavAguvudariMda viSaya jAcnxnavuMTAguvudo, hAgeyeV BUta saMbaMdhadiMda Atamxnige budidhx iMdirxya modalAdavugaLa saMbaMdhadiMda vijAcnxnavu aBivayxkatxvAguvudu kaMDide, saMvidUrxpavAda jAcnxna visheVSavu BwtikavAgalu BUtagaLige beVreyAda Atamxna jAcnxnavAdare EnoMdu parxyoVjanavU ilalx.
\end{artha}

\vishaya{parihAra}

\begin{shl}
iteyxVvaM coVditeV keYshicxdatarx parxtividhiVyateV || \\
vayxtirikotxV na ceVjAjxcnXtA BUteVBoyxV\s BuyxpagamayxteV \hfill || 1121 ||  
\end{shl}

\begin{artha}
I riVtiyAgi kelavaru AkeSxVpa mADalu ililx parxtiyAgi heVLide paMca BUtagaLige beVreyAgi Atamxnu ilalxdeV hoVdare-
\end{artha}

\begin{shl}
tulayxtAvxdUBxtamAtArxNAM savxpanxjAgarxdavasathxyoVH || \\
vijAcnxnajanamx yugapatatxdA pArxponxVti teV dhurxvamf \hfill || 1122 ||  
\end{shl}

\begin{artha}
savxpanx-jAgaradashegaLalilx paMcaBUtasUkaSxgaLu samAnavAdadxriMda 
AvAga EkakAladalilx ninanx matadalilx jAcnxnavu huTaTx beVkAgi baruvudu, idu nishicxta.
\end{artha}

\vishaya{I parxsaMgavu namage iSaTxveMdu pUvaRvAdi heVLidalilx- utatxra-}

\begin{shl}
karxmavaqtetxVsutx boVdhasayx na boVdhoV BwtikasatxtaH || \\
anayxseyxYva hi saMvitAsxyXtasxvXpanxbudAdhxnatxniVDataH \hfill || 1123 || 
\end{shl}
				
\begin{shl}
itayxsAyxthaRsayx vijacnxpetxyXY daqSATxnotxV\s tArxBidhiVyateV ||  \\
karxmeVNa saMcarananxdAyxM mahAmatosxyXV yathA baliV \hfill || 1124 ||  
\end{shl}

\begin{artha}
boVdhavu karxmavAgi huTuTxvudu adariMda boVdhavu Bwtikavalalx, adakekx beVreyAda AtamxnigeVnaye saMvitf (anuBavavu) Aguvudu elilx? eMdare savxpanx-jAgarxtutxgaLa maneyalilx.
\end{artha}

\vishaya{I athaRvanunx tiLisalu idakekx daqSATxMtavanunx heVLide-}

\begin{artha}
nadiyalilx karxmavAgi saMcarisuva baliSaThxvAda mahAmatasxyXvu heVgo hAge eMdu.
\end{artha}

\begin{shl}
matAsxyXBisaMgatiyaRdavxtapxyARyeVNeVha kUlayoVH || \\
karxmeVNa jAcnxtaqsaMbanadhxsatxdavxtasxvXpanxparxboVdhayoVH \hfill || 1125 ||  
\end{shl}

\begin{artha}
ililx eraDu daDagaLalUlx matasxyXkekx karxmavAgi heVgeV saMbaMdhaviruvudo hAgeye savxpanx jAgaragaLalilx karxmavAgi jAcnxna mADikoLuLxva (Atamxni)ge saMbaMdhaviruvudu.
\end{artha}

\vishaya{jAcnxna mADikoLuLxva Atamxnige deVhAtirikatxvAdadalUlx deVhAtamxvAdadalilx ApAdisida ApAdane ideyeMdu shaMkisuvudu-}

\footnotetext[1]{deVhavu AtamxnAdalilx savxpanx jAgaragaLalilx eraDu shariVragaLalUlx EkakAladalilx BoVgavu Aga beVkAgi baruvudeMdu heVLida doVSavu deVhakekx beVreyAgi AtamxnidAdxneMbavAdadalUlx samAna Atamxnu deVha eraDu deVhagaLalUlx vAyxpisidadxriMda EkakAladalilx jAcnxnavAguva saMBavavideyeMbudu ApAdisuvavana Ashaya ``AtamxnoV pi vAyxpitevxV savxpanxbudAdhxnatxyoVH sAninxdAdhxyXtf ywga padeyxVna upalabidhx vijAcnxneVna Bavitavayxmiti hivadanitx
'' (AnaM, TiVke)}
\begin{shl}
\footnotemark[1]vAyxpitAvxdAtamxnoV\s peyxVvaM yugapajAjxcnXtaqteVti ceVtf || \\
\footnotemark[2]BUtamAtArxvaditeyxVvaM samAnaM coVdayxmAvayoVH \hfill || 1126 || 
\end{shl}
\footnotetext[2]{heVge deVhAtamxvAdadalilx eraDu deVhagaLigU utApxdakavAda BUta sUkaSxmXgaLu samAnavAdadxriMda BoVgavu EkakAladalilx Aga beVkAgi baruvudeMdu ApAdisalapxTiTxdeyo hAgeyeV ninanx matadalUlx hiMde heVLida niVtiyiMda ApAdaneyu samAnavAguvudeMdu parxshanx sAmayxvanunx toVriside.}
%%%%%%%shloka footnote[1, 2]
\begin{artha}
Atamxnu vAyxpakanAgiruvudariMda I riVtiyAgi (deVhAtamxvAdadalilx heVLidaMte) EkakAladalilx savxpanx jAgaragaLalilx jAcnxna mADikoLaLxbeVkAda parxsaMgavu baruvudeMdu ApAdisidare (shariVroVtApxdakavAda) BUta sUkaSxmXgaLu iruvaMte namimxbabxrigU parxshenxyu samAnavAguvudu.
\end{artha}

\section*{baq 4 - 3 - 18}

\vishaya{jAcnxna karxmakekx deVvategaLu kAraNaveMba pakaSxnirAkaraNe-}

\begin{artha}
ibabxralUlx jAcnxnavu EkakAladalilx Aguvudu iSaTxvilalxdidadxre karxmavAgi Agalu kAraNa parxshenx mADuvudu.
\end{artha}

\footnotetext[1]{AditAyxdideVvategaLu shariVradalilx anugArxhakarAgi nelasiveyAdadxriMda avareV jAcnxnavu karxmavAgi savxpanx jAgara eraDu shariVragaLalUlx huTaTxlu kAraNareMdu deVvatAtamxvAdiyu heVLuvanu.}
\begin{shl}
kasatxhiR karxmaheVtuH sAyxtf \footnotemark[1]deVvatAheVtukoV\s sutx saH || \\
karxmeVNa vaqtitxsAtxsA sAyxdadhiSAThxtaqtavxkAraNAtf \hfill || 1127 ||  
\end{shl}

%%%%%shloka footnote[1]
\begin{artha}
hAgAdare karxmakekx kAraNa yAvudu? (eMba parxshenx) utatxra-deVvategaLeV nimitatxvAgidudx jAcnxnagaLa karxmavuMTAgideyeMdAgali, EkeMdare: deVvategaLu shariVradalilx adhiSATxtaqvAgiruva kAraNadiMda avugaLalilx karxmavAgi jAcnxna mADikoDuva vAyxpAravididxVtu.
\end{artha}

\vishaya{I deVvatA pakaSxdalUlx jAcnxnagaLu EkakAladalilx AgabeVkAgi baruvudeMdu dUSisuvudu-}

\begin{shl}
atArxpi yugapajAjxcnXnamukatxyoVH pakaSxyoVyaRthA || \\
vAyxpitAvxdedxVvatAnAM sAyxtasxvXpanxjAgarxdavasathxyoVH \hfill || 1128 ||  
\end{shl}

\begin{artha}
ililxyU (AtamxvAyxpaka eMdaMte) deVvategaLU vAyxpakarAdadxriMda savxpanx jAgara eraDu avasethxgaLalUlx meVle eraDu pakaSxgaLalUlx EkakAladalilx jAcnxnavu huTaTxbeVkAgi baruvudu.
\end{artha}

\vishaya{deVvatAvAdiyu jAcnxnada karxmavanunx matotxMdu riVtiyalilx heVLuvudu-}

\begin{shl}
asaMBavAtapxrXyatanxsayx yugapadedxVvatAtamxnaH || \\
nAtaH sAyxduyxgapajAjxcnXnamiti ceVnanx tathA\s pi tatf \hfill || 1129 ||  
\end{shl}

\begin{artha}
\footnote{dhamARdi itara kAraNa sahakAradiMda deVvategaLu jAcnxnavanunxMTu mADuvaru, dhamARdigaLa parxyatanxvu EkakAladalilx naDeyuvudilalx, adu parxyoVjakavAdarU karxmavaritu odaguvudariMda deVvategaLu vAyxpakavAgidadxrU I dhamARdigaLa sahakAradiMdaleV karxmavAgi jAcnxnagaLu huTuTxvaveMdu deVvatAvAdiyu badalu heVLuvanu.}EkakAladalilx dhamARdi parxyatanxgaLu saMBavisuvadilalxvAdadxriMda deVvatAtamxniMda EkakAladalilx jAcnxnavAgalAradeMdu heVLidare- utatxra- \footnote{AdarU jAcnxna karxmakekx deVvateyu kAraNavalalxveMdu dUSisuvudu-}AdarU adu sariyalalx.
\end{artha}

\begin{shl}
pArxhA\s \s tamxpakaSxvAdayxtarx sidadhxM noV yatasxmiVhitamf ||  \\
kutasatxditi ceVnamxtatxH shaqNu savaRM yathoVcayxteV \hfill || 1130 ||  
\end{shl}

\begin{artha}
AtamxpakaSxvAdiyu ililx namamx iSATxthaRvu sididhxsiteMdu heVLidanu, adu heVgeMdare nAnu heVLuvadanenxlalx keVLu. heVge heVLuveno (hAge)
\end{artha}

\begin{shl}
savaRtarx ywgapadayxM sAyxdivxButAvxdedxVvatAtamxnaH || \\
aishavxyARcacx parxyatonxVtathxkAyeVRSavxsAyx\s \s tamxnoV na tu \hfill || 1131 ||  
\end{shl}

\begin{artha}
(deVvatA nimitatxdiMda) huTiTxda jAcnxna elAlx kaDeyalUlx EkakAladalilx huTaTxbeVkAgi baruvudu, idakekx deVvatAtamxvu vAyxpakavAgiruvudeV kAraNa, alalxde parxyatanxdiMda (dhamARdi nimitatxdiMda) huTiTxda kAyaRgaLalilx deVvatege sAvxtaMtarxviruvudariMdalUlx (dhamARdigaLa nimitatxdiMda Aguvudilalx) Adare Atamxnige sAvxtaMtarxyXvilalxdadxriMda \footnote{``ukatxM hi deVvatAyA viButAvxtf parxyatanxkAyeVRSu ywga padayxM, na tu vijAcnxnAtamxnaH''eMdu BataqRparxpaMcaru heVLidAdxre (AnaM-TiVke)}jAcnxnakarxmavu sididhxsuvudilalx.
\end{artha}

\vishaya{I muMde heVLuva kAraNadiMdalU deVvateyu karxmavAda jAcnxnakekx kAraNavalalx-}

\begin{shl}
adhiSeThxVyeVnidxrXyANAM hi savxpanxBUmAvasaMBavAtf ||  \\
vAyxpAroV nAsatxyXtaH savxpenxV deVvatAnAM manAgapi \hfill || 1132 ||  
\end{shl}

\begin{artha}
savxpanx sAthxnadalilx deVvatAdhiSAThxnavAda iMdirxyagaLU iruvudilalxvAdadxriMda deVvategaLige savxpanxdalilx savxlapxvU vAyxpAravu \footnote{tananx adhiSATxnavAda niyamayx iMdirxyagaLeV savxpanxdalilx ilalxdiruvAga avugaLa meVle parxButavxvanunx toVrisuva adhikAra deVvategaLige ilalx.}iruvudilalx.
\end{artha}

\section*{baq a 4 - 3 - 18}

\vishaya{deVvatege karxmavAgi jAcnxnavu huTuTxvudakekx kAraNavilalxvAdadxriMda Atamxnu parxteyxka sididhxsuvanu-}

\begin{shl}
atoV BoVkAtx savxyaMsidodhxV yathoVketxVneYva heVtunA || \\
parxyatAnxywgapadeyxVna savxpanxjAgarxdavxyAdidhxrukf \hfill || 1133 ||  
\end{shl}

\begin{artha}
adariMda BoVkatxyXvAda Atamxnu hiMde heVLida kAraNadiMdaleV aMdare parxyatanxvu EkakAladalilx iruvudilalxvAdadxriMdalelx savxpanx jAgaragaLa eraDU deVhagaLigiMta BinanxnAgi savxyaM parxkAshavAgi iruvuneMdu sididhxsuvudu.
\end{artha}

\vishaya{Atamxnu vAyxpakanAgiruvudariMda parxyatanxvU EkakAladalilx saMBavisuvudariMda jAcnxnavu karxmavaritu huTaTxlu kAraNavilalxveMbudu sariyalalxveMdare-}

\footnotetext[1]{``tasayxhi savxpAnxnatx budAdhxnatxyoVH vAyxpitevxV\s pi yoV\s sw darxSaTxyXtevxVna kataqRtavxM katarxRtavxM parxti yatonxV na sa ywga padeyxVna saMBavati tasAmxtf tasayx payARyeVNa darxSaTxqqtevxV parxvaqtitx ritayxvasithxtaM Bavati'' eMdu BataqRparxpaMcara ukitxyu vAtiRkadalilx anuvadisalapxTiTxde.}
\begin{shl}
\footnotemark[1]AtamxnoV BoVkatxqqtAsididhxjAcnxRtaqtavxM ca parxsidhayxti || \\
vAyxpitevxV\s payxsayx yatonxV ya AtamxnaH kataqRtAM parxti \hfill || 1134 ||  
\end{shl}
				
\begin{shl}
na saMBavatayxsw yatonxV ywgapadeyxVna niVDayoVH || \\
daqSaTxtAvxtapxripAThAyx sAyxtatxsAmxdavxqqtitxriti sithxtamf \hfill || 1135 ||  
\end{shl}

%%%%%shloka footnote[1]
\begin{artha}
Atamxnu vAyxpakanAgidadxrU BoVkftxqqtavxvu jAcnxna kataqRtavxvU sididhxsuvudu, hAgU Atamxna kataqRtavxkekx beVkAda yAva yatanxvu ideyo adU sididhxsuvudu Adare eraDu manegaLalilx (savxpanx jAgara deVhagaLalilx I parxyatanxvu EkakAladalilx saMBavisuvudilalx adariMda oMdAda meVle)
\end{artha}

\begin{artha}
matotxMdu eMba paripATiyiMda jAcnxnavAguvudu kaMDideyAdadxriMda A AtamxniMda karxmavAgiyeV jAcnxnavAguvudu.
\end{artha}

\begin{shl}
aywgapadayxdashiRtavxmAtAmxsitxtevxV\s numeVSayxteV || \\
tathA tadavxyXrikatxtevxV jAgarxtasxvXpanxkulAyataH \hfill || 1136 ||  
\end{shl}

\begin{artha}
Atamxnu deVhakekx beVreyAgidAdxneMbudakekx EkakAladalilxlalxde (karxmavAgi) noVDuvudu (jAcnxna mADikoLuLxvudeMba) anumAna parxmANavu saMmatavAgide, hAgU jAgara savxpanxgaLa gUDAgiruva (eraDu shariVra)gaLiMdalU Atamxnu BinenxYsidAdxneMbudakekx (\footnote{``upalabadhxyXywna padayx mAtAmxsitxtevxV liknAgxmf'' eMdu BataqRparxpaMcara vacanavu, AtamxnidAdxneMbudakekx jAcnxnavu EkakAladalilx Agadiruvudu (aMdare karxmavAgi AguvudeV) heVtuvu. ``AtAmx deVhAdeVBiRnanxH, karxmadarxSaTxyXtAvxtf gaMdharasAdidarxSaTxqqdeV vadatatxvatf'' eMbudu takaR (AnaM-TiVke)}anumAna parxmANavAguvudu).
\end{artha}

\begin{shl}
na ca savxpanxvinimARNeV deVvatAvAyxpaqtiBaRveVtf || \\
yathAsavxM sAthxnamAyAnitx tA maqtisAvxpayoVyaRtaH \hfill || 1137 || 
\end{shl}

\begin{artha}
\footnote{yathAvu:- ``nahi deVvatAnAM sAvxpakAleV savxvAyxpAraH, tA ucacxkarxmiSoVH loVkAnatxraMparxti visheVSakAyARNi AdhAyxtimxkAni hitAvx sAvxni sAthxnAni parxtipadayxnetxV, aninxMvAgapeyxVti itAyxdi shurxteVH'' eMdu BataqRparxpaMcaru deVvategaLige niderxkAladalilx tamamx tamamx vAyxpAravilalxveMbudakekx A deVvategaLu loVkAMtarakekx hoVgalu bayasuva jiVvana AdhAyxtimxkavAda visheVSa kAyaRgaLanunx biTuTx tamamx tamamx sAthxnakekx hoVguvavu, aginxyanunx vAgaBimAni deVvateyu pArxNadeVvateyu vAyuvanunx, cakuSxraBimAni deVvateyu sUyaRnanunx itAyxdiyAgi shurxtiyalilx heVLidaMte tamamx tamamx sAthxnavanunx hoMduvavu, hAgeye savxpanx kAyaRvanunx paDeyalu iciCxsuvAtanige deVvategaLu horage sariyuvavu eMdU heVLidAdxre ``tathA savxpanxkAyaRM parxtipatosxVH deVvatA apakArxnAtx Bavanitx'' eMdu (AnaM- TiVke)}matutx savxpanxvanunx saqSiTxsuvAga deVvatA vAyxpAravu iruvudilalx, EkeMdare; A deVvategaLu maraNa matutx niderxyalilx tamamx tamamx sAthxnavanunx atikarxmisade tapapxde hoMduvavu.
\end{artha}

\vishaya{deVvategaLu Asharxyisade biTiTxruvAga savxpanxkAyaRvu sididhxsuvudariMda jAcnxnavu deVvategaLiMda AgavudilalxveMdu heVLidadxra meVle punaH shaMkisuvudu-}

\begin{shl}
mataM yathA maqtw deVvAsatxyXkAtxvX savxM savxmanugarxhamf || \\
deVhArameBxV punadeVRvA yathAsavxM kuvaRteV kirxyAmf \hfill || 1138 ||  
\end{shl}
				
\begin{shl}
tadavxtusxSuposxVrutakxrXmayx deVvatAH sAvxdhikArataH ||  \\
savxpanxsageVR punasAtxH savxmadhikAraM parxkuvaRteV \hfill || 1139 ||  
\end{shl}

\begin{artha}
heVge maraNAnaMtara deVvategaLu tananx tananx anugarxhavanunx biTuTx hosadAgi matotxMdu deVhavu huTiTxkoMDane adeV deVvategaLu tananx tananx vAyxpAravanunx heVge mADuvavo hAgeye niderxmADalu bayasuvavana deVhadiMda horage horaTu aMdare deVvategaLu tananx adhikAradiMda horage baMdu savxpanxsaqSiTxyalilx punaH AdeVvategaLu tananx adhikAravanunx mADuvavu.
\end{artha}

\begin{shl}
iteyxVvaM coVditeV\s thAtarx parihAroV\s BidhiVyateV || \\
liknagxkAyARnaBivayxketxVnARlaM kAyARya deVvatAH \hfill || 1140 ||  
\end{shl}

\begin{artha}
IriVtiyAgi AkeSxVpavu baMdare anaMtara idakekx parihAravu heVLalapxDuvudu, EneMdare! (deVvatArUpavAda) \footnote{savxpanxdalUlx deVvategaLu barutitxdadxrU jAgaradalilx baruvaMte baralu samathaRrAgilalx, Adare hiraNayxgaBaRna pArxNatatatxvXveMba samaSiTx liMga shariVra rUpavanunx tALiruvaru, deVvatArUpavAda liMga shariVraveMba pArxNatatatxvXkekx kAyaRmADikoDuvaMte aBivayxkitxyiruvudilalx, hAgidadxre jAgarAvasethxgU savxpAnxvasethxgU visheVSaveV ilalxveMtAguvudu, adariMda savxpanxdalilx kAyaRvanunx mADalu samathaRvAgilalxveMdu tAtapxyaR. ``na hi deVvatA liMgaM jahAti sAtavxnaBivayxkitxV liMgeV nAlaM visheVSakAyARya'' eMdu BataqRparxpaMcara ukitxyu I viSayakekx mUla (AnaM-TiVke)}aMgada kAyaRvu aBivayxkatx vAguvudilalxvAdadxriMda savxpanxdalilx kAyaRvanunx mADalu deVvategaLu samathaRrAgilalx.
\end{artha}

\vishaya{hiMde heVLida viSayavanunx daqSATxMtadiMda sapxSaTxpaDisuvudu-}

\begin{shl}
kAraNAnayxsamathARni yathA deVhamaqteV maqtw || \\
savxkAyaRsAyxBiniSapxtwtx tathA savxpenxV\s pi deVvatAH \hfill || 1141 ||  
\end{shl}
				
\begin{shl}
karaNAnayxnapAshirxtayx nAlaM sAvxdhikaqtiM parxti ||  \\
savxpenxV ca karaNABAvaH shurxteyxYva parxtipAditaH \hfill || 1142 ||  
\end{shl}

\begin{artha}
maraNavAdalilx deVhavilalxde iMdirxyagaLu heVge tamamx kAyaRvanunx naDesalu asamathaRvAguvavo hAgeyeV savxpanxdalUlx deVvategaLu iMdirxyagaLanunx avalaMbisade tananx adhikAravanunx naDesalu asamathaRvAguvuvu, savxpanxdalilx iMdirxyagaLilalxveMbudu shurxtiyiMdaleV parxtipAdisalapxTiTxde.
\end{artha}

\begin{shl}
yata EvamataH savxpenxV nA\s \s shaMkA deVvatAH parxti ||  || \\
adhikAroV yatasAtxsAmAtamxnoV deVhasaMgatw \hfill || 1143 ||  
\end{shl}

\begin{artha}
hiVge (deVvategaLige savxpanxdalilx vAyxpAravilalxveMdu sidadhxvAyito) adariMda savxpanxdalilx deVvategaLanunx kuritu jAcnxnakarxmakekx kAraNareMdu shaMkeyU baruvudilalx, Adare Atamxnige sUthxla shariVrada saMbaMdhavu idAdxgaleV A deVvategaLige adhikAraviruvudu.
\end{artha}

\vishaya{savxpanxdalUlx sUthxla shariVra saMbaMdhavideyaSeTx, hAgAdare deVvatA vAyxpAravU irali: eMdu parxshinxsidare-utatxra}

\begin{shl}
muketxvXYva deVhasaMbanadhxM savxponxV\s yaM parxtayxgAtamxnaH || \\
bahiSukxlAyAvacanAdeVtacAcxdhayxvasiVyateV \hfill || 1144 ||
\end{shl}

\begin{artha}
deVha saMbaMdhavilalxdeye parxtayxgAtamxnige I savxpanxvAguvudu. ``bahiSukxlAyAdamaqta shacxritAvx'' eMba vacanadiMda idanunx niNaRyiside
\end{artha}

\vishaya{BataqRparxpaMcara savxpakaSxvanunx upasaMharisuvudu-}

\begin{shl}
nAtoV BwtikameVtatAsxyXnAnxpi sAyxdedxVvatAkaqtamf || \\
ceYtanayxM savxpanxgaM sAvxthaRM shariVradavxyavajaRnAtf \hfill || 1145 ||  
\end{shl}

\begin{artha}
adariMda I vijAcnxnavu Bwtikavalalx, deVvategaLiMdalU Aguvudilalx, savxpanxdalilxruva ceYtanayxvu sAvxthaR = savxtaMtarx kAraNaveMdare : eraDu shariVragaLU iruvudilalx, adariMda
\end{artha}

\footnotetext[1]{aMga shariVravAda samaSiTxyalilx heVLida layakUkx beVreyAgi deVvategaLa layavanunx `niranugarxhamf'eMbudariMda sUciside.}
\footnotetext[2]{iMdirxyagaLu saha savxpanxdalilx laya hoMdiveyeMdu `niSakxraNamf' eMbudariMda sUciside.}
\footnotetext[3]{suSupitxgU savxpanxkUkx vayxtAyxsavanunx sUcisalu vAsanA mAtorxVpAdhiyeMdu heVLide.}
\begin{shl}
\footnotemark[1]niranugarxhaM \footnotemark[2]niSakxraNaM \footnotemark[3]vAsanoVpAdhimAtarxkamf || \\
savxpanxdashaRnameVtatAsxyXtf \footnotemark[4]pxrXtayxgidhxVvasutxsaMsharxyamf \hfill || 1146 ||  
\end{shl}
\footnotetext[4]{idu sAkiSx mAtarxkekx parxkAshavAguvudeMdu I visheVSaNadiMda tiLidide.}

%%shloka footnote[1, 2, 3, 4]
\begin{artha}
deVvatAnugarxhavilalxdeyU iMdirxyagaLU ilalxdeyU, saMsAkxra mAtarxveV upAdhiyAgidudx I savxpanx jAcnxnavu Aguvudu, adu parxtayxkf savxrUpa budidhxyeMba vasutxvanenxV avalaMbiside.
\end{artha}

\vishaya{I BataqRparxpaMca vAyxKAyxnadalilx deVhAtamxvAdavanunx nirAkarisalu horaTu naDuve deVvateyu karxmavAgi jAcnxnavu namamx shariVradalilx huTaTxlu kAraNaveMbudanunx shaMkisi nirAkarisidudx sariyalalxveMdu heVLuvaru-}

\begin{shl}
paqthaketxvXV vA\s paqthaketxvXV vA BUteVBoyxV deVvatAtamxnaH ||  \\
kimathaRM deVvatAshaknAkx kirxyateV hirugAtamxnaH \hfill || 1147 ||  
\end{shl}

\begin{artha}
deVvatAtamxvu BUtagaLigiMta BinanxveMdarU, aBinanxveMdarU Atamxnige beVreyAgi deVvateyeV jAcnxnada karxmakekx kAraNaveMta A shaMkeyanunx EtakAkxgi mADuvudu?
\end{artha}

\vishaya{Iga sheyxVna vAkayxkekx BASayxdalilx heVLida saMbaMdhavanunx heVLalu AraMBiside}

\begin{shl}
matasxyXdaqSATxnatxvacasa EtAvataPxlamiSayxteV || \\
maqtuyxrUpamidaM savaRM kAyaRM ca karaNAni ca \hfill || 1148 ||  
\end{shl}

\begin{artha}
matasxyXdaqSATxMtada vacanakekx iSeTxVPalaveMbudu saMmatavAgide adeVneMdare:- kAyaR matutx karaNagaLu (shariVra iMdirxyagaLu) I elalxvU maqtuyxvina rUpavu (Atamxna dhamaRvalalx)
\end{artha}

\begin{shl}
tABAyxM vilakaSxNasAtxvXtAmx savxpenxV yoV\s yaM parxpashicxtaH || \\
sUthxlasUkaSxmXshariVrABAyxM keVvalashecxVtanoV\s kirxyaH \hfill || 1149 ||  
\end{shl}

\begin{artha}
savxpanxdalilx yAva I AtamxnanenxV sUthxla sUkaSxmXveMbuva shariVragaLigU vilakaSxNaneMdU keVvala, ceVtana kirxyA shUnayxneMdU visAtxravAgi heVLididxto avaneV A eraDakUkx vilakaSxNavAgiruva Atamxnu.
\end{artha}

\vishaya{alalxde parxyoVjakavAda kAmakamaRgaLa saMbaMdhavAdare kAyaRkaraNagaLanunx punaH Atamxnu sivxVkarisabahudaSeTx; eMdare-}

\begin{shl}
tatheYva kAmakamaRBAyxmatayxnatxM sAyxdivxlakaSxNaH || \\
parxyoVjakABAyxM deVhasayx tatheYva karaNasayx ca \hfill || 1150 ||  
\end{shl}

\begin{artha}
hAgeye kAmakamaRgaLeMba parxyoVjakagaLigU bahaLa vilakaSxNanAgiruvanu, adariMda baruva shariVra matutx iMdirxyagaLigU vilakaSxNanAgiruvanu (beVpaRTiTxruvanu).
\end{artha}

\begin{shl}
avidAyxkaqtameVvAtaH saMsAritavxM na tu savxtaH || \\
atoV\s vidAyxsamuciCxtwtx mukitxH sAyxtapxramAtamxnaH \hfill || 1151 ||  
\end{shl}

\begin{artha}
adariMda saMsArasaMbaMdhavu Atamxnige avideyxyiMdAdadudx, savxtaH savxrUpadalilxruvudalalx, adariMda avideyxyu nAshavAdare mAtarx Atamxnige mukitxyAguvudu.
\end{artha}

\footnotetext[1]{pUvaRvAkayx ``savA ayaM puruSoV jAyamAnaH'' itAyxdiyAgi AraMBisi matasxyXvAkayx payaRMtaviruva vAkayx, Idaqsha:- eMdare shariVra iMdirxyagaLu kAmaka kamaRgaLu ivugaLilalxde savxyaM parxkAshavAgi savxtaH saMsAriyAgirada parxtayxgAtamxneMdathaR}
\begin{shl}
ituyxkatxH \footnotemark[1]pUvaRvAkeyxVna samudAyAdhaR IdaqshaH || \\
pwvARpayaRM samiVkASxyX\s \s tAmx niSikxrXyaH keVvaloV\s davxyaH \hfill || 1152 ||  
\end{shl}

%%%%%shloka footnote[1]
\begin{artha}
I riVtiyAgi hiMdina vAkayxdiMda I bageya parxyoVjaka sahitavAda shariVra iMdirxyagaLige vilakaSxNaneMbuva motatxvAda athaRvu heVLalapxTiTxde, adeVneMdare:-
pUvARparavanunx cenAnxgi noVDi (pariVkiSxsi) Atamxnu kirxyAshUnayxnU, keVvalanU, adivxtiVyanU eMdu.
\end{artha}

\vishaya{muMdina garxMthakekx piVThike-}

\begin{shl}
apAsatxkAyaRkaraNakamARsaknogxV na tu kavxcitf || \\
AtomxVpapAditaH sAkASxdavxcaseyxVkatarx yatanxtaH \hfill || 1153 ||  
\end{shl}
				
\begin{shl}
sAsaknagxshacx samaqtuyxshacx sakAyaRkaraNasatxthA || \\
avidayxyA yatoV boVdha AtAmx\s yamupalakaSxyXteV \hfill || 1154 ||  
\end{shl}

\begin{artha}
kAyaR, karaNa, kamaR, kAmagaLu ilalxda Atamxnanunx elilxyU neVra oMdu vacanadalilx yatanxdiMda upapAdane mADalilalx'' EkeMdare:- kAmasahitanAgiyU, maqtuyxsahitanAgiyU, kAyaRkaraNagaLiMda kUDiyU I Atamxnu ecacxrinalilx ajAcnxnadiMda kUDiruvaMte kaMDu baruvanu\footnote{kAmAdigaLalilx yAvudAdaroMdu biTuTx hoVdarU matotxMdu ilalxveMbudu athARtf sididhxsuvudu, AdarU sAkASxtf sapxSaTxvAkayxdiMda sididhxsilalxveMbudanunx tiLisalu `sAkASxtf'eMdu ililx heVLide 2 adariMda jAgara dasheyanunx heVLuva vAkayxdalilx hiMde heVLida parxtayxgAtamxnu parxtipAdisalapxTiTxlalx eMdu ililx sheVSa pUraNa mADikoLaLxbeVku.}.
\end{artha}

\vishaya{savxpanx vAkayxdalilx aMtaha Atamxnu sididhxsilalxve? eMdare-}

\begin{shl}
savxpenxV tu vAsanAkAmasaMykotxV maqtuyxvajiRtaH || \\
saMparxsAdeV parxsananxshacx tathA\s saknogxV\s pi viVkaSxyXteV \hfill || 1155 ||  
\end{shl}

\begin{artha}
savxpanxdalAlxdarU vAsanAkAmagaLiMda kUDiruvanu matutx maqtuyxviniMda biDalapxTiTxruvanu, suSupitxyalilx parxsananxnAgiyU saMgarahitanAgiyU iruvudu toVrutatxde
\end{artha}

\section*{baq - a 4 bArx 3 - 19 kaMDike}

\begin{shl}
tadayxthAsimxnAnxkAsheV sheyxVnoV vA supaNoVR vA viparipatayx shArxnatxH saMhatayx pakwSx saMlayAyeYva dhirxyata EvameVvAyaM puruSa EtasAmx anAtxya dhAvati yatarx supotxV na kacnacxna kAmaM kAmayateV na kacnacxna savxpanxM pashayxti ||19||
\end{shl}

\section*{vAtiRka}

\begin{shl}
tadAvx aseyxYtaditayxtarx yathoVkatxM rUpamAtamxnaH || \\
vakaSxyXmANamatasatxsayx daqSATxnotxV\s yamihoVcayxteV \hfill || 1156 ||  
\end{shl}

\begin{artha}
``tadAvx aseyxYtadaticaCxnAdx apahatapApAmxBayaM rUpamf'' itAyxdiyAda muMdina shurxtivAkayxdalilx Atamxna rUpavu hiMde heVLidaMte heVLalapxDuvudu, adariMda adakekx I daqSATxMtavu ililx heVLalapxTiTxde.
\end{artha}

\begin{shl}
savxtoV budadhxM ca shudadhxM ca mukatxM rUpamihA\s \s tamxnaH || \\
EkavAkoyxVpasaMhAreV na pucnijxVkaqtayx dashiRtamf \hfill || 1157 ||  
\end{shl}

\begin{artha}
ililx Atamxna savxrUpavu savxtaHtAnAgiyeV jAcnxnavAgiyU shudadhxvAgiyU mukatxvAgiyU ideyeMbudanunx EkavAkayxdalilx upasaMharisuvAga oTATxgi toVrisilalx.
\end{artha}

\vishaya{utatxra garxMthada saMbaMdhavanunx tiLisuvudu}

\begin{shl}
tadayxthA sheyxVna ituyxkAtxyX tadadxqqSATxnatxparxdashaRnAtf || \\
shudadhxbudAdhxdikaM rUpamAtamxnaH saMparxdashayxRteV \hfill || 1158 ||  
\end{shl}

\begin{artha}
``tadayxthAsheyxVnaH'' eMba vacanadiMda A daqSATxMtavanunx toVrisuva mUlaka Atamxna shudadhx budAdhxdi savxrUpavu toVrisalapxDuvudu.
\end{artha}

\vishaya{BASayxdalilx heVLida saMbaMdhavanunx heVLutAtx beVre saMbaMdhavanunx heVLuvaru-}

\begin{shl}
savxpanxbudAdhxnatxyoVveVRha daqSATxnatxH saMparxdashiRtaH || \\
saMparxsAdasayx daqSATxnatxH sheyxVneVnAthAdhunoVcayxteV \hfill || 1159 ||  
\end{shl}

\begin{artha}
savxpanx jAgaragaLalilx mahAmatasxyXdaqSATxMtavanunx toVrisidAdxyitu. suSupitxge sheyxVnadaqSATxMtavanunx Iga heVLuvudu.
\end{artha}

\vishaya{sheyxVna matutx supaNaR padagaLige athaRBeVdavanunx heVLutAtx punarukitxyanunx pariharisuvaru-}

\begin{shl}
sheyxVnaH shashAdoV vijecnxVyoV baqhatAkxyashacx roVhitaH || \\
kiSxparxH sheyxVnaH supaNaRsutx balavAnalapxvigarxhaH \hfill || 1160 ||  
\end{shl}

\begin{artha}
sheyxVna: eMdare molavanunx tinunxva doDaDx shariVravuLaLx keMpu baNaNxda hakikx giDuga idu bahucuruku veVgashAli supaNaRvAdarU baliSaThxvAda saNaNx shariravuLaLx hakikx garuDa.
\end{artha}

\begin{shl}
sheyxVnaH shArxnotxV yathA\s \s kAsheV BakaSxyXheVtoVH pariBarxmanf || \\
pakwSx vitatayx niVDaM savxmeVti hi sharxmahAnayeV \hfill || 1161 || 
\end{shl}

\begin{artha}
heVge giDugavu AkAshadalilx tiMDigAgi eraDu rekekxgaLanunx bicicxkoMDu sututxtAtx idudx sharxmavanunx hoVgalADisalu tananx gUDige baMdu seVruvudoV.
\end{artha}

\begin{shl}
yathA tathA\s yamapAyxtAmx jAgarxtasxvXpeVnx pariBarxmanf || \\
shArxnatxsatxcaCxrXmahAnAthaRM barxhamxniVDaM parxpadayxteV \hfill || 1162 ||  
\end{shl}

\begin{artha}
A sheyxVnavu heVgo hAgeyeV I Atamxnu jAgara savxpanxgaLalilx sututxtAtx baLali A sharxmavanunx kaLedukoLaLxlu barxhamxveMba gUDanunx hoMduvudu.
\end{artha}

\vishaya{anatx shabadxdiMda barxhamxvanunx heVge garxhisuvudu? eMdare-}

\begin{shl}
anAtxya dhAvatiVtuyxkatxM tasAyxnatxsayx visheVSaNamf || \\
yatarx supotxV na kamiti jAgarxtasxvXpanxniSeVdhakaqtf \hfill || 1163 ||  
\end{shl}

\begin{artha}
A `anAtxyadhAvati' eMdu shurxtiyalilx heVLide aMta padAthaRkekx ``yatarxsupotxV na kaMcana kAmayateV na kaMcana savxpanxMpashayxti'' eMbudu visheVSaNa, adu jAgara matutx savxpanxgaLeraDanunx niSeVdhisuvudu (aderaDU ilalxdedxMdu tiLisuvudu).
\end{artha}

\vishaya{`supatx: = malagisidavaneMdu heVLidadxriMdaleV jAgarAvasethxyanunx ilalxveMdu tiLiyuvudakekx baruvAga `nakaMcanakAmamf' eMdu kAmavanunx niSeVdhisidudx vayxthaRvalalxve? eMdare}

\footnotetext[1]{parxteyxkavAgi savxpanx jAgara, suSupitxgaLanunx mUru mUrAgi modaleV viMgaDisi heVLide yAdadxriMda sAmAnayxvAgi Atamxnu avasAthxtarxyadalUlx malagiyeV iruvanu, adariMda asAdhAraNavAgi suSupitxyanunx tiLisalu `yatarxsupotxV nakaMcana'....itAyxdiyAgi visheVSaNadiMda savxSaTx paDisuvudu mUru avasethxgaLalUlx malagidAdxneMbudakekx aBipArxyavidu supitxyeMdare malaguvudu nidirxsuvudu idoMdu loVka rUDhiyAda athaR Adare tatatxvXvanunx tiLiyadiruvudeMbuva tatAtxvXgarxhaNaveMba niderxyadu, `anayxthA gaqNahxtaH savxponxV, nidArx tatavx majAnataH' eMdu gwDavAda kArikeyalilx (gw, Agama parxkaraNa 15ne kArike) hiVgeMdu vayxvahariside I athaRdalilx mUru avasethxgaLalUlx malagiruvaneMdu sAmAnayxvAgi heVLide.}
\begin{shl}
supatx\footnotemark[1]sitxsaqSavxvasAthxsu sAmAneyxVna yatasatxtaH || \\
vishinaSATxyXtamxnaH sAvxpaM yatarx supatxgirA suPxTamf \hfill || 1164 ||  
\end{shl}

%%%%%shloka footnote[1]
\begin{artha}
mUru avasethxgaLalUlx sAmAnayxvAgi malagidavaneV ililx heVLalapxTiTxruvanu adariMda Atamxna supitxyanunx `yatarxsupotxV nakaMcana'....itAyxdi vacanadiMda sapxSaTxvAgi (visheVSaNa) seVrisiruvudu.
\end{artha}

\vishaya{Agali AdarU `na kaMcana kAmaM kAmayateV' eMdu kAmavanenxVniSeVdhisuvudu, jAgarxtf savxpanxgaLeraDanunx niSeVdhisideyeMdu heVLidudx heVge? -eMdare.}

\footnotetext[1]{Atamxna supitxge `yatarx sapotxV na kaMcana kAmaM kAmayateV' eMdu oMdu visheVSaNa, eraDaneyadu `nakaMcanakAmaMkAmayateV' eMdu hiVge eraDu visheVSaNagaLigU parxyoVjanavide, yAva dasheyalilx malagidavanu oMdu kAmavanUnx bayasuvudilalx'veMdu savxpanx jAgaragaLalilxruva kAmavelalxvanunx ilalxveMdu nirAkarisidadxriMda EnoMdu kAmavu ilalxdudx suSupitxyeMdu athaRvAguvudu hAgeyeV ``nakaMcana savxpanxM pashayxti'' eMbudara yAvudoMdu jAgara savxpanxgaLalilxruva savxpanxvanunx kANuvudilalxvo eMdu heVLidadxriMda yAva savxpanxvu ilalxdudx suSupitxyeMdu athaRvAguvudu, Adare modalina visheVSaNavu kAmavanenxV muKayxmADi AmUlaka Aguva savxpanx jAgaragaLeraDanunx niSeVdhisideyeMdu tiLiyabeVku. muMdina visheVSaNavu savxpanxvanenxV parxdhAna mADi eraDavasethxgaLanunx ilalxveMdu nirAkarisideyeMta tiLiya beVku  jAgaradalilx Enanunx nAvu noVDuvevo adU saha savxpanxveMdu shurxtiya tAtapxyaRvide adariMdaleV `nakaMcana savxpanxM pashayxti!' eMdu suSupitxyalilx yAvudoMdu savxpanxvanunx kANuvudilalxveMta tiLiyuvudu, jAgaravanunx savxpanxveMdu tiLiyuvudakekx kAraNavidu-anayx garxhaNaveMbudeV savxpanx, ecacxrinalilx AtamxtatatxvXvanunx tiLiyade idadxdariMda adakekx vipariVtavAgi kataqRBoVkatxyX itAyxdi rUpadalilx BeVda jAcnxnavAguvudu, idu savxpanx dasheyalUlx samAna, anayxthAR garxhaNaveMbudu adhAyxsa mithAyxjAcnxna. adu ilalxdudx suSupitxyeMta tiLiyabeVku ``anayxthA gaqNahxtaH savxponxV, nidArx tatatxvX majAnataH ||'' eMdu gwDapAda kArikeyalilx tatavxvanunx beVre riVtiyAgi tiLiyuvavanalilx savxpanxveMbudanunx vayxvahariside tatAvx garxhaNavanunx niderxyeMdu karedidAdxre'' `baq-BASayxdalUlx-BASayxkAraru `jAgariteV\s pi yadadxshaRnaM tadapi savxpanxM manayxteV-' `shurxtiH ata Aha nakaMcanasavxpanxM pashayxtiVti' yeMdu heVLidAdxre.}
\begin{shl}
\footnotemark[1]na kaMcaneVti kAmoV yaH savxpanxjAgarxdavasathxyoVH || \\
niSidhayxteV suSupetxV\s sw tathA savxponxV\s pi yasatxyoVH \hfill || 1165 ||  
\end{shl}

%%%%%shloka footnote[1]
\begin{artha}
`nakaMcanakAmamf' eMdu heVLida yAva kAmavideyo adu savxpanx jAgaragaLalilxdadedxV suSupitxyalilx niSeVdhisalapxTiTxde. hAgU A avasethxgaLa savxpanxvU kUDa niSeVdhisalapxTiTxde.
\end{artha}

\vishaya{modalina visheVSaNakekx beVre athaRvU ide-}

\begin{shl}
savxpenxV vA kAmavirahAdayxthoVketxVneYva vatamxRnA || \\
\footnotemark[2]jAgarxtAkxmaniSeVdhoV\s tasatxtarx kAmAdisaMBavAtf \hfill || 1166 ||  
\end{shl}
\footnotetext[2]{jAgarxtitxnalilx kAmAdigaLu abAdhitavAgi iruvavu, savxpanxdalilx hAgilalx eMdathaR.}

%%%%%shloka footnote[2]
\begin{artha}
athavA `asaknogxhayxyaM puruSaH' eMdu heVLida mAgaRdalelxV savxpanxdalilx kAmavilalxvAdadxriMda jAgatitxna kAmavanenxV niSeVdhisuvudu, A jAgaradalilx kAmAdigaLu saMBavisa bahudAdadxriMda (adU savxpanxdalilxlalxveMdu heVLide).
\end{artha}

\vishaya{divxtiVya visheVSaNakekx beVre athaRvU ide-}

\begin{shl}
EkeYkasAyxmavasAthxyAM jAgarxtasxvXpanxsuSupitxBiH || \\
yadi vA vidayxteV BeVdaH kaMcaneVti ca liknagxtaH \hfill || 1167 ||  
\end{shl}

\begin{artha}
athavA oMdoMdavasethxyalUlx jAgarxtf, savxpanx-suSupitxgaLiMda BeVdavideyenanx beVku, idakekx `kaMcana'eMdu heVLidedxV gamakavAgide.
\end{artha}

\vishaya{elAlx avasethxgaLu parxteyxVka mUru bageyeMbudakekx gamakavAda shurxti-}

\begin{shl}
tarxya AvasathA iti tathA ca shurxtishAsanamf || \\
jAgarxtasxvXpanxsuSupAtxnAM terxYvidhayxparxtipAdakamf \hfill || 1168 || 
\end{shl}

\begin{artha}
``tarxya AvasathAH tarxyaH savxpAnxH'' eMdu \footnote{ililx Avasatha eMdare mane sAthxnagaLeMdathaR, mUru manegaLu avasethxgaLu ivu Atamxnige tananx tatatxvX jAcnxnavilalxdiruvAga toVruvavu ivu savxpanxdaMtiruvudariMda mUru savxpanxgaLeMdu shurxtiyalilx heVLalapxTiTxde.}aitareVya shurxtiyu jAgara-savxpanx-suSupitxgaLu mUru bageyAgive eMbudanunx parxtipAdisuvudu.
\end{artha}

\vishaya{parxkaqtakekx A viBAgadiMda parxyoVjana:-}

\begin{shl}
tatarx tarxyANAM sAthxnAnAM jAgarxtasxvXpanxniSeVdhanamf || \\
na kaMcaneVti vAkeyxVna tayoVreVva garxhaH shurxteVH \hfill || 1169 ||  
\end{shl}
				
\begin{shl}
savxpanxM na kaMcaneVtuyxkAtxyX savaRsavxpanxniSeVdhanamf || \\
iti parxboVdhasavxpAnxBAyxM vivikatxM sAthxnamucayxteV \hfill || 1170 ||  
\end{shl}

\begin{artha}
adaralilx mUru sAthxnagaLa jAgarasavxpanxgaLanunx `na kiMcana' eMba vAkayxdiMda niSeVdhisuvudu, kAraNaveVneMdare `kAma-savxpanx' eMba shabadxgaLiMda adeV jAgarasavxpanxgaLu shurxtiyiMda tiLidive, adariMda'' savxpanxvanunx yAvudoMdanunx kANuvudilalxveMdu heVLidadxriMda savaRsavxpanxvanunx niSeVdhisiruvudu, adariMda jAgara savxpanxgaLiMda biDalapxTaTx sAthxnavu (aMta padadiMda) heVLalapxDuvudu.
\end{artha}

\vishaya{``aMtAya dhAvati" eMbudara athaRvanunx upasaMharisuvudu}

\begin{shl}
jAgarxtasxvXpAnxtamxkw pakwSx vitatAyx\s \s tAmx\s \s tamxmoVhavAnf || \\
BukAtxvX BoVgAnatha shArxnatxH pakwSx saMhaqtayx cA\s \s tamxni \hfill || 1171 ||  
\end{shl}

\begin{artha}
jAgarxtf savxpanxgaLeMba eraDu rekekxgaLanunx bicicxkoMDu Atamxnu savxrUpada ajAcnxnavuLaLxvanAgi BoVgagaLanunx anuBavisi baLali tananxlilxyeV (jAgara savxpanxgaLeMba) rekekxgaLanunx muduvarisikoMDu iruvanu.
\end{artha}

\vishaya{`saMhatayx pakASx' eMba daqSATxMtadiMda tiLidu baruva athaR}

\begin{shl}
avidAyxyAmavasAthxnaM tadudUBxtasayx vasutxnaH || \\
saMhatayxpakoSxVpamayA shurxteyxVha parxtipAdayxteV \hfill || 1172 ||  
\end{shl}

\begin{artha}
\footnote{eraDu rekekxgaLanunx tanonxLage mudurisikoMDu tananx gUDinalilx giDugavu heVge kuLitu koLuLxvudo hAgeyeV I daqSATxMta shurxtiyiMda ceYtanAyxBAsavAda avideyxyalilx avideyxyiMda kalipxtavAda jagatitxge layaveMdathaR keVvala avideyxyalilx layavalalx ceYtanAyxshirxtavAda avideyxya layasAthxna, adeV mUlakAraNa, adaralilx layaveVyukatx.}avideyxyiMda huTiTxkoMDa vasutxgaLu avideyxyalelxV iruvudu, idanunx saMhatayx pakASx eMba daqSATxMtadiMda shurxtiyu ililx parxtipAdiside.
\end{artha}

\vishaya{`saMlayAyeYva dhirxyateV' eMbudariMda tiLidu baruva athaR-}

\begin{shl}
saMlayAyeVti yacuCxdadhxM rUpaM sAyxtapxrXtayxgAtamxnaH || \\
parxtayxkicxdABa Agatayx dhirxyateV parxtayxgAtamxneV \hfill || 1173 ||  
\end{shl}

\begin{artha}
`saMlayAya' eMdu parxtayxgAtamxna yAva shudadhx savxrUpaviruvudo? adu vivakiSxta A parxtayxgAtamxnigAgi\footnote{giDugavu punaH tananx gUDinalilx cenAnxgi aDagikuLitukoLaLxlu tananx gUDigAgi tananxnunx nililxsi koLuLxvudu, adaraMte parxtayxgAtamxna shudadhx savxrUpavanunx paDedu tananxnunx adaralilx layagoLisalu jiVvAtamxnU saha tananx biMbasAthxnavAda parxtayxgAtamxnidadxlilxge baMdu tananxnunx nililxsikoLuLxvudu" parxtayxgAtamxna ajAcnxnadalilx budidhx modalAdavu aDaguvalilx upAdhiyilalxde tAnu parxteyxVka iralArade tananx savxrUpadalelxV liVnavAgi nilulxvaneMdathaR.} parxtayxkf ceYtanayxdaMte toVruva jiVvanu baMdu tananxnunx dharisuvanu.
\end{artha}

\vishaya{daqSATxMtadiMda adanunx savxSaTxpaDisuvudu}

\begin{shl}
budAdhxyXdikAyaRsaMhAreV parxtayxkecxYtanayxrUpiNaH || \\
cidibxmabxsAyxpi saMhAroV jalAkaRparxvilApavatf \hfill || 1174 ||  
\end{shl}

\begin{artha}
budidhx modalAda kAyaRvu upasaMhAravAdalilx parxtayxkf ceYtanayx rUpavAda ceYtanayx parxtibiMba niVrinalilx parxtibiMbisida sUyaRnige (upa saMhAra)layavAguvaMte layavAguvudu.
\end{artha}

\vishaya{suSupitxyalilx Atamxnu barxhamxdalilxdadxrU beVre avasethxyalilx beVre yAgirali eMdare}

\begin{shl}
avidAyxvAnupxrA yoV\s BUdavidAyxkAyaRgashicxtiH || \\
avidayxyA viBAgoV\s sayx cidibxmabxsoyxVpajAyateV \hfill || 1175 ||  
\end{shl}

\begin{artha}
hiMde (jAgara, savxpanx dashegaLalilx) yAvanu avidAyxvaMtanAgidadxnoV matutx cidavxsutxvu avideyxya kAyaRdalilx (budidhx modalAdavugaLalilx) ididxto. I ceYtanayxda biMbakekx avideyxyiMda viBAgavAguvudu (nijavAgilalx).
\end{artha}

\begin{shl}
parxtAyxKAyxteYva sA\s vidAyx sakAyAR savaRdA\s \s tamxnA || \\
parxtAyxcaSeTxV tu nAvidAyx nirAtamxtAvxcicxdAtapxrXnaH \hfill || 1176 ||  
\end{shl}

\begin{artha}
(vasutxsithxtiyalilx) A\footnote{Atamxnalilx viBAgavu (BeVdavu) toVralu kAraNavAdadudx avideyx, A avideyxyu nAshavAda meVleyeV jiVvanu barxhamxvAgiruvudu adakekx muMce barxhamxsavxrUpavelilxMda eMdu parxshenx baMdare I vAtiRkadalilx utatxriside. Atamxnu sadA cidUrxpanAdadxriMda avanalilx avideyxyu viroVdhiyAdadxriMdaleV yAvAgalU iralAradu, adariMda Atamxnu adanunx tirasakxrisiyeV iruvanu. aMdare adanunx savxBAvavAgiyeV dUra mADiruvanu, avideyxyAdaro Atamxnanunx tirasakxrisilalx adu savxtaMtarxvAda satetx sUPxtiRyuLaLxdadxlalxvAdadxriMda savxrUpavilalxdudx adariMda Atamxnanunx dUriVkarisuvudilalx, avideyxyu nanage ide. toVrutatxdeyeMbudanunx tiLiyalu sAkiSx ceVtanavu beVkeV beVku AtamxnidAdxne toVrutAtxneMbudakekx beVre yAvudara sahAyavu beVDa, savxtaHsidadhx, savxparxkAshaneMdu tAtapxyaR.} avideyxyu kAyaRsahitavAgi yAvAgalU AtamxniMda tirasakxrisalapxTeTxV ide, Adare avideyxyu cidAtamxnigiMta savxtaMtarxvAgidadx savxrUpavilalxdadxriMda Atamxnanunx tirasakxrisuvudilalx (dUramADuvudilalx).
\end{artha}

\vishaya{Atamxnu asaMganAdadxriMda saMbaMdhaveV ilalx hiVgiralu avideyx matutx adara kAyaRvanunx heVge dUra mADuvudu? eMdare-}

\begin{shl}
avidAyxtajajxkAyARBAyxmAtAmx\s nAyxnavasheVSataH || \\
saMbadhayxteV savxtoV\s saknagx AtamxtAvxdeVva kAraNAtf \hfill || 1177 ||  
\end{shl}

\begin{artha}
avideyx matutx adara kAyaRgaLiMda savxtaH asaMganAdarU saMbaMdhisuvanu, Atamxkekx beVreyAgi yAvudU nijavAgi uLidilalxvAdadxriMda avugaLoDane saMbaMdhisuvanu. (Eke uLidilalxveMdare) tAnu avugaLigU AtamxnAgiruva kAraNadiMda (uLidilalx)
\end{artha}

\section*{vAtiRka}

\begin{shl}
avidAyx yacacx tatAkxyaRmAtAmxnaM sAvxtamxsidadhxyeV ||  \\
saMbiBatasxtayxnAtamxtAvxdAtAmx tanAnxnumanayxteV \hfill || 1178 ||  
\end{shl}

\begin{artha}
avideyxyU matutx adara kAyaR yAvuduMTo adU saha tananx savxrUpavu sididhxsalu Atamxnalilx seVralu iciCxsuvudu. EkeMdare avu anAtamxnAgiruvavu aMdare jaDavAgiruvavu, Atamxnu mAtarx adanunx bayasuvudilalx.
\end{artha}

\begin{shl}
niHsaknagxsayx sasaknegxVna kUTasathxsayx nirAtamxnA || \\
AtamxnoV\s nAtamxnA yoVgoV vAsatxvoV noVpapadayxteV \hfill || 1179 ||  
\end{shl}

\begin{artha}
saMgavilalxda Atamxnige saMgaviruva vasutxvinoDaneyU, niviRkAranige (vikAraviruva) nishacxlasavxBAvavilalxda vasutxvinoDaneyU, Atamxnige anAtamxdoDaneyU vAsatxvikavAda saMbaMdhavu hoMduvudilalx.
\end{artha}

\vishaya{AdarU avugaLige avAsatxvika saMbaMdhavu heVge?}

\begin{shl}
parxtAyxKAyxtA\s \s tamxneYveVyaM parxtayxgAtAmxnameVkalamf || \\
avidAyx\s \s liknagxteV vahinxM GaqtapiNaDx ivoVlabxNamf \hfill || 1180 ||  
\end{shl}

\begin{artha}
I avideyxyu AtamxniMdaleV tirasakxrisalapxTeTxV idudx EkarUpadiMda toVruva parxtayxgAtamxnanunx bahaLa uriyuva beMkiyanunx \footnote{uriyuva beMkiyanunx muTiTxdare sahisalArade gaTiTxyAda tupapxvu karagi hoVguvudu, hAgeye Atamxtatavxvanunx muTaTxlu baMda avideyxyu asaMga cidUrxpananunx muTaTxlu Agade dUravAguvudeMdathaR.}tupapxda gaTiTxyu sapxshiRsuvaMte sapxshiRsuvudu.
\end{artha}

\begin{shl}
avicAritasaMsidAdhx saMgatiH paramAtamxnaH || \\
avidAyxtajajxkAyARBAyxmeVvameVveVti gamayxtAmf \hfill || 1181 ||  
\end{shl}

%%%%%%shloka footnote[1]
\begin{artha}
paramAtamxnige avideyx matutx adara kAyaRgaLoMdige saMbaMdhavU vicAra mADadiruvAga sididhxsida ideV riVtiyAgiyeV iruvudeMdu tiLiya takakxdudx.
\end{artha}

\begin{shl}
EvaM satayxsayx bAdhaH sAyxtapxramAthARtamxsaMsharxyAtf || \\
saMsArAnathaRsaMbanadhxheVtoVrAtamxparxmANataH \hfill || 1182 ||  
\end{shl}

\begin{artha}
hiVgAdare saMsArada anathaR saMbaMdhavuMTAgalu kAraNavAda I videyxge paramAthaRvAgiruva Atamxnanunx avalaMbisi baruva AtamxtatatxvXjAcnxnadiMda nivaqtitxyU Aguvudu.
\end{artha}

\begin{shl}
parxtayxgidhxVmAtarxtaH parxtayxknaknxdavxyatavxvisheVSaNaH || \\
niHsaMbanAdhxtamxsaMbanAdhxdanAtAmx tAvxtamxvananx sanf \hfill || 1183 ||  
\end{shl}

\begin{artha}
parxtayxgAtamxna jAcnxnamAtarxdiMda parxtayxgAtamxna adivxtiVyatapxvisheVSaNadiMda kUDi I anAtamxvu (ahaMkArAdigaLu) saMbaMdhavilalxda Atamxna saMbaMdhadiMda toVridadxrU AtamxnaMte satatxlalx (nijavalalx).
\end{artha}

\vishaya{BataqRparxpaMca mata nirAkaraNe}

\begin{shl}
samasatxvayxsatxtAmeVvaM sati vAyxcakaSxteV\s tarx yeV || \\
kaSaRnitx nAsikAgerxVNa kaNaRmUlaM suKeVna teV \hfill || 1184 ||  
\end{shl}

\begin{artha}
I Atamxna viSayadalilx yAru hiVgidadxrU samasatx vayxsatxrUpavanunx (Ekatavx-aneVkatavxgaLanunx) heVLuvaro, avaru suKavAgi mUgina tudiyiMda kiviya buDavanunx eLeyuvaru.
\end{artha}

\begin{shl}
AtamxvasatxvXtireVkeVNa nAnayxdavxsatxvXsitx mAnataH || \\
tanAmxnamiti ceVnemxYvaM yatasatxtApxrXkapxrXboVdhataH \hfill || 1185 ||  
\end{shl}

\begin{artha}
Atamx vasutxvigiMta beVreyoMdu vasutx parxmANanusAravAgi ilalx, adara viSayadalilx parxtayxkASxdi parxmANaveMdare:- hiVgU sariyalalx. yAvudariMda jAcnxnakekx muMceyeV adu parxmANa (anaMtara parxmANavenisadu).
\end{artha}

\footnotetext[1]{anuBavasidadhxvAda anAtamxvasutxvanunx aMgikarisi asutx eMdu heVLide, Adare Atamxnalilx aneVkatavx matutx EkatavxgaLanunx vAsatxvavAgiyeV opupxvadu sariyalalx, jAcnxnakikxMta muMce opipxdarU Atamxnalilx aneVkatavxvU anAtamxvasutxsaMbaMdhavU vAsatxvavAgilalx, asaMgapuruSanige nijavAgi savxtaH anAtamxsaMbaMdhavu bAradu, matotxMdariMda baMdiruvudU ilalx, avideyxyanunx biTuTx matotxMdu adara saMbaMdhavanunxMTu mADuvudilalx, avideyxyU saha kArakavalalxvAdadxriMda ilalxda saMbaMdhavanunx vAsatxvikavAgi huTiTxsuvudilalx, avideyxyu vAsatxva kAyaRkAriyalalx, saMbaMdhavu vAsatxvavalalxvAdare samasatx vayxsatx rUpavU vAsatxvavalalxveMdu tAtapxyaR.}
\begin{shl}
\footnotemark[1]asutx vasatxvXnatxraM cAnayxtatxsAyxpAyxtAmxBisaMgatw || \\
koV heVturiti vakatxvayxM nAvidAyx\s kArakatavxtaH \hfill || 1186 ||  
\end{shl}

%%%%%%shloka footnote[1]
\begin{artha}
beVre vasutxvirali adakUkx Atamxna saMbaMdhavuMTAgalu Enu? kAraNa eMbudanunx heVLabeVku, avideyxyeV saMbaMdhakekx kAraNavalalx, adu kArakavalalx (hosadAgi matotxMdanunx huTiTxsuvudalalx).
\end{artha}

\begin{shl}
samasatxvayxsatxtAvAyxKAyx tasAmxdanadhxparaMparA || \\
veYshAvxnaravarAdeVva na tu nAyxyAnusArataH \hfill || 1187 ||  
\end{shl}

\begin{artha}
adariMda samasatx vayxsatxveMdu vAyxKAyxna mADiruvudu aMdhaparaMpareyeV sari, adU saha veYshAvxnarana varadiMdaleV sididhxsuvudu, nAyxyAnusAravAgi alalx.
\end{artha}

\vishaya{sheyxVnavAkayxkekx BatayxRparxpaMcara vAyxKAyxna}

\vishaya{BUmike}

\begin{shl}
asayx hi devxYtaviSayeV vijAcnxnaM BAnurashimxvatf || \\
\footnotemark[2]vikiVNaRM boVdhadeVshiVyaM samanAtxtapxrXthateV daqsheVH \hfill || 1188 ||  
\end{shl}
\footnotetext[2]{``asayxhidevxYta viSayeV visheVSa vijAcnxna mAditayxrashimxvatf vikiVNaRM jAgarita deVshavatf samanAtxtf avaBAsate'' eMdu BataqRparxpaMcara BASayxvAkayx.}

%%%%%shloka footnote[2]
\begin{artha}
I Atamxna ceYtanayxvu devxYtaviSayadalilx sUyaRna kiraNadaMte ecacxru parxdeVshadalilx \footnote{`mithAyxjAcnxnaM BUtAvx' eMdu AnaMda girigaLu TiVkeyalilx vivarisidAdxre.}(mithAyxjAcnxnavAgi) ceYtanayxda sutatxlU haraDi koMDidudx toVruvudu.
\end{artha}

\footnotetext[1]{``tatarxkAmaH kamaRNaH PalaM kamaRNa EvatuvAsanA'' eMdu BataqRparxpaMcara BASayxvacanavide (AnaM-TiVke)}
\begin{shl}
tatarx kamaRparxsUtoV\s yaM kAmaH savaRparxvaqtitxkaqtf || \\
jAtA kamaRNa EveVyaM BAvanA\s sAyxnatxrAtamxnaH \hfill || 1189 ||  
\end{shl}

%%%%%shloka footnote[1]
\begin{artha}
AtamxjAcnxnavu devxYtaviSayavAgiralu kamaRvu vaqdidhxyAgi adariMda kAmavu huTiTx idu savaRtarx parxvaqtitxyanunxMTu mADuvudu kamaRdiMdaleV I aMtarAtamxnige BAvaneyu (vAsaneyu) huTuTxvudu.
\end{artha}

\vishaya{vAsane Etakekx?-}

\footnotetext[2]{``yathAlakaSxNayeYva BAvanayA tatakxmaRkaqtaM tathAlakaSxNameVva PalaM tasayxvijAcnxnAtamxnaH sA BAvanA tadeVva savxparxyukatxM PalaM parxti EkiVBUtABavati '' eMdu BataqRparxpaMcara BASayxvacana.}
\begin{shl}
\footnotemark[2]yathA lakaSxNayA kamaR kaqtaM BAvanayA\s \s tamxnA || \\
PalaM tAdaqgivxdhaM dAtumeVkiVBavati sA\s \s tamxnaH \hfill || 1190 ||  
\end{shl}

%%%%%shloka footnote[2]
\begin{artha}
yAva lakaSxNavuLaLx vAsaneyiMda kamaRvu taninxMda mADalapxTiTxto A lakaSxNavuLaLx PalavanenxV koDalu Atamxnige A vAsaneyu (taninxMda Aguva Palada bagegx) oMdAguvudu.
\end{artha}

\footnotetext[3]{``tadanuracnijxtoV vijAcnxnAtAmxkAmaM kAmayata iteyxVtadaBxvatiVti''}
\begin{shl}
\footnotemark[3]tayA\s nurAcnijxta iva hAyxtAmx kAmAnapxriVpasxti || \\
vAsanAlakaSxNaH kAmaH savxpanxsagARya jAyateV \hfill || 1191 || 
\end{shl}

%%%%shloka footnote[3]
\begin{artha}
A vAsaneyiMda anuraMjisalapxTaTxMtiruva Atamxnu iSaTxvAdavugaLanunx paDeyalu bayasuvanu, vAsanAlakaSxNavuLaLx kAmavu savxpanx saqSiTxgAgi huTuTxvudu.
\end{artha}

\vishaya{savxpanx saqSiTxyu EtakAkxgi? eMdare-}

\footnotetext[4]{``tatarxHkAmataH savxpanxnimARNaM suKaduHKaM dashaRyati'' eMdu BataqRparxpaMcara vacana'' (AnaM-TiVke)}
\begin{shl}
\footnotemark[4]suKAdiPalaBoVgAya savxpanxH kuSxdarxsayx kamaRNaH || \\
pariceCxVdeVna nimARtaqjAcnxnaM ca parxtayxgAtamxnaH \hfill || 1192 ||  
\end{shl}

%%%%%shloka footnote[4]
\begin{artha}
kuSxdarxvAda kamaRda suKAdiPalavanunx anuBava mADalu savxpanxvu Aguvudu, matutx parxtayxgAtamxna jAcnxnavu (vAsaneyiMda huTuTxva viSayada) nishacxyavanunxMTu mADuvudu
\end{artha}

\vishaya{kamaRkUkx kataqRtavxvanunx aMgiVkarisiruvudariMda Atamxna jAcnxnavu pariceCxVdakavenunxvudu yukatxvalalx eMdu heVLidare-}

\begin{shl}
vijAcnxnAnuvidhAyiV hi vikAraH kamaRNoV yataH || \\
\footnotemark[1]vijAcnxnaM vipariVtAthaRdashaRnaM ceVha BaNayxteV \hfill || 1193 ||  
\end{shl}
\footnotetext[1]{``vijAcnxnaM puna BARvanA pariceCxVda vinimARtaq vijAcnxnAnuvidhAyiniV hi kamaRNoV vikirxyeVtiH'' ``vijAcnxna mapeyxVtadivxpariVta dashaRna mavideyxVti'' eMbudu BataqRparxpaMcara BASayx vacanavide. (AnaM-TiVke)}

%%%%%shloka footnote[1]
\begin{artha}
vijAcnxnavanunx anusarisiyeV kamaRda vikAravu yAvudariMda Aguvudo adariMda yukatxve ililx vijAcnxnavU vipariVtavAda viSaya jAcnxna (BArxMti jAcnxna)veMdu heVLalapxDuvudu.
\end{artha}

\begin{shl}
sA\s vidAyx sati bAheyxV\s theVR darxSaTxveyxV\s laM na cAsati || \\
kAmaH sa Eva boVdheV\s sayx vijAcnxnAtamxna iSayxteV \hfill || 1194 ||  
\end{shl}

\begin{artha}
noVDatakakx \footnote{``sA\s pi sati darxSaTxveyxV vipariVtaM gArxhayeVtf '' ``sa ESa bAhoyxV vijAcnxnAtamxna'' (AnaM-TiVke) BataqRparxpaMcada vacana}horagina viSayavu idadxre A avideyxyu (mithAyxjAcnxnavu) adara niNaRya kAyaRdalilx samathaRvAguvudu, adilalxdidadxre AgalAradu. adeV (avidAyxsahitavAda) kAmaveV I jiVvAtamxnige \footnote{``vAtiRkadalilx bAhayxH'' eMba pAThaveV sari vAyxKAyxnakUkx I vacanakUkx hoMdide, `boVdheV' eMbudu sariyalalx, adaraMte anuvAda mADide.}horagiruvudu
\end{artha}

\footnotetext[1]{ililx Atamxnige bAhayxvAda kAmAdigaLige sUkaravilalxde kAyARBeVdavanunx toVriside, kAma vAsaneyU puruSanige viSayagaLa parxvaqtitxyanunxMTu mADuvudu, kamaRvu puruSanige suKa duHKAdi vikAragaLanunxMTu mADuvudu. ideV kAyARBeVda, anayxthA niNaRyavu (mithAyx jAcnxnavu) vipariVtavAgi tiLiyalu kAraNavAguvudu.}
\begin{shl}
\footnotemark[2]parxyoVkitxrXV BAvanA tatarx kamaR tavxsayx vikArakaqtf || \\
vipariVtagarxhAya sAyxdanayxtheVti vinishicxtiH \hfill || 1195 ||
\end{shl}

%%%%%shloka footnote[1]
\begin{artha}
vAsaneyu (puruSanige viSayadalilx) parxvaqtitxyuMTu mADuvudu, ivana kamaRvu mAtarx (suKaduHKAdi) vikAragaLanunxMTu mADuvudu, beVre riVtiyAgi niNaRyisuvudu BArxMti jAcnxnakekx kAraNavAguvudu (tatatxvX niNaRyakekx Agadu).
\end{artha}

\vishaya{hAgAdare kAmAdivasutxgaLu parasapxra apeVkeSxyilalxdavugaLAdadxriMda savxtaMtarxvAgiveye? eMdare-}

\begin{shl}
kAmeVhABAvanAjAcnxnapadAthARnAM yathAkarxmamf || \\
parasapxravayxpeVkaSxtAvxdeVvaM sAyx\footnotemark[2]deVkavAkayxtA \hfill || 1196 ||  
\end{shl}
\footnotetext[2]{``EkavAkayxte'' eMdare EkakAyaRvuLaLxdedxMdathaR}

%%%%%shloka footnote[2]
\begin{artha}
kAma, kamaR, BAvane (vAsane) jAcnxna- eMba I padAthaRgaLu karxmavaritu oMdakokxMdu sApeVkaSxvAdadxriMda ivugaLige EkavAkayxte (Eka kAyaR saMbaMdhavu) iruvudu.
\end{artha}

\vishaya{hAgAdare avugaLige kAyaRBeVdavu heVge:-}

\begin{shl}
tatarx BAvanayeYvA\s \s tAmx kAmAnAkxmayateV tathA || \\
vidayxyA viVkaSxteV cAthARnavidoyxVtAthxpitAniha \hfill || 1197 ||  
\end{shl}

\begin{artha}
avugaLalilx BAvaneyiMdaleV Atamxnu iSaTxvAdavugaLanunx kAmisuvanu, hAgeyeV jAcnxnadiMda avideyxyiMda kalipxtavAda vasutxgaLanunx I devxYtAvasethxyalilx noVDuvanu.
\end{artha}

\footnotetext[1]{``adevxYta viSayeV punaH suSupatxsAthxneV bAhayxsayx parxyoVjayxsayx aBAvAtf kamaR nisapxnadxmAsetxV, vijAcnxnAtamxgatA\s piBAvanA kAyARBAvAcACxmayxtiVti''}
\begin{shl}
\footnotemark[1]saMparxsAdeV tu bAhayxsayx vasutxnoV\s saMBavAdidamf || \\
kamaR nisapxnadxmeVvA\s \s setxV BAvanA\s pi ca shAmayxti \hfill || 1198 ||  
\end{shl}

%%%%%shloka footnote[1]
\begin{artha}
suSupitxyalAlxdarU bAhayx vasutxvu saMBavisuvudilalxvAdadxriMda I kamaRvu alAlxDade tananx kAyaRvanunx mADade iruvudu, vAsaneyU shAMtavAgide.
\end{artha}

\begin{shl}
parxyoVjayxvirahAtakxmaRkAyaRsAyxsaMBavAtatxthA \hfill || 1199 ||  \\
vidAyx\s pi \footnotemark[2]vipariVtasayx parxviBakatxsayx vasutxnaH || \\
aBAvAdavisheVSAtamxjAcnxnoV\s yaM vayxvatiSaThxteV \hfill || 1200 || 
\end{shl}
\footnotetext[2]{``tadAhuH vidAyxpi vipariVtasayx parxviBakatxsayxgArxhayitavayxsAyxBAvAtf na visheVSavijAcnxnAya kalapxta iti''}

%%%%%shloka footnote[2]
\begin{artha}
parxyoVjayxvu ilalxvAdadxriMda kamaRda kAyaRvu saMBavisadeyiruvudariMda videyxyU shAMtavAguvudu, vipariVtavAda viBakatx vasutx ilalxvAdadxriMda visheVSavilalxda Atamx jAcnxnavuLaLxvanAgi nilulxvanu.
\end{artha}

\begin{artha}
daqSATxMtadiMda suSupAtxtamxna rUpavanunx toVrisuvudu
\end{artha}

\begin{shl}
dAhAyxBAvAdayxthA vahinxH sAvxtamxneyxVvoVpashAmayxti ||  \\
kamaRNayxsatxmiteV tadavxtapxrXtayxknAknxsetxV savxBAvataH \hfill || 1201 ||  
\end{shl}

\begin{artha}
suDalapxDuva kASATxdi vasutxvilalxvAdadxriMda beMkiyu tananxlilxyeV heVge shAMtavAguvudo, hAgeyeV kamaRvu asatxvAda meVle parxtayxgAtamxnu savxBAvavAgiyeV irutAtxne.
\end{artha}

\begin{shl}
EvaM kaqtevxVdamatArx\s \s ha shurxtiH kAmaM na kaMcana || \\
kAmakamARdinimuRkatxM suSupetxV rUpamAtamxnaH \hfill || 1202 ||  
\end{shl}

\begin{artha}
I riVtiyAgi I suSupitxyalilx (kAmAdigaLu nAshavAguvavu eMdu) mADikoMDu `na kaMcanakAmaMkAmayateV' eMba shurxtiyu Atamxna kAmakamARdigaLiMda biDalapxTaTx rUpavanunx heVLiruvudu.
\end{artha}

\vishaya{BataqRparxpaMcara vAyxKAyxnada upasaMhAra}

\begin{shl}
iteyxVvaM sheyxVnadaqSATxnatxH keYshicxdAvxyXKAyxyi sAdareYH || \\
tatarx yukatxmayukatxM vA shurxtivAkAyxnusArataH \hfill || 1203 ||  
\end{shl}

\begin{shl}
nAyxyeYjaRgati saMsidedhxYH savxyameVva vicAyaRtAmf || \\
apakaSxpAtapatiteYviRdavxdiBxvaRsutxsidadhxyeV \hfill || 1204 ||  
\end{shl}

\begin{artha}
adaralilx yukatxvo? ayukatxvo? eMbudanunx shurxtivAkayxvanunx anusarisi loVkasidadhxvAda nAyxyagaLiMda pakaSxpAtadalilx biVLade vidAvxMsaru vasutx niNaRyakAkxgi savxyaM vicAra mADabeVkAdudu.
\end{artha}

\vishaya{IvaregU sheyxVna vAkayxdalilx heVLida parxtayxgUrxpakathana}

\begin{shl}
Etadasayx savxtoV rUpaM yadatorxVpaparxdashiRtamf || \\
avidAyxkAmakamARdivivikatxM yatusxSupatxgamf \hfill || 1205 ||  
\end{shl}

\begin{artha}
idu ivana sAvxBAvikavAda rUpa, yAvudu ililx suSupitxyalilx iruvudo adu, adu yAvudeMdare! avideyx kAma, kamaRmodalAdavugaLiMda biDalapxTaTx rUpavu, adu I (sheyxVna vAkayxdalilx) toVrisalapxTiTxtu.
\end{artha}

\begin{shl}
itoV\s nayxthA tu yadUrxpaM jAgarxtasxvXpanxsavxlakaSxNamf || \\
tadasayx paratoV jecnxVyamAtAmxjAcnxneYkaheVtukamf \hfill || 1206 ||  
\end{shl}

\begin{artha}
idakekx beVreyAgi jAgarxtutx savxpanx ivugaLalilxruva (kataqRtAvxdi) lakaSxNada rUpavu yAvuduMTo, adu matotxMdariMda baMdidedxMdu  tiLiya beVku. EtariMda eMdare Atamxna ajAcnxnaveMba oMdeV kAraNadiMda (baMdadudx).
\end{artha}

\vishaya{muMdina vicAra viSayaveVnu:-}

\begin{shl}
yadedhxVtukamidaM rUpaM sA\s vidAyx\s nathaRkAriNiV || \\
sA savxtaH paratoV vA\s seyxVteyxVtadatArxdhunoVcayxteV \hfill || 1207 ||  
\end{shl}

\begin{artha}
yAva nimitatxdiMda I rUpavu baMdidiyo adu avideyx anathaRkAri yAdudu, adu savxtaH (sAvxBAvikavAgi) I Atamxnige iruvude? athavA inonxMdariMda baMdideye eMbudu Iga heVLalapxDuvudu.
\end{artha}

\begin{shl}
na tAvxganutxravideyxVyamanimoVRkaSxparxsaknagxtaH || \\
AtamxsavxBAvoV\s videyxVyaM na veVteyxVtadivxcAyaRteV \hfill || 1208 ||  
\end{shl}

%%%%%shloka footnote[1]
\begin{artha}
\footnote{``sA cA vidAyxH tasAyx avidAyxyAH kiMsAvxBAvikatavx mAhoVsivxtf kAmakamARdi vadAganutxkatavxmf yadicAganutxkatavxM tatoV vimoVkaSx upapadayxteV . tasAyxshAcx\s \s ganutxkatevxV koVpapatitxH kathaM vA nA\s \s tamxdhamoVR\s videyxVti savARnathaRbiVja BUtAyA avidAyxyAH satatavxvXvadhAraNAthaRMparAkaNiDxkA\s \s raBayxteV-'' I BASayxdalilx AMgutukaveMdare asAvxBAvikaveMdathaR, sAvxBAvikatavxmf eMdu modalina shaMke mADi naMtaraveV kAma kamARdigaLaMte AgaMtukatavxveMba vikalapx mADidadxriMda adakekx vipariVtavAgi asAvxBAvikaveMdeV athaRmADabeVku I vAtiRkadalUlx asAvxBAvikaveMdeV AgaMtu padakekx athaR Adare sAdi, athavA anAdiyo? eMba vicAravu ililx vivakiSxtavalalx. AgaMtu eMbudakekx sAdi. utapxtitxyuLaLxdedxMdu athaR mADidare avideyxyu sAdiyalalx, Iga itatxlU??????deMdu heVLidarU moVkaSxveV bArade hoVguvudu yAvAgalo baruvudeMdAdare moVkaSxvAda meVlU tirugi adu huTiTx edudxkoLaLx bahudu baruvudakekx EnU nimitatxvilalxvAda AkAshadalilx moVDavu baruvaMte I Avideyxyu AmeVlU AkasimxkavAgi barabahudu, adakekx sAdiyenanxbahudu. I riVtiyAgi animoVkaSx parxsaMgavanunx doVSavAgi AroVpisidAdxre'' adakekx heVtuvidadxrU Atamxkekx beVreyAgi matotxMdu kAraNavilalxvAdadxriMda AtamxvanenxV heVtuveMdu aMgiVkarisabeVku AgalU Atamx yAvAgalU iruvudariMda avideyxyu huTiTxkoLaLx bahudu I pakaSxdalUlx mokaSxveV ilalxveMdAguvudu'' adariMda avideyxyu sAdi, athavA anAdiyeMba vicAravu ililx vivakiSxtavalalx.....avideyxyu savxBAvaveV enunxva pakaSxdalUlx savxBAvavu eMdigU hoVguvudilalxvAdadxriMda moVkaSxveV Agade hoVguvudeMdu doVSavu samAnavAdadxriMda savxBAvave? alalxve? eMba vicArakUkx avakAshavelilx? eMdu keVLabAradu. Atamxnu jaDanalalxvenunxvavarige ajAcnxnanivaqtitxyeMba mukitxyu beVreyAgiruvudu adariMda savxBAvavo athavA alalxvo eMdu sAvxBAvikapakaSxvanenxtitx adakekx virudadhxvAda AgaMtuka (asAvxBAvika)veMba pakaSxvanunx etitx vicAra mADiruvudu saMgatavAgide.} 
I avideyxyu AgaMtukavalalx sAdiyalalx EkeMdare ! moVkaSxveV ilalxvAgi biDuvudu. adariMda I avideyxyu Atamxna savxBAvavo? athavA alalxvo eMbudanunx vicAra mADuvudu.
\end{artha}

\vishaya{vicArakekx mUlaveVneMdare}

\footnotetext[2]{avideyxyu sAdi eMba pakaSxdalUlx savxBAvapakaSxdaMte moVkaSxvu avidAyxnAshakekx beVreyAgide yAdadxriMda adu huTuTxva pakaSxdalUlx moVkaSxvilalxde hoVguvudeMba doVSavu baruvudilalx. adariMda `sAdiyo? AnAdiyo? eMba vicAravU yukatxveV eMdu keVLidare utatxra avideyxyu sAdiyAdalilx adara kAraNavanunx huDukabeVkAguvudu niviRkAra Atamxnu kAraNavenunxvudakUkx bAradu, avideyxyilalxde adu kAraNavAgadu. pUvARvideyxyu muMdina avideyxge kAraNaveMdu heVLalu bAradu, adariMda anAdiyAda avideyxyeV adakekx kAraNavenanx beVku AvAgalU I vikalapxvu huTaTxdu A kAraNadiMda savxBAvavo? alalxvo eMbuva vicAraveV yukatx. hiVgeMdu anaMda girigaLu I meVlina eraDu vAtiRkagaLa athaRvanunx mathana mADi heVLidAdxre.}
\begin{shl}
\footnotemark[2]avidAyxkAmakamARdiparxvivikatxmihA\s \s tamxnaH || \\
rUpaM pUvaRmupanayxsatxM tasayx sAkASxcicxkiVSaRyA \hfill || 1209 ||  
\end{shl}

%%%%%shloka footnote[2]
\begin{artha}
ililx Atamxna savxrUpavu avideyx kAma kamaR modalAdavugaLiMda biDalapxTiTxdaMte hiMde (suSupitxyalilx) heVLalapxTiTxde. Iga adanunx parxtayxkaSxvAgi parxtipAdane mADalu iceCxyiMda (vicAra mADuvudu)
\end{artha}

\vishaya{nADiVvAkayxkekx matotxMdu aBipArxyavide-}

\begin{shl}
avidAyxyAshacx yatAkxyaRM tacacx vAcayxmasheVSataH || \\
itAyxviSakxqqtisidadhxyXthaRM paroV garxnothxV\s vatAyaRteV \hfill || 1210 ||  
\end{shl}

\begin{artha}
avideyxya kAyaRvu yAvuduMTo, adelalxvanunx heVLa beVkAdadudx, idanunx sapxSaTxpaDisuvudakAkxgi muMdina garxMthavu iLisalapxTiTxde.
\end{artha}

\section*{muMdina garxMtha (baq - 4 - 4 - 3 - 20 ne kaMDike)}

\begin{shl}
tA vA aseyxYtA hitA nAma nADoyxV yathA keVshaH sahasarxdhA BinanxsAtxvatANimAnx tiSaThxnitx shukalxsayx niVlasayx piknagxlasayx haritasayx loVhitasayx pUNAR atha yaterxYnaM GanxnitxVva jinanitxVva hasitxVva vicACxyayati gataRmivapatati yadeVva jAgarxdaBxyaM pashayxti tadatArxvidayxyA manayxteV\s tha yatarx deVva iva rAjeVvAhameVveVdaM savoVR\s simxVti manayxteV soV\s sayx paramoV loVkaH || 20 ||
\end{shl}

\vishaya{vAtiRka}

\vishaya{nADiV viSayavanunx heVLuva udedxVsha-}

\begin{shl}
nADAyxyatAtx yatoV\s vidAyxkAyaRdaqSiTxrataH paramf || \\
nADiVnAM sAyxdupanAyxsasAtx vA aseyxVtivAkayxtaH \hfill || 1211 ||  
\end{shl}

\begin{artha}
avideyxya kAyaRvAda (kataqRtAvxdi) budidhxyu nADigaLige adhiVnavAgide, adariMda muMde nADigaLa viSayavanunx `\stext'eMba vAkayxdiMda heVLide.
\end{artha}

\vishaya{athavA nADiV viSayoVpanAyxsakekx beVre PalavU ide-}

\begin{shl}
yadAvx maqSAtavxsidadhxyXthaRM nADuyxpanAyxsa iSayxteV || \\
atayxnatxtanutoV hayxnatxviRnAdhxyXdeVriVkaSxNaM kutaH \hfill || 1212 ||  
\end{shl}

\begin{artha}
athavA (eraDu avasethxgaLU) mitheyxyeMdu sididhxsalu nADiV viSayoVpanAyxsavu beVkAgide (adariMda mitheyxyeMdu heVge? sididhxsuvudeMdare)- nADigaLu atayxMta sUkaSxvAgiruvudariMda adaroLage (doDaDxdAda) viMdhayx pavaRtAdigaLu kANuvudu EtariMda heVge? (eMdu keVLabeVku)-
\end{artha}

\footnotetext[1]{avideyxyu aGaTitaGaTanAsamathaRvAdudu aMdare hoMdadiruvudanunx hoMdisalu samathaRvAdadudx, adariMda sUkaSxmXnADigaLalilx ati doDaDxdAda pavaRtAdigaLu kANuvudu, avideyxyiMda, idu yukatxveMdu shaMkiside. adanenxV muMde aMgiVkariside, avideyxyu asatayxvAdaMte adariMda kANuva savxpanx jAgara dashegaLu adara viSayagaLU satayxvalalxveMdu nADiV viSayada upanAyxsavu upaniSatitxnalilx baMdideyeMdu tAtapxyaR.}
\begin{shl}
avidAyxkAyaRmeVtacecxVtf\footnotemark[1] avidAyxvanamxqqSeVSayxteV || \\
atoV maqSAtavxsidadhxyXthaRM tanunADiVparigarxhaH \hfill || 1213 ||  
\end{shl}

%%%%%shloka footnote[1]
\begin{artha}
idu avideyxya kAyaRveMdu utatxrisidedxV Adare avideyxyaMte mitheyxyananx beVkAguvudu, adariMda mithAyxtavxvu sididhxsuvudakAkxgi sUkaSxmXnADigaLanunx (shurxtiyu heVLalu) etitxkoMDide.
\end{artha}

\vishaya{nADi garxMthada vAyxKAyxnAraMBa}

\begin{shl}
tA vA asayx hitA nAma nADayxH sUkASxmX haqdi sithxtAH || \\
apakaSeVRtarw yABiH kAyaRsAyx\s \s tAmx samiVkaSxteV \hfill || 1214 ||  
\end{shl}

\begin{artha}
A nADigaLu hita eMba hesarinavu, shariVrakekx saMbaMdhisi, sUkaSxmXvAgi haqdaya koVshada meVle iruvavu, yAva nADigaLiMdaleV kamaRPalada utakxSaR apakaSaRgaLanunx Atamxnu noVDuvano, (A nADigaLeV hitanAmaka nADigaLu).
\end{artha}

\vishaya{`yathAkeVshaH sahasarxdhABinanxH' eMbudara athaR}

\begin{shl}
sahasarxtamaBAgeVna keVshasayx pariNAhataH || \\
nADoyxV\s NimAnx samAnAsAtx nAnAnanxrasasaMBaqtAH \hfill || 1215 ||  
\end{shl}

\begin{artha}
heVge (oMdu tale kUdalu sAvira BAgavAdalilx) adara sAvirada oMdu BAgavu parimANadalilx saNaNxdAgiruvudo hAgeyeV A nADigaLu sUkaSxmXvAda parxmANada adakekx samAnavAgidudx aneVka bageya ananx rasagaLiMda tuMbiruvavu.
\end{artha}

\vishaya{adariMda Atamxnige EnAyitu? eMdare-}

\begin{shl}
tAsavxnanxrasapUNARsu sa AtAmx\s vidayxyA\s \s tamxnaH || \\
rakatxpiVtAdirUpatavxM manayxteV savxpanxmadhayxgaH \hfill || 1216 ||  
\end{shl}

\begin{artha}
ananxrasa sapUNaRvAda A nADigaLalilx A Atamxnu Atamxna ajAcnxnadiMda rakatxpiVta muMtAda vaNaRvanunx savxpanx madhayxdalilxruvavanu tananxlilx aBimAnisuvanu.
\end{artha}

\begin{shl}
rasavaNARnuroVdheVna rakatxpiVtAdirUpatAmf || \\
parxtiVcoV BoVgasidadhxyXthaRM deVvateYti maqSAtimxkAmf \hfill || 1217 ||  
\end{shl}

\begin{artha}
ananxrasadalilxruva baNaNxvanunx anusarisi asatayxvAda keMpu haLadi muMtAda rUpavanunx parxtayxgAtamxnige BoVgasididhxgAgi (nADiyalilxruva) deVvateyu hoMduvudu.
\end{artha}

\vishaya{nAlakxneV adhAyxyadalUlx AraneV adhAyxyadalUlx nADigaLanunx heVLidadxriMda punaH heVLuvudakekx PalaveVnu?-}

\footnotetext[3]{I vAtiRkadalilx jAcnxnavu bwdadhxveMdide, aMdare budidhxya pariNAmavAda vaqtitxjAcnxna, GaTapaTAdi jAcnxna, idu jAgarAvasethxyalilx matutx savxpanxdalilx vikAsavAguvudu hecucxvudu, aMdare biDibiDiyAgi oMdoMdara visheVSajAcnxnavAguvudu, I vikAsaveV vAtiRkadalilx vikeSxVpaveMdu kareyalapxTiTxde, idara saMhAra eMdare saMkoVcavAguvudu, saMkoVcaveMdare mudurikoLuLxvudeMdu sAmAnAyxthaR, kamalavu hagalu araLidudx rAtirxyalilx muduri mogigxnaMtAguvudu, A athaRvu ililxlalx matetxVneMdare visheVSavAgi biDibiDiyAgi jAcnxnavAguvudu hoVgi suSupitxyalilx sAmAnayx riVtiyalilx EkaceYtanayx savxrUpadalilx nilulxvudu, athavA citatx vaqtitxgaLu niMtiruvudariMda saMkucitavAyiteMdu heVLiruvudu, idanunx heVLalu nAlakxneV adhAyxyadalilx nADigaLanunx heVLide. I riVtiyAda vaqtitx vikAsa saMkoVcagaLige dAvxravAdadudx nADigaLu, adariMda heVLide, idU saha tavxMpadAthaRda shoVdhanegAgi, I elAlx saMkoVca vikAsagaLigU cidUrxpavAda AtamxneV sAkiSxyenunxvudariMda Atamxnu shudadhxnAgiruvaneMdu tiLiyutatxdeyeMdu I vAtiRkada vivaraNe.}
\begin{shl}
\footnotemark[3]bwdadhxvijAcnxnavikeSxVpasaMhArakathanAya tu || \\
\footnotemark[4]catutheVR nADuyxpanAyxsoV vijAcnxnAtamxvishudadhxyeV \hfill || 1218 ||  
\end{shl}
\footnotetext[4]{``atha yadA suSupotxV Bavati yadA nakasayxcana veVda hitAnAma nADoyxVdAvxsapatxtiH sahasArxNi haqdayAtf puriVtata maBiparxtiSaThxnetxV.....'' eMdu I upaniSatitxna 2neV adhAyxyada 16 ne kuDikeyalilx nADigaLanunx heVLide hAgU upaniSatitxna (4ne a. bArx. 2)-3 ne kaMDikeyalilx ``yathAkeVshaH sahasarxdhABinanxH Eva maseyxYtAhitAnAmanADoyxV\s natxhaqRdayeV parxtiSiThxtA Bavanitx EtABivAR EtadAsarxva dAsarxvati'' eMdu heVLide baqhadAraNayxka lekakxdalilx 9ne adhAyxyavAgutatxde. hiMdinadu 4ne adhAyxyavAgutatxde.}

%%%%%shloka footnote[3, 4]
\begin{artha}
budidhxya pariNAmavAda vijAcnxnada vikAsa matutx saMhAravanunx heVLuvudakokxVsakxra nAlakxne adhAyxyadalilx nADiyanunx heVLiruvudu, alalxde vijAcnxnAtamxna shoVdhanegAgi (heVLiruvudu).
\end{artha}

\begin{shl}
deVvatAnanxrasANutavxjAcnxnAya ca punagarxRhaH || \\
SaSAThxdw sUkaSxmXnADiVnAM tadUdxrA\s \s tAmxvabudadhxyeV \hfill || 1219 || 
\end{shl}	

\begin{artha}
deVvateya ananxrasada sUkaSxmXteyanunx tiLiyuvudakAkxgi AraneV adhAyxyada Adiyalilx sUkaSxmXnADigaLanunx punaH shurxtiyu tegedu koMDide, alalxde A mUlaka Atamxvanunx tiLiyuvudakAkxgi tegedukoMDide.
\end{artha}

\vishaya{Iga ililx nADigaLanunx heVLuvudakekx Enu parxyoVjana? eMdare-}

\begin{shl}
BoVkutxH savxrUpavijacnxpAtxyX iha nADiVparigarxhaH || \\
kirxyateV kAmakamARdiviveVkasayx vivakaSxyA \hfill || 1220 ||  
\end{shl}

\begin{artha}
ililx punaH nADiV viSayavanunx tegedukoMDiruvudu kAma matutx kamaR modalAda viveVkavanunx heVLuva udedxVsha mADiruvudariMda jiVva savxrUpavanunx tiLisuvudakAkxgi ||
\end{artha}

\vishaya{jiVvAtamxnu kAmAdirUpavuLaLxvanAgiralu avaniMda kAmAdigaLanunx heVge beVpaRDisuvudu?}

\begin{shl}
na vijAcnxnAtamxnoV rUpaM ceYtanAyxdanayxdiSayxteV || \\
sAvxtAmxvideyxYkaheVtevxVva tasayx rUpAnatxraM yataH \hfill || 1221 ||  
\end{shl}

\begin{artha}
vijAcnxnAtamxnige (jiVvanige) ceYtanayxkikxMta beVreyoMdu rUpavanunx opupxvudilalx, kAraNaveVneMdare? Atanige tananx nijavAda savxrUpavu tiLiyadiruva oMdeVkAraNadiMda beVre rUpavu baruvudaSeTxV (adariMda ceYtanayx savxrUpavoMde beVreyilalx)
\end{artha}

\begin{shl}
vijAcnxnapuruSANAM hi saveVRSAmeVkarUpatA || \\
ceYtanAyxtamxtayA jecnxVyA tadaboVdhAtutx BinanxtA \hfill || 1222 ||  
\end{shl}

\begin{artha}
vijAcnxna puruSarelalxrigU (budidhx upAdhiyAgiruva elAlx jiVvarigU) ceYtanayxsavxrUpadiMdaleV EkarUpaveMdu tiLiyabeVku, adara savxrUpavu tiLiyadiruvadariMda mAtarx avaralilx Binanxteyu (kANuvudu).
\end{artha}

\vishaya{adeVke eMdare?-}

\begin{shl}
adhAyxtAmxdiviBAgoV\s yaM na savxtaH parxtayxgAtamxnaH || \\
tadaboVdheYkaheVtutAvxdapArxpetxYkAtamxyXvasutxkaH \hfill || 1223 ||  
\end{shl}

\begin{artha}
I adhAyxtamx muMtAda viBAgavu parxtayxgAtamxnige savxtaH iruvudilalx, adara tatAtxvXjAcnxnavoMdeV kAraNavAgiruvudariMda EkAtamx savxrUpavAda vasutx laBisade (I viBAgavu Agiruvudu).
\end{artha}

\vishaya{hAgAdare BeVda rUpavu yAvudaradudx?}

\begin{shl}
liknagxmeVva tatoV rUpamAtAmxvideyxYkaheVtukamf || \\
kamoVRtAthx BAvanA yAshacx rUpaM tA api liknagxvatf \hfill || 1224 ||  
\end{shl}

\begin{artha}
adariMda Atamxna avideyxyeMba heVtuviniMda Ada BeVdarUpavu liMgavanenxV Asharxyiside (aMdare avideyxyiMdAda liMgashariVraveMba budidhxyanenxV avalaMbiside) hAgeyeV kamaRdiMda Aguva yAva kataqRtAvxdi vAsanegaLU saha liMga shariVradedxV Agive liMga shariVradalilxruva BeVda rUpadaMte ivu ive.
\end{artha}

\vishaya{kataqRtAvxdigaLu liMga shariVradalilxruvudAdare aMtha shariVravu Eke parxtayxkaSxvAgilalx?-}

\begin{shl}
liknagxM cApayxtisUkaSxmXM tanAnxDiVmadhayxgatikaSxmamf || \\
sAvxmikamaRvashAtatxcacx rUpaM gaqhANxti deYvatamf \hfill || 1225 || 
\end{shl}

\begin{artha}
(budidhxyeMbuva) \footnote{liMga shariVraveMdare paMcajAcnxneMdirxya, paMca kameRVMdirxya paMca pArxNagaLu budidhx, manasusx oTuTx I hadineVLu AdarU I saMdaBaRdalilx avugaLalilx muKayxvAgi budidhx tatatxvXvanenx tegedukoLaLx beVku, idu `budidhxtatatxvX, samaSiTxpArxNatatatxvXda aBimAni yAda hiraNayxgaBaRH veMbuvadeVvatA savxrUpavAdudu. Eka Eva pArxNoVdeVvatA' eMdu hiMdeyeV tiLiside vayxSiTxyAda budidhx pArxNagaLa tatatxvXvU samaSiTxyalelxV seVrikoMDiruvudariMda vAyxpakavAda samaSiTx deVvatA savxrUpaveV idu eMdu `deYvatamf' eMbudariMda vAtiRkadalilx sUcitavAgide. idu sUkaSxmXvAdadxriMda sUthxla shariVradaMte parxtayxkaSxvAguvudilalx, idu vasutxtaH aceVtanavAdarU jiVvAtamxna kamARdhiVnavAgidudx adaqSaTxvashadiMda shabadx sapxshaR rUparasa gaMdhagaLeMba horagina viSayagaLanunx hoMduvudu bAhayx viSayAkAravanunx hoMduvudu, iMdirxyagaLa mUlaka AyAya viSayagaLu aMtaHkaraNadalilx parxtibiMbisi araginalilx acucxhoyadxMte adaralilx toVruvuvu.}liMga shariVravU saha ati sUkaSxmXvAgide adu nADigaLa naDuve athavA nADigaLoLage saMcarisuvudakekx samathaRvAgide, A deVvatAtamxvAda liMga shariVravu jiVvana kamARdhiVnavAgi (shabadxsavxshARdi viSayagaLa) rUpavanunx garxhisutatxde.
\end{artha}

\footnotetext[2]{I riVtiyAgi budidhxyu viSayAkAravanunx hoMdidalilx PalaveVnu? eMdu keVLidare Atamxna BoVgaveV Pala, budidhxyalilx biMbisida shabadx savxrARdigaLanunx jAgaradalilx anuBavisuvudeMbudeV BoVgavu Atamxna BoVgavanunx biTuTx budidhxyu viSayAkAravAguvudakekx beVroMdu Palavilalx, hAgAdare Atamxnu sakirxyanAguvudilalxve? eMdare ilalx, I BoVgavu budidhxyalilx parxtibiMbisida ceYtanAyx BAsamAtarxdalilxruvudariMda shudadhxvAda Atamxnalilx I BoVgavikAravu leVpisuvudilalx, Adare AtamxneV tananx BoVgakAkxgi liMgoVpAdhiyilalxde neVra viSayAkAravanunx hoMduvudilalx kAraNa adu yAvAgalU asaMga savaRsaMbaMdha shUnayxveMdu tAtapxyaR}
\begin{shl}
\footnotemark[2]BoVgamAtarxparxsidadhxyXthaRM ceYtanAyxBAsamAtarxtaH ||  \\
nAnAtamxrUpamAtAmx\s yaM kiMcidanayxdapeVkaSxteV \hfill || 1226 ||  
\end{shl}

%%%%shloka footnote[2]
\begin{artha}
BoVga mAtarx sididhxsalu I Atamxnu ceYtanAyxBAsavAgiruva rUpakikxMta matotxMdu anAtamx rUpavanunx oMdanUnx beVDuvudilalx hAgAdare BoVgavAguvudu heVge?
\end{artha}

\begin{shl}
vijAcnxnapuruSaH soV\s yamadhideYvAdivAsanaH || \\
liknagxdeVhAnuroVdhiV salilxknegxV katArxRdivikirxyAmf || \\
avidoyxVpahatAtAmx sanAnxtamxtevxVnABimanayxteV \hfill || 1227 ||  
\end{shl}

\begin{artha}
A I vijAcnxnamayanAda Atamxnu adhideYvAdivAsanegaLuLaLxvanAgi liMgashariVravanunx anusarisutAtx, liMgashariVradalilxruva kataqRtAvxdivikAravanunx anusarisutAtx, liMga shariVradalilxruva kataqRtAvxdivikAravanunx ajAcnxnadiMda mucicxda nijasavxrUpavuLaLxvanAgi tananxdAgi aBimAnisuvanu.
\end{artha}

\vishaya{vAtiRka}

\vishaya{liMga shariVrada I vikAravU savxtaH adakekx iruvudilalx:-}

\begin{shl}
liknagxsayx vikirxyA yA\s pi sA rasAnuvidhAyiniV \hfill || 1228 ||  \\
adhideYvAdideVhasayx rasaheVtuH sithxtiyaRtaH || \\
EtABivAR iti tathA teVneYtA iti ca shurxtiH \hfill || 1229 ||  
\end{shl}

\begin{artha}
liMga shariVrada (budidhxya) yAva vikAravideyo adU saha ananxrasavanunx anusarisi Aguvudu, EkeMdare? adhideYvAdi shariVrada sithxtiyu ananxrasada nimitatxvAgiruvudu, idakekx ``EtABivAR EtadAsarxvadAsarxvati'' eMba shurxtiyU matutx ``teVneYtA............'' eMbudU (parxmANavAgide).
\end{artha}

\begin{shl}
rasoV\s pi cAyaM jagAdhxnanxpariNAmamapeVkaSxteV ||  \\
yatoV\s toV vAtapitAtxdiVnAdhxtUnosxV\s payxnurudhayxteV \hfill || 1230 ||  
\end{shl}

\begin{artha}
yAvudariMda I rasavU kUDa tiMda ananxda pariNAmavanunx apeVkiSxsutatxdeyo adariMda vAta, pitatx modalAda dhAtugaLanunx rasavu anusarisutatxde.
\end{artha}

\vishaya{adakekx kAraNavidu-}

\begin{shl}
dhAturUpAnuroVdhiV sanfrasoV rakAtxdimeVtayxtha || \\
taM vaNaRM deVvatAdeVhoV yathoVkatxmanurudhayxteV \hfill || 1231 ||  
\end{shl}

\begin{artha}
dhAtugaLa baNaNxvanunx anusarisi rasavu rakatx modalAda baNaNxvanunx hoMdutatxde, A vaNaRvanunx deVvatAshariVravu (liMga shariVravu) hiMde heVLidaMte anusarisutatxde.
\end{artha}

\vishaya{AdarU parxtayxgAtamxnige EnAyiteMdu keVLidare-}

\begin{shl}
AtamxnoV\s vikirxyaseyxYva liknegxV savxBAsavatamxRnA ||  \\
dashaRnaM shukalxpiVtAdeVjARyateV kamaRNoV vashAtf \hfill || 1232 ||  
\end{shl}

\begin{artha}
vasutxtaH niviRkAranAgiruva Atamxnige liMgashariVradalilx tananx cidABAsada mUlaka biLupu, keMpu muMtAda budidhxyu kamaRvashadiMda Aguvudu'' (anaMtara BoVgavu sididhxsuvudu).
\end{artha}

\vishaya{kamaRdiMda shukAlxdivaNARnuBavavAguvudAdare daqshayxvu vAsatxvavAguvudilalxve? eMdare-}

\begin{shl}
AtamxneyxVSa yataH savaRsatxdavideyxYkaheVtukaH || \\
adhAyxtAmxdiparxpacnocxV hi shukitxkArajatAdivatf \hfill || 1233 ||  
\end{shl}

\begin{artha}
AtamxnaleV I elalxvU adara avideyxyeMba oMdu nimitatxdiMda AgiruvudariMda adhAyxtamx modalAda parxpaMcavu mutitxna cipitxnalilx toVruva beLiLx modalAdavugaLaMte (asatayx).
\end{artha}

\vishaya{elalxvU asatayxvAdare `satayx anaqta' eMba viBAga heVgAyitu?-}

\begin{shl}
satAyxnaqtaviBAgoV\s yaM tathA\s payxnaqtavasutxni || \\
jAgarxdUBxmw parxsidodhxV\s sayx loVkeV savxpenxV yathA tathA \hfill || 1234 ||  
\end{shl}

\begin{artha}
AdarU elalxvU anaqta vasutxgaLAdarU satayx anaqtaveMba viBAgavu savxpanxdalilx heVgoV hAge loVkadalilx jAgarxtitxna dasheyalUlx iruvudu.
\end{artha}

\begin{shl}
citasxvXBAvAtireVkeVNa nAvidAyxdeVH parxsidhayxti || \\
satayxtavxmiva mAneVna mithAyxtavxmapi vasutxnaH \hfill || 1235 ||  
\end{shl}
				
\begin{shl}
cinAmxtarxteYva teVnAsayx kAyaRkAraNavasutxnaH || \\
shurxteyxVhoVcecxYH parxtayxpAdi tatatxvXmasAyxdirUpayA \hfill || 1236 ||  
\end{shl}

\begin{artha}
ceYtanayxsavxBAvakikxMta beVreyAgi avidAyxdivasutxgaLige satavxtavxdaMte vasutxvige mithAyxtavxvU parxmANadiMda sididhxsuvudilalx. adariMda I kAyARkAraNa vasutxgaLu ceYtanayxmAtarxsavxrUpagaLeMdu I shAsatxrXdalilx tatxvXmasi modalAda rUpadalilx shurxtiyu gaTiTxyAgi parxtipAdisuvudu.
\end{artha}

\vishaya{Ivarege jAgarxtitxna parxpaMcakekx mithAyxtavxvanunx heVLide inunx muMde ``atheyaterxYnaM GunxnitxVva''.....itAyxdi vAkayxdiMda savxpanx parxpaMcada mithAyxtavxvanunx heVLuvaru-}

\begin{shl}
mithAyxBAvanamAtArxdheVniRrasatxvikaqteVriha || \\
parxtiVcoV dashaRnaM savxpenxV shurxtAyx mitheyxVti vaNayxRteV \hfill || 1237 ||  
\end{shl}

\begin{artha}
mithAyxvAsane mAtarxveV upAdhiyAgiruva savaRvikArashUnayxnAda parxtayxgAtamxnige savxpanxdalilx kANuvudu mitheyxyeMdu shurxtiyiMda vaNiRsalapxDuvudu.
\end{artha}

\vishaya{Iga `shukalxsayx niVlasayx piknAgxlasayx'....itAyxdi vAkayxda tAtapxyaR-}

\begin{shl}
shukalxM sAyxtakxPaBUyasetxvXV vAtapitatxsamAgamAtf || \\
niVlaM kaqSaNxM ca \footnotemark[1]vAtasayx pariNAmaM parxcakaSxteV \hfill || 1238 ||  
\end{shl}
\footnotetext[1]{vAtavu hecAcxdare shelxVSamxvU pitatxvU savxlapx kaDimeyAgidudx tiMda ananxda pariNAmavu kapApxguvudeMdathaR.}

%%%%shloka footnote[1]
\begin{artha}
kaPavu (shelxVSamxvu) jAsitxyAdare vAtapitatxgaLu kamimxyAgi seVruvudariMda (nADiyalilxruva ananx rasavu) beLaLxgAguvudu, niVlavaNaR aMdare kapupx baNaNxvu vAtavu hecAcxdare adara pariNAmaveMdu heVLutAtxre.
\end{artha}

\begin{shl}
piknagxlaM pitatxbAhulAyxdadhxritoV manadxpitatxkaH || \\
iti veYSamayxtoV vaNoVR dhAtusAmeyxV\s tiloVhitaH \hfill || 1239 ||  
\end{shl}

\begin{artha}
pitatxvu hecAcxgiruvudariMda (ananxrasavu) haLadi baNaNxvAgiruvudu (vAta shelxVSamxgaLu jAsitxyAgidudx) pitatxvu kaDimeyAgidadxlilx adu hasiru baNaNxvAguvudu, I riVtiyAgi dhAtu veYSamayxdiMda baNaNxvu badalAgavudu, dhAtugaLu samavAgidadxlilx bahaLa keMpAgiruvudu.
\end{artha}

\footnotetext[1]{``susharxtanUkUDa aruNAH shirA vAtavahA niVlAH pitatxvahAH shirAH | asaqgavxhAsutxroVhiNoyxVgwyaRH shelxVSamxvahAH shirAH || yAjacnxvalakxyXrU saha - ananAtx rashamxya satxsayxdiVpavadavxyXvasithxtAhaqdi | sitAsitAH kaNuDxniVlAH kapilAH piVtaloVhitAH ||'' athaR keMpAda nADigaLu vAta rakatxvanunx harisuvuvu, niVlavAda nADigaLu pitatx rakatxvanunx, keMpAda nADigaLu rakatxvanunx harisuvuvu, beLaLxgiruva nADigaLu shelxVSamx rakatxvanunx harisuvuvu eMdu sushurxta kArikeya athaR oLapu, hoMbaNaNx, hasirubaNaNx, keMpu baNaNx I bageya baNaNxda ananxrasagaLiMda sUkaSxmXvAda nADigaLu pUNaRvAgiveyeMdu heVLideyaSeTx, avugaLalilx rasagaLige vividhavAda baNaNxgaLu vAtapitatx shelxVSamxgaLu oMdakokxMdu seVruvAga hecucx kaDimeyAgi viSama riVtiyalilx seVridare vicitarxvAda baNaNxgaLu uMTaguvuvu adu heVgeMdare- tiMdananxvu pariNAma visheVSavanunx hoMduvudu. vAtavu hecAcxgidadxre ananxrasavu niVli baNaNxvAguvudu, pitatxvu hecAcxdare hoMbaNaNxvAguvudu, shelxSamXvu hecAcxdare biLupAguvudu, pitatxvu kaDimeyAdare hasirAguvudu. dhAtugaLu samanAgidadxre keMpAgiruvudu, hiVge parasapxra vAtapitatx shelxVSamxgaLu seVrikeyalilx veYSamayxvuMTAguvudariMdalU sAmayxdiMdalU vicitarxvAgiyU aneVkavAgiyU ananxrasagaLAguvavu sushurxta garxMthadalilx idanunx visatxrisideyeMdu (AnaMda girigaLu sUcisuvaru) I bageyAgi nAnA vaNaRda ananxrasagaLiMda tuMbiruva sUkaSxmXnADigaLu deVhadalelxVlAlx vAyxpisiruvavu avugaLalilx liMga shariVravu (hadineVLu padAthaRgaLa guMpu) iruvudu, I liMga shariVradalelxV aneVka bageya vAsanegaLu aneVka bageya saMsAra dhamaRda anuBavadiMda huTiTx koMDiruvuvu, vAsanAsharxyavAda liMgavu sUkaSxmXvAdadxriMda savxcaCxvAgi sapxTikamaNiyaMtidudx nADiyalilxruva ananxrasagaLa upAdhi saMpakaRdiMda pApa puNayxgaLa perxVraNAnusAra aneVka vaqtitxgaLu tananxlilx huTiTx, ratha, gaja, sitxrXV muMtAda visheVSagaLiMda kUDi vAsanegaLiMda toVruvudu -ideV muMde vaNiRsuva savxpanx padAthaRgaLa vivaraNe-}
\begin{shl}
\footnotemark[1]iteyxVvamAdi bahudhA pariNAmaM parxcakaSxteV || \\
cikitAsxshAsarxtatatxvXjAcnxjagadhxsAyxnanxsayx nADigamf \hfill || 1240 ||  
\end{shl}

%%%%%shloka footnote[1]
\begin{artha}
I riVtiyAgi veYdayxshAsatxrXda tatatxvXvanunx tiLidavaru ideV modalAda riVtiyalilx aneVka bageyAgi tiMda ananxda pariNAmavu nADigaLalilxruvudanunx heVLuvaru.
\end{artha}

\begin{shl}
atheYtasimxnanxnanxrasapariNAmeV yathoVditeV ||  \\
parxtayxknomxVhasayx yatAkxyaRmadhunA tatapxrXpacnacxyXteV \hfill || 1241 || 
\end{shl}

\begin{artha}
anaMtara IriVtiyAgi hiMde heVLida ananxrasada pariNAmadalilx parxtayxgAtamxna ajAcnxnada kAyaRvanunx Iga vivarisutAtxre.
\end{artha}

\vishaya{savxpanxdalilx avideyxya kAyaRvilalx ``na tatarx rathAH" eMba shurxtiyaMte adanunx visatxrisuvudakekx muMdina vAkayxveMbudu heVge yukatx? -}

\begin{shl}
jagadAtamxni nimARya sAdhiBUtAdhideYvatamf || \\
shukAlxdAyxkaqtinADiVsathxmAtAmx pashayxtayxvidayxyA \hfill || 1242 ||  
\end{shl}

\begin{artha}
avideyxyiMda Atamxnalilx adhiBUta, adhideYvata sameVtavAgi jagatatxnunx nimiRsi shukalx muMtAda Akaqtiya nADigaLalilxruvudanunx Atamxnu noVDuvanu.
\end{artha}

\vishaya{avideyxya paramAvadhiyanunx tiLisuvudu-}

\footnotetext[1]{``athayoV\s nAyxM deVvatAM pashayxti'' itAyxdiyAgi AraMBisi avideyxya kAyaRgaLanunx visatxriside yAdarU hiMde ecacxravanunxMTu mADuva avideyxya kAyaRgaLanunx visatxrisididxtu, ililx savxpanxkekx kAraNavAda avideyxya visAtxravanunx heVLiruvudu. oMdeV avideyxyu kAyaRBeVdadiMda BinenxYside, nijavAgi BinanxvAgilalx ideV sidAdhxMta}
\begin{shl}
\footnotemark[1]avidAyxyAH parA kASAThx\s theVdAniVmucayxteV suPxTamf || \\
tatAkxyaRtAratameyxVna sA\s vidAyx BidayxteV yataH \hfill || 1243 ||  
\end{shl}

%%%%%shloka footnote[1]
\begin{artha}
avideyxya paramAvadhiyanunx Iga muMde sapxSaTxvAgi heVLuvudu adara kAyaRgaLu taratamavAgi (BinanxBinanxvAgi)ruvudariMda A avideyxyu BinanxvAguvudu.
\end{artha}

\vishaya{sAvxBAvikavAgiyeV avidAyx BeVdavanunx heVLuvavara mata nirAkaraNe-}

\begin{shl}
liknAgxdikAyaRBeVdeVna sA\s vidAyx BidayxteV sadA || \\
savxsatxvXvidAyxBeVdoV\s tarx manAgapi na vidayxteV \hfill || 1244 ||  
\end{shl}

\begin{artha}
liMga shariVra modalAda kAyaRvasutxgaLa BeVdadiMda A videyxyu yAvAgalU BinenxYsutatxde savxtaH savxrUpadalilx savxlapxvU adaralilx BeVdaviruvudilalx.
\end{artha}

\vishaya{A liMgashariVravu heVgideyeMdare}

\footnotetext[1]{liMga shariVraveMdare jAcnxneVMdirxya aidu, kameRVMdirxya aidu, aidu pArxNApAnAdigaLu budidhx manasusx oTuTx I hadineVLu idu aNuvinaSuTx sUkaSxmXvAdadxriMda sUthxlashariVradaMte kaNiNxge kANisuvudilalx sapxTikadaMte savxcaCxnimaRlavAdadxriMda avidAyx saMsAkxragaLu, jAgarxtf kAlada saMsAkxragaLu hiVgeyeV janamxjanAmxMtarada saMsAkxragaLu (vAsanegaLu) adaralilx aDagikoLuLxvuvu. nADigaLu sUkaSxvAda ananxrasagaLiMda tuMbiruvudariMda avugaLalilx niVla, piVta, shukAlxdi vaNaR saMsAkxragaLu iruvudu yukatxvAgidadxrU nADiyalilxruva nAnA rasagaLeMba upAdhi saMpakaRdiMda liMga shariVraveV vAsanAsharxyavAgi kANuvudu saPxTikavu nimaRlavAgidadxrU dAshavALa hUvina saMpakaR keMpAgi, niVli baNaNxda hUvina samiVpadalilxruvAga adara saMpakaRdiMda niVlavAgiyU kANuvudu adaraMte idU kANuvudu.}
\begin{shl}
\footnotemark[1]talilxknagxM vAsanAniVDaM sUkaSxmXyXM savxcaCxsavxBAvakamf || \\
nADiVgatarasoVpAdhisasaMgARtasxPXTikAdivatf \hfill || 1245 || 
\end{shl}

%%%%shoka footnote[1]
\begin{artha}
A liMga shariVravu vAsanegaLige AsharxyavAgi sUkaSxmXvAgi saPxTikAdigaLaMte vAda savxBAvavuLaLxdAdxgiruvudu. alalxde nADigaLalilxruva ananxrasagaLeMba upAdhi saMbaMdhadiMda (aneVka vaNaRgaLuLaLxdAgide).
\end{artha}

\vishaya{I riVtiyAgi liMga shariVravu sAkiSxge parxtayxkaSxvenunxtAtxre-}

\footnotetext[2]{idu savxpanxdalilx sAkiSxge goVcarisuvudu aMdare idu kAma korxVdhAdi nAnA vaqtitxgaLa rUpadalilx (liMga shariVradalilx muKayxvAgi) aMtaHkaraNavu toVruvudu, aMdare jAgarxtAkxladalilx BoVgavanunxMTu mADuva adaqSaTx modalAda kAraNadiMda huTiTxkoMDa ratha kudure, ane, sitxrXV muMtAda nAnA kAragaLiMda kUDidudx tiMdu kuDidu idadx ananx nAnAdigaLa paripAkadiMda pariNamisida rasagaLa sahakAradiMda kUDi kAma korxVdhAdi vaqtitxgaLa rUpadalilx sAkiSxge goVcarisuvudeMdu tAtapxyaR}
\begin{shl}
\footnotemark[2]dhamARdiperxVrakoVdUBxtarathasitxrXVhasitxlakaSxNa \\
nAnAkaqtirasAdAyxtamx hAyxtamxnoV\s lupatxcakuSxSaH || \\
parxthateV puratoV\s vidAyxmAtarxtatatxvXM vinashavxramf \hfill || 1246 ||  
\end{shl}

%%%%%%shloka footnote[2]
\begin{artha}
adu dhamaR modalAda perxVrakagaLiMda uMTAda ratha, sitxrXV, Ane muMtAda savxrUpada aneVka AkAragaLiMda kUDidudx (tiMda) ananxrasAdi rUpavAgiyU idudx, nitayx daqSiTxyuLaLx Atamxnige (sAkiSx ceVtanakekx) edurAgi toVruvudu Adare adu avidAyx mAtarx savxrUpavuLaLxdudx (nijavAdadadxlalx) nashisuvaMtahudu.
\end{artha}

\begin{shl}
EvaM tAvadavideyxYkaniVDAnAmuditoV vidhiH || \\
BAvanAnAM viniSapxtwtx boVdheV savxpenxV\s tha BaNayxteV \hfill || 1247 ||  
\end{shl}

\begin{artha}
I riVtiyAgi avideyxyoMdeV AsharxyavAgiruva vAsanegaLu jAgarxtitxnalilx huTuTxva viSayakekx niyamavanunx heVLidAdxyitu, inunx muMde savxpanxdalilx (avugaLu avideyxya kAyaRveMbudanunx) heVLuvudu.
\end{artha}

\vishaya{``atheyaterxYnaM GunxnitxVva'' itAyxdi padagaLige vAyxKAyxna}

\begin{shl}
atha yatArx\s \s tamxmoVhoVtathxM GanxnitxVveYnamitiVkaSxNamf || \\
jinanitxVveYnamiti ca maqSeYvamaBimanayxteV \hfill || 1248 ||  
\end{shl}

\begin{artha}
A meVle (savxpanxvu heVLalapxDuvudu), yAva kAladalilx tananx ajAcnxnadiMda (edugoMDidudx) utapxnanxvAgidudx `Itananunx yAro kolulxtitxruvaMteyU hAgU Itananunxvasha paDisikoLuLxtitxruvaMteyU kANuvudo adeV savxpanx adu vAsatxvavAgilalx, suLALxgiyeV Itanu aBimAnisutAtxne.
\end{artha}

\vishaya{savxpanxdalUlx jAgaradaMte kolulxvudeV modalAdavu satayxvAgiruvaMte irali- eMdare-}

\begin{shl}
nAsatxyXtarx hanAtx jeVtA vA kiMtavxjAcnxnoVtathxmeVva tatf || \\
vAsanAviSaTxvijAcnxnoV hananAdayxBimanayxteV \hfill || 1249 ||  
\end{shl}

\begin{artha}
I savxpanxdalilx kolulxvanU jeYsuvanU yArU ilalx, adelalxvU ajAcnxnadiMda huTiTxkoMDadedx, Adare vAsanegaLiMda tuMbida manasusxLaLxvanu (`nAneV kolulxvenu, jeYsuveneMdu') hananAdi kirxyegaLanunx aBimAnisuvanu.
\end{artha}

\vishaya{``hasitxva'' itAyxdi vAkayxda athaR-(tAtapxyaR)}

\footnotetext[1]{Adi padadiMda avidAyx pakaSxvanunx tegedu koLaLxbeVku.}
\footnotetext[2]{gatARDi eMbuvalilx aginx jala muMtAdavanunx `patanAdi' eMbalilx BAvana (ODuvudu) itAyxdiyanunx garxhisabeVku ililx toVruva viSayAkAravAgi toVruva parxtayxkaSxvu dashaRnABAsa nijavalalx.}
\begin{shl}
adhamARdiparxka\footnotemark[1]SoVRtathx \footnotemark[2]gatARdipatanAdikamf \hfill || 1250 || \\
parxtiVcoV dashaRnaM savxpenxV mUDhaseyxVhoVpajAyateV ||  \\
\footnotemark[3]mithAyxtavxkAraNaM pArxha yadeVveVti shurxtiH savxyamf \hfill || 1251 ||  
\end{shl}
\footnotetext[3]{`yadeVva' itAyxdi shurxtiyeV avideyxyiMdaleV kANuvaneMdu sapxSaTxvAgi heVLiruvudariMda nAvu parxteyxVka savxpanx mithAyxtavxvanunx sAdhisabeVkAgilalx.}

%%%%%shloka footnote[1, 2, 3]
\begin{artha}
adhamaR modalAdavugaLu hecicxdAga adariMda haLaLx modalAda sathxLagaLalilx biVLuvudu modalAda kAyaRgaLu mUDhanAda parxtayxgAtamxnige (jiVvanige) parxtayxkaSxvAgi I savxpanxdalilx kANuvudu, idu mitheyx satayxvalalxveMbudakekx beVkAda kAraNavanunx `yadeVva jAgarxtf pashayxti tadatarx avidayxyAmanayxteV' eMba shurxtiyu savxyaM heVLutatxde.
\end{artha}

\vishaya{Iga A shurxtiya pada vAyxKAyxna}

\begin{shl}
jAgarxdivxSaya EvAyaM yadeYkiSxSaTx purA\s payxsatf || \\
tadatarx savxpenxV\s saMBAvayxM manayxteV\s vidayxyeYva saH || \\
nAvidAyx nApi tatAkxyaRM yasAmxdAtamxsamiVkaSxNeV \hfill || 1252 ||  
\end{shl}

\begin{artha}
jAgarAvasethxyalelxV I Atamxnu yAvudanunx hiMde noVDidadxno, adu asatayxveV adanenxV I savxpanxdalilx asaMBavavAdadadxnunx avanu avideyxyiMda kANuvanu adariMda adu asatayxveMdu (tiLiyabeVku) vasutxtaH Atamxnanunx cenAnxgi noVDidare (tatatxvXtaH sAkASxtAkxra) mADikoLuLxvalilx avideyxyU ilalx adara kAyaRvU ilalx.
\end{artha}

\begin{shl}
BayaM tevxVkAnatxtoV\s vidAyxkAyaRmAhuviRpashicxtaH ||  \\
yatoV vijAcnxtatatAtxvXnAM BiVtinARsitx kutashacxna \hfill || 1253 ||  
\end{shl}

\begin{artha}
eraDu avasethxgaLalUlx Aguva Bayavu niyatavAgi avideyxya kAyaRveMdu vidAvxMsaru heVLutAtxre, adakekx kAraNa? eMdare tatatxvXvanunx anuBava mADikoMDavarige yAvudariMdalU BayaveV iruvudilalx.
\end{artha}

\begin{shl}
EtadukatxM Bavatayxtarx pUvoVRpAtatxsayx kamaRNaH || \\
PalaM parxboVdheV yaduBxkatxM taceCxVSoV BAvanoVcayxteV \hfill || 1254 ||  
\end{shl}

\begin{artha}
(hiMde heVLida vAkayxdalilx tAtapxyaRvanunx I riVti heVLidaMte Aguvudu) ililx idu heVLidaMtAguvudu- adeVneMdare- hiMde mADida kamaRda Palavu ecacxrinalilx yAvudu anuBavisalapxTiTxto adara sheVSaveV vAsaneyeMdu heVLalapxDuvudu adeV savxpanxdalilx sapxSaTxvAgi kANuvudu.
\end{artha}

\vishaya{ecacxra kAladalilx ecacxrina BoVgavanunx koDuva kamaRvu anuBavisalapxTiTxruvudariMda adara vAsanegaLU savxpanx BoVgadiMda samAsatxvAgiruvudariMda punarutAthxna matutx adara anuSAThxnavAgalAradeMdu keVLidare-}

\footnotetext[1]{yadayxpi savxpanxdalilx puruSana BoVgakAladalilx vayxkatxvAda jAgaravAsaneyu Palavanunx koTuTx samApitxyAgide, AdarU punaH jAgaradalilx matotxMdu vAsaneyu hosakamaRvanunx mADisuvudu, avideyxyiruvAga punaH jiVvanu ELabahudu, EnU aDiDxyilalx, vAsanegaLalUlx kelavu savxpanx BoVgavanunx inunx kelavu jAgarxtitxna BoVgavanunx koDuvuvu jAgarxtitxna kamaRkUkx kAraNavAguvuvu adariMda anuSAThxnavu sididhxsuvudeMdu IriVti sidAdhxMtadalilx vAtiRkada meVlina shaMkAparihAravu.}
\begin{shl}
\footnotemark[1]yadi nAmAvasitAthAR PalaM datetxvXVha BAvanA || \\
puMBoVgasamayeV kamaR parxyuknetxV sA punanaRvamf \hfill || 1255 ||  
\end{shl}

%%%%%shloka footnote[1]
\begin{artha}
ililx vAsaneyu Palavanunx koTuTx tananx kAyaRvanunx mugisikoMDideyeMdu oMdu veVLe heVLidare puruSana BoVgakAladalilx samApatxvAgidadxrU punaH hosadAda kamaRvanunx mADisuvudu.
\end{artha}

\begin{shl}
utatxpXtitxBoVgayoVreVvaM BAvanA kamaRNaH sadA || \\
parxyoVkitxrXV BAvaneYvA\s \s tamxkataqRBoVkatxqqtavxyoVmaqRSA \hfill || 1256 ||  
\end{shl}

\begin{artha}
I riVtiyAgi kamaRvu huTuTxvudakUkx adara BoVgakUkx BAvaneyeV (vAsaneyeV) sadA parxvataRka, (parxyoVjaka), Atamxna kataqRtavx matutx BoVkatxqtavx eraDu viSayakUkx parxyoVjakavAdarU adu nijavAgi satayxvalalx.
\end{artha}

\vishaya{AtamxneV vAsanegaLige AsharxyaveMba matavanunx nirAkarisutAtxre-}

\begin{shl}
EvaM sati parxboVdheV yaduBxkatxM pArxkakxmaRNaH Palamf || \\
tasAyxnuvAsanA dhiVsAthx BAvaneVtayxBidhiVyateV \hfill || 1257 ||  
\end{shl}

\begin{artha}
hiVgiralu ecacxrinalilx hiMde yAva kamaRPalavu anuBavisalapxTiTxtoV, adara (Pala BoVgavAda)naMtara huTiTxda vAsaneyu aMtaHkaraNadalelxV iruvudu, adanenxV BAvaneyeMdu heVLuvudu.
\end{artha}

\vishaya{daqSATxMtadiMda aMtaHkaraNadalilxdeyeMbudanunx tiLisutAtxre}

\begin{shl}
AtamxceYtanayxbimebxVna bimibxtA dhiVyaRthA tathA || \\
kamoVRtathxPalabimebxVna bimibxteVyaM matiH sadA \hfill || 1258 ||  
\end{shl}

\begin{artha}
heVge AtamxceYtanayx biMbadiMda budidhxyu biMbisalapxTiTxruvudo hAgeyeV kamaRdiMda huTuTxva PalabiMbadiMda budidhxyu yAvAgalU biMbisalapxDuvudu.
\end{artha}
