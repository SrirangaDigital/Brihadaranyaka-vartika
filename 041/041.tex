\section*{baq. 4ne adhAyxya 3ne bArxhamxNa}

\section*{vAtiRka 1860 riMda 1975 pUNAR}

\vishaya{`sa Eka AjAna deVvAnAmf' eMbudara athaR -}

\begin{shl}
sagaRsayx jananAdAdw deVvatavxM yeV parxpeVdireV || \\
AjAnadeVvAsetxV\s tarx suyxH pUveVRBayxH sUkaSxmXmUtaRyaH \hfill || 1860 ||
  
\end{shl}

\begin{artha}
saqSiTxya AdiyalelxV huTiTx deVvatA rUpavanunx yAru hoMdidaroV, avareV
ililx AjAnadeVvareMdu heVLalapxDuvaru, ivaru hiMdinavarigiMta
sUkaSxmXshariVradavaru.
\end{artha}

\vishaya{I AjAna deVvarige AnaMdavu hecAcxgiruvudakekx kAraNaveVnu ?}

\begin{shl}
vAyxpiVni teVSAM sUkASxmXNi shariVrANi mahAtamxnAmf || \\
pUvARnanedxVBayx EteVSAmAnanodxV\s toV mahAnamxtaH \hfill || 1861 ||
  
\end{shl}

\begin{artha}
mahAtamxrAda I deVvategaLa shariVragaLu sUkaSxmXvAgiyU vAyxpakavAgiyU
ive, idariMda hiMdina AnaMdagaLigiMtalU ivara AnaMdavu hecicxnadeMbudu
saMmata.
\end{artha}

\begin{shl}
nAnAdavxMdovxVpaGAtAthaRheVtUnAM bahulatavxtaH || \\
AjAnadeVvAnanodxV\s taH pUveVRBayxH sAyxcaCxtAdhikaH \hfill || 1862 ||
  
\end{shl}

\begin{artha}
aneVka vidhavAda davxMdavxgaLu duHKa rUpavAda Palakekx kAraNavAgiratakakxvu bahaLavAgiruvudariMda idakekx hiMdidadx AnaMdagaLigiMta AjAna deVvategaLa AnaMdavu nUrakUkx hecAcxgiruvudu.
\end{artha}

\vishaya{`yashacxshoVrxtirxyaH' eMbalilxruva shorxVtirxya padAthaR -}

\begin{shl}
shorxVtirxyoV\s dhiVtaveVdaH sAyxjAjxcnXtaveVdAthaR Eva yaH || \\
kaqtasxnXcoVditakAritAvxtasxvaRpApavivajiRtaH \hfill || 1863 ||
  
\end{shl}

\begin{artha}
veVdAdhayxyana mADidavanU veVdAthaRvanunx tiLidavanU yAroV avaneV
shorxVtirxya, vihitavAda kAlakamaRvanunx mADuvudariMda (avaqjina)
aMdare samasatx pApagaLiMdalU vajiRsalapxTaTxvanu (shUnayxnu).
\end{artha}

\vishaya{`akAmahataH' eMdare -}

\begin{shl}
AjAnadeVvAvadhikakAmAnupahatAshayaH || \\
tataH pareVSu BoVgeVSu jAtataqSaNxshacx yaH pumAnf \hfill || 1864 ||
  
\end{shl}

\begin{artha}
AjAnadeVvara payaRMtaraviruva hiMdina AnaMdagaLalilx kAmaneyiMda
keDada manasusxLaLxvanU, alilxMdAcege iruva BoVgagaLalilx
(utapxnanxvAda AseyuLaLxvanU Agi) yAva puruSanu iruvano, avanu
(akAmahatanu).
\end{artha}

\begin{shl}
AjAnadeVvAnanedxVna samaM tasayx ca yatusxKamf ||  \\
asAyxkAmahatatevxVna suKoVtakxqqSiTxrihoVcayxteV \hfill || 1865 ||
  
\end{shl}

\begin{artha}
AjAna deVvana AnaMdadoMdige samAnavAda yAva suKavideyo adu ivanu
kAmahatanalalxvAdadxriMda suKada utakxSaRveV ililx heVLalapxDuvudu.
\end{artha}

\begin{shl}
shorxVtirxyAvaqjinatevxV devxV tuleyxV EvA\s \s viricnicxtaH || \\
akAmahatatAheVtoVvaqRdwdhx hAlxdoV vivadhaRteV \hfill || 1866 ||
  
\end{shl}

\begin{artha}
shorxVtirxyatavx avaqjinatavxgaLu eraDU barxhamxna payaRMtara
samAnavAgiye ive, I akAmahatatavxveMbudu hecicxdalilx AnaMdavU
hecucxvudu.
\end{artha}

\begin{shl}
pUvaRpUvoVRpaBoVgeVBoyxV yAvadAyxvaninxvataRteV || \\
kAmoV, vivadhaRteV puMsasAtxvatAtxvatusxKaM haqdi \hfill || 1867 ||
  
\end{shl}

\begin{artha}
kAmavu hiMduhiMdina upaBoVgagaLiMda eSeTxSuTx hiMtiruguvudo,
himemxTuTxvudo, manuSayxnige aSaTxSuTx suKavu haqdayadalilx
hecucxvudu.
\end{artha}

\begin{shl}
kAmeYkakaSxyatanetxrXYva yasAmxtupxMsaH suKoVnanxtiH || \\
akAmahatateYvAtaH pUvARBAyxM sAdhanaM paramf \hfill || 1868 ||
  
\end{shl}

\begin{artha}
kAmavoMdara nAshadiMda mAnavana suKavu hecucxvudu,
akAmahatatavxveMbudu hiMdina eraDu sAdhanagaLigiMta utatxma sAdhanavu 
\end{artha}

\vishaya{jAcnxna kamaR samucacxyavu vivakiSxtaveMbudara KaMDane -}

\begin{shl}
sAdhanatavxM samAnaM sAyxtarxrXyANAmiha yadayxpi || \\
kAmakaSxyaparxkaSoVR\s tarx huyxtakxqqSATxhAlxdasAdhanamf \hfill || 1869 ||
  
\end{shl}

\begin{artha}
yadayxpi mUrakUkx sAdhanatavxvu ililx samAnavAgide, idaralilx kAmavu
hecucx kaSxyisiruvudeV utakxqqSaTx AnaMdakekx sAdhana.
\end{artha}

\begin{shl}
samucacxyavivakASx\s tarx na manAgapi yujayxteV || \\
kataqRtAvxdisamuceCxVdiMjAcnxnaseyxVhA\s \s shirxtatavxtaH \hfill || 1870 ||
  
\end{shl}

\begin{artha}
jAcnxna\footnote[1]{kelavaru akAmahataveMba shabadxkekx
barxhamxjAcnxnaveMdU shorxVtirxya avaqjinaveMba shabadxgaLige
kamaRveMdU athaR vivakiSxtaveMdu BAvisi I mUrU iruvavanige
paramAnaMda lABaveMdu shurxtiya tAtapxyaRvanunx heVLuvaru, adu
sariyalalxveMdu vAtiRkakAraru roVrisi koTiTxdAdxre.} kamaRgaLa samucacxyavanunx heVLuva udedxVshavu savxlapxvU
yukatxvAgilalx, kAraNaveVneMdare ? \footnote[2]{koneya payARyadalilx
tatavxjAcnxnavanunx kataqRtAvxdigaLanunx bAdhisuva riVriyalilx
heVLiruvudariMda kamaRsamucacxyavu ililx vivakiSxtavalalx.}kataqRtavx
modalAdavugaLanunx nAshagoLisuva jAcnxnavanenxV ililx Asharxyiside.
\end{artha}

\begin{shl}
samucacxyanivaqtatxyXthaRM nAyxyashAcxpi puroVditaH || \\
nAtaH samucacxyAsheVha kataRvAyx sUkaSxmXdashiRBiH \hfill || 1871 ||
  
\end{shl}

\begin{artha}
I samucacxyavanunx nirAkarisalu nAyxyavanUnx hiMdeyeV
(saMbaMdhagarxMthadalilx) heVLidAdxyitu. adariMda ililx sUkaSxmX
vicArakaru samucacxya Aseyanunx mADabAradu.
\end{artha}

\begin{shl}
sAdhanatavxM yatasutxlayxM shorxVtirxyAvaqjinatavxyoVH || \\
avareVSavxpi BoVgeVSu na coVtatxmasuKaM tataH \hfill || 1872 ||
  
\end{shl}

\begin{artha}
matutx shorxVtirxyatavx-avaqcinatavxgaLige sAdhanatavxvu samAnavAgide
nikaqSaTxvAda BoVgagaLa viSayakukx samAna, adariMda utatxma suKavu
baruvudU ilalx.
\end{artha}

\begin{shl}
akAmahatateYvAtarx tAratamAyxtamxkatavxtaH || \\
BeVdAdutatxmaswKayxsayx sAdhanaM coVtatxmA BaveVtf \hfill || 1873 ||
  
\end{shl}

\begin{artha}
kAmada hoDetavilalxdiruvudeV taratamaBAvarUpadalilxruvudariMda
BinenxYsi utatxma swKayxkekx utatxmasAdhanavAguvudu.
\end{artha}

\begin{shl}
yuvA sAdhuyuveVteyxVvaM teYtitxriVyashurxtiVraNAtf || \\
adhareVSavxpi vAkeyxVSu shorxVtirxyAdi vivakiSxtamf \hfill || 1874 ||
  
\end{shl}

\begin{artha}
\footnote[1]{`sAdhuyuvA' eMbudariMda pApavilalxda yuvakaneMdU
`adhAyxyakaH' eMbudariMda shorxVtirxyaneMdU heVLidadxriMda
avaqjinatavx matutx shorxVtirxyatavxveraDU hiMdina payARyagaLalUlx
teYtatxriVyadalilx `shorxVtirxyasayxcAkAmahatasayx' eMdu heVLalapxTiTxve, adariMda yAva
payARyagaLalilx eraDanunx sapxSaTxvAgi heVLalilalxvo alilxyU
iTuTxkoLaLxbeVkeMdu tAtapxyaR.}`yuvAsAyxtf sAdhuyuvAdhAyxyakaH' ideV bageyAgi teYtatxriVya shurxtiyu
heVLuvudariMda keLagina vakayxgaLalilx shorxVtirxya muMtAdavu vivakiSxtavu.
\end{artha}

\vishaya{``sa EkaH parxjApati loVkaH na EtoV barxhamxloVka AnaMdaH'' - eMbudaravAyxKAyxna}

\begin{shl}
virATapxrXjApatijecnxVRyaserxYloVkAyxtamxkadeVhaBaqtf || \\
hiraNayxgaBoVR barxhAmx\s tarx tathA barxhAmxgiroVcayxteV \hfill || 1875 ||
  
\end{shl}

\begin{artha}
mUru loVka savxrUpavAda sUthxla shariVravanunx hoMdiruva virATa
puruSanu parxjApatiyeMdu tiLiyabeVku, hiraNayx gaBaRneMba barxhamxneV
ililx barxhamx padadiMda (gArxhayxvAgi) heVLalapxTiTxde.
\end{artha}

\vishaya{atheYSa itAyxdivAkayxvanunx vAyxKAyxnakekx tegedukoLuLxvaru -}

\begin{shl}
ataH paramananatxtAvxdagxNitaM vinivataRteV || \\
yata EvamataH pArxhAtheYSa EveVti naH shurxtiH \hfill || 1876 ||
  
\end{shl}

\begin{artha}
ililxMda Acege suKagaLige koneyilalxvAdadxriMda gaNitavu niMtide. I
kAraNadiMdale shurxtiya ``atheYSa Eva parama AnaMda ESabarxhamxloVka
sasxmArxTf'' eMdu namage heVLiruvudu.
\end{artha}

\begin{artha}
idanunx heVge opupxvudu ? eMdu keVLidare ajacnxrigAgi parxmANavanunx heVLutAtxre -
\end{artha}

\begin{shl}
aneVnAtishayavatA hayxsamxdogxVcaravatiRnA || \\
apAsAtxtishayAnanadxM suKeVneVhAnumiVyateV \hfill || 1877 ||
  
\end{shl}

\begin{shl}
\footnotemark[1]dhavxsAtxtishayaniSaThxtAvxlolxVkeV sAtishayAtamxnaH ||  \\
yatoV\s toV\s tishayavatA gamayxteV\s natishayaM suKamf \hfill || 1878 ||
  
\end{shl}
\footnotetext[1]{`suKoVtakxSaRtAratamayxM kavxcidivxshArxnatxmf, taratamatAvxtf parimANa tAratamayxvatf, yadayxtf taratamatavx, tatatxtf kavxcitf (niratishayeV) vishArxnatxmf' eMbudeV anumAna parxmANa athavA
`sAtishayAnanadxH niratishayAnanedxVvishArxnatxH sAtishayAnanadxtAvxtf yoVyaH sAtishayaH sasaH niratishayeV vishAvxnatxH yadhA mahatAvxdivatf paqthivAyxdiH' eMdu anumAna parxmANa idanunx ariyabeVku, yAvudeV oMdu
vasutxvinalilx tAratamayxvu hecucx kaDimeyu kANisuvudo, adakekx
miVrida hecucx kaDimeyilalxda matotxMdu vasutxvu iraleVbeVku. udAH
mahatavxveMba parimANavu parxpaMcadalilx paqthivAyxdi
vasutxgaLalilx kANutitxruvudu hecucx kaDimeyiMda ide, paqthiviyu
doDaDxdu vishAlavAdadudx matutx vAyxpaka, adakikxMta
vAyxpakavAdadudx jalatatavx, adakikxMta aginxtatavx doDaDxdu
vAyxpaka, idakUkx vAyu doDaDxdu adakUkx vayxpaka niratishaya
mahatavxviruvudu AkAsha, EkeMdare idaralelxV vAyu, aginx, jala,
paqthivi elAlx barxhAmxMDavU iruvudu avakAshavilalxdidadxre ivu
yAvudU nilalxlu sAdhayxvilalx, avakAshavanunx koDuvudu AkAshaveV
du savARdhAravAda savaRvAyxpaka, I vAyxpakatavxvu koneyalilx
AkAshadalelxV nilulxvudu, iMtaha niratishaya vAyxpakatavx
parimANaviruva vasutx AkAshavideyeMdu takiRsuvaMte. AnaMdada
tAratamayxdiMda elalxdakUkx migilAda hiraNayxgaBARnaMda
payaRMtaviruva AnaMdadiMda aLatege sikakxda niratishayavAda AnaMda
vasutxvideyeMdu takiRsidare ajacnxrige  adara asitxtavxvu
toVrabahudu.}

%%%% footnote shloka [1] {}
\begin{artha}
atishayavilalxda (tAratamayxvilalxda)
vasutxvinalilxruvudariMda atishayadiMda kUDida vasutxvu
atishayavilalxda (niratishaya) vasutxvinalilx
parisamApitxgoLuLxvudariMda atishayaviruva (suKadiMda) niratishayavAda
suKavu ideyeMdu toVruvudu (UhisalapxDuvudu).
\end{artha}

\begin{shl}
yaterxYtAni samasAtxni niSAThxM yAnitx parAtamxni || \\
paramoV\s sAvihA\s \s nanadxH savARnanAdxtilaknaGxnAtf \hfill || 1879 ||
  
\end{shl}

\begin{artha}
yAva paramAtamxnalilx I elAlx (lwkika suKagaLU) samApitxgoLuLxvavo
(nele nilulxvavoV) adeV ililx paramAnaMdaveMdu heVLalapxDuvudu,
EkeMdare ? adu samasatx AnaMdagaLanunx miVriruvudu.
\end{artha}

\begin{shl}
ESoV\s seyxVti parxtijAcnxta AnanadxH soV\s numAnataH || \\
niSAThxM parxtiVcigamita utatxroVtatxravaqdidhxtaH \hfill || 1880 ||
  
\end{shl}

\begin{artha}
`ESoV\s sayx paramAnanadx' eMdu parxtijecnx mADida AnaMdavanunx takaRdiMda
parxtayxgAtamxnalilxruvudeMdu utatxroVtatxra mahatavxdiMda tiLipaDiside.
\end{artha}

\vishaya{tatatxvXjAcnxnigaLanunx kuritu anuBavavanenxV udAharisutAtxre}

\begin{shl}
avijAcnxtaparAnanAdxnapxrXteyxVSA\s numitimaRtA ||  \\
sAkASxjAjxcnXtAtamxtatAtxvXnAM parxtayxkaSxtamameVva tatf \hfill || 1881 ||
  
\end{shl}

\begin{artha}
parAnaMdavanunx tiLiyada ajacnxranunx kuritu I anumAna parxmANavanunx
toVriside, Atamxtatavxvanunx tiLidavarige neVra sAvxnuBavakekx parxtayxkaSxvAgide.
\end{artha}

\vishaya{AtAmxnaMdada mahatavx paramatavx sAdhane -}

\begin{shl}
akAmahatadhiVgamayx AnanadxH parxtayxgAtamxni || \\
yaH sa Eva yathoVketxVBayxH paramaH sAyxdananatxtaH \hfill || 1882 ||
  
\end{shl}

\begin{artha}
akAmahatana budidhxge toVruva parxtayxgAtamxnalilxruva AnaMdavu yAvudu
ideyo, adeV hiMde heVLida AnaMdagaLigiMtalU vAyxpakavAdudariMda
paramavAgiruvudu.
\end{artha}

\vishaya{veVdavAyxsara saMmatiyU ide -}

\begin{shl}
tathAca BagavAnAvxyXsaH savaRveVdAthaRtatatxvXvitf || \\
savxyaM pArxheVmameVvAthaRM kAmAnathaRjihAsayA \hfill || 1883 ||
  
\end{shl}

\begin{shl}
yacacx kAmasuKaM loVkeV yacacx divayxM mahatusxKamf || \\
taqSANxkaSxyasuKaseyxYteV nAhaRtaH SoVDashiVM kalAmf \hfill || 1884 ||
  
\end{shl}

\begin{shl}
yatoV yatoV nivataRteV tatasatxtoV vimucayxteV ||  \\
nivataRnAdidhx savaRtoV na veVtitx duHKamaNavxpi \hfill || 1885 ||
  
\end{shl}

\begin{artha}
hAgeye BagavAnf veVdavAyxsaru samasatx veVdAthaRtatavxvanunx
tiLidavaru tAveV heVLiruvaru, BoVgadiMda uMTAguva anathaRvanunx
pariharisaliceCxyiMda ideV viSayavanenxV heVLiruvaru ``adeVneMdare
loVkadalilx yAva kAmasuKaviruvudo, matutx deVvaloVkada hecicxna
suKaviruvudo, ivugaLu Aseyu kaSxyisuvudariMda baruva suKada hadinAraneV
oMdu BAgavanunx hoMdalAravu'' yAva yAva (lwkika suKadiMda) hiMdakekx
tiruguvano AyAya viSayadiMda Itanu vimukatxnAguvanu, I savaR
nivaqtitxyiMda athavA savaRBoVgavasutxviniMdalU nivaqtitx
hoMduvudariMda oMdu aNuvaSuTx duHKavanunx Itanu kANuvudilalx.
\end{artha}

\vishaya{`ESa Eva parama Ananadx ESa barxhamxloVkaH' eMbudara tAtapxyaR -}

\begin{shl}
salilAdigirA yoV\s thaRH parxtayxjAcnxyi purA\s KilaH || \\
barxhamxloVkAnatxvAkeyxVna tasayx sAyxdupasaMhaqtiH \hfill || 1886 ||
  
\end{shl}

\begin{artha}
`salila EkoVdarxSATx\s devxYtaH' eMbuva salilAdi vAkayxdiMda yAva athaRvanunx hiMde pUNaRvAgi
heVLalapxTiTxdidxto, adanunx barxhamxloVka padavu madheyx baruva
``ESa barxhamxloVka'' eMba vAkayxdiMda upasaMhAra mADide\footnote[1]{`salila EkoVdarxSATx'
eMdu upakarxmisi-- `mAtArx mupajiVvanitx' eMbuva payaRMtaviruva vAkayxdiMda
kAyaR kAraNagaLiMda biDalapxTaTx niratishayavADa AnaMda
savxrUpavuLaLx yAva parxtayxgAtamxnanunx
savaRvAyxpakanAgiruvavanunx parxtijecnx mADiheVLiruvudo, adanenxV
`sayoV manuSAyxNAmf' ililxMda AraMBisi, `atheYSa Eva' eMbuva payaRMtaraviruva
vAkayxdiMda parxtipAdisi, `ESa' itAyxdi vAkayxdalilx Iga upasaMhAra mADideyeMdathaR.}.
\end{artha}

\vishaya{`soV\s haM BagavateV sahasarxMdadAmi ata UdhavxRM vimoVkASxyeYva bUrxhi' eMbudakekx athaR -}

\footnotetext[2]{janakamahArAjana parxshenxyu mukitxPalavuLaLx
tatatxvXjAcnxnakAkxgi baMdididxtu, I viSayavanunx nidhaRrisidadxrU,
nidhaRrisuvudakekx modalu parxshinxsidaMte IgalU rAjanu mukitxge
muKayxsAdhanavanunx heVLi eMdu keVLiruvanu.}
\begin{shl}
\footnotemark[2]parxshAnxtheVR\s simxnasxmApetxV\s pi pUvaRvatatxM muniM naqpaH || \\
anavxyuknAkxta UdhavxRM tavxM mukatxyeV bUrxhi yatapxramf \hfill || 1887 ||
  
\end{shl}

%%%% footnote shloka {2}
\begin{artha}
(rAjanu) parxshenxya viSayavu idu mugididadxrU hiMdinaMte
A muniyanunx kuritu parxshinxsidanu, EneMdare inunx meVle niVvu
mukitxgebeVkAda muKayx sAdhanavanunx heVLi eMdu.
\end{artha}

\begin{shl}
yAjacnxvalokxyXV\s pi rAjecnxYvaM paqSaTxH sanUpxvaRvatatxdA || \\
aviBeVdanayxtoV heVtoVnaR tavxsAmathayxRkAraNAtf \hfill || 1888 ||
  
\end{shl}

\begin{artha}
yAjacnxvalakxyXrU saha rAjaniMda I riVtiyAgi parxshinxsalapxTaTxvarAgi
AvAga hiMdinaMte beVre kAraNadiMda\footnote[3]{vakatxqqtavx
sAmathayxRvilalxdiruvudAgali, utatxravu tiLidilalxveMbudAgali muniyu
hedaruvudakekx kAraNavalalx, beVre oMdu kAraNavideyeMdathaR.} hedarikoMDaru, Adare
(utatxrisalu) sAmathayxRvilalxveMba kAraNadiMdalalx.

EkeMdare-\footnote[4]{muniyu barxhimxVBUtanAdadxriMda savaRjacnxnU
savaRshakatxnU Agiruvanu, adariMda ajAcnxnavAgali asAmathaRyxvAgali
idu I Bayavu baMdiruvudilalx, meVdhAviVrAjA itAyxdiyAgi muMde
heVLuva kAraNavu beVreyide.}
\end{artha}

\begin{shl}
savaRjacnxtAvxnumxneVnARBUtapxrXshAnxthARjAcnxnatoV Bayamf ||  \\
kAraNaM tavxnayxdeVvAtoV yataH shurxtiraBASata \hfill || 1889 ||
  
\end{shl}

\begin{artha}
muniyu savaRjacnxnAdudariMda parxshAnxthaRvu tiLiyadiruvudariMda
baruva Bayavu Atanige iralilalx, adariMda shurxtiyu beVreye Ada
kAraNavanunx heVLiruvudu.
\end{artha}

\section*{vAtiRka}

\begin{shl}
avivakuSxmayaM rAjA kAmaparxshanxbalAsharxyAtf || \\
kiMjoyxVtiriteyxVvamAdimapArxkiSxVnAmxM punaH punaH \hfill || 1890 ||
  
\end{shl}

\begin{shl}
aparxtAyxKeyxVyoV hayxthaRshacx satayxsAyxvashayxrakaSxNAtf || \\
savxyaMjoyxVtiSaTxvXniNiVRtiH kaqtA\s toV\s navasheVSataH \hfill || 1891 ||
  
\end{shl}

\begin{shl}
niNiVRteV\s payxtha mAM rAjA punaH punarapaqcaCxta || \\
ata UdhavxRmiti girA niruNadedhxyXVva mAM naqpaH \hfill || 1892 ||
  
\end{shl}

\begin{artha}
I rAjanu heVLalu iciCxsida nananxnunx kuritu iSaTxvAdadadxnunx
keVLabahudeMdu (modalukoTaTxvarada) balavanunx Asharxyisi `kiMjoyxVti'
itAyxdiyAgi padeV padeV parxshinxsidadxnu.
\end{artha}

\begin{artha}
satayxvanunx avashayxkApADabeVkAdadadxriMda I viSayavanunx
tirasakxrisuva hAgiralilalx, adakAkxgi savxyaMjoyxVti
savxrUpaveMbudara niNaRyavanunx savxlapxvU uLiyadaMte
mADikoTiTxdAdxyitu, niNaRyamADidadxrU rAjanu padeV padeV heVLidanu,
adariMda ``ata UdhavxRM vimoVkASxyeYva bUrxhi'' eMba mAtiniMda nananxnunx rAjanu nibaRMdhisiruvanu.
\end{artha}

\begin{shl}
kAmaparxshAnxknukxsheVneYva mAM vashiVkaqtayx madagxtamf || \\
samAditasxti niHsheVSaM jAcnxnaM rAjA\s tipaNiDxtaH \hfill || 1893 ||
  
\end{shl}

\begin{artha}
rAjanu iSaTx parxshenxya aMkushadiMdaleV nananxnunx vashapaDisikoMDu
nananxlilxruva samasatx jAcnxnavanunx sivxVkarisalu iciCxsuvanu
adariMda ivanu doDaDxpaMDitanu.
\end{artha}

\begin{shl}
iteyxVSa BayaheVtuH sAyxdAyxjacnxvalakxyXsayx nAnayxtaH || \\
BayaheVtoVravidAyxyAH savaRjacnxtAvxdasaMBavAtf \hfill || 1894 ||
  
\end{shl}

\begin{artha}
ideV yAjacnxvalayxrige Bayakekx kAraNavAgide, beVre kAraNadiMdalalx,
savaRjacnxnAdadadxriMda BayakekxkAraNavAda ajAcnxnavu Itanalilx
saMBavisuvudilalx.
\end{artha}

\begin{shl}
asakaqninxNaRyoV\s kAri paqSeTxV vasutxnayxsheVSataH || \\
arwtisxVnAmxM tathA\s peyxVSa savaRsAvxditasxyA naqpaH \hfill || 1895 ||
  
\end{shl}

\begin{artha}
vasutx viSayavanunx parxshinxsidAga elAlx niSakxSeRyanunx aneVkasala
mADidAdxyitu, AdarU rAjanu samasatxvanunx sivxVkarisalu
iciCxsidadxriMda nananxnunx nibaRMdhisiruvanu.
\end{artha}

\begin{shl}
meVdhAviV paNiDxtoV\s toV\s yaM barxhamxsAvxdAnakAraNAtf || \\
na biBeVti yatasatxsAmxdeBxVtavayxM janakAdaBxqqshamf \hfill || 1896 ||
  
\end{shl}

\begin{artha}
I mahArAjanu paMDitanu. barxhAmxsAvxdaneyakAraNadiMda hedaruvudilalx,
adariMda janakaniMdaleV hecAcxgi hedarabeVku.
\end{artha}

\vishaya{`savA ESa' itAyxdi vAkayxda meVle AkeSxVpa -}

\begin{shl}
nanu parxshAnx yathoVkAtxshecxVninxNiVRtAthARH puroVkitxBiH || \\
aniNiVRtaM kimudidhxshayx naqpoV\s parxkiSxVnumxniM punaH \hfill || 1897 ||
  
\end{shl} 

\begin{artha}
(Aditayx joVyxti modalAdavugaLanunx) heVLida hiMdina vacanagaLiMda I
parxshenxgaLu niNaRyasalapxTaTx viSayagaLuLaLxvugaLeMdu Agidadxre
rAjanu niNaRyavAgadiruva yAva viSayavanunx udedxVshisi muniyanunx
kuritu parxshinxsidanu ?
\end{artha}

\vishaya{samAdhAna -}

\begin{shl}
savxpanxbudAdhxnatxsaMcAra AtamxnoV yaH puroVditaH || \\
daqSATxnatxtevxVna rAjAcnx\s sw savoVR\s piVha vivakiSxtaH \hfill || 1898 ||
  
\end{shl}

\begin{artha}
Atamxna savxpanx jAgarAvasethxgaLalilxna saMcAravanunx modalu
yAvudanunx heVLididxto, adelalxvU rAjaniMda daqSATxMtavAgiye udedxVsha
mADalapxTiTxtu.
\end{artha}

\vishaya{AdarU parxshenxge avakAshavelilx ? eMdare -}

\begin{shl}
tasayx dASATxRnitxkoV yoV\s thoVR yAvatAsxkASxnanx kathayxteV || \\
mumukaSxti na tAvatatxM rAjA parxshAnxthaRsheVSataH \hfill || 1899 ||
  
\end{shl}

\begin{artha}
adakekx yAva viSayavanunxdASATxRMtikavAgi elilxyavarege neVra
heVLuvudilalxvo, aduvaregU rAjanu parxshanx viSayavu uLidiruvudariMda
biDugaDehoMdalu apeVkiSxsuvudilalx.
\end{artha}

\begin{shl}
daqSATxnatxsayx suSupetxVshacx nAthaRM dASATxRnitxkaM jagw || \\
barxhAmxsimxVtAyxgamAdobxVdhaH suSupotxVdAhaqteVmaRtaH || \\
dASATxRnitxkoV\s thaRH pArxjacnxsayx daqSATxnatxsAyxvasheVSataH \hfill || 1900 ||
  
\end{shl}

\begin{artha}
matutx suSupitxdaqSATxMtakekx beVkAda dASATxRMtikada viSayavanunx
heVLalilalx, suSupitxya udAharaNege beVkAda dASATxRMtikada athaR
yAvudeMdare `barxhAmxsimx' eMbuva AgamadiMda baruva jAcnxnaveV, I
riVtiyAgi pArxjacnx daqSATxMtakekx beVkAda viSayavu uLidide.
\end{artha}

\begin{shl}
uketxV dASATxRnitxkeV\s theVR\s simxnapxrXshAnxthaRsayx samApitxtaH || \\
savaRmukatxM BaveVdayxsAmxdataH soV\s thoVR\s dhunoVcayxteV \hfill || 1901 ||
  
\end{shl}

\begin{artha}
IdASATxRMtikada viSayavu heVLalapxDalu parxshAnxthaRvu
mugidiruvudariMda elalxvanUnx heVLida hAgAyitu, adariMda A
viSayavanenxV IvAga heVLutetxVve.
\end{artha}

\begin{shl}
asaknogxV matasxyXvatapxrXtayxknAkxmakamARdiBiH karxmAtf || \\
sameVti savxpanxbudAdhxnwtx yathA\s yaM savxvashasatxthA \hfill || 1902 ||
  
\end{shl}

\begin{shl}
saheVturasayx saMsAraH paraloVkeVhaloVkayoVH || \\
savisatxraH sa vakatxvayx itayxtheVRyaM parA shurxtiH \hfill || 1903 ||
  
\end{shl}

\begin{artha}
doDaDxmiVninaMte\footnote[1]{heVge doDaDx miVnu nadiya eraDu
tiVragaLigU ililxMda alilxge alilxMdilalxge karxmavAgi saMcarisuvudo
hAgeye I Atamxnu tatavxtaH kAmakamARdigaLa saMbaMdhavilalxdavanAdarU
jAgara matutx savxpAnxvasethxgaLanunx hoMdidAga savxtaMtarxnAgiyeV
ODADuvanu, savxtaMtarxnAgi eraDu loVkagaLige saMcarisuvudeV baMdhavu
idanunx idara kAraNavanunx veYrAgayxvu sididhxsalu avashayxheVLabeVkeMdu udedxVshisi I saMsAra parxkaraNavu baMdide.} asaMganAdarU parxtayxgAtamxnu kAma
kamARdigaLiMda karxmavAgi savxpAnxvasethxyanunx jAgarAvasethxyanUnx
hoMdiruvanu, heVge Itanu savxtaMtarxno hAgeye biVja sahitavAda
saMsAravanunx paraloVka-ihaloVkagaLalilx visAtxravAgiruvaMte ivanige
heVLatakakxdAdxgide idakAkxgi muMdina I shurxtiyu baMdiruvudu.
\end{artha}

\vishaya{saMkiSxpatxvAda athaR}

\begin{shl}
banodhxV banadhxnaheVtushacx moVkaSxsatxdedhxVtureVva ca || \\
savisatxraH parxvakatxvayxsatxdukatxyXthAR parA shurxtiH \hfill || 1904 ||
  
\end{shl}

\begin{artha}
saMsAra, saMsArada kAraNa, moVkaSx matutx adara kAraNa ivugaLanenxV
visAtxravAgi heVLabeVkAgide, adanunx heVLuvudakAkxgi muMdina shurxtiyu
baMdide.
\end{artha}

\section*{baq. 4ne a. 3 ne bArxhamxNa. kaMDike 34}

\begin{shl}
sa vA ESa EtasimxnasxvXpAnxnetxV ratAvx caritAvx daqSeTxvXYva puNayxM ca pApaM ca punaH parxtinAyxyaM parxtiyoVnAyxdarxvati budAdhxnAtxyeYva || 34 ||
\end{shl}

\vishaya{itara saMdaBaRvanunx sUcisi `savAESa' itAyxdi shurxtiya tAtapxyaRvanunx heVLalu horaTidAdxre -}

\footnotetext[1]{`budAdhxnetxV ratAvx' itAyxdi vAkayxdiMda hiMde
jAgarAvasethxyiMda Atamxnu `savxpAnxnAtxyeYva' eMdu
savxpAnxvasethxyoLage parxveVshisidaneMdu boVdhiside, Iga adeV
savxpAnxvasethxyiMda jAgarAvasethxge Atamxnanunx oyiyxdaMte
toVrisabeVkAgide, EtakekxMdare ? saMsArada karxmavanunx
tiLisuvudakAkxgi, keVvala savxpAnxvasethxyalelx dASATxRMtikavAda
saMsAra baMdhavanunx sapxSaTxvAgi tiLisuvudakekx sAdhayxvAguvudilalx.
adariMda savxpanxdiMda jAgarAvasethxge baMdaneMdu heVLalu muMdina
vAkayxvu baMdideyeMdathaR.}
\begin{shl}
\footnotemark[1]jAgarxtAsxthXnAtatxtaH pUvaRM savxpanxmAtAmx parxveVshitaH || \\
jAgarxdUBxmiM sa neVtavoyxV dASATxRnitxkavivakaSxyA \hfill || 1905 ||
  
\end{shl}

%%%% footnote shloka {1}
\begin{artha}
jAgarxtitxna sAthxnadiMda adakekx modalu savxpanxvanunx
Atamxnu parxveVshisidadxnu. avananunx punaH jAgarxta sAthxnakekx
oyayxbeVku, dASATxMtikavAda (saMsAra baMdhada) udedxVshayxvAgi.
\end{artha}

\vishaya{matotxMdu tAtapxyaR}

\footnotetext[2]{yAvAga `savxpAnxnAtxyeYva'
eMbalilx suSupitxyavacanavu hAgU mahAmatasxyX daqSATxMtaviruva
vAkayxdiMda AraMBisi `ESa Eva paramaAnanadxH' eMbuva tanaka iruva
vAkayxvU baMdideyeMdu iTuTxkoLuLxviro, adariMda parxtijecnx mADida
viSayavanunx udedxVshisi mugiyisuvudakAkxgi muMdina vAkayxveMdU
heVLuvude udedxVshavAgideyenanxbeVku, AvAgalU suSupitxyiMda
jAgarxdadxshege baMdaneMdu Atamxnige heVLabeVkAdadudx avashayxvide.
anayxthA Adare saMsArada karxmavanunx toVrisuvudakekx Aguvudilalx ||}
\begin{shl}
\footnotemark[2]AnanadxniNaRyAnatxM tu saMparxsAdavacoV yadA || \\
tadA nigamanAthaRM tatapxrXtijAcnxtAthaRgoVcaramf \hfill || 1906 ||
  
\end{shl}

%%%%% footnote shloka {1, 2}
\begin{artha}
suSupitxya vAkayxvu AnaMda savxrUpa
niNaRya mADuva payaRMtara iruvudeMdu yAvAga idu heVLalapxDuvado, AvAga
parxtijAcnxtavAda vasutx viSayadalilx mugisuvudakAkxgi baMdide.
\end{artha}

\section*{vAtiRka}

\vishaya{parxtijecnxmADida viSayaveVnu ? -}

\begin{shl}
alupatxdaqSiTxrAtAmx\s yaM yathoVkatxM savxpanxboVdhayoVH || \\
pArxjecnxV\s pi ca tatheYvAyaM yadevxY taditivAkayxtaH \hfill || 1907 ||
  
\end{shl}

\begin{artha}
I \footnote[1]{ililx AnaMda girigaLu heVge eraDu sAthxnagaLalilx
parxtayxgAtamxnU kUDa naSaTxvAda jAcnxnavuLaLxvanAgiruvano, hAgeye
suSupitxyalUlx naSaTxvAda jAcnxnavuLaLxvaneMdu heVLabeVkalalxve,
eMdu Atamxna jAcnxna daqSiTxyanunx pariVkiSxsuvAga idanunx
heVLideyeMdathaRveMdu vAyxKAyxnisiruvaru.}Atamxnu loVpavAgadiruva jAcnxnavuLaLxvaneMdu heVge
savxpanx jAgara sAthxnagaLalilxruvano hAge suSupitxyalUlx iruvaneMdu
`yadevxYtananxpashayxti pashayxnf veYtananxpashayxtiV' itAyxdi vAkayxgaLiMda tiLisalapxTiTxvanu.
\end{artha}

\vishaya{parxtijecnx mADida matotxMdathaRvanunx heVLutAtxre -}

\begin{shl}
atikArakaheVtushacx yathA\s \s tAmx\s yaM suSupatxgaH || \\
kUTasathxdaqSiTxmAtarxtAvxtatxthA savxpanxparxboVdhayoVH \hfill || 1908 ||
  
\end{shl}

\begin{artha}
heVge Atamxnu suSupitxyalilx (parxmAtaq muMtAda devxYtavilalxde)
iruvano, matutx kAraka matutx kAraNagaLanunx miVridavanAgiruvano,
hAgeye niviRkAravAda (nitayx) daqSiTx mAtarxvuLaLxvanAdadxriMda
savxpanx jAgaradashegaLalUlx iruvanu.
\end{artha}

\footnotetext[1]{Atamxna nitayx daqSiTxyU parxmAtaq parxmANAdi devxYta
shUnayxteyU I eraDU ililx savxpAnxnatx padadiMda mugisalapxDuvudu,
savxpanx shabadxdiMda visheVSadashaRnavu (BeVdajAcnxnavu)
heVLalapxDuvudu, adara aMtaveMdare samApitx suSupitx (supatx)veMdu
heVLalapxDuvudu, hiVgeMdare parxmAtArxdidevxYtavilalxda tatavxvu
sididhxsuvudu, adara hAniyU kUDa tatAkxladalilx niviRkAravAda
boVdhadiMdaleV goVtAtxguvudu (Aguvadu) beVre
sAdhakavilalxvAdadadxriMda, adariMda savxpAnxnatxpadadiMda
nitayxdaqSiTxyanunx devxYtABAvavanunx mugiyisuva vAkayxvu
savxpanxdiMda Atamxnige jAgaradasheyu uMTAguvudanunx toVrisuvudeMdu
tAtapxyaR}
\begin{shl}
iteyxVvaM pUvaRsidedhxV\s theVR \footnotemark[1]nigamAthaRM punavaRcaH || \\
sa vA itAyxdikaM jecnxVyaM \footnotemark[2]na tu nAshAdishaknakxyA \hfill || 1909 ||
  
\end{shl}
\footnotetext[2]{malagidavanu satatxvanaMte naSaTxvAguvanoV
athavA deVshAMtarakekx hoVgiruvano ? eMdu shaMkisidare bAyikaLedu
malaruvudariMda satatxvanaMte toVri baruvudariMdalU I shaMkeyanunx
nirAkarisalu muMdina vAkayxvu baMdideyeMdarU sarihoVguvudilalx,
EkeMdare ?  nitayxsidadhx niSikxrXya niguRNAtamxkavAda
parxtayxgAtamxnalilx A shaMkege avakAshavilalxvAdadxriMda avanu
malagidadxrU maqtanaMtealalx, muKadalilx parxsananxteyu
kADuvudariMda maqti hoMdilalx, usirADuvudariMdalU maqtapaTiTxlalx}

%%%% footnote shloka {1,2}
\begin{artha}
I riVtiyAgi modaleV niNaRyavAda
viSayadalilx matetx heVLuva `savA ESa' itAyxdi vAkayxvu
mugisuvudakAkxgi baMdideyeMdu tiLiyabeVku, Adare `AtamxneV
nAshavAguvaneMba riVriyalilx shaMkeyu baMdiruvudeMbudariMda alalx.
\end{artha}

\section*{baq. 4ne adhAyxya 3 neV bArxhamxNa kaMDike 35}

\begin{shl}
tadayxthAnaH susamAhitamutasxjaRdAyxyAdeVvameVvAyaM shAriVra AtAmx pArxjecnxVnAtamxnAnAvxrUDha utasxjaRnAyxti yaterxYtadUdhovxVRcACxvXsiV Bavati || 35 ||
\end{shl}

\vishaya{uLida bArxhamxNada tAtapxyaR}

\begin{shl}
ita AraBayx saMsAra AtAmxnoV vaNaRyeV\s dhunA ||  \\
savxpAnxdobxVdhApitxvacAcxsAmxlolxVkAlolxVkAnatxraM gataH \hfill || 1910 ||
  
\end{shl}

\begin{shl}
itayxthaRparxtipatatxyXthaRM daqSATxnotxV\s tArxBidhiVyateV || \\
suKAvaboVdhasidadhxyXthaRM shorxVturatheVR vivakiSxteV \hfill || 1911 ||
  
\end{shl}

\begin{artha}
`tadayxdhA anaH susamAhita.' ililxMda AraMBisi Atamxna saMsAravanunx vaNiRsuvudu,
savxpAnxvasethxyiMda jAgarAvasethxyanunx hoMduvaMte I loVkadiMda
loVkAMtaravanunx hoMduvaneMdu I viSayavanunx tiLipaDisuvudakAkxgi
ililx daqSATxMtavanunx udidxSaTxvAda viSayakekx kUDiruvudu,
EtakAkxgi ? eMdare shorxVtaqvige suKavAgi jAcnxnavAguvudakokxVsakxra.
\end{artha}

\vishaya{maMtarxvAyxKAyxna}

\begin{shl}
nAnAthaRsAdhaneYmARgeVR yathA\s naH susamAhitamf || \\
shabAdxnAnxnAvidhAnukxvaRdugxruBAraparxpiVDanAtf \hfill || 1912 ||
  
\end{shl}

\begin{shl}
varxjeVcACxkaTikeVneVha deVshAnatxramadhiSiThxtamf || \\
anasasatxdagxtAcAcxthARdivxlakaSxNavapuBaqRtA \hfill || 1913 ||
  
\end{shl}

\begin{shl}
sAvxtheVRnAdhiSiThxtaM gaceCxVtapxthi shAkaTikeVna tatf || \\
daqSATxnAtxtheVRna saMbanadhx EvameVveVti BaNayxteV \hfill || 1914 ||
  
\end{shl}

\begin{shl}
BukatxdeVhAdidaM liknagxmutAkxrXnatxM BoVgasaMkaSxyAtf || \\
yAti deVhAnatxraM tadavxtakxmaRvidAyxdisaMBaqtamf \hfill || 1915 ||
  
\end{shl}

\begin{artha}
aneVka PalasAdhanavAda vasutxgaLiMda tuMbida gADiyu heVge
bahuBAradiMda uMTAda piVDaneyiMda aneVkabageyAda shabadxgaLanunx
mADutAtx gADi hoDeyuvavaniMda EralapxTuTx hoVguvudo (adaralUlx) gADigU
gADiyalilxruva vasutxgaLigU beVreyAda AkAravanunx dharisida
savxtaMtarxgADi cAlakaniMda EralapxTuTx dAriyalilx heVge hoVguvado
hAgeyeV aMdare I daqSATxMtada athaRdaMte ideV riVtiyAgi anuBava mADida
deVhavanunx biTuTx BoVgavu mugididadxriMda meVlakekxdudx I liMga
shariVravu kamaR videyx modalAda saMBAradiMda tuMbidudx beVre
deVhakekx hoVguvudu.
\end{artha}

\begin{shl}
viyukatxM deVvatABiH satakxmaRsaMBArasaMBaqtamf || \\
anoVvalilxknagxmeVteyxVtatapxrXtayxgAtAmxthaRsidadhxyeV \hfill || 1916 ||
  
\end{shl}

\begin{artha}
anugArxhakavAda deVvategaLiMda biDalapxTa I liMga shariVravu gADiyaMte
kamaRgaLa horeyiMda tuMbidAdxgi parxtayxgAtamxna iSATxthaRsididhxgAgi
horaDuvudu.
\end{artha}

\vishaya{`shAriVraH' eMba padadiMda gArxhayxvAda athaR -}

\begin{shl}
shAriVravacasA cAtarx liknagxmeVvABidhiVyateV || \\
shariVradeVshasaMsathxtAvxnanx tu parxtayxknaknxsaMhateV \hfill || 1917 ||
  
\end{shl}

\begin{artha}
shAriVraveMba vacanadiMda ililx liMga shariVraveV heVLalapxDuvudu,
shariVradeVshadalilxruvudariMda (shAriVraveMdu), Adare
parxtayxgAtamxneMdu heVLalapxDuvudilalx, EkeMdare ? samUhadalilx
seVruvudilalxvAdadxriMda alalx.
\end{artha}

\vishaya{``pArxjecnxnAtamxnA'' eMbudara athaRveVnu ?}

\begin{shl}
pArxjecnxVnAdhiSiThxtaM gaceCxVdarxthinA sAvxthaRrUpiNA || \\
parxtayxgidhxyaH samAroVhamanAvxtAmx\s \s rUDhavadayxtaH \hfill || 1918 ||
  
\end{shl}

\begin{shl}
pArxjecnxVneVhA\s \s tamxnA tasAmxdanAvxrUDhoV\s BidhiVyateV ||  \\
\footnotemark[1]BAnuneVvoVdapAtArxdeVrAroVhoV nA\s \s tamxnaH savxtaH \hfill || 1919 ||
  
\end{shl}
\footnotetext[1]{jalapAtArxdigaLige tananxlilx parxtiMbisuva sUyaRna
mUlaka vAyxpaneyeMbuvaMte pArxjacnxnige ceYtanAyxBAsada mUlaka
budidhxyalilx vAyxpitxyide savxtaH asaMganAda puruSanige
nijavAgilalx.}

%%% footnote[1]{}
\begin{artha}
sAvxthaRsavxrUpavuLaLx (savxtaMtarxnAda) pArxjacnxneMba
rathikaniMda EralapxTuTx (I liMga shariVravu) hoVguvudu, parxtayxgfdhiVyeMbuva (ceYtanAyxBAsavu) aMtaH karaNadalilx hatitxruvudanunx (vAyxpisiruvudanunx) anusarisi AtamxnU hatitxruvavanaMte toVruvanu, adariMda pArxjacnxneMba AtamxniMda AroVhaNa mADalapxTaTxvaneMdu heVLalapxTiTxde, udAH- (tananxlilx parxtibiMbisida) sUyaRniMda jalapAtArxdigaLige vAyxpaneyiruvaMte, Atamxnige savxtaH AroVhaNavu ilalx. 
\end{artha}

\vishaya{pArxjAcnxtamxnu Eke beVku ? tAnAgiye liMgashariVravu
horaDabArade ? eMdare -}

\begin{shl}
pArxkocxVkatxmAtamxneYvAyaM joyxVtiSeVtAyxdikaM vacaH || \\
AtAmxnaM rathinaM vididhxVtayxpi shurxtayxnatxreV vacaH \hfill || 1920 ||
  
\end{shl}

\begin{artha}
hiMdeyU ``AtamxneYvAyaM joyxVtiSAsetxV'', itAyxdi vacanavanunx heVLidAdxyitu, alalxde `AtAmxnaMrathinaM vididhx' eMdu beVre shurxtiyalUlx (kaThadalUlx) I mAtu ide.
\end{artha}

\vishaya{`utasxjaRnf yAti' eMbudara athaRvanunx vishadapaDisuvudu}

\begin{shl}
mamaRkaqnatxnasaMBUtaveVdanAdiRtamAnasamf || \\
bikikxkAlakaSxNaM shabadxM maqtikAla upasithxteV \hfill || 1921 ||
  
\end{shl}

\begin{shl}
utasxjaRdAyxti liknagxM taditi parxtayxkaSxgoVcaraH ||  \\
tadidaM BaNayxteV shurxtAyx puMsAM veYrAgayxjanamxneV \hfill || 1922 ||
  
\end{shl}

\begin{artha}
mamaRvanunx katatxrisuvadariMda uMTAda duHKadiMda piVDitavAda
manasusxLaLx liMga shariVravu (ceYtanAyxBAsavuLaLx pArxNa tatavx)
maraNa kAlavubaMdAga bikf bikf eMdu shabadx mADutAtx hoVguvudeMbudu
anuBavadalilx kANuvudu, I viSayavanunx shurxtiyu mAnavarige
veYrAgayxvanunx huTiTxsalu heVLiruvudu.
\end{artha}

\begin{shl}
mamaRsUtakxqqtayxmAneVSu mucayxmAneVSu saMdhiSu || \\
mumUSaRtoV\s tarx yadudxHKaM samxyaRtAM tanumxmukuSxBiH \hfill || 1923 ||
  
\end{shl}

\begin{artha}
mamaRsAthxnagaLu kitetxLayelalxDutitxralu saMdhi sAthxnagaLU
biDisalapxDutitxralu sAyutitxruvavanige ililx yAva duHKavAguvadoV
adanunx mumukuSxgaLu samxrisabeVku.
\end{artha}

\begin{shl}
kasimxnAkxla idaM janotxVH kiMnimitatxM ca jAyateV ||  \\
itayxsayx parxtipatatxyXthaRM paroV garxnothxV\s vatAyaRteV \hfill || 1924 ||
  
\end{shl}

\begin{artha}
yAva kAladalilx pArxNige idu uMTAgutatxde ? yAva nimitatxdiMda
Agutatxde ? eMba I viSayavanunx tiLisuvudakokxVsakxra muMdina
garxMthavanunx (vAyxKAyxnakAkxgi) iLisuvudu.
\end{artha}

\vishaya{- baq. nAlakxne adhAyxya 3 ne bArxhamxNa kaMDike 36 -}

\begin{shl}
sa yatArxyamaNimAnaM neyxVti jarayA voVpatapatA vANimAnaM nigacaCxti tadayxthAmarxM voVdumabxraM vA pipapxlaM vA banadhxnAtapxrXmucayxta EvameVvAyaM puruSa EBoyxV\s knegxVBayxH samapxrXmucayx punaH parxtinAyxyaM parxtiyoVnAyxdarxvati pArxNAyeYva || 36 ||
\end{shl}

\vishaya{modalina parxshenxge utatxra -}

\begin{shl}
udAnavAyw parxbaleV yaterxYtatAsxyXnumxmUSaRtaH ||  \\
UdhovxVRcACxvXsiV pumAMsatxtarx yathoVkatxM jAyateV maqtw \hfill || 1925 ||
  
\end{shl}

\begin{artha}
sAyuvavanige udAna vAyuvu parxbalavAdAga yAva maraNavu samiVpisuvado
AvAga maraNadalilx A manuSayxnu meVlusuruLaLxvanAgutAtxne.
\end{artha}

\vishaya{eraDane parxshenxge utatxra}

\begin{shl}
piNoDxV\s NutavxM yadA\s BeyxVti jarayA\s BayxdiRtasatxdA || \\
upatApAnivxtoV vA sanayxthoVkatxM parxtipadayxteV \hfill || 1926 ||
  
\end{shl}

\begin{artha}
sUthxla shariVravu yAvAga saNaNxdAguvudo (kaqshateyanunx hoMduvado)
AvAga mupipxniMda sutatxlU piVDisalapxTuTxdAgiyoV javxrAdigaLa
tApadiMda kUDiyo idudx hiMde heVLida (meVlusiranunx eLeyuva
sithxtiyanunx hoMduvanu)
\end{artha}

\begin{shl}
aNimAnamaNoVBARvaM kAshayxRM deVhoV yadeYtayxyamf || \\
jarayA roVgaheVtoVvAR UdhovxVRcACxvXsiV tadA BaveVtf \hfill || 1927 ||
  
\end{shl}

\begin{artha}
yAvAga I deVhavu mupipxniMdaloV roVga nimitatxdiMdalo aNima rUpavanunx
aMdare saNaNxgiruvaMte kaqshateyanunx hoMduvado, AvAga UdhovxRV
cACxsiyAguvanu.
\end{artha}

\vishaya{mupipxniMdalAgali javxrAdiroVgagaLiMdalAgali kaqshateyu heVge ? Aguvudu ? eMdare -}

\begin{shl}
jarayA roVgatoV vA\s sayx jAyateV viSamAginxtA || \\
samayxkapxkitxsatxtoV\s nanxsayx manAdxgenxVnoVRpajAyateV \hfill || 1928 ||
 
\end{shl}

\begin{shl}
rasAdidhAtuBideVRhasatxtoV\s nupacayAdayamf || \\
abalaH sanapxtateyxVSa jiVNaRmanidxravatikxSXtw \hfill || 1929 ||
  
\end{shl}

\begin{artha}
mupipxniMdalAgali roVgadiMdAgali Itanige jATharAginxyu
parxkoVpisuvudu, AvAga maMdAginxyuLaLx Itanige ananxvu sariyAgi
jiVNiRsuvudilalx, rasAdidhAtugaLiMda beLeyadiruvudariMda I deVhavu
(kaqshavAgi) dubaRlavAgi BUmiyalilx haLeya maneyu bidudxhoVguvaMte
bidudx hoVguvudu.
\end{artha}

\vishaya{``tadayxthA\s\s maMvA'' itAyxdi shurxtiya tAtapxyaR}

\begin{shl}
atheYvaM kaqshatApanenxV deVheV\s natxrupajAyateV ||  \\
yA vaqtitxH soVcayxteV sAkASxdadxqqSATxnotxVkAtxyX parxyatanxtaH \hfill || 1930 ||
  
\end{shl}

\begin{artha}
I riVtiyAgi deVhavu kaqshateyanunx hoMdutitxralu yAva vaqtitxyu oLage
huTuTxvado adanunx Iga daqSATxMta vacanadiMda neVra parxyatanxdiMda
tiLisuvudu.
\end{artha}

\begin{shl}
AmarxM yathA rasAdADheyxVR savxsAmxdfvaqnAtxtapxrXmucayxteV || \\
tatheVhAnanxrasagAlxnw liknagxM deVhAtapxrXmucayxteV \hfill || 1931 ||
  
\end{shl}

\begin{artha}
mAvina haNuNx heVge tananx rasavu gaTiTxyAgilalxvAdare (koLetare)
(athavA shoSaNeyAdare) tananx toTiTxniMda kaLacuvudo hAgeye
ananxrasavu hAniyanunx hoMdidare I liMga shariVravu deVhadiMda
biDalapxDuvudu.
\end{artha}

\vishaya{daqSATxMtadalilxruva baMdhana padada athaR.}

\begin{shl}
vaqnetxV nibadhayxteV yeVna raseVnA\s \s paripAkataH ||  \\
sa rasoV banadhxnaM jecnxVyoV vaqnatxM vA banadhxnaM matamf \hfill || 1932 ||
 
\end{shl}

\begin{artha}
yAva rasadiMda paripakavxvAgadiruvAga toTiTxnalilx (haNuNx) hiDidu
nililxsalapxDuvado, A rasaveV baMdhanaveMdu tiLiyabeVku, toTATxdarU
baMdhanaveMbudu iSaTx.
\end{artha}

\begin{shl}
vaqnetxVnA\s \s marxsayx saMbanodhxV rasasAyx\s \s paripAkataH || \\
bAleyxV daqDhoV yathA tadavxnenxYvaM pakavxrasasayx saH 1933 ||  
\end{shl}

\begin{artha}
toTiTxnoDane mAvina kAyige saMbaMdhavu rasavu
paripAkavAgadiruvudariMda iruvudu, adu eLetAgiruvAga gaTiTxyAgiruvudu,
idu heVgo hAgeye rasavu pari pakavxvAgiruvAga (kAyige) A toTiTxna
saMbaMdhavu iruvudilalx. AdarU haNuNx biVLuvudakekx pakavx rasavu heVge kAraNa ? veMdare 
\end{artha}

\begin{shl}
kelxYdayxmApadayxmAnoV\s tha pAkakAla upasithxteV || \\
rasa AmarxM dhArayituM neYveVhoVtasxhateV\s balaH \hfill || 1934 ||
  
\end{shl}

\begin{artha}
pAkakAlavu (haNANxguva kAlavu) samiVpisidAga niVrAgi mADalapxDalu I
rasavu dubaRlavAgi mAvina haNaNxnunx nililxsikoLaLxlu utAsxha
paDuvudilalx, (shakatxvAguvudilalx).
\end{artha}

\begin{shl}
gurutAvxdabxnadhxnAdavxqqnAtxdarxsAdAvx\s tha parxmucayxteV \hfill || 1935 ||\\
nAnAheVtukapAtasayx Palasayx parxtipatatxyeV ||  \\
daqSATxnAtxnAmihoVkitxH sAyxdUBxyasAM BinanxrUpiNAmf \hfill || 1936 ||
  
\end{shl}

\begin{artha}
tUkavAgiruvudariMda toTeTxMbuva baMdhanadiMdalo rasadiMdalo
anaMtaraveV haNuNx kaLacibiDuvudu, nAnA kAraNagaLiMda Aguva
patanaveMba kAyaRvanunx tiLisuvudakAkxgiye BinanxrUpavAda aneVka
daqSATxMtagaLanunx ililx heVLiruvudu.
\end{artha}

\begin{shl}
vaqnAtxdeVvA\s \s marxpAtoV\s tarx vaqnetxVnoVdumabxraM saha ||  \\
patatayxshavxtathxpAtasutx pAkeV\s payxnayxnimitatxtaH \hfill || 1937 ||
  
\end{shl}

\begin{artha}
toTiTxniMdaleV mAvina haNuNx biVLuvudu, toTiTxnoMdige seVri atitxhaNuNx
biVLuvudu, ashavxtathxda haNuNx paripakavxvAdarU beVre nimitatxdiMda
biVLuvudu.
\end{artha}

\begin{shl}
bahuparxkArasidadhxyXthaRM vAshabadxvAyxhaqtisitxvXha ||  \\
bahuheVtumaqRtiyaRsAmxtApxrXNinAM jagatiVkaSxyXteV \hfill || 1938 ||
  
\end{shl}

\begin{artha}
aneVka parxkAravideyeMbudanunx tiLisalu vAshabadxvanunx shurxtiyalilx
heVLiruvudu, pArxNigaLa maraNavu aneVka nimitatxdiMda AguvudeMbudu
loVkadalilx kANutatxde.
\end{artha}

\begin{shl}
yathA\s yamukotxV daqSATxnatxH puruSoV\s peyxVvameVva hi ||  \\
liknAgxtAmx puruSoV jecnxVyasatxseyxYveVhAknagxsaMgateVH \hfill || 1939 ||
  
\end{shl}

\begin{artha}
heVge I daqSATxMtavanunx heVLideyo hAgeye puruSanU kUDa aMgagaLiMda
biDalapxDuvanu, ililx puraSaveMdare liMga shariVraveMdu tiLiyabeVku,
adakekx ililx (deVhadalilx) aMga saMbaMdhaviruvudu (adariMda hAge tiLiyabeVku)
\end{artha}

\vishaya{aMgashabadxda athaR}

\begin{shl}
shorxVtarxtavxgAdinADayxshacx porxVcayxnetxV\s tArxknagxsaMjacnxyA || \\
teVSu vuyxheyxYva liknAgxtAmx yatoV\s laM sAyxtasxvXkamaRNeV \hfill || 1940 ||
  
\end{shl}

\begin{artha}
shorxVterxVMdirxya tavxgiMdirxya modalAdavugaLU nADigaLU saha aMgaveMba
padadiMda heVLalapxTiTxve, EkeMdare ? avugaLalelxV liMga shariVrada
Atamxvu seVrikoMDeV tananx kAyaRvanunx mADalu samathaRvAgutatxde.
\end{artha}

\vishaya{liMga shariVravu shariVradalilxruvudakekx kAraNaveVnu ? -}

\begin{shl}
aknagxsayx kaqSaNxsArasayx cakuSxSaH karaNasayx ca || \\
AsharxyAsharxyisaMbanedhxV heVturananxrasoV BaveVtf \hfill || 1941 ||
  
\end{shl}

\begin{artha}
kapApxda aleyiruva avayavakUkx (kaNuNxguDeDxgU) cakuSxriMdiyakukx
Asharxya AshirxtaveMba saMbaMdhaviralu ananxrasaveV kAraNavAgiruvudu.
\end{artha}

\vishaya{hiMdeyeV liMga shariVravu deVhadoLage iralu ananxrasaveV
kAraNaveMdu heVLide,}

\begin{shl}
ananxM dAmeVti cApuyxkatxM rasAdipariNAmataH || \\
tadAyxvakatakxThinaM deVheV tAvalilxknagxM sithxraM BaveVtf \hfill || 1942 ||
  
\end{shl}

\begin{artha}
ananxvanunx (catuthARdhAyxyadalilx) dAmaveMdu rasAdi pariNAmadiMda adu
eSuTx kAla gaTiTxyAgiruvudo aSuTx kAla I deVhadalilx sithxravAgiruvudu.
\end{artha}

\vishaya{adeV ananx rasavu paripakavxvAdare shariVradiMda adu horaDalu
kAraNavAguvudeMdu heVLutAtxre -}

\begin{shl}
jarAdiheVtupAkeV tu patatAyxmArxdivadf durxtamf || \\
EBoyxV\s knegxVBayx iti girA tadeVtadaBidhiVyateV \hfill || 1943 ||
  
\end{shl}

\begin{artha}
mupepxV modalAda kAraNadiMda haNANxdare mAvina haNuNx
modalAdavugaLaMte shiVGarxvAgi I aMgagaLiMda (liMga shariVravu horaTu)
sUthxla shariVravu biVLuvudeMdu ``EBoyxV\s knegxVBoyxVH'' eMba vAkayxdiMda adeV I
viSayavanunx heVLiruvudu.
\end{artha}

\vishaya{`saMparxmucayx' eMbalilx upasagaRgaLa athaRvanunx heVLutAtxre -}

\begin{shl}
upasagaRH samiteyxVSa EkiVBAvaparxsidadhxyeV || \\
tathA parxkaSaRsidadhxyXthaRM porxVpasagaRH parxyujayxteV \hfill || 1944 ||
  
\end{shl}

\begin{artha}
samf eMba upasagaRvu Ekatavxvanunx tiLisuvudakAkxgiyU `parx' eMba
upasagaRvu hecAcxgi eMba athaRvanunx tiLisuvudakUkx
parxyoVgisalapxTiTxve.
\end{artha}

\vishaya{EkiV BAvavu yAvudu ? eMdare -}

\begin{shl}
vayxsAtxni sAthxnasaMbanAdhxtakxraNAni yatasatxtaH || \\
savxsAthxneVBoyxV\s pakaqSATxni maraNeV yAnatxyXtheYkatAmf \hfill || 1945 ||
  
\end{shl}

\begin{artha}
iMdirxyagaLu (modalu tananx tananx sAthxnadalilx) idadxdadxriMda
biDiyAgidadxvu, adariMda tamamx sAthxnagaLiMda Icege
eLeyalapxTaTxvugaLAgi maraNakAladalilx Ekatavxvanunx hoMdutatxve
(oMdAgutatxve).
\end{artha}

\vishaya{`parx' eMba upasagARthaRda vivaraNe -}

\begin{shl}
savxpenxV pArxNAvasheVSatAvxtAsxthXnAtakxraNasaMhaqteVH || \\
maqtw saheYva cAneVna parxkaSoVR\s tArxta IkaSxyXteV \hfill || 1946 ||
  
\end{shl}

\begin{artha}
savxpanxdalilx (suSupitxyalilx) pArxNavu uLiyuvudariMda iMdirxyagaLu
savxsAthxnadiMda upasaMhAravanunx hoMdiruvudakikxMtalU maraNadalilx I
pArxNada joteyalelxV upasaMhAravAguvudu hecicxnadeMdu (upasagaRda
athaR) parxkaSaRvu ililx adariMda kANuvudu.
\end{artha}

\vishaya{`punaH parxtinAyxyamf' eMbalilx punaH padada athaR -}

\begin{shl}
deVhAnatxrAdamuM deVhaM pUvaRmAgAdayxtasatxtaH || \\
tadapeVkaSxyX punaHshabadxH shurxteyxYveVhABidhiVyateV \hfill || 1947 ||
  
\end{shl}

\begin{artha}
yAvudoV oMdu hiMdina beVre deVhadiMda I pUvaR deVhakekx baMda jiVvavu (adakUkx beVreyAgi) punaH (matotxMdanunx hoMduvadeMdu) shurxtiye ililx heVLiruvudu.
\end{artha}

\vishaya{udAharaNeyiMda idanunx sapxSaTxpaDisuvudu -}

\begin{shl}
savxpAnxdayxtheVtoV budAdhxnatxM savxpanxM budAdhxnatxtaH punaH || \\
pUvaRdeVhAtatxtheYvA\s \s tAmx yAti deVhAnatxraM punaH \hfill || 1948 ||
  
\end{shl}

\begin{artha}
savxpanxdiMda jAgarakUkx jAgaradiMda punaH savxpanxkUkx jiVvanu baMdu
hoVguvudu heVgo, hAgeyeV pUvaRdeVhadiMda Atamxnu punaH beVre deVhakekx
hoVguvanu.
\end{artha}

\vishaya{idanenxV punaH saMkeSxVpisi heVLutAtxre -}

\begin{shl}
bAhAyxdedxVhAtatxtheYvAnatxliRknagxM sAvxpanxM parxpadayxteV || \\
yathAnAyxyaparxtinAyxyw yathA tatorxVditw purA \hfill || 1949 ||
  
\end{shl}

\begin{shl}
deVhAdedxVhAnatxrapArxpwtx tatheYveVti vinidiRsheVtf ||  \\
parxtiyoVni yathAsAthxnamitayxtArxpi tatheYva tatf \hfill || 1950 ||
  
\end{shl}

\begin{artha}
heVge horagina (jAgara) shariVradiMda oLagiruva
savxpAnxvasethxyalilxruva liMgashariVravanunx I Atamxnu paDeyuvano
hAge, yathAnAyxya\footnote[1]{parxtinAyxya padada vAyxKAyxnave
yathAnAyxyaveMbudu, beVreyalalx.}, parxtinAyxyaveMba shabadxgaLu
`sa vA ESa Etasimxnf saMparxsAdeVratAvx' itAyxdi sathxLagaLalilx hiMde heVLalapxTiTxve, hAgeyeV oMdu
deVhadiMda matotxMdu deVhavanunx hoMduva saMdaBaRdalilx
nideRshisabeVku, parxtiyoVgi eMbudu yathAsAthxnamf eMbathaRdalilx
hiMde vAyxKayxnisalapxTiTxde hAgeye ililxyU iruvudeMdu (tiLiyabeVku).
\end{artha}

\vishaya{oTuTx vAkAyxthaR -}

\begin{shl}
shurxtakamARnurUpeVNa cakuSxrAdayxnurUpataH || \\
parxtiyoVni yathAsAthxnamAdarxvateyxVSa janamxneV \hfill || 1951 ||
  
\end{shl}

\begin{artha}
I Atamxnu tAnu mADida kamaRkekx takakxMte cakuSxrAdi iMdirxyagaLige
anuguNavAgi parxtiyoVni, aMdare yathAsAthxnakekx janamx
etutxvudakAkxgi baruvudu.
\end{artha}

\begin{shl}
deVhAnatxragaqhiVtayxthaRM pArxNinAM maraNaM yataH || \\
pArxNAyeVti tatoV vakitx janamxnayxsati noV maqtiH \hfill || 1952 ||
  
\end{shl}

\begin{artha}
beVredeVhavanunx paDeyuvudakAkxgiye pArxNigaLige sAvu baruvudu,
adariMda `pArxNAya' eMdu (janamxvetutxvudakAkxgi eMdu) shurxtiyu
heVLade, janamxveV ilalxvAdare namage sAvU ilalx.
\end{artha}

\vishaya{pArxNa sahitavAgiyeV jiVvavu horaDuvAga `pArxNAya'
pArxNakAkxgi eMdu shurxti heVLuvudu sariyeV ? eMdare -}

\footnotetext[1]{pArxNagaLeMdare pArxNAdi paMcavAyugaLU matutx
iMdirxyagaLu, ivu mukitx payARMtara idedxV irutatxve, avugaLanunx
oTuTx kUDisuvudu niSapxrXyoVjana, adakAkxgi beVre deVvanunx
tegedukoLuLxvudeMbudu yadayxpi sariyalalx, AdarU AyAya
iMdirxyagaLanunx AyAya sAthxnagaLalilx irisida vinaha Atamxnu
kamaRvanunx yAvudeV oMdu vayxvahAravanunx mADalu
samathaRnAguvudilalx, adakAkxgi pArxNagaLanUnx AyAya sAthxnadalilx
nililxsuvudakAkxgiyeV deVhAMtarakekx seVrikoLuLxvanu pArxNavUyxhakAkxgi
eMbudakekx AyAya AyAya goVLakAdisAthxnagaLalilx
nililxsuvudakAkxgiyeMdeV athaRvu, I athaRvu AnaMdagirigaLa
vAyxKAyxnadiMda gotAtxguvudu, tathApi na tatarxdogxVlakAdisAthxneVSu karaNasAthxpanaMvinA\s ya mAtAmx kamaR nimARta miVSeTxV tasAmxdedxVhAnatxra garxhaNaM pArxNa vUyxhAthaRmf athaRpatf  eMdu vAtiRkaTiVke, Adare
pArxNavUyxha shabadxkekx AyAya iMdirxyavanunx AyAya sAthxnadalilx
nililxsuvudeMba mAtarx athaRvalalx, iMdirxyagaLige AyAya
sAthxnadalilx visheVSavAda aBivayxkitxyuMTAguvudeMbuva payaRMta
athaR. baq. BASayx TiVkeyalilx ``pArxNAvUyxhAya pArxNAnAM visheVSABivayxkitxlABAyeVtiyAvatf'' eMdu vAyxKAyxna mADive,
`pArxNAyeYva' eMdu shurxtiyalilxde, adakekx pArxNavUyxhAyeYva eMdu
BASAyxnusAra vAtiRkabaMdide, hiVge athaRvanunx mADadidadxre keVvala
pArxNakAkxgi deVhAMtaravanunx tegedukoLuLxvudeMbudu
vayxthaRvAguvudu, horaDuvAga pArxNasameVtanAgiyeV jiVvanu horaDuvanu
`pArxNAya' eMdu heVLabeVkAgilalx idara sAthaRkateyanunx toVrisalu
BASayxkAraru `pArxNavUyxhAya eMdu athaRvanunx mADiruvaru, keVvala
pArxNavidadxrU BoVgavAguvudilalx, adakAkxgi iMdirxyagaLu
AvashayxkaveMdu pArxNa padakekx iMdirxyigaLeMdu athaRmADivUyxhAya
eMdidAdxre. Adare 1951 ne vAtiRkadalilx `ESajanamxne' eMbudAgi
heVLidadxriMda `pArxNAya' eMbudakekx janamxkAkxgi eMdu
lAkaSxNikAthaRvanunx mADidadxru. adakekx `dhurxvaMjanamxmaqtasayxca' eMba giVteye AdhAra
I athaR vAtiRkakAraru modalu mADida athaR.}
\begin{shl}
\footnotemark[1]pArxNavUyxhAya ceYvAyamAdarxvatayxnayxdeVhataH || \\
sAthxneVSavxvUyxDhakaraNoV nAlaM sAyxtakxmaRNeV yataH \hfill || 1953 ||
  
\end{shl}

%%% footnote shloka {1} 
\begin{artha}
beVre deVhadiMda beVredeVhakekx pArxNavUyxhakAkxgi
(pArxNagaLa sAthxpanegAgi) Itanu hoVguvanu, EkeMdare ? iMdirxya
sAthxnagaLalilx iMdirxyagaLanunx sAthxpane mADade idadxre Itanu
kamaRvanunx mADalu samathaRnAguvudilalx.
\end{artha}

\vishaya{(AkeSxVpa) shaMke.}

\begin{shl}
pArxNavUyxhasayx saMsididhxH kathamaseyxVti shaknakxyXteV || \\
shariVragarxhaNAshaketxVravUyxDhakaraNatavxtaH \hfill || 1954 ||
  
\end{shl}

\begin{artha}
shariVravanunx tegedukoLuLxva shakitxyilalxvAdadxriMda
pArxNavUyxhada racaneyAguvudilalx I kAraNadiMda
pArxNavUyxhavu Itanige heVge ? sidhidxsuvudeMdu shaMkemADide.
\end{artha}

\begin{artha}
samAdhAnavanunx hiVge mADabahudalalx, kelavu BaqtayxsAthxnadalilxdudx
deVhAMtaravanunx racisikoMDu jiVvanu baruvudanenxV niriVkiSxsutAtx
kuLitiveyeMdu heVLabahudalalx ? eMdare
\end{artha}

\vishaya{samAdhAna -}

\begin{shl}
na ca deVhAnatxraM kaqtAvx rAjocnxV BaqtAyx ivA\s \s sateV || \\
yataH kamaRsahAyoV\s yameVkAkeyxVveVha gacaCxti \hfill || 1955 ||
  
\end{shl}

\begin{artha}
matutx beVre deVhavanunx saqSiTxsi rAjana BaqtayxraMte niriVkiSxsutAtx
yArU kuLitU ilalx, yAvakAraNadiMda eMdare - I jiVvanu
kamaRsahAyavuLaLxvanAgi obabxMTiganAgiyeV ililx (saMsAra maMDaladalilx) hoVgavanu.
\end{artha}

\vishaya{I meVlina shaMkeya samAdhAna -}

\begin{shl}
itayxsayx parihArAthaRM daqSATxnotxV\s tArxBidhiVyteV ||  \\
kaqtasxnXM jagatasxvXBoVgAthaRM puMsoVpAtatxM savxkamaRNA \hfill || 1956 ||
  
\end{shl}

\begin{artha}
idanunx pariharisuvudakAkxgi ililx daqSATxMtavanunx heVLide, elAlx
jagatutx mAnavaniMda tananx BoVgakAkxgi tananx kamaRdiMda
saMpAdisikoMDadedxMdu (tiLiyabeVku)
\end{artha}

\begin{shl}
tatakxmaRPalaBoVgAthaRM kaqtAvx deVhaM parxtiVkaSxteV ||  \\
shuBaM vA yadi vA pApaM shurxtakamARnurUpataH \hfill || 1957 ||
  
\end{shl}

\begin{artha}
AkamaRPalada BoVgakAkxgi deVhavanunx mADikoMDu (iMdArxdi deVvategaLo
mAnavaro yArAdarU) niriVkiSxsutAtxre\footnote[1]{baratakakx
deVhasAvxmiya BoVgakAkxgi iMdArxdi deVvategaLU manuSayxrU
deVhavanunx saqSiTxsi BoVkatxqqvAda jiVvananunx niriVkiSxsutAtxre,
iMdArxdi deVvategaLu BoVkatxqq jiVvana pApa puNAyxnusAravAgi
perxVritarAgi nimiRsuvaru.}. (pUvaRdalilx
saMpAdisida) upAsane, jAcnxna, kamaRgaLige anukUlavAguvaMte
(takakxMte) \footnote[2]{hiMdina janamxdalilx sheVKarisidadx kamaRgaLiMdaleV pArxNigaLu (iMdArxdi deVvatAjaMtugaLU saha) perxVreVpisalapxDuvavu, kamaRkekx takakx deVvatAshariVravanunx racisuvaru, ``tadABxvitAH parxpadayxnetxV tasAmx tatatxsayxroVcateV'' eMba samxqqtiyu karxmavAsaneyiMda saMsakxqqtarAda jiVvaru AyAya janamxvanunx paDeyuvaru, vAsanege takakxMte AyAya shuBA shuBakamaRgaLu rucisutatxveyeMdathaR.}shuBavanonx, ashuBavanonxV (hoMduvanu athavA) mADuvanu.
\end{artha}

\begin{shl}
kamoVRpAtAtxni BUtAni tatapxrXyukAtxni deVhinaH || \\
kaqtAvx deVhAnatxraM yoVgayxM tasUthx rAjocnxV\s garxgA iva \hfill || 1958 ||
  
\end{shl}

\begin{artha}
pArxNiya kamaRdiMda saMpAdisida (deVvAdi) pArxNigaLu kamaRdiMda
perxVreVpisalapxTuTx yoVgayxvAda beVre deVhavanunx nimiRsi rAjana
muMde muMdALAgi janaru nilulxvaMte nilulxvavu.
\end{artha}

\begin{shl}
taterxYtasimxnayxthoVketxV\s theVR daqSATxnotxV\s payxBidhiVyateV ||  \\
vivakiSxtAthaRvijAcnxnaparxsididhxH sAyxtakxthaM nivxti \hfill || 1959 ||
  
\end{shl}

\begin{artha}
I viSayadalilx hiMde heVLida viSayadalilx daqSATxMtavanunx
heVLuvudu, vivakiSxtavAda athaRjAcnxnavu heVge parxsidadhxvAguvudeMdu
(toVrisuvudakAkxgi).
\end{artha}

\section*{baq. 4 ne adhAyxya. 3 ne bArxhamxNa, kaMDike 37}

\begin{shl}
tadayxthA rAjAnamAyAnatxmugArxH parxteyxVnasaH sUtagArxmaNoyxV\s nenxYH pAneYrAvasatheYH parxtikalapxnetxV\s yamAyAtayxyamAgacaCxtiVteyxVvaM heYvaMvidhaM savARNi BUtAni parxtikalapxnatx idaM barxhAmxyAtiVdamAgacaCxtiVti || 37 ||
\end{shl}

\section*{vAtiRka}

\begin{shl}
yathA rAjAnamAyAnatxM sadamxpAnAdisAdhaneYH || \\
budAdhxvX kalxqqpetxYH parxtiVkaSxnatx ugArxH sUtAdayoV janAH \hfill || 1960 ||
  
\end{shl}

\begin{artha}
heVge rAjanu baruvudanunx tiLidu (sUta) sArathi modalAda ugarx eMbuva
janaru pArxsAda, pAna muMtAda (rAjoVcitavAda) saMBAragaLanunx
kalipxsikoMDu niriVkiSxsuvaro.
\end{artha}

\vishaya{ugarx padakekx athaR}

\begin{shl}
ugerxV kamaRNi yeV rAjAcnx niyukAtxH pApakAriSu || \\
ugArxsetxV\s tArxBidhiVyanetxV \footnotemark[1]jAtitoV vA\s BidhA BaveVtf \hfill || 1961 ||
  
\end{shl}
\footnotetext[1]{ugarxeMba jAti}

\begin{artha}
pApavanunx mADuvavara viSayakekx kUrxravAda shikeSx, modalAda
kAyaRdalilx yAru rAjaniMda niyoVgisalapxTiTxruvaro avaru ugarxreMdu
ililxheVLalapxDuvaru, athavA jAtinimitatxvAgi ugarx eMbudu hesarU Aguvudu.
\end{artha}

\vishaya{parxteyxVnasaH eMbudakekx athaR - }

\begin{shl}
Ena EnaH parxti tathA niyukAtx yeV narAdhipeYH || \\
parxteyxVnasoV\s tarx teV jecnxVyA BinAnxsetxV sAvxdhikArataH \hfill || 1962 ||
  
\end{shl}

\begin{artha}
oMdoMdu pApakAyaRkekx (shikaSxkarAgiruvaMte) tamamx tamamx adhikAra
nimitatx beVre beVreyAgi yAru rAjariMda niyoVgisalapxDuvaro avaru
ililx parxteyxVnasaH eMdu heVLalapxDuvaru.
\end{artha}

\vishaya{sUtAdi padagaLa athaR}

\begin{shl}
rathavAhAshacx sUtAH sUyx rathavAhanakoVvidAH || \\
gArxmaNoyxV gArxmaneVtAraH seVnAdhipatayoV matAH \hfill || 1963 ||
  
\end{shl}

\begin{artha}
rathavanunx ODisuvudaralilx jANarAda rathavanunx naDeyisuvavaru
sUtarenisuvaru, gArxmanAyakarU seVnAdhipatigaLU ililx gArxmANiyaru.
\end{artha}

\begin{shl}
teV yathA kamaRNoVpAtAtx rAjacnxH pArxgeVva cA\s \s gamAtf || \\
ananxpAnagaqhAdayxthARnakxqqtAvx rAjaparxtiVkiSxNaH || \\
rAjA\s yaM naH samAyAtiVtAyxsateV yatanxmAsithxtAH \hfill || 1964 ||
 
\end{shl} 

\begin{artha}
heVge avaru rAjanu baruvudakekx modaleV kamaRdiMdalaBisidudx ananx, pAna,
pArxsAda itAyxdi (satAkxra saMBAra) vasutxgaLanunx kalipxsi, `namamx I
rAjanu baruvaneMdu' rAjananunx niriVkiSxsi takakx parxyatanxvanunx
mADikoMDu iruvaro.
\end{artha}

\vishaya{`Enamf' itAyxdi maMtarxBAgada vAyxKAyxna -}

\footnotetext[1]{savaRBUtagaLeMdare shariVra nimARpakarU iMdirxyAnu
gArxhakaru Ada AditAyxdi deVvategaLu BoVkatxqqvinakamaRperxVritarAgi
sidadhxvAda kamaR Pala BoVga vasutxgaLiMda  niriVkiSxsuvudeMdu athaR.}
\begin{shl}
yatheYvaM \footnotemark[1]savaRBUtAni BoVkAtxrX kamARdikAraNeYH || \\
ajiRtAnuyxcitaM deVhaM BoVkutxBoVRgakaSxmaM tathA \hfill || 1965 ||
  
\end{shl}

\begin{shl}
kaqtAvx tatApxrXpitxtaH pUvaRM yathA sUtAdayoV naqpamf || \\
parxtiVkaSxnetxV parxyatenxVna barxhemxYtiVtayxBikAknakxSXyA \hfill || 1966 ||
  
\end{shl}

%%%% footnote shloka {1}
\begin{artha}
ideV riVtiyAgi samasatxpArxNigaLU BoVga paDuva jiVvaniMda kamaRmuMtAda
sAdhanagaLiMda saMpAdisalapxTaTxvugaLAgi, BoVga paDuvavanige BoVgakekx
samathaRvAda deVhavanunx racisi avanu baruvudakekx modaleV sUtAdigaLu
rAjananunx niriVkiSxsidaMte barxhamxnu baruvaneMdu AkAMkeSxyiMda
parxyatanxdiMda niriVkiSxsutAtxre.
\end{artha}

\vishaya{baruva jiVvAtamxnanunx `barxhAmxyAti' eMdu kareyuvudakekx
kAraNa -}

\begin{shl}
vijAcnxnAtamxnuyxpakArxnetxV barxhemxVti yadihoVcayxteV || \\
savxtoV barxhamxtavxtasatxsayx saMsAritavxmavidayxyA \hfill || 1967 ||
  
\end{shl}

\begin{artha}
jiVvAtamxneV ililx parxsutxtavAgiruvAga yAvAtanu `barxhamx'veMdu
heVLalapxTiTxruvano, avanige savxtaH barxhamxsavxrUpaviruvudariMda
heVLalapxTiTxruvanu. saMsAri rUpavu mAtarx avideyxyiMda Agiruvudu.
\end{artha}

\vishaya{jiVvanalilx barxhamx shabadx parxyoVgakekx matotxMdu tAtapxyaR -}

\begin{shl}
kathaM nAma matisatxsayx barxhAmxsimxVti BaveVdiha || \\
shurxtisatxtatxtatxvXboVdhAthaRM barxhemxVtAyxha tatoV naramf \hfill || 1968 ||
  
\end{shl}

\begin{artha}
Itanige `nAnu barxhamxneV AgiruveneMba' budidhxyu heVge tAne ?
uMTAdiVtu eMdu udedxVshisi shurxtiyu avana nijasavxrUpavanunx tiLisalu
jiVvAtamxnanunx kuritu `barxhamx' eMdu heVLiruvudu.
\end{artha}

\vishaya{``EvaM heVyxvaM vidamf'' eMbuvalilx `EvaMvitf' eMbudara athaR -}

\footnotetext[1]{maraNavu savaRsAdhAraNavAgiruvudariMda
utAkxrXMtiyAguvudeMbudu vidhiyalilx ajAcnxtajAcnxpanavalalx, Adare
loVkasidadhxvAda utAkxrXMtiyanunx deVhAMtaravanunx
sivxVkarisuvaneMbudanunx tiLisalu anuvAdamADide.}
\begin{shl}
anuvAdoV na tu vidhireVvaMvidamitiVyaRteV \hfill || 1969 ||\\
\footnotemark[1]savARnapxrXtayxvishiSaTxtAvxdeVvaMvidamitiVraNamf ||  || \\
nAnuvAdaH PaloVkitxvAR yathAvAyxKAyxtaveVdinaH \hfill || 1970 ||
  
\end{shl}

%%%% footnote shloka {1}
\begin{artha}
`EvaMvidamf' eMbudu anuvAdaveMdu heVLalapxTiTxde. Adare
adu vidhiyalalx, elalxranunx kuritu heVLidudx samAnavAgiruvudariMda
vidhiyalalx. athavA \footnote[2]{shAsatxrXdalilx heVLidaMte
utAkxrXMtiyanunx anusaMdhAna mADuvavanige saMsAradalilx
virakitxyuMTAguvudu, iMtaha mahA puruSananunx kuritu `savARNi BUtAni parxtiVkaSxnetxV'
eMbudAgi Palavanunx tiLisuvadariMda EvavidaM eMba vAkayxdiMda
vidhiye, anuvAdavalalx, keVvala anuvAda mADuvalilx
niSaPxlavAguvudu, adilalxdidadxrU deVhAMtaravanunx
sivxVkarisuvaneMdu heVLuvudakekx swkayaRvide. adariMda anuvAdavalalx
eMdu vidhivAdigaLa matavide.}anuvAdavalalx, hiMde
vAyxKAyxnisidaMte tiLidavanige Palavacanavidu.
\end{artha}

\section*{vAtiRka}

\begin{shl}
karaNAnugarxhiVtaqNi BUtAni ca mumUSaRtaH || \\
kaqtAvx deVhaM parxtiVkaSxnetxV naqpaseyxVva gaqhaM narAH \hfill || 1971 ||
  
\end{shl}

\vishaya{---athaR bareyabeVku---}

\vishaya{`tadayxthA rAjAnaM' itAyxdi muMdina vAkayxda meVle parxshenx -}

\begin{shl}
EvaM jigamiSuM deVvamanugacaCxnitx keV tu tamf || \\
gacaCxnatxH kasayx gatAyx vA sAvxtanetxrXyXVNA\s \s tamxnoV\s thavA \hfill || 1972 ||
  
\end{shl}

\begin{artha}
I riVtiyAgi hoVgalu iciCxsuva A puruSananunx yAru anusarisihoVguvaru ?
hoVguvavarAdarU yAra gamanavanunx anusarisi hoVguvaru ? athavA
savxtaMtarxvAgiyo ? tananx kamaRdiMda (perxVritarAgiyo) heVge
anusarisi hoVguvaru ?
\end{artha}

\begin{shl}
itayxthaRmapi daqSATxnatxH pUvaRvadaBxNayxteV punaH || \\
itaH parxvasatoV deVhAdedxVhinaH kamaRsaMkaSxyAtf \hfill || 1973 ||
  
\end{shl}

\begin{artha}
idakAkxgiyU modaliniMte punaH daqSATxMtavanunx shurxtiyu koTiTxruvudu. ililxMda deVhadiMda kamaRkaSxyavAgi horaDuva jiVvAtamxnige (I daqSATxMtavu)
\end{artha}

\vishaya{daqSATxMtavivaraNe -}

\begin{shl}
yatheVha parxyiyAsanatxmugarxsUtAdayoV janAH || \\
rAjAnamaBisaMyAnitx pArxNAsatxdavxnamxqqtw naramf \hfill || 1974 ||
  
\end{shl}

\begin{artha}
ililx hAge parxyANa mADalu udedxVshisida rAjananunx ugarx sUta
modalAda janaru edurAgi oTuTx kUDibaruvaroV hAgeye maraNakAladalilx
mAnavananunx vAgAdi iMdirxyagaLu muMdALAgi oTATxgi seVri baruvavu.
\end{artha}

\begin{shl}
tadayxthA rAjAnaM parxyiyAsanatxmugArxH parxteyxVnasaH sUtagArxmaNoyxV\s BisamAyanetxyXVvameVveVmamAtAmxnamanatxkAleV saveVR pArxNA aBisamAyanitx yaterxYtadUdhovxVRcACxvXsiV Bavati || 38 ||
\end{shl}

\begin{shl}
yadoVdhovxVRcACxvXsiteYvAsayx puruSasayx Bavatayxtha || \\
suKayanatxsatxdA pArxNAH saMyAnutxyXtAkxrXnitxveVdinamf \hfill || 1975 ||
  
\end{shl}

\begin{artha}
yAvAga I puruSanige meVlumxKavAgi usiru eLeyuvudo AvAga iMdirxyagaLu
utAkxrXMti viSayavanunx tiLidavananunx AnaMdagoLisutAtx muMdALAgi
oTuTxgUDi baruvuvu.
\end{artha}

\begin{center}
iMteMbalilxge shirxV baqhadAraNayxka upaniSadf BASayxvAtiRkadalilx
nAlakxne adhAyxyada mUrane joyxVti bArxhamxNavu  pUNaRvAgide.
\end{center}

\section*{baqhadAraNayxka vAtiRka (4-4-1 kaMDike)}

\section*{- catuthaRbArxhamxNa -}

\begin{shl}
sa yatArxyamAtAmxbalayxM neyxVtayx samomxVhamiva neyxVtayxtheYnameVteV pArxNA aBisamAyanitx sa EtAsetxVjoVmAtArxH samaBAyxdadAnoV haqdayameVvAnavxvakArxmati sa yaterxYSa cAkuSxSaH puruSaH parAknapxyARvataRteV\s thArUpajocnxV Bavati || 1 ||
\end{shl}

\section*{- vAtiRkakAraru heVLida pUvaRpara saMgati -}

\vishaya{EvaMvidaM eMbudu vidhirUpaveMba pakaSxdalilx pUvaR parasaMbaMdhavu}

\begin{shl}
yata utAkxrXnitxruketxVyaM vishiSaTxPalasaMgateVH || \\
saveVRSAmavishiSATx\s taH sa yaterxVtuyxcayxteV\s dhunA \hfill || 1 ||
  
\end{shl}

\begin{artha}
asAdhAraNavAda Palavanunx heVLidadxriMda heVLida I utAkxrXMtiyu
savaRrigU samAnavAgide adariMda `sayatarx' eMdu Iga heVLide ||
\end{artha}

\vishaya{`EvaMvidaM'- eMbudu anuvAda rUpavenunxva pakaSxdalUlx pUvAR
paragarxMthagaLa saMbaMdhavu -}

\begin{shl}
karaNAnAM samutAkxrXnitxgaRmanaM ca tayoVH samamf || \\
pArxgukatxM tatarx yanonxVkatxM tadeVveVhoVcayxteV\s dhunA \hfill || 2 ||
  
\end{shl}

\begin{artha}
iMdirxyagaLige utAkxrXMtiyU matutx gamana I (eraDU) A (jiVva matutx
pArxNigaLige) samAnavAgideyeMdu hiMdeye heVLidAdxyitu, avugaLalilx
yAvudanunx heVLilalxvo adanenxV ililx Iga heVLuvudu.
\end{artha}

\vishaya{saMsAra parxkaraNavu elilxya payaRMtaraveMdare -}

\begin{shl}
saMsArasAyxdhikAroV\s yamA sholxVkoVdAhaqteVmaRtaH || \\
sa yaterxVtayxta AraBayx puMsaH saMsAravaNaRnamf \hfill || 3 ||
  
\end{shl}

\begin{artha}
`sayatArxyaM' eMbalilxMda AraMBisi ``tadeVvasakatxH'' saha kamaRNeVti
eMba sholxVkasahitavAgiruva garxMthavanunx udAharisiruvudariMda
(ililxya payaRMtaviruva garxMthavu) saMsArada parxkaraNavu,
EkeMdare ? jiVvana saMsAravanunx vaNiRsuvudu.
\end{artha}

\vishaya{Isha BASoyxVkatxvAda saMbaMdhavanunx heVLi punarukitx
parihAravanunx mADutAtxre}

\begin{shl}
puMsaH saMsaraNaM pUvaRM sUtirxtaM yatasxmAsataH || \\
visatxrasatxsayx vakatxvayx itayxthAR vA parA shurxtiH \hfill || 4 ||
  
\end{shl}

\begin{artha}
pUvaRdalilx jiVvana saMsAravanunx saMkeSxVpavAgiyAvudanunx
vagiRsididxtoV adara visAtxravanunx heVLabeVkeMdu idakAkxgi I muMdina
shurxti baMdiruvudu.
\end{artha}

\begin{shl}
tatasxMparxmoVkaSxNaM kasimxnupxMsaH kAleV\s BijAyateV || \\
kathaM veVtAyxdikoV\s tArxthoVR visatxreVNoVpavaNayxRteV \hfill || 5 ||
  
\end{shl}

\begin{artha}
A deVhAMgagaLiMda jiVvana biDugaDeyu yAva kAladalilx Aguvudu ? heVge
Aguvudu ? itAyxdi viSayavanunx ililx visAtxravAgi vaNiRsuvudu.
\end{artha}

\begin{shl}
parxkaqtAtamxparAmashaRH sashabedxVna vivakaSxyXteV || \\
saMsArAnathaRsaMbanadhxvijAcnxnAya tamasivxnaH \hfill || 6 ||
  
\end{shl}

\begin{artha}
`saH' eMba shabadxdiMda parxkaqtavAda Atamxnanunx parAmashiRsuvudu,
adu ajAcnxnige saMsArada anathaR saMbaMdhavu tiLiyuvudakAkxgi.
\end{artha}

\begin{shl}
ya AtAmx parxsutxtoV\s vidAvxnasx yadeVdaM shariVrakamf || \\
kAshayxRM pArxpayayx pUvoVRketxYjaRrAroVgAdiheVtuBiH \hfill || 7 ||
  
\end{shl}

\begin{shl}
yadi vA savxyameVveYtayx deVhABeVdatavxheVtutaH || \\
dwbaRlayxmeVvaM saMpArxpayx saMmoVhamiva yAtayxtha \hfill || 8 ||
  
\end{shl}

\begin{artha}
yAva Atamxnu ajAcnxniyAgiruvanoV avaneV ililx parxsutxtanAgidudx yAvAga
I shariVravanunx pUvoVRkatxvAda mupupx roVga muMtAda kAraNagaLiMda
kaqshavAguvaMte mADi athavA tAnAgiye deVhatAdAtamxyXviruva kAraNadiMda
kaqshateyanunx hoMdiyo dubaRlateyanunx I riVtiyAgihoMdi mUCeRyanunx
(ajAcnxnavanunx) A kUDaleV hoMduvano,  (AvAga vAgAdi iMdirxyagaLu eduru
baruvudu).
\end{artha}

\begin{shl}
mUCARdAviva saMmoVhamihApAyxtAmx nigacaCxti || \\
boVdhamAterxYkayAthAtAmxyXnAnxyaM saMmoVhaBAgayxtaH ||  \\
saMmUDhabudidhxsAkiSxtAvxtasxMmUDha iva BAtayxtaH \hfill || 9 ||
  
\end{shl}

\begin{artha}
mUCeR modalAda avasethxyalilxruvaMte I maraNakAladalUlx Atamxnu
ajAcnxnavanunx hoMduvanu. Adare I Atamxnu jAcnxna mAtarxvoMdeV
nijavAda savxrUpavuLaLxvanAdadxriMda ajAcnxnakekx nijavAgi BAgiyalalx,
mUDhabudidhxge sAkiSxyAdadxriMda mUDhanAgiruvaMte toVribaruvanu. aSeTx.
\end{artha}

\vishaya{Atamxnalilx nijavAgi ajAcnxna leVpavilalxveMbudakekx matotxMdu kAraNa}

\begin{shl}
saMmoVhaheVtukaM kAyaRM saMmoVhAtamxkamiSayxteV ||  \\
akAyaRkAraNaM parxtayxgoVjxyXtiH saMmuhayxteV kutaH \hfill || 10 ||
  
\end{shl}

\begin{artha}
ajAcnxna nimitatxdiMda AdakAyaRvu ajAcnxna savxrUpavuLaLxdedxMdu
opipxde, kAyaRkAraNagaLeraDU alalxda parxtayxgAtamx joyxVtiyu heVge
mUDhavAguvudu ? (ilalx)
\end{artha}

\begin{shl}
kAshayxRsaMmoVhasaMbanodhxV yatoV nAsayx savxtasatxtaH || \\
moVhoVtathxdeVhabudAdhxyXdisaMgateVrivagiVriyamf \hfill || 11 ||
 
\end{shl}

\begin{artha}
kaqshate, ajAcnxna ivugaLa saMbaMdhavu savxtaH Atamxnige ilalx,
adariMda ajAcnxnadiMda huTiTxkoMDa deVha, budidhx modalAdavugaLa
saMbaMdhadiMdale eMbudakAkxgi I ivashabadxvu (parxyukatxvAgide).
\end{artha}

\vishaya{saMmoVha padakekx matotxMdu athaRvide -}

\begin{shl}
utAkxrXnitxkAleV pArxNAnAM savxsAthxnAdADhayxRheVtukA || \\
savxgoVcareVSavxshakitxyAR saMmoVhoV\s sAvihA\s \s tamxnaH \hfill || 12 ||
  
\end{shl}

\begin{artha}
utAkxrXMtiyAguva kAladalilx iMdirxyagaLige tananx sAthxnavu
daqDhavAgilalxdiruva nimitatxvAgi tananx viSayagaLalilx (aMdare
shabadx sapxshaRdi viSayagaLa jAcnxnavanunxMTu mADikoDuva
viSayadalilx) yAva ashakitxyuMTAguvudo, adeV Atamxnige ililx
saMmoVhavenisuvudu.
\end{artha}

\vishaya{iMdirxyagaLa ashakitx atamxnige heVge ?}

\begin{shl}
dAhaceCxVdAdayoV yadavxdadAhAyxceCxVdayxtanivxnaH || \\
paroVpAdhinimitAtxH suyxduRbaRlatAvxdayasatxthA \hfill || 13 ||
  
\end{shl}

\begin{artha}
suDuvudu, katatxrisuvudu itAyxdi dhamaRgaLu yAva riVtiyAgi
suDuvudakekxbArada matutx katatxrisalAgada sUkaSxmX savxrUpavAda
Atamxnige matotxMdu upAdhiya nimitatxdiMda baMdiveyo, hAgeye dubaRlate
muMtAda dhamaRgaLu.
\end{artha}

\vishaya{maraNakAladalilx horagina sithxti -}

\begin{shl}
bAhAyx tAvadiyaM vaqtitxvAyxRKAyxtA parxtayxgAtamxnaH || \\
kAshayxRsaMmoVharUpA yA parxsidAdhx jagatiVdaqshiV \hfill || 14 ||
  
\end{shl}

\begin{artha}
yAvudu kaqshate, ajAcnxna ivugaLa rUpadalilx parxtayxgAtamxna
sithxtiyu I bageyAgi loVkadalilx parxsidadhxvAgididxto, adu
bAhayxsithxtiyeMdu vAyxKAyxna mADalapxTiTxtu.
\end{artha}

\vishaya{maraNakAladalilx oLagina sithxti -}

\begin{shl}
mariSayxtoV\s sayx yA vaqtitxrAnatxriV sA\s dhunoVcayxteV ||  \\
haqtasxdamxnuyxpasaMhAroV yathA sAyxdinidxrXyAtamxnAmf \hfill || 15 ||
  
\end{shl}

\begin{artha}
sAyuva I jiVvana oLagina sithxtiyu yAvudidiyo adanunx Iga heVLuvudu,
adeVneMdare - iMdirxya savxrUpagaLige haqdaya maMdiradalilx
upasaMhAravAguvudu heVgo hAge heVLalapxDuvudu.
\end{artha}

\vishaya{atha itAyxdi padagaLa vAyxKAyxnavu -}

\begin{shl}
atheYnamucicxkirxmiSu vijAcnxnAtAmxnamiVshavxramf || \\
pArxNA vAgAdayaH saveVR taM samAyanitx kaqtasxnXtaH \hfill || 16 ||
  
\end{shl}

\begin{artha}
I maraNakAladalilx I vAgAdi iMdirxyagaLu elalxvU utAkxrxMti hoMdalu
edurAda jiVvAtamxneMba sAvxmiya hatitxra saMpUNaRvAgi eduru EkavAgi
oTuTxgUDi barutatxve.
\end{artha}

\vishaya{`aBisamAyanitx' eMbudara vAyxKAyxna}

\begin{shl}
aBiVti cA\s \s BimuKeyxV\s theVR saM tu sAmasatxyX iSayxteV ||  \\
avadhayxtheVR tathA\s \s knatarx yanitxVtayxsayx visheVSaNamf \hfill || 17 ||
 
\end{shl}

\begin{artha}
aBi eMbudu ABimuKayx (eduru) eMbathaRdalilxde, samf eMba upasagaRvu
samasatx eMbathaRdalilx saMmata, Ajf eMba upa sagaRvu
avadhiyeMbathaRdalilx, hAgU `yanitx' eMbuva dhAtuvige visheVSaNavAguvavu.
\end{artha}

\begin{shl}
sAvxsharxyeVBoyxV yiyAsanitx karaNAni haqdiVshavxramf || \\
yadeYSAM sAthxnasaMbanadhxvimoVkaSxH sAyxtatxdeYva tu \hfill || 18 ||
  
\end{shl}

\begin{artha}
yAvAga tananx Asharxya sAthxnagaLiMda iMdirxyagaLu sAvxmiyidedxDege
iciCxsutatxveyo, AvAgaleV ivugaLige sAthxna saMbaMdhavu cuyxtiyAguvudu.
\end{artha}

\begin{shl}
kathaM tamaBisaMyAnitxVtuyxketxV shurxtAyx\s BidhiVyateV || \\
AtAmxnamaBisaMyAnitx vAgAdiVni yathA suPxTamf \hfill || 19 ||
 
\end{shl}

\begin{artha}
avu heVge avana eduru barutatxve ? eMdu keVLidare shurxtiyu vAgAdi
iMdirxyagaLu Atamxnalilxge heVge baMdu seVrutatxve eMbudanunx
sapxSaTxvAgi heVLuvudu.
\end{artha}

\vishaya{``sa E tA setxVjoVmAtArxH samaBAyxdadhAnaH'' idara vAyxKAyxna}%%% 51

\begin{shl}
sa AtAmx parxkaqtasetxvXVtAshacxkuSxHshocxVtArxdilakaSxNaH ||  \\
teVjoVmAtArx yathAdeVshaM samayeV maqtikamaRNaH \hfill || 20 ||
  
\end{shl}

\begin{artha}
saH eMdare parxsAtxpisida Atamxnu cakuSx, shorxVtarx muMtAda
rUpavuLaLx teVjoV mAterxgaLanunx tananxdeVshadiMda
maraNavAguvakAladalilx tegedukoMda haqdaya sAthxnakekx baruvanu.
\end{artha}

\begin{shl}
udUBxtAkUtavijAcnxnoV maqtiM parxti yadA tadA || \\
AkUtAnuvidhAyiVni jAyanetxV karaNAnayxtha \hfill || 21 ||
  
\end{shl}

\begin{artha}
maraNa viSayadalilx horapaTaTx aBipArxya (sUcane koDuva)
jAcnxnavuLaLxvanAgi yAvAga iruvano AvAga iMdirxyagaLu
aBipArxyavanunx anusarisi irutatxve.
\end{artha}

\vishaya{IvAga `samaBAyxdadAna' eMbudalilx shAnacf parxtayxyada athaRvanunx heVLutAtxre.}

\begin{shl}
sAvxkUtAnuvidhAyitavxM yatatxdA karaNAtamxnAmf || \\
aBAyxdadAna iti tatakxtaqRtavxM sAyxdihA\s \s tamxnaH \hfill || 22 ||
  
\end{shl}

\begin{artha}
(mumUGaRvAda) Atamxna aBipArxyavanunx anusarisi naDeyuvadeMbudeMbudu
iMdirxya savxrUpagaLige yAvuduMTU, adeV kataqRtavxvu ililx Atamxnige
`aBAyxdadAnaH' eMdu heVLidudx.
\end{artha}

\begin{shl}
EtatakxtaqRtavxmApeVkaSxyX shurxteyxYvamaBidhiVyateV || \\
aBAyxdadAna iti tu teVjoVmAtArxH savxdeVshataH \hfill || 23 ||
  
\end{shl}

\begin{artha}
tananx sAthxnadiMda teVjoVmAtarxgaLanunx (iMdirxyagaLanunx) (oMdanunx
biDade tegedukoLuLxvaneMdu) gwNavAda kataqRtavxvanenxV shurxtiyeV
heVLiruvudu.
\end{artha}

\vishaya{`teVjoVmAtArxH' eMbudakekx athaR}

\begin{shl}
miVyanetxV viSayA yABimARtArxsAtxshacxkuSxrAdayaH || \\
teVjoVvikaqtiheVtutAvxtetxVjoVmAtArxshacx tAH samxqqtAH \hfill || 24 ||
  
\end{shl}

\begin{artha}
yAvudariMda viSayagaLu tiLiyutatxveyo avugaLu cakuSx modalAda
iMdirxyagaLu, teVjasesxMba vikAravanenxV nimitatx mADikoMDiruvudariMda
teVjoV mAtarxveMdu avugaLu heVLalapxTiTxve.
\end{artha}

\vishaya{teVjaH shabadxda athaR, (sAMKayxmatadaMte) -}

\footnotetext[1]{satavxguNavu ahaMkAra rUpadalilxdudx shabadxmodalAda
viSayagaLanunx parxkAshagoLisuvudariMda karaNa rUpadiMda (iMdirxya
rUpadiMda) pariNAmavanunx hoMduvudu. `sAtivxka EkAdashakaH parxvataRteV veYkaqtAdahaMkArAtf' eMdu heVLidadxriMda
sAMKayxmatadaMte ililx vAyxKAyxna mADide.}
\begin{shl}
\footnotemark[1]satatxvXM teVjoV\s tarx vijecnxVyaM tadeVva karaNAtamxnA || \\
parxviBakatxM hi tacaCxbadxsapxshARdayxthARvaBAsanAtf \hfill || 25 ||
  
\end{shl}

%%% footnote shloka {1}
\begin{artha}
ililx teVjasusx eMdare satavxguNa, adeV iMdirxya
rUpadalilx viMgaDavAgide, shabadx sapxshaRmodalAda athaRvanunx
parxkAsha goLisuvudariMda (hiVgeMdu tiLiyalapxDuvadu).
\end{artha}

\footnotetext[1]{``aginxreVva shariVreV pitAtxnatxgaRtaH kupitAkupitAni shuBAshuBAni karoVti'' eMdu heVLide pitatxdoVSadoLagiruva teVjasesxV shariVradalilx parxkoVpa, parxkoVpavilalxdiruvudu oLeyadu keTaTxdudx
ivugaLanunx mADuvudeMdu athaR. matutx AmAshayadalilxruva pitatxvu rasavanunx baNaNxvAgi
mADuvudariMda raMjakaveMdU, budidhx, meVdhA shakitx, aBimAna muMtAda
vaqtitxgaLiMda iSaTxvAda viSayavanunx sAdhisikoDuvudariMda
sAdhakaveMdU, haqdayalilxdadxpitatxvu rUpAloVcaneyiMda
kaNiNxnalilxdudx AloVcakaveMdU tavxkikxnalilxdudx camaRvanunx
hoLeyuvaMte mADuvudariMda BArxjakaveMdu heVLalapxDuvudu. hiVgeMdu
matAMtaravu TiVkeyalilxde.}
\begin{shl}
\footnotemark[1]pitAtxKAyxM vA BaveVtetxVjasatxdaMshAshacxkuSxrAdayaH ||  \\
iteyxVvamAyuveVRdajAcnxH karaNAni parxcakaSxteV \hfill || 26 ||

\end{shl}

%%% footnote shloka {1}
\begin{artha}
pitatxveMbahesuruLaLx vasutxveV teVjasusx, adara aMshagaLeV cakuSx
modalAda iMdirxyagaLeMdu I riVtiyAgi AyuveRVdavanunx tiLidavaru
iMdirxyagaLeMdu heVLutAtxre.
\end{artha}

\vishaya{sidAdhxMtadalilx iMdirxyagaLu parxkAsha rUpavAdadudx}

\begin{shl}
yadA pacnAcxvatiSaThxnetxV jAcnxnAni manasA saha || \\
iti parxkAsharUpatavxM karaNAnAM shurxtijaRgw \hfill || 27 ||
  
\end{shl}

\begin{artha}
yAvAga aidu jAcnxna parxkAshavAda iMdirxyagaLu manasisxnoDane
irutatxveyo eMdu shurxtiyu iMdirxyagaLige parxkAsha rUpavanunx
heVLiruvudu.
\end{artha}

\footnotetext[2]{``tadAhu BaRtaqRparxpaMcAH'' eMdu BataqR parxpaMcara matavanunx meVlina
viSayakekx AnaMda girigaLu udAharisi - punaH AyuveRVdavanunx tiLida
paMDitaru heVLidadxnunx ililx udAharisidAdxre - 
``EkeYkAdhikayukAtxni KAdiVnA miMdirxyANi ca |" iti \\
KAdiVni budidhxravayxkatx mahaMkAra satxthA\s  SaTxmaH |\\
BUta parxkaqti rudidxSATx vikArAH SoVDasheYvatu ||\\
budidhxVnidxrXyANi paMceYva paMcakameVRMdirxyANi ca |\\
samanasAkx shacx pacnacxthARH vikArA iti saMjicnxtAH ||\\
idariMda oTiTxnalilx tiLiyuvudeVneMdare teVjoVmAtArxH eMbuvalilx
teVjaH padakekx paMcaBUtagaLU athaR, avugaLakAyaR iMdirxyagaLa samUha,
avu parxkAshaka BwtikavAda viSayagaLanunx parxkAsha paDisuvudariMda
beLakinaMte parxkAshaveMdu sidAdhxMtavAguvudu. sAMKayx sidAdhxMtada
athaRvAgali, itare matavAgali saMmatavAguvudilalxveMdu tiLiyabeVku.
AyuveRVda sidAdhxMtadaMte shurxti sidAdhxMtavu iMdirxyagaLelalxvU
BwtikaveMdeV barxhamxsUtarxkArariMdalU CAMdoVgayx upaniSatitxnalilx
heVLidaMte tiVmARnisalapxTiTxde. `ananxmayaM hi swmayxmanaH ApoVmayaH pArxNaH teVjoV mayiV vAkf' eMdu CAMdoVgayxdalilx
manasusx paqthiviVtatavxda vikAra, pArxNa jalada vikAra, vAgiMdhiya
teVjasisxna vikAraveMdu tiLiside. (CAM - SaSAThxdhAyxya)}
\begin{shl}
Bwtikasutx parxkAshoV\s yaM BwtikAthaRparxkAshanAtf || \\
parxdiVpavananx BUteVBoyxV jAtayxnatxramatoV BaveVtf \hfill || 28 ||
  
\end{shl}

%%%% footnote shloka {2}
\begin{artha}
paMcaBUtagaLiMda Ada viSayagaLanunx
parxkAshapaDisuvudariMda BUtagaLa pariNAmavAda iMdirxyagaLu diVpadaMte
parxkAsha rUpavAgiruvavu, adariMda paMcaBUtagaLigiMta beVre jAtiyavu alalx.
\end{artha}

\vishaya{iMdirxyagaLu BwtikaveMbudakekx vAkayx sheVSavU hoMdutatxde.}

\begin{shl}
sapxSaTxM ca vakaSxyXteV\s thoVdhavxRmeVteVBayx iti hi shurxtiH || \\
akASxNi BwtikAneyxVva nAtaH shakatxya AtamxnaH \hfill || 29 ||
  
\end{shl}

\begin{artha}
alalxde ``EteVBoyxV BUteVBayxH samutAthxya'' eMbuva muMdina vAkayxsheVSadalilx sapxSaTxvAgi
shurxtiyu heVLiruvudu, adariMda iMdirxyagaLu BwtikaveV,
Atamxna shakitxgaLeMbuva (matAMtaravU) sariyalalx.
\end{artha}

\vishaya{iMdirxyagaLu BwtikaveMdu meVlina vAkayxsheVSadalilx
tiLiyuvudilalx, heVge adariMda BwtikaveMdu tiVmARnisuvudu ?}

\begin{shl}
liknAgxtamxkAnAM BUtAnA nideVRshoV\s vidayxyA saha || \\
tanAnxshamanu nAshaH sAyxdayxtoV duHKAtamxnasatxtaH \hfill || 30 ||
  
\end{shl}

\begin{artha}
liMga shariVra rUpavAda paMcaBUtagaLanunx avideyxyoMdige seVrisi
`BUteVBayx' eMdu nideRVshamADide, kAraNaveVneMdare - avugaLa
nAshavanunx anusarisi duHKamayanAda (jiVvanigU) nAshavu ideyeMdu
yAvakAraNadiMda shurxtavAgideyo (AkAraNadiMda)
\end{artha}

\vishaya{sUthxla deVhavu nAshavAdarU jiVvanige nAshavu iruvudariMda
sUthxladeVhavanenxV `BUteVBayxH' eMbuvalilx heVLideyeMdu Eke
heVLabAradu ? eMdare -}

\begin{shl}
piNaDxnAsheV\s pi neYvAsayx nAshaH saMsAriNoV yataH || \\
\footnotemark[1]avidAyxdiVni BUtAni tatorxVcayxnetxV tatoV dhurxvamf \hfill || 31 ||
  
\end{shl}
\footnotetext[1]{avidAyxsameVtavAda hadineVLu padAthaRgaLu
paMcajAcnxneVMdirxyagaLu, paMcakamoRVMdirxyagaLu budidhx manasusx
ivugaLu seVri hadineVLU saha paMcaBUtagaLa pariNAmagaLu adariMda
`BUteVBayxH' eMbududariMda sUthxlashariVra matutx (sUkaSxmX
shariVra) iMdirxyagaLU heVLalapxDuvudu. jiVvAtamxnige aupAdhikavAda
nAshavanenxV atAtivxkavAgi heVLideyaSeTx, jiVvAtamxnige nijavAgi
savxrUpataH nAshavilalxveMbudanunx ariyabeVku.}

%%% footnote shloka {1}
\begin{artha}
sUthxlashariVravu nAshavAdarU I jiVvanige nijavAgi
nAshavilalxveMbuva kAraNadiMda avidAyxdiyAgi paMca BUtagaLeV
`BUteVBayxH' eMbudalilx heVLalapxTiTxve eMbudu nishicxtavu.
\end{artha}
