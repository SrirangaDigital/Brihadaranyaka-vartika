\section*{jiVvana saMsAravaNaRne}

\section*{(baq. 4 ne adhAyxya. bArx. 4. kaMDike 1)}

\section*{vAtiRka 32 riMda ---}

\vishaya{iMdirxyagaLu BwtikaveMbudakekx vAkayx sheVSadalilxruva
liMgaveV gamaka -}

\footnotetext[1]{savARvataH = BUta BwtikagaLa avayagaLeV savARshabAdxthaR
`savAR vA BUta BwtikamAtArxH asayx saMsagaRkAraNaBUtA vidayxnatx iti savARvAnf' eMdu BASayx,}
\begin{shl}
\footnotemark[1]mAtArxsaMsagaR EvAsayx tathAceYva parxvakaSxyXteV || \\
vijAcnxneVnAtha vijAcnxnamAdAyeVtayxpi cAvadatf ||  32 || 
\end{shl}

%%% footnote shloka
\begin{artha}
`mAtArx\s naMsagaRsatxvXsayxBavati' eMdu muMde
heVLalapxDuvudu, vijAcnxneVna vijAcnxna mAdAya eMdU heVLiruvudu
(gamaka)
\end{artha}

\begin{artha}
`asayx loVkasayx savARvatoV mAtArx ma pAdAya parxsivxpiti shukarx mAdAya punareYti sAthxnamf' eMbudAgi elAlx sathxLagaLalU keVvala BUta mAtarxgaLanunx
tegedukoMDiruvudu.
\end{artha}

\begin{shl}
asayx loVkasayx ceVtuyxkatxM shukarxmitAyxdi cAparamf || \\
saveVRSevxVSu parxdeVsheVSu BUtamAtArxgarxhaH shurxtw ||  33 ||  
\end{shl}

\footnotetext[2]{Atamxnige beVreyAgiruva kANuva vasutx eMdare BUta
BwtikagaLu, ivanunx biTuTx beVreyilalx, BUtagaLigU AtamxnigU
beVreyAdadedxMdare BUtavikAraveV AgabeVku, adu BUta mAtarxve
AguvudeMdu elAlx kaDeyalUlx `BUtamAtArx' eMdu tegedukoMDide,
vAtiRkadalilx BUtagaLa viveVkaveMdare ililx atayxMta parxkAsha
rUpavAda pariNAma iMdirxya samudAya, adeV teVjoVmAtArx eMdu
ukatxvAgide (AvaM-TiVke).}
\begin{shl}
\footnotemark[2]BUteVBoyxV nAparaM vasutx yasAmxdAtamxna IkaSxyXteV ||  34 || \\
atoV viveVkoV BUtAnAM yaH paroV\s tiVva shudidhxtaH || \\
teVjoVmAtArxdivacasA sa EvAtArxBidhiVyateV ||  35 ||  
\end{shl}

%%% footnote shloka
\begin{artha}
yAvakAraNadiMda Atamxnige paMcaBUtagaLigiMta matotxMdu vasutx iruvudu
kANuvudilalxvo, adariMda matutx paMcaBUtagaLa viveVkavU
bahushoVdhakavAdudariMda muKayxvAgideyoV AviveVkaveV ililx
`teVjoVmAtArxH' eMba vacanadiMda heVLalapxDuvudu.
\end{artha}

\vishaya{viSayagaLU iMdirxyagaLU elalxvU Atamxna vikAra eMbuvara
matavanunx nirAkarisuvudu -}

\begin{shl}
inidxrXyANiVnidxrXyAthARshacx na vikAraH parAtamxnaH || \\
atoV na jAyata iti tadivxkAraniSeVdhataH ||  36 ||  
\end{shl}

\begin{artha}
iMdirxyagaLU, iMdirxya viSayagaLU kUDa paramAtamxna vikAravalalx,
adariMdaleV `na jAyata surxyateVvA' eMdu shurxtiyalilx Atamxna
vikAravanunx niSeVdhisiruvudu.
\end{artha}

\begin{shl}
janAmxdivikirxyASaTakxM sAkASxnanx paramAtamxnaH || \\
apUvARnaparAduyxketxVneVRti neVtAyxdivAkayxtaH ||  37 ||  
\end{shl}

\begin{artha}
janamxmodalAda vikAragaLu ArU saha paramAtamxnige neVra iruvudilalx,
neVti neVti itAyxdi vAkayxdiMdalU `apUvaR manaparamf'
itAyxdivacanadiMdalU hiVgeMdu tiLiyuvudu.
\end{artha}

\vishaya{Adare paramAthaRvasutxvige beVre vasutx nijavAgilalx -}

\begin{shl}
na ca veVdAnatxsidAdhxnetxV paramAtAmxtireVkataH || \\
iSaTxM vikAravadavxsutx yathA kApilashAsaneV ||  38 || 
\end{shl}

\begin{artha}
veVdAnatx sidAdhxMtadalilx paramAtamxnige beVreyAgi iSaTxvAga
vikAravuLaLx vasutx kapilashAsatxrXdalilx (sAMKayxshAsatxrXdalilx) idadxMte ilalx.
\end{artha}

\vishaya{kUTasathx niviRkAra eMdAdare eraDaneyadu ilalxvAdare
kAraNavilalxde jagatutx baMditeMdu AguvudaSeTx ? eMdu keVLidare -}

\begin{shl}
savxtaH kUTasathxtatatxvXsayx tadasaMboVdhatasatxtaH || \\
janAmxdivikirxyASaTakxsaMgatiH sAyxtapxrAtamxnaH ||  39 ||  
\end{shl}

\begin{artha}
savxtaH niviRkAravAda tatavxkekx paramAtamxnige adara ajAcnxnadiMda
janAmxdi AruvikAragaLu saMbaMdhisuvavu aSeTx.
\end{artha}

\begin{shl}
AtamxkAraNavAdoV\s yameVvaM satuyxpapadayxteV || \\
na tu vidhavxsatxniHsheVSajanamxnAshAdikAraNeV ||  40 ||  
\end{shl}

\begin{artha}
hiVge avidAyxnimitatxvAgiye AtamxkAraNaveMbavAdavu hoMduvadu anayxthA
Adare samasatx janamx, nAshAdikAraNagaLeV ilalxda vasutxvinalilx
hoMduvudilalx.
\end{artha}

\vishaya{iMdirxyagaLu BwtikaveMbavAdada upasaMhAra -}

\begin{shl}
teVjoV\s toV BwtikaM savaRmanayxtarx paramAtamxnaH || \\
savxyaMjoyxVtiHparxsaknegxVna tadukatxM pArxgapi shurxtw ||  41 ||  
\end{shl}

\begin{artha}
adariMda paramAtamxnigiMta beVreyAda teVjasusx elalxvU Bwtika, adanunx
hiMdeye savxyaM joyxVti vicAra mADuva parxsaMgadalilx shurxtiyalilx
heVLiye ide.
\end{artha}

\begin{shl}
AditAyxdiVni teVjAMsi tathA\s dhAyxtAmxdhiBUtayoVH || \\
BwtikAneyxVya tAniVti parxtayxknf teVBoyxV vilakaSxNaH ||  42 ||  
\end{shl}

\begin{artha}
Aditayx modalAda teVjasusxgaLu hAgU adhAyxtamx adhiBUtagaLalilx
teVjasusxgaLU BwtikaveV eMdu heVLidAdxyitu avugaLige
vilakaSxNanAdavanu parxtayxgAtamxneMdu (Iga Palisitu).
\end{artha}

\begin{shl}
tA EtAsetxVjasoV mAtArxH samasAtxshAcx\s \s BimuKayxtaH || \\
sAmasetxyXVnA\s \s dadAnaH sanahxqqtasxdamxnuyxpasapaRti ||  43 ||  
\end{shl}

\begin{artha}
A elAlx teVjasisxna avayavagaLanunx (iMdirxyagaLanunx) edurAgi neVra
saMpUNaRvAgi tegedukoLuLxtAtx aMdare savxlapxvU vAsaneyU uLiyadaMte
saMharisuvavanAgi haqdayada manege baMdu seVruvanu.
\end{artha}

\vishaya{iMdirxyagaLanunx tegedukoLuLxvudakUkx haqdayasathxLavanunx
akarxmisuvudakukx pwvARpayaRkAlavuMTe ? eMdare - ilalx -}

\begin{shl}
aBAyxdadAna EvAyamanavxvakArxmatiVshavxraH || \\
na tAvxkarxmayx samAdatetxV kathaM taditi BaNayxteV ||  44 ||  
\end{shl}

\begin{artha}
I (shariVra) sAvxmiyAda jiVvanu iMdirxyagaLanunx tegudukoLuLxtatxleV
(haqdayavanunx) Akarxmisuvanu, Adare modalu Akarxmisi iMdirxyagaLanunx
sivxVkarisuvudilalx, adu heVgeMbudanunx heVLuvudu.
\end{artha}

\vishaya{idanenx visatxrisiheVLuvudu -}

\begin{shl}
visheVSajAcnxnalABoV\s sayx liknAgxtAmxnuvidhAnataH || \\
yatasatxdapakaSeVRNa vijAcnxnAtomxVpasaMhaqtiH ||  45 ||  
\end{shl}

\begin{shl}
tadAvxyXpwtx cApi saMvAyxpitxrApAdatalamasatxkamf || \\
savxtasutx vAyxpitxsaMhArarahitatAvxtapxrAtamxnaH ||  46 ||  
\end{shl}

\begin{artha}
Itanige liMga shariVravanunx anusarisuvudariMda visheVSajAcnxna lABavu
Aguvudu, adanunx eLedukoMDidadxriMda jiVvAtamxnige upasaMhAravu, adara
liMgashariVrada vAyxpaneyu (deVhadalilx) Aguvalilx kAliniMda hiDidu
taleya payaRMta AtamxnavAyxpane yAguvudu, EkeMdare ! savxtaH
paramAtamxnige I riVtiyAda vAyxpane, upasaMhAragaLu ilalxvAdudariMda
(aupAdhikavAgiyeV ideyeMdu tiLiyabeVku).
\end{artha}

\vishaya{liMgashariVravanunx anusarisuva bage -}

\begin{shl}
seYnadhxvAdirasavAyxpetxV piVyamAneV yathoVdakeV || \\
pAnaM salavaNaseyxYvaM liknAknxtamxnuvidhAyitA ||  47 ||  
\end{shl}

\begin{artha}
seYnadhxva modalAda rasadiMda kUDi niVranunx kuDiyutitxdadxre upupx
sameVta niVranunx kuDiyuvudu heVgo, hAgeye liMgashariVravanunx
anusarisi naDeyuvudu (aMdare liMga shariVropAdhiyiMda kUDida Atamxnige
tadanu sAriyAda naDeyuvike).
\end{artha}

\vishaya{iMdirxya sivxVkaraNa, haqdaya parxveVsha kAyaRgaLanunx mADuva
kataRnige iMdirxyAdigaLa BeVdavirabeVkaSeTx ? eMdare -}

\begin{shl}
soV\s BAyxdAnasayx katAR\s tarx liknagxM yoV\s simxVti manayxteV || \\
AkArxmadadhxqqdayaM liknagxmanavxvakArxmatiVva saH ||  48 ||  
\end{shl}

\begin{artha}
ililx liMga shariVravanunx yAru nAneV eMdu aBimAnisuvanoV avaneV
iMdirxyagaLanunx tegedukoLuLxva kataRnAguvanu. avanu haqdayavanunx
Akarxmisutitxruva liMga shariVravanunx anusarisi A AtamxnU
AkarxmisuvavanaMte toVruvanu.
\end{artha}

\vishaya{`haqdaya meVva' eMbuvalilx haqdaya padAthaR -}

\begin{shl}
tathA haqdayashabedxVna tatAsxthX dhiVraBidhiVyateV || \\
EveVtayxvadhaqtishAcxmarx savxpanx\footnotemark[1]pArxjacnxnivaqtatxyeV ||  49 ||  
\end{shl}
\footnotetext[1]{pArxjacnx eMdare ililx suSupitx, yeMdathaR}

%%% footnote shloka
\begin{artha}
hAgU haqdaya padadiMda haqdayadalilxruva budidhxyu heVLalapxDuvudu. `Eva eMba avadhAraNavu savxpanx matutx suSupitxyanunx beVpaRDisuvudakAkxgi.
\end{artha}

\vishaya{adu heVge ? eMdare.}

\begin{shl}
visheVSakAyaRvAhiBayxH sorxVtoVBayxH savxpanxBUmigaH || \\
visheVSAnanusaqpAyx\s \s tAmx yAtAvx haqdayamAsharxyamf ||  50 || 
\end{shl}

\begin{shl}
puriVtatapxrXmuKaM deVhaM vaqtAtxyX sAmAnayxrUpayA || \\
tapatxloVhAdivadAvxyXpayx sheVteV pArxNAtamxtAM gataH ||  51 ||  
\end{shl}

\begin{artha}
savxpanx sAthxnadalilxruva Atamxnu (vAsanArUpavAda visheVSa
kAyaRgaLanunx harisuva nADigaLa mUlaka vAsanArUpavAda visheVSagaLanunx
anuBavisi naMtara AsharxyavAda haqdayavanunx hoMdi, puriVtatatxnunx
muKayxmADikoMDu deVhavanenxV jAcnxna sAmAnayx rUpavAda ceYtanAyxBAsa
vaqtitxyiMda kAda kabibxNadalilx beMkiyu vAyxpisuvaMte vAyxpisi pArxNa
savxrUpavanunx hoMdi malagiruvanu.)
\end{artha}

\begin{shl}
iha tevxVveVti niHsheVSA liknagxsAyxsoyxVpasaMhaqtiH || \\
visheVSeVBoyxV\s visheVSeVBoyxV yukatxM parxtayxvasapaRNamf ||  52 ||  
\end{shl}

\begin{artha}
`ihatevxVva' eMdu liMga shariVrakekx savxlapxvU uLiyadaMte
upasaMhAravAguvudu, adariMda visheVSa viSayagaLiMdalU sAmAnayx
viSayagaLiMdalU hiMtiruguvudu yukatx.
\end{artha}

\vishaya{EvakArada athaR}

\begin{shl}
avadadherxV\s ta EveVti mA BUtasxvXpAnxdivatisxthXtiH || \\
sAmAnayxM vA visheVSoV vA na yatoV\s tArxvashiSayxteV || \\
kAtesxnXyXVRna haqdaya Eva piNiDxVBAvaM varxjatayxtaH ||  53 ||  
\end{shl}

\begin{artha}
adariMdale Eva eMbudu savxpAnxdigaLaMte iruva sithxtiyalalxveMdu
avadhAraNa mADiruvudu. I (maraNa dasheyalilx) sAmAnayxvAgali
visheVSavAgali (oMdu vaqtitxyU) uLiyuvudilalx, adariMda haqdayadalelxV
GaniVBAvavanunx hoMduvanu.
\end{artha}

\vishaya{``sayaterxYSa cAkuSxSaH puruSaH'' eMbudara tAtapxyaR -}

\begin{shl}
vidhirAkaSaRNoV tAvatetxVjoVmAtArxsharxyoV gataH || \\
athAsaMveVdanAvidhimuRmUSoVRrucayxteV yathA ||  54 ||  
\end{shl}

\begin{artha}
teVjoVmAtarxgaLa viSayada AkaSaRNe viSayadalilx heVLabeVkAda
parxkAravu mugiyitu, Iga mumUSuRvige jAcnxnavAgadiruva parxkAshavanunx heVLuvudu.
\end{artha}

\vishaya{shurxtiyalilxruva cAkuSxSa padAthaR -}

\begin{shl}
sAdhAraNAtamxnoV yoVM\s shoV BoVkatxqqkamaRvashiVkaqtaH || \\
adhAyxtamxM cakuSxSi raveVshacxkuSxvAyxRpArasididhxkaqtf ||  55 ||  
\end{shl}

\begin{shl}
kamoVRpaBoVgasidadhxyXthaRM pariciCxnanxH savxcakuSxSA || \\
\footnotemark[1]cAkuSxSoV\s taH sa vijecnxVyaH sAmAnAyxtAmx\s pi BAsakxraH ||  56 ||  
\end{shl}
\footnotetext[1]{``Aditayx shacxkuSx BURtAvx akiSxNiV pArxvishatf'' eMba aitareVya udAharaNeyiMda}

%%% footnote shloka
\begin{artha}
elAlx kaDeyalUlx sAmAnayxvAgiruva sUyaRna yAva aMshavideyo
adu BoVkatxnAda jiVvana kamaRkekx adhiVnavAgi shariVradoLage kaNiNxna
sAthxnadalilx cakuSxriMdirxya vAyxpAravanunx sAdhisikoDuvudu, adariMda
savaRsAdharaNanAgidadxrU sUyaRnu jiVvana kamaR PalaBoVga sAdhanegAgi
tananx kaNiNxniMda pariciCxnanxnAgi; (BAgeYsidaMte idadxvanAgi)
cAkuSxSaneMdu avananunx ililx tiLiyabeVku.
\end{artha}

\vishaya{uLida shurxtiya vAkayxkekx athaR -}

\begin{shl}
sa ESa puruSoV yatarx tatapxrXyoVkatxqqkirxyAkaSxyeV || \\
keSxVtarxjAcnxthARtapxrAkneVva payARvataRta Atamxni ||  57 ||  
\end{shl}

\begin{artha}
I puruSanu (aMdare cakuSxriMdirxyakekx anugarxha mADuva sUyaRna aMshavAdavanu)
yAva kAladalilx A BoVkatxqqvina kamaRkaSxyavAguvudo
AkAladalilx BoVkatxqqvina BoVgayx vasutxviniMda vimuKanAgiye
Atamxnalilx (aMdare tananx aMshiyAda sUyaR deVvateyalilx) hiMtirugi
nilulxvanu.
\end{artha}

\vishaya{payARvataRne eMbudara vAyxKAyxnavu -}

\begin{shl}
niranugarxhateYvAsayx payARvataRnamucayxteV ||  58 ||  \\
haqlilxknagxparxsaqtoV yoVM\s sha upaBoVgAthaRmakiSxNi || \\
soV\s sayx kamaRkaSxyAdakaSxNXH parAknAvataRteV haqdi ||  59 ||  
\end{shl}

\begin{artha}
Itanige payARvataRnaveMdare anugarxha mADadiruvudeV (beVreyalalx)
haqdayaveMba liMga shariVradiMda parxsAravAda yAva aMshavideVyo adu
cakuSxsAthxnadalilxdadx jiVvana upaBoVgakAkxgiruvudu, adu kamaR kaSxyadiMda
cakuSxsAthxnavanunx biTuTx hiMtirugi haqdayasAthxnadalilx nilulxvudu.
\end{artha}

\vishaya{`sayatarx' - itAyxdi maMtArxthaRda upasaMhAravu -}

\begin{shl}
cAkuSxSeV puruSeV tasimxnAnxvaqtetxV kamaRNaH kaSxyAtf || \\
deVvatA deVvatAmeVti liknegxYkayxM karaNaM tathA ||  60 ||  
\end{shl}

\begin{artha}
kamaRvu kaSxyisidadxriMda A cAkuSxSa puruSanu hiMtirugalu aBimAni
deVvateyu (aMshavu) deVvateyanunx seVruvudu, iMdirxyavu liMgadalilx
(haqdayadalilx) aikayxvanunx heVge hoMduvudo, hAge.
\end{artha}

\begin{shl}
atheYvaM sayxrUpajocnxV deVvatAkaraNacuyxteVH ||  \\
AtemxYvamukAtxnuketxVSu yoVjayxM pArxNeVSavxsheVSataH ||  61 ||  
\end{shl}

\begin{artha}
anaMtarave hiVgAgalu Atamxnu deVvateyU iMdirxyavu tapipxhoVdadadxriMda
rUpavanunx tiLiyade iruvanu, I riVtiyAgi heVLida viSayagaLalUlx
heVLadiruva (viSayagaLalUlx) samasatx iMdirxyagaLalUlx elalxvanunx
yoVjisikoLaLxbeVku.
\end{artha}

\begin{shl}
ceVtanAvadadhiSAThxnAtakxraNAdeVjaRDAtamxnaH || \\
parxvaqtitxniRyatA daqSATx loVkeV\s toV na vipayaRteV ||  62 ||  
\end{shl}

\begin{artha}
alalxde jaDarUpavAda iMdirxyAdigaLige ceYtanayxviruva Atamxna
Asharxyavidudx parxvaqtitxyuMTAguvudeMbudu loVkadalilx niyatavAgi
kaMDide, adariMda adakekx badalAgi EnU kANuvudilalx.
\end{artha}

\begin{shl}
arUpajacnxtavxheVtuH ka itAyxshaknokxyXVtatxraM vacaH || \\
loVkaparxsididhxmAcaSaTx EkiVtAyxdi mumUSaRti ||  63 ||  
\end{shl}

\begin{artha}
rUpavanunx Eke tiLiyuvudilalx ? veMdu AshaMke mADi utatxravAkayxvu
`EkiVBavati' itAyxdi vAkayxvu sAyuvAtana viSayadalilx
loVkaparxsididhxyanunx heVLuvudu.
\end{artha}

\section*{baq. shurxti 4-4-2}

\begin{shl}
EkiV Bavati na pashayxtiVtAyxhureVkiV Bavati na jiGarxtiVtAyxhureVkiV Bavati na rasayata itAyxhureVkiV Bavati na vadatiVtAyxhureVkiV Bavati na shaqNoVtiVtAyxhureVkiV Bavati na manuta itAyxhureVkiV Bavati na sapxqqshatiVtAyxhureVkiV Bavati na vijAnAtiVtAyxhusatxsayx heYtasayx haqdayasAyxgarxM parxdoyxVtateV teVna parxdoyxVteVneYSa AtAmx niSAkxrXmati cakuSxSoTxV vA mUdhonxVR vAneyxVBoyxV vA shariVradeVsheVBayxsatxmutAkxrXmanatxM pArxNoV\s nUtAkxrXmati pArxNamanUtAkxrXmanatxM saveVR pArxNA anUtAkxrXmanitx savijAcnxnoV Bavati savijAcnxnameVvAnavxvakArxmati ||taM vidAyxkamaRNiV samanAvxraBeVteV pUvaRparxjAcnx ca || 2 ||
\end{shl}

\begin{shl}
EkiV Bavati sAvitarxH savitArxM\s shasatxtheYva ca || \\
liknAgxMshashAcxkuSxSoV vUyxDhoV liknegxYkayxM parxtipadayxteV ||  64 ||  
\end{shl}

\begin{artha}
savitaqdeVvateya (aMshiyU) savitarxdeVvateya aMshavU oMdAguvudu
cakuSxsAthxnadalilxdadx liMgAMshavu cAkuSxSa puruSanU liMgeYkayxvanunx
(haqdayeYkayxvanunx) hoMduvudu.
\end{artha}

\vishaya{`napashayxtiVtAyxdi' vAkayxda tAtapxyARthaR -}

\begin{shl}
sAvxMshiBAyxmeVkatApatwtx deVvatAliknagxBAgayoVH || \\
athArUpajacnxtAmeVti yaH pArxgUrxpAdiveVdayxBUtf ||  65 ||  
\end{shl}

\begin{artha}
tananx aMshigaLoDane deVvatAMshakUkx liMgAMshakUkx EkatavxvuMTAgalu
yAva jiVvanu hiMde rUpAdigaLanunx tiLiyutatxlidadxno avanu I
rUpajAcnxnavanunx hoMduvudilalx.
\end{artha}

\vishaya{``itAyxhuH'' eMbudara athaR -}

\begin{shl}
atheYnaM pAshavxRgAH pArxhubaRnadhxvoV\s sayx mumUSaRtaH || \\
vAkayxmeVkiV BavatiVti nirAshAsatxsayx jiVviteV ||  66 ||  
\end{shl}

\begin{artha}
anaMtarave I Atamxnanunx kuritu pakakxdalilxdadx I sAyuvavana
baMdhugaLu Atanu badukuvudaralilx nirAsharAgi `EkiVBavatinapashayxti'
itAyxdi vAkayxvanunx heVLuvaru.
\end{artha}

\vishaya{anuBavAnusAravAgi I kAladalilx Aguva ajAcnxna karxmavanunx
tegedukoLaLxbeVke vinaha pATha karxmavananxlalx enunxtAtxre -}

\begin{shl}
karaNeV deVvatA yasimxnUpxvaRM mucnacxtayxnugarxhamf || \\
taseyxYva parxthamaM vaqtitxniroVdha upajAyateV ||  67 ||  
\end{shl}

\begin{artha}
yAva iMdirxyadalilx deVvateyu modalu anugarxhisuvudanunx biDuvudo
adara vaqtitx niroVdhaveV modalu Aguvudu.
\end{artha}

\vishaya{EkakAladalelxV iMdirxyAdigaLa utakxrXmaNavAgali,
karxmadiMdeVnAgutatxde ? eMdare -}

\begin{shl}
kasimxMshicxtakxraNeV vaqtitxniroVdhaH pUvaRmiVkaSxyXteV || \\
aywgapadeyxVnoVtAkxrXnwtx talilxknagxM deVvatAtamxnAmf ||  68 ||  
\end{shl}

\begin{artha}
yAvudo oMdu iMdirxyadalilx modalu vaqtitx niroVdhavanunx kANuvudu
EkakAladalilx deVvatA savxrUpagaLige utAkxrXMtiyAgadiruvudakekx adeV
gamaka.
\end{artha}

\begin{shl}
yadA\s \s loVcanamAtarxM sAyxdUrxpAdw cakuSxrAdiBiH || \\
parxviveVkAtx tu rUpAdeVH samayxknenxYva pumAneBxveVtf ||  69 ||  
\end{shl}

\begin{shl}
tadoVtAkxrXnAtxM vijAniVyAnamxnasoV deVvatAmitaH || \\
keVvalAloVcanAlilxknAgxtatxthA budedhxVshacx deVvatAmf ||  70 ||  
\end{shl}

\begin{artha}
cakuSx muMtAdi iMdirxyagaLiMda rUpAdi viSayadalilx yAva AloVcane
mAtarx vAdiVtu, Adare rUpAdigaLanunx sariyAgi I puruSanu
viveVcisalAranu, AvAga I manasisxniMda deVvateyu utAkxrXMtiyanunx
hoMdideyeMdu tiLiyabeVku matutx keVvala AloVcaneyeMba heVtuviniMda
budidhxsAthxnadiMdalU deVvateyu utAkxrXMtiyanunx hoMdideyeMdu
tiLiyabeVku.
\end{artha}

\begin{shl}
EvaM tadedxVvatoVkArxnAtxveVkiVBAveVna saMgateVH || \\
manuteV noV na jAnAtiVteyxVvamAhusatxthA janAH ||  71 ||  
\end{shl}

\begin{artha}
I riVtiyAgi A deVvateyu utAkxrXMtiyanunx hoMdidalilx oMdAguvudariMda
Itanu namamxnunx tiLiyuvudilalxveMdu hAgeyeV janaru heVLuvaru.
\end{artha}

\begin{shl}
ukatxM vimoVkaSxNaM tAvatakxraNAnAM savxdeVshataH || \\
asaMvijAcnxnatA coVkAtx haqdayeV coVpasaMhaqtiH ||  72 ||  
\end{shl}

\begin{artha}
I riVtiyAgi iMdirxyagaLu tamamxsAthxnadiMda biDugaDeyAguvudanUnx
jAcnxnavilalxdiruvudanunx haqdayadalilx elalxvudakukx upasaMhAravanunx
heVLidAdxyitu.
\end{artha}

\begin{shl}
athoVpasaMhaqtAsheVSakaraNasAyx\s \s tamxnoV yathA || \\
loVkAnatxroVpasaMkArxnitxsatxtheYtadaBidhiVyateV ||  73 ||  
\end{shl}

\begin{artha}
AmeVle samasatx iMdirxyagaLU upasaMhAravAgiruva I Atamxnige beVre
loVkadalilx saMbaMdhavu heVge AguvudeMbudanunx (I muMde) heVLuvudu.
\end{artha}

\vishaya{``tasayx heYtasayx haqdayasAyxgarxM parxdoyxVtateV'' eMbudara vAyxKAyxna -}

\begin{shl}
taseyxYtasayx yathoVkatxsayx kaqtasxnXpArxNoVpasaMhaqteVH || \\
nADayxgarxM haqdayasAyxtha parxkaSeVRNa parxkAshateV ||  74 ||  
\end{shl}

\begin{shl}
saveVRSu saMhaqteVSevxVvaM karaNeVSu savAyuSu ||  \\
budedhxVH parxdoyxVtateV\s thAgarxM haqdayasayx yiyAsataH ||  75 ||  
\end{shl}

\begin{artha}
I hiMde heVLida I jiVvanige elAlx iMdirxyagaLU
upasaMhAravAgiruvudariMda haqdayada nADi tudiyu hecAcxgi
parxkAshisuvudu, pArxNavAyu sameVtavAgi elAlx iMdirxyagaLU upasaMhAra
hoMdalu parxyANa mADalu, bayasida jiVvana yAva haqdayavideyoV adara
tudiyu hecucx parxkAshisuvudu.
\end{artha}

\vishaya{haqdayA garxvu parxkAshisuvudakekx kAraNaveVneMdare -}

\begin{shl}
sa ESa kamaRjoV budedhxVH parxkAshoV jAyateV maqtw || \\
savxkamaRnimiRtaM loVkaM yeVnA\s \s tAmx\s yaM parxpashayxti ||  76 ||  
\end{shl}

\begin{artha}
yAva budidhxya parxkAshavu maraNakAladalilx Aguvudo adu kamaRdiMda
Aguvudu, yAvudariMda I Atamxnu tananx kamaRdiMda nimiRsida loVkavanunx
noVDuvano (A parxkAshavidu).
\end{artha}

\vishaya{ahaMteyU mamateyU Agale toVrabahudaSeTx ? eMdare -}

\begin{shl}
kamaRNeYvAsayx vijAcnxnaM tAdAtamxyXmupaniVyateV || \\
pashAcxdApananxtadABxvoV deVhameVtaM vimucnacxti ||  77 ||  
\end{shl}

\begin{artha}
kamaRdiMdale Itanige muMde baruva deVha jAcnxnavu Aguvudu adariMdaleV
(ahaM) eMba tAdAtamxyXvu baruvudu, anaMtarave BAvi deVhadalilx tAdAtamxyX
BAvavanunx hoMdida Itanu I deVhavanunx biDuvanu.
\end{artha}

\vishaya{sakala iMdirxyagaLu uparatavAgiralu muMdina deVhada sUPxtiR
heVge Aguvudu ? eMdare}

\footnotetext[1]{maraNavAguva modaleV sumAru Aru mAsagaLiMdalU AraMBisi
BAvideVhadalilx, `ahamasimx' eMba aBimAnavu puruSanige uMTAguvudu.
adeV ililx parxdoyxVtaveMdu heVLalapxDuvudeMdu (AnaM-TiVke). idakekx
AdhAra (4 - 4 - 119 - 120 ne vAtiRkaveV idu pArxyikavAda mAtu) Adare
ArumAsagaLige hiMdeyeMba niyamaveVnU ilalx maraNakekx modalo
deVvaloVkadalilx nAnu deVvarAguvuneMdu BAvaneyanunx upAsaneyoMdige
mADida PalavAgi padeV padeV anusaMdhAna mADida baladiMda
`nAneVdeVvanAgidedxVneMba jAcnxnavu A maraNakAladalUlx Aguvudu. `sadA tadABxva BAvitaH'' eMdu giVteyalilx heVLidaMte muKayxvAgi modalu AyAya deVvatA BAvavu satata aBayxsatxvAgirabeVkAdadu Avashayxka, adriMdale
AkAladalilx I jAcnxnavu baruvudeMdu namamx aBipArxya.}
\begin{shl}
\footnotemark[1]BAviloVkAtimxkA yA\s sayx parxtayxkecxYtanayxbimibxtA || \\
vAsaneYvA\s \s tamxnaH porxVkAtx parxdoyxVtavacasA suPxTamf ||  78 ||  
\end{shl}

%%% footnote shloka
\begin{artha}
`parxdoyxVta' eMba padadiMda muMdina loVkada (shariVrada) vAsaneyu
parxtayxgAtamxceYtanayx parxtibiMbisidudx yAvuduMTo adeV heVLalapxTiTxruvudu.
\end{artha}

\vishaya{maraNakAladalilx manasUsx ilalxvAdadxriMda BAvideVha
jAcnxnavu heVge baruvudu -}

\begin{shl}
mAtorxVpAdAnarUpeVNa sevxVna BAsA purA\s barxviVtf || \\
sevxVneYva joyxVtiSA savxpenxV yathA tadavxdihApi tatf ||  79 || 
\end{shl}

\begin{artha}
heVge savxpanxdalilx jAgarxtitxna vAsanAsivxVkArarUpavAda tananx
BAnadiMda (vAsanArUpavAda vaqtitxyiMda) matutx savxrUpa ceYtanayx
joyxVtiyiMdalU jAcnxnavAguvudeMdu hiMde (joyxti bArxhamxNadalilx)
shurxtiyu heVLiruvudo hAgeye ililxyU (ImaraNa dasheyalUlx A
jAcnxnavAguvudu\footnote{maraNakAladalilx budidhxyu
upasaMhAravAgiruvudilalx. savxpanxdalilx jAcnxnavAguvaMte
budidhxyalilx IjAcnxnavu Aguvudu. adariMda shaMkege avakAshavilalx.}.
\end{artha}

\vishaya{`teVna' itAyxdi vAkayxda tAtapxyaR}

\begin{shl}
parxdoyxVteVna yathoVketxVna parxdoyxVtitapathA kaqta -\\
kamaRkAyoVR\s tha haqdayAninxSAkxrXmati yathAsuKamf ||  80 ||  
\end{shl}

\begin{artha}
\footnote{ililx `yathAsuKamf' eMbudakekx `asuKaM suKamiti vAceCxVdaH'
eMdu padadeCxVda mADide (AnaM - TiVke) puNAyxnusAra suKavAgiyU,
pApAnusAra duHKavAgiyU eMdathaR virabeVku.}hiMdeheVLida parxdoyxVtadiMda toVrisalapxTa mAgaRdiMda tAnu
mADida kamaR PalavuLaLxvanAgi atha AmeVle (Pala BoVgakAla baMdAga)
suKavAgi horaDuvanu.
\end{artha}

\begin{shl}
sAdhevxVvAtaH parxyatenxVna kamaR kAyaRM vipashicxtA || \\
pashayxtA pArxNinAmeVvaM kamaRmUlAmimAM gatimf ||  81 ||  
\end{shl}

\begin{artha}
adariMda aviVkiyAdavanu I riVtiyAgi pArxNigaLige kamaR mUlakavAgi
baruva I gatiyanunx noVDutAtx parxyatanx pUvaRka utatxmakavaRvanenxV
mADabeVku.
\end{artha}

\begin{shl}
AtAmx savxkamaRNoVpAtatxM parxdoyxVteVna yathoVcitamf || \\
loVkaM pashayxnasxvXhaqdayAninxSAkxrXmati yathAyathamf ||  82 ||  
\end{shl}

\begin{artha}
Atamxnu tananxkamaRdiMda ajiRtavAda matutx jAcnxnadiMda
parxkAshitavAdaMte shariVravanunx noVDutatx tananx haqdaya
sAthxnadiMda heVge horaDabeVko hAgeye horaDuvanu.
\end{artha}

\vishaya{`yathAyathaM' eMbudara vivaraNe -}

\begin{shl}
cakuSxSoTxV vA\s tha mUdhonxVR vA yaM yaM loVkaM parxpatasxyXteV || \\
tadUdxreVNeYva niSAkxrXmananx kacitapxrXtihanayxteV ||  83 ||  
\end{shl}

\begin{artha}
kaNiNxniMdaloV athavA shirasisxniMdaloV (netitxyiMdaloV) (\quad) yAva
yAva loVkavanunx muMde paDeyuvano, adara dAvxradiMdaleV horuDutAtx,
(jiVvanu) elilxyU taDeyalapxDade iruvanu.
\end{artha}

\footnotetext[1]{AditayxloVka pArxpitxge nimitatxvAda upAsane athavA kamaRvu
ididxdedx Adare eMdathaR}
\begin{shl}
AditayxloVka\footnotemark[1]saMpArxpwtx cakuSxSoTxV\s yaM gigacaCxti || \\
barxhamxloVka\footnotemark[2]paripArxpwtx mUdhanxR AtAmx nigacaCxti ||  84 ||  
\end{shl}
\footnotetext[2]{barxhamx loVka pArxpitxge
nimitatxvAda upAsane kamaRvu idadxre eMdathaR}

%%% footnote shloka
\begin{artha}
AditayxloVkavanunx hoMdalu kaNiNxniMda Itanu
hoVguvanu, barxhamx loVkavanunx paDeyuvAga mUdhaRsAthxnadiMda Atamxnu
hoVguvanu.
\end{artha}

\vishaya{-jiVvana saMsAra vaNaRne-}

\vishaya{`aneyxVBoyxVvA shariVra deVsheVBayxH' eMbudara vAyxKAyxna -}
	
\begin{shl}
kamaRshurxtAnuroVdheVna hayxneyxVBoyxV vA yathAyathamf || \\
haqdayasaqtanADiVBirita AtAmx nigacaCxti ||  85 ||   
\end{shl}

\begin{artha}
tAnu mADida kamaR matutx shAsotxrXVkatx upAsane ivugaLanunx anusarisi
beVre parxdeVshagaLiMdalU (shariVrada beVre dAvxragaLiMdalU)
hoVguvanu, adu heVgeMdare ? haqdayadiMda horaTiruva nADigaLa mUlaka
ililxMda Atamxnu hoVguvanu.
\end{artha}

\vishaya{deVhavanunx biTuTx edidxruva jiVvanige aneVka parxtihatigaLu
(taDegaLu) horage uMTAguvudilalxve ? eMdare - ilalxvenunxtAtxre -}

\begin{shl}
liknagxM ca savaRtoV gacaCxnanx kacitapxrXtihanayxteV || \\
atisUkaSxmXsavxBAvatAvxdapi loVhasamudarxgamf ||  86 ||  
\end{shl}

\begin{artha}
liMga shariVravu elAlx kaDeyalUlx hoVgutatxlidadxrU elilxyU
taDeyalapxDuvudilalx, EkeMdare, adu ati sUkaSxmXvAda
savxBAvavuLaLxdudx, adariMda, iSeTxVke ? (aBeVdayxvAda) loVhada
hatitxra hoVdarU taDeyalapxDuvudilalx.
\end{artha}

\vishaya{`tamutArx manatxH pArxNoV\s nUtAkxrXmati' itAyxdi shurxtiya athaR -}

\begin{shl}
utAkxrXmanatxM tamAtAmxnaM yathoVketxVneVha vatamxRnA || \\
pArxNoV\s nUtAkxrXmati tataH pArxNaM pArxNAsatxthA pareV ||  87 ||  
\end{shl}

\begin{artha}
utAkxrXMti hoMdutitxruva A Atamxnanunx, anusarisi hiMde heVLida
mAgaRdiMda pArxNavu utAkxrXMtiyanunx hoMduvudu. anaMtara pArxNavanunx
anusarisi itare pArxNagaLu (iMdirxyagaLa) hAgeye utAkxrXMtiyanunx hoMduvavu.
\end{artha}

\vishaya{ililx shurxtiyu heVLida karxmada meVle AkeSxVpa matutx
samAdhAnagaLu -}

\begin{shl}
nanAvxtamxpArxNavAgAdeVranoyxVnayxvayxtimisharxNAtf || \\
deVshakAlAdayxsaMBeVdAtakxrXmeVNoVtakxrXmaNaM kathamf ||  88 || 
\end{shl}

\begin{shl}
neYSa doVSoV yatoV neVha karxmakAloV vivakaSxyXteV || \\
parxyoVjakaparxyoVjayxtavxM yatoV\s miVSAM vivakiSxtamf ||  89 ||  
\end{shl}

\begin{artha}
Atamx pArxNa, vAgAdi iMdirxyagaLu ivelalxvU oMdakokxMdu
seVrikoMDiruvudariMda deVshakAlAdigaLU misharxvAguvudu, adariMda
(oMdAda meVle matotxMdu eMba) karxmadiMda utAkxrXMtiyeMdu heVLidudx
heVge ?
\end{artha}

\begin{artha}
ideVnU doVSavalalx, yAvudariMda eMdare ? ililx karxmakAlavu
udedxVshisalapxTiTxlalx, parxyoVjaka, parxyoVjayx BAvavu (guNa
parxdhAnaBAvavu) ivugaLige vivakiSxtavAgide.
\end{artha}

\begin{shl}
ihoVcicxkarxmiSA yA\s sayx vijAcnxnAtemxYkasaMsharxyA || \\
\footnotemark[1]pArxNoVtAkxrXnetxVH parxyoVkitxrXV sA vAgAduyxtakxrXmaNasayx ca ||  90 ||  
\end{shl}
\footnotetext[1]{Atamxnige pArxdhAnayx adanunx hiMbAlisida muKayxpArxNakekx
apArxmuKayxte idU saha Atamxn daqSiTxyiMda, pArxNavu vAgAdi
iMdirxyagaLa daqSiTxyiMda parxdhAna, adanunx anusarisida itara
iMdirxyagaLu aparxdhAna jagatitxnalilx jiVvanige utakxrXmaNada
parxvaqtitxyu loVkaparxsidadhxvAgidadxre adu pArxNoVtakxrXmaNakekx
parxyoVjakavAguvudu, pArxNoVtakxrXmaNavu vAgAdi iMdirxyagaLa
utakxrXmaNakekx parxyoVjaka, ivugaLalilx jiVvana utakxrXmaNakekx
parxyoVjaka yAvudeMdare ? muKayx pArxNada utakxrXmaNa, savxtaH
jiVvAtamxnige utakxrXmaNa ``kasimxnavxha mutAkxrXnetxV utAkxrXnotxV BaviSAyxmi, kasimxnf vA parxtiSiThxteV parxtiSAThxsAyxmi''}

%%% footnote shloka
\begin{artha}
ililx jiVvAtamxnanenxV avalaMbisida utAkxrXMtiVceCxyU saha Itanige
pArxNotAkxrXMtiyanunx avalaMbiside, hAgUvAgAdi iMdirxyagaLa
utAkxrXMtiyU pArxNotAkxrXMtiyanunx parxyoVjakaveMdu avalaMbiside.
\end{artha}

\begin{shl}
muKayxpArxNasayx yoVtAkxrXnitxH parxyoVkitxrXV seYva nAparA || \\
\footnotemark[2]vijAcnxnAtemxYkaniVDAyA BAvanAyAH parxdhAnatA ||  91 ||  
\end{shl}
\footnotetext[2]{`savijAcnxnoVBavati' eMbuva muMdina vAkayxda
vAyxKAyxnakAkxgi idu avataraNike. BAvaneyu parxdhAnaveMdu
AraMBiside, adu heVge ? eMbudanunx muMde vivarisuvudu.}

%%% footnote shloka
\begin{artha}
muKayxpArxNada yAva utakxrXmaNavideyo adeV parxyoVjaka, beVreyalalx,
jiVvAtamxnobabxnanenxV Asharxyisida BAvanege pArxdhAnayxvu.
\end{artha}

\vishaya{pUvaR parxjecnxyeMbuva BAvaneyu heVge ? parxdhAna eMdare -}

\begin{shl}
sA hi deVhAnatxrapArxpAtxvAtamxnoV mAgaRdashiRniV || \\
tayA parxyukatxH pArxNoV\s yaM pArxNAnAdAya ceVtarAnf ||  92 ||  
\end{shl}

\begin{shl}
AtamxnA\s nanayxBUtaH sacnajxlapAtArxkaRvadabxhiH || \\
niSAkxrXmati yathAkamaR savijAcnxnoV Bavatayxtha ||  93 ||  
\end{shl}

\begin{artha}
A BAvaneyu Atamxnige deVhAMtaravulaBisuvudakekx (athavA
deVhAMtaravanunx hoMduvudakekx) mAgaRdashiRyAgide, heVge
jalapAtarxdalilx parxtibiMbisuva sUyaRnu jalapAtarxvanunx
anusarisuvano, hAgeye AtamxnoDane EkavAgidudx (Atamxnanunx anusarisi)
A BAvaneyiMda perxVritavAda I muKayx pArxNavu itare pArxNagaLanunx
(iMdirxyagaLanunx) tegedukoMDu kamaRdaMte (utakxrXmaNakekx modaleV)
muMde baruva deVhada vijAcnxnadiMda kUDidudx anaMtara deVhada horage
horaDuvudu.
\end{artha}

\begin{shl}
nanUpasaMhaqtAsheVSakaraNatAvxdadhiVH pumAnf || \\
tadAkUtAnuguNayxM sAyxtApxrXNAdeVH kathamucayxteV ||  94 ||  
\end{shl}

\begin{artha}
samasatx iMdirxyagaLU upasaMhAravAgiruvudariMda aMtaHkaraNavU ilalxde
puruSanu (nilulxvanu), adariMda avana aBipArxyavanunx pArxNAdigaLu
anusarisuvavu - eMbudanunx heVge opupxvudu ? eMdu (shaMkisidare)
ucayxteV - eMdu samAdhAnavu heVLalapxDuvudu.
\end{artha}

\vishaya{`saMjAcnxna meVvAnavxvakArxmati' eMdu mAdhayxMdina pAThadaMte samAdhAna mADalu
vAyxKAyxnavu AraMBavAgide  -}

\begin{shl}
parxtayxBijAcnxtamxkaM jAcnxnaM saMjAcnxnamiti BaNayxteV || \\
pUvaRM yadadhxqqdayasAyxgarxdoyxVtaneVna parxkAshitamf ||  95 ||  
\end{shl}

\begin{shl}
\footnotemark[1]BAvinoV\s thaRsayx vijAcnxnaM parxtayxBijAcnxnamucayxteV ||  \\
sa tathoVdUBxtavijAcnxna AtAmx deVhAninxgacaCxti ||  96 || 
\end{shl}
\footnotetext[1 2]{parxtayxBijAcnxnaveMdareVnu ? muMdu baruva deVhada
viSayadalilx pUvARnuBavavilalxdeyiruvudariMda adara saMsAkxradiMda
huTuTxva jAcnxnaveMdu heVLuvudu yukatxvalalxveMdu parxshenx baMdare
adakekx utatxravAgi pUvaRM itAyxdi vAtiRkavu, adara aBipArxyavu
AnaMdagirigaLiMda hiVge horaDisalapxTiTxde - I mumUSuRvAda
puruSanige anusaMdhAna mADida viSayadalilx baruva jAcnxnaveV
haqdayAgarx parxdoyxVtaveMdide, idu maraNakekx muMceye Aru
mAsagaLiMdalU huTiTx naDedu baMdide, jiVvaviruvAga toVruvaMte maraNa
dasheyalUlx, `nAnu deVvaneV AgidedxVneMba aBimAna rUpavAda
BAvanAtamxka jAcnxnavu baruvudu ideV parxtayxBijAcnxnaveMdathaR,
pAThakarxmadaMte utakxrXMtiya naMtara I jAcnxnavu AguvudeMbudu
sariyalalx, pATha karxmavanunx biTuTx athaR karxmavanunx
anusarisabeVku, AvAga A mumuGaRvu hiMde heVLida parxtayxBijAcnx
rUpavAda jAcnxnadiMda kUDi anaMtaraveV deVhada horage hoVguvaneMdu I
jAcnxnavu modaleV uMTAgideyeMdu athaRmADikoLaLxbeVku pAThakarxmavu
muKayxvalalx, hAgAdare utakxrXMti karxmavU muKayxvenanxbeVkAguvudu.}


%%% footnote shloka
\begin{artha}
\footnotemark[2]parxtayxBijAcnxrUpavAda jAcnxnaveV `saMjAcnxnaM' eMbudariMda
heVLalapxDuvudu, parxtayxBijAcnxnaveMdare yAvudu modalu haqdayada
nADiVdAvxrada parxkAshadiMda parxkAshisalapxTiTxto, A muMdina viSayada
(deVhada) vijAcnxnaveV eMdu heVLalapxDuvadu, A (mumUSuRvAda) Atamxnu A
riVtiyAgi huTiTxda vijAcnxnavuLaLxvanAgi deVhadiMda horaDuvanu.
\end{artha}

\vishaya{shaMke}

\begin{shl}
haqdayAgarxparxkAsheVna niSAkxrXnatxsAyxpi deVhataH || \\
niHsaMboVdhasayx gamanaM kathaM deVhAnatxraM parxti ||  97 || 
\end{shl}

\begin{artha}
haqdayada agarxvu parxkAshavAguvudariMda deVhadiMda horage
horaTavanAdarU jAcnxnavilalxdiruva Itanige beVre deVhakekx
hoVguvudeMbudu heVge ?
\end{artha}

\vishaya{samAdhAna}

\begin{shl}
itayxsayx parihArAya sa itAyxdi paraM vacaH || \\
sa ESa jacnxH paroV deVvaH savijAcnxnoV Bavatayxtha ||  98 ||  
\end{shl}

\begin{artha}
I shaMkeyanunx pariharisalu `sa ESa' itAyxdi muMdina vacanavu - adeV
idu - ``sa ESa jacnxH paroVdeVvaH savijAcnxnoV Bavatayxtha'' || 
\end{artha}

\vishaya{(idu mAdhayxMdina pAThadaMtiruvadu) idara athaR -}

\begin{shl}
saMhaqtAsheVSakaraNoV BAvanAkamaRheVtutaH || \\
savijAcnxnoV yathA savxpenxV tathoVtAkxrXnAtxvapiVSayxteV ||  99 ||  
\end{shl}

\begin{artha}
savxpanxdalilx heVgo hAgeye utAkxrXMtiya kAladalUlx sakala
iMdirxyagaLu upasaMhAravAgidadxrU BAvane, kamaR I kAraNagaLiMda
jAcnxnadiMda kUDiye iruvaneMdu opipxruvudu.
\end{artha}

\vishaya{kaNavx-mAdhayxMdina pAThagaLalilx muMdina deVhada jAcnxnavu
kamaRda apeVkeSxyilalxde ceVtanavAdadxriMda tAnAgiye
uMTAgabahudaSeTx ? eMdare -}

\begin{shl}
visheVSajAcnxnasaMbanadhxH kamaRNeYvAsayx heVtunA || \\
na tu sAvxtaMtarxyXtoV laBayxH sAvxtanAtxrXyXsaMBavAdiha ||  100 ||  
\end{shl}

\begin{artha}
Itanige kamaRveMba nimitatxdiMdaleV visheVSa jAcnxna saMbaMdhavu
uMTAguvudu, sAvxtaMtarxyXvu ililx saMBavisuvudilalxvAdadxriMda
sAvxtaMtarxyXdiMda laBisuvudalalx.
\end{artha}

\vishaya{sAvxtaMtarxyXvilalxdiruvudakekx kAraNaveVnu ?}

\begin{shl}
kaqtakaqtoyxV BaveVtasxvoVR visheVSajAcnxnasaMgatiH || \\
sAvxtanetxrXyXVNeVha ceVlalxBAyx na tu laBAyx tathA\s \s ha ca ||  101 || 
\end{shl}

\begin{shl}
yaM yaM vA\s pi samxranABxvaM tayxjatayxnetxV kaleVvaramf ||  \\
taM tameVveYti kwnetxVya sadA tadABxvaBAvitaH ||  102 ||  
\end{shl}

\begin{artha}
sAvxtaMtarxyXdiMdaleV visheVSajAcnxna saMbaMdhavu laBisuvudeV
Agidadxre elalxrU kaqtakaqtayxrAgi biDutitxdadxru, Adare hAge
laBisuvudilalx, hAgeye BagavaMtanu heVLiruvanu. yAva yAva deVvatA
savxrUpavanunx samxrisutAtx koneyalilx deVhavanunx biDuvano AyAya
deVvaBAvavanenxV hoMduvanu ajuRna keVLu (kAraNaveVneMdare) yAvAgalU
AyAya deVvatA savxrUpa BAvaneyiMda modale susaMsakxqqtanAgiruvanu ||
eMdu (giVte  a.\quad) 
\end{artha}

\vishaya{upasaMhAra}

\begin{shl}
EvamAtAmx savijAcnxnoV ganatxvayx yatupxrA\s jiRtamf || \\
tadAtamxBAvavijAcnxnasatxdeVvAtoV nigacaCxti ||  103 ||  
\end{shl}

\begin{artha}
I riVtiyAgi vijAcnxnadiMda kUDidavanAgi aMdare ABAvi deVhadalilx
mADida Atamx BAvanA rUpavAda vijAcnxnavuLaLxvanAgi tAnu modalu yAva
gaMtavayx sAthxnavanunx hiMdeyeV saMpAdisidadxno adeV sAthxnakekx I
(kamaRda vashadiMda) hoVguvanu.
\end{artha}

\begin{shl}
yata EvamataH pumiBxH sAvxtanAtxrXyAthaRM parxyatanxtaH ||  \\
yoVgAdisAdhanABAyxsaH kataRvayxH puNayxsaMcayaH ||  104 ||  
\end{shl}

\begin{artha}
yAvakAraNadiMda hiVge ideyo adariMda sAvxtaMtarxyXkAkxgi yoVgAdi
sAdhanagaLanunx parxyatanx pUvaRka aBAyxsa mADabeVku, puNayx
saMcayavanunx mADabeVku (puNayxgaLanunx kUDihAkikoLaLxbeVku).
\end{artha}

\begin{shl}
savaRshAsatxrXsamAramaBxsatxdathoVR\s yaM mahAniha || \\
vAknaBxnaHkAyasAdhAyxnAmupAyAnAM parxbudadhxyeV ||  105 ||  
\end{shl}

\begin{artha}
elAlx shAsatxrXgaLa AraMBavU adakAkxgi (aMdare maraNakAladalilx tananx
sAvxtaMtarxvanunx sAdhisivudakAkxgi) I vayxvahAradalilx
doDaDxdAgiruvudu, alalxde vAkukx, manasusx, shariVragaLiMda
sAdhisabeVkAda sAdhanagaLa jAcnxnakAkxgiruvudu.
\end{artha}

\begin{shl}
tathA\s nathaRparipArxpAtxvupAyAnAM niSeVdhanamf || \\
EtAvAneVva shAsAtxrXthaRH katarxRdhiVnaH suKApatxyeV ||  106 ||  
\end{shl}

\begin{artha}
hAgU anathaRvu baralu kAraNavAda (surApAnAdi) sAdhanagaLanunx
niSeVdhisiruvudu, adariMda suKalABakAkxgi iSeTxVshAsAtxrXthaRvanunx
kataqRvige adhiVnavAgiruvudanunx mADatakakxdudx.
\end{artha}

\begin{shl}
mAtArxlakaSxNamAdAnaM kaqtAvx savxpanxsayx sajaRnamf ||  \\
yatheVha na tathA kiMcidupAdAnaM samiVkaSxyXteV ||  107 ||  
\end{shl}

\begin{artha}
jAgarxtitxna saMsAkxravanunx sivxVkarisi savxpanxvanunx saqSiTxsuvudU
savxpanxdalilx heVge ideyo hAge I maraNakAladalilx oMdanUnx
sivxVkarisuvudeMbudu kANuvudilalx.
\end{artha}

\begin{shl}
atheYvaM deVvatAtayxketxV liknegxV deVhAdabxhigaRteV || \\
loVkAnatxragatw heVtuloVRkArameBxV ca BaNayxtAmf ||  108 ||  
\end{shl}

\begin{artha}
anaMtara I riVtiyAgi deVvateyu biTiTxruva liMga shariVravu deVhada
horage hoVguvalilx beVre loVkakekx hoVguvudakUkx beVre loVkada (divayx
shariVrada) huTiTxgU kAraNaveVneMbudanunx heVLabeVku.
\end{artha}

\begin{shl}
AtamxnaH paraloVkAya yatAsxyXdagxmanakAraNamf ||  \\
BuknekxtXV gatAvx ca yatatxtarx deVhArameBxV ca kAraNamf ||  109 ||  
\end{shl}

\begin{shl}
liknAgxnasa itoV deVhAdedxVhamanayxM nigacaCxtaH || \\
saMBAraH koV\s sayx gatayxthoVR deVhArameBxV ca kathayxtAmf ||  110 ||  
\end{shl}

\begin{artha}
Atamxnige paraloVkakekx hoVgalubeVkAda sAdhanavu yAvudu ideyo, matutx
hoVgi alilx yAvudanunx anuBavisuvano, alilx deVhavu huTaTxlu kAraNavu
yAvudu ideyo, matutx I deVhavanunx biTuTx beVre deVhavidadxDege
hoVguvAtanige liMga shariVraveMbagADiyuLaLxvanige hoVguvudakekx
beVkAda saMBAravu yAvudu ideyo, alalxde deVhoVtapxtitxgebeVkAda
sAdhanavu yAvudo avelalxvanunx heVLabeVku.
\end{artha}

\vishaya{`taM vidAyxkamaRNiV samanAvxraBeVteV' - eMbavAkayxvu adanunx tiLisalu baMdide enunxtAtxre -}

\begin{shl}
itoV jigamiSuM vidAyxkamaRNiV yeV purA\s jiRteV || \\
taM samanAvxraBeVteV teV yA cABUtUpxvaRvAsanA ||  111 ||  
\end{shl}

\begin{artha}
hiMde saMpAdisidadx jAcnxna kamaRgaLu yAvuduMTo avugaLu ililxMda
hoVgalu bayasida A jiVvAtamxnanunx hiMbAlisuvavu yAva pUvaRvAsaneyu
ididxto adU saha hiMbAlisuvudu.
\end{artha}

\begin{shl}
vijAcnxnaM saMshayajAcnxnaM mithAyxjAcnxnamathApi vA || \\
parxmANatoV\s parxmANAdAvx savaRM videyxVti BaNayxteV ||  112 ||  
\end{shl}

\begin{artha}
vijAcnxna matutx saMshaya jAcnxna, mithAyxjAcnxna (BArxMtijAcnxna)
yAvudAdarAgali parxmANadiMdalo aparxmANadiMdalo huTuTxva elAlx
jAcnxnavU (shurxtiyalilxruva) vidAyxpadadiMda heVLalapxDuvudu.
\end{artha}

\begin{shl}
saMsArakAraNadhavxMsi yatutx jAcnxnaM parAtamxgamf ||  \\
tadatarx na parigArxhayxM savARpatAkxraNApanutf ||  113 ||  
\end{shl}

\begin{artha}
saMsAra kAraNavanunx nAshamADuvaMtaha paramAtamx viSayadalilx huTuTxva
jAcnxnaveMbudu yAvuduMTo, adanunx ililx (vidAyxpadadiMda)
tegedukoLuLxvudalalx, idu samasatxvipatutxgaLigU kAraNavAda
ajAcnxnavanunx hoVgalADisuvudu.
\end{artha}

\begin{shl}
saMsArakAraNaM tasAmxdAtAmxjAcnxnAviroVdhi yatf || \\
apArxpatxparamAthARthaRM jAcnxnamAtarxM jiGaqkiSxtamf ||  114 ||  
\end{shl}

\begin{artha}
adariMda Atamxna ajAcnxnakekx aviroVdhiyAdudx yAva
saMsArakAraNavideyo matutx paramAthaRvanunx muTaTxdeyiruvudo A jAcnxna
mAtarx ililx gArxhayxvAgide.
\end{artha}

\begin{shl}
vAknaBxnaHkAyasAdhayxM ca shAsatxrXtoV yadi vA\s nayxtaH || \\
daqSATxdaqSATxthaRrUpaM yatatxcacx kameVRti gaqhayxteV ||  115 ||  
\end{shl}

\begin{artha}
vAgiMdirxya manasusx kAya ivugaLiMda mADalu sAdhayxvAda matutx
shAsatxrXdaMteyo athavA adakekx beVreyAda (loVkanAyxyadaMteyo)
mADalapxDuva daqSaTx matutx adaqSaTx PalakAkxgiruva vAyxpAravelalxvU
kamaRveMdu garxhisalapxDuvudu.
\end{artha}

\begin{shl}
anAvxraBeVteV gacaCxnatxM yathoVketxV jAcnxnakamaRNiV || \\
gacaCxnatxM puruSaM yasAmxdanevxVteV savxsavxBAvataH || \\
gacaCxtoV\s toV\s nushabodxV\s tarx pashAcxdatheVR parxyujayxteV ||  116 ||  
\end{shl}

\begin{artha}
hoVgutatxliruva jiVvAtamxnanunx hiMbAlisi I hiMde heVLida I jAcnxna
kamaRgaLu EraDU baruvudu, yAvakAraNadiMda hoVgutitxruva jiVvananunx
hiMbAlisiye (I jAcnxna kamaRgaLu) tananxtananx savxBAvadiMda
hoVguvavo, adariMdale anushabadxvu ililx AmeVle eMbathaRdalilx
parxyoVgisalapxDuvudu.
\end{artha}

\vishaya{`pUvaRparxjAcnx' eMbudara athaR -}

\begin{shl}
gamanAdividhw puMsaH sAdhanatavxM nigacaCxtaH ||  117 ||  \\
kamaRNaH kirxyamANasayx saMsAkxroV yoV haqdi shirxtaH || \\
tataPxlasayx ca Bukatxsayx pUvaRparxjecnxVti soVcayxteV ||  118 ||  
\end{shl}

\begin{artha}
puruSanu parxyANa modalAdavugaLanunx mADalu sAdhanatavxvanunx
hoMdiruva (sAdhanavAda) acarisalapxDuva kamaRda saMsAkxravu yAvudu
haqdayalilxruvudo, adU matutx BoVgisida kamaR Palada saMsAkxravU pUvaR
parxjecnxyeMdu heVLalapxDuvudu.
\end{artha}

\begin{shl}
pUvoVRpacitasaMsAkxraheVtuBayxH sA\s BijAyateV ||  119 || \\
SaNAmxsasheVSaporxVdUBxtA vAsanA yA\s sayx deVhinaH || \\
mariSayxtoV\s nayxdeVhAthaRM pUvaRparxjecnxVti tAM viduH ||  120 ||  
\end{shl}

\begin{artha}
hiMde kUDiTaTx saMsAkxraveMba kAraNagaLiMda (maraNakAlakekx hiMde) Aru
tiMgaLu kAla uLidiruvAgaleV yAva vAsaneyu sAyuva jiVvAtamxnige
huTuTxvudo, adu beVre deVhavanunx paDeyalu (huTiTxkoLuLxvudu) adanenx
pUvaR parxjecnxyeMdu (jAcnxnigaLu) tiLidiruvaru.
\end{artha}

\begin{shl}
samathAR seYva teV yasAmxdudobxVdudhxM jAcnxnakamaRNiV || \\
narasAyxtaH parxdhAnatAvxtapxqqthakatxsAyx garxhaH kaqtaH ||  121 ||  
\end{shl}

\begin{artha}
A vAsaneyeV manuSayxna jAcnxna kamaRgaLanunx ebibxsalu yAva
kAraNadiMda samathaRvAgideyo, adariMda adu muKayxvAgiruvudariMda
adanunx parxteyxVkavAgi ililx tegedukoMDide.
\end{artha}

\vishaya{vidAyx kamaRgaLige pUvaR parxjecnxyaMte samAsa mADuvudu
beVDaveMdare utatxra}

\begin{shl}
samAseVneYva nidiRSeTxV kAraNatAvxvisheVSataH || \\
anoyxVnayxkAraNatAvxcacx shurxteyxVha jAcnxnakamaRNiV ||  122 ||  
\end{shl}

\begin{artha}
jAcnxnakamaRgaLu (pUvaRparxjecnxge) kAraNaveMbudu
samAnavAgiruvudariMdalU (pUvaR parxjecnxgU ivugaLigU)
parasapxrakAraNatavxviruvudariMdalU shurxtiyiMda samAsa padadiMdaleV
nideRVshisalapxTiTxve.
\end{artha}

\begin{shl}
pUvaRparxjAcnxta udUBxtiviRdAyxyAH kamaRNoV yataH || \\
tABAyxM ca BAvanoVdUBxtiniRdeVRshoV\s toV yathoVditaH ||  123 ||  
\end{shl}

\begin{artha}
yAvudariMda pUvaRparxjecnxyiMda jAcnxnavU kamaRvU huTiTx koMDiruvudo
matutx adeV jAcnxna-kamaRgaLiMda pUvaR parxjecnxyeMbuva BAvaneya
utapxtitxyAgi Aguvudu, adariMda hiMde heVLidaMte nideRVshamADiruvudu.
\end{artha}

\vishaya{BAvanA eMdareVnu ?}

\begin{shl}
kamaRNoV BujayxmAnasayx parisheVSoV hi BAvanA || \\
mUlaM ca jAyamAnasayx parxdhAnaM teVna BaNayxteV ||  124 ||  
\end{shl}

\begin{artha}
anuBava mADalapxDuva kamaRda parisheVSaveV (Palada vAsaneyeV) BAvanA
eMbudu ideV huTuTxva kamaRkUkx mUla, adariMda adu muKayxveMdu
heVLalapxDuvudu.
\end{artha}

\begin{shl}
pariceCxVtirxrXV vinimARtirxV vidAyx loVkAnatxrasayx hi || \\
vikataqR kamaR voVDhirxV ca pUvaRparxjecnxVha pUvaRyoVH ||  125 ||  
\end{shl}	

\begin{artha}
videyxyu beVre deVhavanunx saNaNxdAgi mADuvudu matutx nimiRsuvudu,
kamaRvu vikAragoLisuvudu  modalina eraDaralilx I pUvaRparxjecnxyu
oyuyxvudu.
\end{artha}

\vishaya{jiVvagati viSayadalilx matAMtaragaLu -}

\begin{shl}
nAnAvikalapxsadABxvAtasxMdeVhoV gamanaM parxti || \\
\footnotemark[1]pakiSxVva vaqkASxtikxM tAvatapxriciCxnonxV varxjatayxyamf ||  126 ||  
\end{shl}
\footnotetext[1]{digaMgabara sidAdhxMta jiVvanu AyAya deVhadaSuTx
AkAravuLaLxvanu giLiyu maradiMda marakekx (kupapxLisi) hAri
hoVguvaMte oMdu deVhadiMda matotxMdu deVhakekx hoVgi seVruvanu.}

%%%% footnote shloka
\begin{artha}
nAnA matagaLu iruvudariMda loVkAMtaragamana viSayadalilx
saMshayavu baruvudu  (A matagaLalilx modalaneyadu)- digaMbaramata I jiVvanu
pariciCxnanxnAgidudx (deVhadaSuTx parimANaviruvavanAgi) hakikxyu
maradiMda marakekx hAri hoVguvaMte Itanu beVre deVhakekx hoVguvano ?
\end{artha}

\vishaya{deVvAtAvAdi mata, sAMKayx mata}

\footnotetext[2]{deVvatAvAdi mata - deVvateyu yAva deVhadiMda kUDida
jiVvananunx paraloVkakekx oyuvudo adeV deVhavu AtivAhika deVhaveMdu
heVLalapxDuvudu, I deVhavanunx biTaTxkUDaleV AtivAhika
shariVraveMbudu huTiTxkoLuLxvudu adariMda jiVvAtamxnanunx deVvateyu
BoVgadeVhakekx oyuyxvudu.}
\begin{shl}
\footnotemark[2]AtivAhikadeVheVna kiMvA deVhAnatxraM parxti || \\
gati\footnotemark[3]viRkAsasaMkoVcw BinanxkumaBxsathxdiVpavatf || 
\end{shl}
\footnotetext[3]{iMdirxyagaLu ahaMkAra tatavxda
pariNAmagaLu, ahaMkAravu jaganamxMDaladalelxV vAyxpakavAdudariMda
iMdirxyagaLu vAyxpaka, hiVgeyeV AtamxnU vAyxpaka, IveraDU elalxgU
hoVguvudilalx, hosa deVhavu baMdalilx kamARnusAravAgi vaqtitxgaLu
AdeVhadalilx huTuTxtatxveyeMbudu sAMKayxmata Atamxnu saMkoVca
vikAsagaLanunx hoMduvudeV gatiyeMbudu kelavaru, modalu
gaDigeyalilxTaTx diVpavu saMkucitavAgi aSaTxralelxV parxkAshisuvudu,
oDedu hoVdalilx diVpavu parxBeyiMda visatxrisuvudu, hAgeye eMdu
deVhadalilx saMkucitavAgi anaMtara vikAsagoLaLxvanu.}

%%% footnote shloka
\begin{artha}
AtivAhika deVhadiMda Itanu beVre deVhavidedxDege
oyayxlapxDuvane ? athavA oDedu hoVda gaDigeyalilxruva diVpadaMte
jiVvanige vikAsa, saMkoVcagaLeV gatiyo ?
\end{artha}

\vishaya{veYsheVSika mata}

\footnotetext[4]{neYsheVSika matadalilx Atamxnu vAyxpakanu adariMda Atamxnu
hoVguvudilalx, adare manasusx mAtarx deVhAMtaraviruvalilxge
hoVguvudu, iMdirxyagaLu hosadAgiye saqSiTxyAguvavu eMdu heVLiruvudu,
Atamxnu siSikxyanAdarU avanalilx vaqtitxyu
huTuTxvudu. cAvARkamatadalilx deVhaveV BasamxvAguvudu jiVvavu
deVhAMtarakekx hoVguvudeMbudeV iruvudilalx.}
\begin{shl}
mataM vA kiM\footnotemark[4] manoVmAtarx deVhAdedxVhAnatxraM varxjeVtf ||  127 ||  
\end{shl}

%%% footnote shloka 
\begin{artha}
athavA keVvala manasesxV deVhadiMda matotxMdu deVhakekx
matotxMdu deVhakekx saMcarisuvudo ?
\end{artha}

\vishaya{aupa niSada dashaRna}

\footnotetext[5]{sidAdhxMtadalilx elalx iMdirxyagaLU vAyxpakavAdavu, AdarU
kamARnusAravAgi AyAya deVhagaLalilx saMkoVca vikAsagaLanunx hoMduvavu.}
\begin{shl}
kiMvA\footnotemark[5] savaRgatAnAM sAyxtakxraNAnAmihA\s \s tamxni || \\
shurxtakamARnuroVdheVna vaqtitxhAnuyxdaBxvw kavxcitf ||  128 ||  
\end{shl}

%%% footnote shloka
\begin{artha}
savaRvAyxpakavAda iMdirxyagaLige I sidAdhxMtadalilx
Atamxnalelx deVvatAjAcnxna, kamaR ivugaLanunx anusarisi vaqtitxnAsha
matutx vaqtitxya utapxtitxgaLu oMdu kaDeyAguvavo.
\end{artha}

\vishaya{sidAdhxMta -}

\begin{shl}
iti BUrivikalApxyAM BUmw sidAdhxnatx ucayxteV || \\
ananAtxH savaR EveYteV vAknamxnaHpArxNalakaSxNAH ||  129 ||  
\end{shl}

\begin{artha}
I riVti aneVka vikalapxgaLige sAthxnavAda (loVkAMtara gamana
viSayadalilx) sidAdhxMtavu heVLalapxDuvudu. vAkukx, manasusx, pArxNa
eMbudavu elalxvU vAyxpakavAgive.
\end{artha}

\begin{shl}
pArxNikamARnuroVdheVna pulxSAyxdiparimANatA || \\
iti shurxtuyxkitxtasetxVSAM parxtipArxNi yathAmati || \\
yathAkamaR gatiH sAthxnaM pariceCxVdoV\s tha visatxqqtiH ||  130 ||  
\end{shl}

\begin{artha}
pArxNigaLa kamARnusAravAgi hutatxda huLu modalAda parxmANavuLaLxdAdxgi
iMdirxyagaLu Aguvavu eMdu shurxtiye (``samaH pulxSiNA samoV mashakeVna samoV nAgeVna sama EBisitxrXBi loVRkeYH samoV\s neVna siveVRNa'') eMdu heVLiruvudariMda parxtiyoMdu pArxNigU avugaLa jAcnxnaviruvaMteyU, kamaRdaMteyU gamana
matutx sithxti, parimANa, (saMkoVca) vikAsa ivugaLu Agutatxve.
\end{artha}

\begin{shl}
sAvxtanatxrXyXM pAratanatxrXyXM vA\s NimAdeyxYshavxyaRmeVva vA || \\
karaNAnAmidaM savaRM jAcnxnakamARdiheVtukamf ||  131 ||  
\end{shl}

\begin{artha}
sAvxtaMtarxyXvAgali, parataMtarxteyAgali aNimAdi aSeTxYshavxyaRveV
Agali idelalxvU iMdirxyagaLige upAsane-kamaR itAyxdi nimitatxvAgi baruvudu.
\end{artha}

\begin{shl}
asamxdAdeVyaRthA tadavxdivxdAyxkamARnuroVdhataH || \\
anAtamxnoV\s pi cA\s \s nanatxyXM na savxtaH sidadhxmAtamxvatf ||  132 ||  
\end{shl}

\begin{artha}
namamxMthavarige heVge jAcnxna matutx kamARnu sAravAgi anaMtavxvu
iruvudo hAgeyeV anAtamx vasutxvigU iruvudu, AtamxnaMte adu savxtaH
sidadhxvAgiruvudalalx.
\end{artha}

\begin{shl}
itiVmaM pakaSxmAshirxtayx shurxtAyx daqSATxnatx ucayxteV ||  133 || \\
taqNAgarxsAthx jalUkeVha pArxpayx tasAmxtatxqqNAnatxramf || \\
saMharatAyxtamxnA\s \s tAmxnaM pUvaRsAmxtatxtarx sA karxmAtf ||  134 ||  
\end{shl}

%%% footnote shloka
\begin{artha}
I riVtiyAda I \footnote[1]{iMdirxyagaLu vAyxpakaveMba pakaSxvanunx Asharxyisi
aMtaHkaraNadalilxruva BAvaneyu muMdina deVhavanunx muTiTxkoMDu AgaleV
deVvanAgidedxVneMdu BAvisi naMtara hiMdina deVhavanunx biDuvudu, I
viSayakekx daqSATxMtavanunx `tadayxthAtaqNA jalUkA' itAyxdi shurxtiyu koTiTxruvudu.}pakaSxvanunx avalaMbisi shurxtiyiMda
daqSATxMtavu heVLalapxDuvudu  jigaLeyu hulilxna kaDiDxya
tudiyalelxVniMtu beVre hulalxnunx kacicxkoMDu punaH tAneV tananxnunx
hiMdina hulilxniMda biDisuvudu, anaMtara karxmavAgi muMdina
hulukaDiDxyalilx seVrikoLuLxvudu.
\end{artha}

\section*{baq. a. 4 - bArx. 4,  3 kaMDike}

\begin{shl}
tadayxthA taqNajalAyukA taqNasAyxnatxM gatAvxnayxmAkarxmamAkarxmAyxtAmxnamupasaM harateyxVvameVvAyamAtemxVdaM shariVraM nihatAyx\s vidAyxM gamayitAvxnayxmAkarxmamAkarxmAyxtAmxnamupasaMharati || 3 ||
\end{shl}

\begin{shl}
yatheYvameVvamAtemxVdaM shariVraM kamaRNaH kaSxyAtf || \\
nihatAyxceVSaTxmApAdayx sAvxtamxliknogxVpasaMhaqteVH ||  135 ||  
\end{shl}

\begin{shl}
gamayitAvx tathA\s vidAyxM jADayxM niHsaMjacnxtAmidamf ||  \\
atoV\s nayxmAkarxmaM deVhaM pArxpayx BAvanayA\s cnijxtaH || \\
pUvaRdeVhasathxmAtAmxnaM saMharatAyxtamxnA\s \s tamxni ||  136 ||  
\end{shl}

\begin{shl}
jalUkAvadagxtiriyamAtamxnaH parxtipAditA || \\
tatarx deVhAnatxrAramaBx upAdAnaM kimAtamxnaH ||  137 ||  
\end{shl}

\begin{artha}
jigaLeyaMte Atamxna gamanavu parxtipAdisalapxTiTxruvudu, Atamxnige
beVre deVhavu huTuTxvudakekx upAdAna kAraNa yAvudu ? eMbudanunx
heVLabeVku.
\end{artha}

\vishaya{hosa deVhakekx upAdAna kAraNa vicAra}

\begin{shl}
upamaqdoyxVpamaqdAyxsayx nitoyxVpAtatxM kimiSayxteV ||  \\
punaH punarapUvaRM vA deVhAramABxya kalapxteV ||  138 ||  
\end{shl}

\begin{artha}
I Atamxnige anAdiyAgi (tAnu) sivxVkarisidadx (paMcaBUtagaLu) (padeV
padeV deVhavu nAshavAgi nAshavAgi) matotxMdu deVhavanunx huTiTxsalu
samathaRvAguvudo ? athavA padeV padeV hosadAgiruva (paMcaBUtagaLe)
deVhAraMBakekx samathaRvAguvudo ?
\end{artha}

\vishaya{utatxra}

\begin{shl}
iti daqSATxnatxvacasA niNaRyoV\s soyxVpavaNayxRteV ||  139 ||
\end{shl}

\begin{artha}
idanunx daqSATxMta vAkayxdiMda niNaRya mADiruvudanunx vaNiRsuvudu ||
\end{artha}

\section*{baq. a.4. bArx. 4. kaMDike 4.}

\begin{shl}
tadayxthA peVshasAkxriV peVshasoV mAtArxmapAdAyAnayxnanxvataraM kalAyxNataraM rUpaM tanuta EvameVvAyamAtemxVdaM shariVraM nihatAyxvidAyxM gamayitAvxnayxnanxvataraM kalAyxNataraM rUpaM kuruteV pitarxyXM vA gAnadhxvaRM vA deYvaM vA pArxjApatayxM vA bArxhamxM vAneyxVSAM vA BUtAnAmf || 4 ||
\end{shl}

\section*{vAtiRka}

\begin{shl}
peVshasAkxriV yathA mAtArxmupAdAyeVha peVshasaH || \\
vimaqdayx racanAM pUvARM kuruteV racanAnatxramf ||  140 ||  \\
navAnanxvataraM rUpaM kalAyxNatarameVva ca || \\
kalAyxNAditi daqSATxnotxV yathA tadavxdihApi ca ||  141 ||  
\end{shl}

\begin{artha}
suvaNaRkAranu cinanxda oMdu BAgavanunx tegedukoMDu modalu mADida
racaneyanunx (ABaraNavanunx) aLisi matotxMdu racaneyanunx
(ABaraNavanunx) heVge mADuvano, aMdare idadx hosa rUpakikxMtalU
matatxSuTx hosadAda rUpavuLaLx; modalina aMdakikxMtalU hecucx
aMdavAgiruva rUpadalilxruvaMte mADuvanoV, eMdu I daqSATxMtavu heVgo
hAgeye ililxyU kUDa iTuTxkoLaLxbeVku.
\end{artha}

\begin{artha}
navaneMdu heVLabeVkAgidadxrU navataraveMdu heVLidudx heVge ? - eMdare -
\end{artha}

\begin{shl}
ApeVkaSxyX navamAdAnaM tatAkxyaRM navamucayxteV || \\
shurxtAyx navataraM rUpaM kalAyxNatarameVva ca ||  142 ||  
\end{shl}

\begin{artha}
paMcaBUtagaLa kAyaRvAda hosarUpada daqSiTxyiMda adakukx hosadAda
kAyaRvAda shariVragarxhaNavanunx shurxti navatara kalAyxNatara
rUpaveMdu heVLiruvudu.
\end{artha}

\vishaya{I viSayakekx mUtARmUtaRbArxhamxNaveV parxmANa -}

\begin{shl}
nitoyxVpAtAtxni BUtAni karaNAni ca deVhinaH || \\
devxV vAva barxhamxNoV rUpeV iti pUvaRmavAdiSamf ||  143 ||  
\end{shl}

%%% footnote shloka
\begin{artha}
\footnote{``yacacxkuSxreVSa tapatiVti ca sUthxloV deVhoV lakiSxtoV, yoV\s yaM dakiSxNeV\s kaSxnf puruSoV ya ESa Etasimxnf manaDxleV puruSa iti ca sUkoSxmXV deVhoV gaqhiVtaH, tatarx sa teV heyxVSa rasa satxyXsayx heyxVSa rasa iti ca tayoVBURta paMcakArabadhxtavx mukatxmf atArxpi nitoyxVpAtatxmeVva tadArabaBxkamiSaTxmiti'' eMdu AnaMdagirigaLu vivarisidaMte `yacacxkuSxreVkaSxtapati' eMbuvalilx sUthxla deVha sUcitavAgide, `yo\s yavf... maMDaleV
puruSaH' eMbuvalilx sUkaSxmX deVhavu garxhisalapxTiTxde. `adaralilx
satoVheyxVSarasaH tayxsayxheyxVSarasaH' eMdu averaDU
paMcaBUtagaLiMda huTiTxkoMDideyeMdu tiLidu baruvudu. ideV mUtAR
mUtaR bArxhamxNadalilx heVLalapxTiTxruvudu.}nitayxvU tAnu sivxVkarisidadx paMcaBUtagaLu matutx iMdirxyagaLu
shariVrige vivakiSxtaveMbudanunx ``devxVvAda barxhamxNoV rUpeV'' eMba
mUtARmUtaRbArxhamxNadalilx modaleV heVLiruvenu.
\end{artha}

\begin{shl}
upamaqdoyxVpamaqdeyxYSAM racanAM pArxkatxniVM punaH || \\
yathAkamaR yathAjAcnxnaM kuruteV racanAnatxramf ||  144 ||  
\end{shl}

\begin{artha}
I paMcaBUtagaLa hiMdina racaneyanunx aLiyisi aLiyisi kamaR, matutx
jAcnxna (upAsane) ivugaLige anusAravAgi beVre racaneyoMdanunx mADuvanu.
\end{artha}

\vishaya{`pitarxyXMvA' itAyxdi shurxti vAkayxvanunx vAyxKAyxnisuvudu -}

\begin{shl}
pitArxdiyoVgayxM pitArxyXdi huyxtatxmAdhamamadhayxmamf ||  145 ||  \\
pitArxyXdiloVkeVSAvxtAmx\s yaM yathAkamaR yathAshurxtamf ||  \\
tanuteV deVhajAtAmi BUrirUpANayxvidayxyA ||  146 ||  
\end{shl}

\begin{artha}
`pitarxyX' muMtAdavu pitaqdeVvateyeV modalAda (deVvategaLige)
yoVgayxvAda utatxma, madhayxma, adhama rUpavAda shariVravanunx I
Atamxnu pitaqloVka modalAda loVkagaLalilx tananx-kamaRdaMteyU
upAsaneyaMteyU racisuvanu, shariVra samudAyagaLu aneVkarUpavAgi,
avideyxyiMda Agive.
\end{artha}

\begin{shl}
na tu ceYtanayxvatAsxkASxtasxMsAroV\s sayx savxtoV mataH || \\
itayxthaRparxtipatatxyXthaRmAjagAmoVtatxraM vacaH ||  147 ||  
\end{shl}

\begin{artha}
Adare ceYtanayxdaMte saMsAravu I Atamxnige savxtaH (savxrUpadalilx)
iruvudilalx - I viSayadalilx jAcnxnavuMTAgalu muMdina vacanavu baMdiruvudu.
\end{artha}

\begin{shl}
yacAcxsayx vAsatxvaM rUpaM yacAcxvidoyxVtathxmAtamxnaH || \\
sa vA itAyxdinA tasayx niNaRyaH kirxyateV\s dhunA || 148 ||
\end{shl}

\begin{artha}
yAvudu Atamxna vAsatxvavAda rUpa, yAvudu ? avideyxyiMda
huTiTxkoMDadudx, I eraDara niNaRyavanunx Iga mADuvudu.
\end{artha}

\begin{shl}
taninxNaRyAdasheVSoV\s thAR\s niNiVRtaH sAyxtakxthaM nivxti ||  148 ||  
\end{shl}

\begin{artha}
adara niNaRyadiMda elAlx veVdAthaRvanunx niNaRya mADidaMtAguvudeMbudu
heVge ? eMdare -
\end{artha}

\begin{shl}
saMsAriV yoV yathoVketxVna garxnethxVna parxtipAditaH || \\
tadagxqqhiVteyxY sashabodxV\s yaM tatasxmXqqtayxthaRM tathAca veY ||  149 ||  
\end{shl}

\begin{artha}
yAva saMsAriyAda Atamxnu hiMde heVLida garxMthadiMda parxti
pAdisalapxTiTxruvano, avananunx ililxgarxhisalu `sa', eMba shabadx
vidU, hAgU avana samxraNegAgi veYshabadxvU baMdide.
\end{artha}

\section*{baq - 4 - 4 - kaMDike 5}

\begin{shl}
sa vA ayamAtAmx barxhamx vijAcnxnamayoV manoVmayaH pArxNamayashacxkuSxmaRyaH shorxVtarxmayaH paqthiviVmaya ApoVmayoV vAyumaya AkAshamayasetxVjoVmayoV\s teVjoVmayaH kAmamayoV\s kAmamayaH korxVdhamayoV\s korxVdhamayoV dhamaRmayoV\s dhamaRmayaH savaRmayasatxdayxdeVtadidamamxyoV\s doVmaya iti yathAkAriV yathAcAriV tathA Bavati sAdhukAriV sAdhuBaRvati pApakAriV pApoV Bavati puNayxH puNeyxVna kamaRNA Bavati pApaH pApeVna ||athoV KalAvxhuH kAmamaya EvAyaM puruSa iti sa yathAkAmoV Bavati tatakxrXtuBaRvati yatakxrXtuBaRvati tatakxmaR kuruteV yatakxmaR kuruteV tadaBisamapxdayxteV || 5 ||
\end{shl}

\begin{shl}
anAtamxBUta EtasimxnAkxyaRkAraNalakaSxNeV || \\
saMsAreV parxthateV yoV\s thaR AtamxnA\s nanayxmAnagaH ||  150 ||  
\end{shl}

\begin{artha}
I anAtamx rUpavAda I kAyARkAraNa lakaSxNavuLaLx saMsAradalilx yAva
athaRvu savxyaM parxsidadhxvAgi beVre parxmANavanunx hoMdilalxvo
(aMdare beVre parxmANadiMda tiLiyade) iruvudo adanunx Iga heVLuvudu.
\end{artha}

\vishaya{meVle heVLida shurxtiyalilx Atamx shabadxda athaR -}

\begin{shl}
yatAsxkiSxkw yathoVkatxsayx BAvABAvw parxsidhayxtaH || \\
saMsAravasutxnaH soV\s yamAtemxVtayxtArxBidhiVyateV ||  151 ||  
\end{shl}

\begin{artha}
yAva sAkiSxyiMdale hiMde heVLida saMsAra padAthaRda asitxtavxvU
nAsitxtavxvU parxsidadhxvAgudo avaneV niviRkAra ceYtanayx rUpanu I
`ayamAtAmx' eMbuvalilx heVLalapxDuvanu.
\end{artha}

\vishaya{avanu kUTasathx niviRkAraveMbudu heVgeMdare -}

\begin{shl}
vayxBicAroV na yasAyxsitx saveVRSu vayxBicAriSu || \\
tadavaSaTxmaBxtaH sidedhxVvayxRBicArasayx savaRdA ||  152 || 
\end{shl}

\begin{artha}
elalxvU vayxBicaritavAgidadxrU yAva Atamxkekx vayxBicAravu (aBAvavu) iruvudilalxvo matutx A Atamxna avalaMbaneyiMdaleV yAvAgalU vayxBicAravU sidadhxvAguvudo (avanu niviRkAranu).
\end{artha}

\vishaya{iMtaha savaRsAkiSxyu `ayaM' eMdu parxtayxkaSxdaMte heVge heVLalapxTiTxruvanu ?}

\begin{shl}
parxtayxkatxyA\s sayx sAkASxtAtxvXdakirxyAkArakatavxtaH ||  \\
ananayxboVdhamAnatAvxdayamituyxcayxteV tataH ||  153 ||  
\end{shl}

\begin{artha}
I Atamxnu \footnote{budAdhxyXdigaLigU aMtara sAkiSxyAgi aparoVkaSxnAgiruva Atamxnu `ayamf eMdu parxtayxkaSxvAgiruvaMte heVLalapxTiTxruvaneMdathaR}parxtayxkf AgiruvudariMda aparoVkaSxnAgiruvanAgiruvudariMdalU kirxyAkAraka shUnayxnAgiruvudariMdalU savxtaMtarx boVdha parxmANavAdadxriMdalu ayaM eMdu ililx heVLalapxDuvanu.
\end{artha}

\vishaya{`ayamf' eMdu GaTAdi viSayavanunx heVLidaMte Atamxnanunx heVLidare pariciCxnanxnAguvudilalxveV ? eMdare -}

\begin{shl}
niHsheVSAnAtamxtadedhxVtunirAkaraNavatamxRnA || \\
AtamxtavxmAtamxnaH sidheyxVnAnxnayxsheVSamanAtamxvatf ||  154 ||  
\end{shl}

\begin{artha}
samasatx anAtamx (jaDa vasutx)vanUnx adara kAraNavanunx nirAkarisuva
mAgaRdiMda (aMdare niVti neVti eMba riVtiyalilx) Atamxnige Atamxtavxvu
(apariciCxnanxteyu) sididhxsuvudu, anAtamx vasutxvinaMte idu
matotxMdakekx adhiVnavalalx.
\end{artha}

\vishaya{Atamxnige beVreyAgi kAyaRkAraNagaLu iralu A mUlaka AtamxniNaRyavu heVge Aguvudu? :- eMdare}

\footnotetext[1]{`neVti neVti' eMdu sakala anAtamx visheVSavanunx
nirAkarisidadxriMda niviRsheVSa vasutx loVkAtirikatxvAgi
sididhxsiruvudu, I riVti asAdhAraNa sididhxyiMdaleV Atamxna
sididhxyeMdu tiLiyabeVku, AdarU I sholxVkada vAyxKAyxna anuvAda,
taqpitxkaravAgilalx, namage tiLididadxnunx baredidedxVve.}
\begin{shl}
\footnotemark[1]asAdhAraNasidedhxyXYva sididhxH sAyxdAtamxvasutxnaH || \\
yatoV\s ta AtamxvasetxvXVva kAyaRkAraNavajajxgatf ||  155 ||  
\end{shl}

%%\footnote shloka
\begin{artha}
asAdhAraNa vasutx sididhxsuvudariMdaleV Atamx vasutxvina
sididhxyAguvudu, adariMda kAyaRkAraNagaLiruva jagatutx Atamxvasutxve
(idanunx biTuTx beVreyAgilalx).
\end{artha}

\footnotetext[2]{Atamxnige BinanxvAgiyU jagatitxna sithxtiyu sididhxsadu,
Atamxnige aBinanxvAgiyU sididhxsadu, jaDa, ajaDa eraDu vasutxgaLigU
EkatavxvUbAradu eMdu BAvavu.}
\begin{shl}
\footnotemark[2]na hiVdamAtamxnaH sAthxnaM tatoV\s nayxtArxpi vA\s shunxteV || \\
AtamxnoV\s vayxtireVkeVNa yatoV\s nAtAmx parxsidhayxti ||  156 ||  
\end{shl}

%%% footnote shloka
\begin{artha}
I jagatutx Atamxnalilx sithxtiyanunx hoMdide. adanunx biTuTx beVre
sathxLadalilx sithxtiyanunx hoMduvudilalx, matutx Atamxnige
aBinanxvAgiyU anAtamxvu sididhxsuvudU ilalx.
\end{artha}

\vishaya{anAtamxkekx BinanxvAgiruva AtamxnaMte anAtamxvU Atamxnige
beVreyAgirali eMdare -}

\begin{shl}
AtAmx tavxnAtamxparxtayxkAtxvXdavxyXtireVkaM na soV\s haRti || \\
sarxjiVva sapaRdaNADxdeVH sarxgavidoyxVtathxvasutxnaH || \\
sarxkatxtatxvXvayxtireVkeVNa sididhxnARnayxtarx kutarxcitf ||  157 ||  
\end{shl}

\begin{artha}
AtamxnAdaro anAtamxvasutxvina AMtaravasutxvAgiruvudariMda avananunx
biTuTx A anAtamxvasutx beVreyAgiralAradu,  udA:- hagagxdalilx kANuva
sapaR, koVlu muMtAdavu hagagxda-ajAcnxnadiMda huTiTxkoMDa vasutxgaLu,
adariMda hagagxda tatavxrUpavanunx biTuTx beVreyAgi beVreyAva
sathxLadalUlx sididhxsuvudilalx, eMbuvate.
\end{artha}

\vishaya{jagatutx yAvariVtiyalUlx sididhxsade hoVdare shUnayxdalelxV payARvasAnavAgali! eMdu keVLidare -}

\begin{shl}
na cABAvAvasAyeyxVtadaBAvasAyxpi BAvavatf || \\
parxtayxknAmxterxYkayAthAtAmxyXdashurxtatAvxnanx cAthaRtaH ||  158 ||  
\end{shl}

\begin{artha}
matutx I jagatutx shUnayxdalelx nele nilalxdu, aBAvakukx
BAvavasutxvige heVLuvaMte parxtayxgAtamxvoMdeV nijasavxrUpavAdadxriMda
shurxtiyalilx (shUnayxveMbudAgi) heVLilalxvAdadxriMdalU, athARpatitx
parxmANadiMdalU (shUnayxpayaRvasAnavAguvudilalx).
\end{artha}

\begin{shl}
anUdayx niKilaM vishavxM tatatxtatxvXparxtipatatxyeV || \\
AtemxYveVti shurxtaM yasAmxnAnxtoV\s nayxtikxMcidiSayxteV ||  159 ||  
\end{shl}

\begin{artha}
elalx jagatatxnunx anuvAdamADi adara tatavxvu tiLiyuvudakAkxgiye
AtamxveV eMdu sapxSaTxpaTiTxde. yAvudariMda I AtamxnigiMta
beVreyoMdanunx opupxvudilalxvo.
\end{artha}

\begin{shl}
parxtAyxKAyxya na cA\s \s tAmxnamanAtAmx vayxtiricayxteV || \\
vayxtireVkasavxBAvatAvxnAnxpi cA\s \s tamxni sidhayxti ||  160 ||  
\end{shl}

\begin{artha}
matutx Atamxnanunx nirAkarisi savxtaMtarxvAgi anAtamx vasutx
beVreyAgiradu, matutx Atamxnalilx aBinanxvAgiyU sididhxsuvudilalx,
EkeMdare! (ceYtanayxkekx) beVreyAda (jaDa) savxrUpavAgiruvudu (sididhxsuvudilalx).
\end{artha}

\begin{shl}
parxtAyxcaSeTxV shurxtirataH savaRM neVtiVti cA\s \s tamxni || \\
savaRmAtemxVti ca tathA vayxtireVkaM niSeVdhati ||  161 ||  
\end{shl}

\begin{artha}
idariMda `neVti neVti' eMba shurxtiyu Atamxnalilx elalxvU ilalxveMdu
nirAkarisiruvudu, `savaRmAtAmx' eMba shurxtiyu jagatutx
vayxtirikatxveMbudanunx niSeVdhiside.
\end{artha}

\begin{shl}
apUvARnaparAnanatxrAbAhayxM barxhamxlakaSxNamf || \\
ukAtxtamxvasutxsAvxBAvAyxdAtAmx barxhemxVtayxtoV vacaH ||  162 ||  
\end{shl}

\begin{artha}
barxhamxna lakaSxNavu apUvaR, anapara, anaMtara abAhayxveMbude, hiMde
heVLida Atamx vasutxvina savxBAvave (I barxhamxvAgiruvudariMda) `AtAmx
barxhamx'veMdu vacanavu.
\end{artha}

\begin{shl}
parxtayxkatxvXM barxhamxNasatxtatxvXM barxhamxtavxM cA\s \s tamxnasatxthA || \\
paroVkaSxdavxyahAneVna hAyxtAmx barxhemxVti boVdhayxteV ||  163 ||  
\end{shl}

\begin{artha}
barxhamxda tatavx parxtayxkf savxrUpa, Atamxna tatavxvU barxhamx
savxrUpa, paroVkaSx matutx BeVda ivugaLanunx biDuvudariMda Atamxnu
barxhamxveMdu boVdhisalapxDuvanu.
\end{artha}

\vishaya{hiMdina vAtiRkada athaRvanenx sapxSaTxpaDisuvudu -}

\begin{shl}
avAyxvaqtAtxnanugatoV barxhamxshabAdxthaR iSayxteV ||  \\
nA\s \s tamxnoV\s nayxtarx laBoyxV\s sw nApAyxtAmx barxhamxNoV\s nayxtaH ||  164 ||  
\end{shl}

\begin{shl}
AtamxnoV\s pi paroVkaSxtavxM barxhamxNoV\s vidayxyA yathA || \\
AtamxnaH sadivxtiVyatavxM barxhamxNoV\s pi tathA mama ||  165 ||  
\end{shl}

\begin{artha}
visheVSa, sAmAnayxgaLu ilalxda vasutxveV barxhamx shabadxda
athaRveMbudu saMmatavAgide, I athaRvu Atamxnanunx biTuTx beVre
sathxLadalilx siguvudilalx, matutx Atamxnu barxhamxkikxMta
beVreyAgilalx.
\end{artha}

\vishaya{AdarU barxhamxna paroVkaSxte matutx devxYta saMbaMdha
ivugaLanunx heVge nirAkarisuvudu ? eMdare}

\begin{shl}
atoV\s vidAyxsamuciCxtwtx yathAvasatxvXvaboVdhataH || \\
AtAmx barxhemxYva sanenxVSa barxhAmxpeyxVti savxtoV\s davxyamf ||  166 ||  
\end{shl}

\begin{artha}
AtamxnigU barxhamx paroVkaSxvAgiruvudu avideyxyiMda eMbudu heVgo
hAgeye Atamxna sadivxtiVyatavxvu barxhamxnigU avideyxyiMda Agiruvudu.
\end{artha}

\begin{artha}
adariMda nijavAda vasutxjAcnxnadiMda ajAcnxnavu nAshavAguvalilx
Atamxnu barxhamxveV Agidudx Itanu savxtaH adavxyavAda barxhamxvanenxV
seVruvanu (EkavAguvanu).
\end{artha}

\vishaya{vijAcnxnamaya itAyxdi vAkayxda tAtapxyaR -}

\begin{shl}
yathoVkatxboVdhavirahAdasAyxnathaRparaMparA || \\
vijAcnxnAdayxBisaMbanodhxV yathA tadadhunoVcayxteV ||  167 ||  
\end{shl}

\begin{artha}
hiMde heVLida jAcnxnavilalxdiruvudariMda Itanige anathaR
paraMpareyanUnx vijAcnxna muMtAda guNagaLa saMbaMdhavanunx heVge
heVLabeVko adanunx Iga heVLuvudu.
\end{artha}

\begin{shl}
AtAmx barxhemxYva sanenxVSa dhameYRyARvadiBxranivxtaH || \\
ajAcnxnAtasxMsaratayxtarx vaNayxRteV tatasxmAsataH ||  168 ||  
\end{shl}

\vishaya{(yeV\s sayxbaMdhana saMjacnxkAluvAdhi BUtAH, yeYH saMyukatx satxnamxyoV\s ya miti viBAvayxteV, teV padAthARH pucnijxVkaqtayx iheYkatarx parxtinidiRshayxnetxV  - itAyxdi BASayxda yoVjane -)}

\begin{artha}
I Atamxnu barxhamxveV Agidudx elAlx saMsAra dhamaRgaLiMda kUDi
ajAcnxnadiMda ODADuvanu eMbudanunx ililx (vijAcnxnamayavAkayxdalilx)
saMkeSxVpavAgi vaNiRside.
\end{artha}

\vishaya{I vaNaRneyiMda siguva parxyoVjanaveVnu ?}

\begin{shl}
yatoV\s vidAyxnavxyeV\s sheVSasaMsArAnathaRsaMgatiH || \\
tadadhxvXsAtxvAtamxnasatxsAmxtupxruSAthaRH samApayxteV ||  169 ||  
\end{shl}

\begin{artha}
yAvudariMda avidAyx saMbaMdhavidadxre saMsArada sakala anathaR
saMbaMdhavuMTAguvudoV, adu nAshavAdalilx Atamxnige adariMda
puruSAthaRvu pUNaRvAguvudu.
\end{artha}

\begin{shl}
avidAyxmAtarxheVtUkAtxyX hAyxtamxnoV\s nathaRsaMgatiH || \\
itayxsayx parxtipatatxyXthaRM paroV garxnothxV\s vatAyaRteV ||  170 ||  
\end{shl}

\begin{artha}
avideyxyoMdeV muKayxkAraNaveMdu heVLuvudariMda Atamxnige anathaR
saMbaMdhavuMTAguvudeMba I viSayavanunx tiLisalu muMdina garxMthavu
baMdiruvudu.
\end{artha}

\vishaya{saMsAra baMdhavu ajAcnxnadiMda eMbudakekx yukitx -}

\begin{shl}
yadayxvideyxYkaheVtusAyxtasxMsAritavxM tadA\s \s tamxnaH ||  \\
vidAyxthoVR\s yaM samAramoBxV yujayxteV nAnayxthA sati ||  171 ||  
\end{shl}

\begin{artha}
avideyxyoMdeV nimitatxvAgidudx saMsAra saMbaMdhavu Atamxnige
baMdiruvudAdare videyxgAgi I shAsAtxrXraMBa mADiruvudu yukatxvAguvudu,
anayxthA iruvalilx yukatxvAguvudilalx.
\end{artha}

\vishaya{A vivakiSxtavAda pAThakarxmavanunx biTuTx `pArxNamayaH'
eMbudara athaRvanunx vivarisuvaru -}

\begin{shl}
pArxNAtamxtAvxBimAniV sanayxtaH pArxNaH parxsUyateV || \\
pArxNapArxNoV\s pi sanomxVhAtApxrXNanAdi parxpadayxteV ||  172 ||  
\end{shl}

\begin{artha}
(pArxNavasutxvinalilx AtAmxBimAna mADikoMDavanAgi) yAvudariMda
pArxNavAyuvu huTuTxvudoV avanu pArxNakUkx pArxNanAgi (jiVvadAna
mADuvavanAgi) idadxrU ajAcnxnadiMda pArxNavasutxvinalilx
AtAmxBimAnavuLaLxvanAgidudx shAvxsoVcACxvXsa muMtAda
(kirxyegaLanunx) hoMdiruvanu.
\end{artha}

\vishaya{vijAcnxnamaya, matutx manoVmaya eMbudara vAyxKAyxna}

\begin{shl}
tatoV budidhxsamutapxtwtx vijAcnxnoV\s simxVtiBAvataH || \\
vijAcnxnamayatAmeVti sarxkasxpaRmayatAmiva ||  173 ||  
\end{shl}

\begin{artha}
alilxMda budidhxyu huTiTxdalilx nAneV vijAcnxnavAgidedxVneMdu
BAvaneyiMda hagagxvu sapaRrUpavanunx hoMduvaMte
vijAcnxnamayatavxvanunx hoMduvanu.
\end{artha}

\begin{shl}
manasoV garxhaNaM cAtarx budidhxvaqtutxyXpalakaSxNamf || \\
asubudidhxV yatoV heVtU saveVRSAminidxrXyAtamxnAmf ||  174 ||  
\end{shl}

\begin{artha}
manasusxnunx ililxtegedukoMDiruvudu budidhx vaqtitxgaLigelAlx gamaka,
kAraNaveVneMdare elAlx iMdirxyasavxrUpagaLigU pArxNa matutx budidhx
ivugaLu kAraNavAgive.
\end{artha}

\begin{shl}
kameVRnidxrXyANAM sAthARnAM pArxNaH kAraNamucayxteV || \\
sa Eva budadhxyXtishayaH shorxVtArxdeVrapi kAraNamf ||  175 ||  
\end{shl}

\begin{artha}
viSaya sahitavAda kameRVMdirxyagaLige pArxNavAyuvu
kAraNaveMdu heVLalapxDutatxde. adeV budidhxya atishayavu shorxVtArxdi
iMdirxyagaLigU kAraNavAgide.
\end{artha}

\vishaya{pArxNAdigaLu Atamxna kAyaRveMbudakekx EnukAraNa ?}

\begin{shl}
sAvxBayxsatxBAvanAtoV\s sayx shurxtakamARnuroVdhataH || \\
pArxNoV budidhxmaRnashacxkuSxHshorxVtArxdayxjacnxsayx jAyateV ||  176 || 
\end{shl}

\begin{artha}
tAnu aBAyxsamADi BAvaneyiMda matutx shAsoVtxrXkatx deVvatoVpAsane,
kamaR ivugaLige anusAravAgi ajAcnxniyAdavanige pArxNa, budidhx,
manasusx, cakuSx, shorxVtarx modalAda iMdirxyagaLu huTuTxtatxve.
\end{artha}

\vishaya{I parxkaraNadalilx pArxNamaya vijAcnxnamaya itAyxdiyAgi
kANuva mayaTf parxtayxyada athaR -}

\begin{shl}
pArxyAtheVR ca mayatf knecnxVyoV vikArAdeVniRSeVdhanAtf || \\
avijAcnxtAtamxtatatxvXsayx vikAroV vA\s satxvXdoVSataH ||  177 ||  
\end{shl}

\begin{artha}
pArxcuyaR (bAhuLayx)veMba athaRdalilx mayaTf parxtayxyavu
baMdideyeMdu tiLiyabeVku. EkeMdare ? vikArAdi athaRvanunx
niSeVdhisiruvudariMda. athavA Atamx tatavxvanunx ariyadiruvavanige
vikAravU (vikArAthaRvU) doVSavilalxdadxriMda irali.
\end{artha}

\begin{artha}
`EtasAmx jAjxyateV pArxNoV manaH sasxveVRMdirxyANi ca' eMba athavaRNa shurxtiyalilx AtamxniMda manasusx elAlx
iMdirxyagaLu huTiTxveyeMdu heVLide. AdarU Atamxnu paMcaBUtagaLanunx
saqSiTxsi avugaLa rUpadiMda pArxNavAgi jAcnxna
shakitxkirxyAshakitxgaLiMda elAlx iMdirxyagaLanunx saqSiTxsuvanu
eMdu athavaRNa shurxtiya athaR, pArxNaveV budidhx vilakaSxNavAda
AkAravuLaLxdAdxgi shorxVtArxdi iMdirxyagaLigU viSayagaLigU
kAraNavAguvudu, manasisxna sapxMdaneye calaneye devxYtaveMva
tAtapxyaR, pAThakarxmavu vivakiSxtavalalxdadxriMda pArxNabudidhxgaLa
kAyaRvAdadadxriMda iMdirxyagaLu avugaLa vaqtitxyeMdu heVLuvudeMdu
BAvavu.
\end{artha}

\begin{shl}
sapARdayoV yathA rajAjxvX vikArAH suyxraboVdhataH || \\
ajAcnxnAdAtamxnasatxdavxtetxVjoVbanAnxdivikirxyA ||  178 ||  
\end{shl}

\begin{artha}
hagagxda ajAcnxnadiMda sapARdigaLu heVge adara vikAravAguvavo hAgeye
Atamxna ajAcnxnadiMda teVjasusx, jala, paqthiviV muMtAda vikAragaLu
Aguvavu.
\end{artha}

\vishaya{parxdhAnAdigaLu kAraNaveMdu heVLabahudaSeTxV ajAcnxni Atamxnu
kAraNaneMdu Eke heVLabeVku? eMdare}

\begin{shl}
na hi veVdAnatxsidAdhxnetxV hayxjAcnxtAtAmxtireVkataH || \\
sAMKAyxnAmiva sidAdhxnetxV laBayxteV kAraNAnatxraNf ||  179 ||  
\end{shl}

\begin{artha}
veVdAMta sidAdhxMtadalilx ajAcnxtavAda Atamxnige beVreyAgi sAMKayxra
sidAdhxMtadalilxruvaMte kAraNavu siguvudilalxvaSeTx.
\end{artha}

\vishaya{A riVtiyAgi ajAcnxta Atamxnu-kAraNavAdalilx parxkaqtakekx Enu
parxyoVjana ? EMdare -}

\begin{shl}
pArxNAdimayatAM yAtAvx tadavxqqtitxVnAmaboVdhataH || \\
AtAmx\s katAR\s pi kataqRtavxmeVti tAsAM samudaBxveV ||  180 ||  
\end{shl}

\begin{artha}
paramAtamxnu pArxNAdirUpavanunx hoMdi avugaLa vaqtitxgaLige
kataRnalalxdidadxrU avugaLa utapxtitx viSayadalilx kaqtaqRtavxvanunx
hoMduvanu.
\end{artha}

\vishaya{cakuSxmaRya itAyxdi padagaLige athaR -}

\begin{shl}
cakuSxSashacxkuSxrapeyxVvaM yathA cakuSxmaRyasatxthA || \\
shorxVtArxdimayatA\s payxsayx vAyxKeyxVyA parxtAyxgAtamxnaH ||  181 ||  
\end{shl}

\begin{artha}
kaNiNxge kaNANxgidadxrU cakuSxmaRyanu heVgo hAgeye
shorxVtAdimayatavxvanunx parxtayxgAtamxnige vAyxKAyxnisabeVku.
\end{artha}

\begin{shl}
sa vijAcnxnamanaHpArxNacakuSxHshorxVtArxdi moVhajamf || \\
manAvxnoV\s vidayxyA\s \s temxYti tanamxyatavxM na tu savxtaH ||  182 ||  
\end{shl}

\begin{artha}
A Atamxnu budidhx, manasusx, pArxNa, cakuSx, shoVrxtarx modalAda ajAcnaxna kalipxta vasutxgaLanunx ajAcnaxnadiMdaleV AtamxveMdu tiLiyutAtx tanamxyatavxvanunx hoMduvanu, savxtaH tanamxyateyanunx hoduvudilalx.
\end{artha}


\vishaya{paqthiviVmaya itAyxdivAkayxda tAtapxyaR}

\begin{shl}
samAsavAyxsatasatxdavxtapxcnacx BUtAnayxvidayxyA || \\
ashabAdxdimayoV\s pAyxtAmx tanamxyatavxM nigacaCxti ||  183 ||  
\end{shl}

\begin{artha}
(Atamxnu heVge pArxNamayatavx modalAdavugaLanunx paDedano) hAgeye
adaraMteye shabadxmaya muMtAda rUpavalalxdidadxrU
shabadxmayatavxvanunx modalAdavanunx avideyxyiMda hoMduvanu, heVge ?
eMdare! paqthivAyxdipaMcaBUtagaLanunx avideyxyiMda
Ekatavx,\footnote[1 2]{virATasavxrUpavu vAyxsarUpa, biDiyAda deVva,
mAnuSa, pashupakASxyXdirUpa sUtArxtamxnu hiraNayxgaBaR, ivanu
EkarUpa, samAsarUpa, EkarUpadiMdalU aneVka rUpagaLiMdalU
parxteyxVkavAgi aBimAnisutAtx paMca BUta mayanAguvaneMdathaR.}, nAnAtavx\footnotemark rUpagaLiMda aBimAnisutAtx
(tanamxyanAguvanu).
\end{artha}

\footnotetext[3]{`vijAcnxnamayaH' eMbalilxMda paqthiviVmayaH eMbudakUkx modaliruva vAkayxdiMda sUkaSxmX deVhavU
vAyxpakavAdadxriMda parxdhAnaveMdu vAyxKAyxnisalapxTiTxde, hAgeyeV
`paqthiviVmayaH' eMbalilxMda AraMBisi `kAmamayaH' eMbudakUkx Icege
iruva vAkayxdiMda sUthxladeVhavu vAyxpAyxvAdadadxriMda guNavAgi
(amuKayxvAgi) vAyxKAyxnisalapxTiTxde, I eraDu shariVragaLU
paMcaBUtagaLiMda niSapxnanxvAdavu, AdarU parasapxra kAyaR kAraNa
rUpavAgiveyeMdu hiMde heVLida karxmadiMda tiLisalapxTiTxve,
oTiTxnalilx Atamxnige uBayadeVha mayatavxvu avideyxyiMda
sidadhxvAyiteMdu tAtapxyaR.}
\begin{shl}
\footnotemark[3]liknagxdeVhAvimAveVvaM pacnacxBUtamayAvuBw || \\
parxdhAnaguNavaqtotxyXVkwtx sUkaSxmXsUthxlaviBAgataH ||  184 ||  
\end{shl}

%%% footnote shloka 
\begin{artha}
ideV riVtiyAgi sUkaSxmX shariVra sUthxla shariVragaLu eraDU
paMcaBUtAtamxkavAgive AdarU parxdhAna vaqtitxyiMdalU
(muKayxvAyxpAradiMdalU) guNavaqtitxyiMdalU (amuKayxvAyxpAradiMdalU
sUkaSxmX sUthxlagaLeMba BeVdadiMda toVrisalapxTiTxve.)
\end{artha}

\section*{vAtiRka}

\vishaya{`kAmamaya' itAyxdi vAkayxda tAtapxyaR -}

\begin{shl}
ukatxyoVrAtamxnoVranatxyaRdUrxpaM BAvanAmayamf || \\
kAmAdimayatoVketxyXVha tadidAniVM viBAvayxteV ||  185 ||  
\end{shl}

\begin{artha}
heVLida eraDu AtamxgaLalilx (eraDu deVhagaLalilx) oLagina
BAvanAmayavAda rUpavu yAvuduMTo, adanunx I ajAcnxniyalilx kAmAdi
mayatavxvanunx heVLidadxriMda Iga visheVSa riVtiyAgi BAvisuvudu.
\end{artha}

\vishaya{oTuTx vAkAyxthaR}

\begin{shl}
kAmaM korxVdhaM tathA dhamaRM tadivxrudadhxM ca moVhataH ||  \\
saMBAvayanapxrXtiVcAyxtamx tanamxyatavxM nigacaCxti ||  186 || 
\end{shl}

\begin{artha}
kAmavanunx koVpavanUnx hAgU dhamaRvanunx adakekx virudadhxvAdadadxnunx
A parxtayxgAtamxnalilx ajAcnxnadiMda AroVpisikoLuLxtAtx
tanamxyatavxvanunx hoMduvanu.
\end{artha}

\vishaya{`idamamxya' itAyxdi padadavxyada tAtapxyaRkekx modalu
BUmikeyanunx toVrisuvaru -}

\begin{shl}
parxvaqtatxya iheVkaSxyXnetxV vAknaBxnaHkAyasAdhanAH || \\
yAvatoyxV BAvanAH puMsAmapi tAvatayx Eva tu ||  187 ||  
\end{shl}

\begin{shl}
iteyxVvamAdayoV\s neVkeV koVshAH suyxBARvanAmayAH || \\
asaMKeyxVyA bahutAvxtetxV saMkeSxVpoV\s toV\s tarx BaNayxteV ||  188 ||  
\end{shl}

\begin{artha}
I puruSanalilx parxvaqtitxgaLu vAkukx, manasusx, kAyagaLeMba
sAdhanavuLaLxvugaLAgi (vAcika, mAnasika, kAyika)veMbudAgi
kANisutatxve, hiVgeye (rAgAdivAsanA rUpavAda) BAvanegaLu eSuTx iveyo
aSeTx BAvanegaLu mAnavarige ive.
\end{artha}

\begin{artha}
I riVtiyeV modalAda aneVka koVshagaLu BAvanArUpavAdavu ive, avugaLu
bahaLavAgiruvudariMda lekikxsuvudakekx shakayxvilalx, adariMda ililx
saMkeSxVpavAgi heVLide.
\end{artha}

\vishaya{`idaMmayaH adoVmayaH' eMba padagaLa athaR -}

\begin{shl}
idamiteyxVva yatAsxkASxtikxMcitakxmoVRpalakaSxyXteV || \\
paroVkaSxM BAvanArUpaM teVnAdoVmayateVSayxteV ||  189 ||  
\end{shl}

\begin{artha}
yAvudoMdu parxtayxkaSxvAgiruva kamaRvidadxrU adu idamf eMdeV
gArxhayxvAgide, BAvanA rUpavAda kamaRvu paroVkaSxvAgiruvudu, adariMda
adoVmayaneMdu heVLuvudu.
\end{artha}

\begin{shl}
idaMmayeVna liknegxVna saMbanodhxV\s doVmayAtamxnaH || \\
parxtayxkaSx Atamxni yathA tathA\s nayxtArxpi liknagxyXteV ||  190 ||  
\end{shl}

\begin{artha}
\footnote{heVge obabxnalilxruva tananxrAgAdigaLu, tAnu mADida
puNAyxpuNayx kamaRgaLU sAvxnuBavasidadhxvAgiruvavo hAgeye
matotxbabxnalUlx kANuva shubAshuBakamaR vAyxpAravanunx noVDi
avanalilxruva rAgAdigaLanunx matotxbabxnu UhisuvudakekxV sAdhayxvideyeMdathaR}idaMmayaveMba heVtuviniMda adoVmayAtamxna saMbaMdhavanunx parxtayxkaSxvAda tananxlilx heVge Uhisabahudo hAgeye beVrobabxnalUlx Uhisabahudu.
\end{artha}

\vishaya{savxyaM AtamxdaqSATxMtavanenxV sAdhisuvudu -}

\begin{shl}
dhUmAgonxyXVriva saMbanadhxsatxyoVdaqRSaTx ihA\s \s tamxni || \\
\footnotemark[1]akAmaparxmuKeYyoVRgeYsatxthA kAmAdiyoVgataH ||  191 || 
\end{shl}
\footnotetext[1]{athaRkarxmavanunx anusarisi kAyaRkAraNamayaveMdu heVLi
savaRmaya eMbudara athaRvanunx ililx heVLide.}

\begin{shl}
dhamARdhamaRmayoV BUtAvx pumAnasxvaRmayoV BaveVtf || \\
dhamARdhameYRkaheVtutAvxtasxvaRsayx jagatasatxtaH ||  192 ||  
\end{shl}

%%% footnote shloka
\begin{artha}
A idaMmayatavx-adoVmayatavxgaLa saMbaMdhavu hoge
beMkigaLige iruva saMbaMdhavu (avinABAvavu) kaMDiruvaMte I Atamxnalilx
kaMDiruvudu akAma muMtAda vaqtitxgaLiMda  saMbaMdhavAgi
dhamaRmayanAgiyU, kAmAdi vaqtitxgaLa saMbaMdhadiMda adhamaRmayanAgiyU
I puruSanu idudx savaRmayanAguvanu, EkeMdare ? elAlx jagatutx
dhamAdhamaRgaLeMba oMdeV nimitatxgaLiMda saqSiTxyAguvudu, (adariMda
dhamARdhamaRmayanu savaRmayanAguvanu).
\end{artha}

\vishaya{hiMde heVLidara saMkeSxVpa}

\footnotetext[2]{savaRmayatavxvanunx vAyxKAyxna mADida naMtara
pAThakarxmadaMte adara athaRvanunx anuvadisuvaru.}
\begin{shl}
\footnotemark[2]adoVmayatavxM liknagxM sAyxlilxknegxVneVdaMmayAtamxnA || \\
BAvanAkamaRvidAyxnAmeVvamuketxVna vatamxRnA || \\
BUriBeVdAnanx shakayxnetxV vakutxM rUpANayxsheVSataH ||  193 ||  
\end{shl}

%%%% footnote shloka
\begin{artha}
adoVmayatavxvanunx idaMmayarUpavAda liMgadiMda
(heVtuviniMda) anumAnamADalu (takiRsalu) yoVgayxvAguvudu, BAvane,
kamaR, videyx (upAsane)gaLu hiMde heVLida mAgaRdalilx aneVka
bageyAgiruvudariMda avugaLa savxrUpagaLanunx elalxvanunx heVLuvudakekx
sAdhayxvAguvudilalx.
\end{artha}

\begin{shl}
kiM kAraNaM pumAnayxsAmxdayxthAkAriV Bavatayxyamf || \\
yathAcAriV ca loVkeV\s simxMsatxthArUpoV Bavatayxsw ||  194 ||  
\end{shl}

\begin{artha}
(asaMga matutx adivxtiVyanAda Atamxnu) pArxNAdimayanAgalu Enu
kAraNa? eMdare-yAvudariMda Itanu heVge kamaRvanunx moDuvano, heVge
Acarisuvano, hAgeye I loVkadalilx (tananx avideyxyiMda mADuva kamaR,
AcaraNegaLige anusAravAgi pArxNAdimayanAguvanu.
\end{artha}

\vishaya{karaNa, caraNa shabadxgaLige iruva athaRBeVda -}

\begin{shl}
karaNaM niyataM ceYva tatoV\s nayxcacxraNaM tathA || \\
karaNaM kamaRshakitxvAR caraNaM parxtayxyAtamxkamf ||  195 ||  
\end{shl}

\begin{artha}
karaNaveMdare avashayxkataRvayxvAdadudx, adakekx beVreyAdadudx caraNa
athavA karaNa eMdare kamaR mADuva shakitx, caraNaveMdare jAcnxna
rUpavAdadudx.
\end{artha}

\vishaya{`sAdhukAriV' itAyxdi maMtarxBAgada vAyxKAyxna -}

\begin{shl}
sAdhukAriV pumAnayxH sAyxtAsxdhureVva Bavatayxsw || \\
pitaqganadhxvaRdeVvAdw sAdhusAdhanasaMpadA ||  196 ||  
\end{shl}

\begin{artha}
(oLeLxya kamaRvanunx mADida manuSayx) yAva manuSayxnu utatxma
kamaRvanunx mADuvano avanu sAdhuveV AgutAtxne, utatxma sAdhana
saMpatitxniMda pitaq, gaMdhavaR, deVvategaLeV modalAdavaralilx
(huTuTxvanu).
\end{artha}

\begin{shl}
pApakAriV ca pApashacx sAthxNAvxdAvaBijAyateV || \\
kAmakorxVdhAdiBUyiSaThxH pumAnugerxVNa kamaRNA ||  197 ||  
\end{shl}

\begin{artha}
pApavanunx mADida pApa puruSanu kAma korxVdha modalAdavugaLiMda
hecicxdavanAgi kUrxrakamaRdiMda sAthxNu modalAdavugaLalilx huTuTxvanu ||
\end{artha}
