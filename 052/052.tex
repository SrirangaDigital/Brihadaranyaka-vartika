\section*{baq. vA. 4, bArx. 4}

\vishaya{|| dakiSxNAmUtaRyeV namaH||}

\begin{shl}	
niHsheVSapuruSAthoVR\s yaM samApotxV barxhamxboVdhataH || \\
savARnathaRnirAsAthaRmeVtAvAneVva cA\s \s daqteYH \hfill || 1305 ||  
\end{shl}
				
\begin{shl}
kataRvoyxV yatanxmAsAthxya kaqteV yasimxnakxqqtAthaRtA || \\
nAnayxtaH kaqtakaqtayxH sAyxdukatxjAcnxnAtireVkataH \hfill || 1306 ||  
\end{shl}

\begin{artha}
barxhamxjAcnxnadiMda samasatx puruSAthaRvU pUNaRvAyitu, elalxrU AdaravuLaLxvarAgi samasatx anathaRgaLanunx kaLedukoLaLxlu yatanxmADi iSeTx mAtarx (barxhamxjAcnxnavanunx) sAdhisabeVku, yAvudanunx sAdhisidare-kaqtAthaRteyu uMTAguvudo, adanenx sAdhisabeVku. hiMde heVLida jAcnxnakekx beVreyAda matotxMdu sAdhanadiMda kaqtAthaRnAguvudilalx.
\end{artha}

\vishaya{baq. 4 - 4 - kaMDike 24, 25}

\vishaya{muMdina eraDu kaMDikegaLa tAtapxyaR}

\begin{shl}
vAyxKAyxtoV yoV\s yamatArx\s \s tAmx sa vijAcnxnAnuroVdhataH || \\
PalAya sAyxdavijAcnxta iteyxVtadadhunoVcayxteV \hfill || 1307 ||  
\end{shl}

\begin{artha}
ililx yAva I Atamxnanunx vAyxKAyxnisidAdxyito A Atamxnu jAcnxnavanunx anusarisi paramapuruSAthaRkekx anukUlisuvanu, tiLiyadiruvalilx saMsAradaPalakAkxgi nilulxvudu, I viSayavanunx Iga heVLuvudu.
\end{artha}

\begin{shl}
sa vA ESa mahAnaja AtAmxnAnxdoV vasudAnoV vinadxteV vasu ya EvaM veVda || 24 ||
\end{shl}

\begin{shl}
sa vA ESa mahAnaja AtAmxjaroV\s maroV\s maqtoV\s BayoV barxhAmxBayaM veY barxhAmxBayaM hi veY barxhamx Bavati ya EvaM veVda || 25 ||
\end{shl}

\vishaya{paMcamAdhAyxyada koneyalilx ``rAtidARtuH parAyaNaM tiSaThxmAnasayx tadivxdaH'' eMbuvalilx kamaRjAcnxnagaLa Palavanunx heVLide, punaH heVLidalilx punarukitxyAguvudilalxve? eMdare {\rm --}}

\begin{shl}
pacnacxmAnatx upanayxsatxM rAteVdARtuH parAyaNamf || \\
tadivxdasitxSaThxmAnaseyxVtayxsayx ceVhoVpasaMhaqtiH \hfill || 1308 ||  
\end{shl}

\begin{artha}
paMcamAdhAyxyada koneyalilx `rAteVdARtuH parAyaNaM tadivxdaH tiSaThxmAnasayx' eMdu heVLidadxnenx ililx \footnote{kamaRPalavanunx upasaMhAra mADalu modalina vAkayx, jAcnxnaPalavanunx upasaMharisalu eraDane vAkayx.}upasaMhAra mADuvudu.
\end{artha}

\vishaya{kUTasathx niviRkAra Atamxnige `anAnxda', `vasudAna' eMba visheVSAthaRvu heVge ? saMgata ? eMdare \mdash  }

\begin{shl}
AtAmxnetxVnAsayx vAkeyxVna savxtasatxtatxvXmihoVcayxteV || \\
anAnxdavasudAnABAyxmavidoyxVtathxM tu BaNayxteV \hfill || 1309 ||  
\end{shl}

\begin{artha}
savxtaH sahajavAda tatavxvu ililx heVLalapxTiTxde, `anAnxdo vasudAnaH' eMba vAkayxdiMda ajAcnxnadiMda kalipxtavAda \footnote{ajAcnxna kalipxtavAda tatavx mUlakAraNaveMdathaR.}tatavxvu heVLalapxTiTxde.
\end{artha}

\vishaya{anAnxda padadiMda heVge A tatavxvu ukatxvAgide ?}

\footnotetext[3]{ajAcnxtavAda paramAtamxtatavxvu pArxNasavxBAvavuLaLxvanu - aMdare jagatAkxraNasavxrUpavuLaLxvanu. samasatx ananxvanunx BakiSxsuvaneMbudakekx sakalakAyaRvanunx saMharisuvavaneMdathaR, `atAtxcarAcaragarxhaNAtf' eMbuvalilx atatxrX eMdare carAcarAtamxka jagatfsaMhArakataRneMdathaRveMbudu sidAdhxMtavAgide. I lakaSxNaviruva pArxNAtamxneMdare jagatAkxraNanAda Ishavxra eMdathaR. idara upapAdane heVge ? eMdare jagatetxlAlx Atamxna kAyaR, pariNAma, Atamxnu kAraNa, adariMda avanu saMhArakataRnAgabahudu.}
\begin{shl}
\footnotemark[3]niHsheVSAnAnxdanAdAtAmx pArxNAnotxV\s nAnxda ucayxteV || \\
AtAmx\s yaM kAraNaM yasAmxtAkxyaRmitayxKilaM tataH \hfill || 1310 ||  
\end{shl}

%%%%footnote in shloka
\begin{artha}
samasatx ananxvanunx BakiSxsuvudariMda Atamxnu pArxNasavxBAvavuLaLxvanu (kAraNa savxBAvavuLaLxvanu) anAnxda eMdu heVLalapxDuvanu. yAvudariMda I Atamxnu kAraNanAgiruvano, adariMda elalxvU kAyaR.
\end{artha}

\begin{shl}
kAraNeV\s nupayukatxM yatAkxyaRM tanenxVha vidayxteV || \\
kAyeVR\s payxnupayukatxM yananx tatAkxraNamucayxteV \hfill || 1311 ||  
\end{shl}

\begin{artha}
kAraNakekx upayukatxvalalxdudx (apeVkiSxtavalalxdudx) yAva kAyaRvo adu ililx (I vayxvahAradalilx) iruvudilalx, kAyaRkUkx upayukatxvalalxdudx yAvudo, adu kAraNaveMdu heVLalapxDuvudilalx.
\end{artha}

\vishaya{idariMda kAyaRkAraNagaLu sApeVkaSxvasutxgaLAdare PalitAMshaveVnu ?}

\begin{shl}
anoyxVnAyxthaRvayxpeVkiSxtAvxtAkxyaRkAraNavasutxnoVH || \\
nAnoyxVnAyxthARtireVkeVNa sidheyxVteV kAyaRkAraNeV \hfill || 1312 ||  
\end{shl}

\begin{artha}
I kAyaRkAraNa vasutxgaLu parasapxra vasutxvanunx apeVkiSxsuvudariMda parasapxra vasutxvanunx biTuTx kAyaRkAraNa sididhxsuvudilalx.
\end{artha}

\begin{shl}
vasUpakaraNaM porxVkatxM tadadxdAtiVshavxratavxtaH || \\
yatasatxtoV\s ja AtAmx\s yaM vasudAna ihoVcayxteV \hfill || 1313 ||  
\end{shl}

\begin{artha}
vasu eMbudu upakaraNaveMdu heVLalapxTiTxde, adanunx yAvudariMda IshavxranAdadadxriMda koDutAtxneyo, adariMda huTaTxdiruva I Atamxnu vasudAna eMdu ililx heVLalapxDuvanu.
\end{artha}

\begin{shl}
anAnxdavasudAnABAyxM guNABAyxM yaH samiVkaSxteV ||  \\
IshavxraM sa yathAdaqSiTx PalamAponxVti mAnavaH \hfill || 1314 ||  
\end{shl}

\begin{artha}
anAnxda-vasudAnaguNagaLiMda kUDi yAva mAnavanu Ishavxrananunx noVDuvano, avanu kaMDa daqSiTxge takakx Palavanunx hoMdutAtxne.
\end{artha}

\begin{shl}
iteyxVvamayathAvasutxdashiRnaH PalamiVritamf || \\
yathAvasutxdaqshoV\s peyxVvaM yathA rajAjxvXdidashiRnaH \hfill || 1315 ||  
\end{shl}

\begin{artha}
I riVtiyAgi paramAthaRvalalxda vasutxvanunx noVDuvavanige Palavanunx heVLidAdxyitu, hAgeye paramAthaRvasutxvanunx kANuvavananunx noVDuvavanigU rajujx modalAda paramAthaRvasutxvanunx noVDuvavanige Aguva PaladaMteye (aMdare sapaRBayAdi nivaqtitxyaMte) saMsAraduHKa nivaqtitxyAguvudu.
\end{artha}

\vishaya{hiMde heVLida daqSiTxyeMbuva upAsanege takakx PalaveMbudakekx giVteya udAharaNe \mdash  }

\begin{shl}
yeV yathA mAM parxpadayxnetxV tAMsatxtheYva BajAmayxhamf ||  \\
iti cA\s \s heVshavxroV vAkayxmukAtxthaRparxtipatatxyeV \hfill || 1316 ||  
\end{shl}

\begin{artha}
yAru heVge  nananxnunx tiLiyuvaro avaranunx hAgeyeV nAnu seVvisuvenu' eMdu Ishavxranu I vAkayxvanunx hiMde heVLida athaRjAcnxnakAkxgi upadeVshisiruvanu.
\end{artha}

\begin{shl}
yathAtatatxvXM tu yoV\s jAcnxtaM vasutx sAkASxtapxrXpadayxteV ||  \\
yathAvasetxvXVva tasAyxpi PalaM  sAyxditi BaNayxteV \hfill || 1317 ||  
\end{shl}

\begin{artha}
yAvanu ajAcnxtavAda vasutxvanunx tatavxvu iruvaMteye neVra noVDuvano avanigU vasutxvu iruvaMte PalavAguvudeMdu heVLalapxDutatxde.
\end{artha}

\vishaya{kAMDadavxyakekx beVre oMdu athaR \mdash  }

\begin{shl}
kANaDxdavxyasayx vA yoV\s thaRH sa saMkiSxpAyxBidhiVyateV || \\
sa vA itAyxdivAkeyxVna sArAthaRsayx jiGaqkaSxyA \hfill || 1318 ||  
\end{shl}

\begin{artha}
eraDu kAMDagaLa yAva athaRvideyo adanunx sArAthaRvanunx garxhisuva udedxVshadiMda `savA itAyxdi vAkayxdiMda saMkeSxVpisi heVLuvudu.
\end{artha}

\vishaya{`savA' eMbuva padagaLige athaR \mdash  }

\begin{shl}
avidAyxvAnupxrA yoVkotxV maqtuyxjanAmxdisaMsaqtiH || \\
sa vA iti girA soV\s tarx sAmxyaRteV barxhamxsaMgatw \hfill || 1319 ||  
\end{shl}

\begin{artha}
hiMde (bArxhamxNada Adiyalilx) yAva maqtuyx janamx modalAda saMsAravuLaLx Atamxnu heVLalapxTiTxruvano, avaneV I koneya kaMDikeyalilx barxhamxna saMbaMdhavanunx heVLalu udedxVshisi `saveY' eMba padagaLiMda samxrisalapxTiTxruvanu.
\end{artha}

\vishaya{`ESaH' eMbudara athaR \mdash  }

\begin{shl}
sa vA ESa puroVketxVna yoV vAkeyxVna parxkAshitaH || \\
parxdhavxsAtxjAcnxnatajajxH sanapxrAM nivaqRtimagataH \hfill || 1320 ||  
\end{shl}

\begin{artha}
`saveY' eMdu yAva tavxMpadAthaRvu heVLalapxTiTxto, adu `barxhamxloVkaH' eMdu hiMde heVLida vAkayxdiMda parxkAshapaDisalapxTiTxdudx ajAcnxna matutx adara kAyaRgaLiMda nAshagoMDu paranivARNavanunx (moVkaSxvanunx) hoMdiruvudu. (A AtamxneV `ESaH' eMdu heVLalapxTiTxruvanu).
\end{artha}

\begin{shl}
mahAnitayxBidhAneVna satayxjAcnxnAdilakaSxNamf ||  \\
barxhemxYva porxVcayxteV sAkASxtapxrXtayxgAtamxvisheVSaNamf \hfill || 1321 ||  
\end{shl}

\begin{artha}
`mahAnf' eMba shabadxdiMda satayxjAcnxnAdi lakaSxNavuLaLx barxhamxveV sAkASxtAtxgi heVLalapxTiTxde. adu parxtayxgAtamxnige visheVSaNavu.
\end{artha}

\vishaya{jarAmaraNAdidhamaRvuLaLx Atamxnu heVge barxhamxvAdAnu ? eMdare}

\begin{shl}
barxhemxYvA\s \s tAmx savxtoV\s boVdhAdabarxhemxYva parxkAshateV || \\
jarAdidhamaRvAMsatxsAmxdabxrXhamxNeYva visheVSayxteV \hfill || 1322 ||  
\end{shl}

\begin{artha}
barxhamxveV AtamxsavxrUpajAcnxnavilalxdadxriMda A barxhamxveV Agi parxkAshisutatxliruvanu, adariMda jarAmaraNAdi dhamaRvuLaLxvanAgiruvanu barxhamxnoMdige visheVSaNavAgi seVrisalapxTiTxruvanu.
\end{artha}

\vishaya{padagaLige sAmAnAdhikaraNayxda BAvAthaR}

\begin{shl}
parxmAnatxrAnadhigatA barxhamxtA parxtayxgAtamxnaH || \\
parxtayxkatxvXM barxhamxNasatxdavxdavxcaseYva\footnotemark[1] parxboVdhayxteV \hfill || 1323 ||  
\end{shl}
\footnotetext[1]{`mahAnAtamx' eMdu oMdAgi heVLida vacanadiMda.}

%footnote in shloka
\begin{artha}
beVre parxmANadiMda tiLiyadiruva barxhamxsavxrUpavu parxtayxgAtamxnigU, barxhamxnige parxtayxkfsavxrUpavU hAgeye idudx vacanadiMdaleV boVdhisalapxDuvudu.
\end{artha}

\vishaya{`aja Atamx' eMbudara tAtapxyaR}

\begin{shl}
ahaM barxhemxVtivAkoyxVtathxvijAcnxnAtapxrXtayxgAtamxnaH ||  \\
dhavxsetxV\s jAcnxneV sakAyeVR\s tha yadUrxpaM tadihoVcayxteV \hfill || 1324 ||  
\end{shl}

\begin{artha}
`ahaM barxhamx' eMbuva vAkayxdiMda huTiTxda vijAcnxnadiMda parxtayxgAtamxna ajAcnxnavu kAyaRsahitavAgi nAshavAgalu anaMtara iruva rUpavanunx ililx heVLiruvudu.
\end{artha}

\begin{shl}
rajujxH sapARdineVvA\s \s tAmx vinA\s vidAyxM na jAyateV || \\
kAyARtamxnA yatasatxsAmxdAtAmx\s ja iti BaNayxteV \hfill || 1325 ||  
\end{shl}

\begin{artha}
rajujxvu sapARdirUpadiMda avideyxyilalxde heVge pariNamisalArado hAgeye Atamxnu kAyaRrUpadiMda avideyxyilalxde pariNamisuvudilalx, adariMda Atamxnu ajanu (huTaTxdiruvanu) eMdu heVLalapxDuvanu.
\end{artha}

\vishaya{AkeSxVpa}

\begin{shl}
savaRBAvavikArANAM janamx mUlaM yatasatxtaH || \\
saveVRSAM parxtiSeVdhaH sAyxninxSeVdhAdAtamxjanamxnaH \hfill || 1326 ||  
\end{shl}

\begin{artha}
samasatx BAva pariNAmagaLigU mUlavAdadudx janamx. adariMda Atamxnige janamxvanunx niSeVdhisuvudariMda samasatx pariNAmagaLanunx niSeVdhisidaMte Aguvudu.
\end{artha}

\vishaya{samAdhAna}

\begin{shl}
tathA\s pi tAcnushxrXtiyaRtAnxjajxrAdiVnapxrXtiSeVdhati || \\
sAvxBAvikatAvxshaknAkxyAH parxtiSeVdhasayx sidadhxyeV \hfill || 1327 ||  
\end{shl}

\begin{artha}
AdarU shurxtiyu parxyatanxdiMda A jarA muMtAda BAva vikAragaLanunx niSeVdhisuvudu, EtakAkxgi ? eMdare ivugaLu Atamxnige sAvxBAvikavAgiveyeMba shaMkeyanunx pariharisuvudakAkxgi.
\end{artha}

\vishaya{ajara padavanunx vAyxKAyxnisuvudu \mdash  }

\begin{shl}
kAlAtamxnA hayxvaceCxVdAnAnxyaM kAlaMjaratavxtaH || \\
deVhAdivajajxrAmeVti tasAmxdAtAmx\s jaraH samxqqtaH \hfill || 1328 ||  
\end{shl}

\begin{artha}
ajara pada vAyxKAyxna - I Atamxnu kAlavanenx jiVNiRsikoLuLxvudariMda aMdare pariciCxnanxnalalxdadxriMda kAlarUpadiMda pariciCxnanxvAda deVhAdigaLaMte mupapxnunx hoMduvudilalx, adariMda ajaraneMdu heVLalapxTiTxruvanu.
\end{artha}

\begin{shl}
savaRsayx pariNAmasayx hAnoVpAdAnamAtarxtaH || \\
jarAjanamxniSeVdheVna savoVR\s toV\s tarx niSidhayxteV \hfill || 1329 ||  
\end{shl}

\begin{artha}
elAlx pariNAmavU hAna, upAdAna mAtarxveV AgiruvudariMda mupapxnunx jananavanunx niSeVdhisidadxriMdale elalx pariNAmavanunx I Atamxnalilx niSeVdhisidaMte Aguvudu.
\end{artha}

\vishaya{I viSayakekx gamakaveVnu ?}

\begin{shl}
pariNAmoV\s sayx yeVnAnatxyXH shurxteyxVha parxtiSidhayxteV \hfill || 1330 ||  \\
deVhasithxteVravasitimaqRtishabedxVna BaNayxteV ||  \\
taninxSeVdhoV\s maroVkatAyx\s tarx nitayxsAyx\s \s tamxna ucayxteV \hfill || 1331 ||  
\end{shl}

\begin{artha}
I vAkayxdalilx (janamxjarAdigaLanunx niSeVdhisida naMtara) Atamxnige nAshaveMba koneya pariNAmavanunx shurxtiyu niSeVdhiside, maqti eMba shabadxdiMda deVhasithxtiya avasAnavu heVLalapxDuvudu, amara shabadxdiMda ililx nitayxnAda Atamxnige maraNavanunx niSeVdha mADiruvudu.
\end{artha}

\begin{shl}
nAjasAyxpariNAmasayx maraNaM jagatiVkaSxyXteV || \\
ajoV\s jarashacx teVnoVkotxV\s marashecxYSa tataH samxqqtaH \hfill || 1332 ||  
\end{shl}

\begin{artha}
huTaTxdiruva pariNAmavilalxda vasutxvige nAshaveMbudu jagatitxnalelx kANuvudilalx, adariMda aja, ajara eMdAguvanu, adariMda I Atamxnu amaranU eMdu heVLalapxTiTxruvanu.
\end{artha}

\begin{artha}
amarapadadiMdale aMtayxvikAravanunx nirAkarisiruvAga amaqta eMba matotxMdu padavu Etakekx ?
\end{artha}

\begin{shl}
pariNAmAtamxkoV maqtuyxramaroVkAtxyX nivAritaH || \\
amaqtoVkAtxyX\s tha nAshoV\s sayx vAyaRteV yoV niranavxyaH \hfill || 1333 ||  
\end{shl}

\begin{artha}
pariNAma rUpavAda maraNavanunx amarapadadiMda nivAriside, amaqtapadadiMda niranavxya nAshavu nivArisalapxDuvudu.
\end{artha}

\begin{shl}
avinAshiVtayxpi tathA vinAshadavxyamAtamxnaH || \\
shurxtAyx niSidhayxteV sAkASxtapxrXtayxkwtxTasathxyXsidadhxyeV \hfill || 1334 ||  
\end{shl}

\begin{artha}
hAgeye `avinAshi' eMba shabadxdiMda Atamxnige eraDu bageya nAshagaLanunx shurxtiyu niSeVdhiside. EtakekxMdare ? parxtayxgAtamxnalilx niviRkAra rUpavanunx tiLisuvudakAkxgi.
\end{artha}

\begin{shl}
kAmakamaRtamoVBAvAdamaroV\s maqta Eva ca || \\
tata EvABayaH parxtayxknaBxyaM hi tadaboVdhataH || \\
kAraNasayx niSeVdhoV\s toV BayakAyaRniSeVdhataH \hfill || 1335 ||  
\end{shl}

\begin{artha}
kAma, kamaR, ajAcnxnagaLu ilalxdiruvudariMda Atamxnu amaranu, amaqtaneV Agiruvanu. adariMdale aBayanu, Bayavu adara ajAcnxnadiMdaveMbudu parxsidadhx (kAraNavAda ajAcnxnavanunx parxtayxgAtamxnalilx niSeVdhisiruvudu BayaveMba kAyaRvanunx niSeVdhisiruvudariMda) kAyaRvAda BayavilalxveMbudariMda kAraNavAda ajAcnxnavu ilalxveMdu tiLiyuvudu.
\end{artha}

\begin{shl}
kutoV\s BayatavxsaMsididhxH parxtiVca iti shaknikxteV || \\
barxhemxVtAyxhA\s \s tamxnasatxtatxvXM barxhamx tavxBayameVva hi \hfill || 1336 ||  
\end{shl}

\begin{artha}
parxtamxgAtamxnige aBayarUpavu heVge. sididhxsuvudeMdu shaMkisidalilx barxhamxve Atamxna tatavxveMdu heVLiruvudu. EkeMdare barxhamxve aBayavAgiruvudu.
\end{artha}

\begin{shl}
ahaM barxhemxVtayxtaH sAkASxdayxthoVketxVneYva vatamxRnA  || \\
BayaheVtoVniRrAseVna sa barxhAmxBamashunxteV \hfill || 1337 ||  
\end{shl}

\begin{artha}
ahaMbarxhamx' eMbuva sAkASxtAkxradiMda hiMde heVLida mAgaRdalelx Baya nimitatxvu hoVgiruvudariMda Atanu aBayavAda barxhamxvanunx paDeyuvanu.
\end{artha}

\vishaya{koneya kaMDikeya athaRvanunx upasaMharisuvudu \mdash  }

\begin{shl}
savoVRpaniSadAmeVSa saMkiSxpotxV\s thaR ihoVditaH || \\
utapxtAtxyXdivikalopxV\s yamaseyxYva parxtipatatxyeV \hfill || 1338 ||  
\end{shl}

\begin{artha}
elAlx upaniSatutxgaLa I saMkiSxpAtxthaRvanunx ililx heVLidAdxyitu, I saqSiTx muMtAda vikalapxvu I tatavxvanunx tiLiyuvudakAkxgiye baMdiruvudu.
\end{artha}

\vishaya{saqSiTx modalAdavu AtamxjAcnxnakekx kAraNaveMbudakekx daqSATxMta \mdash  }

\footnotetext[1]{reVKeyu nijavAgi saMKeyxyalalx, AdarU adaralilx AroVpisikoMDa meVle nAvu tiLiyuva 1,2,3, itAyxdi saMKeyxyu gaNisalAgada anaMta vasutxgaLanunx leKaKxmADalu gaNisi tiLiyalu upAyavAguvudu. gaNaniVyavAda vasutxvilalxde saMKeyxya jAcnxnavAguvudilalx, hAgeye parabarxhamx vasutxvinalilx ilalxdiruva jagatfsaqSiTxyanunx kalipxsikoMDa naMtara adara mUlaka tatavxjAcnxnavAguvudeMdu tAtapxyaR, I vAtiRkadalilx inonxMdathaRvu ide - oMdu eraDu muMtAda riVtiyalilx gaNisuva vasutx reVKeya saMkeVta mUlaka gaNisalAgada vasutxgaLanunx tiLisuvudu. saMKeyxyanonx, athavA gaNisalapxDuva vasutxvanonx reVKAvisheVSagaLalilx AroVpisi adara mUlaka nAnA vasutxgaLa jAcnxnavAguvudu. adaraMteyeMdu tiLiyabeVku.}
\begin{shl}
\footnotemark[1]EkaparxBaqtisaMKeyxVyaM reVKAsaMkeVtavatamxRnA || \\
parxtipatitxrasaMKeyxVyavasutxnaH sAyxdayxthA tathA \hfill || 1339 ||  
\end{shl}
				
\begin{shl}
savaRparxmANaviSayalaknf GinoV\s kAyaRkAraNa-\\
vasutxnaH parxtipatitxH sAyxjajxnAmxdidhavxMsavatamxRnA \hfill || 1340 ||  
\end{shl}

\begin{artha}
I oMdu modalAda saMKeyxyu reVKeya saMkeVtada mAgaRdiMda gaNisalAgada vasutxvina jAcnxnakekx kAraNavAguvudu. idu heVgo hAgeye samasatxparxmANa sAthxnavanunx miVrida kAyaRkAraNa BinanxvAda vasutxvina jAcnxnavu saqSiTx modalogxMDu saMhAra payaRMta heVLida mAgaRdiMda Aguvudu.
\end{artha}

\vishaya{I meVlina daqSATxMtavanenx sAdhisuvudu \mdash  }

\begin{shl}
saMKeyxVyatavxM na reVKAyAH saMKAyxtavxM vA\s cnajxseVSayxteV  || \\
tadadhAyxroVpagateyxYva tathA\s pi parxtipatatxyeV \hfill || 1341 ||  
\end{shl}

\begin{artha}
reVKeyu gaNaniVya vasutxvalalx, saMKeyxyU nijavAgi saMmatavalalx, AdarU averaDanunx reVKeyalilx AroVpisuva mUlaka adanunx tiLiyalu sAdhayxvAguvudu.
\end{artha}

\begin{shl}
yatheYvamavagacaCxnitx parxdhavxsAtxsheVSakalapxnamf || \\
AtAmxnaM sithxtijanAmxdikalapxnoVpAyamAtarxtAH \hfill || 1342 ||  
\end{shl}

\begin{artha}
idu heVgo hAgeye samasatx kalapxnegaLU nijavAgilalxda Atamxnanunx saqSiTx, sithxti modalAda kalapxneyeMba upAyamAtarxdiMda tiLiyuvaru.
\end{artha}

\vishaya{beVre daqSATxMtavanunx koDutAtxre \mdash  }

\begin{shl}
patarxkajajxlareVKABiyaRthA vA\s dhAyxsavatamxRnA || \\
vaNARcnAjxnanatxyXkArAdiVnapxtArxdiBoyxV vilakaSxNAnf  \hfill || 1343 ||  
\end{shl}

\begin{artha}
Olegari, kADige, reVKe ivugaLiMda adhAyxsa mAgaRdiMdaleV akArAdivaNaRgaLanunx Olegari modalAdavugaLige vilakaSxNavAgiruvavugaLanenx janaru heVge tiLiyuvaro.
\end{artha}

\begin{shl}
tathoVtapxtAtxyXdikAdhAyxsavatamxRnA tadivxlakaSxNamf || \\
janAmxdiheVtudhavxMseVna neVtiVti barxhamx gamayxteV \hfill || 1344 ||  
\end{shl}

\begin{artha}
hAgeye saqSiTx modalAdavugaLa AroVpaNeya mAgaRdiMdale saqSATxdi lakaSxNavilalxda barxhamxvu utapxtitx modalAdavugaLa kAraNavu nAshavAdadxriMda `neVti neVti' eMdu tiLisalapxDuvudu.
\end{artha}

\vishaya{parxkaqtoVpasaMhAra \mdash  }

\begin{shl}
iti kANaDxdavxyoVkotxV\s thaRH saMhaqtayx saPaloV\s KilaH || \\
iha parxdashiRtoV vAkeyxV tatasxmApitxvivakASx \hfill || 1345 ||  
\end{shl}

\begin{artha}
I riVtiyAgi eraDu kAMDagaLa athaRvanunx parxyoVjana sahitavAgi elalxvanunx I koneya kaMDikeya vAkayxdalilx eraDu kAMDavanunx samApitxgoLisuva udedxVshadiMda toVrisikoTiTxde.
\end{artha}

\bigskip
\begin{center}
{\bf iMtebuvalilxge shirxVbaqhadAraNayxka upaniSadfBASayxda}

\smallskip
{\bf vAtiRkadalilx nAlakxne adhAyxyadalilx}

\smallskip
{\bf nAlakxne shAriVraka bArxhamxNavu pUNaRgoMDide}
\end{center}
\bigskip

\section*{adhAyxya 4, bArxhamxNa 5 AraMBa}

\vishaya{|| dakiSxNAmUtaRyeV namaH||}

\begin{shl}
niradhAri purA barxhamx madhukArxNADxgameVna yatf || \\
yAjacnxvalikxVyakANeDxVna taduyxkAtxyX\s thoVpapaditamf \hfill || 1 ||  
\end{shl}

\begin{artha}
madhukAMDa shurxtiyiMda yAva parabarxhamxvu hiMde nidhaRrisalapxTiTxtoV adanunx yukitxyiMda yAjacnxvalakxyXkAMDadiMda upapAdane mADidAdxyitu.
\end{artha}

\begin{shl}
jalapxnAyxyeVna tatUpxvaRM pacnacxmeV parxtipAditamf || \\
vAdanAyxyeVna tadUBxyaH SaSeThxV samayxkapxrXpacnicxtamf \hfill || 2 ||  
\end{shl}

\begin{artha}
jalapxkatheyiMda aidane adhAyxyadalilx yAvudanunx hiMde parxtipAdisididxto adanunx vAda katheyiMda Arane adhAyxyadalilx punaH cenAnxgi visatxriside.
\end{artha}

\begin{shl}
athAdhunA nigamanasAthxniVyamidamucayxteV || \\
meYterxVyiVbArxhamxNaM shurxtAyx tathAca nAyxyavidavxcaH \hfill || 3 ||  
\end{shl}

\begin{artha}
anaMtara IvAga nigamasAthxnadalilxruva meYterxVyi bArxhamxNavanunx shurxtiyu heVLuvudu, adara bagegx nAyxyashAsatxrXjacnxra vacanavU iruvudu.
\end{artha}

\vishaya{nigamana eMdare \mdash  }

\footnotetext[1]{``heVtavxpadeVshAtf parxtijAcnxyAH punavaRcanaM nigamanamf'' eMdu nAyxyasUtarx (adhAyxya 1, Anihxka 1 - sU. 39)`vAkayxsayx' eMbuvalilx parAthARnumAna rUpavAda paMcAvayavavuLaLx mahAvAkayxvanunx garxhisabeVku, parxtijAcnx, heVtu, udAharaNa, upanaya, nigamana eMdu aidu bageyAgiruvudu, adaralilx koneyadu `pavaRtoVvanihxmAnf' eMbudu parxtijAcnx, `dhUmavatAvxtf' eMbudu heVtu; `yoVyoVdhUmavAnf sa vanihxmAnf' eMdu udAharaNa, tathAcAyaM eMbudu upanaya, tasAmxtatxthA (dhUmavatAvxtf vanihxmAnf eMbudu nigamana).}
\begin{shl}
\footnotemark[1]heVtUkitxtaH parxtijAcnxyAH sidAdhxthARyA yadutatxramf || \\
vacoV nigamanaM tatAsxyXdAvxkayxsAyxvayavoVkitxBAkf \hfill || 4 ||  
\end{shl}

%%%%footnote in shloka
\begin{artha}
heVtuvacanadiMda sidadhxvAda athaRvuLaLx parxtijAcnxvAkayxkekx muMdina vacanavu nigamanaveMdu Aguvudu. adu paMcAvayavagaLiMda kUDida vAkayxdalilx EkadeVsha vacanavAguvudu.
\end{artha}

\section*{vAtiRka}

\vishaya{I bArxhamxNakekx beVre tAtapxyaRvU ide \mdash  }

\begin{shl}
sasaMnAyxsA\s \s tamxvidAyx yA madhukANADxgamoVditA ||  \\
upapatitxparxdhAneV\s pi seYva moVkeSxV\s vasiVyateV \hfill || 5 ||  
\end{shl}

\begin{artha}
madhukAMDadalilx meYterxVyiVbArxhamxNadiMda AgamadiMda heVLida saMnAyxsa sahitavAgi heVLida Atamxvideyx yAvuduMTo. adeV videyxyu yukitxparxdhAnavAda moVkaSxdalUlx (aMdare munikAMDadalilx heVLida mukitxge sAdhanavAgideyeMdu) nishacxyisalapxDuvudu. 
\end{artha}

\vishaya{kahoVLa parxshenxyiMda nAyxyadiMda adu niNaRyavAgideyAgiralu punaH I bArxhamxNadiMda Enu parxyoVjana ? {\rm --}}

\begin{shl}
AtamxjAcnxnaM sasaMnAyxsaM moVkASxyeVtAyxgamAdayxthA || \\
yukitxtoV\s pi tathA jecnxVyamiti ceVhoVpasaMhaqtiH \hfill || 6 ||  
\end{shl}

\begin{artha}
AtamxjAcnxnavu saMnAyxsa sahitavAgi moVkaSxkekx kAraNaveMbudu heVge AgamadiMda sidadhxvAyito, hAgeye yukitxyiMdalU adu sidadhxvAguvudeMdu tiLiyabeVkeMdu ililx upasaMhAra mADide.
\end{artha}

\begin{shl}
vAyxKAyxtatAvxtapxdAthARnAM catutheVR\s dhAyxya Eva tu || \\
tadAvxyXKAyxnAya yatonxV\s toV na BUyaH kirxyateV\s dhunA \hfill || 7 ||  
\end{shl}	

\begin{artha}
nAlakxne adhAyxyadalelx padAthaRgaLanunx vAyxKAyxnisidadxriMda Iga adara vAyxKAyxnakAkxgi punaH parxyatanx mADuvudilalx.
\end{artha}

\vishaya{padAthaRvanunx heVLabeVkAgilalxvAdare tAtapxyaRvanunx heVLiruvudariMda bArxhamxNavu ililxge mugiyuvudilalxve ? eMdare \mdash  }

\footnotetext[1]{`sayathA\s \s derxYRdhAgenxV' itAyxdi sathxLadalilx `iSaTxM hutamf' itAyxdiyAgi hecAcxgi heVLidudx kaMDubaMdide, adara aBipArxyaveVneMbudanunx BASayxdaMteye ililx heVLide, catuthARdhAyxyadalilx shabadxsaqSiTxyu AtamxniMda AgiruvudeMdu heVLide, adariMdale loVkAdi athaRsaqSiTxyanunx heVLida hAgAyitu, heVgeMdare ! athaRvilalxde shabadxveMbudu hoMduvudilalx, adariMda beVre adara saqSiTxyanunx heVLalilalx, savaRshAsAtxrXthaRda upasaMhAravanunx I adhAyxyadalilx mADabeVkAgiruvudariMda athaRsidadhxvAda athaRvanunx sapxSaTxpaDisabeVkAgideyAdadxriMda parxteyxVkavAgi heVLiruvudu.}
\begin{shl}
\footnotemark[1]shabadxseyxYvA\s \s tamxnoV janamx catutheVR parxtipAditamf || \\
savoVRpasaMhaqteVratarx yAgAdeVrapi BaNayxteV \hfill || 8 ||
\end{shl}
				
\begin{artha}
nAlakxne adhAyxyadalilx shabadxda huTaTxnunx parxtipAdiside, ililx sakalakUkx ililx upasaMhAraviruvudariMda yAgAdigaLa utapxtitxyu heVLalapxDuvudu.
\end{artha}

\footnotetext[2]{yadaqgevxVdoyajuveRVdaH itAyxdi maMtarxda aBipArxya \mdash   I shabadxvu maMtarx bArxhamxNa rUpadalilx BinanxvAgide, nAmavelalxvU paramAtamxniMdale huTiTxruvudeMdu ``yadaqgevxVdaH'' eMbudariMda AraMBisi ``vAyxKAyxnAni'' eMbuvavarege shurxtiyu upasaMhariside, `iSaTxM hutaM' eMdu AraMBisi ayaMcaloVka eMbudakUkx hiMdeye yAgAdikamaRgaLU paramAtamxniMda huTiTxdedxMdu upasaMhariside, `ayaMcaloVkaH' eMbudAgi AraMBisi muMdina garxMthadiMda rUpa saqSiTxyanunx upasaMhariside}
\begin{shl}
\footnotemark[2]shabodxV\s yaM bahudhA BinonxV nAmeYva paramAtamxjamf || \\
yAgadAnAdikaM kamaR rUpaM loVkapuraHsaramf \hfill || 9 ||  
\end{shl}

\begin{artha}
I shabadxvu aneVka riVtiyAgi BinenxYsiruvudu, paramAtamxniMda huTiTxruvudu, yAga, dAnAdi, kamaRvU rUpavU loVkapuraHsaravAgide.
\end{artha}

\section*{baq. 4, bArx. 5 - 11 neV kaMDike}

\begin{shl}
sa yathAderxYRdhAgenxVraBAyxhitasayx paqthagUdhxmA vinishacxranetxyXVvaM vA areV\s sayx mahatoV BUtasayx nishavxsitameVtadayxdaqgevxVdoV yajuveVRdaH sAmaveVdoV\s thavARknigxrasa itihAsaH purANaM vidAyx upaniSadaH sholxVkAH sUtArxNayxnuvAyxKAyxnAni vAyxKAyxnAniVSaTxM hutamAshitaM pAyitamayaM ca loVkaH parashacx loVkaH savARNi ca BUtAnayxseyxYveYtAni savARNi nishavxsitAni || 11 ||
\end{shl}

\section*{vAtiRka}

\begin{shl}
savaRmeVtadayatenxVna tata Eva viniHsaqtamf || \\
ananayxBUtaM teVnAtaH sAkASxtatxtatxvXM parxgiVyateV \hfill || 10 ||  
\end{shl}

\begin{artha}
I elalxvU parxyatanxvilalxde A paramAtamxniMdale horage horaTubaMdide, adariMda A kAraNavasutxvige beVreyAgiradeV ananayxvAgide, adariMda kAyaRda tatavxvu ade eMdu heVLalapxDuvudu.
\end{artha}

\section*{baq. 4, bArx. 5, kaMDike 13}

\begin{shl}
sa yathA savARsAmapAM samudarx EkAyanameVvaM saveVRSAM sapxshARnAM tavxgeVkAyanameVvaM saveVRSAM ganAdhxnAM nAsikeV EkAyanameVvaM saveVRSAM rasAnAM jihevxYkAyanameVvaM saveVRSAM rUpANAM cakuSxreVkAyanameVvaM saveVRSAM shabAdxnAM shorxVtarxmeVkAyanameVvaM saveVRSAM saknakxlApxnAM mana EkAyanameVvaM savARsAM vidAyxnAM haqdayameVkAyanameVvaM saveVRSAM kamaRNA hasAtxveVkAyanameVvaM saveVRSAmAnanAdxnAmupasathx EkAyanameVvaM saveVRSAM visagARNAM pAyureVkAyanameVvaM saveVRSAmadhavxnAM pAdAveVkAyanameVvaM saveVRSAM veVdAnAM vAgeVkAyanamf || 13 ||
\end{shl}

\vishaya{`yathAseYnadhxva GanaH' itAyxdi maMtarxda athaR \mdash  }

\begin{shl}
anatxbaRhiVrasaGanaH seYnadhxvasayx GanoV yathA || \\
vijAcnxnaGana EvAyaM vijAcnxnAtAmx tatheYva ca \hfill || 11 ||  
\end{shl}

\begin{artha}
upipxna gaTiTxyu heVge oLagU horagU oMde rasavuLaLxdAdxgiruvudo, hAgeye I vijAcnxnAtamxnu (oLagU horagU) jAcnxna Ganave Agiruvanu.
\end{artha}

\vishaya{`EteVBoyxV BUteVBayxH' eMbudara athaR \mdash  }

\begin{shl}
savxvikAroVpasaMshelxVSAdivxjAcnxnAtamxtavxmeVtayxtaH || \\
tadedhxVtunAshAtapxrXkaqtiM sAvxmeVva parxtipadayxteV \hfill || 12 ||  
\end{shl}

\begin{artha}
ajAcnxnaveMba tananx vikArada saMbaMdhadiMda Atamxnu jiVvaBAvavanunx hoMdiruvanu, adara kAraNavu nAshavAdadxriMda tananx savxrUpavanenxV hoMdiruvanu.
\end{artha}

\vishaya{`avinAshiV anuciCxtitxdhamAR' eMba eraDu padagaLa punarukitxyanunx pariharisuvudu \mdash  }

\begin{shl}
pariNAmaniSeVdhaH sAyxdavinAshigirA\s \s tamxnaH || \\
anuciCxtitxgirA nAshoV vAyaRteV yoV niranavxyaH \hfill || 13 ||  
\end{shl}	

\begin{artha}
avinAshi eMba padadiMda Atamxnige pariNAmavanunx niSeVdhiside, `anuciCxtitxdhamAR' eMba padadiMda nibiRja nAshavanunx nivAriside.
\end{artha}

\vishaya{``mAtArx\s saMsagaRsatxvXsayx Bavati'' eMba mAdhayxMdina shurxtiya athaR \mdash  }

\begin{shl}
mAtArxsaMsagaRjasetxvXVSa yoV vinAshAdidhamaRvAnf ||  \\
avidAyxmAtarxheVtUtothxV rajujxsapARdivanamxtaH \hfill || 14 ||  
\end{shl}

\begin{artha}
yAvudu (jiVvavasutx) nAsha modalAda dhamaRvuLaLxdodx, adu viSaya saMpakaRdiMda uMTAdadudx, alalxde rajujxsapARdigaLaMte ajAcnxnamAtarxveMba kAraNadiMda huTiTxdudx.
\end{artha}

\vishaya{barxhamxpArxpitxge heVLida sAdhanagaLu beVre beVre adhAyxyagaLalilx heVLidadxriMda barxhamxvu aneVka rasavoV ? eMdare \mdash  }

\begin{shl}
barxhemxYveYkaH sa AtomxVkotxV hayxdhAyxyeVSu catuSavxRpi || \\
upAyamAtarxBeVdoV\s tarx na tUpeVyaH parxBidayxteV \hfill || 15 ||  
\end{shl}

\begin{artha}
barxhamxvoMdeV A AtamxneMdu nAlukx adhAyxyagaLalUlx heVLalapxTiTxruvudu, ililx upAyamAtarxdalilx BeVdavidadxrU paDeyuva vasutx BinenxYsuvudilalx.
\end{artha}

\vishaya{paDeyuva vasutxvinalilx BeVdavilalxveMdu sAdhisutAtx `saESaH' itAyxdi maMtarxda tAtapxyaRvanunx heVLutAtxre \mdash  }

\section*{baq. - 4, bArx. 5, 15 neV kaMDikeya BAga}

\begin{shl}
....... sa ESa neVti neVtAyxtAmxgaqhoyxV na gaqhayxteV\s shiVyoVR na hi shiVyaRteV\s saknogxV na hi sajayxteV\s sitoV na vayxthateV na riSayxti vijAcnxtAramareV keVna vijAniVyAdituyxkAtxnushAsanAsi meYterxVyeyxVtAvadareV Kalavxmaqtatavxmiti hoVkAtxvX yAjacnxvalokxyXV vijahAra || 15 ||
\end{shl}

\begin{shl}
neVti neVti catutheVR\s sw yatheYva parxtipAditaH || \\
neVtiVti pacnacxmeV\s peyxVvaM niviRkalopxV\s vadhAritaH \hfill || 16 ||  
\end{shl}
				
\begin{shl}
SaSoThxV\s pi janakAKAyxnapArxrameBxV tadavxdiVritaH || \\
niviRkalopxV yatheYkoV\s thoVR  neVtiVtayxtorxVpasaMdaqtaH \hfill || 17 ||  
\end{shl}

\begin{shl}
neVtiVti shAsArxvasitw tatheYvA\s \s tomxVpasaMhaqtaH || \\
EkarUpamatasatxtatxvXM savaRterxYva vivakiSxtamf \hfill || 18 ||  
\end{shl}

\begin{artha}
`neVti neVti' eMdu nAlakxne bArxhamxNadalilx (adhAyxyadalilx) heVge parxtipAdisalapxTiTxtoV hAgeye aidane bArxhamxNadalUlx neVti eMdu niviRkalapxvAgiruvudeMdu niNaRyisalapxTiTxruvudu, Arane bArxhamxNadalUlx janakana kathAraMBadalUlx adeV riVtiyAgi heVLalapxTiTxruvudo, niviRkalapxvAda adivxtiVya vasutxve eMdu `neVti' eMdu I shAriVraka bArxhamxNadalilx upasaMharisalapxTiTxruvudu, `neVti' eMdeV shAsatxrXvanunx upasaMharisuvAgalU hAgeye Atamxnu upasaMharisalapxTiTxruvanu, adariMda Atamxvu EkarUpavAgiruvudeMba tatavxvu elAlx kaDeyalUlx tAtapxyaR viSayavAgide.
\end{artha}

\begin{shl}
neVti neVtAyxtamxkAtatxtAtxvXtapxrXkArANAM shateYrapi || \\
nirUpayxmANeV nAnAyxdaqgayxsAmxtatxtavxM samiVkaSxyXteV \hfill || 19 ||  
\end{shl}
				
\begin{shl}
takaRtoV yadi vA vAkAyxdata EveVdameVva tu || \\
neVtineVtAyxtamxvijAcnxnamaqtatevxYkasAdhanamf \hfill || 20 || 
\end{shl}
				
\begin{shl}
sasaMnAyxsaM vinishecxVyamiti shAsarxsayx saMgarxhaH || \\
amaqtapArxpatxyeV\s laM sAyxtAsxdhanAnatxranisapxqqhamf \hfill || 21 ||  
\end{shl}
				
\begin{shl}
yathoVditamidaM jAcnxnaM sahakAyaRnayxsAdhana-\\
nirapeVkaSxmalaM mukAtxyX iteyxVtadaBidhiVyateV \hfill || 22 ||  
\end{shl}

\begin{artha}
`neVti neVti' eMbuva tatavxkikxMta beVre taraha tatavxvu aneVka parxkAragaLiMda pariVkiSxsidalUlx kANuvudilalx, takaRdiMdAgali, vAkayxdiMdAgali kANuvudilalx, adariMdale ideV tatavxvu, `neVti neVti' eMba Atamx vijAcnxnavu moVkaSxkekx sAdhana. adu saMnAyxsa sahitavAdadedxMdu nishacxyisabeVkeMbudeV shAsatxrXda saMkeSxVpa, moVkaSxvu laBisuvudakekx beVre sAdhanagaLa apeVkeSxyilalxde hiMde heVLida I jAcnxnaveV samathaRvAgide, sahakArisAdhanagaLanunx biTuTx beVre sAdhanagaLanunx apeVkiSxsade idoMdeV mukitxge sAkAgideyeMbudanunx muMde heVLuvudu.
\end{artha}

\vishaya{`EtAvadareV Kalavxmaqtatavxmiti hoVkAtxvX yAjacnxvalokxyXV vija\null{hA}ra' eMbudara vAyxKAyxna \mdash  }

\begin{shl}
EtAvadara EveYtadamaqtavxsayx sidadhxyeV || \\
parxtayxgAyxthAtamxyXvijAcnxnaM nAnayxtikxMcidapeVkaSxyXteV \hfill || 23 ||  
\end{shl}

\begin{artha}
eleY meYterxVyi, moVkaSxsididhxge iSeTxVsAku, adeVneMdare ! parxtayxgAtamxna nijasavxrUpajAcnxnavoMde, beVre oMdanunx savxlapxvU apeVkiSxsabeVkAgilalx.
\end{artha}

\vishaya{vija\null{hA}ra eMbudara athaR \mdash  }

\begin{shl}
ituyxkAtxthaRparijAcnxnadADhAyxRthaRM savaRsAdhanamf || \\
kamARNi cApanudAyx\s \s shu parxvavArxjAvicArayanf \hfill || 24 ||  
\end{shl}

\begin{artha}
I riVtiyAgi heVLida athaRjAcnxnada daqDhategAgi savaRsAdhakavAdadudx saMnAyxsave eMdu nidhaRrisi muniyu ciMtisade, kamaRgaLanunx tayxjisi shiVGarxvAgi horaTaru.
\end{artha}

\vishaya{`vija\null{hA}ra' eMbudakekx beVre athaR}

\begin{shl}
na keVvalamidaM jAcnxnaM kamaRsAdhananisapxqqhamf || \\
niHsheVSakamaRsaMnAyxsApeVkASx\s pi sAyxdayxtoV muniH \hfill || 25 || 
\end{shl}
				
\begin{shl}
samayxgivxjAcnxtatatatxvXtAvxtakxqqtAthoVR\s payxKilaM savxyamf || \\
tatAyxja kamaR tavxrayA vAknaBxnaHkAyasAdhanamf \hfill || 26 ||  
\end{shl}

\begin{artha}
I jAcnxnavu keVvala kamaR sAdhanada apeVkeSxyilalxdeyiruvudilalx, samasatx kamaR saMnAyxsada apeVkeSxyu iruvudu, adariMda muniyu sariyAgi tatavxvanunx tiLididadxriMdale kaqtAthaRnAgidudx savxyaM tavxreyiMda vAkukx, manasusx, kAya ivugaLa sAdhanavAgidadx kamaRvanunx biTaTxnu. (kamaRsanAyxsa mADidanu).
\end{artha}

\vishaya{PalitAMshaveVneMdare \mdash  }

\begin{shl}
atoV\s vagamayxteV nUnaM na kamaR sahateV\s Kilamf || \\
AtamxjAcnxnaM yatoV\s tAyxkiSxVtasxmayxgAjxcnXnoV\s pi tanumxniH \hfill || 27 ||  
\end{shl}

\begin{artha}
idariMda nishicxtavAgi hiVge tiLiyutatxde, EneMdare ? AtamxjAcnxnavu samasatx kamaRvanunx sahisuvudilalx, adariMda muniyu tatavxjAcnxnavuLaLxvanAdarU adariMda kamaRvanunx biTaTxnu.
\end{artha}

\vishaya{jAcnxna kamaRgaLige viroVdha heVge ?}

\begin{shl}
kamaRheVtuvirudadhxtAvxtasxmayxgAjxcnXnasayx kamaRNA || \\
viroVdhoV\s tayxthaRmeVvAtasatxtatxyXkatxM jAcnxnashAlinA \hfill || 28 ||  
\end{shl}

\begin{artha}
kamaRheVtuvAda ajAcnxnakekx viroVdhiyAdadxriMda tatavxjAcnxnakekx kamaRdoDane atayxMta viroVdhaveV ide, adariMda jAcnxnasaMpananxniMda kamaRvu tayxjisalapxTiTxtu.
\end{artha}

\begin{shl}
EvaM kANaDxdavxyeVneVyaM saPalA\s navasheVSataH || \\
savaRsaMnAyxsaniSAThx ca barxhamxvidAyx samiVritA \hfill || 29 ||  
\end{shl}

\begin{artha}
I riVtiyAgi eraDu kAMDagaLiMdalU PalasahitavAgi barxhamxvideyxyanUnx savaRsaMnAyxsa niSeThxyanunx vakatxvAyxMshavu savxlapxvU uLiyadaMte heVLidAdxyitu.
\end{artha}

\begin{shl}
EtAvAnupadeVshaH sAyxdevxVdeV sherxVyoVthiRnAM naqNAmf || \\
kaqtakaqtoyxV BaveVtikxSXparxmeVtajAjxcnXtAvx\s nushAsanamf \hfill || 30 ||  
\end{shl}

\begin{artha}
sherxVyasasxnunx bayasuva mAnavarige veVdadalilx iSuTx mAtarx upadeVsha mADiruvudu, idanunx shiVGarxvAgi tiLidu mAnavanu kaqtAthaRnAguvanu - eMbudu shurxtiya anushAsanavu.
\end{artha}

\begin{center}
iMteMbalilxge shirxVbaqhadAraNayxkoVpaniSadf BASayxda vAtiRkadalilx nAlakxne adhAyxyadalilx aidane bArxhamxNavu pUNaRvAyitu
\end{center}

\begin{shl}		
tirxsAhasirxV tathA pacnacx shatAnayxtarx samAsataH || \\
catAvxriMshatatxthA sholxVkAH SaSAThxdhAyxyasayx vAtiRkeV ||  
\end{shl}

\begin{center}

\vishaya{araNayxkadaMte Arane adhAyxyada vAtiRkadalilx mUrusAvirada ainUru nalavatutx (3540) sholxVkagaLu oTuTx irutatxve.}

\vishaya{(upaniSatitxnaMte ililxge 4neV adhAyxyavu pUNaRgoMDitu)}

\vishaya{|| dakiSxNAmUtaRyeV namaH ||}
\end{center}
