%~ \vishaya{baq - 6 - 2 - 11}

%~ \vishaya{vAtiRka 101 riMda}

\begin{shl}
ayaM veY loVkoV\s ginxrwgxtama tasayx paqthiveyxVva samidaginxdhURmoV rAtirxraciRshacxnadxrXmA aknAgxrA nakaSxtArxNi visuPxliknAgxsatxsimxnenxVtasimxnanxgwnx deVvA vaqSiTxM juhavxti tasAyx AhutAyx ananxM samaBxvati || 11 ||
\end{shl}

\vishaya{vAtiRka}

\begin{shl}
loVkoV\s yamaginxviRjecnxVyaH paqthiviV samiducayxteV | \\
paqthivAyx hi samidodhxV\s yaM samitetxVna kiSxtimaRtA \hfill|| 101 || 
\end{shl}

\begin{shl}
aginxdhURmasatxdutAthxnAdArxtirxraciRsatxtheYva ca | \\
rAtirxmAhuH kiSxticACxyAmaknAgxrAshacxnadxrXmAsatxthA \hfill|| 102 || 
\end{shl}

\begin{shl}
shAnatxtavxvatuRlatAvxBAyxM visuPxliknagxsamatavxtaH | \\
nakaSxtArxNi suPxliknAgxH suyxsatxsimxninxtAyxdi pUvaRvatf \hfill|| 103 || 
\end{shl}

\begin{shl}
ananxsayx saMBavasatxsAmxtupxmagwnx tacacx hUyateV | \\
puruSoV\s ginxriti dheyxVyoV vAyxtatxM tasayx samitasxmXqtA \hfill|| 104 || 
\end{shl}

\begin{shl}
vAyxtetxV muKeV hi tadidxVpitxdhURmaH pArxNoV muKoVtithxteVH | \\
puMdiVpetxVvARkninxmitatxtAvxdaciRvARketxVna BaNayxteV \hfill|| 105 || 
\end{shl}

\begin{shl}
sithxteVraknAgxravacacxkuSxraknAgxrAH shorxVtarxmeVva ca | \\
visuPxliknAgx iti jecnxVyaM tasayx vikeSxVpasaMsithxteVH \hfill|| 106 || 
\end{shl}

\begin{shl}
ananxM juhavxti tatArxgwnx saMBavoV reVtasasatxtaH | \\
yoVSidaginxriti jecnxVyA upasathxshacx samitatxthA \hfill|| 107 || 
\end{shl}

\begin{shl}
tadupasethxVna saMdiVpetxVdhURmoV loVmAni sAmayxtaH | \\
aciRvaRNaRsamAnatAvxdoyxVniraciRBaRveVtatxtaH \hfill|| 108 || 
\end{shl}

\begin{shl}
anatxH karoVti yatAkxmiV teV\s knAgxrAsatxtasxmAnataH | \\
visuPxliknagxH suKalavAH kaSxNikatevxYkaheVtutaH \hfill|| 109 || 
\end{shl}

\begin{shl}
reVtoV juhavxti tatArxgwnx deVvAshecxVnidxrXyarUpiNaH | \\
pacnacxmAyx AhuteVsatxsAyxH puruSaH saMBavatayxyamf \hfill|| 110 || 
\end{shl}

\begin{artha}
I loVkaveV aginx, paqthiviyeV samitetxMdu heVLide, paqthiviyiMda I
loVkavu parxkAshagoMDideyeMba kAraNadiMda paqthiviyanunx samitetxMdu
aBipArxyapaTiTxde.
\end{artha}

\begin{artha}
aginxyanunx dhUmaveMdu adariMda edidxruvudariMda heVLide, rAtirxyeV
aciR (jAvxle) \footnote{``EtAni hi caMdarxM rAterxV satxmasoV maqtoyxVbiRBayxta matayxpArayanf '' eMba shurxtiyiMda rAtirxyanunx
  tamasesxMdu tiLiyuvudu `tasayx ca maqtuyxveYRtamashACxyA teVneYva tajojxyXVtiSA matuyxM tamashAPxyAMtarati' eMba shurxtiyiMda BUmiya CAyeyeMbuva
tamaseVsx rAhuveMdU parxsidadhxvAgide. \\`udadhxyXtayxpaqthiviVcACxyAM nimiRtaM maMDalAkaqti||\\
savxBARnoVsatxdf baqhatf sAthxnaM taqtiVyaM yatatxmoV mayaM ||' \\eMba samxqqtiyiMda
paqthiviya neraLanunx meVlakekxtitx maMDalAkAravAgi nimiRtavAdaMtiruva
rAhuvina doDaDxsAthxnavu tamoVrUpavAdudeMdu heVLalapxTiTxde.}rAtirxyanunx BUmiya neraLeMdu heVLuvaru,
caMdarxneV aMgAra (keMDa) shAMtavAgiyU duMDAgiyU iruvadariMda hiVge
heVLide, nakaSxtarxgaLu kiDigaLaMte iruvudariMda kiDigaLu, A I
loVkaveMba aginxyalilx deVvategaLu vaqSiTxyeMba A hutiyanunx
koDuvareMbudAgi adariMda ananxvu Aguvudo elalx vAyxKAyxnavU
hiMdinaMteye.
\end{artha}

\vishaya{baq - 6 - 2 - 12 ne kaMDike}

\begin{shl}
puruSoV vA aginxrwgxtama tasayx vAyxtatxmeVva samitApxrXNoV dhUmoV vAgaciRshacxkuSxraknAgxrAH shorxVtarxM visuPxliknAgxsatxsimxnenxVtasimxnanxgwnx deVvA ananxM juhavxti tasAyx AhuteyxY reVtaH samaBxvati || 12 ||
\end{shl}

\begin{artha}
ananxvu (dhAnayxvu) A vaqSiTxyiMda huTuTxvudu, adu puruSAginxyalilx
hoVma mADalapxDuvudu, adariMda puruSaneV aginxyeMdu dhAyxnisabeVku. A
puruSana tereda bAyiyu samitutx, tereda bAyiyiMda puruSanu
mAtanADuvudu adhayxyana mADuvudU itAyxdi diVpanavu Aguvadu, pArxNaveV
dhUma, bAyiMdale pArxNavAyavu horaDuvudu. puruSana parxkAshavu
vAkikxna nimitatxdiMdaleV AgavudariMda vAkakxnunx aciR (jAvxle)yeMdu
heVLide, kaNuNx keMDadaMtiruvudariMda keMDaveMdU shorxVterxVMdirxyavu
atitxtatx vikeSxVpagoLuLxvudariMda kiDigaLeMdU dhAyxnisabeVku. A
puruSAginxyalilx ananxvanunx hoVmamADuvaru, adariMda reVtasusx
huTuTxvudu. 

\textbf{yoVSA vA aginx rwgxtama tasAyx upasathx Eva samitf loVmAnidhUmoV yoVniraciRH yadanatxH karoVti teV\s knAgxrAH aBinanAdx visuPxliknAgxH tasimxnenxVtasimxnf nanxgwnx deVvA reVtoV juhavxti tasAyx AhuteyxY puruSaH saMBavati sa jiVvati yAva jijxVvati||.....||}
\end{artha}

\begin{artha}
sitxrXVyeV aginx, upasethxyeV samitutx, adariMdaleV
udidxVpanavAguvudariMda samitetxMdu BAvisabeVku, roVmagaLe
kapApxgiruva hoVlikayiMda dhUmaveMdU yoVniye aciRyeMdu keMpu baNaNxda
hoVlikeyiMda heVLalapxDuvudu, adaroLage kAmuka puruSanu Enu mADuvano
ade\footnote{keMDagaLu beMkiyu shAMtavAgiruvudakekx kAraNa, adaraMte
  kAmukapuruSanu neDesuva meYthuna kamaRvu sitxrXVyeMba aginxgU,
  puruSanigU teVjoVviVyARdigaLa nAshakekx kAraNavAdadxriMda
  shAMtikaravAgide. A kamaRdalilx aMgAra BAvaneyanunx mADabeVkeMdu
 `tatf samAnataH' eMba vAtiRkada padadiMda sUcitavAguva aBipArxyavu.} aMgAraveMdU suKaleVshagaLeV kiDigaLeMdU kaSxNikatavx
mAtarx nimitatxdiMda tiLiyabeVku, AsitxrXVyeMba aginxyalilx
iMdirxyarUpavAda deVvategaLu reVtasasxnunx hoVma mADuvaru, aidane
AhutiyiMda I puruSanu huTuTxvanu.
\end{artha}

\vishaya{vAtiRka}

\begin{shl}
yathoVkatxvatamxRnA hAyxpaH pArxpatxH puMsapxriNAmatAmf | \\
pAkajaH pariNAmoV\s yameVvaM pacnacxBiraginxBiH \hfill|| 111 || 
\end{shl}

\begin{artha}
hiMde heVLida mAgaRdalilx Ahuti pariNAma sUkaSxmX jalavu
puruSAkAradalilx pariNAmavanunx hoMdutatxde. ideV riVtiyalilx
paMcAginxgaLa mUlaka pAkadiMda idu pariNAmavAguvudu.
\end{artha}

\vishaya{upasaMhAra -}

\begin{shl}
tasAmxdApaH sUkaSxmXBAvAH sUthxlatAM yAnitx pAkataH | \\
aginxBiH pacnacxBiH pakAvxH puruSAKAyx Bavanitx hi \hfill|| 112 || 
\end{shl}

\begin{artha}
adariMda sUkaSxmXvAda jalatatatxvXvu pAka visheVSadiMda
sUthxlarUpavanunx hoMduvudu. paMcAginxgaLiMda pakavxvAda Ajala
tatatxvXvu puruSaveMba hesaranunx paDeyuvudu.
\end{artha}

\vishaya{avataraNike}

\begin{artha}
paMcAginx videyxyuLaLxvarige muMde baruva gatiyanunx heVLalu  moTaTxmodalu adakekx upayoVgiyAda I videyxyanunx tiLida gaqhasathxnadeVhavanunx I riVtiyAgi vaNiRsiruvaru atha yadAmirxyateV -
\end{artha}

\vishaya{baq - 6 - 2 - 14 kaMDike.}

\begin{shl}
atheYnamaganxyeV haranitx tasAyxginxreVvAginxBaRvati samitasxmidUdhxmoV dhUmoV\s ciRraciRraknAgxrA visuPxliknAgx visuPxliknAgxsatxsimxnenxVtasimxnanxgwnx deVvAH puruSaM juhavxti tasAyx AhuteyxY puruSoV BAsavxravaNaRH samaBxvati || 14 ||
\end{shl}

\begin{shl}
tasAmxnamxqqtaM pirxyaM banudhxM haranatxyXganxya QutivxjaH \hfill|| 113 | \\
yatheYvA\s \s havaniVyAgenxVH parxsidadhxM samidAdikamf | \\
shamxshAnAgenxVsatxtheYveYtatasxvaRM jecnxVyaM na kalapxyXteV \hfill|| 114 || 
\end{shl}

\begin{artha}
(avanu yAvAga maraNa hoMduvano anaMtaraveV Itananunx (avana  deVhavanunx) aginxgAgi oyuyxvaru A puruSanige parxtayxkaSx aginxyeV aginx, hAgeye samitutx dhUma, jAvxle aMgAra, kiDigaLu elalxvU  parxtayxkaSxvAgiruvaveV horatu hiMdinaMte yAvudU kalipxtavalalxveMdu  shurxtiya athaR -)
\end{artha}

\vishaya{vAtiRkada anuvAda}

\begin{artha}
adariMda (kamaRkaSxyavAdanaMtara) maqtanAda pirxya, athavA baMdhu
ivananunx Qutivxjaru aginxgAgi oyuyxvaru, heVge parxsidadhxvAda
AhavaniVyAginxge samitutx modalAdavu iruvavo hAgeyeV shamxshAnada
aginxgU elalxvU iruvudeMdu tiLiyabeVku. yAvudanUnx hosadAgi
kalipxsuvudilalx.
\end{artha}

\begin{shl}
anatxyXsaMsAkxrasidadhxyXthaRM tasimxnanxgwnx yathoVditeV | \\
QutivxjoV juhavxti naramanAtxyXhuteyxY vidhAnataH | \\
AhuteVjARyateV tasAyxH pumAnABxsavxrarUpaBaqtf \hfill|| 115 || 
\end{shl}

\begin{artha}
aMtayx saMsAkxravu sididhxsalu A aginxyalilx hiMde heVLidadxralilx
Qutivxjaru I puruSananunx aMtAyxhutigAgi vidhiyaMte hoVma mADuvaru, A
AhutiyiMda puruSanu parxkAshavAda (sAtivxka) rUpavanunx dharisuvanu.
\end{artha}

\begin{shl}
rAjasaM tAmasaM rUpamitoV hayxnayxtarx vakaSxyXteV \hfill|| 116 | \\
itoV\s ginxBoyxV\s ginxmeVvAyaM sAvxM yoVniM parxtipatasxyXteV | \\
iti loVkeV samAcArAdaginxBayxH saMBavasatxtaH \hfill|| 117 || 
\end{shl}

\begin{artha}
I sAtitxvXka rUpavanunx biTuTx uLida rAjasa tAmasarUpavanunx
(kamiRyalilx) muMde heVLuvudu. I paMcAginxgaLiMda beVre
parxtayxkaSxvAda aginxyanenxV tananx utapxtitxge kAraNavAgiruvudaneVnx I
puruSanu seVruvanu I riVtiyAgi loVkadalilx (shurxtimAgARnusAriyAda
vidAvxMsara vagaRdalilx) vayxvahAraviruvudariMda aginxgaLiMda puruSana
utapxtitxyu Aguvudu.
\end{artha}

\vishaya{gati nirUpaNe}

\begin{shl}
pacnacxmAyxmAhutAveVvaM punAMmonxV janamx kiVtiRtamf | \\
gatisatxsAyxtha vakatxvAyx udagadxkiSxNaBeVdataH \hfill|| 118 || 
\end{shl}

\begin{artha}
I riVtiyAgi aidane Ahutiyalilx puruSaneMbuvana jananavanunx
heVLidAdxyitu, inunx muMde utatxra dakiSxNaveMba BeVdadiMda avana
gatiyanunx heVLabeVkAgide.
\end{artha}

\begin{shl}
teV ya EvameVtadivxduyeVR cAmiV araNeyxV sharxdAdhxM satayxmupAsateV teV\s ciRraBisamaBxvanatxyXciRSoV\s harahanx ApUyaRmANapakaSxmApUyaRmANapakASxdAyxnaSxNAmxsAnudaknAknxditayx Eti mAseVBoyxV deVvaloVkaM deVvaloVkAdAditayxmAditAyxdevxYduyxtaM tAnevxYduyxtAnupxruSoV mAnasa Etayx barxhamxloVkAnagxmayati teV teVSu barxhamxloVkeVSu parAH parAvatoV vasanitx teVSAM na punarAvaqtitxH || 15 ||
\end{shl}

\vishaya{paMcAginx videyxyuLaLxvarige aciRrAdi mAgaR}

\vishaya{vAtiRka}

\vishaya{parxthama parxshenxya utatxra}

\begin{shl}
teV ya EtadayxthAjAtaM jAcnxnaM pacnAcxginxsaMsharxyamf | \\
viduraciRH samAyAnitx bahUketxVshacx divxjAtayaH \hfill|| 119 || 
\end{shl}

\begin{artha}
teV = avaru yAru hiVge paMcAginxyanunx avalaMbisida upAsaneyanunx
tiLidiruvaro avaru (teV ye) eMdu bahuvacanadiMda divxjaru enunxvaru
aciR deVvateyanunx hoMduvaru.
\end{artha}

\begin{shl}
utapxtitxsaMsithxtilayA yathoVkAtxgenxyXVkaheVtavaH | \\
itathxM yeV viduraciRsetxV saMBavanAtxyXtamxBAvitAH \hfill|| 120 || 
\end{shl}

\begin{artha}
`nananx huTuTx sithxti layagaLu hiMde heVLida paMcAginxgaLeMba
muKayxkAraNadiMda Aguvavu' eMdu yAru tiLidiruvaro, avaru tananxlilx
mADida anusaMdhAnavuLaLxvarAgi aciR eMba mAgaRvanunx hoMduvaru.
\end{artha}

\vishaya{yeV ca eMbalilx heVLida adhikArigaLanunx tiLisuvaru -}

\begin{shl}
iSATxpUtaRkaqtoV yeV veY gArxmasethxVBayxshacx yeV pareV | \\
araNayx iti gaqhayxnatx itareVSAM paqthagagxrXhAtf \hfill|| 121 || 
\end{shl}

\begin{artha}
iSaTx\footnote{iSaTx} pUtaR eMba kamaRgaLanunx yAru mADiruvaro
matutx yAru gArxmasathxrigU beVreyAgiruvaro avarU ililx `araNeyx' eMba
mAtiniMda tegedukoLaLxbeVku. itararanunx beVreyAgi
tegedukoMDiruvudariMda (avaru ililx gArxhayxralalx).
\end{artha}

\vishaya{araNayxdalilxruva vAnaparxsathxrige iSATxdi kamaRgaLu elilxMda baMdavu ? eMdare -}

\begin{shl}
yajacnxdAnatapAMsiVha gaqhasathx iva tApaseV | \\
gaqhasAthxcAyaRvAsAcacx na garxhoV barxhamxcAriNAmf \hfill|| 122 || 
\end{shl}

\begin{artha}
yajacnx, dAna, tapasusxgaLu ililx gaqhasathxnalilxruvaMte tApasanalUlx
irabahudu, gaqhasathxnalUlx AcAyaRnalUlx vAsamADuvudariMda ililx
barxhamxcArigaLanunx tegedukoLuLxvudilalx.
\end{artha}

\begin{shl}
nAraNayxsAthx na ca gArxmAyx apeVkaSxnetxV\s tarx vidayxyA | \\
sAmAnayxvacasoVpAtetxVnaR visheVSaparigarxhaH \hfill|| 123 || 
\end{shl}

\begin{artha}
araNayxdalilxruvavarU, gArxmadalilxruvavarU I paMcAginx videyxge
apeVkiSxtaralalx, `teV ya Eva meVtadivxduH' eMdu sAdhAraNavAda
padavanenxV ililx parxyoVgisiruvudariMda visheVSAthaRvanunx
tegedukoLuLxvudalalx.
\end{artha}

\vishaya{Ivarege `yeV cAmiVaraNeyxV' eMbalilx gaqhasathxranunx vAna
  parxsathxranunx tegedukoLaLx beVkeMdu pUvaRpakaSxvanunx toVrisidaru -
inunx muMde sidAdhxMtavAgi vAnaparxsathxranunx parama haMsaranunx
biTuTxLida yatigaLanunx tegedukoLuLxvudeMdu niNaRya mADalu
horaTidAdxre -}

\begin{shl}
sasharxdadhxsAyxpi satayxsayx yadayxpayxnayxtarx saMBavaH | \\
tathA\s pi yatayoV gArxhAyxH araNeyxVna visheVSaNAtf \hfill|| 124 || 
\end{shl}

\begin{artha}
sharxdedhxyiMda kUDida satayxvu satayxhiraNayxgaBoRV (pAsaneyu) beVre
gaqhasathxralUlx saMBavisabahudu, AdarU araNayx eMdu visheVSaNavanunx
koTiTxruvudariMda yatigaLanunx (matutx vAnaparxsathxranUnx) ililx
garxhisabeVkAdudu.
\end{artha}

\vishaya{beVre oMdu pakaSxvanunx toVrisutAtxre -}

\begin{shl}
teV ya EvaM viduriti yadi vA gaqhiNAM garxhaH | \\
aginxsaMbanadhxtoV nAyxyoyxV vanasathxsAyxpi saMBavAtf \hfill|| 125 || 
\end{shl}

\begin{artha}
athavA `teV ya EvaMviduH' eMdu gaqhasathxranunx tegedukoLuLxvadu aginx
saMbaMdhaviruvudariMda nAyxyavAgide, hAgU vAnaparxsathxranunx
tegedukoLuLxvudu, avaralilx aginx saMbaMdhavu saMBavisuvudariMda
yukatx.
\end{artha}

\begin{artha}
kelavaru paMcAginx dashaRnavu aginxhoVtarxda sutxtiyeMdu tiLidu adu
udidxSaTxvalalxveMdU adakekx saMbaMdhisidaMte I gatiyU
udidxSaTxvalalxveMdU BAvisi ililx utatxrAyaNAdi mAgaRvanunx hiDidavara
nirUpaNeyeV asAthxneV saMBarxma eMbaMte ayukatxvenunxtAtxre - ivara
matavanunx I muMde tirasakxrisutAtxre -
\end{artha}

\begin{shl}
na cAginxhoVtarxsheVSatavxmukatxdaqSeTxVriheVSayxteV | \\
sAmAneyxVna garxhAtatxsAyxH pacnAcxginxriti liknagxtaH \hfill|| 126 || 
\end{shl}

\begin{artha}
hiMdeV heVLida paMcAginx dashaRnavu aginx hoVtarxkekx aMgaveMbudu
iSaTxvAgilalx, paMcAginx daqSiTxyanunx sAmAnayxvAgiyeV
tegedukoMDiruvudariMdalU `paMcAginxVnfveVda' eMdu paMcAginxgaLeMdu
tiLidu baMdiruvudariMdalU (inonxMdakekx aMgavAguvudilalx).
\end{artha}

\begin{shl}
tirxloVkiVsAdhanatAyxgAnAnxpi saMnAyxsinoV garxhaH | \\
deYviV vidAyx hi vitatxM sAyxdivxtAtxcacx vuyxtithxtiyaRtaH \hfill|| 127 || 
\end{shl}

\begin{artha}
matutx mUru loVkagaLa sAdhanegaLanunx tayxjisidadxriMda ililx
saMnAyxsigaLige garxhaNavilalx, matutx deVvatoVpAsaneyeV vitatx I
vitatxvanunx tayxjisi I sanAyxsigaLu niMtavaru eMbuvudariMdalU
saMnAyxsigaLanunx tegedukoLaLxbAradu.
\end{artha}

\vishaya{yatigaLu loVkatarxya sAdhanavanunx tayxjisuvareMbudakekx
  parxmANa -}

\begin{shl}
parxjayA kiM kariSAyxma AkeSxVpoV barxhamxveVdanAtf \hfill|| 128 | \\
pacnAcxginxjAcnxnavadABxvAyx gatirapayxtarx savaRdA | \\
yatoV\s toV gatirapuyxkAtx nAnayxthA tadudiVraNamf \hfill|| 129 || 
\end{shl}

\begin{artha}
`kiM parxjayA kariSAyxmoV yeVSAMnoV\s ya mAtAmxyaM loVkaH' eMdu barxhamxjAcnxnadiMda parxjA (putarxveMba) sAdhanavanunx  nirAkarisiruvu kaMDide. (adariMda loVkatarxyasAdhanavanunx  tayxjisuvudu) Adare paMcAginxvideyxyaMte ililx adara gati  (mAgaR)yanunx yAvAgalU ciMtisabeVku, adariMdaleV gatiyanUnx  heVLiruvudu beVre aBipArxyadalilx adanunx heVLidadxlalx.
\end{artha}

\vishaya{aciR padadiMda jAvxleyeV gArxhayxvalalx.}

\begin{shl}
deVvatoVpAsanaseyxVha parxkaqtatAvxtatxthA gateVH | \\
aciRHshabedxVna teVneVha deVvateYvAtarx gaqhayxteV \hfill|| 130 || 
\end{shl}

\begin{artha}
ililx deVvatoVpAsaneyeV parxkaqtavAgiruvudariMda hAgU parxkaqta
upAsaneya samiVpadalelxV gatiyanunx heVLiruvudariMda `aciRH' eMba
padadiMda adara aBimAni deVvateyeV ililx garxhisalapxTiTxde. 
\end{artha}

\footnotetext[1]{ililx deVvateya Ekatavxvanunx  hoMdutAtxreMbudanunx heVLidudx heVge eMbudanunx tiLiyabeVku,  vidAvxMsaru vimashiRsabeVku, BASayxdalilx hiVgilalx.}
\begin{shl}
aciRdeVRvatayeYkatavxM pArxpAyxhadeVRteYkatAmf\footnotemark[1] | \\
saMBavanitxVti savaRtarx saMbanodhxV\s tArxnuSajayxteV \hfill|| 131 || 
\end{shl}

\begin{artha}
aciRdeRVvateyoDane Ekatavxvanunx hoMdi ahadeRVvateyoDane Ekatavxvanunx
hoMduvareMdu `saMBavanitx' eMbudanunx muMde elalx kaDeyalUlx anavxya
mADalu joVDisikoLaLxbeVku.
\end{artha}

\vishaya{`aciRSoV\s haH' - eMbudara athaR -}

\footnotetext[2]{aciRrAdi padagaLiMda aciRrAdi aBimAniyAda deVvateyanunx  tegedukoLaLxdidadxre rAtirx kAladalilx maraNahoMduvavarige  hagalanunx heVLuva ahaHsisxna saMbaMdhavu Agade hoVguvudu,  ahasisxnalelxV maraNavanunx hoMduvareMbudu niyamavilalx, rAtirxyU  hoVguvaru, ahasasxnunx ivaru niriVkiSxsuvudU ilalx, shukalx pakaSxvU  niyatavalalx, utatxrAyaNavU niyatavalalx, adariMda ililx ahasusx  shukalxpakaSx utatxrAyaNa muMtAda kAlavAcakashabadxgaLiMda  aciRdeVvateyaMte deVvategaLeV namage  gArxhayxvAgive. `AtivAhikAsatxlilxMgAtf' iti sUtarx nAyxyadiMdalU  deVvateyeV gArxhayxvAgide. shariVravanunx biTuTx meVlakekx hoVguva  kamiR jiVvigaLige dAritoVrisuvavaru yAru, aciRrAdi shabadxgaLu  aceVtanavAgiruvalilx avugaLu jaDavAdadxriMdaleV toVrisalAravu savxtaH  jiVvarige IshariVravanunx biTuTx matotxMdu shariVradalilx  seVrikoLuLxva payaRMta I aMtarALadalilx liMga shariVradiMda  kUDidadxrU jAcnxnavAguvaMtilalx, adariMda toVrisuvavU jaDavAgi  hoVguva jiVvarU mUDharAgidadxre loVkAMtarakekx hoVguva dAriyu kANade  hoVguvudu, adariMda aciRrAdi shabadxgaLu aBimAnideVvategaLe, ivaru  AtivAhakaru, I shariVravanunx biTaTx kUDale kamaR upAsanegaLa  baladiMda hoVguva jiVvigaLige `ililxge hoVgu' ililxMda muMdakekx  hoVgu eMdu dAri toVrisutAtx jiVvigaLanunx mUMde muMde oyuyxva  deVvategaLu AtivAhakareMdu sUtarxkAraru sidAdhxMta  paDisidAdxre. adaraMte utatxrAyaNa mAgaRdalilxruva aciRrAdigaLu  AtivAhaka deVvategaLu, hiVgeyeV dakiSxNa mAgaRdalilxruva dhUmAdigaLU  deVvategaLeV eMdu tiLiyabeVku. mAgaRdashiRgaLanunx mAgaRveMdu  aupacArikavAgi vayxvaharisuvaMte ililxyU vayxvahariside. eMdu  BAvAthaR.}
\begin{shl}
\footnotemark[2]gaqhayxteV deVvatA noV ceVdaciRrAdigirA tadA | \\
asaMBavoV\s nayxtarx maqteVritayxpAthAR gatishurxtiH \hfill|| 132 || 
\end{shl}

%%% footnote shloka
\begin{artha}
aciRrAdi shabadxgaLiMda deVvateyanunx tegedukoLaLxdidadxreAvAga upAsakanige Palavu saMBavisuvudilalx, beVre kAladalUlx (kaqSaNxpakASxdigaLalUlx maraNavu AguvudariMda gati shurxtiyu athaRshUnayxvAguvudu.)
\end{artha}

\begin{artha}
BiVSaru maraNakAkxgi utatxrAyaNa kAlavanunx niriVkiSxsidudx
itihAsadalilxruvAga aharAdi shabadxvanunx kAlavAcakagaLeMde ? Eke heVLabAradu ? eMdare-
\end{artha}

\begin{shl}
yatUtxdagayanApeVkASx BiVSamxsayx shUrxyateV samxqqtw | \\
satayxvAditavxsidadhxyXthaRM shaMtanoVsatxdapiVkaSxyXtAmf \hfill|| 133 || 
\end{shl}

\begin{artha}
BiVSamxnige utatxrAyaNa niriVkeSxyidadxdudx samxtiyalilx(BAratadalilx) sapxSaTxvAgide yeMbudU saha shaMtanuvu (`niVnusavxcaCxnadx maqtuyxvuLaLxvanAgu' eMdu AshiVvaRdisidaMte) satayxvAdiyeMdu parxsididhxpaDisalu baMdiruvudeMdu\footnote{shurxti  samxqqtigaLalilx mAgaRpavaRvAgi kAla visheVSavaneVnx heVLuvudeMdu oMdeV  aBipArxyavidadxlilx maraNakAlavu niyatavalalxdadxriMda rAtirx dakiSxNAyana itAyxdikAladalilx hoVdaro ahasusx, utatxrAyaNa itAyxdi  kAla niriVkeSx mADabeVkAda parxsaMga baruvudu, hAge yAru  niriVkiSxsade  rAtirxyalolx hagalalolx, dakiSxNAyanadalolx,  utatxrAyaNadalolx sAyuvaraSeTx, ``atashAcxyaneV\s pidakiSxNeV" eMdu  sUtarxkArarU idaneVnx heVLutAtx kAlavisheVSavu  udidxSaTxvAgilalxvenunxtAtxre, adariMda kAladeVvatA vAcakagaLeMdeV  ahaH, ApUyaR mANapakaSx, utatxrAyaNada Aru mAsagaLu, saMvatasxra ivu  niNaRyavAgive. mahABAratadalilx} ililx tiLiyabeVku.
\end{artha}

\begin{shl}
anayxthA kaqtanAshaH sAyxdakaqtABAyxgamasatxthA | \\
deVvatAgarxhaNAtatxsAmxdedxVvateYvAciRrucayxteV \hfill|| 134 || 
\end{shl}

\begin{artha}
beVre athaRvanunx iTaTxre (gatiyilalxda jiVvarugaLige) hiMde mADida
puNayxgaLu nAshavAguvavu, matutx Iga mADadiruva kamaRgaLa Palavu
barabeVkAguvudu, adariMda deVvateyaneVnx tegedukoMDiruvudariMda ililx
aciR eMdu deVvateyeV heVLalapxTiTxde.
\end{artha}

\vishaya{`teVya'... itAyxdi maMtArxthaRda upasaMhAra -}

\begin{shl}
tasAmxdeVvaMvidoV dhiVrA aciRrAdayxBimAniniVmf | \\
karxmeVNa deVvatAM yAtAvx veYricnacxM yAnitx tatapxdamf \hfill|| 135 || 
\end{shl}

\begin{artha}
adariMa I riVti paMcAginxvideyxyanunx tiLida vidAvxMsaru (upAsakaru)
aciR ! modalAda aBimAnideVvateyanunx paDedu karxmavAgi barxhamxna
padaviyanunx seVrutAtxre.
\end{artha}

\vishaya{yeV ca - itAyxdi maMtArxthaRda upasaMhAra}

\begin{shl}
sharxdadxdhAnAsatxpasayxnatxH satayxM barxhamx samAhitAH | \\
upAsateV bahigArxRmAdaciRsetxV saMBavanatxyXtaH \hfill|| 136 || 
\end{shl}

\begin{artha}
sharxdedhxyuLaLxvarAgi tapasusx mADuvavaru, samAdhiyalilxrutAtx
gArxmada horage satayxbarxhamxvanunx (aparabarxhamxvanunx) upAsane
mADuvavarU elalxrU I aciR deVvateyanunx hoMduvaru.
\end{artha}

\vishaya{vAkayxkekx matotxMdu athaR  -}

\begin{shl}
barxhamxNaH sAthxnamAyAnitx yadAvx sAvxsharxmakamaRBiH | \\
saMnAyxsAdabxrXhamxNaH sAthxnaM tathAca samxqqtishAsanamf \hfill|| 137 || 
\end{shl}

\begin{artha}
athavA tamamx Asharxma dhamaRgaLiMda kUDi saMnAyxsa mADidadxriMda
barxhamxna sAthxnavanunx hoMduvaru `sanAyxsAdf barxhamxNasAthxnaM'- eMbuva sumxqqtiya
shAsanavU \footnote{modalu satayx shabadxkekx sUtArxtamxneMba
  athaRvanunx iTuTxkoMDu satayxvideyxyiMda Aguva Palavanunx heVLidaru,
Iga satayxvanunx heVLuvudeMbuva muKayxvAda AsharxmadhamaRvanunx
avalaMbisi yamaniyamAdi sAvxsharxmadhamaRgaLanunx AcarisuvudariMa
keVvala saMnAyxsadiMdalU utatxmAdhikArigaLU hiraNayxgaBaRna
sAthxnavanunx paDeyuvareMdu I vAtiRkada aBipArxya.}iruvudu.
\end{artha}

\vishaya{vAtiRka}

\vishaya{`aciR SoV\s ha''- itAyxdi maMtarx vAyxKAyxna}

\begin{shl}
tatoV\s hadeVRvatAM yAnitx shukalxpakaSxM tataH karxmAtf | \\
SaNAmxsAMshacx tatoV yAnitx huyxtatxrAyaNalakaSxNAnf \hfill|| 138 || 
\end{shl}

\begin{shl}
deVvatAM ca tatoV yAnitx deVvaloVkABimAniniVmf | \\
tata AditayxmAyAnitx veYduyxtaM cApi BAsakxrAtf \hfill|| 139 || 
\end{shl}

\begin{shl}
barxhamxNA manasA saqSoTxV mAnasaH puruSasatxtaH | \\
barxhamxloVkAnasx nayati soV\s payxBeyxVtAyxtha veYduyxtAnf \hfill|| 140 || 
\end{shl}

\begin{shl}
teV teVSu barxhamxloVkeVSu diVpayxmAnAH parAH samAH | \\
bArxhamxmAnAH samA gArxhAyx barxhamxloVkeVSu tacuCxrXteVH \hfill|| 141 || 
\end{shl}

\begin{artha}
A aciR deVvateyiMda ahadeRVvateyalilxge barutAtxre anaMtara alilxMdaAru utatxrAyaNada mAsadeVvategaLanunx seVruvaru anaMtaradeVvaloVkABimAniyAda deVvateyanunx hoMduvaru alilxMda AditayxdeVvateyanunx A nitayxniMda viduyxtf deVvateyanunx hoMduvaru alilxMdaAcege barxhamxna manaH saMkalapxdiMda saqSiTxyAda (amAnava) mAnasapuruSanu baMdu viduyxdedxVvateya adhiVnadalilxdadx I upAsakaranunxbarxhamx loVkakekx oyuyxvanu. avaru barxhamxloVkagaLalilxparxkAshisutAtx utatxmavAda saMvatasxragaLa payaRMta alilxyeV vAsamADuvaru, ililx A saMvatasxragaLu yAvudeMdare  barxhamxna mAnarUpavAgi tegedukoLaLxbeVku. EkeMdare ! barxhamxloVkadalilxnavatasxraveMdu heVLiruvudariMda adanenxV tegedukoLaLxbeVku.
\end{artha}

\vishaya{satayx loVkavu oMdAgidadxrU `barxhamxloVkeVSu' eMdu bahuvacanavanunx EtakAkxgi parxyoVgiside ? eMdare .}

\begin{shl}
sASiTxRsAloVkayxsAyujayxvayxpeVkASx bahugiVriyamf | \\
samaSiTxvayxSiTxBeVdaM vA vayxpeVkaSxyX bahuvAgiyamf \hfill|| 142 || 
\end{shl}

\begin{artha}
sASiTxR - sAloVkayx - sAyujayxveMba mukitxgaLa daqSiTxyiMda
bahuvacanavanunx parxyoVgiside, athavA samaSiTx (sAmAnayx) vayxSiTx
(visheVSa)gaLa BeVdavanunx iTuTxkoMDu I bahu vacanavu baMdide.
\end{artha}

\vishaya{`teVSAM na punarAvaqtitx :-' eMbudara athaR -}

\begin{shl}
AvaqtitxnaR punasetxVSAM yAvadABUtasaMpalxvamf | \\
ABUtasaMpalxvasAthxnamamaqtatavxM hi BASayxteV \hfill|| 143 || 
\end{shl}

\begin{artha}
avarige samasatx pArxNigaLu layavAguvavaregU punaH Avaqtitxyilalx
(saMsAra maMDalakekx hiMtirugi baruvudilalx), I bageyalilx
amaqtatavxveMbudanunx ``ABUta saMpalxvaM sAthxna mamaqtatavxM
hi BASayxteV'' eMdu (samxqqtikAraru heVLuvaru).
\end{artha}

\vishaya{athavA anAvaqtitx shurxtige muKAyxthaRvU irabahudu, heVge ? -}

\begin{shl}
aikAtamxyXdhiVsamutapxtetxVH yadivA barxhamxloVkataH | \\
mucayxnetxV na nivataRnetxV yathA dhUmAdimAgaRgAH \hfill|| 144 || 
\end{shl}

\begin{artha}
athavA alilxyeV EkAtamxsavxrUpajAcnxnavu AguvudariMda (aMtayxdalilx)
barxhamx loVkadalelxV mukatxrAgutAtxre. Adare dhUmAdi mAgaRvanunx
hoMdida kamiRgaLaMte hiMtiruguvudilalx.
\end{artha}

\vishaya{I divxtiVya pakaSxdalilx heVLuva athaRkekx gamakaveVnu ? eMdare -}

\begin{shl}
imaM mAnavamAvataRmitAyxdayxsayx visheVSaNAtf | \\
aterxYva kalepxV\s nAvaqtitxnaR tavxneyxVSavxnivAraNAtf \hfill|| 145 || 
\end{shl}

\begin{artha}
`imaM mAnava mAvataRmf' eMdu anAvatiR viSayadalilx visheVSaNavu  kANuvudariMda I kalapxdalelxV punarAvaqtitxyilalx, Adare beVre  kalapxgaLalilx anAvaqtitxyu ilalx (AvaqtitxyuMTu) Eke ? taDeyuvudakekx  yAvudU ilalxvAdadxriMda.
\end{artha}

\section*{baq 6 - 2 - 16 kaMDike.}

\begin{shl}
atha yeV yajecnxVna dAneVna tapasA loVkAcnajxyanitx teV dhUmamaBisamaBxvanitx dhUmAdArxtirxM rAterxVrapakiSxVyamANapakaSxmapakiSxVyamANapakASxdAyxnaSxNAmxsAnadxkiSxNAditayx Eti mAseVBayxH pitaqloVkaM pitaqloVkAcacxnadxrXM teV canadxrXM pArxpAyxnanxM Bavanitx tAMsatxtarx deVvA yathA soVmaM rAjAnamApAyxyasAvxpakiSxVyasevxVteyxVvameVnAMsatxtarx BakaSxyanitx teVSAM yadA tatapxyaRveYtayxtheVmameVvAkAshamaBiniSapxdayxnatx AkAshAdAvxyuM vAyoVvaqRSiTxM vaqSeTxVH paqthiviVM teV paqthiviVM pArxpAyxnanxM Bavanitx teV punaH puruSAgwnx hUyanetxV tatoV yoVSAgwnx jAyanetxV loVkAnapxrXtuyxtAthxyinasayx EvameVvAnuparivataRnetxV\s tha ya Etw panAthxnw na vidusetxV kiVTAH pataknAgx yadidaM danadxshUkamf || 16 ||
\end{shl}

\begin{shl}
deVvayAnaH samAseVna panAthx yatAnxtapxrXpacnicxtaH | \\
vAyxKAyx\s tha pitaqyANasayx samayxgAraBayxteV\s dhunA \hfill|| 146 || 
\end{shl}

\begin{artha}
deVvayAna mAgaRvanunx saMkeSxVpagarxMthadiMda parxyatanxdiMda
visAtxravAgi iLisidAdxyitu. Iga pitaqyANa mAgaRvanunx cenAnxgi
vAyxKAyxnisalu AraMBiside.
\end{artha}

\begin{shl}
deVvAdijAcnxnahiVneVna yeV\s tha yajecnxVna sadidxvXjAH | \\
loVkAcnajxyanitx dAneVna sadedxVshAdimatA tathA \hfill|| 147 || 
\end{shl}

\begin{shl}
niHsheVSakalamxSadhavxMsitapasA vA\s vipashicxtaH | \\
maqtAsetxV dhUmamAyAnitx dhUmAdArxtirxM tamasivxnaH \hfill|| 148 || 
\end{shl}

\begin{shl}
rAterxVraparapakaSxM ca dakiSxNAyanadeVvatAmf | \\
mAseVBayxH pitaqloVkaM hi pitaroV yatarx sheVrateV | \\
pitaqloVkAcacx teV canadxrXM yAnatxyXnanxM tadidxvwkasAmf \hfill|| 149 || 
\end{shl}

\begin{artha}
yAru deVvatA jAcnxnavilalxde keVvala yajacnxdiMda loVkagaLanunx
jayisuvaro hAgU satfpAtarx, deVsha modalAdavugaLiMda kUDida
dAnadiMdalU jayasuvaro, A sadivxjarU athavA niHsheVSavAgi
kalamxSavanunx nAshagoLisuva tapasisxniMdalo avidAvxMsarAdavaru maraNa
hoMdidavarAgi, dhUmadeVvateyanunx hoMduvaru, dhUmadeVvateyiMda
rAtirxdeVvateyanunx paDeyuvaru, rAtirx deVvateyiMda apara pakaSx
deVvateyanunx, alilxMda dakiSxNAyana deVvateyanunx, aMdare
mAsadeVvategaLanunx hoMduvaru, mAsa deVvategaLiMda muMde
pitaqloVkavanunx seVruvaru, yAva loVkadalilx pitaqgaLu nelasuvaro
(A loVkaveV pitaqloVkavu) pitaqloVkadiMda avaru caMdarxnanunx hoMdi
deVvategaLige ananxvAguvaru.
\end{artha}

\vishaya{`teV canadxrXM' pArxpayx itAyxdi maMtarxda athaR -}

\begin{shl}
canadxrXM pArxpAyxnanxBAvaM ca teV yatasatxtarx saMsithxtAH | \\
asakaqdaBxkaSxyanetxyXVtAnApAyxyAyx\s \s pAyxyayx soVmavatf \hfill|| 150 || 
\end{shl}

\begin{artha}
avaru caMdarxnanunx hoMdi ananx savxrUpavanunx paDedidadxriMda
alilxyeV idadx deVvategaLu I kamiRgaLanunx yajacnxdalilx
soVmarasavanunx pAna mADidaMte capapxrisi capapxrisi yAvAgalU
BakiSxsuvaru.
\end{artha}

\begin{shl}
pakeSxV shukelxV tamApAyxyayx kaqSeNxV taM BakaSxyanatxyXtha \hfill|| 151 | \\
BoVgasAdhanaBAvAshacx BakaSxyanitxVti BaNayxteV | \\
na tavxBayxvahaqtinAyxRyAyx veVdavatamxRni tiSaThxtAmf \hfill|| 152 || 
\end{shl}

\begin{artha}
shukalxpakaSxdalilx A caMdarxnanunx tuMbikoMDu kaqSaNxpakaSxdalilxavananunx deVvategaLu BakiSxsuvaru, `parxkaqta BakaSxyanitx'eMbudariMda BoVga sAdhanagaLanAnxgi mADikoLuLxvareMdu heVLide, BakaSxNaveMdare tinunxvudeMdalalx, yukatxvU alalx, EkeMdare! veVdamAgaRdalilxruva kamiRgaLanunx tiMdu hAkuvareMbudu\footnote[1]{satatxvX guNaveV adhikavAgiruva ahiMsAdi sAdhanagaLalilx parxvaqtatxrAgiruva  janareV veVdamAgaRdalilxruvavaru, iMthakamaRTharanunx deVvategaLu  savxgaRkekx baMdoDane tiMdu hAkuvudu satayxvAdare anathaRveV Aguvudu  savxgARBilASigaLige kamaR sAdhanegaLalilx parxvaqtitxyuMTAgalArdu  adariMda ivaralilx ananxveMdu shabadx parxyoVgamADidudx gwNa,  BakaSxNavU gwNAthaR, aMdare I kamiRgaLanunx deVvategaLu  upayoVgisikoMDu BoVgapaDuvareMdeV athaR, rAjaru (rANiyarana)  Baqtayxranunx seVvA mUlaka upayoVgisikoMDu BoVgapaDuvaMte, Adare  ivarige BoVgaveV ilalxveMdalalx, Baqtayxrige iruvaMte ivarigU  deVvategaLa sahavAsadalilx suKavU ide. adariMda ananxveMdare BoVga  sAdhanavAda BoVgayx vasutx eMdeV gwNAthaR, ``BAkatxMvA\s  nAtamxvitAvxtf `` eMba sUtarxdalilx niNaRyamADide.} nAyxya badadhxvAgilalx.
\end{artha}

\vishaya{`teVSAMyadA tatapxyaR veYti' - - - eMbudara athaR -}

\begin{shl}
yadA tUpacitaM kamaR payaRvasayxti BoVgataH | \\
teVSAmatheYtamadhAvxnamAvataRnetxV yathAgatamf \hfill|| 153 || 
\end{shl}

\footnotetext[2]{ililx aBoVginaH eMbudakekx BoVgavilalxdaveMdathaR batatx  muMtAda dhAnayxgaLalilx baMdu seVradavarige sasigaLige Aguva hiVna  janamxda duHKAnuBavavu ivarige ilalxveMde tiLisalu  mAnavAdishariVradalilx huTuTxvavaregU ivarige BoVgavilalxveMdu  tiLisalu I padavanunx parxyoVgisiruvaru batatx muMtAda sasigaLalilx  baMdu seVruvaru aSeTx. ``anAyxdhiSiTxteV pUvaRvadaBilApAtf'' eMba  sUtarxdalilx idara niNaRyavide. (barx. sU...) dhAnAyxdi rUpadalilx  baMdu seVrikoMDiruva I kamiRgaLu I jAgavanunx bahaLa kaSaTxdiMda  shariVradalilx oMdu seVruvaru anaMtara kamARnusAravAgi sitxrXV  puruSa davxMdavxdiMda nAnA bageyAgi huTuTxvareMdu BAvAthaR.}
\begin{shl}
AkAshamanuniSapxdayx vAyumAyAnatxyX\footnotemark[2]BoVginaH | \\
vAyoVvaqRSiTxmavApAyxtha vaqSeTxVshacx paqthiviVM tataH | \\
ananxBAvaM paqthivAyx\s tasetxV samAyAnitx BUmigAH \hfill|| 154 || 
\end{shl}

\begin{shl}
GaTiVyanatxrXvadashArxnAtx EvameVva punaH punaH | \\
parivataRnitx saMsAreV kamaRvAyusamiVritAH \hfill|| 155 || 
\end{shl}

%%% footnote shloka
\begin{artha}
yAvAga paripakavxvAda kamaRvu BoVgadiMda samApitxgoLuLxvudo AvAga avaru anaMtara heVge hoVgidadxro hAgeye I mAgaRkekx iLiyutAtxre hiMdirugutAtxre. heVgeMdare ? baruvAga AkAshavanunx hoMdi anaMtaravAyuvidadxlilxge baruvaru, vAyuviniMda vaqSiTxyanunx hoMdi adaroMdigeseVri alilxMda vaqSiTx mUlaka BUmige baruvaru, BUmige baMdavaru (naDuveBoVgavilalxdavarAgi) virxVhiyavAdi dhAnayxdoMdige seVrikoLuLxvaruhiVgeye rATeyaMte savxlapxvU  vishArxMtiyilalxde pade padeV IsaMsAradalilx kamaRveMba vAyuviniMda taLaLxlapxTaTxvarAgi sututxtatxleV irutAtxre.
\end{artha}

\vishaya{atha itAyxdi maMtarx BAgavanunx pUvaRsaMbaMdhavanunx tiLisi vAyxKAyxnisuvudu -}

\begin{shl}
dakiSxNasayx pathoV vAyxKAyx yathAvadanuvaNiRtA \hfill|| 156 | \\
yathoVkatxlakaSxNw yeV tu panAthxnAvutatxreVtArw | \\
na vidusetxV BavanitxVha kiVTAdAyx duHKaBoVginaH \hfill|| 157 || 
\end{shl}

\begin{artha}
dakiSxNa mAgaRvanunx idadxMteye vAyxKAyxna mADidAdxyitu hiMdeheVLida lakaSxNavuLaLx utatxra matutx dakiSxNa mAgaRgaLanunx yAvajiVvaru paDeyuvudilalxvo avaru ililx kiVTAdirUpavuLaLxvugaLAgiduHKakekx pAtarxrAguvaru.
\end{artha}

\vishaya{kAraNaveVneMdare ? -}

\begin{shl}
nA\s \s dirxyanatx idaM jAcnxnamudaknABxgARpitxsAdhanamf | \\
kamaR vA pitaqyANApwtx teV kiVTAdAyxmiyugaRtimf \hfill|| 158 || 
\end{shl}

\begin{artha}
utatxra mAgaR lABakekx sAdhanavAda I jAcnxnavanunx (upAsaneyanunx)yAru Adarisuvudilalxvo, hAgU pitaqyANa mAgaRvu laBisuvudakUkxsAdhanavAda kamaRvanunx apeVkiSxsuvudilalxvo, A jiVvigaLukirxmikiVTAdi rUpavAda dugaRtiyanunx paDeyuvaru.
\end{artha}

\begin{shl}
goVmayAduyxdaBxvAH kiVTAH pataknAgxH shalaBAsatxthA | \\
daMshAshacx mashakAshecxYva danadxshUkAH sapananxgAH \hfill|| 159 || 
\end{shl}

\begin{artha}
goVmaya modalAda kashamxla parxdeVshadalilx huTuTxva huLugaLU, diVpada meVle eraguva shalaBagaLU, pataMgagaLU, kaDiyuva noNagaLU, hAvu muMtAda viSajaMtugaLU eMdathaR.
\end{artha}

\vishaya{aidu parxshenxgaLanunx nAyxyavAgi niNaRyisabeVkAdadudx adara niNaRyavanunx mADade ideVnu akAMDatAMDavanunx mADide ? eMdare \mdash }

\begin{shl}
yatithAyxmiti yaH parxshanxH sa puMjanomxVkitxtoV gataH | \\
tadananatxrameVvoVkAtx vAyxvaqtitxshacx pathoVdavxRyoVH \hfill|| 160 || 
\end{shl}

\begin{artha}
`yatithAyxmf' eMdu yAva parxshenxyididxto adanunx puruSana janamxvanunx heVLidadxriMdaleV mugisidAdxyitu, adara naMtaraveV eraDu mAgaRgaLa parxshenxgaLigU avugaLanunx beVpaRDisi utatxrisidAdxyitu.
\end{artha}

\begin{shl}
AkAshAdayxBisaMBUtAyx punarAvataRnaM gatamf | \\
deVvayAnaM ca yatakxqqtAvx pitaqyANaM ca laBayxteV \hfill|| 161 || 
\end{shl}

\begin{shl}
sAdhanaM tadapiVhoVkatxM jAcnxnakamaRsavxlakaSxNamf | \\
loVkasAyxpUraNeV heVtuH samApAtxvuditaH suPxTaH \hfill|| 162 || 
\end{shl}

\begin{artha}
punarAvaqtitxyu heVge ? eMba eraDane parxshenxyu AkAshAdi mAgaRvanunx hoMduvareMdu heVLidadxriMdaleV mugiyitu. yAvudanunx mADi deVvayAna mAgaRvu laBisuvudu ? hAgU pitaqyANa mAgaRvu laBisuvudu ? eMba aidane parxshenxgU jAcnxna (upAsane) matutx kamaRveMbuva sAdhanagaLeV eMdu utatxriside. paraloVkavu pUNaRvAgadiralu kAraNaveVnu ? mUrane parxshenxyanunx samApitxyalilx bagehariside.
\end{artha}

\vishaya{adu heVge ? eMdare \mdash }

\begin{shl}
kiVTAdiVnAmagamanAdagxtAnAM cA\s \s gateVsatxthA | \\
EvaM pacnAcxpi niNiVRtAH parxshAnx yeV pArxkapxrXcoVditAH \hfill|| 163 || 
\end{shl}

\begin{artha}
kirxmi modalAdavugaLu savxgaRloVkakekx hoVgade iruvudariMdalU alilxge hoVda jiVvarugaLu punaH hiMdakekx baruvudariMdalU savxgaRloVkavu BatiRyAgade hAgeyeV iruvudeMdu tiLiyabeVku. I riVtiyAgi hiMde keVLida aidu parxshenxgaLanunx niNaRya mADidAdxyitu.
\end{artha}

\vishaya{sholxVkagaLa saMKeyx oTuTx 10936 ||}

\begin{center}
{\bf iti shirxVbaqhadAraNayxkoVpaniSadf BASayxda vAtiRkadalilx}
\smallskip

{\bf Arane adhAyxyadalilx eraDane bArxhamxNavu}
\smallskip

{\bf pUNaRgoMDitu.}

\smallskip
{\bf || shirxVdakiSxNAmUtaRyeV namaH ||}
\end{center}

\newpage

\vishaya{baq. a. 6, bArx. 3, 1neV kaMDike}

\vishaya{muMdina bArxhamxNada tAtapxyaR \mdash }

\begin{shl}
vitatxM vinA na sididhxH sAyxdadxqqSATxdaqSATxthaRkamaRNaH | \\
yatoV\s taH kamaR manAthxKayxM tAdatheyxVRnABiVdhiVyateV \hfill|| 1 || 
\end{shl}

\begin{artha}
vitatxvilalxde (darxvayxvilalxde) daqSaTxPalavuLaLxkamaRvU adaqSaTx PalavuLaLx kamaRvU sididhxsuvudilalx, adakAkxgi vitatxkAkxgi maMthaveMba kamaRvanunx muMde heVLuvudu.
\end{artha}

\begin{shl}
sa yaH kAmayeVta mahatApxrXpunxyAmituyxdagayana ApUyaRmANapakaSxsayx puNAyxheV dAvxdashAhamupasadavxrXtiV BUtwvxdumabxreV kaMseV camaseV vA sarwvxSadhaM PalAniVti samaBxqqtayx parisamuhayx parilipAyxginxmupasamAdhAya parisitxVyARvaqtAjayxM saMsakxqqtayx puMsA nakaSxterxVNa manathxM saninxVya juhoVti yAvanotxV deVvAsatxvXyi jAtaveVdasitxyaRcnocxV Ganxnitx puruSasayx kAmAnf teVBoyxV\s haM BAgadheVyaM juhoVmi teV mA taqpAtxH saveYRH kAmeYsatxpaRyanutx sAvxhA yA tirashicxV nipadayxteV\s haM vidharaNiV iti tAM tAvx Gaqtasayx dhArayA yajeV saMrAdhaniVmahaM sAvxhA || 1 ||
\end{shl}

\vishaya{putarxmaMtha bArxhamxNada vAtiRka tAtapxyaR \mdash }

\begin{shl}
pituloVRkoyxV yathA putarxH kamaRNeVhoVpajAyateV | \\
tAdaqkupxtArxthiRnaH kamaR tevxVSAmitAyxdinoVcayxteV \hfill|| 2 || 
\end{shl}

\begin{artha}
ihadalilx mADida yAva kamaRdiMda taMdege eMtaha putarxnu BoVgayxnAgi huTuTxvano aMtaha putarxnanunx bayasuvavanige beVkAda kamaRvanunx ESAmitAyxdi garxMthadiMda heVLuvudu.
\end{artha}

\vishaya{I kamaRkekx adhikAri yAru ? \mdash }

\begin{shl}
yaH sAyxtAkxmagarxhagarxsotxV vitAtxpatoyxVdaBxvaM parxti | \\
taM parxteyxVveVti vijecnxVyaM kameVRdaM na tavxkAminamf \hfill|| 3 || 
\end{shl}

\begin{artha}
yAvanu kAmaveMba mosaLeyiMda nuMgalapxTaTxvanAgi dhana, apatayx idara lABakAkxgi yatinxsuvano avanige I kamaRvanunx mADalu heVLiruvudu. Adare kAmaneyilalxdavanige heVLidadxlalx.
\end{artha}

\vishaya{IvAga maMtarx vAyxKAyxnada AraMBa \mdash }

\begin{shl}
pArxpunxyAM samuhadivxtatxmiti yaH kAmayeVta ha | \\
ituyxkAtxyX\s vApAtxvitatxsayx neVdaM kameVRti gamayxteV \hfill|| 4 || 
\end{shl}

\begin{artha}
yAvanu mahatAtxda dhanavanunx saMpAdisuveneMdu bayasuvano, avanige idu eMdu heVLidadxriMda dhanavanunx saMpAdisikoMDavanige I kamaRvalalxveMdu tiLiyutatxde.
\end{artha}

\vishaya{mahatf' eMbudakekx matotxMdu athaRvanunx heVLutAtx vAkayx yoVjane mADuvudu \mdash }

\begin{shl}
mahatatxvXmAtamxnoV\s tayxthaRM yoV vA kAmayateV gaqhiV | \\
mahatAtx na vinA vitatxM kamARtoV vitatxsidadhxyeV \hfill|| 5 || 
\end{shl}

\begin{artha}
yAva gaqhasathxnu tanage hecicxna mahatatxvXvu beVkeMdu kAmisuvano, avanige A mahatavxvu dhanavilalxde baruvudilalx. adariMda I kamaRvu dhanavanunx sAdhisuvudakAkxgi iruvudu.
\end{artha}

\vishaya{`udagayana' itAyxdi vAkayxda tAtapxyaR \mdash }

\begin{shl}
kAloV vidhiVyateV cAsayx manAthxKayxseyxVha kamaRNaH | \\
shurxtoyxVdagayanAduyxkAtxyX kamaRsidedhxyXY parxyatanxtaH \hfill|| 6 || 
\end{shl}

\begin{artha}
I maMthaveMba kamaRkekx yoVgayxvAda kAlavanunx `udagayaneV' itAyxdi vacanadiMda shurxtiyu kamaRsididhxgAgi parxyatanx pUvaRka vidhiside.
\end{artha}

\vishaya{yAvudAdaroMdu kAladalilx kamaRvanunx AcarisabAradeV ? eMdare \mdash }

\begin{shl}
pArxthaRniVyaH purA kAloV vishudidhxshAcx\s \s tamxnasatxtaH | \\
darxvAyxNoyxVSadhayashecxYva deVshashacx tadananatxramf \hfill|| 7 || 
\end{shl}

\begin{artha}
modalu kAlavanunx ciMtisabeVku, anaMtara Atamxna shudadhxte matutx kamaRkekx beVkAda darxvayxgaLanunx OSadhi vasutxgaLanunx (batatx modalAdavanUnx) anaMtara deVshavanunx ciMtisabeVku.
\end{artha}

\begin{shl}
kAlAdiVnAM guNAnAM ca kamaRNayxsimxnasxmucacxyaH | \\
na vikalapxH samaqdidhxH sAyxtakxmaRNoV\s sayx samucicxtw \hfill|| 8 || 
\end{shl}

\begin{artha}
kAlAdigaLanunx shudidhx, sharxdAdhxdiguNagaLanunx I kamaRdalilx oTuTxgUDisikoLaLxbeVku. vikalapxvu (yAvudAdaroMdu) eMbudu ivugaLalilx AgalAradu. I elalxvanunx kUDisikoMDalilx I kamaRkekx samaqdidhxyuMTAguvudu.
\end{artha}

\vishaya{sUtarxkAraru aginxhoVtArxdigaLaMte I kamaRvanunx Eke ? heVLalilalx ?}

\begin{shl}
kamaRNashAcxsayx niSapxtwtx vidavxtAtx\s pi samashirxtA | \\
yatoV\s taH sUtirxtaM neVdaM sUtarxkaqdiBxyaRthoVditamf \hfill|| 9 || 
\end{shl}

\begin{artha}
I kamaRda sAdhanege jAcnxnavU beVkAguvudu, adariMda sUtarxkAraru idanunx heVLidaMte sUcisalilalx.
\end{artha}

\begin{shl}
yoV ha veY jeyxVSaThxmituyxkAtxyX vidavxtAtx yA puroVditA | \\
tasAyxM satAyxmidaM kamaR jeyxVSAThxyeVtAyxdiliknagxtaH \hfill|| 10 || 
\end{shl}

\begin{artha}
`yoVhaveYjeyxVSaThxM sherxVSaThxM ca pArxNAnAM veVda' vacanadiMda yAva vidavxtatxvXvanunx (pArxNoVpAsaneyanunx) heVLididxto, A upAsaneyu iruvAga I kamaRvanunx Acarisalu heVLide. \footnote{``jeyxVSAThxya sAvxhA sherxVSAThxyasAvxheV tayxgwnxhutAvx'' eMdu muMdina 2neV kaMDikeyalilx jeyxVSaThx matutx sherxVSaThxveMbuva pArxNadeVvatege eraDeraDu AhutigaLanunx koDabeVkeMdu parxkaqtakamaRkekx beVkAda maMtarxda shabadx sAmathayxRdiMda pArxNoVpAsakanige I kamaRveMbudu tiLiyuvudu, I pArxNoVpAsaneyanenxV hiMdina vAtiRkadalilx heVLida vidavxtutx beVreyalalxveMdu tiLiyabeVku.}jeyxVSAThxya `itAyxdi vAkayxvu idakekx gamaka.
\end{artha}

\begin{shl}
BUmishayAyx payaHpAnaM barxhamxcayaRM ca vAgayxmaH | \\
upasadavxrXtameVtAsxyXdamAvAseyxVti liknagxtaH \hfill|| 11 || 
\end{shl}

\begin{artha}
`upasadfvarxtiV' eMbuvalilx heVLida upasadavxrXtaveMdare - nelada meVle malaguvudu, kiSxVrapAna, barxhamxcayaR, mwna eMbuva niyamagaLeV Aguvavu, idakekx gamakaveVneMdare ``udaga yaneV\s mAvAsAyxyAM diVkiSxtAvx pUvaR pakaSxsayx'' itAyxdi shurxtiyiMda `amAvAseyalilx (barxhamxcayARdi niyamaviruvudeMbudu parxsidadhxveMdu) sUcitavAgiruvudu.
\end{artha}

\begin{artha}
`upasada Eva varxtamf' eMbudAgi vuyxtapxtitxyiMda upasatetxMba varxtavuLaLxvaneMdu Eke athaRvirabAradu ? eMdare
\end{artha}

\begin{shl}
aginxmiteyxVkavacanAtatxtheYvoVlelxVKanAditaH | \\
na muKoyxVpasadAmatarx saMBavoV\s sitxVti gamayxteV \hfill|| 12 || 
\end{shl}

\begin{artha}
\footnote{`aginxmupasamAdhAya' eMba shurxtiyiMda AvapathayxveMba iSiTxgaLu avugaLu EkAginxge saMbaMdhisida kamaRvidu eMdu tiLiyuvudu. upasatutxgaLeMbavu joyxVtiSoTxVmAdiyAgagaLalilx parxvagAyxRhadalilx shurxtavAdavu, terxVtAginxgaLige saMbaMdhisive. hAgU ulelxVKana, parisamUhana muMtAdavugaLige saMbaMdhisida kamaRvu idu, upasatutxgaLige ivu saMBavisuvudilalx. adariMda `upasatutx varxtamf' - eMdu vuyxtapxtitxyiMda upasatutxgaLalilx heVLida nAlukx varxta (niyama) gaLuLaLxvaveMdu `upasadfvarxtiV' eMbuvalilx athaRvu. matutx joyxVtiSoTxVmadalilx upasatutxgaLanunx AcarisuvAga goVvina moleyanunx higigxsutatxlU kugigxsutatxlU kiSxVravanunx pAna mADabeVkeMdu niyamavide. ililx adilalxde keVvala payaRpAna mAtarx AcarisabeVkeMba niyamavu I maMthakamaRdalilx, idu sAmxtaRkamaR, adu shwrxtakamaR, I maMthakamaRvU shwrxtakamaRvAgidadxre parxkaqtiyalilx heVLida sakala aMgAnuSAThxnavU irutatxlididxtu, adariMda idu shwrxtavalalx AvasathayxveMba EkAginxyalelxV idanunx Acarisuvaru.}`aginxmupasamAdhAya' eMbuvalilx aginxmf' eMdu EkavacanaviruvudariMda hAgU ulelxVKanAdigaLU iruvudariMda ililx muKayxvAda upasatetxMba kamaRgaLige saMBavavilalxveMdu toVrutatxde.
\end{artha}

\vishaya{maMtarxdalilx auduMbara eMdare tAmarxveMdu kelavaru. hAgalalx \mdash }

\begin{shl}
tAmarxM nwdumabxraM gArxhayxmidhemxYsatxsayx viroVdhataH | \\
vAnasapxtayxmatoV gArxhayxM kamaRNwyxdumabxraM shurxtw \hfill|| 13 || 
\end{shl}

\begin{artha}
auduMbaraveMdare tAmarxveMdu ililx gArxhayxvalalx. `auduMbara idhamxH' eMbudaroDane A athaRkekx viroVdhaviruvudu. adariMda vanasapxtige saMbaMdhisidadxnenx auduMbaraveMdu kamaRdalilx upayukatxvAgiruvaMte shurxtiyalilx heVLide.
\end{artha}

\vishaya{`savwRSadhamf' itAyxdi vAkayxda athaR \mdash }

\begin{shl}
gArxmAyxNi dasha dhAnAyxni PalapuSApxNi savaRtaH | \\
aparxmAdakANiVha BakaSxyXmeVdhAyxni yAni ca \hfill|| 14 || 
\end{shl}

\begin{artha}
gArxmayxvAda dasha dhAnayxgaLu elalxvU, matutx parxmAdavanunxMTu mADadiruva Pala puSapxgaLU, yajacnxkekx yoVgayxvAda yAva pavitarx BoVjayxvasutxgaLiveyo 
\end{artha}

\vishaya{avelalxvU I kamaRdalilx gArxhayxvAgive || `vAkayxsheVSadalilx dasha' eMdu heVLidadxra tAtapxyaR \mdash }

\begin{shl}
dasheVti niyamAthaRM sAyxnanx tatoV\s nayxniSeVdhaneV | \\
BUridoVSoVpaduSaTxtAvxtapxrisaMKAyxvidheVriti \hfill|| 15 || 
\end{shl}

\begin{artha}
\footnote{`dasha' eMba hatatxnunx niyama mADide, adariMda, gArxmadalilx beLeyuva dhAnayxgaLalilx hatutx dhAnayxgaLanunx tegedukoLaLxbeVkeMdu niyama mADuvudaralilx tAtapxyaR, beVreyadanunx beVDaveMdu heVLuvudaralilx tAtapxyaRvilalx, hecAcxgi tegedukoMDare `dasha' eMdu visheVSaNavu bAdhakavAguvudilalx. anayxvanunx tirasakxrisuva parisaMKAyxvidhiyanunx bahudoVSayukatxvAdadxriMda ililx keY biTiTxde. shurxtiyalilx heVLida savaRvisheVSaNavanunx biDabeVkAguvudu. heVLadiruva `anayx niSeVdhaveMba athaRvanunx sivxVkarisabeVkAguvudu. gArxmayxvAda OSadhigaLeMdare I muMde heVLida hadineVLu dhAnayxgaLu {\rm --} virxVhi, yava, goVdhUma muMtAdavugaLeMdu TiVkAkAraru udAharisidAdxre \mdash  \\ virxVhayashacx yavAshecxYva goVdhUmA aNAvasitxlAH ||\\
pirxyaMgavoV\ BuyxdArAshacx koVradUSAH saciVNakAH ||\\
mASA mudAgxH masUrAshacx niSApxvAH sakuLitathxkAH ||\\
ADhakayx shacxNakAshecxYva shAyxmAH sapatxdashasamxqqtAH ||\\
iteyxVtA auSadiVnAM tu gArxmAyxNAM jAtayaH samxqqtAH ||\\ batatx, jave goVdhi, goVdhi, aNu eLuLx, navaNe, abhuyxdAra, kaDalekAyi, hAraka, ududx, hesaru, kADukaDale, avaDe, huruLi, togari, kaDale, shAyxmAka ivu 17 gArxmayxdhAnayxgaLu. yAvaka (alasaMdi).}hatutx eMdu niyamakAkxgi heVLide. adariMda beVre adhikavAgiruvudanunx niSeVdhisuvudakekx alalx, parisaMKAyxvidhiyu bahudoVSagaLiMda duSaTxvAgiruvudariMda adanunx ililx opipxlalx.
\end{artha}

\vishaya{vAtiRka}

\vishaya{kaMseVvA itAyxdi vacana vikalapxkekx tAtapxyaR \mdash }

\begin{shl}
kaMsAdAyxkArasidadhxyXthaRM kaMsAduyxkitxrihoVcayxteV | \\
taNuDxlAnaPxlapuSApxNi piSATxvX pAterxV nidhApayeVtf \hfill|| 16 || 
\end{shl}

\begin{shl}
dadhAnx ca madhunA piSaTxM samayxgAloVDayx pAtarxgamf | \\
sAthxpayeVtakxqqtarakaSxM sacuCxcw deVsheV parxyatanxtaH \hfill|| 17 || 
\end{shl}

\begin{artha}
\footnote{ililx kaMsadaMte AkAravirabeVku duMDAgirabeVku, athavA camasadaMte cwkavAgirabeVku. Adare adu auduMbaradalelxV (atitxmaradalelx) mADida pAterxyAgirabeVkeMdu mAtarx niyama. AkAra meVle heVLidaMte eraDaralilx oMdAgirabeVku.}kaMsAdi AkAravanunx tiLiyalu kaMsAdi shabadxvanunx ililx parxyoVgiside. akikxyanunx haNuNx, hUvugaLanunx aredu pAterxyalilx irisabeVku. pAterxyalilxruva hiTaTxnunx mosariniMdalU, jeVnutupapxdiMdalU cenAnxgi goTAyisi, kaDagoVliniMda  cenAnxgi kaDedu goTAyisi (kAge, nAyi itAyxdigaLa bAdheyiradaMte) shuciyAda sathxLadalilx kApADabeVku.
\end{artha}

\vishaya{`parisamuhayx parilapayx' itAyxdi vAkayxda tAtapxyaR \mdash }

\begin{shl}
pariVtAyxdigirA cAtarx deVshasaMsAkxra ucayxteV | \\
saMsakxqqteV BUparxdeVsheV\s tha hayxginxM saMsAthxpayeVdagxqqhiV \hfill|| 18 || 
\end{shl}

\begin{artha}
`parisamuhayx parilapayx' itAyxdi padagaLiMda ililx \footnote{parisamUhana, upaleVpana, ulelxVKana, udadhxraNa, aBuyxkaSxNa, eMbuva saMsAkxragaLu deVsha saMsAkxra kamaRgaLu. parxkaqta parisamUhana parileVpanagaLu BUmi saMsAkxragaLu.}deVsha saMsAkxravanunx (sathxLa shudidhxyanunx) heVLide, gaqhasathxnu saMsakxrisida BUparxdeVshadalilx anaMtara aginxyanunx parxtiSAThxpisabeVku.
\end{artha}

\begin{shl}
leVpanAdi ca kataRvayxM kaqtapUveVR\s pi tadivxdheVH | \\
\footnotemark[2]iti kAtiVyavacanamadaqSATxthaRtavxkAraNAtf \hfill|| 19 || 
\end{shl}
\footnotetext[2]{``daqSATxthaRtayA kaqta shoVdhaneV\s pideVsheV leVpanAdikataRvayxmfs'' eMdu kAtAyxyanaru racisida garxMthavAkayxvu ide. parisamuhayx, upalipayx, ulilxKayx, udadhxqqtayx, aBuyxkaSxyX itAyxdi. adaraMte daqSATxthaRvAgi shudidhx mADida BUmiyalUlx punaH adaqSATxthaRvAgi I saMsAkxragaLanunx mADabeVku, adakAkxgi parisamUhanAdigaLanunx heVLideyeMdu tiLiyabeVku. aginxsAthxpanAdikarxmavanunx gaqhayxsUtarxdalilx heVLidaMte iTuTxkoLaLxbeVku.}

%%%%footnote in shloka
\begin{artha}
shoVdhane mADidadxrU A BUparxdeVshadalilx leVpanAdi saMsAkxravanunx adara vidhiyaMte mADabeVkeMdu kAtAyxyanara vacanaviruvudu. EkeMdare, adu adaqSATxthaRvAgiruva kAraNadiMda kataRvayxvAgide.
\end{artha}

\footnotetext[1]{daBeRgaLanunx parisatxraNakAkxgi hAkuvudeMbudanunx ililx `parisitxVyARvaqtA'' eMdu sUcisidadxriMda pAkayajacnxdavidhiyaMte barxhAmxsana, AsatxraNa, parxNiVtA pAtArxsAdhana, parisatxraNa itAyxdigaLanunx vidhivatAtxgi mADi nivARsa mADi, Ajayxvanunx daBaRdalilx iTuTx payaRginxkaraNavanunx mADabeVkeMbudeV pAkayajacnx vidhAnaveMbudara athaR. oTiTxnalilx sAthxliVpAka parxyoVgadaMte Ajayx saMsAkxrAdigaLanunx sArxvavarxtapanAdigaLanunx mADabeVkeMdu athaR.}
\begin{shl}
\footnotemark[1]pAkayajacnxvidhAneVna saMsakxqqtAyx\s \s jayxM yathAvidhi | \\
\footnotemark[2]puMsA hasAtxdinakeSxVRNa manathxmAniVya sidadhxyeV \hfill|| 20 || 
\end{shl}
\footnotetext[2]{puruSa nakaSxtarxvanunx heVLidudx I maMthakamaRdiMda dhanasaMpatutx keYgUDuvudakekx, hAgilalxdidadxre `puMsA' eMdu shurxtiyalilx visheVSaNavanenxV heVLutitxralilalx.}

\footnotetext[3]{KajakeVna = modalaneya mathanadaMDa kaDagoVlu eMdathaR.}
\begin{shl}
\footnotemark[3]KajakeVneYkadhiVkaqtayXya manathxdarxvayxmataH paramf | \\
AvApasAthxna Ajayxsayx hutAvx \footnotemark[4]nitAyxhutiVsatxtaH \hfill|| 21 || 
\end{shl}
\footnotetext[4]{nitAyxhutigaLeMdare AGArAjayxBAgagaLu, mahAvAyxhaqtigaLu, savaRpArxyashicxtatx, sivxSaTxkaqtf itAyxdi rUpavAdavu.}

\footnotetext[5]{`yAvanotxV deVvAH' itAyxdi eraDu maMtarxgaLa athaR.}
\begin{shl}
\footnotemark[5]yAvanatx ituyxpakarxmayx manetxrXYrAjayxM yathAkarxmamf | \\
hutAvx hutAvx ca manethxV\s tha saMsarxvaM parxkiSxpeVnumxhuH \hfill|| 22 || 
\end{shl}

\begin{artha}
pAkayajacnxda vidhiyiMda shAsotxrXVkatxvAgi Ajayxvanunx saMsakxrisi puMnakaSxtarxvAda hasAtxdi nakaSxtarxdalilx samasatx OSadhi PalagaLa piSaTxvanunx taMdu (mathana mADuva) kaDagoVliniMda (atitxmarada camasa pAtarxdalilx mosaru, jeVnutupapx, tupapx ivugaLanunx hAki ApiSaTxvanunx hAki mathana mADi, anaMtara aginxgU tanagU balaBAgadalilx sAthxpisabeVku) Ahutiyanunx koDuva parxdeVshadalilx AjayxdiMda nitAyxhutigaLanunx koTuTx, `yAvanotxV deVvA''... itAyxdi maMtarxgaLiMda upakarxmisi karxmavAgi Ajayxvanunx hoVmamADi mADi saMsarxva BAgavanunx (sutxvadalilx leVpisida Ajayxvanunx) pade padeV maMtha darxvayxdalilx hAkabeVku.
\end{artha}

\vishaya{baq. a. 6, bArx. 3, 3neV kaMDike}

\begin{shl}
jeyxVSAThxya sAvxhA sherxVSAThxya sAvxheVtayxgwnx hutAvx manethxV saMsarxvamavanayati pArxNAya sAvxhA vasiSAThxyeY sAvxheVtayxgwnx hutAvx manethxV saMsarxvamavanayati vAceV sAvxhA parxtiSAThxyeY sAvxheVtayxgwnx hutAvx manethxV saMsarxvamavanayati cakuSxSeV sAvxhA samapxdeV sAvxheVtayxgwnx hutAvx manethxV saMsarxvamavanayati shorxVtArxya sAvxhAyatanAya sAvxheVtayxgwnx hutAvx manethxV saMsarxvamavanayati manaseV sAvxhA parxjAteyxY sAvxheVtayxgwnx hutAvx manethxV saMsarxvamavanayati reVtaseV sAvxheVtayxgwnx hutAvx saMsarxvamavanayati || 2 ||
\end{shl}

\begin{shl}
aganxyeV sAvxheVtayxgwnx hutAvx manethxV saMsarxvamavanayati \\
soVmAya sAvxheVtayxgwnx hutAvx manethxV saMsarxvamavanayati \\
BUH sAvxheVtayxgwnx hutAvx manethxV saMsarxvamavanayati\\
BuvaH sAvxheVtayxgwnx hutAvx manethxV saMsarxvamavanayati \\
savxH sAvxheVtayxgwnx hutAvx manethxV saMsarxvamavanayati \\
BUBuRvaHsavxH sAvxheVtayxgwnx hutAvx manethxV saMsarxvamavanayati \\
barxhamxNeV sAvxheVtayxgwnx hutAvx manethxV saMsarxvamavanayati \\
kaSxtAtxrXya sAvxheVtayxgwnx hutAvx manethxV saMsarxvamavanayati \\
BUtAya sAvxheVtayxgwnx hutAvx manethxV saMsarxvamavanayati \\
BaviSayxteV sAvxheVtayxgwnx hutAvx manethxV saMsarxvamavanayati \\
vishAvxya sAvxheVtayxgwnx hutAvx manethxV saMsarxvamavanayati \\
savARya sAvxheVtayxgwnx hutAvx manethxV saMsarxvamavanayati \\
parxjApatayeV sAvxheVtayxgwnx hutAvx manethxV saMsarxvamavanayati || 3 ||
\end{shl}

\vishaya{vAtiRka}

\begin{shl}
sAvxhAkArAvasAnAH suyxmaRnAtxrXH saveVR yathoVditAH \hfill|| 23 | \\
manatxrXdavxyeVna jeyxVSAThxdw hoVmaH kAyoVR vijAnatA | \\
agAnxyXdAveVkashaH kAyoVR yAvanamxnAthxvamashaRnamf \hfill|| 24 || 
\end{shl}

\begin{artha}
I meVlina maMtarxgaLalilx sAvxhAkAradiMda koneyalilx munisida BAgavelalxvU maMtarxgaLeV. eraDu maMtarxgaLiMda jeyxVSAThxdigaLige eraDereDu AhutigaLanunx hoVma mADuvavanu tiLidu mADabeVku, reVtaseVsAvxhA, aganxyeVsAvxhA itAyxdi maMtarxdiMda oMdoMdu Ahutiyanunx maMthadarxvayxda aBimashaRna mADuva payaRMta koDabeVku.
\end{artha}

\vishaya{baq. a. 6, bArx. 4, 3neV kaMDike}

\begin{shl}
atheYnamaBimaqshati Barxmadasi javxladasi pUNaRmasi parxsatxbadhxmaseyxVkasaBamasi hiknakxqqtamasi hiknikxrXyamANamasuyxdigxVthamasuyxdigxVyamAnamasi shArxvitamasi parxtAyxshArxvitamasAyxderxVR sanidxVpatxmasi viBUrasi parxBUrasayxnanxmasi joyxVtirasi nidhanamasi saMvagoVR\s siVti || 4 ||
\end{shl}

\vishaya{vAtiRka}

\begin{shl}
atha sivxSaTxkaqdanetxV\s simxnohxVmAnanatxrameVva tatf | \\
divxtiVyeVna mathA\s \s loVDayx hayxthABimaqshati pANinA \hfill|| 25 || 
\end{shl}

\begin{artha}
anaMtara sivxSaTxkaqtf eMbuva hoVmapayaRMta ivanunx mADida naMtaraveV eraDaneya kaDagoVliniMda A maMthadarxvayxvanunx mosarinalUlx jeVnutupapxdalUlx, tupapxdalUlx mathana mADi kUDaleV keYyiMda sapxshiRsabeVku.
\end{artha}

\vishaya{adara maMtarx matutx maMtArxthaR \mdash }

\begin{shl}
BarxmasiVtAyxdinA manathxM samxraMsatxdedxVvatAM haqdA | \\
Barxmasi pArxNaBUtasatxvXM na heyxVkatArxvatiSaThx meV \hfill|| 26 || 
\end{shl}

\begin{artha}
`Barxmadasi' - itAyxdi maMtarxdiMda adara deVvateyanunx manasisxnalilx samxrisutAtx maMthadarxvayxvanunx sapxshiRsabeVku. niVnu pArxNavAgidudx sututxtatxliruve, oMdu kaDeyalUlx niVnu nilulxvudilalxvaSeTx. (eMdu maMtarxda athaR)
\end{artha}

\vishaya{`javxladasi pUNaRmasi' eMbudara athaR}

\begin{shl}
javxlajAjxjavxlayxmAnasatxvXM jaTharasAthxnanxpAkakaqtf | \\
pUNaRM ca savaRtoVvAyxpi savaRtArxnavaKaNiDxtamf \hfill|| 27 || 
\end{shl}

\begin{artha}
niVnu hecAcxgi uriyutAtx hoTeTxyoLagiruva ananxvanunx pacana mADuvavanAgiruvudariMda javxlisuve, matutx pUNaRvAgididx. EkeMdare ? elalx kaDeyalUlx vAyxpisididx, elAlx sathxLadalUlx barxhamxrUpadiMda apariciCxnanxnAgiruve.
\end{artha}

\vishaya{`parxsatxbadhxmasi' eMbudara athaR \mdash }

\begin{shl}
parxsatxbodhxV\s si sithxratAvxcacx viyadavxnanx vikamapxseV \hfill|| 28 || \\
savaRtoV\s paripanithxtAvxtasxveYRshAcxpayxnukUlataH | \\
tavxmeVkashaPamituyxkatxM shaPeYkA yA tavxdAtimxkA \hfill|| 29 || 
\end{shl}

\begin{artha}
sithxravAgiruvudariMdalU parxsatxbadhxnAgiruve AkAshadaMte alAlxDade iruve, alalxde elilxyU (taDeyilalxdavanAdadxriMda) parxtibaMdhakavalalxdadxriMdalU elalxrigU anukUlavAgiruvudariMda niVnu EkashaPanAgiruve, yAva kAlugorasideyo adu ninanx savxrUpaveV Agiruvudu (aMdare samasatx jagatutx ninanx oMdu vAdada aMshavAgiruvudariMda ninanx savxrUpaveV Agiruvudu).
\end{artha}

\begin{shl}
udAgxtArx hiMkaqtaM pUvaRM sotxVtarxmudAgxyatA suPxTamf | \\
yajAcnxrameBxV tadA madheyxV giVyamAneV\s tha sAmani \hfill|| 30 || 
\end{shl}

\begin{shl}
api hiMkirxyamANoV\s si tathoVdigxVtheV vinidiRsheVtf | \\
shArxvitoV\s dhavxyuRNA ca tavxmaginxVdhArx ca tathoVtatxramf \hfill|| 31 || 
\end{shl}

\begin{artha}
udAgxtaq eMba QutivxjaniMda modalu sotxVtarx mADuvAga yajAcnxraMBadalilx hiMkAravu mADalapxDuvudu hAgU sAmagAnavanunx mADuvAgalU madhayxdalUlx himf' eMdu shabadx mADalapxDuvudu. adariMda niVnu hiMkaqta, matutx hiMkirxyamANa' eMdu Agiruve, hAgU udigxVthaveMba sAmaBakitxyanunx heVLuvAga `udigxVthavAgididx', niVnu adhavxyaRviniMdalU anaMtara aginxVdharxniMdalU sharxvaNa mADisalapxTiTxruve.
\end{artha}

\vishaya{`AderxRV saMdiVpatxmasi' eMbudara athaR}

\begin{shl}
AderxVR meVdhoVdareV viduyxtasxMdiVpotxV\s siVti kathayxteV | \\
vividhaM tavxmeVva Bavasi yatoV\s toV viBurucayxteV \hfill|| 32 || 
\end{shl}

\begin{shl}
soVmoV vaqSATxyXdiBAveVna parxBuH parxBavasiVtayxtaH | \\
AditayxH pArxNaBAveVna hayxnanxBAveVna soVmatA \hfill|| 33 || 
\end{shl}

\begin{artha}
AdarxR aMdare moVDada oLaBAgadalilx miMciniMda hoLeyutAtx iruve eMdu heVLalapxDuvudu, hAgU niVneV nAnA bageyAgi iruveyAdadxriMda viBu eMdU heVLalapxDuve, vaqSiTx modalAda rUpadiMda samathaRnAgiruveyAdadxriMda parxBuvAgiruve, `joyxVtirasi' eMdare pArxNarUpadiMda AditayxnAgididx, ananxmasi' eMdu BoVgayxrUpadalilx soVmanAgididx.
\end{artha}

\begin{shl}
anAnxnAnxdadavxyaM savaRM tavxmeVva tadapi parxBoV | \\
nidhanaM kAraNatAvxtatxvXM kAraNeV kAyaRsaMlayaH \hfill|| 34 || 
\end{shl}

\begin{artha}
ananx, anAnxda (BoVgayx, BoVkatx) eMdu eraDAgiruva elAlx jagatutx niVneV AgididxVye, alalxde niVneV nidhana kAraNanAgiruvudariMda AgididxVye, kAraNadalelxV kAyaRkekx layaveMbudu parxsidadhxvAgide.
\end{artha}

\begin{shl}
vAgagAnxyXdAyxtamxsaMpAtAtasxMvagoVR\s siVti BaNayxteV | \\
iti manatxrXH samApotxV\s yamaBimashaRnakamaRNi \hfill|| 35 || 
\end{shl}

\begin{artha}
AdhAyxtamxdalilx vAgAdi iMdirxyagaLu pArxNadalilx upasaMhAravAguvudariMdalU, adhideYvadalilx aginx modalAda deVvategaLu vAyuvinalilx upasaMhAravAguvudariMdalU niVnu saMvagaR eMdAgiruve. ililxge \footnote{maMthadarxvayxkekx pArxNaveV deVvateyAgiruvudariMdalU A samaSiTx pArxNadeVvateyu savARtamxkavAgiruvudariMdalU pArxNadeVvateyoDane I darxvayxvanunx oMdAgi mADi savARtamxkaveMdu I maMtarxgaLalilx heVLide, samasatx deVhagaLalUlx pArxNarUpadiMda `Barxmadasi' eMdu heVLide pArxNavu calanarUpavAdadxriMdalU pArxNarUpavAdadxriMdalU darxvayxvanunx BarxmaNashiVlavAgide eMdu heVLide. aginxrUpadiMda javxlisuvAga javxladasi' eMdu heVLide. aginxyu parxkAshAtamxka, I darxvayxvU ArUpavAgide. barxhamxrUpadiMda pUNaRvAgiruve eMdu pUNaRmasi eMdide, AkAsharUpadiMda `parxsatxbadhx masi' kaMpanavilalxde niMtiruve. ide riVtiyAgi hiMde vAyxKAyxnisidaMte hiMkaqta, hiMkirxyamANa, udigxVtha itAyxdi rUpadalilx I darxvayxvanunx oMdAgi mADi sutxtisiruvudeMdu maMtarxda saMkiSxpatx tAtapxyaR, idariMda maMthadarxvayxvanunx aBimashiRsabeVku.}maMtarxvu maMthadarxvayxvanunx aBimashaRna mADuvudakekx upayukatxvAgidudx samApitxgoMDitu.
\end{artha}

\vishaya{vAtiRka}

\begin{shl}
atheYnaM manatxrXpUtaM sadudayxcaCxti yathoVditamf | \\
AmaMsiVtAyxdimanetxrXVNa tadathARviSakxqqtisitxvXyamf \hfill|| 36 || 
\end{shl}

\begin{artha}
anaMtara maMtarxdiMda pavitarxgoLisida I maMthadarxvayxvanunx hiMde heVLidaMte `AmaMsi' itAyxdi maMtarxdiMda udayxmana mADuvudu aMdare pAterxyoMdige keYyalilx hiDiyuvudu. A maMtarxda athaRvanunx parxkaTiside - 
\end{artha}

\vishaya{baq. a. 6, bArx. 3, 5neV kaMDike}

\begin{shl}
atheYnamudayxcaCxtAyxmaM sAyxmaM hi teV mahi sa hi rAjeVshAnoV\s dhipatiH sa mAM rAjeVshAnoV\s dhipatiM karoVtivxti || 5 ||
\end{shl}

\begin{shl}
AmaMsiVti BaveVdUrxpaM jAcnxnAthaRseyxYva manayxteVH | \\
leVTAyxknUpxvaRsayx saMsidadhxM shabulxkiVtayxvadhAraNAtf \hfill|| 37 || 
\end{shl}

\begin{artha}
Aknf upasagaR pUvaRkavAda mana eMbuva jAcnxnAthaRvuLaLx dhAtuvina leVTf eMba lakAradalilx AmaMsi' eMdu rUpavAguvudu. heVge ? eMdare shapf' eMba vikaraNa parxtayxyakekx lukf (loVpavu) uMTAgalu sidadhxvAguvudu.
\end{artha}

\vishaya{adara athaR \mdash }

\begin{shl}
AsamanAtxdivxjAnAsi sUkASxmXdi jagati sithxtamf | \\
jecnxVyaM yAvajajxgatikxMcidAmaMsiVti tataH sadA \hfill|| 38 || 
\end{shl}

\begin{artha}
jagatitxnalilx yAvudoMdu tiLiyabeVkAda sUkaSxmX modalAda vasutxvidadxrU avelalxvanunx yAvAgalU niVnu tiLidiruve adariMda AmaMsi eMdu heVLide.
\end{artha}

\vishaya{AmaMhi' eMbudara athaR \mdash }

\begin{shl}
yathA\s sAmxMsatxvXM vijAnAsi tatheYva tAvxM vayaM sadA | \\
manAyxmaheV variVyAMsaM guNavadoBxyXV guNAdhikamf \hfill|| 39 || 
\end{shl}

\begin{artha}
heVge niVnu namamxnunx tiLidididxVyo, hAgeyeV ninanxnunx nAvu yAvAgalU tiLididedxVve. heVgeMdare guNavaMtarigU hecicxna guNavuLaLxneMdU sherxVSaThxneMdU tiLidiruvevu.
\end{artha}

\vishaya{`AmaM hi' eMdu eraDu padavAgi garxhisiyU athaR mADabahudu \mdash }

\begin{shl}
mahi mahatatxvXM jAniVma AmoV\s si tavxM tathA parxBoV | \\
na tavxM pAkasamAyoVgAtaPxlavanAnxshamaqcaCxsi \hfill|| 40 || 
\end{shl}

\begin{artha}
ninanx mahi - eMdare mahimeyanunx nAvu tiLididedxVve, adariMda eleY parxBuve niVnu Ama tiMdu AgididxVye, matutx niVnu pAkavisheVSadiMda haNiNxnaMte nAshavanunx hoMduvavanalalx.
\end{artha}

\begin{shl}
AmoV\s pakaM tavxyi gataM mahatatxvXmapi BaNayxteV | \\
yasAmxdArxjA\s sayx savaRsayx mAM ca deVvaH sa savaRdA \hfill|| 41 || 
\end{shl}

\begin{shl}
rAjAnaM ca tatheVshAnaM karoVtavxdhipatiM ca mAmf | \\
udayxmAyxneVna manetxrXVNa BakaSxyatayxtha BAgashaH \hfill|| 42 || 
\end{shl}

\begin{artha}
yAvudariMda I elalx jagatitxgU rAjanAgiruvano (hAgU IshavxranU adhipatiyU) Agiruvano adariMda A pArxNadeVvanu nananxnunx rAjananAnxgiyU IshavxrananAnxgiyU adhipatiyanAnxgiyU mADali.
\end{artha}

\begin{artha}
`atheYna mAcAmati tatasxvituvaRreVNayxmf |' I maMtarxdiMda adanunx meVlakekx keYyalilx etitx hiDidukoMDu BAga mADi BakiSxsabeVku.
\end{artha}

\vishaya{baq. a. 6, bArx. 3, 6neV kaMDike}

\vishaya{maMtarx viBAga karxma}

\begin{artha}
{\bf atheYnamAcAmati tatasxvituvaRreVNayxmf | madhu vAtA QutAyateV madhu kaSxranitx sinadhxvaH | mAdhivxVnaRH sanotxvXVSadhiVH | BUH sAvxhA ||\\
BagoVR deVvasayx dhiVmahi | madhu nakatxmutoVSasoV madhumatApxthiRvaM rajaH | madhu dwyxrasutx naH pitA | BuvaH sAvxhA ||\\
dhiyoV yoV naH parxcoVdayAtf | madhumAnonxV vanasapxtimaRdhumAM asutx sUyaRH | mAdhivxVgARvoV Bavanutx naH | savxH sAvxheVti ||\\
savARM ca sAvitirxVmanAvxha savARshacx madhumatiVrahameVveVdaM savaRM BUyAsaM BUBuRvaH savxH sAvxheVtayxnatxtaH\\
Acamayx pANiV parxkASxlayx jaGaneVnAginxM pArxkishxrAH saMvishati pArxtarAditayxmupatiSaThxteV dishAmeVkapuNaDxriVkamasayxhaM \\
manuSAyxNAmeVkapuNaDxriVkaM BUyAsamiti yatheVtameVtayx jaGaneVnAginxmAsiVnoV vaMshaM japati || 6 ||}
\end{artha}	
