%%%% From 055.tex

\centerline{\vishaya{adhAyxya - 6,   bArxhamxNa - 2}}

%~ \begin{center}
%~ eraDaneV bArxhamxNa
%~ \end{center}

\begin{shl}
yananx saMBAvitaM pUvaRM vasutxpArxdhAnayxheVtutaH | \\
vakutxM tadatarx vakatxvayxM KilakANADxdhikArataH \hfill|| 1 || 
\end{shl}

\begin{artha}
hiMde (jAcnxnakAMDadalilx) yAvudanunx vasutxpArxdhAnayxda 
nimitatxdiMda heVLuvudakekx saMBavisade ididxto adanunx ililx 
KilakAMDa parxkaraNavAdadxriMda heVLabeVkAgide.
\end{artha}

\begin{shl}
sapatxmAvasitAvukatxM mAgaRpArxthaRnamaginxtaH | \\
supatheVti shurxtaM tatarx shurxtAyx mAgaRvisheVSaNamf \hfill|| 2 || 
\end{shl}

\begin{artha}
ELane adhAyxyada koneyalilx aginxyiMda mAgaRpArxthaRne mADiruvudanunx 
heVLide. `agenxV naya supathA rAyeV' eMdu shurxtiyiMdaleV mAgaRvisheVSaNayukatxvAgi sapxSaTxpaTiTxde.
\end{artha}

\vishaya{alilx supathA eMba visheVSaNaveVtakekx? eMdare {\rm --}}

\footnotetext[1]{`saMBavavayxBicArABAyxM sAyxdivxsheVSaNamathaRvatf' eMba 
nAyxyadaMte `niVlamutapxlamf' itAyxdi sathxLadalilx 
rakotxVtapxlavanunx vAyxvaqtitxmADalu niVla eMba visheVSaNavanunx 
koTiTxruvudu. idu sAthaRka, niVlaveMba visheVSaNavu ilalxvAdare 
visheVSayxvu aMdare utapxlavu rakatxvaNaRvAgiyU irabahudeMdu 
athaRvAdiVtu. adakAkxgi I visheVSaNavu sAthaRkavAguvudu. adaraMte 
`supathA' eMdu visheVSaNavu vayxBicAravu parxsakatxvAgalu 
sAthaRkavAgide.}
\begin{shl}
\footnotemark[1]saMBaveV vayxBicAreV ca visheVSaNavisheVSayxyoVH | \\
daqSaTxM visheVSaNaM loVkeV yatheVhApi tatheVkaSxyXtAmf \hfill|| 3 || 
\end{shl}

%% shloka footnote
\begin{artha}
visheVSaNa visheVSayxgaLalilx visheVSaNavu saMBavisuvudidadxrU 
vayxBicAravu (visheVSaNAthaRvu kelavu kaDe ilalxdiruvudU) 
parxsakatxvAgidadxlUlx visheVSaNavanunx parxyoVgisuvudu loVkadalilx 
kaMDide. idu heVgo ililxyU hAgeyeV iruvudeMdu kANuvudu.
\end{artha}

\vishaya{hAgAdare supathA eMbudu vAyxvataRka visheVSaNavAgali? eMdare {\rm --}}

\begin{shl}
supatheVti tatoV yukatxM saMBaveV BUyasAM pathAmf | \\
visheVSaNamatoV vAcAyxH panAthxnaH kamaRheVtavaH \hfill|| 4 || 
\end{shl}

\begin{artha}
aneVka mAgaRgaLu saMBavisuvAga supathA eMba visheVSaNavu yukatxvAgide. 
idariMda kamaRnimitatxvAda mAgaRgaLanunx heVLabeVkAgide. (adakAkxgi I 
bArxhamxNavu baMdide)
\end{artha}

\vishaya{hAgAdare yAva mAgaRgaLu?}

\begin{shl}
dakiSxNoVdagadhoVmAgAR vihitaparxtiSidadhxyoVH | \\
vipAkAH kamaRNoVvARcAyxsatxdevxYrAgayxparxsidadhxyeV \hfill|| 5 || 
\end{shl}

\begin{artha}
vihita kamaRkUkx niSidadhx kamaRkUkx Aguva paripAkagaLenisida 
dakiSxNamAgaR, utatxramAgaR, adhoVgati eMbuvugaLanunx adaralilx 
veYrAgayxvu sididhxsalu avashayx heVLabeVkAgide.
\end{artha}

\vishaya{veYrAgayxveVtakekx? eMdare {\rm --}}

\begin{shl}
nAvirakatxsayx niHsheVSasAMsArikapumathaRtaH | \\
parxvaqtitxmuRkatxyeV tasAmxcuCxrXtAyx yatAnxtatxducayxteV \hfill|| 6 || 
\end{shl}

\begin{artha}
(virakatxnAgadiruva) samasatx saMsAradalilxruva puruSAthaRdalilx 
virakitxyanunx hoMdadavanige mukitxgAgi parxvaqtitxyeV huTuTxvudilalx.
adariMda shurxtiyu parxyatanxpUvaRka A veYrAgayxvanunx heVLuvudu.
\end{artha}

\vishaya{kamaRvipAkadalilx veYrAgayxvu heVge? baMdiVtu? kamaRdiMdaleV sakala puruSAthaRvalalxve? eMdare {\rm --}}

\begin{shl}
shakunxvanitx na kamARNi savaRkAmasamApanamf | \\
niSeVdudhxM vA\s KilAnathARMsatxtaPxlasAyxtiPalugxtaH \hfill|| 7 || 
\end{shl}

\begin{artha}
kamaRgaLu samasatx kAmanegaLanunx pUNaRgoLisalu samathaRvAgilalx. 
athavA elalx anathaRgaLanunx nivArisalu samathaRvAgilalx. avugaLa 
Palavu bahaLa kuSxdarxvAgiruvudariMda samathaRvAgilalx.
\end{artha}

\vishaya{kamaRgaLige mukitxyu Palavalalx {\rm --}}

\begin{shl}
na kamaR kAraNaM muketxVnARginxdaRhajavxrApanutf | \\
kamaRBoVyx janamx niyataM janamx ceVninxvaqRtiH kutaH \hfill|| 8 || 
\end{shl}

\begin{artha}
mukitxge kamaRvu kAraNavAguvudilalx, aginxyu dAha, javxravanunx 
hoVgalADisuvudilalx. kamaRgaLiMda janamxvu kaDADxyavAgi baruvudu. 
janamxvidadxre mukitxyu elilxMda barabeVku?
\end{artha}

\vishaya{AtamxsavxrUpavAda moVkaSx viSayadalilx kamaRvu niSaPxla eMbudakekx shurxtiyeV parxmANa-}

\begin{shl}
na kamaRNA kaniVyasAtx mahatatxvXM cAnatxrAtamxnaH | \\
iti bAhumivoVdadhxqqtayx veVdAnetxYGoVRSaNA kaqtA \hfill|| 9  || 
\end{shl}

\begin{artha}
aMtarAtamxnige kamaRdiMda alapxteyU mahatatxvXvU baruvudilalx; hiVgeMdu veVdAMtagaLu keYyanunx meVlakekxtitx heVLuvaMte GoVSaNe mADiruvavu.
\end{artha}

\vishaya{adeV viSayakekx yAjacnxvalakxyXra samxqqtiyU parxmANavide {\rm --}}

\begin{shl}
na tatarx dakiSxNA yanitx vidayxyeYva tadApayxteV | \\
iti shirxVyAjacnxvalekxyXVna mukatxkaNaThxmudAhaqtamf \hfill || 10 || 
\end{shl}

\begin{artha}
`natatarx dakiSxNAyanitx vidayxyeYva ta dApayxteV' eMdu shirxV yAjacnxvalakxyXru mukatx kaMThadiMda heVLidAdxre.
\end{artha}

\begin{shl}
atoV mukitxM pariVcaCxdiBxrutapxtAtxyXdiviroVdhiniVmf | \\
tayxkAtxvX kamARNayxtheYkAtamxyXjAcnxnaM savARtamxnA\s \s sharxyeVtf \hfill|| 11 || 
\end{shl}

\begin{artha}
adariMda janAmxdigaLige viroVdhiyAda mukitxyanunx apeVkiSxsuvavaru 
kamaRgaLanunx biTuTx EkAtamx savxrUpajAcnxnavanunx savaRvidhadalUlx 
AsharxyisabeVku.
\end{artha}

\vishaya{kamaRvu mukitxya parxtibaMdhakavanunx kaLeyuvudilalx - EkeMdare {\rm --}}

\begin{shl}
tamoVnatxrAyatoV muketxVnARnatxrAyoV\s paroV\s sitx hi | \\
tamoVhatinaR kamaRBoyxV jAcnxnAtAsx vayxcnajxkatavxtaH \hfill|| 12 || 
\end{shl}

\begin{artha}
mukitxge ajAcnxnaveV parxtibaMdhaka. adanunx biTuTx matotxMdu parxtibaMdhakavilalx. ajAcnxnada nivaqtitxyu kamaRgaLiMda Aguvudalalx. jAcnxnadiMdaleV Aguvudu. EkeMdare jAcnxnavu parxkAshakavAdudariMda (adariMdaleV nivaqtitxyu).
\end{artha}

\vishaya{pUvoRVtatxra saMbaMdha viveVcane - `supathA' eMba 
visheVSaNadiMda tiLida aneVka mAgaRgaLanunx nirUpisuvudakAkxgiyU, 
jAcnxnaveV mukitxge kAraNa. kamaRgaLu baMdhanakekx kAraNaveMbudariMda 
adakekx saMbaMdhapaDuvudilalxveMdu heVLuvudakAkxgiyU I bArxhamxNavu 
baMdideyeMdu heVLidAdxyitu. matotxMdu bageyalilx pUvaRgarxMtha 
saMbaMdhavanunx heVLalu aginxhoVtarx parxkaraNadalilx heVLida 
viSayavanunx  anuvAdisutAtxre {\rm --}}

\footnotetext[1]{aginxhoVtarx parxkaraNadalilx niVnu I 
aginxhoVtArxhutigaLa apUvaRveMba sUkaSxmX vasutxgaLa 1. utAkxMtiyanunx 
2. gatiyanunx 3. parxtiSeThxyanunx 4. taqpitxyanunx 5. 
punarAvaqtitxyanunx 6. loVkAMtarakekx edudx niMtiruvudanunx 
tiLidididxVyA? tiLidilalx. adakAkxgi parxtivacanavu `teVvAEteV 
AhutiVhuteV utAkxrXmataH' itAyxdiyAgi baMdide. adaralilx AhutigaLa 
kAyaRvanunx heVLide. aginxhoVtarxdalilx sAyaMkAla pArxtaHkAlagaLalilx 
koDalapxDuva eraDu AhutigaLu ive. avugaLa sUkASxmXMshavAda apUvaRvu 
aMtarikASxdi rUpadalilxruva loVkadalilx tirugi ELuvudariMda Aru 
bageyAgi jagatitxna rUpadalilx pariNAmavAguvudeMdu shurxtiyu apUvaR 
vikAravanunx heVLuvudeMdu tAtapxyaR:}
\begin{shl}
\footnotemark[1]na tu tavxmeVtayoVveVRtethxVtuyxtAkxrXnAtxyXdisavxlakaSxNamf | \\
SaDavxdhaM pariNAmAthaRmaginxhoVtArxhutiVhayoVH \hfill|| 13 || 
\end{shl}

%% shloka footnote
\begin{artha}
Aru bageyAgi (utAkxrXMti modalAda rUpadalilx) aginxhoVtarxda AhutigaLa 
apUvaR pariNAmavAda (jagatatxnunx) niVnu tiLidilalxveMdu 
AkeSxVpayukatxvAda parxshenxyanunx (janakanu yAjacnxvalakxyX 
muniyanunx kuritu) `natu tavxmeVtayoVveRVtathx' eMba vAkayxdiMda 
mADidadxnu.
\end{artha}

\footnotetext[2]{I Ahutiya pariNAmaveV Ahuti lakaSxNavuLaLx kamaRda Pala. 
kataRqvilalxde avananunx avalaMbisade I kamaRvu savxtaMtarxvAgi 
utAkxrXMti modalAda kAyARraMBavanunx mADalAradu. kamaRda kAyARraMBaveV 
kataRqvigAgi, kamaRvu sAdhanavanunx Asharxyiside. alilx A 
parxkaraNadalilx aginxhoVtarxvanunx sutxtisalu Aru bageyanunx 
aginxhoVtarxda kAyaRveMdeV heVLide. parxkaqta I vidAyxparxkaraNadalilx 
adeV kataRqvige baruva PalaveMdu upadeVshisuvudu. kamaRPala 
vijAcnxnavu apeVkiSxtavAgiruvudariMda Aru bageyAgiyU tiLiside. A 
mUlaka paMcAginx videyxyu ililx utatxramAgaRpArxpitxge sAdhanaveMdu 
tiLisuva udedxVsha mADide. elAlx saMsAragatiyanunx upasaMharisidaMte 
Aguvudu. adariMda sakala kamaRvipAkavanunx upasaMharisidaMte Aguvudu. 
adakAkxgi I muMdina KilakAMDada bArxhamxNavu baMdideyeMdu BASayxdaMte 
vivarisida viSayavanunx gamanisabeVku.}
\begin{shl}
\footnotemark[2]itiparxshanxparxtivacasetxV vA itAyxdikaM jagw | \\
loVkaM parxtuyxtithxtaM yAvadaginxhoVtArxdudxtiVhayoVH \hfill|| 14 || 
\end{shl}

%% shloka footnote
\begin{artha}
I riVtiyAgi mADida parxshenxge utatxravAgi `teVvA' itAyxdi 
vacanavanunx hADiruvanu elilxyavarege? eMdare aginxhoVtArxhutigaLu 
paraloVkadalilx tananx Asharxyada utithxtige kAraNavAda 
pariNAmapayaRMtaravAgi.
\end{artha}

\begin{shl}
apUvaRpariNAmoV\s yamaginxhoVtArxKayxkamaRNaH | \\
utAkxrXnAtxyXdigireVhoVkatx AloVkoVtAthxnavAkayxtaH ||  15 ||  
\end{shl}

\vishaya{\mdash athaR bareyabeVku\mdash }

\begin{shl}
AhutoyxVraginxhoVtarxsayx hayxnatxrikASxdiBeVdataH | \\
A loVkoVtAthxnataH shurxtAyx ukAtx\s pUvaRsayx vikirxyA ||  16 ||  
\end{shl}

\vishaya{\mdash athaR bareyabeVku\mdash }

\begin{shl}
tadeVvoVkatxmihA\s \s lamavxyX tadaginxVkaSxNasidadhxyeV | \\
shevxVtakeVturiti garxnathxH para AraBayxteV\s dhunA \hfill|| 17 || 
\end{shl}

\begin{artha}
IvAga adeV jagatatxnenxV hiMde heVLidadxnunx avalaMbisi I KilakAMDadalilx A jagatitxnalilx iDabeVkAda aginxdaqSiTxyu sididhxsuvudakAkxgi `shevxVtakeVtu haRvA AruNiVyaH' itAyxdi muMdina garxMthavanunx AraMBisiruvudu.
\end{artha}

\centerline{\vishaya{baq. 6 - 2 - 1 kaMDike}}

\begin{shl}
shevxVtakeVtuhaR vA AruNeVyaH pacnAcxlAnAM pariSadamAjagAma sa AjagAma jeYvaliM parxvAhaNaM paricArayamANaM tamudiVkASxyXBuyxvAda kumArA3 iti sa BoV3 iti parxtishushArxvAnushiSoTxV\s navxsi piterxVtoyxVmiti hoVvAca || 1 ||
\end{shl}

\vishaya{I AKAyxyikeya tAtapxyaR}

\begin{shl}
pacnAcxginxvidAyx yatenxVna vaqdedhxVnAbArxhamxNAdapi | \\
hitAvx dhanaM ca mAnaM ca labedhxVtuyxkitxH sutxtidhiRyaH \hfill|| 18 || 
\end{shl}

\begin{artha}
bArxhamxNanalalxdavaniMdalU vaqdadhxnAdavanU saha dhana, aBimAna idanenxlAlx biTuTx parxyatanxdiMda paMcAginxvideyxyanunx saMpAdisikoMDaneMdu heVLidudx videyxya sutxtigAgi.
\end{artha}

\vishaya{duyxloVkAdigaLu aginxyalalxdadxriMda avugaLalilx aginxshabadx parxyoVgavu heVge sAdhu? eMdare}

\begin{shl}
pariNAmoV hi pAkeVna pAkashacx na vinA\s ginxnA | \\
dashaRnAtapxriNAmasayx pakAtxHsavaRtarx pAvakaH \hfill|| 19 || 
\end{shl}

\begin{artha}
yAvude oMdu pariNAmavu pAkadiMda Aguvudu. A pAkavU aginxyilalxde 
Aguvudilalx. \footnote[1]{duyxloVka, pajaRnayx muMtAdavugaLa 
adhiVnavAgidudx deVhavu pariNAmagoLuLxvudeMdu `iti tu paMcamAyx mAhutAvApaH puruSa vacasoV Bavanitx' itAyxdi 
shurxtiyalilx I viSayavu tiLiyuvudu. aginxyeV 
paripAkavanunxMTumADuvudeMba niyamadaMte ivugaLalilx aginxshabadx 
parxyoVgavanunx mADiruvudu yukatx.}I deVha rUpavAda pariNAma kaMDiruvudariMda 
pAkamADuvudu elalx sathxLadalUlx aginxyeV.
\end{artha}

\centerline{\vishaya{avataraNike}}

\begin{shl}
samApAtxsheVSavidayxM hi samAvatayxR pitA sutamf | \\
samApAtxsheVSavidoyxV\s siVteyxVvamAhoVtasxsajaR ca \hfill|| 20 || 
\end{shl}

\begin{artha}
taMdeyu samasatxvidAyxBAyxsavanunx pUNaRgoLisi gurukuladiMda 
hiMdakekx baMdiruva magananunx kuritu, niVnu elAlx videyxgaLanunx 
pUNaRgoLisiruveyeMdu heVLidanu, matutx biTuTxkoTaTxnu. 
\end{artha}

\vishaya{taMde gawtamanu putarxnanunx biDalu udedxVshaveVneMdare {\rm --}}

\begin{shl}
nikaSoVpalasaMsethxVSu veVdavitusx pariVkaSxyXtAmf | \\
videyxVyaM yatakxtoV vatasx darxDhimenxV macuCxrXtasayx ca \hfill|| 21 || 
\end{shl}

\begin{artha}
eleY magane? oregalilxna sAthxnadalilxruva veVdajacnxralilx ninanx 
videyxyanunx parxyatanxdiMda pariVkiSxsikoLaLxbeVku. EkeMdare 
naninxMda sharxvaNa mADida shAsatxrXda dADhaRyxkAkxgi.
\end{artha}

\vishaya{itare vidavxtf pariSatutxgaLidadxrU paMcAladeVshada pariSatitxge Itanu Eke baMdidudx?}

\begin{shl}
parxsidAdhx\s tiVva vidavxtAtx pacnAcxlabArxhamxNeVSu hi | \\
tAmeVva pariSadaM tasAmxdAjagAma tavxrAnivxtaH \hfill|| 22 || 
\end{shl}

\begin{artha}
paMcAla deVshada bArxhamxNarugaLalilx vidavxtutx bahaLavAgi ididxteMdu 
parxsidadhxvAgididxtu. adariMda adeV pariSatitxge tavxreyiMda kUDi 
baMdanu.
\end{artha}

\vishaya{putarxnu ililxge baruva udedxVshaveVnu?}

\begin{shl}
pacnAcxlabArxhamxNAcnijxtAvx vidoyxVtakxSeYRkaheVtutaH | \\
rAjAnamapi jeVSAyxmiVtAyxjagAma naqpaM tataH \hfill|| 23 || 
\end{shl}

\begin{artha}
paMcAla bArxhamxNaranunx jayisi videyxya hirimeya nimitatxvAgi 
rAjananunx gelulxtetxVneMdu udedxVshisi rAjanalilxge baMdanu.
\end{artha}

\begin{artha}
\textbf{1{\rm --}2ne kaMDikegaLa sArAMsha-kathAraMBa}\mdash AruNiya maganAda 
shevxVtakeVtu eMbuvanu gurukuladalilx sakalavideyxgaLanunx pUNaRvAgi 
kalitu tananx videyxyanunx saBegaLalilx parxkAshagoLisi yashasasxnunx 
gaLisalu taMdeya AjecnxyaMte pAMcAla deVshada bArxhamxNavidAvxMsara 
pariSatitxge baMdanu. avaranunx gedudx rAjananunx jayisutetxVneMdu 
rAjana saBegU baMdanu. parxvAhaNa eMba jiVvalana putarxneV pAMcAla 
deVshada rAjanAgidadxnu. seVvakariMda seVveyanunx keYgoLuLxtitxruvAga 
I rAjana hatitxra shevxVtakeVtuvu baruvAgaleV Itana vidAyxBimAna 
gavaRviruvudanunx modaleV keVLi tiLididadx rAjanu ivananunx 
viniVtananAnxgi mADabeVkeMdu tiLidu avananunx baMda kUDale avana 
aBipArxyavanunx Uhisi hiVge heVLidanu {\rm --}\\
kumAra, eMdu karedu hedarisidanu. shevxVtakeVtuvU saha koVpisikoMDu 
`BoVH' eMdu kaSxtirxyanige anucitavAgidadxrU hiVge saMboVdhisidanu. 
taMdeyiMda AjAcnxpisalapxTaTxvanAgi baMdiruveyeVnu? eMdu rAjanu 
parxtuyxtatxravitatxnu. idakekx shevxVtakeVtuvu hawdu eMdanu. ninage 
saMshayavidadxre keVLu eMdanu. adakekx rAjanu muMde heVLuvaMte Aru 
parxshenxgaLanunx hAkidanu. yAvudoMdu parxshenxgU utatxravanunx koDalu 
tiLiyade, nanage gotitxlalxveMdeV shevxVtakeVtuvu heVLidanu.
\end{artha}

\begin{shl}
taM jigiVSuM samAyAnatxmunAmxgeVR saMsithxtaM divxjamf | \\
sanAmxgaRparxtipatatxyXthaRM rAjoVvAca savxshAsatxrXtaH \hfill|| 24 || 
\end{shl}

\begin{artha}
jayisalu iciCxsi barutatxliruva mAgaRvanenxV biTiTxruva A 
bArxhamxNananunx kuritu sanAmxgaRda tiLivaLike uMTumADalu tananx 
shAsAtxrXnusAravAgi rAjanu hiVge heVLidanu.
\end{artha}

\begin{shl}
AmanatxrXyAmAsa ca taM kumAra! iti bAlavatf | \\
parxtishushArxva soV\s puyxkotxV BoV ituyxkAtxyX guruM yathA \hfill|| 25 || 
\end{shl}

\begin{artha}
avananunx huDugananunx kareyuvaMte kumAra eMdu rAjanu karedanu. AtanU 
I riVtiyAgi kareyalapxTaTxvanAgi BoVH eMdu guruvanunx heVLidaMte 
rAjananunx kuritu heVLidanu.
\end{artha}

\vishaya{`anushiSoTxVnu' eMba vAkayxvanunx vAyxKAyxnakekx tegedidAdxre {\rm --}}

\begin{shl}
dapoVRtesxVkasamAveVshAnAnxnushiSoTxV\s yamAdarAtf | \\
piterxVti jAtasaMdeVhaH payaRpaqcaCxdatoV naqpaH \hfill|| 26 || 
\end{shl}

\begin{artha}
dapaRvu hececxdudx kANuvudariMda Itanu taMdeyiMda shikiSxsalapxTiTxruvane? athavA ilalxve? eMdu saMshayavu huTiTx rAjanu I parxshenxyanunx mADidanu.
\end{artha}

\begin{shl}
anushiSoTxV\s si kiM pitArx utAhoV neVti BaNayxtAmf | \\
nAnushiSaTxsayx jagati vaqtatxmiVdaqkasxmiVkaSxyXteV \hfill|| 27 || 
\end{shl}

\begin{artha}
taMdeyiMda shikiSxsalapxTaTxvane? athavA ilalxve eMbudanunx heVLabeVku. AjAcnxpisalapxTaTxvanige jagatitxnalilx iMtaha naDavaLike kANutitxlalx.
\end{artha}

\vishaya{OmitAyxdi vAkayxda athaR {\rm --}}

\begin{shl}
bADhaM pitArx\s nushiSoTxV\s simx kiM na pashayxsi majajxyamf | \\
tavxtapxNiDxteVSu saveVRSu paqcaCx mAM yadi shaknakxseV \hfill|| 28 || 
\end{shl}

\begin{artha}
hawdu, taMdeyiMda nAnu vidAyxBAyxsa mADiruvenu nananx jayavanunx niVnu 
kANalilalxveVnu? ninanx elalx paMDitaralUlx keVLu. nananx meVle niVnu 
anumAnapaTiTxdadxre.
\end{artha}

\begin{shl}
EvaM rAjocnxV yathoVkotxVkAtxyX hayxBuyxpeVteV\s nushAsaneV | \\
shevxVtakeVtumathApArxkiSxVtapxcnacx parxshAnxnakxrXmAnanxqqpaH \hfill|| 29 || 
\end{shl}

\begin{artha}
I riVtiyAgi rAjanu heVLida mAtiniMda taMde mADida anushAsanavanunx 
shevxVtakeVtuvu aMgiVkarisalu rAjanu anaMtaraveV I aidu 
parxshenxgaLanunx shevxVtakeVtuvanunx kuritu keVLidanu.
\end{artha}

\section*{baq. 6 - 2 - 2 kaMDike}

\begin{shl}
veVtathx yatheVmAH parxjAH parxyatoyxV viparxtipadayxnAtx 3 iti neVti hoVvAca (3) veVtothxV yatheVmaM loVkaM punarApadayxnAtx 3 iti neVti heYvoVvAca (4) veVtothxV yathAsw loVka EvaM bahuBiH punaH punaH parxyadiBxnaR samUpxyaRtA3 iti neVti heYvoVvAca veVtothxV yatithAyxmAhutAyxM hutAyAmApaH puruSavAcoV BUtAvx samutAthxya vadanitxV3 iti neVti heYvoVvAca (5) veVtothxV deVvayAnasayx vA pathaH parxtipadaM pitaqyANasayx vA yatakxqqtAvx deVvayAnaM vA panAthxnaM parxtipadayxnetxV pitaqyANaM vApi hi na QuSeVvaRcaH shurxtaM devxV saqtiV ashaqNavaM pitaqRNAmahaM deVvAnAmuta matAyxRnAM tABAyxmidaM vishavxmeVjatasxmeVti yadanatxrA pitaraM mAtaraM ceVti nAhamata Ekacnacxna veVdeVti hoVvAca || 2 ||
\end{shl}

\vishaya{modalaneV parxshenxya vAyxKAyxna {\rm --}}

\begin{shl}
anushiSoTxV\s si ceVdUbxrXhi tuleyxV\s pi maraNeV parxjAH | \\
yathA viparxtipadayxneVtx BinanxvatamxRparxBeVdataH \hfill|| 30 || 
\end{shl}

\begin{artha}
elAlx pArxNigaLigU maraNavu samAnavAgidadxrU mAnavaru sAyuvavaru 
Binanx mAgaRgaLa parxBeVdadiMda heVge? beVrebeVreyAgi hoVguvaro (hAge 
niVnu tiLidididxVyA?)
\end{artha}

\vishaya{adaneVnx punaH vivarisuvudu {\rm --}}

\begin{shl}
yeVna kamaRvisheVSeVNa samAnAyAM maqtw parxjAH | \\
anAyx aneyxVna saMyAnitx yathA\s neyxVnAparAsatxthA \hfill|| 31 || 
\end{shl}

\begin{artha}
janaralilx maraNavu samAnavAgidadxrU obabxru beVre mAgaRdiMdalU 
matotxbabxru matotxMdu mAgaRdiMdalU yAva kamaRvisheVSadiMda janaru 
heVge? hoVguvaro, hAge tiLidididxVyA?
\end{artha}

\vishaya{I parxshenxge shevxVtakeVtuvina utatxra {\rm --}}

\begin{shl}
tavxyoVkatxM na viveVdAhaM nAnushiSiTxrihAsitx meV | \\
veVtethxVha tA yathA BUya vataRnetxV parxjA iti  \hfill|| 32 || 
\end{shl}

\begin{artha}
niVnu heVLidadxnunx nAnu tiLidilalx. nanage I viSayadalilx 
shikaSxNavilalx. (eraDane parxshenx) satutxhoVda janaru 
(loVkAMtarakekx hoVdavaru) heVge ililxge punaH baruvaro? hAge niVnu 
tiLidididxVyA?
\end{artha}

\vishaya{idara vivaraNe}

\begin{shl}
yathA yeVna maqtAH satoyxV heVtunA\s neVna ca parxjAH | \\
taM veVtathx sivxnanx veVtuyxkotxV neVti hoVvAca taM punaH \hfill|| 33 || 
\end{shl}

\begin{artha}
janaru maqtapaTaTxvaru yAvudariMda hiMdakekx ililxge baruvaro adanunx 
niVnu tiLidididxVyA? athavA ilalxveMdu parxshinxsidAga shevxVtakeVtuvu 
ilalx, tiLidilalxveMdu punaH utatxrisidanu.
\end{artha}

\vishaya{2ne parxshenx - idara vAyxKAyxna {\rm --}}

\begin{shl}
parxyadiBxrasakaqdUBxteYmaRhadiBxbaRhuBiH sadA | \\
neYvAsw pUyaRteV loVkoV yathA veVtathx tathA\s tarx kimf \hfill|| 34 || 
\end{shl}

\begin{artha}
aneVkAvatiR janaru huTiTx bahumaMdi doDaDxvarAgidadxrU avariMda 
yAvAgalU I paraloVkavu BatiRyAgalilalx. idu heVgeyeMbudanunx niVnu 
tiLidididxVyA?
\end{artha}

\vishaya{utatxra}

\begin{shl}
neVti hoVvAca paqSaTxH sanArxjA paparxcaCx taM punaH | \\
hutAyAmAhutw veVtathx yatithAyxM puruSABidhAH \hfill|| 35 || 
\end{shl}

\begin{artha}
parxshinxsida naMtara Atanu nAnu tiLidilalxveMdu heVLidanu. rAjanu 
avananunx kuritu matetx parxshinxsidanu - yAva saMKeyxya (eSaTxneya) 
Ahutiyalilx hoVma mADidadxralilx jalavu puruSa shabadxvuLaLxdAdxgi 
Aguvudu?
\end{artha}

\vishaya{punaH I parxshenxyanunx vivarisuvudu {\rm --}}

\begin{shl}
Apa Eva samutAthxya puruSAkaqtayoV hutAH | \\
parxvadanitx yathA veVtathx tathA\s \s shu parxtipadayxtAmf \hfill|| 36 || 
\end{shl}

\begin{artha}
hoVma mADidadx jalaveV meVlakekx edudx puruSAkAravuLaLxdAdxguvudu eMdu heVge heVLuvaro adanunx tiLidididxVyA? hAgidadxre shiVGarxvAgi parxtipAdisu.
\end{artha}

\begin{shl}
rAjAnaM neVti hoVvAca nAhaM veVdimx tavxyoVditamf | \\
payaRpaqcaCxdatoV rAjA shAnatxdapaRM divxjaM punaH \hfill|| 37 || 
\end{shl}

\begin{shl}
pathasatxvXM deVvayAnasayx pitaqyANasayx vA\s cnajxsA | \\
veVtathx parxtipadaM kiMvA na veVtisxVtayxBidhiVyatAmf \hfill|| 38 || 
\end{shl}

\begin{artha}
adakekx shevxVtakeVtuvu rAjananunx kuritu niVnu heVLidadxnunx nAnu 
aritilalx eMdanu. anaMtara rAjanu dapaRvu iLiduhoVgidadx I 
bArxhamxNananunx kuritu matetx parxshenx mADidanu. 2. niVnu deVvayAna 
athavA pitaqyANada parxtipatatxnunx tiLidididxVyA? athavA tiLidilalxvo 
Enu heVLu eMdu parxshinxsidanu.
\end{artha}

\vishaya{parxtipatf eMdare {\rm --}}

\footnotetext[1]{yAva kamaRvanunx aMdare aginxhoVtarx muMtAda 
vishiSaTxtama kamaRvanunx mADi, athavA upAsaneyanonx mADi, deVvayAna 
mAgaRvanonx pitaqyANa mAgaRvanonx I janaru hoMduvaro A kamaRveV 
parxtipatf eMdu heVLalapxDuvudu. parxtipadayxte = pArxpayxteV yayA 
aginxhoVtArxdi kirxyayA upAsanA kirxyayA sA parxtipatf eMba 
vuyxtapxtitxyiMda ililx mAgaRdavxyavanunx hoMdisuva kamaR, athavA 
upAsaneyeMdathaR.}
\begin{shl}
parxtipadavxcanasAyxthaRM \footnotemark[1]yatakxqqtevxVtAyxdarAcuCxrXtiH | \\
vAyxcaSeTxV deVvayAnAdiparxtipatwtx \footnotemark[2]kirxyeYva sA \hfill|| 39 || 
\end{shl}
\footnotetext[2]{}

%% shloka footnote
\begin{artha}
parxtipatetxMba padakekx shurxtiyu `yatakxqqtAvx' eMbudAgi AdaradiMda 
vAyxKAyxnisuvudu. adeVneMdare - deVvayAna pitaqyANa mAgaRvanunx 
paDeyalu beVkAda kamaRveV \textbf{parxtipatf} eMdu heVLalapxDuvudu.
\end{artha}

\vishaya{mAgaRdavxyavU pArxmANikavenunxtAtxre {\rm --}}

\begin{shl}
sAvxBUyxha iti mA shaknikxVyaRtoV mAgaRdavxyeV\s pi naH | \\
QuSeVmaRnatxrXsayx sharxvaNamasitx tacacx viBAvayxteV \hfill|| 40 || 
\end{shl}

\begin{artha}
mAgaRgaLalilx eraDaralUlx tananx UheyeMdu ililx niVnu shaMkisabeVDa. 
namage I athaRvanunx parxkAshapaDisuva maMtarxvu shurxtavAgide. 
adanunx parxmANavAgi BAvisidedxVve.
\end{artha}

\vishaya{QugevxVda maMtarx}

\begin{shl}
``devxV saqtiV ashaqNavaM pitqRNAmf............. mAtaraMca"
\end{shl}

\vishaya{I maMtarxda vAyxKAyxna {\rm --}}

\begin{shl}
devxV saqtiV ashaqNavaM sAkASxtasxMbanidhxnwyx divwkasAmf | \\
pitaqRNAM cApi matAyxRnAM mArwgx tAvadhikArataH \hfill|| 41 || 
\end{shl}

\begin{shl}
tABAyxM savaRmidaM gacaCxdayxthAkamaR yathAshurxtamf | \\
sameVti madheyxV BoVgAya roVdasoyxVH kamaRNoV jagatf \hfill|| 42 || 
\end{shl}

\begin{artha}
deVvategaLigU pitaqgaLigU neVrA saMbaMdhisida eraDu mAgaRgaLu iveyeMdu 
keVLidedxVne. avugaLu manuSayxnige tamamx (kamaR, matutx upAsanegaLa) 
adhikArAnusAravAgi laBayxvAda mAgaRgaLu. I eraDu mAgaRgaLiMda I 
jagatetxlalxvU vAyxpatxvAgi naDeyutitxruvudu. tananx kamaRkekx 
yoVgayxvAgiyU upAsanegU yoVgayxvAgiyU iruvaMte dAyxvApaqthivigaLa 
naDuve kamaRda BoVgakAkxgi A mAgaRgaLoMdige seVriruvudu.
\end{artha}

\begin{shl}
tavxdukAtxtapxrXshanxgaNatoV na veVdemxyXVkamapiVritamf | \\
parxshanxM mA mAmataH pArxkiSxVrituyxkAtxvX\s vAkishxrA hayxBUtf \hfill|| 43 || 
\end{shl}

\begin{artha}
rAjane? niVnu keVLida parxshenxgaLa samudAyadalilx oMdanUnx nAnu 
tiLidilalx. adariMda niVnu idakUkx meVle parxshenxyanunx nananxlilx 
keVLabeVDa. eMdu heVLi shevxVtakeVtuvu taleyanunx tagigxsikoMDu 
niMtanu.
\end{artha}

\vishaya{`adheYnaMvasayxtAyx'... itAyxdi vacanada athaR}

\begin{shl}
nidhURtAsheVSakaluSaM shAnatxdapaRM samiVkaSxyX tatf | \\
vasatAyx\s \s manatxrXyAMcakarx uSayxtAmiti pAthiRvaH \hfill|| 44 || 
\end{shl}

\begin{artha}
samasatx kalamxSavU hoVgi, dapaRvU shAMtavAgiruva A 
shevxVtakeVtuvanunx rAjanu tananxlilx vAsamADalu AhAvxnisidanu. 
ililxyeV vAsa mADu eMdanu.
\end{artha}

\vishaya{baq. 6 - 2 - 3 kaMDike}

\begin{shl}
atheYnaM vasatoyxVpamanatxrXyAcnacxkerxV\s nAdaqtayx vasatiM kumAraH parxdudArxva sa AjagAma pitaraM taM hoVvAceVti vAva kila noV BavAnupxrAnushiSATxnavoVca iti kathaM sumeVdha iti pacnacx mA parxshAnxnArxjanayxbanudhxrapArxkiSxVtatxtoV neYkacnacxna veVdeVti katameV ta itiVma iti ha parxtiVkAnuyxdAjahAra ||3||
\end{shl}

\vishaya{`anAdaqtayx\mdash itivAva' idara vAyxKAyxna {\rm --}}

\begin{shl}
hirxVtoV roVSAcacx tadAvxkayxM vasatayxthaRmudiVritamf | \\
anAdaqtayx parxdudArxva yatArx\s \s setxV gwtamaH pitA \hfill|| 45 || 
\end{shl}

\begin{artha}
nAcikepaTaTx shevxVtakeVtuvu koVpadiMdalU vAsamADalu heVLida mAtanunx 
tirasakxrisi elilx taMde gawtamanu iruvano alilxge ODi hoVdanu.
\end{artha}

\begin{shl}
pArxpAyxtha pitaraM roVSAtAsxBayxsUyaM nirAha saH | \\
iti \footnotemark[1]vAveVti vacanaM pUvoVRtatxraviroVdhataH \hfill|| 46 || 
\end{shl}
\footnotetext[1]{I shurxtiya aBipArxyavanunx koDalu horaTa I eraDu 
vAtiRkagaLa Ashayavidu - rAjana AmaMtarxNavanunx tirasakxrisi baMdu 
taMdeya hatitxra hiVge heVLidanu. `niVvu hiMde gurukuladiMda hiMdakekx 
baMdu nananxnunx kuritu' elAlx videyxgaLalUlx sushikiSxtanAgididxVye 
eMdu heVLididxVrA, alalxve? idu Iga nananx BAgakekx suLALxgide. 
EkeMdare? nAnu tamimxMda pUNaRvAgi elAlx videyxgaLalUlx shikaSxNa 
hoMdidedxVneMdu naMbi rAjanidadxlilxge hoVde. Atanu Aru 
parxshenxgaLanunx hAkidanu. nAnu yAvudoMdakUkx utatxra koDalu 
sAdhayxvAgalilalx. tamimxMda sakala vidAyxshikaSxNavu Agidadxre rAjana 
parxshenxge utatxrakoDalu ajAcnxnaveVke? avana parxshenxyu tiLiyade 
idadxkAraNa, nimamx mAtu suLALxyitu.}

%% shloka footnote
\begin{artha}
anaMtara koVpadiMda taMdeyanunx hoVgi kuritu asUyeyiMda kUDiruvaMte 
(beYgaLiMda kUDiruvaMte) Atanu heVLidanu. samAvataRne kAladalilx 
`niVnu elAlx videyxgaLalUlx sushikiSxtanAgiruve' eMdu hiMdumuMdina 
mAtugaLige viroVdhaviruvaMte niVvu heVLididxVralalxve? eMdu heVLidanu.
\end{artha}

\vishaya{iti eMbudara athaR}

\footnotetext[2]{pitaqshikaSxNavu suLuLx eMdu rAjana parxshanx vAkayxdiMda 
nanage tiLiyitu. idu takaRdiMda, rAjana parxshenxyeV takaR heVtu, 
nanage A shikaSxNavAgilalxveMbudu modalu parxtayxkaSxvAgiralilalx, Iga 
tiLiyitu. adariMdaleV `anumAnAdidhx tadagxtiH' eMdu heVLiruvadu.}
\begin{shl}
itiVtuyxkatxparAmashoVR \footnotemark[3]vAkoVvAkayxM naqpeVritamf | \\
\footnotemark[2]apArxtayxkASxyXtikxleVtuyxkitxranumAnAdidhx tadagxtiH \hfill|| 47 || 
\end{shl}
\footnotetext[3]{sholxVkadalilx vAkoVvAkayx eMbudakekx 
takaRshAsatxrXveMdathaRvidadxrU parxkaqta adakekx samAnavAda rAjanu 
heVLida parxshanxrAshiyanunx garxhisabeVku. takaRshAsatxrXveMbathaRvu 
parxkaqtakekx hoMdadu. I parxshenxvAkayxgaLe heVtuvAgidudx adariMda 
pitaqvAkayxvu niVnu elAlx videyxgaLanunx kalitu 
pUNaRgoLisididxyeMbuva mAtu adakekx virudadhxvAgideyeMbudu samxraNege 
baruvudeMdu Ashaya. rAjanu keVLida parxshenxgaLu tiLiyade 
ajAcnxnavidadxdudx kaMDubaMdadadxriMda taMde heVLida mAtu asatayx, 
suLeLxMdu tiLidu baMditeMdu heVLidudx sari.}

%% shloka footnote
\begin{artha}
`iti' eMbudu hiMde heVLidadxnunx parAmashiRsuvudakekx rAjanu heVLida 
parxshanx samudAyaveMba heVtuvAkayx. kila eMba mAtu taMdeya 
shikaSxNavu (pUNaRvAgi) parxtayxkaSxvAgalilalxvAdadxriMda idanunx
sUcisuvudakAkxgi, takaRdiMdaleV alalxde adu tiLidiruvudu.
\end{artha}

\begin{shl}
naqpoVkatxyXBiBavAlilxknAgxdavxcnicxtoV\s simxVti liknagxyXteV | \\
yathAvadanushiSaTxsayx nABiBUtiyaRtoV\s nayxtaH \hfill|| 48 || 
\end{shl}

\begin{artha}
rAjanu heVLida vacanagaLiMda Ada tirasAkxraveMba heVtuviniMda 
nAnu taMdeyiMda moVsahoVdeneMbudu UhisalapxTiTxde. EkeMdare? 
nijavAgiyeV shikiSxtanAgiruvavanige matotxbabxriMda tirasAkxravu 
AgalAradaSeTx.
\end{artha}

\vishaya{adu heVge taMdeyiMda vaMcitanAde? eMdare}

\begin{shl}
taM mAmananushiSeyxYva kimituyxkatxM tavxyA purA | \\
anushiSoTxV\s si puterxVti vacnicxtoV\s simxVtayxtoV matiH \hfill|| 49 || 
\end{shl}

\begin{artha}
A nananxnunx pUNaRvAgi vidAyxBAyxsa mADisade niVvu hiMde`putarxne? 
niVnu vidAyxBAyxsa mADidavanAgididxV', eMdu heVLidadxriMda nimimxMda 
Iga moVsa hoVdeneMdu tiLivaLike baMdide.
\end{artha}

\begin{shl}
kathaM tavxM nAnushiSoTxV\s si bUrxhi tatAkxraNaM mama | \\
pacnacx mAmitayxtoV\s voVcadayxthA hayxnanushAsanamf \hfill|| 50 || 
\end{shl}

\begin{artha}
niVnu heVge shikiSxtanAgilalx? nanage adara kAraNavanunx heVLu eMdu 
taMdeyu heVLutitxralu `paMca mAparxshAnxnf rAjanayx banudhx 
rapArxkiSxVtf' eMdu rAjanu nananxnunx aidu parxshenxgaLanunx 
keVLidanu. (adanunx tiLidilalxveMdanu).
\end{artha}

\begin{shl}
parxshAnxsetxV katameV vatasx yAMsatxvXM na jAcnxtavAnasi | \\
parxshanxparxtiVkAnavadatapxqqSaTxH pitArx samAsataH \hfill|| 51 || 
\end{shl}

\begin{artha}
magu! A parxshenxgaLu yAvuvu? yAvudanunx niVnu tiLiyade hoVdeyo? 
adanunx heVLu eMdu taMdeyu punaH keVLalu saMkeSxVpavAgi parxshenxgaLa 
EkadeVshagaLanunx heVLidanu.
\end{artha}

\vishaya{baq. 6 - 2 - 4 kaMDike}

\footnotetext[1]{I kaMDikeya sArAMsha - taMdeyAda AruNiyu kupitanAda 
magananunx samAdhAnapaDisalu I riVti heVLidanu. eleY vatasx niVnu 
namamxnunx hAge tiLiyabeVku nAnu tiLida vijAcnxnavelalxvanunx ninage 
heVLikoTiTxruveneMdeV tiLidiko. ninagiMtalU pirxVtipAtarxnu nanage 
beVre yAridAdxre? yArigAgi nAnu jAcnxnavanunx bacicxDabeVku? nAnU kUDa 
rAjanu keVLida parxshenxyanunx tiLidavanalalx. adariMda bA hoVgoVNa. 
ililxMda hoVgi rAjanalelxV barxhamxcayaR varxtadalelxV idudx 
jAcnxnakAkxgi vAsisoVNa eMdu taMdeyu shevxVtakeVtuvige heVLidanu {\rm --}}
\begin{shl}
\footnotemark[1]sa hoVvAca tathA nasatxvXM tAta jAniVthA yathA yadahaM kicnacx veVda savaRmahaM tatutxBayxmavoVcaM perxVhi tu tatarx parxtiVtayx barxhamxcayaRM vatAsxyXva iti BavAneVva gacaCxtivxti sa AjagAma gwtamoV yatarx parxvAhaNasayx jeYvaleVrAsa tasAmx AsanamAhaqtoyxVdakamAhArayAcnacxkArAtha hAsAmx aGayxRM cakAra taM hoVvAca varaM BagavateV gwtamAya dadamx iti ||4||
\end{shl}

\begin{shl}
bArxhamxNajAcnxnatoV\s nayxtarx vidAyxM paparxcaCx BUmipaH | \\
na hi bArxhamxNavijAcnxneV kiMcidasitx tavxyA\s gatamf \hfill|| 52 || 
\end{shl}

\begin{artha}
rAjanu bArxhamxNanu tiLida videyxge beVreyeV Ada viSayadalilx 
videyxyanunx parxshinxsidanu. bArxhamxNanige tiLida vijAcnxnaveV 
Agidadxre nininxMda tiLiyada aMshavu oMdU irutitxralilalx.
\end{artha}

\begin{shl}
iteyxVtadadhxqqdayeV kaqtAvx tathA na iti soV\s vadatf | \\
mA shaknikxSAThxsatxtoV mAM tavxM noVkatxM savaRM mameVti hi \hfill|| 53 || 
\end{shl} 

\begin{artha}
I riVtiyAgi aBipArxyavanunx haqdayadalilxTuTxkoMDu saH = A taMdeyu A 
parxkAradalilx namamxnunx hAge tiLiyabeVku eMdu maganige heVLidanu. 
EneMdare? niVnu nanage elalxvanunx heVLikoDalilalxveMdu nananx meVle 
shaMkepaDabeVDa eMdu. (nanage tiLididedxlAlx videyxyanunx 
heVLikoTiTxruveneMdeV tiLi).
\end{artha}

\begin{shl}
perxVhi tatarx gamiSAyxvasatxdivxdAyxlabadhxsidadhx\\
barxhamxcayaRM ca vatAsxyXva AvAM tatarx naqpeV gatw \hfill|| 54 || 
\end{shl}

\begin{artha}
bA, alilxge hoVgoVNa. A videyxya lABavu Agalu barxhamxcayaRdalilx 
nAvibabxrU rAjanalelxV vAsamADoVNa.
\end{artha}

\vishaya{BavAneVva itAyxdi shurxtiya athaR}

\begin{shl}
yAtu tatarx BavAneVva nAhaM taM ganutxmutasxheV | \\
ituyxkatxH sUnunA hirxVtaH savxyameVva jagAma tamf \hfill|| 55 || 
\end{shl}

\begin{shl}
sasaMBarxmaH sa coVtAthxya tasAmx AsanamAharatf | \\
apa AhArayAMcakarx aGayxRpAdAyxthaRsidadhxyeV \hfill|| 56 || 
\end{shl}

\begin{shl}
satakxqqtayx ca yathAshAsarxM rAjA\s tha tamuvAca ha | \\
varaM kAmaM parxyacACxmoV yaH kAmoV vAcniCxtasatxvXyA \hfill|| 57 || 
\end{shl}

\begin{shl}
QuSirAhAtha rAjAnaM kAmitAthaRsayx sidadhxyeV | \\
parxtijAcnxtoV varasAtxvadaBxvatA\s pArxthiRtoV\s pi sanf \hfill|| 58 || 
\end{shl}

\begin{artha}
niVneV alilxge hoVgu. nAnu avana hatitxra hoVgalu iciCxsuvudilalx eMdu 
putarxnu heVLida meVle nAcikepaTuTx tAneV A rAjana hatitxra hoVdanu. 
rAjanU kUDa saMBarxmadiMda toreyiMda meVlakekxdudx Itanige Asanavanunx 
koTaTxnu. niVranunx tarisikoTaTxnu, EtakAkxgi? aGaRyx, 
pAdAyxdigaLanunx apiRsalu. anaMtara shAsatxrXdaMte satAkxravanunx mADi 
avanu A AruNiyanunx kuritu heVLidanu. niVvu yAva varavanunx 
bayasuviro, adanunx nAvu yatheVSaTxvAgi koDuvevu eMdu heVLalu rAjanige 
bArxhamxNanu hiVge utatxrisidanu - niVnu iSATxthaRsididhxgAgi 
naninxMda pArxthiRsalapxDadidadxrU varavanunx (beVDidadxnunx) 
koDutetxVneMdu parxtijecnx mADiruve.
\end{artha}

\vishaya{baq. a.6, bArx. 1, kaMDike 2}

\begin{shl}
sa hoVvAca parxtijAcnxtoV ma ESa varoV yAM tu kumArasAyxnetxV vAcamaBASathAsAtxM meV bUrxhiVti ||5||
\end{shl}

\begin{shl}
sa hoVvAca deYveVSu veY gwtama tadavxreVSu mAnuSANAM bUrxhiVti ||6||
\end{shl}

\begin{shl}
asutx satayxparxtijocnxV\s tarx parxtijAcnxtaM tavxyeVha yatf | \\
deVhi parxshAnxtimxkAM vAcaM yAM matUsxnoVraBASathAH \hfill|| 59 || 
\end{shl}

\begin{shl}
rAjA\s pi tamuvAcAtha deYveVSivxti paraM vacaH | \\
deYveVSavxyaM varaH sidodhxV mAnupANAM varaM vaqNu \hfill|| 60 || 
\end{shl}

\begin{shl}
na hi mAnuSatoV deYvaH pArxthaRniVyoV vijAnatA | \\
mAnuSasUtxcitoV dAtumAdAtuM mAnuSAdavxraH \hfill|| 61 || 
\end{shl}

\begin{artha}
adakekx badalAgi bArxhamxNanu utatxravitatxnu - niVnu 
satayxparxtijecnxyuLaLxvanAgu. niVne yAvudanunx modale parxtijecnx 
mADididxVyo, adanenx koDu. nananx maganige parxshanxrUpavAda yAva 
mAtanunx ADideyo, adanenx koDu, (adeV nananx vara). rAjanu 
bArxhamxNanige heVLidanu, `niVnu yAvudanunx beVDuveyo adu deYva 
varagaLalilx seVride, mAnuSa varagaLalilx oMdu varavanunx keVLu' 
eMdanu. manuSayxriMda deYva varavanunx tiLidavaru pArxthiRsuvudalalx, 
mAnuSa varavanenx koDuvudu ucita. manuSayxriMda sivxVkarisuvudakUkx 
adeV yoVgayxvAda vara - eMdanu.
\end{artha}

\vishaya{baq. a.6, bArx. 2, kaMDike 7}

\begin{shl}
sa hoVvAca vijAcnxyateV hAsitx hiraNayxsAyxpAtatxM goVashAvxnAM dAsiVnAM parxvArANAM paridAnasayx mA noV BavAnabxhoVrananatxsAyxpayaRnatxsAyxBayxvadAnoyxV BUditi sa veY gwtama tiVtheVRneVcACxsA ituyxpeYmayxhaM Bavanatxmiti vAcA ha semxYva pUvaR upayanitx sa hoVpAyanakiVtoyxVRvAsa || 7 ||
\end{shl}

\begin{shl}
mamApayxsetxyXVva tatasxvaRM yadayxdidxtasxsi mAnuSamf | \\
vijAcnxyateV mayeYvA\s \s dw BavatA\s pi parxmAnatxrAtf \hfill|| 62 || 
\end{shl}

\begin{artha}
gawtamanu nanagU adelAlx idedxV ide. yAva yAva mAnuSa vasutxgaLanunx 
koDalu iciCxsideyo, nanage modaleV gotitxde, ninagU beVre parxmANadiMda 
tiLide ide.
\end{artha}

\begin{shl}
na ca tatApxrXthaRniVyaM meV BUri yadivxdayxteV mama | \\
tasAmxdedxYvoV varoV mahayxM diVyatAM nAsatxyXsw mama \hfill|| 63 || 
\end{shl}

\begin{artha}
nanage yAvudu hecAcxgi ideyo, adanunx nAnu beVDatakakxdadxlalx. adariMda deYva varavaneVnx nanagAgi koDabeVkAdadudx, idu nanage iruvudilalx.
\end{artha}

\vishaya{adanunx koDuvudakekx Aguvudilalx nanage loVBa Ase ide - eMdare {\rm --}}

\begin{shl}
bahoVrananAtxpayaRnatxdeYvavitatxsayx loVBataH | \\
mA BUraBayxvadAnayxsatxvXM dAtA BUtevxVha naH parxti \hfill|| 64 || 
\end{shl}

\begin{artha}
bahaLavAgiyU anaMtavAgiyU koneye ilalxdaSuTx irada deYvavitatxda loVBadiMda niVnu nanenxduru ililx dAtaqvAgidudx dAniyAgade irabeVDa. (koDade irabeVDa eMdu gawtamanu heVLidanu)
\end{artha}

\vishaya{`saveYgawtama' itAyxdi vAkayxda tAtapxyaR {\rm --}}

\begin{shl}
deYvaM varaM na saMdAtuM parxtAyxKAyxtuM ca taM divxjamf | \\
iti duHKitavxmApananxsitxVtheVRneVceCxVtuyxvAca tamf \hfill|| 65 || 
\end{shl}

\begin{artha}
deYva varavanunx koDuvudakUkx tirasakxrisuvudakUkx rAjanu 
shakatxnAgalilalx. A bArxhamxNananunx tirasakxrisuvudakUkx Agade 
duHKakekx oLagAgi avananunx kuritu `\footnote[1]{tiVthaR eMdare 
nAyxya, shAsatxrXvihitavAda mAgaRveMdathaR.}tiVtheRVneVcACx' eMdu heVLidanu.
\end{artha}

\vishaya{hAgAdare rAjanu bArxhamxNananunx tirasakxrisuvavanaMte Eke heVLidudx? eMdare {\rm --}}

\footnotetext[2]{`nAtiVtheVRna SaDf bAheyxVna AcAyoVR\s SiRtoV\s pi tasemxY nadadAyxtf' eMdu manAvxdi vAkayxveMdu AnaMdagirigaLu ililx udAharisidAdxre. shiSAyxdi Aru janaralilx obabxru tiVthaR (satApxtarx) venisuvaru. avariMda pArxthiRsalapxTaTx guruvu shiSayxvaqtitxyiMda tananx hatitxra baMdavanige avashayxvAgi videyxyanunx dAna mADabeVku. Aru janaralilx seVradiruvavane atiVthaR, avaniMda pArxthiRsalapxTaTxrU avanige videyxyanunx dAna mADabAradeMdu heVLide. adanunx anusarisi rAjanU kUDa tiVthaRmAgaRdiMdale naninxMda niVnu videyxyanunx paDe eMdu rAjanu ililx heVLidanu.}
\begin{shl}
tiVtheVRna vidAyx deVyeVti \footnotemark[2]nAtiVtheVRneVti cA\s \s gamaH | \\
yatoV\s tasitxVthaRsaqteyxYva matotxV vidAyxM tavxmApunxhi \hfill|| 66 || 
\end{shl}

%% shloka footnote
\begin{artha}
nAyxyadiMda videyxyanunx dAna mADabeVku. nAyxyavilalxde dAna mADabAradeMba Agamavu iruvudu. adariMda nAyxyamAgaRdiMdaleV naninxMda niVnu videyxyanunx paDe eMdanu.
\end{artha}

\vishaya{`upeYmiVti' eMba vAkAyxthaR {\rm --}}

\begin{shl}
shAsArxthaRM sAmxritaH soV\s tha rAjocnxVpeYmiVtayxthoVcivAnf | \\
\footnotemark[3]vAceYva hayxvarAnUpxvaR upayanitx yatasatxtaH \hfill|| 67 || 
\end{shl}
\footnotetext[3]{bArxhamxNanu videyxgAgi rAjana hatitxra barabahude? eMdu 
keVLidare utatxravidu. `vAceYva hayxvarAnf pUveRV' eMdu. udAdxlaka 
modalAda bArxhamxNareV ashavxpatiyeMba rAjanalilx videyxgoVsakxra 
AsharxyaNa mADiruvuduMTu. adariMda modalu huTiTxda bArxhamxNaru 
modaligarAdarU ApatAkxladalilx anaMtara baMda kaSxtirxyaranunx 
vidAyxjaRnegAgi shiSayxvaqtitxyiMda Asharxyisabahudu. hAgeye 
kaSxtirxyarU Apatitxnalilx veYshayxranunx Asharxyisabahudu. Adare 
guruvina pAdoVpasapaRNa guru shushUrxSAdigaLiMda Asharxyisuva 
padadhxtiyilalx. keVvala `nAnu videyxgAgi tamamxnunx avalaMbisi 
baMdidedxVne' eMdu vAcA heVLi upasatitx mADabeVkAdudu hiVgeMdu 
`sa veY gwtama tiVtheVRnecACxdA itUyxpeYmayxhaM Bavanatxmiti vAcA ha semxYva pUravxupayanitx sahoVpAyanakiVtAyxR upAsa' eMba shurxtiya athaRvu, matutx I 67, 68ne vAtiRkagaLa tAtapxyaR.}

\begin{shl}
sa hoVpAyanakiVteyxYRva barxhamxcayaRmuvAsa ha | \\
upeYmiVti hi saMkiVteVRnARnAyxtikxMcicacxkAra saH \hfill|| 68 ||  
\end{shl}

%% shloka footnote
\begin{artha}
rAjanu shAsAtxrXthaRvanunx nenapige taMdukoTaTxmeVle `nAnu shiSayxnAgi ninanxnunx paDeyuve eMdu' bAyi mAtinaleVlx bArxhamxNanu heVLidanu. EkeMdare modalina bArxhamxNaru kaniSaThxranunx (kaSxtirxyAdi AcAyaRpuruSaranunx) vAcA videyxgAgi hoMduvaru aSeTx. (pAdoVpasapaRNa shushUrxSAdigaLiMdalU alalx)

Atanu upAyana kiVtaRnadiMdale barxhamxcayaRvanunx avalaMbisi idadxnu. adariMda `upeYmi' eMdu heVLuvudanunx biTuTx avanu beVre oMdanunx mADalilalx.
\end{artha}

\begin{shl}
sAparAdhaM savxmAtAmxnaM rAjA pariharananxtha | \\
kaSxmayAmAsa tamaqSiM tathA na itivAkayxtaH \hfill|| 69 || 
\end{shl}

\begin{artha}
rAjanu AnaMtara tananxnunx aparAdhiyeMdu tiLidu adanunx pariharisalu yatinxsi A QuSiyanunx `tathAnaH' eMba vAkayxdiMda kaSxmisuvaMte mADidanu.
\end{artha}

\vishaya{A vAkayxda vAyxKAyxna {\rm --}}

\vishaya{baq. a.6, bArx. 2, kaMDike 8}

\begin{shl}
sa hoVvAca tathA nasatxvXM gwtama mAparAdhAsatxva ca pitAmahA yatheVyaM videyxVtaH pUvaRM na kasimxMshacxna bArxhamxNa uvAsa tAM tavxhaM tuBayxM vakASxyXmi koV hi tevxYvaM burxvanatxmahaRti parxtAyxKAyxtumiti || 8 ||
\end{shl}

\begin{shl}
mAnoV\s parAdhinoV maMsAthxsatxva pUveVR pitAmahAH | \\
nAmanayxnatx yathA tadavxdaBxvAnapayxparAdhinaH \hfill|| 70 || 
\end{shl}

\begin{artha}
namamxnunx niVvu aparAdhiyeMdu tiLiyabeVDi. nimamx pUviRkarAda 
pitAmaharu. nananx pUvaRpitAmaharanunx aparAdhigaLeMdu tiLidiralilalx. 
niVvu avaraMte tiLiyabeVDiri eMdu (rAjanu kaSxme keVLidanu).
\end{artha}

\begin{shl}
tavxtasxMparxdAnataH pUvaRM videyxVyaM hi kadAcana | \\
noVvAsa bArxhamxNeV sAdhivxV sAkASxdapi baqhasapxtw \hfill|| 71 || 
\end{shl}

\begin{artha}
ninage I vidAyxdAna mADuva muMce, I utatxmavAda videyxyu yAvatUtx 
bArxhamxNanalilx iralilalx, sAkASxtf baqhasapxtiyalUlx iralilalx||
\end{artha}

\begin{shl}
EvaM gupAtxmapi tu tAM vakASxyXmeyxVvAhamacnajxsA | \\
parxtAyxKAyxtuM samathaRH koV burxvanatxM bArxhamxNaM naqpaH \hfill|| 72 || 
\end{shl}

\begin{artha}
I riVtiyAgi rakiSxsalapxTiTxdadxrU A videyxyanunx nAnu neVrA heVLiye 
koDuvenu. yAva rAjanu I riVti heVLuva bArxhamxNananunx tirasakxrisalu 
samathaRnAguvanu?
\end{artha}

\vishaya{baq. a.6, bArx. 2, kaMDike 8, 9}

\begin{shl}
sa hoVvAca tathA nasatxvXM gwtama mAparAdhAsatxva ca pitAmahA yatheVyaM videyxVtaH pUvaRM na kasimxMshacxna bArxhamxNa uvAsa tAM tavxhaM tuBayxM vakASxyXmi koV hi tevxYvaM burxvanatxmahaRti parxtAyxKAyxtumiti || 8 ||
\end{shl}

\begin{shl}
asw veY loVkoV\s ginxrwgxtama tasAyxditayx Eva samidarxshamxyoV dhUmoV\s haraciRdiRshoV\s knAgxrA avAnatxradishoV visuPxliknAgxsatxsimxnenxVtasimxnanxgwnx deVvAH sharxdAdhxM juhavxti tasAyx AhuteyxY soVmoV rAjA samaBxvati || 9 ||
\end{shl}

\vishaya{nAlakxne parxshenxyaneVnx modalu niNaRya mADalu Enu kAraNaveMdu parxshinxsuvudu {\rm --}}

\begin{shl}
asAviti karxmoV\s BeVdi kasAmxdedhxVtoVritiVyaRtAmf | \\
parxshanxseyxVha catuthaRsayx pArxdhAnAyxdiBxdayxteV karxmaH \hfill|| 73 || 
\end{shl}

\begin{artha}
`asawveY' itAyxdiyAgi heVLida karxmavu yAva kAraNadiMda beVpaRTiTxteMbudanunx heVLabeVku. ililx nAlakxne parxshenxge samAdhAna pArxdhAnayxviruvudariMda karxmavu beVpaRTiTxde (eMbudeV utatxra).
\end{artha}

\vishaya{pArxdhAnayxviruvudariMda nAlakxneya parxshenxyanunx modalina niNaRyakekx tegedukoMDideyeMbudanunx samapiRsuvudu {\rm --}}

\begin{shl}
utapxtetxVsatxdadhiVnatAvxjajxnAmxyatAtx sithxtisatxthA | \\
sithxtayxpAyeV parxyANaM ca shurxtAyx\s BeVdi karxmasatxtaH \hfill|| 74 || 
\end{shl}

\begin{artha}
mAnavara janamxvu nAlakxne parxshenxya niNaRya nimitatxvAgiruvudariMda (pArxdhAnayxvidudx) I parxshenxge parxthama sAthxnaviralu, anaMtara janAmxdhiVnavAda sithxtiyU hAgU sithxtige apAyavuMTAdAga parxyANavU saMBavisuvudu. adariMda shurxtiyalilx pAThakarxmavu beVpaRTiTxtu.
\end{artha}

\vishaya{inunx `a sw veY loVkoVginxH'... itAyxdi maMtarxda 
vAyxKAyxnavu {\rm --}}

\begin{shl}
dUratoV\s muSayx loVkasayx sAyxdasAviti giVriyamf | \\
samidUdhxmAdiBiV rUpeYloVRkayxteV loVkagiVrapi \hfill|| 75 || 
\end{shl}

\begin{shl}
veYshabadxH samxraNAya sAyxdaginxsatxtapxriNAmataH | \\
yata AhavaniVyoV\s ginxduyxRloVkAtamxtayA sithxtaH \hfill|| 76 || 
\end{shl}

\begin{artha}
shurxtiya `asw' eMba padavu dUradalilxruva paraloVkavanunx heVLuvudu. 
samitutx, dhUma modalAda rUpagaLiMda kANisuvudariMda loVkaveMba padavU 
ide. veY shabadxvu samxraNAthaRkavAgide. avugaLa pariNAmadiMda 
aginxyeMdu heVLalapxTitxde, kAraNaveVneMdare? AhavaniVyAginxyu 
duyxloVka rUpadalilxruvudu.
\end{artha}

\vishaya{adu heVgeMdare:-}

\begin{shl}
apUvaRpariNAmoV\s yamaginxhoVtArxKayxkamaRNaH | \\
duyxloVkoVpakarxmoV jecnxVyoV yAvatupxruSasaMBavaH \hfill|| 77 || 
\end{shl}

\begin{artha}
aginxhoVtarxveMba kamaRda apUvaR pariNAmavidu. adu duyxloVkadiMda 
AraMBisi puruSanalilx huTuTxva payaRMtaraveMdu tiLiyabeVku.
\end{artha}

\vishaya{aginxhoVtarx parxkaraNavanunx yoVcisidarU duyxloVkAdigaLu aginx eMdu heVLuvudu yukatx {\rm --}}

\begin{shl}
AhutoyxVraginxhoVtarxsayx yA viBUtiH puroVditA | \\
seYveVha daqSiTxvidhayxthaRM shurxtAyx vAyxKAyxyateV\s cnajxsA \hfill|| 78 || 
\end{shl}

\begin{artha}
matutx aginxhoVtarxda AhutigaLa viBUtiyanunx hiMde yAvudanunx heVLididxto, adaneVnx ililx duyxloVkadalilx oMdAgi kANabeVkeMba upAsanA daqSiTxyanunx vidhisalu shurxtiyu sAkASxtf vAyxKAyxnisiruvudu.
\end{artha}

\begin{shl}
adhAyxtemxV cAdhiyajecnxV ca hayxdhiloVkAdhideYvayoVH | \\
shurxtirAhavaniVyAdeVvAyxRcaSeTxV visatxqqtiM suPxTAmf \hfill|| 79 || 
\end{shl}

\begin{artha}
shurxtiyu AhavaniVyAdi visAtxravanunx \footnote[1]{`natevxVveYtayoVH satxvXmutAkxrXnitxM nagatiM na parxtiSAThxM nataqpitxM na punarAvaqtitxM na loVkaM parxtuyxtAthxyinaM veVtethxti' eMdu aginxhoVtarx parxkaraNadalilx Aru parxshenxgaLa niNaRyakAkxgi utatxra vacanavu baMdide. `teVvA EteV AhutiV huteV (satwyx) utAkxrXmataH teV anatxrikaSxM parxvishataH teV anatxrikaSx mAhavaniV yaMkuvARteV, vAyuM samidhaM mariVciVreVva shukArx mAhutiM | teV anatxrikaSxM tapaRyataH | teV tata utAkxrXmataH || teV diva mAvishataH | teV diva mAhavaniVyaM kuvARteV AditayxM samidhamf' itAyxdiyAgi heVLidudx kaMDide. yajamAnanu maraNa hoMduvAga avana aginxhoVtarxda AhutigaLeraDU sAdhana padAthaRgaLoMdige biTuTx ELuvavu. heVge badukiruvAga yAva sAdhanagaLiMda kUDiyeV yAva AhutigaLu, kUDalapxTaTxvo aMdare AhavaniVyAginx, samitutx, dhUma, aMgAra (keMDa) kiDigaLU matutx Ahuti darxvayxveMba sAdhanagaLiMda kUDidaveMdu tiLididadxvo hAgeye I loVkavanunx biTuTx paraloVkakekx ededxVLuvavu. aginx, samitutx, dhUma, aMgAra, kiDigaLu, Ahuti darxvayx kiSxVrAdidarxvayx eMba rUpadalelx saqSiTxya Adiyalilx avAyxkaqtadasheyalilxyU sUkaSxmXvAda matotxMdu rUpadiMda (parabarxhamx rUpadalilx) irutatxve. I riVtiyAgidudx aginxhoVtarxveMba kamaRvu sAdhanasahitavAgi apUvaR savxrUpavAgi vayxvasithxtavAgidudx punaH saqSiTxkAladalilx aMtarikASxdigaLige AhavaniVyAginx muMtAda rUpadalilx pariNAmagoLisuvudu. idu heVgoV hAgeye IgalU aginxhoVtarxveMba kamaRvu pariNAmavanunx hoMduvudu. I riVtiyAgi aginxhoVtarxda AhutigaLa apUvaR pariNAmavAgi jagatetxlalxvU iruvudeMdu AhutigaLa sutxtigAgiye heVLide. adariMda utApxrXpitx muMtAda Aru padAthaRgaLanunx kamaRparxkaraNadalilx niNaRyiside. parxkaqta I bArxhamxNadalilx kataRqqvige kamaRveVneMbudanunx heVLalu aMtarikaSx loVkadiMda AraMBisi, yoVSitf payaRMtara iruva 5 padAthaRgaLalilx paMcAginx daqSiTxyanunx kataRvayxveMdu heVLi, adu utatxra mAgaRpArxpitxge sAdhanaveMdU, adeV vishiSaTxvAda kamaRPalada BoVgakAkxtgi vidhisalapxDuvudeMdU, tiLisutAtx ililx modalu duyxloVkAginx muMtAda daqSiTxyanunx heVLalu shurxtiyu AraMBisiruvudu.}adhAyxtamx, adhiyajacnx, adhiloVka, adhideYva ivugaLalilx sapxSaTxvAgi heVLiruvudu. (adariMdalU duyxloVkAdigaLu aginxyeMdu tiLiyabahudu)
\end{artha}

\begin{shl}
loVka AhavaniVyoV\s ginxrasAviti vicinatxyeVtf | \\
AditAyxdiSavxpi tathA samidAdisamiVkaSxNamf \hfill|| 80 || 
\end{shl}

\begin{artha}
I paraloVkave AhavaniVyAginxyeMdu ciMtisabeVku. hAgeyeV AditAyxdigaLalUlx samitutx modalAda daqSiTxyanunx iDabeVku.
\end{artha}

\begin{shl}
saminadhxnAtasxmidABxnU rashamxyoV dhUma itayxpi | \\
samininxgaRmasAmAnAyxdaciRrahasatxtheYva ca \hfill|| 81 || 
\end{shl}

\begin{artha}
cenAnxgi parxkAshagoLisuvudariMda Aditayxnu samitetxMdU, samitutxgaLiMda horage baruva hoVlikeyiMda sUyaRna kiraNagaLeV dhUmaveMdU, hagaleV jAvxleyeMdU ciMtisabeVku.
\end{artha}

\begin{shl}
shAnatxtAvxcacx dishoV\s knAgxrAH samididhx pariNAmataH | \\
aciRraknAgxraBAvasayx yatheYvaM BAnuheVtukAH | \\
rashamxyashacx dishashecxYtA AditayxsamidAsharxyAH \hfill|| 82 || 
\end{shl}

\begin{shl}
avAnatxradishasatxdavxdivxkiSxpatxtevxYkaheVtutaH | \\
visuPxliknAgx iti jecnxVyAsatxsimxnanxgwnx yathoVditeV \hfill|| 83 || 
\end{shl}

\begin{artha}
dikukxgaLeV aMgAragaLu (keMDagaLu) kAraNa shAMtavAgiruvudariMda aciR (jAvxle)yeMbudu aMgAraveMbuva dikukxgaLadudx. heVge kiraNagaLu dikukxgaLu Aditayx samitatxnunx Asharxyisiruvavo hAgeyeV Aditayx samititxna pariNAmavAdadxriMda dikukxgaLeV aMgAragaLeMdu heVLalapxTiTxve.

madhayxvatiRyAda mUledikukxgaLu vikiSxpatxvAgiruva nimitatxdiMda visuPxliMgagaLeM(kiDigaLeM)dU tiLiyabeVku. A aMtarikaSxveMba aginxyaleVlx deVvategaLu hoVma mADuvaru.
\end{artha}

\vishaya{hoVma mADuva deVvategaLu yAru? eMdare {\rm --}}

\footnotetext[1]{aSaTxvasugaLu, EkAdasharudarxru, dAvxdasha Aditayxru 
iMdarx, parxjApati, oTuTx 33 deVvategaLu, adhAyxtamx, adhideYva 
modalAda BeVdadiMda BinenxYsi AyAya BAvavanunx hoMdi yajamAnanu 
irutitxralu avana kamaR, upAsanegaLige anusAravAgi AyAya 
kalipxtAginxgaLalilx hoVma mADuva hoVtaqgaLAgiruvaru. parxkaqta ililx 
AdhAyxtimxkavAda pArxNagaLu (vAgAdigaLu) yajamAnanige 
saMbaMdhapaTiTxruvugaLeV deVvategaLeMdu heVLalapxTiTxve. avugaLe 
hoVtaqgaLAgi AdhideYvikavAgi pariNamisi iMdArxdi deVvategaLAgiruvaru.}
\begin{shl}
\footnotemark[1]tarxyasitxrXMshacacx yeV deVvAH suyxsetxV\s dhAyxtAmxdiBUmigAH | \\
hoVtArasatxtarx tatarx suyxH kamaRjAcnxnAnuroVdhataH \hfill|| 84 || 
\end{shl}

%% shloka footnote
\begin{artha}
yAva mUvatutx mUru deVvategaLu sidadhxvAgiruvaro avareV adhAyxtamx 
modalAda sAthxnagaLalilx hoVtaqgaLAgi viMgaDisalapxTiTxve. alalxlilx 
kamaRjAcnxnAnusAravAgi iruvavu.
\end{artha}

\begin{shl}
QutivxgUrxpeVNa teV hAyxsanayxthA pArxkaqtakamaRNi | \\
hoVtAraH pariNAmeVSu tatheYvoVtatxraBUmiSu \hfill|| 85 || 
\end{shl}

\begin{artha}
hiVge joyxVtiSoTxVmAdi pArxkaqta kamaRgaLalilx QutivxkAkxgi 
hoVtaqgaLAgiruvaro hAgeye kamaRda apUvaR pariNAmavAda muMdina 
duyxloVkAdi BUmigaLalilx A yajamAna \footnote[1]{hiMde heVLidaMte 
yajamAnana pArxNaveMbuva iMdirxyagaLu I vayxvahAradalilx 
aginxhoVtarxda hoVtaqgaLAgi AdhAyxtimxkavenisidudx AdhideYvika 
rUpadalilx pariNamisi iMdArxdi deVvategaLAgi avareV duyxloVkAginxyalilx 
hoVma mADuva hoVtaqgaLAdaru. avaru aginxhoVtarxdx Palavanunx anuBava 
mADalu aginxhoVtarxvaneVnx hoVmADidaru. avare aginxhoVtarxda PalavAgi 
pariNamisuvAga PalaBoVgigaLAguvudariMda alalxlilx hoVtaqsAthxnavanunx 
hoMduvaru. duyxloVka, pajaRnayx, paqthiviV, puruSa, yoVSitf eMba 
aginxsAthxnagaLige yoVgayxvAda rUpadalilx pariNAmavanunx hoMdi 
deVvareMdu heVLalapxDuvaru. ivareV hoVtaqgaLu.}pArxNa deVvategaLu 
hoVtaqgaLAgiruvaru.
\end{artha}

\vishaya{`sharxdAdhxM juhavxti' eMdu sharxdedhxyu hoVma darxvayxveMdu heVLidudx heVge yukatx? {\rm --}}

\begin{shl}
AhutoyxVH pariNAmoV\s yamUgarxRsoV\s pUvaRmitayxpi | \\
tasayx sharxdedhxYkaheVtutAvxcaCxrXdAdhx nAmeVti kiVtayxRteV \hfill|| 86 || 
\end{shl}

\begin{artha}
AhutigaLa pariNAmave idu baliSaThxvAda rasa, matutx apUvaRveMdU heVLalapxDuvudu. adu sharxdedhxyeMba oMdeV kAraNadiMda huTiTxdeyAdadxriMda \footnote[2]{idaralilx aginxhoVtarx kamaRdalilx upayukatxvAda payoVdarxvayxvanunx AhavaniVyadalilx hoVma mADidAga aginxyu adanunx BakiSxsuvudu, aginxyu sivxVkarisidudx kaNiNxge kANadaSuTx sUkaSxmX rUpadalilx pariNAma hoMdi kataRqqvAda yajamAnanoDane paraloVkakekx dhUmAdi mAgaR karxmadalilx horaTu aMtarikaSxdalilx seVri duyxloVkavanunx parxveVshisuvudu. I riVtiyAgi AviyAgi sUkaSxmXvAgi pariNAma hoMdida (kiSxVradarxvayxda Ahuti pariNAmavAda) sUkaSxmXvAda jalavanunx kataRqqsameVtavAgidadxdadxnunx sharxdAdhx eMdu shabadxdiMda shurxtiyalilx karedide. soVmaloVkadalilx (caMdarx loVkadalilx) yajamAnanige divayxshariVravanunx uMTumADalu duyxloVkadalilx adu parxveVshisidAdxga adu hoVma mADalapxDuvudeMdu shurxtiyalilx gawNavAgi heVLalapxTiTxde. `sharxdAdhxM juhavxti' eMdu, A sUkaSxmXvAda payoVdarxvayxveV duyxloVkavanunx parxveVshisi caMdarxmaMDaladalilx kataRqqvina shariVravanunx huTiTxsuvudu. idanenxV `deVvAH sharxdAdhxM juhavxti tasAyx AhuteyxYsoVmoVrAjA saMBavati' eMdu heVLiruvudu.}sharxdAdhx eMba hesariniMda kareyalapxTiTxde.
\end{artha}

\vishaya{utatxravAkayxvanunx vAyxKAyxnisuvaru {\rm --}}

\begin{shl}
tasAyxshAcxpAyxhuteVH soVmoV rAjA saMBavatiVti ca | \\
tasAyxBivaqdidhxH saMBUtinaR tavxBUtajaniyaRtaH \hfill|| 87 || 
\end{shl}

\begin{artha}
A sharxdAdhx eMba AhutiyiMda soVmaveV rAjanAguvanu. (pitaqgaLigU bArxhamxNarigU) rAjanAguvanu (caMdarxnAguvanu)|| adara utapxtitx heVgeMdare? idadx vasutxvina aBivaqdidhxye saMBUti (utapxtitx) yeMdu tiLiyabeVku. EkeMdare? ilalxda vasutxvige utapxtitxyu saMBavisuvudeV ilalx.
\end{artha}

\vishaya{I riVtiyAgi modalane payARyada vAkAyxthaRvanunx saMkeSxVpisi upasaMharisuvudu {\rm --}}

\begin{shl}
dwyxraginxH samidAditayxH sharxdAdhx tasimxnihx hUyateV | \\
sUyeVR samidhi diVpAtxyAM sharxdAdhxM juhavxti deVvatAH \hfill|| 88 || 
\end{shl}

\begin{artha}
duyxloVkaveV aginx, AditayxneV samitutx, adaralilx hoVma mADalapxDuva 
darxvayxveMdare sharxdedhx (sUkaSxmXdarxvadarxvayx), sUyaRsamitutx 
uriyutitxralu A aginxyalilx deVvategaLu sharxdedhxyanunx hoVma mADuvaru.
\end{artha}

\vishaya{sharxdAdhxshabadxvanunx kamaRda sUkaSxmXvasutxvinalilx heVge parxyoVgisidudx?}

\footnotetext[1]{BawtikavAda parxpaMcakekx paMcaBUta sUkaSxmXgaLU avashayxvAgi upAdAna kAraNavAgiruvavu. parxkaqta parxdhAnavAgi tegedukoMDidudx AhutidarxvayxkiSxVra, adara sUkaSxmXvAda jalAMshavanunx ililx parxdhAnavAgiyeV heVLide vinaha itara BUtagaLanunx biTiTxlalx. I sUkaSxmXjalavu itara BUta sUkaSxmX vasutxgaLoMdige Bawtika parxpaMcakekx poVSakavAgiyU AsharxyavAgiyU iruvudariMda sharxdAdhx shabadxda athaRvAgiruvavu.}
\begin{shl}
sharxyateVH sharxdadxdhAteVvAR sharxdedhxVtAyxhuviRpashicxtaH \hfill|| 89 || \\
\footnotemark[1]sharxyaNAdAdhxraNAcAcx\s \s paH sharxdAdhxhAvxH kAraNAtimxkAH | \\
BUtAvx\s \s pa iti liknAgxcacx ApaH sharxdAdhxBidhAsatxtaH \hfill|| 90 || 
\end{shl}

%% shloka footnote
\begin{artha}
`sharxyatiVti sharxdAdhx, sharxdadhxdhAtiVti sharxdAdhx' eMdu vidAvxMsaru sharxdAdhxshabadxda vuyxtapxtitxyeMdu karediruvaru. Bawtika parxpaMcakekx AsharxyavAdadxriMdalU poVSakavAdadxriMdalU I sUkaSxmXvAda jalavu sharxdedhxyeMdu hesarAgiruvavu. kAraNarUpavAgiruvadariMdalU `ApaH puruSavAcoVBUtAvx samutAthxyavadanitx' eMba upakarxmaliMgadiMdalU apf sharxdAdhx shabadxdiMda heVLalapxTiTxde.
\end{artha}

\vishaya{sUkaSxmXvAda apf aBivaqdidhxyAguva bage}

\footnotetext[1]{kaqSaNxpakaSxdalilx caMdarxnu kaSxyisutAtx 
amAvAseyxyalilx sUyaRnoLage seVruvanu. adaralilxruva niVriniMda 
shukalxpakaSxdalilx karxmavAgi aBivaqdidhx hoMduvanu. I vaqdidhxyanenx 
hiMde caMdarxna saMBUti (utapxtitx) yeMdu heVLidudx \\`vivasAvxnaMshuBiH sitxVkeSxNXYH adAyajagatoV jalamf |\\
soVmeV mucnacxtayxtheVnudx shacx vAyunADiVmayeYdivxRjaH ||'\\ sUyaRnu 
tiVkaSxNXvAda kiraNagaLiMda jagatitxna niVranunx caMdarxnalilx 
biDuvanu. anaMtara caMdarxnU kUDa vAyunADigaLa rUpadalilx niVranunx 
eLadukoMDu biDuvaneMdu I samxqqtiya athaR. TiVkAkAraru (AnaM-TiVkA) 
udAharisidAdxre. I sholxVkavu meVlina viSayakekx heVge 
hoMdikeyAguvudo? namage tiLiyadu. sholxVkada BAvAthaRvU namage 
tiLiyadAgide. vidAvxMsaru shoVdhisabeVkAgide.}
\begin{shl}
\footnotemark[1]AkaqSaTxM rashimxBisotxVyamAditeyxV parxtitiSaThxti | \\
tasAmxdAditayxgaH soVmaH kiSxVNa ApAyxyateV punaH \hfill|| 91 || 
\end{shl}

%% shloka footnote
\begin{artha}
sUyaRna kiraNagaLiMda meVlakekx AkaSiRsalapxTaTx niVru Aditayxnalilx 
nilulxvudu. adariMda Aditayxnalilxruva caMdarxnu kiSxVNanAgi punaH 
vaqdidhxgoLuLxvanu.
\end{artha}

\vishaya{caMdarxnu teYjasaneMdu yAru heVLuvaro avara matakekx AhutiyiMda caMdarxnu huTuTxvaneMbudu virudadhxvalalxve? eMdare}

\begin{shl}
pariNAmoV hayxpAM soVmaH shiVtAMshusetxVna soV\s mamxyaH | \\
sharxdAdhxhuteVhiR soVmasayx saMBavaH shAsarx ucayxteV \hfill|| 92 || 
\end{shl}

\begin{artha}
shiVtakiraNavuLaLx caMdarxnu jalada pariNAma. adariMda avanu 
jalamayanu. adariMda shAsatxrXdalilx shudadhxvAda AhutiyiMda caMdarxna 
utapxtitxyanunx heVLide.
\end{artha}

\vishaya{hAgAdare manuSayxloVkAginxyalilx caMdarxnu aMgAravAdadudx heVge? {\rm --}}

\begin{shl}
aknAgxrAshacxnadxrXmAsatxsimxnuhxteV\s gwnx soVmasaMBavaH | \\
soVmacanadxrXmasoVreVvaM BeVdaH shAsetxrXVNa dashiRtaH \hfill|| 93 || 
\end{shl}

\begin{shl}
canadxrXmA maNaDxlaM savxcaCxM canadxrXkeVNa mitoV hi saH | \\
soVmasutx maNaDxleV shevxVtoV vadhaRteV harxsateV ca yaH \hfill|| 94 || 
\end{shl}

\begin{artha}
A duyxloVkAginxyalilx hoVma mADida meVle soVmana (caMdarxna) 
utapxtitxyu Agabahudu. caMdarxnu (manuSayx loVkAginxyalilx) 
aMgAravAgabahudu. I riVtiyAgi soVma matutx caMdarxra BeVdavanunx 
toVriside\footnote[2]{caMdarxnu kAraka, adara Pala soVma eMbudeV BeVda, 
`caMdarxmA aknAgxrAH tasAyx AhuteyxY soVmoVrAjA Bavati' eMba shAsatxrXveV I BeVdakekx parxmANa.}.
\end{artha}

\begin{artha}
\footnote[3]{caMdarxmaMDaladalilx caMdarxmA eMba shabadx parxyoVgakekx kAraNa `caMdarxkeVNa mitaH' eMbudu. aMdare aBarxkakekx samAnavAdadadxriMda caMdarx eMdu heVLuvudu. maMDaladoLagiruva puruSanalilx (Atamxnalilx) soVma shabadxvanunx parxyoVgisuvudakekx nimitatx vaqdidhxhArxsagaLanunx hoMdutatxlilxruvudeV. heVge soVmalateyu shukalxpakaSxdalilx vaqdidhxyanunx kaqSaNxpakaSxdalilx kaSxyavanunx hoMduvado hAgeyeV caMdarxmaMDalada puruSanu hoMduvanu. adariMda soVma eMdu gawNavAgi I puruSananunx kareyuvudu. Adare vaqdidhxkaSxyagaLu caMdarxmaMDaladedxV AgiveyeMdu heVLalAgadu. idakekx niyAmakaveVnU ilalx. `Eva meVnAMsatxtarxBakaSxyanitx' eMdu shurxtiyu caMdarxnanenxV BoVgayxveMdu heVLideyAdadxriMda keVvala maMDalaveV BoVgayxvAguvudilalx. caMdarxnanunx paDeda kamiRgaLanunx deVvategaLu upaBuMjisuvaru. Baqtayxranunx parxBugaLu tamamx BoVgakAkxgi seVvArUpadalilx upayoVgisikoMDu anuBavisuvaro hAgeye deVvategaLu caMdarxnalilx caMdarxna shariVradaMtiruva shariVravanunx hoMdiruva kamiRgaLanunx deVvategaLu seVvA mUlaka anuBavisuvareMdu `EvameVnAM.....' eMba meVlina shurxtiya athaR. adariMda deVvategaLige BoVgayxvAgiruvudu maMDaladoLagina puruSaneMde heVLuvudu yukatxveMdu TiVkAkArara aBipArxyavide.}caMdarxneMbudu savxcaCxvAda maMDala. aBarxkakekx (biMkakekx) samAnavAgide. maMDaladalilxruva puruSanu beLaLxge vaqdidhx hoMduvanu, matutx kaSxyavanunx hoMduvanu yAroV avaneV soVmanu.
\end{artha}

\begin{shl}
canadxrXmAH para AditAyxdavARkosxVmaH shurxteVmaRtaH | \\
AditAyxcacxnadxrXmitAyxha neYteV saMvatasxraM tathA \hfill|| 95 || 
\end{shl}

\begin{artha}
matutx caMdarxnu Aditayxna naMtaraveMdU, Aditayxna Icege soVmaveMdU shurxtiya aBipArxyavide. \footnote[1]{deVvayAna mAgaRvanunx tiLisuvAga `teVciRSamaBisaMBavanitx aciRSoV\s haH...........saMvatasxrAdAditayxmf, AditAyxcacxdarxmasaM caMdarxmasoVviduyxtamf' eMdu heVLide. hAgU pitaqyANa mAgaRvanunx heVLuvAga `teV dhUmamaBisaMBavanitx.......AkAshAcacxMdarxmasa meVSa soVmoV rAjA' eMdu caMdarxna naMtara soVmavanunx heVLide. adariMda caMdarx soVmagaLige savxlapx BeVdavideyeMdu vAtiRkada Ashaya. `neYteV saMvatasxra maBipArxpunxvanitx' eMdu saMvatasxra deVvateya hatitxra I dakiSxNa mAgaRdalilxruva kamiRgaLu baruvudilalxveMdu heVLiyU ide.}``AditAyxcacxMdarxmasamf" eMdu CAMdoVgayxdalilxde.
\end{artha}

\vishaya{upasaMhAra {\rm --}}

\begin{shl}
soVmacanadxrXmasoVsatxsAmxdeBxVdaH samavagamayxteV | \\
\footnotemark[1]deVshABeVdAdaBinwnx tAveVSa soVma iti shurxteVH \hfill|| 96 || 
\end{shl}
\footnotetext[1]{savxrUpaBeVdavidadxre deVshavoMdAdadxriMda oMdu eMdu 
heVLuvudu. parxkaqta `pitaqloVkAcacxnadxrXmeVSasoVmoV rAjA' eMba aBeVdashurxti, soVmavu 
caMdarxmaMDaladalelxV iruvudariMda averaDanUnx oMdAgi heVLide.}

%% shloka footnote
\begin{artha}
adariMda soVma matutx caMdarxgaLige BeVdavu tiLiyuvadu. sathxLavu 
oMdAgiruvudariMda averaDU oMdeV, `ESa soVmaH' eMba shurxtiyu parxmANa.
\end{artha}

\vishaya{hAgAdare averaDU Binanxvenunxva shurxtiyu heVge sari? eMdare}

\footnotetext[2]{ililx dhamaRBeVdaveMdare = savxrUpaBeVda.}
\begin{shl}
Binwnx ca \footnotemark[2]dhamaRBeVdeVna tasAmxduBayathA shurxtiH | \\
\end{shl}

\begin{artha}
alalxde dhamaRBeVdadiMda avugaLu BinanxvU Agive. adariMda eraDu bageyalUlx shurxtiyu saMgatavAgide. 
\end{artha}

\vishaya{eraDaneV payARya vAyxKAyxna {\rm --}}

\vishaya{baq. a.6, bArx. 2, kaMDike 10}

\begin{shl}
pajaRnoyxV vA aginxrwgxtama tasayx saMvatasxra Eva samidaBArxNi dhUmoV viduyxdaciRrashaniraknAgxrA hArxdunayoV visuPxliknAgxsatxsimxnenxVtasimxnanxgwnx deVvAH soVmaM rAjAnaM juhavxti tasAyx AhuteyxY vaqSiTxH samaBxvati || 10 ||
\end{shl}

\begin{shl}
pajaRnoyxV\s ginxriti jecnxVyaH samitasxMvatasxraH samxqqtaH \hfill|| 97 || 
\end{shl}

\begin{shl}
saMvatasxreV samidedhxV hi pajaRnayxsayx sameVdhanAtf | \\
dhUmoV\s BArxNiVti sAdaqshAyxdivxduyxdaciRsatxtheYva ca \hfill|| 98 || 
\end{shl}

\begin{shl}
shAnitxvatuRlatoV\s knAgxrAH pajaRnayxshiKinoV\s shaniH | \\
vikiSxpatxtevxYkasAmAnAyxtusxPXliknAgxH satxnayitanxvaH \hfill|| 99 || 
\end{shl}

\begin{shl}
soVmaM juhavxti tatArxgwnx deVvAshAcxtorxVditAH purA | \\
vaqSeTxVshacx saMBavoV\s payxsAmxtf loVkeV\s simxnf sApi hUyateV \hfill|| 100 || 
\end{shl}

\begin{artha}
pajaRnayxveV aginxyeMdU, saMvatasxraveV samitetxMdU tiLiyabeVku. 
saMvatasxravu uriyalapxDalu pajaRnayxvU uriyuvudariMda hiVge 
kalipxside. meVGagaLeV dhUma. sAdaqshayxviruvudariMda viduyxtetxV 
miMceV jAvxle pajaRnAyxginxya aMgAragaLeMdare ashani, siDilu, 
upashAMti vatuRlagaLiMda guDugina shabadxgaLeV kiDigaLeMdU elelxDe 
haraDiruva sAdaqshayxdiMda heVLalapxDuvavu. I aginxyalilx hiMde 
heVLida deVvategaLu modalu hoVmavanunx mADuvaru. adariMda vaqSiTxya 
utapxtitx. adanenxV I BUloVkadalUlx hoVma mADalapxDuvudu.
\end{artha}


%%%% From 056.tex

%~ \vishaya{baq - 6 - 2 - 11}

%~ \vishaya{vAtiRka 101 riMda}

\begin{shl}
ayaM veY loVkoV\s ginxrwgxtama tasayx paqthiveyxVva samidaginxdhURmoV rAtirxraciRshacxnadxrXmA aknAgxrA nakaSxtArxNi visuPxliknAgxsatxsimxnenxVtasimxnanxgwnx deVvA vaqSiTxM juhavxti tasAyx AhutAyx ananxM samaBxvati || 11 ||
\end{shl}

\vishaya{vAtiRka}

\begin{shl}
loVkoV\s yamaginxviRjecnxVyaH paqthiviV samiducayxteV | \\
paqthivAyx hi samidodhxV\s yaM samitetxVna kiSxtimaRtA \hfill|| 101 || 
\end{shl}

\begin{shl}
aginxdhURmasatxdutAthxnAdArxtirxraciRsatxtheYva ca | \\
rAtirxmAhuH kiSxticACxyAmaknAgxrAshacxnadxrXmAsatxthA \hfill|| 102 || 
\end{shl}

\begin{shl}
shAnatxtavxvatuRlatAvxBAyxM visuPxliknagxsamatavxtaH | \\
nakaSxtArxNi suPxliknAgxH suyxsatxsimxninxtAyxdi pUvaRvatf \hfill|| 103 || 
\end{shl}

\begin{shl}
ananxsayx saMBavasatxsAmxtupxmagwnx tacacx hUyateV | \\
puruSoV\s ginxriti dheyxVyoV vAyxtatxM tasayx samitasxmXqtA \hfill|| 104 || 
\end{shl}

\begin{shl}
vAyxtetxV muKeV hi tadidxVpitxdhURmaH pArxNoV muKoVtithxteVH | \\
puMdiVpetxVvARkninxmitatxtAvxdaciRvARketxVna BaNayxteV \hfill|| 105 || 
\end{shl}

\begin{shl}
sithxteVraknAgxravacacxkuSxraknAgxrAH shorxVtarxmeVva ca | \\
visuPxliknAgx iti jecnxVyaM tasayx vikeSxVpasaMsithxteVH \hfill|| 106 || 
\end{shl}

\begin{shl}
ananxM juhavxti tatArxgwnx saMBavoV reVtasasatxtaH | \\
yoVSidaginxriti jecnxVyA upasathxshacx samitatxthA \hfill|| 107 || 
\end{shl}

\begin{shl}
tadupasethxVna saMdiVpetxVdhURmoV loVmAni sAmayxtaH | \\
aciRvaRNaRsamAnatAvxdoyxVniraciRBaRveVtatxtaH \hfill|| 108 || 
\end{shl}

\begin{shl}
anatxH karoVti yatAkxmiV teV\s knAgxrAsatxtasxmAnataH | \\
visuPxliknagxH suKalavAH kaSxNikatevxYkaheVtutaH \hfill|| 109 || 
\end{shl}

\begin{shl}
reVtoV juhavxti tatArxgwnx deVvAshecxVnidxrXyarUpiNaH | \\
pacnacxmAyx AhuteVsatxsAyxH puruSaH saMBavatayxyamf \hfill|| 110 || 
\end{shl}

\begin{artha}
I loVkaveV aginx, paqthiviyeV samitetxMdu heVLide, paqthiviyiMda I
loVkavu parxkAshagoMDideyeMba kAraNadiMda paqthiviyanunx samitetxMdu
aBipArxyapaTiTxde.
\end{artha}

\begin{artha}
aginxyanunx dhUmaveMdu adariMda edidxruvudariMda heVLide, rAtirxyeV
aciR (jAvxle) \footnote{``EtAni hi caMdarxM rAterxV satxmasoV maqtoyxVbiRBayxta matayxpArayanf '' eMba shurxtiyiMda rAtirxyanunx
  tamasesxMdu tiLiyuvudu `tasayx ca maqtuyxveYRtamashACxyA teVneYva tajojxyXVtiSA matuyxM tamashAPxyAMtarati' eMba shurxtiyiMda BUmiya CAyeyeMbuva
tamaseVsx rAhuveMdU parxsidadhxvAgide. \\`udadhxyXtayxpaqthiviVcACxyAM nimiRtaM maMDalAkaqti||\\
savxBARnoVsatxdf baqhatf sAthxnaM taqtiVyaM yatatxmoV mayaM ||' \\eMba samxqqtiyiMda
paqthiviya neraLanunx meVlakekxtitx maMDalAkAravAgi nimiRtavAdaMtiruva
rAhuvina doDaDxsAthxnavu tamoVrUpavAdudeMdu heVLalapxTiTxde.}rAtirxyanunx BUmiya neraLeMdu heVLuvaru,
caMdarxneV aMgAra (keMDa) shAMtavAgiyU duMDAgiyU iruvadariMda hiVge
heVLide, nakaSxtarxgaLu kiDigaLaMte iruvudariMda kiDigaLu, A I
loVkaveMba aginxyalilx deVvategaLu vaqSiTxyeMba A hutiyanunx
koDuvareMbudAgi adariMda ananxvu Aguvudo elalx vAyxKAyxnavU
hiMdinaMteye.
\end{artha}

\vishaya{baq - 6 - 2 - 12 ne kaMDike}

\begin{shl}
puruSoV vA aginxrwgxtama tasayx vAyxtatxmeVva samitApxrXNoV dhUmoV vAgaciRshacxkuSxraknAgxrAH shorxVtarxM visuPxliknAgxsatxsimxnenxVtasimxnanxgwnx deVvA ananxM juhavxti tasAyx AhuteyxY reVtaH samaBxvati || 12 ||
\end{shl}

\begin{artha}
ananxvu (dhAnayxvu) A vaqSiTxyiMda huTuTxvudu, adu puruSAginxyalilx
hoVma mADalapxDuvudu, adariMda puruSaneV aginxyeMdu dhAyxnisabeVku. A
puruSana tereda bAyiyu samitutx, tereda bAyiyiMda puruSanu
mAtanADuvudu adhayxyana mADuvudU itAyxdi diVpanavu Aguvadu, pArxNaveV
dhUma, bAyiMdale pArxNavAyavu horaDuvudu. puruSana parxkAshavu
vAkikxna nimitatxdiMdaleV AgavudariMda vAkakxnunx aciR (jAvxle)yeMdu
heVLide, kaNuNx keMDadaMtiruvudariMda keMDaveMdU shorxVterxVMdirxyavu
atitxtatx vikeSxVpagoLuLxvudariMda kiDigaLeMdU dhAyxnisabeVku. A
puruSAginxyalilx ananxvanunx hoVmamADuvaru, adariMda reVtasusx
huTuTxvudu. 

\textbf{yoVSA vA aginx rwgxtama tasAyx upasathx Eva samitf loVmAnidhUmoV yoVniraciRH yadanatxH karoVti teV\s knAgxrAH aBinanAdx visuPxliknAgxH tasimxnenxVtasimxnf nanxgwnx deVvA reVtoV juhavxti tasAyx AhuteyxY puruSaH saMBavati sa jiVvati yAva jijxVvati||.....||}
\end{artha}

\begin{artha}
sitxrXVyeV aginx, upasethxyeV samitutx, adariMdaleV
udidxVpanavAguvudariMda samitetxMdu BAvisabeVku, roVmagaLe
kapApxgiruva hoVlikayiMda dhUmaveMdU yoVniye aciRyeMdu keMpu baNaNxda
hoVlikeyiMda heVLalapxDuvudu, adaroLage kAmuka puruSanu Enu mADuvano
ade\footnote{keMDagaLu beMkiyu shAMtavAgiruvudakekx kAraNa, adaraMte
  kAmukapuruSanu neDesuva meYthuna kamaRvu sitxrXVyeMba aginxgU,
  puruSanigU teVjoVviVyARdigaLa nAshakekx kAraNavAdadxriMda
  shAMtikaravAgide. A kamaRdalilx aMgAra BAvaneyanunx mADabeVkeMdu
 `tatf samAnataH' eMba vAtiRkada padadiMda sUcitavAguva aBipArxyavu.} aMgAraveMdU suKaleVshagaLeV kiDigaLeMdU kaSxNikatavx
mAtarx nimitatxdiMda tiLiyabeVku, AsitxrXVyeMba aginxyalilx
iMdirxyarUpavAda deVvategaLu reVtasasxnunx hoVma mADuvaru, aidane
AhutiyiMda I puruSanu huTuTxvanu.
\end{artha}

\vishaya{vAtiRka}

\begin{shl}
yathoVkatxvatamxRnA hAyxpaH pArxpatxH puMsapxriNAmatAmf | \\
pAkajaH pariNAmoV\s yameVvaM pacnacxBiraginxBiH \hfill|| 111 || 
\end{shl}

\begin{artha}
hiMde heVLida mAgaRdalilx Ahuti pariNAma sUkaSxmX jalavu
puruSAkAradalilx pariNAmavanunx hoMdutatxde. ideV riVtiyalilx
paMcAginxgaLa mUlaka pAkadiMda idu pariNAmavAguvudu.
\end{artha}

\vishaya{upasaMhAra -}

\begin{shl}
tasAmxdApaH sUkaSxmXBAvAH sUthxlatAM yAnitx pAkataH | \\
aginxBiH pacnacxBiH pakAvxH puruSAKAyx Bavanitx hi \hfill|| 112 || 
\end{shl}

\begin{artha}
adariMda sUkaSxmXvAda jalatatatxvXvu pAka visheVSadiMda
sUthxlarUpavanunx hoMduvudu. paMcAginxgaLiMda pakavxvAda Ajala
tatatxvXvu puruSaveMba hesaranunx paDeyuvudu.
\end{artha}

\vishaya{avataraNike}

\begin{artha}
paMcAginx videyxyuLaLxvarige muMde baruva gatiyanunx heVLalu  moTaTxmodalu adakekx upayoVgiyAda I videyxyanunx tiLida gaqhasathxnadeVhavanunx I riVtiyAgi vaNiRsiruvaru atha yadAmirxyateV -
\end{artha}

\vishaya{baq - 6 - 2 - 14 kaMDike.}

\begin{shl}
atheYnamaganxyeV haranitx tasAyxginxreVvAginxBaRvati samitasxmidUdhxmoV dhUmoV\s ciRraciRraknAgxrA visuPxliknAgx visuPxliknAgxsatxsimxnenxVtasimxnanxgwnx deVvAH puruSaM juhavxti tasAyx AhuteyxY puruSoV BAsavxravaNaRH samaBxvati || 14 ||
\end{shl}

\begin{shl}
tasAmxnamxqqtaM pirxyaM banudhxM haranatxyXganxya QutivxjaH \hfill|| 113 | \\
yatheYvA\s \s havaniVyAgenxVH parxsidadhxM samidAdikamf | \\
shamxshAnAgenxVsatxtheYveYtatasxvaRM jecnxVyaM na kalapxyXteV \hfill|| 114 || 
\end{shl}

\begin{artha}
(avanu yAvAga maraNa hoMduvano anaMtaraveV Itananunx (avana  deVhavanunx) aginxgAgi oyuyxvaru A puruSanige parxtayxkaSx aginxyeV aginx, hAgeye samitutx dhUma, jAvxle aMgAra, kiDigaLu elalxvU  parxtayxkaSxvAgiruvaveV horatu hiMdinaMte yAvudU kalipxtavalalxveMdu  shurxtiya athaR -)
\end{artha}

\vishaya{vAtiRkada anuvAda}

\begin{artha}
adariMda (kamaRkaSxyavAdanaMtara) maqtanAda pirxya, athavA baMdhu
ivananunx Qutivxjaru aginxgAgi oyuyxvaru, heVge parxsidadhxvAda
AhavaniVyAginxge samitutx modalAdavu iruvavo hAgeyeV shamxshAnada
aginxgU elalxvU iruvudeMdu tiLiyabeVku. yAvudanUnx hosadAgi
kalipxsuvudilalx.
\end{artha}

\begin{shl}
anatxyXsaMsAkxrasidadhxyXthaRM tasimxnanxgwnx yathoVditeV | \\
QutivxjoV juhavxti naramanAtxyXhuteyxY vidhAnataH | \\
AhuteVjARyateV tasAyxH pumAnABxsavxrarUpaBaqtf \hfill|| 115 || 
\end{shl}

\begin{artha}
aMtayx saMsAkxravu sididhxsalu A aginxyalilx hiMde heVLidadxralilx
Qutivxjaru I puruSananunx aMtAyxhutigAgi vidhiyaMte hoVma mADuvaru, A
AhutiyiMda puruSanu parxkAshavAda (sAtivxka) rUpavanunx dharisuvanu.
\end{artha}

\begin{shl}
rAjasaM tAmasaM rUpamitoV hayxnayxtarx vakaSxyXteV \hfill|| 116 | \\
itoV\s ginxBoyxV\s ginxmeVvAyaM sAvxM yoVniM parxtipatasxyXteV | \\
iti loVkeV samAcArAdaginxBayxH saMBavasatxtaH \hfill|| 117 || 
\end{shl}

\begin{artha}
I sAtitxvXka rUpavanunx biTuTx uLida rAjasa tAmasarUpavanunx
(kamiRyalilx) muMde heVLuvudu. I paMcAginxgaLiMda beVre
parxtayxkaSxvAda aginxyanenxV tananx utapxtitxge kAraNavAgiruvudaneVnx I
puruSanu seVruvanu I riVtiyAgi loVkadalilx (shurxtimAgARnusAriyAda
vidAvxMsara vagaRdalilx) vayxvahAraviruvudariMda aginxgaLiMda puruSana
utapxtitxyu Aguvudu.
\end{artha}

\vishaya{gati nirUpaNe}

\begin{shl}
pacnacxmAyxmAhutAveVvaM punAMmonxV janamx kiVtiRtamf | \\
gatisatxsAyxtha vakatxvAyx udagadxkiSxNaBeVdataH \hfill|| 118 || 
\end{shl}

\begin{artha}
I riVtiyAgi aidane Ahutiyalilx puruSaneMbuvana jananavanunx
heVLidAdxyitu, inunx muMde utatxra dakiSxNaveMba BeVdadiMda avana
gatiyanunx heVLabeVkAgide.
\end{artha}

\begin{shl}
teV ya EvameVtadivxduyeVR cAmiV araNeyxV sharxdAdhxM satayxmupAsateV teV\s ciRraBisamaBxvanatxyXciRSoV\s harahanx ApUyaRmANapakaSxmApUyaRmANapakASxdAyxnaSxNAmxsAnudaknAknxditayx Eti mAseVBoyxV deVvaloVkaM deVvaloVkAdAditayxmAditAyxdevxYduyxtaM tAnevxYduyxtAnupxruSoV mAnasa Etayx barxhamxloVkAnagxmayati teV teVSu barxhamxloVkeVSu parAH parAvatoV vasanitx teVSAM na punarAvaqtitxH || 15 ||
\end{shl}

\vishaya{paMcAginx videyxyuLaLxvarige aciRrAdi mAgaR}

\vishaya{vAtiRka}

\vishaya{parxthama parxshenxya utatxra}

\begin{shl}
teV ya EtadayxthAjAtaM jAcnxnaM pacnAcxginxsaMsharxyamf | \\
viduraciRH samAyAnitx bahUketxVshacx divxjAtayaH \hfill|| 119 || 
\end{shl}

\begin{artha}
teV = avaru yAru hiVge paMcAginxyanunx avalaMbisida upAsaneyanunx
tiLidiruvaro avaru (teV ye) eMdu bahuvacanadiMda divxjaru enunxvaru
aciR deVvateyanunx hoMduvaru.
\end{artha}

\begin{shl}
utapxtitxsaMsithxtilayA yathoVkAtxgenxyXVkaheVtavaH | \\
itathxM yeV viduraciRsetxV saMBavanAtxyXtamxBAvitAH \hfill|| 120 || 
\end{shl}

\begin{artha}
`nananx huTuTx sithxti layagaLu hiMde heVLida paMcAginxgaLeMba
muKayxkAraNadiMda Aguvavu' eMdu yAru tiLidiruvaro, avaru tananxlilx
mADida anusaMdhAnavuLaLxvarAgi aciR eMba mAgaRvanunx hoMduvaru.
\end{artha}

\vishaya{yeV ca eMbalilx heVLida adhikArigaLanunx tiLisuvaru -}

\begin{shl}
iSATxpUtaRkaqtoV yeV veY gArxmasethxVBayxshacx yeV pareV | \\
araNayx iti gaqhayxnatx itareVSAM paqthagagxrXhAtf \hfill|| 121 || 
\end{shl}

\begin{artha}
iSaTx\footnote{iSaTx} pUtaR eMba kamaRgaLanunx yAru mADiruvaro
matutx yAru gArxmasathxrigU beVreyAgiruvaro avarU ililx `araNeyx' eMba
mAtiniMda tegedukoLaLxbeVku. itararanunx beVreyAgi
tegedukoMDiruvudariMda (avaru ililx gArxhayxralalx).
\end{artha}

\vishaya{araNayxdalilxruva vAnaparxsathxrige iSATxdi kamaRgaLu elilxMda baMdavu ? eMdare -}

\begin{shl}
yajacnxdAnatapAMsiVha gaqhasathx iva tApaseV | \\
gaqhasAthxcAyaRvAsAcacx na garxhoV barxhamxcAriNAmf \hfill|| 122 || 
\end{shl}

\begin{artha}
yajacnx, dAna, tapasusxgaLu ililx gaqhasathxnalilxruvaMte tApasanalUlx
irabahudu, gaqhasathxnalUlx AcAyaRnalUlx vAsamADuvudariMda ililx
barxhamxcArigaLanunx tegedukoLuLxvudilalx.
\end{artha}

\begin{shl}
nAraNayxsAthx na ca gArxmAyx apeVkaSxnetxV\s tarx vidayxyA | \\
sAmAnayxvacasoVpAtetxVnaR visheVSaparigarxhaH \hfill|| 123 || 
\end{shl}

\begin{artha}
araNayxdalilxruvavarU, gArxmadalilxruvavarU I paMcAginx videyxge
apeVkiSxtaralalx, `teV ya Eva meVtadivxduH' eMdu sAdhAraNavAda
padavanenxV ililx parxyoVgisiruvudariMda visheVSAthaRvanunx
tegedukoLuLxvudalalx.
\end{artha}

\vishaya{Ivarege `yeV cAmiVaraNeyxV' eMbalilx gaqhasathxranunx vAna
  parxsathxranunx tegedukoLaLx beVkeMdu pUvaRpakaSxvanunx toVrisidaru -
inunx muMde sidAdhxMtavAgi vAnaparxsathxranunx parama haMsaranunx
biTuTxLida yatigaLanunx tegedukoLuLxvudeMdu niNaRya mADalu
horaTidAdxre -}

\begin{shl}
sasharxdadhxsAyxpi satayxsayx yadayxpayxnayxtarx saMBavaH | \\
tathA\s pi yatayoV gArxhAyxH araNeyxVna visheVSaNAtf \hfill|| 124 || 
\end{shl}

\begin{artha}
sharxdedhxyiMda kUDida satayxvu satayxhiraNayxgaBoRV (pAsaneyu) beVre
gaqhasathxralUlx saMBavisabahudu, AdarU araNayx eMdu visheVSaNavanunx
koTiTxruvudariMda yatigaLanunx (matutx vAnaparxsathxranUnx) ililx
garxhisabeVkAdudu.
\end{artha}

\vishaya{beVre oMdu pakaSxvanunx toVrisutAtxre -}

\begin{shl}
teV ya EvaM viduriti yadi vA gaqhiNAM garxhaH | \\
aginxsaMbanadhxtoV nAyxyoyxV vanasathxsAyxpi saMBavAtf \hfill|| 125 || 
\end{shl}

\begin{artha}
athavA `teV ya EvaMviduH' eMdu gaqhasathxranunx tegedukoLuLxvadu aginx
saMbaMdhaviruvudariMda nAyxyavAgide, hAgU vAnaparxsathxranunx
tegedukoLuLxvudu, avaralilx aginx saMbaMdhavu saMBavisuvudariMda
yukatx.
\end{artha}

\begin{artha}
kelavaru paMcAginx dashaRnavu aginxhoVtarxda sutxtiyeMdu tiLidu adu
udidxSaTxvalalxveMdU adakekx saMbaMdhisidaMte I gatiyU
udidxSaTxvalalxveMdU BAvisi ililx utatxrAyaNAdi mAgaRvanunx hiDidavara
nirUpaNeyeV asAthxneV saMBarxma eMbaMte ayukatxvenunxtAtxre - ivara
matavanunx I muMde tirasakxrisutAtxre -
\end{artha}

\begin{shl}
na cAginxhoVtarxsheVSatavxmukatxdaqSeTxVriheVSayxteV | \\
sAmAneyxVna garxhAtatxsAyxH pacnAcxginxriti liknagxtaH \hfill|| 126 || 
\end{shl}

\begin{artha}
hiMdeV heVLida paMcAginx dashaRnavu aginx hoVtarxkekx aMgaveMbudu
iSaTxvAgilalx, paMcAginx daqSiTxyanunx sAmAnayxvAgiyeV
tegedukoMDiruvudariMdalU `paMcAginxVnfveVda' eMdu paMcAginxgaLeMdu
tiLidu baMdiruvudariMdalU (inonxMdakekx aMgavAguvudilalx).
\end{artha}

\begin{shl}
tirxloVkiVsAdhanatAyxgAnAnxpi saMnAyxsinoV garxhaH | \\
deYviV vidAyx hi vitatxM sAyxdivxtAtxcacx vuyxtithxtiyaRtaH \hfill|| 127 || 
\end{shl}

\begin{artha}
matutx mUru loVkagaLa sAdhanegaLanunx tayxjisidadxriMda ililx
saMnAyxsigaLige garxhaNavilalx, matutx deVvatoVpAsaneyeV vitatx I
vitatxvanunx tayxjisi I sanAyxsigaLu niMtavaru eMbuvudariMdalU
saMnAyxsigaLanunx tegedukoLaLxbAradu.
\end{artha}

\vishaya{yatigaLu loVkatarxya sAdhanavanunx tayxjisuvareMbudakekx
  parxmANa -}

\begin{shl}
parxjayA kiM kariSAyxma AkeSxVpoV barxhamxveVdanAtf \hfill|| 128 | \\
pacnAcxginxjAcnxnavadABxvAyx gatirapayxtarx savaRdA | \\
yatoV\s toV gatirapuyxkAtx nAnayxthA tadudiVraNamf \hfill|| 129 || 
\end{shl}

\begin{artha}
`kiM parxjayA kariSAyxmoV yeVSAMnoV\s ya mAtAmxyaM loVkaH' eMdu barxhamxjAcnxnadiMda parxjA (putarxveMba) sAdhanavanunx  nirAkarisiruvu kaMDide. (adariMda loVkatarxyasAdhanavanunx  tayxjisuvudu) Adare paMcAginxvideyxyaMte ililx adara gati  (mAgaR)yanunx yAvAgalU ciMtisabeVku, adariMdaleV gatiyanUnx  heVLiruvudu beVre aBipArxyadalilx adanunx heVLidadxlalx.
\end{artha}

\vishaya{aciR padadiMda jAvxleyeV gArxhayxvalalx.}

\begin{shl}
deVvatoVpAsanaseyxVha parxkaqtatAvxtatxthA gateVH | \\
aciRHshabedxVna teVneVha deVvateYvAtarx gaqhayxteV \hfill|| 130 || 
\end{shl}

\begin{artha}
ililx deVvatoVpAsaneyeV parxkaqtavAgiruvudariMda hAgU parxkaqta
upAsaneya samiVpadalelxV gatiyanunx heVLiruvudariMda `aciRH' eMba
padadiMda adara aBimAni deVvateyeV ililx garxhisalapxTiTxde. 
\end{artha}

\footnotetext[1]{ililx deVvateya Ekatavxvanunx  hoMdutAtxreMbudanunx heVLidudx heVge eMbudanunx tiLiyabeVku,  vidAvxMsaru vimashiRsabeVku, BASayxdalilx hiVgilalx.}
\begin{shl}
aciRdeVRvatayeYkatavxM pArxpAyxhadeVRteYkatAmf\footnotemark[1] | \\
saMBavanitxVti savaRtarx saMbanodhxV\s tArxnuSajayxteV \hfill|| 131 || 
\end{shl}

\begin{artha}
aciRdeRVvateyoDane Ekatavxvanunx hoMdi ahadeRVvateyoDane Ekatavxvanunx
hoMduvareMdu `saMBavanitx' eMbudanunx muMde elalx kaDeyalUlx anavxya
mADalu joVDisikoLaLxbeVku.
\end{artha}

\vishaya{`aciRSoV\s haH' - eMbudara athaR -}

\footnotetext[2]{aciRrAdi padagaLiMda aciRrAdi aBimAniyAda deVvateyanunx  tegedukoLaLxdidadxre rAtirx kAladalilx maraNahoMduvavarige  hagalanunx heVLuva ahaHsisxna saMbaMdhavu Agade hoVguvudu,  ahasisxnalelxV maraNavanunx hoMduvareMbudu niyamavilalx, rAtirxyU  hoVguvaru, ahasasxnunx ivaru niriVkiSxsuvudU ilalx, shukalx pakaSxvU  niyatavalalx, utatxrAyaNavU niyatavalalx, adariMda ililx ahasusx  shukalxpakaSx utatxrAyaNa muMtAda kAlavAcakashabadxgaLiMda  aciRdeVvateyaMte deVvategaLeV namage  gArxhayxvAgive. `AtivAhikAsatxlilxMgAtf' iti sUtarx nAyxyadiMdalU  deVvateyeV gArxhayxvAgide. shariVravanunx biTuTx meVlakekx hoVguva  kamiR jiVvigaLige dAritoVrisuvavaru yAru, aciRrAdi shabadxgaLu  aceVtanavAgiruvalilx avugaLu jaDavAdadxriMdaleV toVrisalAravu savxtaH  jiVvarige IshariVravanunx biTuTx matotxMdu shariVradalilx  seVrikoLuLxva payaRMta I aMtarALadalilx liMga shariVradiMda  kUDidadxrU jAcnxnavAguvaMtilalx, adariMda toVrisuvavU jaDavAgi  hoVguva jiVvarU mUDharAgidadxre loVkAMtarakekx hoVguva dAriyu kANade  hoVguvudu, adariMda aciRrAdi shabadxgaLu aBimAnideVvategaLe, ivaru  AtivAhakaru, I shariVravanunx biTaTx kUDale kamaR upAsanegaLa  baladiMda hoVguva jiVvigaLige `ililxge hoVgu' ililxMda muMdakekx  hoVgu eMdu dAri toVrisutAtx jiVvigaLanunx mUMde muMde oyuyxva  deVvategaLu AtivAhakareMdu sUtarxkAraru sidAdhxMta  paDisidAdxre. adaraMte utatxrAyaNa mAgaRdalilxruva aciRrAdigaLu  AtivAhaka deVvategaLu, hiVgeyeV dakiSxNa mAgaRdalilxruva dhUmAdigaLU  deVvategaLeV eMdu tiLiyabeVku. mAgaRdashiRgaLanunx mAgaRveMdu  aupacArikavAgi vayxvaharisuvaMte ililxyU vayxvahariside. eMdu  BAvAthaR.}
\begin{shl}
\footnotemark[2]gaqhayxteV deVvatA noV ceVdaciRrAdigirA tadA | \\
asaMBavoV\s nayxtarx maqteVritayxpAthAR gatishurxtiH \hfill|| 132 || 
\end{shl}

%%% footnote shloka
\begin{artha}
aciRrAdi shabadxgaLiMda deVvateyanunx tegedukoLaLxdidadxreAvAga upAsakanige Palavu saMBavisuvudilalx, beVre kAladalUlx (kaqSaNxpakASxdigaLalUlx maraNavu AguvudariMda gati shurxtiyu athaRshUnayxvAguvudu.)
\end{artha}

\begin{artha}
BiVSaru maraNakAkxgi utatxrAyaNa kAlavanunx niriVkiSxsidudx
itihAsadalilxruvAga aharAdi shabadxvanunx kAlavAcakagaLeMde ? Eke heVLabAradu ? eMdare-
\end{artha}

\begin{shl}
yatUtxdagayanApeVkASx BiVSamxsayx shUrxyateV samxqqtw | \\
satayxvAditavxsidadhxyXthaRM shaMtanoVsatxdapiVkaSxyXtAmf \hfill|| 133 || 
\end{shl}

\begin{artha}
BiVSamxnige utatxrAyaNa niriVkeSxyidadxdudx samxtiyalilx(BAratadalilx) sapxSaTxvAgide yeMbudU saha shaMtanuvu (`niVnusavxcaCxnadx maqtuyxvuLaLxvanAgu' eMdu AshiVvaRdisidaMte) satayxvAdiyeMdu parxsididhxpaDisalu baMdiruvudeMdu\footnote{shurxti  samxqqtigaLalilx mAgaRpavaRvAgi kAla visheVSavaneVnx heVLuvudeMdu oMdeV  aBipArxyavidadxlilx maraNakAlavu niyatavalalxdadxriMda rAtirx dakiSxNAyana itAyxdikAladalilx hoVdaro ahasusx, utatxrAyaNa itAyxdi  kAla niriVkeSx mADabeVkAda parxsaMga baruvudu, hAge yAru  niriVkiSxsade  rAtirxyalolx hagalalolx, dakiSxNAyanadalolx,  utatxrAyaNadalolx sAyuvaraSeTx, ``atashAcxyaneV\s pidakiSxNeV" eMdu  sUtarxkArarU idaneVnx heVLutAtx kAlavisheVSavu  udidxSaTxvAgilalxvenunxtAtxre, adariMda kAladeVvatA vAcakagaLeMdeV  ahaH, ApUyaR mANapakaSx, utatxrAyaNada Aru mAsagaLu, saMvatasxra ivu  niNaRyavAgive. mahABAratadalilx} ililx tiLiyabeVku.
\end{artha}

\begin{shl}
anayxthA kaqtanAshaH sAyxdakaqtABAyxgamasatxthA | \\
deVvatAgarxhaNAtatxsAmxdedxVvateYvAciRrucayxteV \hfill|| 134 || 
\end{shl}

\begin{artha}
beVre athaRvanunx iTaTxre (gatiyilalxda jiVvarugaLige) hiMde mADida
puNayxgaLu nAshavAguvavu, matutx Iga mADadiruva kamaRgaLa Palavu
barabeVkAguvudu, adariMda deVvateyaneVnx tegedukoMDiruvudariMda ililx
aciR eMdu deVvateyeV heVLalapxTiTxde.
\end{artha}

\vishaya{`teVya'... itAyxdi maMtArxthaRda upasaMhAra -}

\begin{shl}
tasAmxdeVvaMvidoV dhiVrA aciRrAdayxBimAniniVmf | \\
karxmeVNa deVvatAM yAtAvx veYricnacxM yAnitx tatapxdamf \hfill|| 135 || 
\end{shl}

\begin{artha}
adariMa I riVti paMcAginxvideyxyanunx tiLida vidAvxMsaru (upAsakaru)
aciR ! modalAda aBimAnideVvateyanunx paDedu karxmavAgi barxhamxna
padaviyanunx seVrutAtxre.
\end{artha}

\vishaya{yeV ca - itAyxdi maMtArxthaRda upasaMhAra}

\begin{shl}
sharxdadxdhAnAsatxpasayxnatxH satayxM barxhamx samAhitAH | \\
upAsateV bahigArxRmAdaciRsetxV saMBavanatxyXtaH \hfill|| 136 || 
\end{shl}

\begin{artha}
sharxdedhxyuLaLxvarAgi tapasusx mADuvavaru, samAdhiyalilxrutAtx
gArxmada horage satayxbarxhamxvanunx (aparabarxhamxvanunx) upAsane
mADuvavarU elalxrU I aciR deVvateyanunx hoMduvaru.
\end{artha}

\vishaya{vAkayxkekx matotxMdu athaR  -}

\begin{shl}
barxhamxNaH sAthxnamAyAnitx yadAvx sAvxsharxmakamaRBiH | \\
saMnAyxsAdabxrXhamxNaH sAthxnaM tathAca samxqqtishAsanamf \hfill|| 137 || 
\end{shl}

\begin{artha}
athavA tamamx Asharxma dhamaRgaLiMda kUDi saMnAyxsa mADidadxriMda
barxhamxna sAthxnavanunx hoMduvaru `sanAyxsAdf barxhamxNasAthxnaM'- eMbuva sumxqqtiya
shAsanavU \footnote{modalu satayx shabadxkekx sUtArxtamxneMba
  athaRvanunx iTuTxkoMDu satayxvideyxyiMda Aguva Palavanunx heVLidaru,
Iga satayxvanunx heVLuvudeMbuva muKayxvAda AsharxmadhamaRvanunx
avalaMbisi yamaniyamAdi sAvxsharxmadhamaRgaLanunx AcarisuvudariMa
keVvala saMnAyxsadiMdalU utatxmAdhikArigaLU hiraNayxgaBaRna
sAthxnavanunx paDeyuvareMdu I vAtiRkada aBipArxya.}iruvudu.
\end{artha}

\vishaya{vAtiRka}

\vishaya{`aciR SoV\s ha''- itAyxdi maMtarx vAyxKAyxna}

\begin{shl}
tatoV\s hadeVRvatAM yAnitx shukalxpakaSxM tataH karxmAtf | \\
SaNAmxsAMshacx tatoV yAnitx huyxtatxrAyaNalakaSxNAnf \hfill|| 138 || 
\end{shl}

\begin{shl}
deVvatAM ca tatoV yAnitx deVvaloVkABimAniniVmf | \\
tata AditayxmAyAnitx veYduyxtaM cApi BAsakxrAtf \hfill|| 139 || 
\end{shl}

\begin{shl}
barxhamxNA manasA saqSoTxV mAnasaH puruSasatxtaH | \\
barxhamxloVkAnasx nayati soV\s payxBeyxVtAyxtha veYduyxtAnf \hfill|| 140 || 
\end{shl}

\begin{shl}
teV teVSu barxhamxloVkeVSu diVpayxmAnAH parAH samAH | \\
bArxhamxmAnAH samA gArxhAyx barxhamxloVkeVSu tacuCxrXteVH \hfill|| 141 || 
\end{shl}

\begin{artha}
A aciR deVvateyiMda ahadeRVvateyalilxge barutAtxre anaMtara alilxMdaAru utatxrAyaNada mAsadeVvategaLanunx seVruvaru anaMtaradeVvaloVkABimAniyAda deVvateyanunx hoMduvaru alilxMda AditayxdeVvateyanunx A nitayxniMda viduyxtf deVvateyanunx hoMduvaru alilxMdaAcege barxhamxna manaH saMkalapxdiMda saqSiTxyAda (amAnava) mAnasapuruSanu baMdu viduyxdedxVvateya adhiVnadalilxdadx I upAsakaranunxbarxhamx loVkakekx oyuyxvanu. avaru barxhamxloVkagaLalilxparxkAshisutAtx utatxmavAda saMvatasxragaLa payaRMta alilxyeV vAsamADuvaru, ililx A saMvatasxragaLu yAvudeMdare  barxhamxna mAnarUpavAgi tegedukoLaLxbeVku. EkeMdare ! barxhamxloVkadalilxnavatasxraveMdu heVLiruvudariMda adanenxV tegedukoLaLxbeVku.
\end{artha}

\vishaya{satayx loVkavu oMdAgidadxrU `barxhamxloVkeVSu' eMdu bahuvacanavanunx EtakAkxgi parxyoVgiside ? eMdare .}

\begin{shl}
sASiTxRsAloVkayxsAyujayxvayxpeVkASx bahugiVriyamf | \\
samaSiTxvayxSiTxBeVdaM vA vayxpeVkaSxyX bahuvAgiyamf \hfill|| 142 || 
\end{shl}

\begin{artha}
sASiTxR - sAloVkayx - sAyujayxveMba mukitxgaLa daqSiTxyiMda
bahuvacanavanunx parxyoVgiside, athavA samaSiTx (sAmAnayx) vayxSiTx
(visheVSa)gaLa BeVdavanunx iTuTxkoMDu I bahu vacanavu baMdide.
\end{artha}

\vishaya{`teVSAM na punarAvaqtitx :-' eMbudara athaR -}

\begin{shl}
AvaqtitxnaR punasetxVSAM yAvadABUtasaMpalxvamf | \\
ABUtasaMpalxvasAthxnamamaqtatavxM hi BASayxteV \hfill|| 143 || 
\end{shl}

\begin{artha}
avarige samasatx pArxNigaLu layavAguvavaregU punaH Avaqtitxyilalx
(saMsAra maMDalakekx hiMtirugi baruvudilalx), I bageyalilx
amaqtatavxveMbudanunx ``ABUta saMpalxvaM sAthxna mamaqtatavxM
hi BASayxteV'' eMdu (samxqqtikAraru heVLuvaru).
\end{artha}

\vishaya{athavA anAvaqtitx shurxtige muKAyxthaRvU irabahudu, heVge ? -}

\begin{shl}
aikAtamxyXdhiVsamutapxtetxVH yadivA barxhamxloVkataH | \\
mucayxnetxV na nivataRnetxV yathA dhUmAdimAgaRgAH \hfill|| 144 || 
\end{shl}

\begin{artha}
athavA alilxyeV EkAtamxsavxrUpajAcnxnavu AguvudariMda (aMtayxdalilx)
barxhamx loVkadalelxV mukatxrAgutAtxre. Adare dhUmAdi mAgaRvanunx
hoMdida kamiRgaLaMte hiMtiruguvudilalx.
\end{artha}

\vishaya{I divxtiVya pakaSxdalilx heVLuva athaRkekx gamakaveVnu ? eMdare -}

\begin{shl}
imaM mAnavamAvataRmitAyxdayxsayx visheVSaNAtf | \\
aterxYva kalepxV\s nAvaqtitxnaR tavxneyxVSavxnivAraNAtf \hfill|| 145 || 
\end{shl}

\begin{artha}
`imaM mAnava mAvataRmf' eMdu anAvatiR viSayadalilx visheVSaNavu  kANuvudariMda I kalapxdalelxV punarAvaqtitxyilalx, Adare beVre  kalapxgaLalilx anAvaqtitxyu ilalx (AvaqtitxyuMTu) Eke ? taDeyuvudakekx  yAvudU ilalxvAdadxriMda.
\end{artha}

\section*{baq 6 - 2 - 16 kaMDike.}

\begin{shl}
atha yeV yajecnxVna dAneVna tapasA loVkAcnajxyanitx teV dhUmamaBisamaBxvanitx dhUmAdArxtirxM rAterxVrapakiSxVyamANapakaSxmapakiSxVyamANapakASxdAyxnaSxNAmxsAnadxkiSxNAditayx Eti mAseVBayxH pitaqloVkaM pitaqloVkAcacxnadxrXM teV canadxrXM pArxpAyxnanxM Bavanitx tAMsatxtarx deVvA yathA soVmaM rAjAnamApAyxyasAvxpakiSxVyasevxVteyxVvameVnAMsatxtarx BakaSxyanitx teVSAM yadA tatapxyaRveYtayxtheVmameVvAkAshamaBiniSapxdayxnatx AkAshAdAvxyuM vAyoVvaqRSiTxM vaqSeTxVH paqthiviVM teV paqthiviVM pArxpAyxnanxM Bavanitx teV punaH puruSAgwnx hUyanetxV tatoV yoVSAgwnx jAyanetxV loVkAnapxrXtuyxtAthxyinasayx EvameVvAnuparivataRnetxV\s tha ya Etw panAthxnw na vidusetxV kiVTAH pataknAgx yadidaM danadxshUkamf || 16 ||
\end{shl}

\begin{shl}
deVvayAnaH samAseVna panAthx yatAnxtapxrXpacnicxtaH | \\
vAyxKAyx\s tha pitaqyANasayx samayxgAraBayxteV\s dhunA \hfill|| 146 || 
\end{shl}

\begin{artha}
deVvayAna mAgaRvanunx saMkeSxVpagarxMthadiMda parxyatanxdiMda
visAtxravAgi iLisidAdxyitu. Iga pitaqyANa mAgaRvanunx cenAnxgi
vAyxKAyxnisalu AraMBiside.
\end{artha}

\begin{shl}
deVvAdijAcnxnahiVneVna yeV\s tha yajecnxVna sadidxvXjAH | \\
loVkAcnajxyanitx dAneVna sadedxVshAdimatA tathA \hfill|| 147 || 
\end{shl}

\begin{shl}
niHsheVSakalamxSadhavxMsitapasA vA\s vipashicxtaH | \\
maqtAsetxV dhUmamAyAnitx dhUmAdArxtirxM tamasivxnaH \hfill|| 148 || 
\end{shl}

\begin{shl}
rAterxVraparapakaSxM ca dakiSxNAyanadeVvatAmf | \\
mAseVBayxH pitaqloVkaM hi pitaroV yatarx sheVrateV | \\
pitaqloVkAcacx teV canadxrXM yAnatxyXnanxM tadidxvwkasAmf \hfill|| 149 || 
\end{shl}

\begin{artha}
yAru deVvatA jAcnxnavilalxde keVvala yajacnxdiMda loVkagaLanunx
jayisuvaro hAgU satfpAtarx, deVsha modalAdavugaLiMda kUDida
dAnadiMdalU jayasuvaro, A sadivxjarU athavA niHsheVSavAgi
kalamxSavanunx nAshagoLisuva tapasisxniMdalo avidAvxMsarAdavaru maraNa
hoMdidavarAgi, dhUmadeVvateyanunx hoMduvaru, dhUmadeVvateyiMda
rAtirxdeVvateyanunx paDeyuvaru, rAtirx deVvateyiMda apara pakaSx
deVvateyanunx, alilxMda dakiSxNAyana deVvateyanunx, aMdare
mAsadeVvategaLanunx hoMduvaru, mAsa deVvategaLiMda muMde
pitaqloVkavanunx seVruvaru, yAva loVkadalilx pitaqgaLu nelasuvaro
(A loVkaveV pitaqloVkavu) pitaqloVkadiMda avaru caMdarxnanunx hoMdi
deVvategaLige ananxvAguvaru.
\end{artha}

\vishaya{`teV canadxrXM' pArxpayx itAyxdi maMtarxda athaR -}

\begin{shl}
canadxrXM pArxpAyxnanxBAvaM ca teV yatasatxtarx saMsithxtAH | \\
asakaqdaBxkaSxyanetxyXVtAnApAyxyAyx\s \s pAyxyayx soVmavatf \hfill|| 150 || 
\end{shl}

\begin{artha}
avaru caMdarxnanunx hoMdi ananx savxrUpavanunx paDedidadxriMda
alilxyeV idadx deVvategaLu I kamiRgaLanunx yajacnxdalilx
soVmarasavanunx pAna mADidaMte capapxrisi capapxrisi yAvAgalU
BakiSxsuvaru.
\end{artha}

\begin{shl}
pakeSxV shukelxV tamApAyxyayx kaqSeNxV taM BakaSxyanatxyXtha \hfill|| 151 | \\
BoVgasAdhanaBAvAshacx BakaSxyanitxVti BaNayxteV | \\
na tavxBayxvahaqtinAyxRyAyx veVdavatamxRni tiSaThxtAmf \hfill|| 152 || 
\end{shl}

\begin{artha}
shukalxpakaSxdalilx A caMdarxnanunx tuMbikoMDu kaqSaNxpakaSxdalilxavananunx deVvategaLu BakiSxsuvaru, `parxkaqta BakaSxyanitx'eMbudariMda BoVga sAdhanagaLanAnxgi mADikoLuLxvareMdu heVLide, BakaSxNaveMdare tinunxvudeMdalalx, yukatxvU alalx, EkeMdare! veVdamAgaRdalilxruva kamiRgaLanunx tiMdu hAkuvareMbudu\footnote[1]{satatxvX guNaveV adhikavAgiruva ahiMsAdi sAdhanagaLalilx parxvaqtatxrAgiruva  janareV veVdamAgaRdalilxruvavaru, iMthakamaRTharanunx deVvategaLu  savxgaRkekx baMdoDane tiMdu hAkuvudu satayxvAdare anathaRveV Aguvudu  savxgARBilASigaLige kamaR sAdhanegaLalilx parxvaqtitxyuMTAgalArdu  adariMda ivaralilx ananxveMdu shabadx parxyoVgamADidudx gwNa,  BakaSxNavU gwNAthaR, aMdare I kamiRgaLanunx deVvategaLu  upayoVgisikoMDu BoVgapaDuvareMdeV athaR, rAjaru (rANiyarana)  Baqtayxranunx seVvA mUlaka upayoVgisikoMDu BoVgapaDuvaMte, Adare  ivarige BoVgaveV ilalxveMdalalx, Baqtayxrige iruvaMte ivarigU  deVvategaLa sahavAsadalilx suKavU ide. adariMda ananxveMdare BoVga  sAdhanavAda BoVgayx vasutx eMdeV gwNAthaR, ``BAkatxMvA\s  nAtamxvitAvxtf `` eMba sUtarxdalilx niNaRyamADide.} nAyxya badadhxvAgilalx.
\end{artha}

\vishaya{`teVSAMyadA tatapxyaR veYti' - - - eMbudara athaR -}

\begin{shl}
yadA tUpacitaM kamaR payaRvasayxti BoVgataH | \\
teVSAmatheYtamadhAvxnamAvataRnetxV yathAgatamf \hfill|| 153 || 
\end{shl}

\footnotetext[2]{ililx aBoVginaH eMbudakekx BoVgavilalxdaveMdathaR batatx  muMtAda dhAnayxgaLalilx baMdu seVradavarige sasigaLige Aguva hiVna  janamxda duHKAnuBavavu ivarige ilalxveMde tiLisalu  mAnavAdishariVradalilx huTuTxvavaregU ivarige BoVgavilalxveMdu  tiLisalu I padavanunx parxyoVgisiruvaru batatx muMtAda sasigaLalilx  baMdu seVruvaru aSeTx. ``anAyxdhiSiTxteV pUvaRvadaBilApAtf'' eMba  sUtarxdalilx idara niNaRyavide. (barx. sU...) dhAnAyxdi rUpadalilx  baMdu seVrikoMDiruva I kamiRgaLu I jAgavanunx bahaLa kaSaTxdiMda  shariVradalilx oMdu seVruvaru anaMtara kamARnusAravAgi sitxrXV  puruSa davxMdavxdiMda nAnA bageyAgi huTuTxvareMdu BAvAthaR.}
\begin{shl}
AkAshamanuniSapxdayx vAyumAyAnatxyX\footnotemark[2]BoVginaH | \\
vAyoVvaqRSiTxmavApAyxtha vaqSeTxVshacx paqthiviVM tataH | \\
ananxBAvaM paqthivAyx\s tasetxV samAyAnitx BUmigAH \hfill|| 154 || 
\end{shl}

\begin{shl}
GaTiVyanatxrXvadashArxnAtx EvameVva punaH punaH | \\
parivataRnitx saMsAreV kamaRvAyusamiVritAH \hfill|| 155 || 
\end{shl}

%%% footnote shloka
\begin{artha}
yAvAga paripakavxvAda kamaRvu BoVgadiMda samApitxgoLuLxvudo AvAga avaru anaMtara heVge hoVgidadxro hAgeye I mAgaRkekx iLiyutAtxre hiMdirugutAtxre. heVgeMdare ? baruvAga AkAshavanunx hoMdi anaMtaravAyuvidadxlilxge baruvaru, vAyuviniMda vaqSiTxyanunx hoMdi adaroMdigeseVri alilxMda vaqSiTx mUlaka BUmige baruvaru, BUmige baMdavaru (naDuveBoVgavilalxdavarAgi) virxVhiyavAdi dhAnayxdoMdige seVrikoLuLxvaruhiVgeye rATeyaMte savxlapxvU  vishArxMtiyilalxde pade padeV IsaMsAradalilx kamaRveMba vAyuviniMda taLaLxlapxTaTxvarAgi sututxtatxleV irutAtxre.
\end{artha}

\vishaya{atha itAyxdi maMtarx BAgavanunx pUvaRsaMbaMdhavanunx tiLisi vAyxKAyxnisuvudu -}

\begin{shl}
dakiSxNasayx pathoV vAyxKAyx yathAvadanuvaNiRtA \hfill|| 156 | \\
yathoVkatxlakaSxNw yeV tu panAthxnAvutatxreVtArw | \\
na vidusetxV BavanitxVha kiVTAdAyx duHKaBoVginaH \hfill|| 157 || 
\end{shl}

\begin{artha}
dakiSxNa mAgaRvanunx idadxMteye vAyxKAyxna mADidAdxyitu hiMdeheVLida lakaSxNavuLaLx utatxra matutx dakiSxNa mAgaRgaLanunx yAvajiVvaru paDeyuvudilalxvo avaru ililx kiVTAdirUpavuLaLxvugaLAgiduHKakekx pAtarxrAguvaru.
\end{artha}

\vishaya{kAraNaveVneMdare ? -}

\begin{shl}
nA\s \s dirxyanatx idaM jAcnxnamudaknABxgARpitxsAdhanamf | \\
kamaR vA pitaqyANApwtx teV kiVTAdAyxmiyugaRtimf \hfill|| 158 || 
\end{shl}

\begin{artha}
utatxra mAgaR lABakekx sAdhanavAda I jAcnxnavanunx (upAsaneyanunx)yAru Adarisuvudilalxvo, hAgU pitaqyANa mAgaRvu laBisuvudakUkxsAdhanavAda kamaRvanunx apeVkiSxsuvudilalxvo, A jiVvigaLukirxmikiVTAdi rUpavAda dugaRtiyanunx paDeyuvaru.
\end{artha}

\begin{shl}
goVmayAduyxdaBxvAH kiVTAH pataknAgxH shalaBAsatxthA | \\
daMshAshacx mashakAshecxYva danadxshUkAH sapananxgAH \hfill|| 159 || 
\end{shl}

\begin{artha}
goVmaya modalAda kashamxla parxdeVshadalilx huTuTxva huLugaLU, diVpada meVle eraguva shalaBagaLU, pataMgagaLU, kaDiyuva noNagaLU, hAvu muMtAda viSajaMtugaLU eMdathaR.
\end{artha}

\vishaya{aidu parxshenxgaLanunx nAyxyavAgi niNaRyisabeVkAdadudx adara niNaRyavanunx mADade ideVnu akAMDatAMDavanunx mADide ? eMdare \mdash }

\begin{shl}
yatithAyxmiti yaH parxshanxH sa puMjanomxVkitxtoV gataH | \\
tadananatxrameVvoVkAtx vAyxvaqtitxshacx pathoVdavxRyoVH \hfill|| 160 || 
\end{shl}

\begin{artha}
`yatithAyxmf' eMdu yAva parxshenxyididxto adanunx puruSana janamxvanunx heVLidadxriMdaleV mugisidAdxyitu, adara naMtaraveV eraDu mAgaRgaLa parxshenxgaLigU avugaLanunx beVpaRDisi utatxrisidAdxyitu.
\end{artha}

\begin{shl}
AkAshAdayxBisaMBUtAyx punarAvataRnaM gatamf | \\
deVvayAnaM ca yatakxqqtAvx pitaqyANaM ca laBayxteV \hfill|| 161 || 
\end{shl}

\begin{shl}
sAdhanaM tadapiVhoVkatxM jAcnxnakamaRsavxlakaSxNamf | \\
loVkasAyxpUraNeV heVtuH samApAtxvuditaH suPxTaH \hfill|| 162 || 
\end{shl}

\begin{artha}
punarAvaqtitxyu heVge ? eMba eraDane parxshenxyu AkAshAdi mAgaRvanunx hoMduvareMdu heVLidadxriMdaleV mugiyitu. yAvudanunx mADi deVvayAna mAgaRvu laBisuvudu ? hAgU pitaqyANa mAgaRvu laBisuvudu ? eMba aidane parxshenxgU jAcnxna (upAsane) matutx kamaRveMbuva sAdhanagaLeV eMdu utatxriside. paraloVkavu pUNaRvAgadiralu kAraNaveVnu ? mUrane parxshenxyanunx samApitxyalilx bagehariside.
\end{artha}

\vishaya{adu heVge ? eMdare \mdash }

\begin{shl}
kiVTAdiVnAmagamanAdagxtAnAM cA\s \s gateVsatxthA | \\
EvaM pacnAcxpi niNiVRtAH parxshAnx yeV pArxkapxrXcoVditAH \hfill|| 163 || 
\end{shl}

\begin{artha}
kirxmi modalAdavugaLu savxgaRloVkakekx hoVgade iruvudariMdalU alilxge hoVda jiVvarugaLu punaH hiMdakekx baruvudariMdalU savxgaRloVkavu BatiRyAgade hAgeyeV iruvudeMdu tiLiyabeVku. I riVtiyAgi hiMde keVLida aidu parxshenxgaLanunx niNaRya mADidAdxyitu.
\end{artha}

\vishaya{sholxVkagaLa saMKeyx oTuTx 10936 ||}

\begin{center}
{\bf iti shirxVbaqhadAraNayxkoVpaniSadf BASayxda vAtiRkadalilx}
\smallskip

{\bf Arane adhAyxyadalilx eraDane bArxhamxNavu}
\smallskip

{\bf pUNaRgoMDitu.}

\smallskip
{\bf || shirxVdakiSxNAmUtaRyeV namaH ||}
\end{center}
