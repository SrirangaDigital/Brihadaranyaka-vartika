%%%% From 055.tex

%~ \begin{center}
%~ \section*{sureVshavxravAtiRka}
%~ {\Large\textbf{adhAyxya - 6,   bArxhamxNa - 1}}
%~ \medskip
%~ \end{center}

\chapter{bArxhamxNa - 1}

\begin{shl}
samApatxH sapatxmoV\s dhAyxyaH pArxpAtxvasara ucayxteV | \\
aSaTxmaH KilakANeDxV\s simxnUpxvaRkANeDxVSavxnukitxtaH \hfill|| 1 || 
\end{shl}

\begin{artha} 
\footnote[1]{OMkAra, damAdi sAdhana, barxhamx, barxhemxVtara 
upAsanegaLu, adara PalagaLu, adakAkxgi heVLida mAgaRgati, AditAyxdi 
deVvateya upasAthxna - ivugaLanunx heVLidAdxyitu. Iga idaralilx 
barxhamxBinanxvAda vasUtxpAsanegaLanunx parxdhAnavAgi heVLuvudu. adara 
PalagaLanunx shirxVmaMthAdi kamaRvanunx hiMde heVLade iruvudanunx heVLalu 
eMTaneV adhAyxyavu baMdide. barxhomxVpAsaneya vicAravu muKayxvAgi 
mugida naMtara barxhemxVtara upAsaneya vicAravanunx pArxdhAnavAgi 
mADalu beVre adhAyxyavu baMdide. parxdhAna vicAravanunx mADade 
aparxdhAna vicAravanunx mADuvaMtilalx, saMBavisuvudU ilalx.}ELaneV adhAyxyavu mugiyitu. Iga (avakAshavu 
laBisiruvudariMda) pUvaRkAMDagaLalilx heVLadeyiruvudariMda Iga 
avakAshavu odagidadxriMda I KilakAMDadalilx eMTaneV adhAyxyavanunx 
heVLuvudu.
\end{artha}

\vishaya{Iga modalane bArxhamxNakekx avAMtara saMbaMdhavanunx heVLalu 
AraMBiside {\rm --}}

\begin{shl}
gAyatArxyXH pArxNaBAvoVkitxH kasAmxdedhxVtoVH puroVditA | \\
na tu vAgAdiBAvoV\s sAyxsatxtarx heVturihoVcayxteV \hfill|| 2 || 
\end{shl}

\begin{artha} 
gAyatirxyanunx pArxNaveV eMdu yAva kAraNadiMda hiMde heVLidudx? AdarU 
I gAyatirxyu vAgAdi savxrUpaveMdu heVLalilalx. idakekx 
kAraNaveVneMbudanunx I bArxhamxNadalilx heVLuvudu.
\end{artha}

\footnotetext[2]{hiMde heVLida pArxNoVpAsaneyu anayxsheVSavAgidadxre 
ukAthxdi upAsanegaLige sheVSavo (aMgavo) athavA gAyatirxV upAsanege 
aMgavo athavA maMthanakamaRkekx aMgavo eMdu vikalapxmADi 
utatxriside. vAgAdiiMdirxyagaLu idadxrU avugaLanunx biTuTx ukAthxdiguNa vishiSaTxvAda pArxNoVpAsaneyaneVnx heVLide. adakekx kAraNavAgi jeyxVSaThxtAvxdiguNavanunx heVLide. I upAsaneyu ukAthxdi upAsanege 
aMgavAgilalx. hiMdina garxMthavAda naMtara I garxMthavu EtakAkxgi 
baMdideyeMbudanunx hoMdisalu jeyxVSaThxtAvxdigaLanunx vaNiRside. adu 
pArxNavoMde upAsayxveMbudakekx kAraNa. guNaBeVdaviruvudariMdalU 
PalaBeVdaviruvudariMdalU pArxNoVpAsaneyu ukothxVpAsanege 
aMgavAguvudilalxveMdu tAtapxyaR.}
\begin{shl}
\footnotemark[2]jeyxVSaThxH sherxVSoThxV yataH pArxNoV na tu vAgAdayasatxtaH | \\
pArxNAtamxBAva EvoVkatx AnanatxyARthaRmeVva tu \hfill|| 3 || 
\end{shl}

%% shloka footnote
\begin{artha} 
yAvudariMda pArxNaveV jeyxVSaThx matutx sherxVSaThxvo adariMda matutx 
vAgAdi iMdirxyagaLu sherxVSaThxvU jeyxVSaThxvU alalxvo adariMda 
gAyatirxyu pArxNasavxrUpaveMdeV heVLide. matutx garxMthada AnaMtara 
saMbaMdhavanunx heVLuvudakAkxgiyU heVLalapxTiTxde.
\end{artha}

\section*{baq. a.6, bArx. 1, kaMDike 1}

\begin{shl}
OM yoV ha veY jeyxVSaThxM ca sherxVSaThxM ca veVda jeyxVSaThxshacx sherxVSaThxshacx sAvxnAM Bavati pArxNoV veY jeyxVSaThxshacx sherxVSaThxshacx jeyxVSaThxshacx sherxVSaThxshacx sAvxnAM Bavatayxpi ca yeVSAM buBUSati ya EvaM veVda || 1 ||
\end{shl}

\begin{shl}
yoV ha veY vasiSAThxM veVda vasiSaThxH sAvxnAM Bavati vAgevxY vasiSAThx vasiSaThxH sAvxnAM Bavatayxpi ca yeVSAM buBUSati ya EvaM veVda || 2 ||
\end{shl}

\begin{shl}
upAsatxyXnatxrameVveYtataPxlavatutx vivakiSxtamf | \\
na tUkatxsheVSateYtasAyx BinonxVpAsitxtavxkAraNAtf \hfill|| 4 || 
\end{shl}

\begin{artha} 
idoMdu PalavuLaLx beVre upAsaneyeMbudeV vivakiSxta. Adare hiMde 
heVLida gAyatirxV upAsanege idu aMgavAgilalx. idu beVre 
upAsaneyAgiruvudeV kAraNa.
\end{artha}

\vishaya{shaMke}

\begin{shl}
manathxkamaRNi yeV manAtxrXH pacnacx jeyxVSAThxdayaH shurxtAH | \\
pArxNAtamxveVdinasetxVSAM parxyoVgoV\s torxVpavaNayxRteV \hfill|| 5 || 
\end{shl}

\begin{artha} 
manathxnakamaRdalilx jeyxVSAThxdi maMtarxgaLu aidu. yAvudu 
shurxtavAgiveyo avugaLa parxyoVgavanunx pArxNAtamxvanunx 
upAsisuvavanige ililx muMde vaNiRsideyeMdu heVLabahudaSeTx? (eMdu 
shaMke).
\end{artha}

\vishaya{parihAra {\rm --}}

\begin{shl}
paqthagAvx PalanideVRshAdoyxV ha vA iti pacnacxdhA | \\
pArxNavidAyx paqthaknamxnAthxnamxnathxsutx mahimAthiRnaH \hfill|| 6 || 
\end{shl}

\begin{artha} 
`yoV havA' eMdu aidu bageyAgi beVre Palavanunx nideRVshisidadxriMda 
pArxNavideyxyu maMthakamaRkikxMta beVreyAgide. (adakekx aMgavilalx) 
maMthanakamaRvu mahimeyanunx bayasuvavanige heVLide.
\end{artha}

\vishaya{mahatatxvX Palavu pArxNavideyxge ilalxve? eMdare {\rm --}}

\begin{shl}
PaleV\s nayxsimxnanxnidiRSeTxV vAkayxsheVSagataM Palamf | \\
tasimxnasxti hi sadABxvAdAvxgAdiVnAM na taM vinA \hfill|| 7 || 
\end{shl}

\begin{artha} 
beVre Palavanunx nideRVsha mADadiruvAga vAkayxsheVSadalilxruva 
Palavanunx garxhisabeVkAguvudu. (beVre Palavu nideRVshisalapxTaTxlilx 
Palavu iruvudariMda vAkayxsheVSadalilxna (mahatatxvXPalavu 
gArxhayxvalalx) pArxNavu idadxre vAgAdi iMdirxyagaLu iruvudariMdalU 
adilalxde ivu ilalxdiruvudariMdalU (vAgAdi iMdirxyagaLige 
sherxVSaThxteyu ilalx).)
\end{artha}

\vishaya{I viSayakekx shAsatxrXsaMvAdavU irutatxde {\rm --}}

\begin{shl}
shAserxVNoVkAtx shariVreV\s simxnavxqqtitxH pArxNasayx jiVvanamf | \\
pUvaRmAvishati pArxNoV deVhaM pashAcxcacx mucnacxti \hfill|| 8 || 
\end{shl}

\begin{artha} 
shAsatxrXvu heVLuvudu EneMdare? I shariVradalilx pArxNada vaqtitxyeV 
jiVvana. (gaBaRdalilx) pArxNavAyuveV modalu shariVravanunx 
parxveVshisuvudu. anaMtaravU pArxNaveV biDuvudu.
\end{artha}

\begin{shl}
jeyxVSaThxH sherxVSaThxshacx saveVRSAM pArxNAnAmAsharxyoV hi saH | \\
sherxVSaThxtA vakaSxyXmANeVna garxnethxVnAsayx viBAvayxteV \hfill|| 9 || 
\end{shl}

\begin{artha} 
elalx vAgAdi pArxNagaLigU A pArxNaveV jeyxVSaThx matutx sherxVSaThx. 
muMde heVLuva garxMthadiMda idakekx sherxVSaThxteyu gotAtxguvudu.
\end{artha}

\vishaya{`yoVhaveY vasiSAThxM' itAyxdi maMtarxda athaR {\rm --}}

\footnotetext[1]{yAru vAkayxvanunx vasiSaThxveMba guNavuLaLxdAdxgiruvaMte upAsane mADuvaro, alalxde jAcnxtigaLalilx hecicxnavanAgi vasiSaThxnAgalu bayasuvaro avaru upAsanege takakxMte vasiSaThxneV Aguvanu. vAkukx vasiSaThxveMbudakekx vAsayati atishayeVna eMba vuyxtapxtitxyiMda vAgimxgaLAdavaranunx hecicxna haNavaMtaranAnxgi mADabalalxdu. adariMda vasiSaThx matutx vasatx iti vasiSAThx eMba vuyxtapxtitxyiMda inonxbabxranunx mucucxvudu. vAkikxniMda matotxbabx vAcAligaLanunx tirasakxrisuvanu. adariMda `vasa nivAse, vasa AcACxdane' eMbuva eraDu dhAtugaLa rUpave ililx vasiSaThxveMba pAdavu. I athaRdalilx vasiSaThxveMba guNagaLiMda kUDi vAkakxnUnx upAsisuvanu, vAgimxyAgi matotxbabxranunx aDagisi ati dhanavaMtanAgi bALuvanu eMdu I maMtarxda athaR. idu BASayxdalilxruva athaR. cakuSxrAdi itara iMdirxyagaLige upahatiyAguvaMte vAkikxge upahatiyilalx. eraDu loVkagaLalUlx idara parxvaqtitxyuMTu. tananx deVhadoLagiruva pArxNavanunx (niroVdhisuva) niyamisuvaMte itara deVhadalilxruva (pArxNa) itara iMdirxyagaLanunx niyamisabalalxdu vAkukx udA:- obabx yajamAnanu manemaMdiyelAlx janaranunx IkaDe noVDikoLiLx, idanunx keVLiri, idanunx tininxri, eMdu vAcA niyamisabalalxnu. hAgeye rAjanu Baqtayxranunx hoVgu, noVDibA, ililx noVDiko, ililx keVLu, hiVge hoVgu, bA eMdu matotxbabxnanunx niyamisuvanu hiVgeye vAkukx elalxra parxvaqtitxgU kAraNavAdadxriMda adu akuMThitavAgiruvudeMdu. itara iMdirxyagaLa parxvaqtitxgU vAkekxV kAraNavAdadxriMda adu adhiraveMdu upAsayxvAgide.}
\begin{shl}
\footnotemark[1]asimxMlolxVkeV parasimxMshacx vAgeVva na vihanayxteV | \\
deVhAnatxrasAthxnApxrXNAMshacx niyuknatx\s sA tatoV\s dhikA \hfill|| 10 || 
\end{shl}

%% shloka footnote
\begin{artha} 
ihaloVkadalUlx paraloVkadalUlx vAkekxMbudeV kuMThitavAguvudilalx. 
beVre deVhadalilxruva pArxNagaLanUnx kUDa niyoVgisuvudu. adariMda adu 
itara pArxNagaLigiMta hecicxnadu.
\end{artha}

\vishaya{baq. a.6, bArx. 1, kaMDike 3}

\begin{shl}
yoV ha veY parxtiSAThxM veVda parxtitiSaThxti sameV parxtitiSaThxti dugeVR cakuSxveYR parxtiSAThx cakuSxSA hi sameV ca dugeVR ca parxtitiSaThxti parxtitiSaThxti sameV parxtitiSaThxti dugeVR ya EvaM veVda || 3 ||
\end{shl}

\vishaya{I meVlina maMtarxda athaR {\rm --}}

\begin{shl}
shurxtAnamxtAtatxthoVkAtxdAvx parxmANaM daqSiTxrAtamxnaH | \\
cakuSxH parxtiSAThx jAcnxnAnAmAtAmx tatarx parxtiSiThxtaH \hfill|| 11 || 
\end{shl}

\begin{artha} 
\footnote[2]{kaNuNx eMbuva cakuSxriMdirxyavu parxtiSAThx nijavAda 
Asharxya. tAnu nilulxvudakekx kAraNa heVge? kaNiNxniMda samaBUmiyalilx 
noVDi nilulxvanu dugaRmavAda pavaRtAdigaLalUlx noVDi nilulxvanu. 
hAgeyeV suBikaSxvAda kAladalUlx duBiRkASxdi kAladalUlx kaNiNxniMda 
noVDi tiLidu nilulxvanu. adariMda tAnu nilulxvudakekx muKayxvAda 
kAraNa cakuSx adu parxtiSAThx Asharxya. I guNaviruvaMte 
cakuSxriMdirxyavanunx upAsane mADidavanu samadeVshadalUlx, 
viSamaparxdeVshadalUlx, samaviSama kAlagaLalUlx noVDi nilalxbalalxnu. 
idu BASayxdaMte athaR. vAtiRkadalilx visheVSAthaRvide. nAvu oMdu 
viSayavanunx keVLi tiLiyuvudakikxMtalU, matutx takiRsuvudakikxMtalU 
KacitavAgi niNaRya koDuvudu daqSiTx. kaNiNxniMda noVDidadxnunx 
naMbabahudu. adariMda itara jAcnxnagaLigU ideV Asharxya. koneya 
niNaRyakekx kAraNa, `nahidaqSeTxV anupapananxM nAma' noVDidadxralilx 
anupapatitxyilalx. saMshayakekx eDeyilalx. adariMda cakuSx Asharxya. 
`yoV\char'263 yaMdakiSxNeV\char'263 kaSxnf puruSaH' eMdu 
balagaNiNxnalilx I Atamxnu visheVSavAgi nelesuvanu. AsAthxnadalilxdudx 
noVDuvaneMdu visheVSavAgi heVLiyU ide. adariMda shorxVtarx itAyxdi 
iMdirxyagaLigU adhikavAdadudx cakuSx, adu upAsayxveMdathaR.}keVLidadxkUkx (tAnu tiLididadxkUkx) takiRsidadxkUkx 
hAgeyeV heVLiruvudakUkx hecicxna parxmANaveMdare tanage kaNaNxlilx 
noVDuvudeV. kAraNaveVneMdare elAlx jAcnxnagaLigU Asharxya 
cakuSxriMdirxya.
\end{artha}

\vishaya{`yoVhaveY saMpada' - idara athaR {\rm --}}

\begin{shl}
vAgiGx saMpadayxteV shorxVtArxdashurxtaM na hi BASateV | \\
savxvaqtetxVH paravaqtetxVshacx saMpacoCxrXVterxV parxtiSiThxtA \hfill|| 12 || 
\end{shl}

\begin{artha} 
vAkukx shorxVterxVMdirxyadiMda pUNaRvAguvudu. keVLade idadxdadxnunx 
tAnu mAtanADuvudilalxvaSeTxV. tananx vAyxpArada PalavAgi baruva 
saMpatutx matotxbabxna vAyxpArada PalavAda saMpatutx saha 
shorxVterxVMdirxyadalelxV niMtiruvudu.
\end{artha}

\vishaya{baq. a.6, bArx. 1, kaMDike 4}

\footnotetext[1]{I meVlina maMtArxthaR - shorxVterxVMdirxya saMpatutx 
heVgeMdare, idu idadxre elAlx veVdagaLu keYgUDuvavu. kivi idadxvane 
veVdavanunx keVLi adhayxyana mADuvudu. veVdadalilx vidhisida 
kamaRgaLige adhiVnavAgi elAlx BoVgasaMpatutxgaLU baruvuvu. adariMda 
shorxVterxVMdirxyavu saMpatetxMdu heVLide. yAva sAdhakanu saMpatetxMba 
guNavuLaLxdAdxgi I shorxVtarxvanunx dhAyxnisuvano avanige I Palavu 
baruvudu yAva BoVgavanunx bayasuvano adeV kAma. adeV BoVgavu ivanige 
odaguvudu.}
\begin{shl}
\footnotemark[1]yoV ha veY samapxdaM veVda saM hAsemxY padayxteV yaM kAmaM kAmayateV shorxVtarxM veY samapxcoCxrXVterxV hiVmeV saveVR veVdA aBisamapxnAnxH saM hAsemxY padayxteV yaM kAmaM kAmayateV ya EvaM veVda || 4 ||
\end{shl}

\vishaya{baq. a.6, bArx. 1, kaMDike 5}

\footnotetext[1]{yAvanu manaseVsx iMdirxyagaLigU viSayagaLigU AsharxyaveMdu 
tiLidu dhAyxnisuvano, avanu tananx jAcnxtigaLigU itararigU 
AsharxyanAguvanu. manasisxnalilx nelesida viSayagaLu Atamxnige 
BoVgayoVgayxvAgutatxve. iMdirxyagaLu manasisxna saMkalapxkekx 
takakxMte naDeyutatxve. manasisxdadxre iMdirxyagaLu horage ODADutatxve 
ilalxvAdare hiMtirugutatxve. adariMda manaseVsx Ayatana Asharxya. 
adariMda adu namage upAsayx.}
\begin{shl}
\footnotemark[1]yoV ha vA AyatanaM veVdAyatanaM sAvxnAM BavatAyxyatanaM janAnAM manoV vA AyatanamAyatanaM sAvxnAM BavatAyxyatanaM janAnAM ya EvaM veVda || 5 ||
\end{shl}

%% maMtarx footnote

\vishaya{I maMtarxda athaR {\rm --}}

\begin{shl}
mana AyatanaM tatarx vAgAdiVnAM hi vaqtatxyaH | \\
sithxtAsatxtUpxviRkAshecxYva dhAyxyataH sAdhanaM hi tatf \hfill|| 13 || 
\end{shl}

\begin{artha} 
iMdirxyagaLigU viSayagaLigU manasesxV Asharxya. adaralilx elAlx vAgAdi 
iMdirxyagaLa vaqtitxgaLU iruvavu. manaHpUvaRkavAgiyeV I vaqtitxgaLu 
adaralilx nelesive. dhAyxna mADuvavanige manasesxV sAdhana.
\end{artha}

\section*{baq. a.6, bArx. 1, kaMDike 6}

\footnotetext[2]{I maMtarxda athaR - yAvanu parxjAti reVtasusx eMbuva 
jananeVMdirxyavanunx parxjAsaMtatige kAraNaveMdu tiLidu adanunx 
upAsisuvano avanu parxjeyiMdalU pashugaLiMdalU saMpananxnAguvanu. 
adariMda adu upAsayxveMdathaR.}
\begin{shl}
\footnotemark[2]yoV ha veY parxjAtiM veVda parxjAyateV ha parxjayA pashuBiV reVtoV veY parxjAtiH parxjAyateV ha parxjayA pashuBiyaR EvaM veVda ||6||
\end{shl}

\vishaya{I maMtarxda tAtapxyaR {\rm --}}

\begin{shl}
upasethxVnidxrXyaM parxjAtiH sAyxtatxsayx janemxYkaheVtutaH | \\
na hi reVtoV vinA janamx pArxNinoV\s tarx samiVkaSxyXteV \hfill|| 14 || 
\end{shl}

\begin{artha} 
parxjAti eMdare upasethxVMdirxya. adu janamxvoMdakekx kAraNa. reVtasusx ilalxde yAva pArxNigU huTeTxMbudeV ilalx. idu ihadalilx hiVge kANuvudilalx.
\end{artha}

\vishaya{`teVha' itAyxdi maMtarxda tAtapxyaR {\rm --}}

\begin{shl}
vaqtitxVnAM pArxNapUvaRtAvxdavxyXpadeVshAcacx tatakxqqtAtf | \\
pArxNAnAM parxthamaH pArxNaH sa hayxtAtx\s nanxM hi tasayx tatf \hfill|| 15 || 
\end{shl}

\begin{artha}
samasatx iMdirxya vAyxpAragaLu pArxNavAyuvina mUlakavAgiruvudariMdalU pArxNa nimitatxvAgiyeV pArxNaveMdu karediruvudariMdalU elAlx pArxNagaLigU (iMdirxyagaLigU) I pArxNavAyuvu modalaneyadu, (jeyxVSaThx, sherxVSaThx), A pArxNaveV BoVkatx. adakekx itara pArxNAdigaLu BoVgayxvaSeTx.
\end{artha}

\vishaya{baq. a.6, bArx. 1, kaMDike 7}

\begin{shl}
teV heVmeV pArxNA ahaMsherxVyaseV vivadamAnA barxhamx jagumxsatxdodhxVcuH koV noV vasiSaThx iti tadodhxVvAca yasimxnavx utAkxrXnatx idaM shariVraM pApiVyoV manayxteV sa voV vasiSaThx iti || 7 ||
\end{shl}

\vishaya{I maMtarxda vAyxKAyxna {\rm --}}

\begin{shl}
teV pArxNAH savxguNeVruketxYvaRsiSaThxtAvxdilakaSxNeYH | \\
sherxVyAnasamxyXhameVveVti vivadanatxH parasapxramf \hfill|| 16 || 
\end{shl}

\begin{shl}
niNaRyAthARya teV barxhamx jagumxrinadxrXM parxjApatimf | \\
koV noV vasiSaThx iti taM paparxcuCxniRNaRyAthiRnaH \hfill|| 17 || 
\end{shl}

\begin{artha}
A pArxNagaLu (iMdirxyagaLu) hiMde heVLida vasiSaThxtavx modalAda 
tamamx guNagaLiMda nAneV sherxVSaThxneMdu anoyxVnayx jagaLavADutAtx 
idara niNaRyakAkxgi barxhamx eMbuva IshavxranAda parxjApatiya hatitxra 
hoVdaru. niNaRyavanunx bayasida I pArxNagaLu namamx peYki yAru 
vasiSaThx? (hecicxna vAgimx?) eMdu keVLidavu.
\end{artha}

\vishaya{adakekx barxhamxnu heVLida utatxra {\rm --}}

\begin{shl}
yasimxnfva iti vAkeyxVna vasiSaThxtavxsayx lakaSxNamf | \\
pArxNeVBayxH pArxbarxviVdabxrXhamx pakaSxpAtaBayAtikxla \hfill|| 18 || 
\end{shl}


\begin{artha}
`yasimxnf vaH' eMba vAkayxdiMda parxjApatiyu pakaSxpAtavAguva 
BayadiMda vasiSaThxtavxda lakaSxNavanunx pArxNagaLige modalu 
heVLidanu.
\end{artha}

\begin{shl}
jAnananxpi vasiSAThxdiguNavatatxvXM yathAthaRtaH | \\
tathA\s pi nAvadadabxrXhamx sAvxnuBUtayxvabudadhxyeV \hfill|| 19 || 
\end{shl}

\begin{artha}
matutx vasiSAThxdi guNagaLu yathAthaRvAgi iruvudanunx tiLidavanAdarU 
parxjApatiyu heVLalilalx. EkeMdare? tamamx anuBavadiMdaleV 
tiLiyabeVkeMbudakAkxgi.
\end{artha}

\vishaya{adeVneMdare:-}

\begin{shl}
utAkxrXnetxV\s nayxtameV yasimxnapxrXteyxVkamapasapaRNeV | \\
pApiVyoV manayxteV loVkoV vasiSoThxV vaH samiVkaSxyXtAmf \hfill|| 20 || 
\end{shl}

\begin{artha}
eleY vAgAdigaLirA, nimamx peYki obabxru parxteyxVkavAgi obobxbabxrU 
shariVravanunx biTuTx edudx Acege saridalilx I shariVravanunx janaru 
pApiSaThxveMdu (atayxMta doVSayukatx asapxqqshayx)veMdu tiLiyuvaro, 
AvAga nimamxlilx obabxnu vasiSaThx eMdu Aguvanu, noVDiri eMdu 
heVLidanu.
\end{artha}

\vishaya{baq. a.6, bArx. 1, kaMDike 8}

\begin{shl}
vAgoGxVcacxkArxma sA saMvatasxraM porxVSAyxgatoyxVvAca kathamashakata madaqteV jiVvitumiti teV hoVcuyaRthAkalA avadanotxV vAcA pArxNanatxH pArxNeVna pashayxnatxshacxkuSxSA shaqNavxnatxH shorxVterxVNa vidAvxMsoV manasA parxjAyamAnA reVtaseYvamajiVviSemxVti parxviveVsha ha vAkf || 8 ||
\end{shl}

\vishaya{I shurxtigaLa tAtapxyaR {\rm --}}

\begin{shl}
anavxyavayxtireVkABAyxM vasiSaThxtAvxvabudadhxyeV | \\
upAsAyxthaRpariVkASxyeY parxvaqtetxYSA parA shurxtiH \hfill|| 21 || 
\end{shl}

\begin{artha}
anavxyavayxtireVkagaLiMda vasiSaThx eMbudanunx tiLiyalu matutx upAsayx vasutxvanunx pariVkiSxsalu I muMdina shurxtiyu baMdiruvudu.
\end{artha}

\vishaya{`sA saMvatasxraM porxVSayx' itAyxdi maMtarxdalilx toVruvaMte 
oMdu vaSaR payARMtara shariVravanunx biTuTx vAgiMdirxyAdigaLu horage 
hoVgidadxveMba athaRvu vivakiSxtave? eMba shaMkeyanunx pariharisuvudu 
{\rm --}}

\begin{shl}
saMhatAnAM kirxyAsidedhxVH karaNAnAM paqthakapxqqthakf | \\
neYkeYkasayx kirxyAsididhxH shibikoVdAvxhavatatxtaH \hfill|| 22 || 
\end{shl}

\begin{artha}
nAlukx janaru seVri palalxkikxyanunx horuvaMte malinavAda 
iMdirxyagaLalelxV kirxyAsididhxyAguvudariMda oMdoMdakekx kirxyeya 
niSapxtitxyAgalAradeMdu tiLiyabeVku. (adariMda vaSaRkAla parxvAsavu 
vivakiSxtavalalx)
\end{artha}

\begin{shl}
pArxNapArxdhAnayxsidadAdhxyXthaRM shurxtAyx\s \s KAyxyikacaCxdamxnA | \\
anavxyavayxtireVkABAyxM nAyxyoV lwkika ucayxteV \hfill|| 23 || 
\end{shl}

\begin{artha}
pArxNa vasutxvige pArxdhAnayxvu sididhxsalu shurxtiyu katheya nepadalilx anavxya vayxtireVkagaLiMda kUDida lawkika nAyxyavanunx heVLiruvudu.
\end{artha}

\vishaya{lawkika nAyxyavanunx I muMde toVrisuvaru {\rm --}}

\begin{shl}
yathA mUkA vinA vAcA yathA\s nAdhxshacxkuSxSA vinA | \\
itAyxdivacasA pArxNeV sati jiVvanamucayxteV \hfill|| 24 || 
\end{shl}

\begin{artha}
heVge bAyiyilalxdidadxvaru mUkarAguvaro, hAgeye kaNiNxlalxdavaru kuruDarAguvaro itAyxdi vacanadiMda pArxNavAyuvu idadxreV jiVvana (baduku) eMdu heVLiruvudu.
\end{artha}

\begin{shl}
utAkxrXnwtx ca parxveVsheV ca hayxlaM dahaH savxkamaRNeV | \\
vAgAdiVnAM na pAtoV\s sayx nApi coVtAthxnamiVkaSxyXteV \hfill|| 25 || 
\end{shl}

\begin{artha}
vAgAdi iMdirxyagaLu shariVravanunx biTuTx meVlakekx horaTAgalU punaH parxveVshisidAgalU (I aMtaradalilx) deVhaveV tananx kAyaRvanunx naDesalu samathaRvAgididxtu. avu biTuTx edAdxga deVhakekx patanavAgalilalx. avu baMdu punaH parxveVshisidAga shariVravu edudxkoMDidUdx kANisalilalx.
\end{artha}

\begin{artha}
\textbf{8-9-10-11-12 maMtarxgaLa sArAthaR {\rm --}}
(parxjApatiyu heVLida naMtara vAgAdi iMdirxyagaLeMba pArxNagaLu tamamx shakitxyanunx pariVkiSxsikoLaLxlu karxmavAgi vAgiMdirxya, cakuSxriMdirxya, shorxVterxVMdirxya, manasusx - matutx jananeVMdirxya - ivugaLu oMdoMdAgi edudx oMdu vaSaR shariVravanunx biTuTx dUra hoVgiralu deVhavu tananx uLida kAyaRvanunx naDesutAtx jiVvisididxtu. vAgiMdirxya horaTAga mAtanADade idadxrU itara iMdirxyagaLiMda dashaRna, sharxvaNAdigaLanunx mADutAtx muKayxvAgi pArxNavAyuviniMda usirADutAtx jiVvisididxtu. hiVgeye kaNuNx horaTidAdxga kuruDAdarU itareVMdirxyagaLiMda pArxNavAyuviniMda vayxvaharisutAtx jiVvisididxtu. shorxVterxVMdirxyavu horaTAga kivuDAdarU itara vayxvahAravanunx sAgisididxtu. hAgeyeV manasUsx Acege hoVdAga mUDharAgidudx tiLiyadavarAgi itara vayxvahAravanunx mADutAtx sAdhisitu. jananeVMdirxyavu horaTAga napuMsakavAgi idudx jiVvisididxtu. oTiTxnalilx yAva jAcnxneVMdirxya kameRVMdirxyagaLU citatxvU ilalxdidadxrU pArxNavAyuviniMdale shariVravu jiVvisididxtu. adariMda itareVMdirxyagaLige pArxdhAnayxvilalxveMdu niNaRyavAyitu.)
\end{artha}

\begin{shl}
atha ha pArxNa utakxrXmiSayxnayxthA mahAsuhayaH seYnadhxvaH paDivxVshashaknUkxnasxMvaqheVdeVvaM heYveVmAnApxrXNAnasxMvavahaR teV hoVcumAR Bagava utakxrXmiVnaR veY shakASxyXmasatxvXdaqteV jiVvitumiti tasoyxV meV baliM kuruteVti tatheVti || 13 ||
\end{shl}

\vishaya{I maMtarxda tAtapxyaR {\rm --}}

\begin{shl}
utAkxrXnetxV pArxNa EvAsAmxcaCxriVraM patati dhurxvamf | \\
utitxSaThxti parxviSeTxV ca pArxNaH sherxVyAMsatxtoV\s nayxtaH \hfill|| 26 || 
\end{shl}

\begin{artha}
pArxNaveV shariVravanunx biTeTxdadxre shariVravu nishacxyavAgi 
biVLuvudu, adeV pArxNavu shariVradoLage parxveVshisidalilx shariVravu 
edudx nilulxvudu. adariMda beVre elalxdakUkx pArxNaveV sherxVSaThx.
\end{artha}

\begin{artha}
(\textbf{maMtArxthaR} {\rm --} anaMtara pArxNavu shariVravanunx biTeTxVLalu 
yatinxsuvAga AgaleV tananx tananx sAthxnadiMda vAgAdi iMdirxyagaLu 
calisidavu. adu heVgeMdare, kudureya savAranu pariVkASxthaRvAgi oMdu 
siMdhu deVshadalilx huTiTxdadx lakaSxNayukatxvAda doDaDx kudureyanunx 
Eralu A kudureyu tananx kAlugaLanunx kaTiTxhAkidadx gUTagaLanunx 
kitutxkoMDu oMdeVsala meVlakekx hAruvudo, hAgeye calisidavu. avugaLu 
calisuvaMte pArxNavu mADuvudu. AvAga A vAgAdi pArxNagaLu heVLidavu 
EneMdare? - pUjayx pArxNave? niVnu shariVravanunx biTuTx horaDabeVDa, 
niVnilalxde nAvu badukuvudakekx Aguvudilalx, eMdu. idanunx keVLi 
pArxNavu heVLitu, niVvugaLu nananx sherxVSaThxteyanunx tiLidideVdx 
Adare nanage kapapxkANikeyanunx opipxsi eMdu.)
\end{artha}

\vishaya{I meVlina maMtarxdalilx `teVhoVcuH' - itAyxdi vAkayxda 
tAtapxyaRvanunx heVLutAtxre {\rm --}}

\begin{shl}
mAmaqteV jiVvituM yUyaM yadayxshakAtxH sathx savaRdA | \\
parxdhAnaM tahiR mAM vitatx BavanatxshAcxparAdhinaH \hfill|| 27 || 
\end{shl}

\begin{shl}
karaM baliM parxdhAnAya datatx vAgAdayoV\s cirAtf | \\
ituyxkAtxsetxV tatheVtUyxcuH savaRsavxM dadateV parxBoVH \hfill|| 28 || 
\end{shl}

\begin{artha}
eleY, vAgAdi iMdirxyagaLirA niVvu shiVGarxvAgi parxdhAnavAda 
pArxNadeVvanige kapapxkANikeyanunx koDiri eMdu heVLalu hAgeyeV AgaleMdu 
heVLidavu. parxBuvige samasatxvanunx salilxsidavu.
\end{artha}

\vishaya{`sAheVtAyxdi' maMtarxda tAtapxyaR {\rm --}}

\vishaya{baq. a.6, bArx. 1, kaMDike 14}

\begin{shl}
sA ha vAguvAca yadAvx ahaM vasiSAThxsimx tavxM tadavxsiSoThxV\s siVti yadAvx ahaM parxtiSAThxsimx tavxM tatapxrXtiSoThxV\s siVti cakuSxyaRdAvx ahaM samapxdasimx tavxM tatasxmapxdasiVti shorxVtarxM yadAvx ahamAyatanamasimx tavxM tadAyatanamasiVti manoV yadAvx ahaM parxjAtirasimx tavxM tatapxrXjAtirasiVti reVtasatxsoyxV  ||
\end{shl}

\begin{artha}
athaR (A vAkukx modalu karavanunx koDalu parxvatiRsi hiVge heVLitu - 
yadAvx nAneV vasiSaThxLAgidedxVne. nananx vasiSaThxtavxvu ninanxdeV 
Agide. A vasiSaThxtavx guNadiMda niVnu vasiSaThxnAgiruve. hiVgeye 
itara iMdirxyagaLu karxmavAgi parxtiSAThx, saMpatutx, Asharxya, 
parxjanana itAyxdi guNavuLaLxvugaLeMdu heVLikoLuLxtAtx avelalxvU 
ninanxdeV AgiveyeMdu pArxNakekx opipxsidavu - idAda naMtara utatxmavAda 
karavanunx niVvu koTiTxdidxVri. I guNagaLiMda kUDida pArxNadeVvanige 
ananx yAvudu? baTeTx yAvudu? tegedukoMDu baninx eMdu pArxNavu heVLitu.)
\end{artha}

\begin{shl}
yadidaM kicnAcxshavxBayx A kaqmiBayx A kiVTapataknegxVBayxsatxtetxV\s nanxmApoV vAsa iti na ha vA asAyxnananxM jagadhxM Bavati nAnananxM parxtigaqhiVtaM ya EvameVtadanasAyxnanxM veVda tadivxdAvxMsaH shorxVtirxyA ashiSayxnatx AcAmanatxyXshitAvxcAmanetxyXVtameVva tadanamanaganxM kuvaRnotxV manayxnetxV || 14 ||
\end{shl}

\begin{shl}
tavxdavxsiSaThxtayeYvAhaM vAgavxsiSeThxVtuyxdAharatf | \\
ituyxkAtxvX\s neyxV\s pi savaRsavxM daduvARgAdayaH surAH \hfill|| 29 || 
\end{shl}

\begin{artha}
pArxNadeVvaneV! ninanxlilxruva vasiSaThxteyeMbuva guNadiMdaleV nAnU 
vasiSaThxnAgiruveneMdu vAkukx heVLitu. hiVgeyeV itara iMdirxyAdigaLU 
heVLidavu anaMtara vAgAdi deVvategaLu tamamx savaRsavxvanunx apiRsidavu.
\end{artha}

\vishaya{eraDu parxshenxgaLa vAyxKAyxna}

\begin{shl}
kimananxM meV buBukoSxVH sAyxdAvxsoV vA meV kimiVyaRtAmf | \\
iti pArxNavacaH shurxtAvx parxtUyxcuH karaNAni tamf \hfill|| 30 || 
\end{shl}

\begin{shl}
A shavxBoyxV yadidaM kiMcidA kaqmiBayxshacx lakaSxyXteV | \\
ananxM tadaBxvataH savaRM vayaM tavxceCxVSaBoVginaH \hfill|| 31 || 
\end{shl}

\begin{artha}
nanage ananx AhAra yAvudu? nanage hasivu Aguvudu, nanage baTeTx 
yAvudu? eMdu pArxNavu heVLida mAtanunx keVLi iMdirxyagaLu A 
pArxNavanunx kuritu nAyigaLiMda AraMBisi kirxmigaLa payaRMtara 
yAvudoMdu kaMDideyo, adelalx ananxvU ninage irali. nAvu niVnu UTa mADi 
uLida sheVSavanunx BuMjisuvavaru.
\end{artha}

\vishaya{ililx oMdu shaMke}

\begin{shl}
jiVvaH pArxNoV\s tarx saMsAriV BoVketxVnidxrXyamanaHparaH | \\
\footnotemark[1]pArxNoV heyxVtAni savARNi BavatiVti ca liknagxtaH \hfill|| 32 || 
\end{shl}
\footnotetext[1]{`AtemxVMdirxya manoVyukatxM BoVketxVtAyxhu maRniVSiNaH' eMba shurxtiyaMte jiVvaneV BoVkAtx. parxkaqta elAlx ananxvU pArxNakekx eMdu heVLidadxriMda pArxNavU BoVkatxqqveMdu heVLuvudu yukatx. saveRVMdirxyagaLu pArxNavaneVnx AsharxyisiruvudariMda iMdirxyasAvxmitavxvu jiVvanigiMta beVre kaDeyilalxvAdadxriMda pArxNaveMbudu jiVvave eMdu aBipArxya.}

%% shloka footnote
\begin{artha}
meVle heVLida maMtarxdalilx pArxNaveMdare iMdirxya manasusxgaLalilx AsakatxnAgi BoVkatx eMbuva saMsAri jiVvaneMdu heVLabeVkaSeTx. idakekx `pArxNoVheyxVtAni savARNi Bavati' eMba shurxtiyu gamakavAgide. adariMdalU (jiVvane pArxNa eMbudara athaR)
\end{artha}

\vishaya{muKayx pArxNakUkx iMdirxyagaLa sAvxmitavxvirabAradeVke? eMdare {\rm --}}

\footnotetext[2]{iMdirxyagaLu huTuTxvAgalU iruvAgalU layavAguvAgalU 
jiVvAdhiVnavAgiruvudariMda iMdirxyagaLige jiVvaveV sAvxmi. 
`tamutAkxrXmanatx pArxNamanUtAkxrXmati' eMba shurxtiyU parxmANavAgide. 
elalxvU pArxNakekx ananxveMdu upAsane mADuvavanige pArxNatAdAtamxyXveV 
baruvudu. upAsakanu pArxNaveV AgiruvudariMda adakekx heVLida 
ananxvelalxvU upAsakanige seVruvudu.}
\begin{shl}
\footnotemark[2]tanamxyA hiVtareV pArxNAsatxnUmxlAsatxninxbanadhxnAH \hfill|| 33 | \\
EvaMvideV hi nAnananxM kiMcidasitxVti dashaRnAtf | \\
PalaM sAyxtApxrXNasAyujayxM savaRM tasayx hi BoVjanamf\hfill|| 34 || 
\end{shl}

%% shloka footnote
\begin{artha}
itara pArxNagaLu muKayxpArxNa savxrUpagaLeV. A pArxNada mUlakavAgiyU pArxNavAyu nimitatxvAgiyU iruvavu. I riVtiyAgi tiLidavanige ananxvalalxdudx yAvudoMdU ilalxveMdu kaMDiruvudariMda pArxNasAyujayxveV PalavAguvudu. avanige elalxvU UTaveV Aguvudu.
\end{artha}

\vishaya{hAgAdare upAsakanu ananxvAgirali eMdare {\rm --}}

\begin{shl}
BavatayxtAtx sa savaRsayx nAnanxM Bavati kasayxcitf | \\
keVvaleV\s vasithxteV\s tatxqqtevxV maqtuyxnA\s pi na giVyaRteV \hfill|| 35 || 
\end{shl}

\begin{artha}
upAsakanu elalxvanunx tinunxva (BoVkatx)nAguvanu. Adare yArigU ananxvAguvudilalx. keVvala BoVkatxqqtavxveV irutitxralu maqtuyxviniMdalU avanu nuMgalapxDuvudilalx.
\end{artha}

\vishaya{pUvaRpakaSx shaMke {\rm --}}

\begin{shl}
adiBxH paridadhateyxVnamashiSayxnatx iti shurxteVH | \\
vidadhAti hayxpAM pAnamapUvaRM shAsatxrXlakaSxNamf \hfill|| 36 || 
\end{shl}

\begin{artha}
UTa mADalapxDuva ananxvanunx UTamADuva pArxNavanunx `adiBxHparidadhAti' eMbudAgiyU `Ena mashiSayxnatx' eMbudAgiyU iruvudariMda apUvaRvAda jalapAnavanunx vidhisuvudu.
\end{artha}

\begin{shl}
shAsarxM muKAyxthaRmeVva sAyxdivxdhayxthaRM yadi kalapxyXteV | \\
yukatxH PalAnuSaknogxV\s sayx vAsoVlABoV vidhiyaRdi \hfill|| 37 || 
\end{shl}

\begin{artha}
shAsatxrX vidhigAgi baMdideyeMdu kalipxsidare muKAyxthaRvuLaLxdedxV Aguvudu. (AvAga Acamanakekx vidhiyu sididhxsuvudu) AvAga pArxNakekx vasatxrXlABavu ide. PalasaMbaMdhavU yukatxvAguvudu.
\end{artha}

\begin{shl}
EkeV ca shAKinoV vayxkatxM vidhirUpamadhiVyateV | \\
manetxrXVNa pArxshanaM cApAmeVkeVSAM shAKinAM matamf \hfill|| 38 || 
\end{shl}

\begin{artha}
kelave shAKeyavaru (CaMdoVgaru) `ashiSayxnAnxcAmeVtf ashitAvxcA cAmeVtf' eMdu vidhisavxrUpavanunx sapxSaTxvAgi adhayxyana mADuvaru. I maMtarxdiMda jalapArxshanavu kelavu shAKeyavarige saMmatavAgide.
\end{artha}

\footnotetext[1]{ililx pUvoRVtatxra pakaSxgaLa sArAMshavanunx vAcakara sawkayaRkAkxgi koTiTxruvevu. `kiMmeVananxM kiMmeVvAsaH' eMdu pArxNavu keVLidadxkekx `yadidaMkiMca AshavxBayxH AkaqmiBayxH satxtetxV\char'263 nanx mApoVvAna iti' - ililx samasatx pArxNigaLU tinunxva ananx AhAravu nAyi, kirxmi payaRMtaviruvudelalxvU pArxNakekx ananxveMdU meVle kuDiyuva niVre vasatxrXveMdu heVLide. hiVgeMdu upAsakanu ciMtisabeVkeMdu heVLiruvudu vasutxsithxti.\\
CAMdoVgayxdalilx `tasAmxdAvx EtadashiSayxnatxH purasAtxcocxVpariSAvx cAcxdidhxH paridadhati' (CAM-5-2-2) hAgeyeV vAjasaneVyadalilx `tadivxdAvxMsaH shorxVtirxyA ashiSayxnatx AcAmanatxyXshitAvx cAcAmanitx EtameVva tadana managanxMkuvaRnotx manayxnetxV ' ide. eraDu shAKegaLalUlx `tasAmxdeVvaMvidashiSayxnAnxcAmeVtf ashitAvx cAcAmeVtf EtameVva tadanamanaganxM kuruteV' iti ililx AcamanavU, Acamana mADida niVrinalilx (anaganxtA ciMtane) vasatxrXBAvane mADuvudU saha pArxNakekx saMbaMdhisidaMte toVrutatxde. ililx eraDU vidhisalapxDuvudeV? athavA AcamanavoMdeV vihitavo? athavA anaganxtA ciMtanavoMdeV vihitavo? eMdu saMshaya baMdare pUvaRpakaSxvidu. eraDU tiLiyuvudariMda eraDU apUvaRvAgiruvudariMdalU vihitave. athavA AcamanavoMdeV vihitavAgide. AcAmeVtf - eMdu vidhiviBakitxyu sapxSaTxvAgide. anaganxtA kiVtaRna keVvala sutxtigAgi eMdu pUvaRpakaSxvu sidAdhxMta-samxqqtiyiMda shudidhxgAgi kataRvayxveMdu vihitavAda sakala kamARMgavAda Acamanavu modaleV parxsidadhx. adanunx anuvAda mADi anaganxtA dhAyxnave vihitavAguvudeMdu sidAdhxMta. `divxjoVnitayxmupasapxqqsheVtf' eMba samxqqti savaRsAmAnayx viSayavAdadudx. elalxrU shudidhxgAgi Acamana mADabeVku eMbudu sAmAnayxvidhi pArxNavidAyx parxkaraNadalilx paThitavAda shurxtiyu pArxNoVpAsaneya aMgavAgide. Adare pArxNavideyxge aMgavAgiyU apUvaRvAgiyU Acamanavu vidhisalapxDuvudilalx. puruSasAmAnayxranunx udedxVshisi vihitavAda AcamanaveV ililxyU toVruvudu. adariMda eraDakUkx vidhiyilalx. adariMda BoVjanakekx modalU AmeVlU mADabeVkAda Acamanavanunx anuvAda mADi AcamanayoVgayxvAda jaladalilx `EtameVva tadana managanxMshuvaRnotxV manayxnetx' eMba vAkayxdiMda anaganxtavx (vasatxrXsavxrUpa)vanunx ciMtisabeVkeMdu pArxNavidAyx saMbaMdhavAgi hosadAgi vidhisalapxDuvudu. Acamanavu vidhiyoVgayxvalalxdadxriMda adara sutxtiyu vivakiSxtavalalx. vasatxrXciMtaneyeV oMdu beVre kirxye. shudidhxgAgi heVLida Acamanavu beVre oMdu kirxye. vasatxrX ciMtaneyu pArxNavAyuvige hodudxkoLuLxvudakAkxgi vihitavAgide. adariMda Acamanavu shudidhxgAgiyU AcACxdanakAkxgiyU ideyeMbudU sariyalalx matutx `yadidaMkiMca' - eMba shurxtiyalilx elAlx vidhavAda ananxvanunx tinanxbeVkeMdu vihitavAgilalx. vidhisuvudakekx sAdhayxvU alalx, hAgeMdu heVLuva vAkayxvilalx, yArigU shakayxvU alalx. elalxvU pArxNadeVvanige ananxveMdu ananxdaqSiTxyanunx vidhiside. adaroMdige Acamanada jaladalilx vasatxrXciMtaneyU vihitavAgide. adariMda apUvaRvAgi jaladalilx AcamaniVya jaladalilx vasatxrXkalapxneyU vihitave horatu jAcnxtavAda Acamanavu punaH vidhiyoVgayxvalalxveMdu sidAdhxMtavu.\\`kAyARKAyxnAdapUvaRmf' eMba barxhamxsUtarxda () sidAdhxMtavidu|| alalxde pArxNoVpAsakanige savARnanx BakaSxNavu vihitavU alalx. tinanxbArada beLuLxLiLx, IruLiLx itAyxdi aBoVjayx BoVjanavanunx mADabeVkAgibaruvudu. aBoVjayx BoVjanada doVSakekx BAgiyAgabeVkAdiVtu. pArxNApAyAdi Apatitxnalilx elilx ananxvanunx tinanxbahudeMdu shAsatxrXvidadxrU Apatitxlalxda kAladalilx savARnanxvanunx tinunxvudakekx shAsatxrX parxmANaveV ilalx. `savARnAnxnumatiH pArxNAtayxyeV tadadxshaRnAtf' eMdu barxhamxsUtarxdalilx savARnanxvanunx tinanxlu anumati koTiTxruvudu pArxNavu hoVguva saMdaBaRvidadxlilx mAtarxveMdu niNaRyavanunx veVdavAyxsaru heVLiruvaru. vishAvxmitarxru shunaka mAMsavanunx tiMdareMbuva katheyu avaru ApatAkxladalilx tiMdareMdu heVLidadxriMda Apatitxlalxdavanige I udAharaNeyu salalxdu. visheVSa vicAravanunx barxhamxsUtarx BASayxdaleVlx tiLiyabahudu.}
\begin{shl}
\footnotemark[1]ananxdashaRnavacecxYkeV vAsoVdaqSiTxM parxcakaSxteV | \\
pArxyatAyxthaRmapAM pAneV daqSeTxVH parxkaraNAdiha | \\
savARBakaSxyXparxsakitxH sAyxdayxdi kamaR vidhitasxyXteV \hfill|| 39 || 
\end{shl}

%% shloka footnote
\begin{artha}
kelavaru ananxdashaRnadaMte (upAsaneyaMte) vasatxrXdaqSiTxyanUnx 
heVLuvaru. shudidhxgAgi jalavanunx pAna mADuvAga adaralilx 
vasatxrXdaqSiTxyanunx parxkaraNadiMda heVLabeVkAgide. 
pArxNoVpAsakanige savARnanx BakaSxNavanunx vidhisuva udedxVshavidadxre 
samasatx aBakaSxyXBakaSxNa parxsaMga baMdiVtu.
\end{artha}

\begin{center}
ililxge shirxVbaqhadAraNayxka BASayx vAtiRkadalilx\\
AraneV adhAyxyadalilx oMdaneV bArxhamxNavu mugidide.\\
|| shirxVdakiSxNAmUtaRyeV namaH ||
\end{center}
