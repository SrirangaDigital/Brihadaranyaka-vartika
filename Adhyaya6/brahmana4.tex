%%%% From 057.tex
\begin{center}
baqhadAraNayxka nAlakxneV bArxhamxNa\\
|| dakiSxNAmUtaRyeV namaH ||
\end{center}

\begin{shl}
ESAmiti naqbiVjasayx sutxtirukAtxyX vivakaSxyXteV | \\
parxtiSAThxM kalapxyAniVti yasAyxM puMsatxvXM parxpadayxteV \hfill|| 1 || 
\end{shl}

\begin{artha}
`ESAmf' eMdu AraMBisi `reVtaH' eMbuva payaRMta puruSabiVjada 
sutxtiyu vivakiSxtavAgide. `sahaparxjApatiriVkASxMcakerxV' eMbuvalilx 
(yAralilx iTaTx reVtasusx puruSatavxvanunx hoMduvudo) A 
AdhAraBUtavAdudanunx saqSiTxsuvaneMdu AloVcisidaneMdathaRvu.
\end{artha}

\vishaya{adanenxV muMde vAyxKAyxnisuvudu {\rm --}}

\begin{shl}
shukarxM niSikatxM yaterxVdaM puruSatavxM nigacaCxti | \\
parxtiSAThxM tAdaqshiVmiVshaH parxjApatiraciVkalxqqpatf \hfill|| 2 || 
\end{shl}

\begin{artha}
yAralilx seVcane mADida puruSaviVyaRvu puruSatavxvanunx hoMduvudo A 
bageya AdhAra vasutxvanunx (sitxrXVyanunx) saqSiTxsuveneMdu 
IshavxranAda parxjApati barxhamxnu AloVcisidanu.
\end{artha}

\vishaya{`sasitxrXya \c sasaqjeV'}

\footnotetext[1]{`patishacx patinxVcABavatAmf' eMbuvalilx heVLida shatarUpA eMbuva patinxyanunx saqSiTxsidaneMdu athaR.}
\begin{shl}
sitxrXyaM sasajaR tadoyxVgAyxM \footnotemark[1]madhukANeDxV yathoVditAmf | \\
saqSATxvX\s thAdhaH parxdeVsheV tAmupAsetxV \footnotemark[2]gArxmayxdhamaRtaH \hfill|| 3 || 
\end{shl}
\footnotetext[2]{gArxmayxdhamaRveMdare kAmashAsotxrXVkatxvAgiruvaMte pashukamaR.}

%% shloka footnote
\begin{artha}
adakekx yoVgayxvAda madhukAMDadalilx heVLidaMtiruva sitxrXVyanunx 
saqSiTxsidanu. saqSiTxmADi anaMtara A sitxrXVyanunx 
gArxmayxdhamaRdiMda (meYthunadiMda) keLaBAgadalilx seVvisidanu.
\end{artha}

\vishaya{`tasAmxtf sitxrXyamadha upAsiVta' eMbudara athaR}

\begin{shl}
apatoyxVtapxtatxyeV sirxVNAmadhoVdeVshaM parxjApatiH | \\
pArxgupAsitavAnayxtAnxdupAsayxM teVna tatatxtaH \hfill|| 4 || 
\end{shl}

\begin{artha}
parxjApatiyu makakxLa utapxtitxgAgi hiMde parxyatanxpUvaRka 
sitxrXVyara keLaBAgavanunx seVvisidanu. adariMda putarxkAmiyAdavanU 
adanunx seVvisabeVku.
\end{artha}

\vishaya{`sa EtaM pArxcnAcxM...... itAyxdi maMtarxda tAtapxyaR'}

\footnotetext[1]{vAjapeVya yAgakekx samAnavAgi I meYthuna kamaRvu 
iruvudeMdu BAvisabeVku. yAgadalilx soVmalateyanunx kuTiTx rasavanunx 
tegeyalu udadxvAda kalalxnunx upayoVgisuvuduMTu. hAgeye I pAshava 
kamaRdalilx A kalilxna sAthxnadalilx tananx jananeVMdirxyavu ideyeMta 
BAvisabeVku, kaThinavAgiruvudariMda hAgeyeV BAvisabeVku.}
\begin{shl}
\footnotemark[1]soVmABiSavarUpatavxkalxqqpatxyeV\s thAdhunoVcayxteV | \\
EtaM gArxvANavaciCxshanxmudapArayadAtamxnaH \hfill|| 5 || 
\end{shl}

%% shloka footnote
\begin{artha}
IvAga soVmABiSava rUpavanunx kalipxsikoLaLxlu heVLuvudu. I tananx 
shishanxvanunx soVmarasavanunx hiMDalu upayoVgisuva kalilxnaMte 
gaTiTxyAgiruvaMte tuMbikoMDanu,
\end{artha}

\vishaya{parxjApatiyu muMdeVnu mADidanu? {\rm --}}

\begin{shl}
pArxcnacxM kaqtAvx\s tha taM shishanxM yatheYvABiSavoVpalamf | \\
EkiVBAveVna taM nAyARM yatAnxtasxmupaveVshayeVtf \hfill|| 6 || 
\end{shl}

\begin{artha}
A shishanxvanunx QujuvAgi diVGaRvAgi mADikoMDu heVge soVmarasavanunx 
tegeyuva kalulx iruvudo hAgeyeV mADikoMDu adanunx sitxrXVyalilx 
(shAsitxrXVya) parxyatanxdiMda oMdAgi seVrisidanu.
\end{artha}

\vishaya{`teVneYnA maBayxsaqjata' eMbudara vAyxKAyxna {\rm --}}

\begin{shl}
tathABUteVna gArxvf NeYtAM sitxrXyamaBayxsaqjanumxhuH | \\
AtamxnaH puruSAthARya yathoVkotxVpAsanaM BaveVtf \hfill|| 7 || 
\end{shl}

\begin{artha}
hAgeyeV iruvaMte BAvisida, kalilxniMda sitxrXVyanunx matetx matetx 
saMbaMdha mADidanu. I meYthuna kamaRvanunx vAjapeVya yAgaveMdu BAvisi 
upAsane mADuvudu tananx puruSAthaRkAkxgi.
\end{artha}

\vishaya{upAsaneya karxmavanunx shurxtiyu heVLuvudu}

\vishaya{baq. a.6, bArx. 4, kaMDike 3}

\begin{shl}
tasAyx veVdirupasothxV loVmAni bahiRshacxmARdhiSavaNeV samidodhxV madhayxtaswtx muSwkx sa yAvAnahx veY vAjapeVyeVna yajamAnasayx loVkoV Bavati tAvAnasayx loVkoV Bavati ya EvaM vidAvxnadhoVpahAsaM caratAyxsAM sitxrXVNAM sukaqtaM vaqknekxtXV\s tha ya idamavidAvxnadhoVpahAsaM caratAyxsayx sitxrXyaH sukaqtaM vaqcnajxteV ||3||
\end{shl}

\begin{shl}
yasAyxM  parxjananaM puMsA sitxrXyAmaBayxnatxriVkaqtamf | \\
tasAyx veVdirupasathxH sAyxdUvoVRrupari dashaRnamf \hfill|| 8 || 
\end{shl}

\begin{shl}
bahiRsatxjAjxni loVmAni camARdhiSavaNeV tathA | \\
cameVRhA\s \s naDuhaM pasheyxVtosxVmABiSavasidadhxyeV \hfill|| 9 || 
\end{shl}

\begin{shl}
PalakeV cAdhiSavaNeV yathAsaMKeyxVna nidiRsheVtf \hfill|| 10 | \\
yoV yoVnayxnatxgaRtoV deVshaH samidadhxshacxmaR nidiRsheVtf | \\
tadedxVshamaBitaswtx yw muSwkx tw vaqSaNAviti \hfill|| 11 || 
\end{shl}

\begin{artha}
yAva sitxrXVyalilx puruSanu tananx jananeVMdirxyavanunx oLapaDisuvano, 
A sitxrXVya guhayx sAthxnaveV veVdiyeMdu BAvisabeVku. adu eraDu toDegaLa 
meVliruvaMte BAvisabeVku. adara meVle huTiTxruva roVmagaLeV 
daBeRgaLeMdU, adara camaRvanenxV etitxna camaRveMdU, eraDu BAgadalUlx 
oLage aDagiruva eraDu mAMsapiMDagaLanunx karxmavAgi soVmarasavanunx 
hiMDuva PalakagaLanAnxgiyU nAvu BAvisabeVku. yoVniya oLagiruva 
parxdeVshavanunx uriyuva aginxyeMdU BAvisabeVku. hiMde heVLida camaRvu 
yoVnideVshada sutatxlU iruvudanunx toVriside. adaroLage iruva 
muSakxgaLeMdare vaqSaNagaLeMdu (mAMsagarxMthigaLeMdu) tiLiyabeVku. I 
riVtiyAda muSakxgaLalilx soVmarasavanunx hiMDuva PalakagaLeMdu 
BAvaneyanunx mADabeVku.
\end{artha}

\vishaya{I riVtiyAgi BAvisi mADuva meYthuna kaqtayxvanunx vAjapeVya 
yAgaveMba utatxma BAvane mADuvavanige baruva Pala {\rm --}}

\begin{shl}
loVkaH sAyxdAvxjapeVyeVna yajamAnasayx mAnataH | \\
tAvanatxM loVkamAponxVti yAvanatxM vAjapeVyataH \hfill|| 12 || 
\end{shl}

\begin{artha}
vAjapeVya yAgavanunx mADuva yajamAnanige yAva loVkavu 
parxmANAnusAravAgi siguvudo, matutx vAjapeVyadiMda eSuTx loVkavanunx 
vAyxpisuvano aSeTxV loVkavanunx I vAjapeVyoVpAsakanU vAyxpisuvanu.
\end{artha}

\vishaya{meYthuna kamaRda aMgagaLalilx vAjapeVyAMgagaLanunx AroVpi 
BAvisidarU, A kamaRvanunx vAjapeVyaveMdu heVge BAvisuvudu? adara 
sAdaqshayxvilalxvalalx? eMdare {\rm --}}

\begin{shl}
anAnxni saMBirxyanetxV hi dasha sapatx ca BAgashaH | \\
vAjapeVyeV karxtAvananxkAmasayx sa vidhiVyateV \hfill|| 13 || 
\end{shl}

\begin{artha}
vAjapeVya yAgadalilx elAlx ananxvanunx bayasuvavanige hadineVLu 
ananxgaLanunx parxjApati deVvatege koDuvaMte sheVKarisalapxDuvavu. 
(`sapatxdasha pArxjApatAyxnf pashUnAlaBeVta' eMdu) A vAjapeVya 
yAgavanunx kataRvayxveMdu vidhisuvudu.
\end{artha}

\begin{shl}
reVtasoV\s nanxrasaseyxYva yatArxnAnxhutiriVkaSxyXteV | \\
vAjapeVyABisaMpatAsxyXnemxYthunAKayxkarxtoVrataH || 
\end{shl}

\begin{artha}
parxkaqta ananxrasavAgiruva reVtasasxnenxV ananxveMba AhutiyeMdu yAva meYthuna vAjapeVyadalilx BAvisabeVkeMdu iruvudo, A meYthunaveMba kamaRdalilx vAjapeVya yAgaveMdu ananxvu eraDaralUlx samAnavAgiruvudariMda BAvisabahudu. idanunx vAjapeVyada Palavanunx bayasuvavarige heVLiruvudu.
\end{artha}

\vishaya{I upAsanege adhika PalavU iruvudu {\rm --}}

\begin{shl}
sAmAnAyxditi vijecnxVyaM vAjapeVyaPalAthiRnAmf \hfill|| 14 | \\
sirxVNAM ca sukaqtaM vaqknekxV savaRmAvajaRyeVcuCxBamf | \\
adhoVpahAsaM yoV vidAvxnayxthoVkatxmupaseVvateV \hfill|| 15 || 
\end{shl}

\begin{artha}
yAva upAsakanu hiMde heVLidaMte sitxrXVya taLaBAgavanunx (vAjapeVya 
BAvaneyiMda) seVvisuvano, avanu sitxrXVyara elalx puNayxgaLanunx 
tananx kaDege eLedukoLuLxvanu. \\ aneVvaMviduSaH puNayxM sukaqtaM vaqcnajxteV sitxrXyaH | \\ I riVtiyAgi tiLiyade iruva 
(vAjapeVya BAvaneyilalxdavanu) mADida puNayxvanunx sitxrXVyaru eLadukoLuLxvaru.
\end{artha}

\vishaya{baq. a.6, bArx. 4, kaMDike 4}

\begin{shl}
Etadadadhx samx veY tadivxdAvxnudAdxlaka AruNirAheYtadadhx samx veY tadivxdAvxnAnxkoV mwdagxlayx AheYtadadhx samx veY tadivxdAvxnukxmArahArita Aha bahavoV mayAR bArxhamxNAyanA nirinidxrXyA visukaqtoV\s sAmxlolxVkAtapxrXyanitx ya idamavidAvxMsoV\s dhoVpahAsaM caranitxVti bahu vA idaM supatxsayx vA jAgarxtoV vA reVtaH sakxnadxti ||4||
\end{shl}

\begin{shl}
meYthanoVpaniSatatxtatxvXmeVtadadhx sAmx\s \s ha BAvitaH \hfill|| 16 || 
\end{shl}

\begin{shl}
savaRdeYvA\s \s hutiVviRdAvxcnujxhoVtayxshanxnanxpaH pibanf | \\
upagacaCxnisxrXyaM tadavxdAruNigoVRtarxtaH kila \hfill|| 17 || 
\end{shl}

\begin{artha}
aMdare I kamaRvu vAjapeVya yAgaveMdu BAvisi tiLidavanAgi UTa 
mADutatxlU, kuDiyutatxlU sitxrXVgamana mADutatxlU iruvAga anusaMdhAna 
mADutAtx reVtasisxna rUpada AhutigaLanunx elAlx pArxNiyU yAvAgalU 
hoVma mADuvudeMdu kANutatxliruva AruNiyu (A goVtarxdavanu) I meYthuna 
rahasayxda tatatxvXvanunx heVLidanu.
\end{artha}

\vishaya{AruNiyu Enu heVLidaneMdare? {\rm --}}

\begin{shl}
bArxhamxNA jAtimAterxVNa sirxVBihaqRtashuBAgamAH | \\
ukatxM vidhimajAnanotxV mirxyanetxV meYthuneV ratAH \hfill|| 18 || 
\end{shl}

\begin{artha}
jAti mAtarxvAgidudx bArxhamxNaru sitxrXVyariMda apaharisalapxTaTx - 
tamamx shuBalABavuLaLxvarAgi heVLida I vidhiyanunx ariyadavarAgi 
meYthunadalilx AsakatxrAgidudx sAyuvaru.
\end{artha}

\begin{shl}
bahevxVtaditi vijecnxVyaM parxyoVjanabahutavxtaH | \\
reVtaH sakxnadxti yatusxpwtx jAgarxtoV vA\s pi kAminaH \hfill|| 19 || 
\end{shl}

\begin{shl}
hasetxVnA\s \s laBayx taderxVtaH pashAcxcAcxpayxnumanatxrXyeVtf \hfill|| 20 | \\
yadadayx meV\s pataderxVta OSadhiVrasaracacx yatf | \\
apoV\s gacaCxtasxvXyoVniM ca tadahaM BUya AdadeV \hfill|| 21 || 
\end{shl}

\begin{artha}
idara parxyoVjanavu bahaLavAgiruvudariMda idU bahuvAgide. 
savxpanxdalilx athavA ecacxravAgidAdxga kAmiyAda ivanige hecAcxgi 
athavA kamimxyAgi reVtasusx saKxlane hoMdidarU A reVtasasxnunx 
keYyiMda sapxshiRsi anaMtaraveV aBimaMtirxsabeVku. hAgU punaH 
tegedukoLaLxbeVku (adakekx beVkAda maMtarxvidu)
\end{artha}

\vishaya{baq. a.6, bArx. 4, kaMDike 5}

\begin{shl}
tadaBimaqsheVdanu vA manatxrXyeVta yanemxV\s dayx reVtaH paqthiviVmasAkxnitxsXVdayxdoVSadhiVrapayxsaradayxdapaH || idamahaM taderxVta AdadeV punamARmeYtivxnidxrXyaM punasetxVjaH punaBaRgaH || punaraginxdhiRSANxyX yathAsAthxnaM kalapxnAtxmitayxnAmikAknugxSAThxBAyxmAdAyAnatxreVNa satxnw vA Burxvw vA nimaqjAyxtf || 5 ||
\end{shl}

\begin{artha}
yanemxVdayx itAyxdi maMtarxda athaR - yAva nananx reVtasusx I dina 
keLakekx (nelada meVle) bididxto OSadhigaLalUlx hoVgi seVrito, hAgU 
tananx upAdAna kAraNavAda jaladalUlx hoVgi seVrito, adanunx punaH nAnu 
sivxVkarisuvenu.
\end{artha}

\begin{shl}
aBimashaRnasayx yoV manotxrXV garxhaNasayx sa Eva tu | \\
punamARmeYtu taderxVtoV vijAcnxnaM teVja ucayxteV \hfill|| 22 || 
\end{shl}

\begin{artha}
yAvudanunx ililx aBimashaRnakekx heVLideyo punaH tegedukoLuLxvudakUkx 
adeV maMtarxveMdu tiLiyabeVku. tirugi nananxlilxge A reVtasusx baMdu 
seVrali. punaHsetxVjaH eMbalilx teVjasesxMdare jAcnxnaveMdu 
heVLalapxDuvudu. (adaraMte jAcnxnavU tirugi baMdu seVrali)
\end{artha}

\vishaya{`punaBaRgaH' eMbudara athaR}

\begin{shl}
swBAgayxM punaraginxH sayxH parishiSATxshacx deVvatAH \hfill|| 23 | \\
parxkAshakatAvxtasxvARsAM dhiSANxyXshAcxpayxganxyoV matAH | \\
gamayanutx yathAsAthxnaM deVvA agAnxyXdayoV mama \hfill|| 24 || 
\end{shl}

\begin{artha}
BagaH eMdare sawBAgayxvu (punaH nananxnanx seVrali) hAgeye aginxyu 
(reVtasisxna rUpadalilx horage hoVdadudx nananxlilxge barali) matutx 
uLida deVvategaLU nananxnunx baMdu seVrali. hAgeye samasatx 
deVvategaLa sAthxnagaLu parxkAshakavAgiruvudariMda aginxgaLeMdu ililx 
opipxde. A aginx modalAda deVvategaLu punaH (BUmiyalilx bididxruva 
nananx reVtasasxnunx) savxsAthxnakekx hoMdisali. (eMdu maMtArxthaRvu)
\end{artha}

\vishaya{`anAmikAMguSAThxBAyxM' itAyxdi maMtarxda athaR}

\begin{shl}
aknugxSAThxnAmikABAyxM taderxVta AdAya cA\s \s tamxnaH | \\
satxnw Burxvw vA nimaqjeVnamxdheyxV ca satxnayoVsatxdA \hfill|| 25 || 
\end{shl}

\begin{artha}
hebebxraLu matutx anAmikA eMba beraLugaLiMda A tananx reVtasasxnunx 
tegedukoMDu satxnagaLanunx, hububxgaLanunx adariMda leVpisabeVku. 
alalxde satxnagaLa naDuve leVpisabeVku\footnote[1]{shirxVmaMtha 
kamaRvanunx mADi patinxyu QutukAlavanunx barxhamxcayaRdalelxV idudx 
niriVkiSxsabeVku. A kAlavu odaguva muMceye atirAgadiMda I puruSana 
ecacxrinalolx savxpanxdalolx reVtasusx saKxlaneyanunx hoMdidare AvAga 
`yanemxVadayxreVtaH.......' I maMtarxvanunx heVLutAtx reVtasasxnunx 
japisabeVku. yAva maMtarxdiMda aBimaMtirxsididxto, adeV maMtarxdiMda 
aMguSaThx anAmikeyeMba eraDu beraLugaLiMda A reVtasasxnunx 
tegedukoLaLxbeVku. anaMtara `punamARmA meYtipxMdirxyaM.......kalapxnAtxmf' eMdu heVLutAtx tegedukoMDu 
patinxya satxnagaLa meVlU hububxgaLa meVlU satxnagaLa naDumadhayx 
BAgadalUlx leVpisabeVku eMdu oTuTx tAtapxyaR.}.
\end{artha}

\vishaya{baq. a.6, bArx. 4, kaMDike 6}

\begin{shl}
atha yaduyxdaka AtAmxnaM pasheyxVtatxdaBimanatxrXyeVta mayi teVja inidxrXyaM yashoV darxviNaM sukaqtamiti shirxVhaR vA ESA sitxrXVNAM yanamxloVdAvxsAsatxsAmxnamxloVdAvxsasaM yashasivxniVmaBikarxmoyxVpamanatxrXyeVta ||6||
\end{shl}

\begin{shl}
reVtaHsavxyoVnAvudaka AtAmxnaM ceVtapxrXmAdataH | \\
pasheyxVnamxnetxrXVNa tatotxVyamaneVneYvAnumanatxrXyeVtf \hfill|| 26 || 
\end{shl}

\begin{artha}
reVtasisxna tananx upAdAna kAraNavAda niVrinalilx reVtaH seVcane 
mADuva puruSanu parxmAdadiMda tananxnunx (tananx CAyeyanunx) noVDidedxV 
Adare I `mayi teVja iMdirxyamf.......' eMba maMtarxdiMda A niVranunx 
aBimaMtirxsabeVku\footnote[2]{hAgeye kadAcitf reVtasasxnunx seVcisuva 
puruSanu reVtasisxgU mUlakAraNavAda jaladalilx tananx CAyeyanunx 
parxmAdadiMda noVDidare hiVge noVDida pApakAkxgi `mayiteVjaH.......sukaqtamf' eMdu I 
niVranunx muTiTx japisabeVku. AvAga noVDida pApakekx aparAdhakekx oMdu 
pArxyashicxtatxvAguvudu.}.
\end{artha}

\vishaya{maMtArxthaR {\rm --}}

\begin{shl}
mayi teVjoV\s sutx vijAcnxnamiti reVtoV\s BidhiVyateV | \\
vishiSATxpatayxheVtutAvxjajxpeVdeVvaM tathoVtatxreV \hfill|| 27 || 
\end{shl}

\begin{artha}
nananxlilx teVjasusx eMdare vijAcnxnavu. adU reVtasesxMdu ililx 
heVLalapxDuvudu. utatxma putarx saMpatitxge kAraNavAdadxriMda A 
reVtasusx nananxlilx irali eMdu tiLidu maMtarxvanunx japisabeVku. 
hiVgeye teVjaH shabadxdaMte muMdiruva (`iMdirxyaM, yashoVdarxviNaM 
sukaqtamf' eMbuva) payARMta reVtasesxMdeV athaR mADikoLaLxbeVku.
\end{artha}

\vishaya{avataraNike}

\begin{artha}
reVtaHseVcane mADuva puruSanige deVshakAlagaLanunx miVri parxmAdadiMda 
reVtaHsaKxlaneyAdadadxkekx pArxyashicxtatxvanunx Ivarege heVLidAdxyitu.
\end{artha}

\vishaya{inunx muMde QutukAladalilx BAyARgamanavanunx heVLalu 
`shirxVhaRvA ESA' itAyxdi maMtarxdiMda BUmikeyanunx raciside. 
adaralilx `maloVdAvxsAH' eMbudara athaRvanunx heVLutAtxre {\rm --}}

\begin{shl}
udagxtaM malavadAvxsashacxtutheVR\s hani yatitxrXyAH | \\
tAM maloVdAvxsasaM patinxVmAhusatxtakxmaRkAriNaH \hfill|| 28 || 
\end{shl}

\begin{artha}
sitxrXVyara naDuve I patinxyu lakiSxmXyeV AgiruvaLu. ivaLige Qutu 
kaLedu nAlakxne dinadalilx maladiMda kUDida niVreyu meVlakekx 
tegeyalapxDuvudu. adariMda A patinxyanunx maloVdAvxsaLeMdu A meYthuna 
kamaR mADuvavaru heVLuvaru.
\end{artha}

\vishaya{I sitxrXVyu lakiSxmXyeMdu heVge heVLidudx?}

\begin{shl}
guNADhAyxpatayxPalavatupxSapxBUtatavxkAraNAtf | \\
malavadAvxsasaM pArxhuH shirxyaM shirxVheVtutaH sitxrXyamf \hfill|| 29 || 
\end{shl}

\begin{artha}
guNasaMpananxnAda putarxPalavuLaLx puSapxvatiyAgiruva kAraNadiMda malavadAvxsavuLaLx (QutusAnxtaLAda)vaLanunx sirige kAraNaLAgiruvudariMda shirxV (lakiSxmX)yeMdu shurxti heVLide.
\end{artha}

\vishaya{`tasAmxtf' itAyxdi maMtarxda athaR {\rm --}}

\begin{shl}
catutheVR divaseV sAnxtAM gatAvx tAmupamanatxrXyeVtf | \\
AtamxnoV\s BimuKiVBAveV vAgayxtonxV\s torxVpamanatxrXNamf \hfill|| 30 || 
\end{shl}

\begin{artha}
QutuvAgi nAlakxne dinadalilx sAnxna mADidadx A patinxyanunx upamaMtirxsabeVku. ililx upamaMtarxNaveMdare tananx kaDege tirugalu vAkf parxyatanx mADuvudu.
\end{artha}

\begin{shl}
perxVmoNxVpamanitxrXtA\s payxsemxY pateyxV dadAyxnanx ceVdasw | \\
vasAtxrXBaraNaBoVgAdeyxYrAtamxnoV vashamAnayeVtf \hfill|| 31 ||  
\end{shl}

\begin{artha}
perxVmadiMda karedAgalU I patige Ikeyu avakAshavanunx oMdu veVLe 
koDadidadxre vasatxrX, ABaraNa muMtAda BoVgavasutxgaLiMda tananx 
vashakekx baruvaMte mADikoLaLxbeVku.
\end{artha}

\vishaya{baq. a.6, bArx. 4, kaMDike 7}

\begin{shl}
sA ceVdasemxY na dadAyxtAkxmameVnAmavakirxVNiVyAtAsx ceVdasemxY neYva dadAyxtAkxmameVnAM yaSATxyX vA pANinA voVpahatAyxtikArxmeVdinidxrXyeVNa teV yashasA yasha Adada itayxyashA Eva Bavati || 7 ||
\end{shl}

\begin{shl}
tathA\s puyxkAtx na ceVdadxdAyxdabxlAtAtxM vashamAnayeVtf | \\
upeVyAtAtxmatikarxmayx shApadAnAya roVSitaH \hfill|| 32 || 
\end{shl}

\begin{shl}
inidxrXyeVNa ta itAyxdimanetxrXVNAtha shapeVdurxSA | \\
patishApAdaputArx sA vashamAshu BaveVtatxdA \hfill|| 33 || 
\end{shl}

\begin{artha}
AdarU heVLidarU patinxyu avakAshavanunx koDuvudilalxvAdare Akeyanunx 
balAtakxrisi vasha mADikoLaLxbeVku. kupitanAgi shApavanunx koDalu 
Akeyanunx atikarxmisi saMgama mADabeVku. koVpadiMda `iMdirx yeVNateV' itAyxdi 
maMtarxdiMda shapisabeVku. patishApadiMda Akeyu putarxhiVnaLAgi 
shiVGarxdaleVlx tananx vashavAguvaLu.
\end{artha}

\begin{shl}
shapAsxyXmi tAvxmiti huyxkAtxvX vashaM tAmAnayeVtapxtiH | \\
dadAyxcACxpaBayAtAsx ceVdanuloVmaM tadA\s \s careVtf \hfill|| 34 || 
\end{shl}

\begin{artha}
ninanxnunx shapisuveneMdu heVLi patiyu avaLanunx tananx vasha 
mADikoLaLxbeVku. AvAga Akeyu shApa BayadiMda avakAshavanunx koDuvaLu. 
AvAga anukUlavAgi tAnu naDeyabeVku.
\end{artha}

\vishaya{anukUlavAda AcaraNeyeMdare Enu? {\rm --}}

\begin{shl}
atha shApaBayAdadxdAyxtapxteyxV kAmitamAdarAtf | \\
tadA nivataRyeVcACxpaM manetxrXVNAneVna satapxtiH \hfill|| 35 || 
\end{shl}

\begin{artha}
anaMtara shApada BayadiMda patige iSaTxvAdudanunx AdaradiMda Akeyu 
koDuvaLu. AvAga A oLeLxya patiyu I maMtarxdiMda shApavanunx 
hiMtirugisabeVku.
\end{artha}

\vishaya{`sayAMkAmayeVta......' itAyxdi maMtarxda athaR {\rm --}}

\begin{shl}
puruSadevxVSiNiVM BAyARM patishecxVdaBikAmayeVtf | \\
mAmiyaM kAmayeVteVti kuyARtatxsAyx imaM vidhimf \hfill|| 36 || 
\end{shl}

\begin{artha}
puruSananunx devxVSisuva heMDatiyanunx patiyu kAmisuvudAdare 
`nananxnunx Ikeyu kAmisali' eMdu avaLigAgi I vidhiyanunx AcarisabeVku.
\end{artha}

\vishaya{baq. a.6, bArx. 4, kaMDike 9}

\begin{shl}
sa yAmiceCxVtAkxmayeVta meVti tasAyxmathaRM niSAThxya muKeVna muKaM sanAdhxyoVpasathxmasAyx aBimaqshayx japeVdaknAgxdaknAgxtasxmaBxvasi haqdayAdadhijAyaseV ||sa tavxmaknagxkaSAyoV\s si digadhxvidadhxmiva mAdayeVmAmamUM mayiVti || 9 ||
\end{shl}

\vishaya{muMde heVLuva kamARdhikAravu elalxrigU iruvudilalx - hAgAdare 
yArige?}

\begin{shl}
ukatxmanathxvidhAneVna  caritavarxta Eva sanf | \\
utatxreVSavxpi kAyeVRSu savaRM tadanuvataRyeVtf \hfill|| 37 || 
\end{shl}

\begin{artha}
hiMde heVLida maMthakamaRvanunx Acarisi niyamavanunx AcarisidavanAgi 
muMde heVLuva kamaRgaLalilx kataRvayxvAdadedxlAlx Acarisatakakxdudx.
\end{artha}

\vishaya{adeVneMdare - `tasAyx mathaRmf' itAyxdi maMtArxthaR {\rm --}}

\begin{shl}
sitxrXVlakaSxNeV parxveVshAyxnatxrAtimxVyaM puMsatxvXlakaSxNamf | \\
vakatxrXM vaketxrXVNa saMdhAya sapxqqSoTxvXVpasathxM japeVdatha \hfill|| 38 || 
\end{shl}

\begin{artha}
sitxrXVyoVniyalilx tananx liMgavanunx oLapaDisi muKavanunx muKadoDane 
seVrisi upasethxyanunx sapxshiRsi muMde heVLida maMtarxvanunx 
japisabeVku.
\end{artha}

\vishaya{vashiVkaraNa mADuva maMtArxthaR - `aknAgx daknAgxtf' itAyxdi 
maMtarxda modalaneya padada athaR {\rm --}}

\begin{shl}
aknAgxdaknAgxtasxMBavasi jagAdhxnanxpariNAmataH | \\
rasAcoCxVNitamitAyxdikarxmAcuCxkarxtayA mama \hfill|| 39 || 
\end{shl}

\begin{artha}
eleY reVtasesx? nananx AyAya aMgadiMda niVnu huTuTxve heVgeMdare? - 
nAnu tiMda AhArada pariNAmavAda rasadiMda rakatx itAyxdi 
karxmadiMda shukarxvAgi pariNAma hoMdi idariMda niVnu huTuTxve.
\end{artha}

\vishaya{haqdayAdadhi jAyaseV - itAyxdi maMtarxda athaR}

\begin{shl}
shukarxparxvahayA nADAyx haqdayAcAcxBijAyaseV | \\
sa tavxmaknagxkaSAyoV\s si digadhxvidAdhxM maqgiVmiva | \\
senxVhoVparoVdhAdeVveYtAM patinxVM meV vashamAnaya \hfill|| 40 || 
\end{shl}

\begin{artha}
haqdayada mUlaka shukarxvanunx parxvahisuva (harisuva) nADiya mUlaka 
niVnu huTuTxve. niVnu aMgagaLa kaSAyarasavAgiruve. auSadhadiMda 
leVpisida bANadiMda hoDeyalapxTaTx heNuNx jiMkeyaMte. nananx 
perxVmavanunx taDedidadxriMda I nananx patinxyanunx nananx vashakekx 
baruvaMte mADu.
\end{artha}

\vishaya{baq. a.6, bArx. 4, kaMDike 10}

\begin{shl}
atha yAmiceCxVnanx gaBaRM dadhiVteVti tasAyxmathaRM niSAThxya muKeVna muKaM sanAdhxyABipArxNAyxpAnAyxdinidxrXyeVNa teV reVtasA reVta Adada itayxreVtA Eva Bavati || 10 ||
\end{shl}

\begin{artha}
I maMtarxda athaR {\rm --}
\end{artha}

\begin{shl}
mA biBagaRBaRmiteyxVmatha yAM kAmayeVta saH | \\
rUpaBarxMshoV hi Bavati yatoV gaBaRsayx dhAraNeV \hfill|| 41 || 
\end{shl}

\begin{shl}
tathA ywvanahAnishacx tasAmxdeVvaM sa kAmayeVtf | \\
tasAyxM savxmathaRM niSAThxya muKeVneVtAyxdi pUvaRvatf \hfill|| 42 || 
\end{shl}

\begin{artha}
A patiyu yAva BAyeRyanunx kAmisuvano, Akeyu gaBaRvanunx dharisadirali 
eMdu (aBipArxyapaTaTxre muMde heVLuvaMte AcarisabeVku) (hiVge Eke 
aBipArxyapaDuvaneMdare) gaBaRvanunx dharisuvalilx rUpavu nAshavAguvudu 
matutx yawvavxnakUkx hAniyAguvudeMdu I kAraNadiMda hiVge avanu 
bayasuvanu. AvAga avaLa yoVniyalilx tananx liMgavanunx irisi 
muKadoDane muKavanunx seVrisi - anaMtara
\end{artha}

\begin{shl}
pArxNAyx\s \s dw reVcakaM kaqtAvx\s pAnayeVtatxdananatxramf \hfill|| 43 || \\
niSikatxmapi taderxVtaH pArxNavaqtAtxyX yathAvidhi | \\
apAnavaqtAtxyX tadadhxvXsatxmiteyxVtakxmaRNaH Palamf \hfill|| 44 || 
\end{shl}

\begin{artha}
modalu pArxNayx = reVcakavanunx mADi keLakekx biDabeVku. anaMtara 
adaralilx biTaTx reVtasasxnunx pArxNavAyxpAradiMda niyamAnusAra 
adhoVmuKavAgi biDuva apAnavAyxpAradiMda adeV dAvxradiMda punaH 
tegedukoLuLxveneMdu aBimAna mADikoLaLxbeVku. AvAga A reVtasusx 
nAshavAguvudu. ide I kamaRda Palavu.
\end{artha}

\vishaya{I riVtiyAgi BAvisi I pAshada kamaRvanunx mADuvavanu `inidxrXyeVNa' 
itAyxdi maMtarxdiMda patinxyanunx aBimaMtirxsabeVku.}

\footnotetext[1]{meYthuna mADuva kAladalilx modalu tananx liMgada mUlaka 
BAyeRya yoVniyalilx pArxNavAyuvanunx biTuTx adeV dAvxradiMdaleV nAnu 
punaH nananx reVtasasxnunx hiMde tegedukoMDidedxVneMdu aBimAna 
mADikoLaLxbeVkeMdu idara athaR.}
\begin{shl}
inidxrXyeVNa ta itAyxdimanotxrXVkAtxyX tAM parAmaqsheVtf \hfill|| 45 || \\
\footnotemark[1]inidxrXyeVNeYva tavxderxVtoV reVtasA AdeVdeV savxyamf | \\
areVtA Eva sA sitxrXV sAyxdeVvaM patAyx\s BimanitxrXtA \hfill|| 46 || 
\end{shl}

%% shloka footnote
\begin{artha}
`inidxrXyeVNata' itAyxdi maMtarxvanunx heVLi A patinxyanunx sapxshiRsabeVku. 
 maMtArxthaR:- iMdirxyadiMdaleV reVtasisxniMdaleV ninanx reVtasasxnunx 
nAnu tegedukoLuLxvenu. idariMda patiyu aBimaMtirxsidadx A sitxrXVyu 
reVtasisxlalxdavaLAguvaLu.
\end{artha}

\vishaya{baq. a.6, bArx. 4, kaMDike 11}

\begin{shl}
atha yAmiceCxVdadxdhiVteVti tasAyxmathaRM niSAThxya muKeVna muKaM sanAdhxyApAnAyxBipArxNAyxdinidxrXyeVNa teV reVtasA reVta AdadhAmiVti gaBiRNeyxVva Bavati || 11 ||
\end{shl}

\vishaya{idara athaR {\rm --}}

\begin{shl}
dadhiVta gaBaRmiteyxVvaM yAmiceCxVtapxtiraknagxnAmf | \\
tAmapAnayx parxyatenxVna pArxNAyxnamxnetxrXVNa kArayeVtf | \\
inidxrXyeVNa ta ituyxkAtxyX AdadhAmiVti satapxtiH \hfill|| 47 || 
\end{shl}

\begin{artha}
patiyu yAva sitxrXVyanunx gaBaRvanunx Ikeyu dharisali eMdu 
apeVkeSxpaDuvano, avanu Akeyanunx udedxVshisi (avaLalilx) 
parxyatanxpUvaRka \footnote[2]{tananx jananeVMdirxyada mUlaka BAyeRya 
jananeVMdirxyadiMda reVtasasxnunx eLedukoMDu putorxVtapxtitxyAgalu 
samathaRvAyiteMdu tiLidu tananx reVtasisxnoDane kUDisi 
adaralilxrisabeVku. ideV vAtiRkada apAna shabadxda athaR. 
pArxNavAyxpAraveMdare hiVge mADi anaMtara reVtasesxVcana 
mADabeVkeMbude, adU saha muMde heVLida maMtarxdiMda}apAnavAyxpAravanunx mADi (tananx 
viVyaRdoDane Akeya shoVNatavu kUDuvaMte mADi) maMtarxdiMda 
pArxNavAyxpAravanunx mADabeVku. A maMtarxvidu ``inidxrXyeVNateV reVtasAreVta AdadhAmi" eMbudu 
maMtarxdiMda nAnu irisuvenu.
\end{artha}

\vishaya{adhayasayx jAyAyeY - itAyxdi maMtarxda tAtapxyaR {\rm --}}

\begin{shl}
athA\s \s BicArikaM kamaR parxsaknAgxdaBidhiVyateV | \\
upAyatevxVna vijacnxpetxyXY sheyxVnavananx vidhiVyateV \hfill|| 48 || 
\end{shl}

\begin{artha}
\footnote[1]{vivAhavAda naMtaraveMdathaR.}anaMtara I \footnote[2]{I 
pashukamaRvanunx heVLuva saMdaBaRdalelxV upapatiyAda jArapuruSananunx 
nAshamADabeVkeMdu aBilASepaDuva patiyu I aBicArikaveMba shaturxmAraNa 
kamaRvanunx mADaleMdu, jArapuruSana mAraNakekx idu upAyaveMdu 
tiLisuvudakAkxgi shurxtiyu boVdhiside. shurxtiyalilx 
`sheyxVneVnABicaranf yajeVta' eMbudAgi tiLisida sheyxVnayAgadaMte 
idoMdu sAdhana. Adare mADaleVbeVkeMdu idanunx shAsatxrXvu vidhisilalx. 
EkeMdare - `mAhiMsAyxtf savARBUtAni' eMdu yAva pArxNiyanunx 
vadhisabAradeMdu niSeVdhisiruvudariMda pApakaravAda I hiMseyanunx 
vidhisuva udedxVshavanunx shurxtiyu iTuTxkoMDilalx. 
shaturxmAraNaveVnoV idariMda sididhxsuvudu. hAgeye adara pApadiMda 
narakavU sidadhxvAguvudu. narakavu baMdarU ciMteyilalx. 
shaturxnAshavAdare sAkeMdu taqpitxpaDuva hiMsABilASiyAdavanu idanunx 
tiLiyalu shurxtiyu boVdhiside eMdu tAtapxyaR.}parxsaMgadalilx aBicArikaveMba 
kamaRvanunx heVLuvudu. idoMdu sheyxVnayAgadaMte sAdhanaveMdu (shaturx 
hiMsege sAdhanaveMdu) tiLisuvudakAkxgi. Adare idu kataRvayxveMdu 
shurxtiyiMda vidhisalapxDuvudilalx.
\end{artha}

\vishaya{baq. a.6, bArx. 4, kaMDike 12}

\begin{shl}
atha yasayx jAyAyeY jAraH sAyxtatxM ceVdidxvXSAyxdAmapAterxV\s ginxmupasamAdhAya parxtiloVmaM sharabahiRsitxVtAvxR tasimxnenxVtAH sharaBaqSiTxVH parxtiloVmAH sapiRSAkAtx juhuyAnamxma samidedhxV\s hwSiVH pArxNApAnw ta AdadeV\s sAviti mama samidedhxV\s hwSiVH putarxpashUMsatx AdadeV\s sAviti mama samidedhxV\s hwSiVH \footnotemark[1]{}iSATx\footnote[2]{}sukaqteV ta AdadeV\s sAviti mama samidedhxV\s hwSiV\footnotemark[3]{}rAshA\footnotemark[4]{}parAkAshw ta AdadeV\s sAviti sa vA ESa nirinidxrXyoV visukaqtoV\s sAmxlolxVkAtepxrXYti yameVvaMvidAbxrXhamxNaH shapati tasAmxdeVvaMvicoCxrXVtirxyasayx dAreVNa noVpahAsamiceCxVduta heyxVvaMvitaparoV Bavati || 12 ||
\end{shl}
\footnotetext[1]{iSaTxM = shawrxtakamaRgaLu}
\footnotetext[2]{sukaqtaM = sAmxtaRkamaRgaLu}
\footnotetext[3]{AshA = pArxthaRne}
\footnotetext[4]{parAkAshaH = vAcA yAvudanunx mADutetxVneMdu heVLi kamaRNA adanunx mADade idudx adanunx niriVkeSx mADuvudeV}

\vishaya{I maMtarxda vAyxKAyxna}

\begin{shl}
atha yasayx gaqhasathxsayx patAnxyX jAroV BaveVkavxcitf | \\
taM ceVdUdxSAyxdurxSeYveYnAmAraBeVta tadA kirxyAmf \hfill|| 49 || 
\end{shl}

\begin{artha}
anaMtara yAva gaqhasathxna patinxge yArAdarU jAra puruSanobabxnu elilxyAdarU idadxre avananunx patiyu roVSadiMda devxVSisuvudAdare AvAga I parxtikirxyeyanunx AraMBisabeVku.
\end{artha}

\vishaya{``divxSAyxtf" eMbudu EtakAkxgi?}

\begin{shl}
na hayxdivxSaTxmanasakxsayx kameYRtatisxdidhxmashunxteV | \\
atoV\s dhikArivijacnxpetxyXY divxSAyxditi visheVSaNamf \hfill|| 50 || 
\end{shl}

\begin{artha}
devxVSa mADada manasusxLaLxvanige I kamaRvu sididhxyanunx hoMduvudilalx. adariMda adhikAravanunx tiLisalu `divxSAyxtf' eMdu visheVSaNavanunx koTiTxde.
\end{artha}

\vishaya{`AmapAterxV' eMdu Etakekx tegedukoMDide?}

\begin{shl}
AmapAterxV\s ginxmituyxkAtxyX hayxBicArAKAyxkamaRNaH | \\
yoVgayxteYvA\s \s mapAtarxsayx BiduratavxsamanavxyAtf \hfill|| 51 || 
\end{shl}

\begin{artha}
`AmapAtarx'dalilx aginxyanunx sAthxpisi eMdu heVLidadxriMda I ABicArika kamaRkekx AmapAtarxda yoVgayxteyanenx tiLiside. (aMdare hasiyAda maDakeyeV I kamaRkekx yoVgayxveMdu tiLiside.) adu heVgeMdare, jArapuruSananunx siVLi hAkuvudeV I kamaRda PalavAdadxriMda adakekx takakxdAda pAterx hasiyAdare siVLuvudeMba athaR saMbaMdhavu baruvudu. (adariMda adeV yoVgayx)
\end{artha}

\vishaya{visheVSaNada tAtapxyaR vaNaRneya upasaMhAra {\rm --}}

\begin{shl}
yathA\s \s maM BiduraM pAtarxmapusx sadoyxV viliVyateV | \\
pArxpatxjAroV\s pi meV shaturxsatxtheYvA\s \s shu vidiVyaRtAmf \hfill|| 52 || 
\end{shl}

\begin{artha}
heVge hasiyAda maNiNxna pAterxyu niVrinalilx kUDale karagi hoVguvudo, hAgeyeV namamxlilxge baMda jArapuruSanu nananx shaturxvAgiralu shiVGarxvAgi siVLihoVgali.
\end{artha}

\vishaya{`aginxmf' aMdare yAva aginxyanunx ililx tegedukoLaLxbeVku?}

\begin{shl}
aginxmiteyxVkavacanAdulilxKAyxdeVshacx liknatxH | \\
AvasathAyxginxnideVRshoV na tu terxVtAginxsaMgarxhaH \hfill|| 53 || 
\end{shl}

\begin{artha}
`aginxmf' eMdu EkavacanadiMdalU ulilxKayx itAyxdi gamaka liMgadiMdalU `Avasathayx' veMba aginxyanunx nideRVshiside, Adare terxVtAginxgaLige ililx garxhaNavilalx.
\end{artha}

\vishaya{`parxtiloVmamf' itAyxdi vacanada athaR}

\begin{shl}
parxtiloVmamavasitxVyaR kamaRNaH parxtiloVmataH | \\
sharabahiRH parxyatenxVna vidAvxnorxVSasamanivxtaH \hfill|| 54 || 
\end{shl}

\begin{shl}
tasimxnanxgwnx shareVSiVkA GaqtAkAtx juhuyAdatha | \\
jArasayx doVSaM parxKAyxpayx manetxrXVNAneVna satavxraH \hfill|| 55 || 
\end{shl}

\begin{artha}
kamaRvu parxtikUlavAgiruvudariMda sharaveMba daBeRyanunx parxtiloVmavAgi parisatxraNa hAki sAvadhAnavAgi I vidAvxnf (pArxNoVpAdakanAgi maMthakamaRvanunx Acarisidavanu) roVSadiMda kUDidudx sharaveMba taqNada kaDiDxgaLanunx tupapxdalilx nenesi A aginxyalilx hoVma mADabeVku. `mama' itAyxdi maMtarxdiMda jArapuruSana doVSavanunx parxkaTapaDisi tavxreyiMda hoVma mADabeVku.
\end{artha}

\vishaya{maMtarx vAyxKAyxna {\rm --}}

\begin{shl}
mama savxBUteV yoVSAgwnx samidedhxV ywvanAdinA | \\
shukArxhutiM yatoV\s hwSiVreVSa teV\s tarx vayxtikarxmaH \hfill|| 56 || 
\end{shl}

\begin{shl}
AdadeV\s toV\s parAdhAtetxV pArxNApAnw jijiVviSoVH | \\
PaTAkxreVNeYva juhuyAcaCxraBaqSiTxVyaRthoVditAH \hfill|| 57 || 
\end{shl}

\begin{artha}
nananx savxtAtxgiruva sitxrXVyeMba aginxyalilx yawvavxna modalAda kAraNadiMda uriyutitxruvAga adaralilx reVtasesxMba Ahutiyanunx yAvudariMda hoVma mADideyo idu ninanx atikarxmaNavAyitu. I aparAdhadiMda badukirabeVkeMdu bayasuva ninanx pArxNa apAnagaLanunx nAnu seLedukoLuLxveneMdu maMtArxthaR, AdadeV eMbudara muMde PaTf eMba pada parxyoVgadiMda sharaveMba taqNada kaDiDxgaLanunx (viloVmavAgiruvaMte mADi tupapxdiMda leVpisi) hoVma mADabeVku.
\end{artha}

\vishaya{hAgeyeV itara maMtarxgaLiMdalU hoVma mADabeVku. avugaLa athaR saMkeSxVpa {\rm --}}

\begin{shl}
tathA putArxnapxshUMshecxYva AdadeV teV\s dayx kAmuka | \\
shwrxtamiSiTxM vijAniVyAtAsxmXtaRM sukaqtamitayxpi \hfill|| 58 || 
\end{shl}

\begin{artha}
eleY kAmuka? ninanx hAgU putarxranUnx pashugaLanunx IgaleV sivxVkarisuve - eMdu maMtArxthaR. ililx iSaTxveMdare shawrxtakamaR, sukaqtaM eMdare sAmxtaRkamaR.
\end{artha}

\begin{shl}
shwrxtaM sAmxtaRM ca yatikxMcitupxNayxM kamaR tavxyA kaqtamf | \\
tatasxvaRM ta AdadeV\s hamAhutiM parxkiSxpeVdurxSA \hfill|| 59 || 
\end{shl}

\begin{artha}
shawrxta matutx sAmxtaR rUpavAda yAvudeV oMdu puNayxkamaRvanunx hiMde mADidadxrU adelalxvanunx nAnu tegedukoLuLxveneMdu (maMtArxthaRvanunx tiLidu) koVpadiMda Ahutiyanunx aginxyalilx hAkabeVku.
\end{artha}

\begin{shl}
pArxthaRnA\s \s sheVti vijecnxVyA parAkAshA parxtiVkaSxNamf | \\
AdAnAnotxV BaveVnamxnatxrXH savaRterxYvaM vinidiRsheVtf \hfill|| 60 || 
\end{shl}

\begin{artha}
maMtarxdalilxruva AshA eMbudu pArxthaRne. parAkAshaH eMbudu parxtiVkaSxNe. maMtarxvu AdadeV eMbuva payaRMta, elalx maMtarxgaLalUlx ideV riVtiyAgi tiLiyabeVku.
\end{artha}

\vishaya{parxtiVkaSxNaveMdare Enu? {\rm --}}

\begin{shl}
vAknABxterxVNa parxtijAcnxtaM kamaRNA noVpapAditamf | \\
tatapxrXtiVkaSxNamAkAknAkxSX parAkAsheVti BaNayxteV \hfill|| 61 || 
\end{shl}

\begin{artha}
bAyimAtiniMda mAtarx parxtijecnx mADidadxnunx kirxyeyiMda naDeyisade 
iruvudeV parxtiVkaSxNa. hiVge iciCxsuvudeV parAkAsha eMbudu.
\end{artha}

\vishaya{ABicArakakamaRda Palavanunx nirUpisuvudu {\rm --}}

\begin{shl}
sa vA itAyxdinA\s thAsayx Palamukatxsayx kamaRNaH | \\
BaNayxteV vacasoVkatxsayx nirinidxrXyapuraHsaramf \hfill|| 62 || 
\end{shl}

\begin{artha}
anaMtara `sa vA ESa nirinidxrXyaH' itAyxdi vAkayxdiMda Itanige hiMde 
heVLida kamaRda Palavanunx, vacanadiMda heVLida shaturxvige 
niriMdirxya modalAda riVtiyalilx heVLuvudu.
\end{artha}

\vishaya{`tasAmxditAyxdi' vAkayxda avataraNike {\rm --}}

\begin{shl}
niHsheVSapuruSAthARpitxloVpakaqtakxmaR vaNiRtamf | \\
jAyayA meYthunAKayxM yacoCxrXVtirxyasayx vipashicxtaH \hfill|| 63 || 
\end{shl}

\begin{artha}
shorxVtirxyanAda vidAvxMsana BAyeRyoDane mADuva meYthunaveMba kamaRvu samasatx puruSAthaRlABakUkx loVpavanunx uMTumADuvudeMdu vaNiRsidAdxyitu.
\end{artha}

\begin{shl}
upahAsamatoV neVceCxVtAsxdhaRM shorxVtirxyajAyayA | \\
meYthunaM tu visheVSeVNa huyxkAtxnathaRjihAsayA \hfill|| 64 || 
\end{shl}

\begin{artha}
idariMda shorxVtirxyana patinxyoDane hAsayxvanunx saha mADalu bayasabAradu, meYthunavanAnxdaro visheVSa riVtiyalilx hiMde heVLida anathaRvanunx tapipxsikoLaLxlu savaRthA bayasabAradu.
\end{artha}

\vishaya{atheVtAyxdi maMtarxda avataraNike {\rm --}}

\begin{shl}
kamAR\s \s BicArikaM porxVkatxM parxsaknAgxnanx parxdhAnataH | \\
yadathaRsutx parxyAsoV\s yaM tatakxmARthaH parxpacnacxyXteV \hfill|| 65 || 
\end{shl}

\begin{artha}
Ivarege ABicArika kamaRvanunx pashukamaRda parxsaMgadalilx heVLidudx, aSeVTx. muKayxvAgi udedxVshisi heVLilalx. inunx muMde putarx saMtAna paDeyalu yAvudanunx AraMBisi heVLideyo A kamaRdalilx (dhamaRkalApavanunx) visatxrisuvudu.
\end{artha}

\vishaya{baq. a.6, bArx. 4, kaMDike 13}

\begin{shl}
atha yasayx jAyAmAtaRvaM vinedxVtatxrXyXhaM kaMseVna pibeVdahatavAsA neYnAM vaqSaloV na vaqSaluyxpahanAyxtitxrXrAtArxnatx Apulxtayx virxVhiVnavaGAtayeVtf || 13 ||
\end{shl}

\begin{shl}
yasayx manathxvidhijacnxsayx jAyAM ceVdAtaRvaM varxjeVtf | \\
tisorxV rAtirxVnaR kAMseyxVna pAnaM kuyARtatxthA\s shanamf  \hfill|| 66 || 
\end{shl}

\begin{artha}
maMthakamaRda vidhAnavanunx tiLida yAva gaqhasathxna BAyeRyu yAvAga 
Qutu dhamaRvanunx hoMduvaLo, AvAga avana BAyeRyu mUru rAtirxgaLu 
kaMcina pAterxyiMda (niVranunx kuDiyabAradu) pAnavanunx mADabAradu. 
hAgU UTavanunx adaralilx mADabAradu.
\end{artha}

\vishaya{`ahatavAsAH' idara athaR {\rm --}}

\begin{shl}
tatheYvAhatavAsAH sAyxdahaHsevxVtuSu shudadhxdhiVH | \\
vaqSaloV vaqSaliV veYnAM noVpahanAyxtakxdAcana \hfill|| 67 || 
\end{shl}

%% shloka footnote
\begin{artha}
matutx I mUru dinagaLalUlx avaLu QutudoVSadiMda kaluSitavAda vasatxrXvuLaLxvaLeMdalalx, AvAgalU shudadhxvAda aMtaHkaraNavuLaLxvaLAgirabeVku. Ikeyanunx (sAnxna mADidavaLanenxV sAnxna mADadavaLanunx) vaqSalanAgali vaqSaliyAgali yAva riVtiyalUlx muTaTxbAradu.
\end{artha}

\begin{shl}
anoyxV vA pApakaqtakxshicxtasxpXshaRsaMBASaNAdiBiH | \\
varxtasAthx noVpahanAyxtAtxmaBiVpisxtaPalApatxyeV \hfill|| 68 || 
\end{shl}

\begin{artha}
beVre yArobabx pApiyidadxrU Atanu sapxshaR, saMBASaNe muMtAdavugaLiMda varxtasathxLAda A shorxVtirxya patinxyanunx keDisabAradu. EkeMdare? satavxtarx jananaveMbuva iSaTxPalavu laBisuvudakAkxgi keDisabAradu.
\end{artha}

\vishaya{`sAtirxrAtArxnetxV' eMbudara athaR {\rm --}}

\begin{shl}
sA tirxrAtArxnatx ApulxtAyxhatavAsAH shuciH satiV | \\
sharxpaNAya caroVBaRtAR virxVhiVMsAtxmavaGAtayeVtf \hfill|| 69 || 
\end{shl}

\begin{artha}
A patinxyu mUru rAtirxgaLu kaLeda meVle (nAlakxne dina) sAnxnavanunx mADi shudadhx vasatxrXvanunx uTuTx shuciyAgiruvaLu. AvAga patiyu caruvanunx pAkamADalu avaLiMda batatxvanunx kuTiTxsabeVku.
\end{artha}

\vishaya{baq. a.6, bArx. 4, kaMDike 14}

\begin{shl}
sa ya iceCxVtupxtorxV meV shukolxV jAyeVta veVdamanuburxviVta savaRmAyuriyAditi kiSxVrwdanaM pAcayitAvx sapiRSamxnatxmashinxVyAtAmiVshavxrw janayitaveY || 14 ||
\end{shl}

\begin{shl}
shukolxV gwroV\s tarx vijecnxVyaH shukolxV vA baladeVvavatf | \\
suvAyxKeyxVyatavxtaH sheVSaH savxyameVvAvAgamayxtAmf \hfill|| 70 || 
\end{shl}

\begin{artha}
I maMtarxdalilx shukalx eMdare haLadi baNaNxdavaneMdu tiLiyabeVku. athavA balarAmanaMte shukalx (shudadhx) eMdAdarU tiLiyabahudu. uLidadadxnunx nAveV suKavAgi vAyxKAyxna mADabahudAdadxriMda tAnAgiyeV tiLidukoLaLxbahudu.
\end{artha}

\vishaya{kiSxVrawdanamitAyxdi maMtArxthaR {\rm --}}

\begin{shl}
kiSxVrwdanaM tayeYvAtha pAcayitAvx savxyaM patiH \hfill|| 71 || \\
daMpatiV GaqtavanatxM tamashinxVyAtAmathwdanamf | \\
savxtanAtxrXviVshavxrw sAyxtAM satupxtarxparxsavAya tw \hfill|| 72 || 
\end{shl}

\begin{artha}
patiyu kiSxVrAnanxvanunx A patinxyiMdaleV pAkamADisi daMpatigaLibabxrU tupapxdoMdige ananxvanUnx BuMjisabeVku. idariMda satupxtarxnanunx utApxdisalu A daMpatigaLu savxtaMtarxrAgabalalxru.
\end{artha}

\vishaya{oTuTx 14ne kaMDikeya tAtapxyARthaR}

\begin{artha}
yAva gaqhasathxnu tanage beLaLxge iruva maganu huTaTxleMdu bayasuvano avanu oMdu veVdavanunx paThisabeVkeMdU matutx nUru vaSaRgaLeMba pUNARyusasxnunx hoMdabeVkeMdu bayasuvano avanu tananx heMDatiyiMdale kiSxVrAnanxvanunx beVyisi tupapxvanunx seVrisi daMpatigaLibabxrU UTa mADabeVku. adariMda aMtaha satupxtarxnanunx paDeyalu samathaRrAguvaru.
\end{artha}

\vishaya{baq. a.6, bArx. 4, kaMDike 15}

\begin{shl}
atha ya iceCxVtupxtorxV meV kapilaH piknagxloV jAyateV dwvx veVdAvanuburxviVtf savaRmAyuriyAditi dadhoyxVdanaM pAcayitAvx sapiRSamxnatxmashinxVyAtAmiVshavxrw janayitaveY || 15 ||
\end{shl}

\vishaya{tAtapxyARthaR}

\begin{artha}
yAva gaqhasathxnu nanage haLadi baNaNxda maganu athavA hoMbaNaNxda maganu huTaTxleMdu bayasuvano avanu eraDu veVdagaLanunx paThisabeVkeMdU, matutx pUNARyusasxnunx hoMdaleMdu bayasuvano avanu patinxyiMda mosarananxvanunx mADisi tupapxdoMdige ibabxrU BuMjisabeVku. adariMda avaribabxru iMtaha putarxnanunx janisalu samathaRrAguvaru.
\end{artha}

\begin{shl}
yathoVkatxputarxparxsaveV yadi vA kiSxparxkAriNw | \\
yathoVkatxkamaRNeYteVna sAyxtAM tAveVva daMpatiV \hfill|| 73 || 
\end{shl}

\begin{artha}
hiMde heVLida putorxVtApxdane mADalu hiMde heVLida caruhoVmAdi kamaRdiMdaleV A daMpatigaLu samathaRrAguvaru. athavA shiVGarxvAgi putarxnanunx utApxdisalu samathaRrAguvaru.
\end{artha}

\vishaya{veVdAdhayxyanakekx anadhikAriyAda magaLige adara jAcnxnavanunx bayasuvudu heVge? `duhitA meV paMDitA' eMbudu heVge saMgata? eMdare}

\begin{shl}
duhatA paNiDxteVtayxtarx sitxrXVNAmucitakamaRsu | \\
tatApxNiDxtayxmiha jecnxVyaM na tu veVdAthaRgoVcaramf \hfill|| 74 || 
\end{shl}

\begin{artha}
`duhitA paMDitA' eMbuvalilx sitxrXVyarige yoVgayxvAda kamaRgaLa viSayadalilx aMdare gaqhakaqtAyxdigaLalilx beVkAda pAMDitayxvanenxV ililx heVLide. Adare veVdAthaRda viSayadalilx heVLidadxlalx.
\end{artha}

\vishaya{vijigiVta itAyxdi maMtarxBAgavanunx vAyxKAyxnisuvudu {\rm --}}

\begin{shl}
vigitaH jagatayxsimxnanxtayxthaRM yoV vishabidxtaH | \\
vidavxtasxBA ca samitisatxdoyxVgayxH samitiMgamaH \hfill|| 75 || 
\end{shl}

\begin{artha}
vi shabadxdiMda atayxMta eMdathaRvu gArxhayx. I jagatitxnalilx bahaLa hogaLisikoMDavaneMdu `vijigiVtaH' eMbudara athaR. `samitiMgamaH' eMbalilx samitiyeMdare vidAvxMsara saBe. adakekx yoVgayxnAdavane samitiMgamaH eMbudara athaR.
\end{artha}

\vishaya{baq. a.6, bArx. 4, kaMDike 18}

\begin{shl}
atha ya iceCxVtupxtorxV meV paNiDxtoV vigiVtaH samitiknagxmaH shushUrxSitAM vAcaM BASitA jAyeVta savARnevxVdAnanuburxviVta savaRmAyuriyAditi mAMswdanaM pAcayitAvx sapiRSamxnatxmashinxVyAtAmiVshavxrw janayitavA aukeSxVNa vASaRBeVNa vA || 18 ||
\end{shl}

\vishaya{`savARnf veVdAnf' eMbudakekx athaRBeVda {\rm --}}

\begin{shl}
parxkaqtAyAM tirxsaMKAyxyAM savaRshabadxparxyoVgataH | \\
parxtiVyAcacxturoV veVdAnasxvaRshabAdxthaRvitatxyeV \hfill|| 76 || 
\end{shl}

\begin{artha}
`tirxVnf veVdAnf' eMdu hiMde parxsutxtavAda mUru saMKeyxyalilx savaRshabadx parxyoVgavanunx mADiruvudariMda (ililx punarukitxyilalxdaMte) savaRshabAdxthaR jAcnxnavAgalu nAlukx veVdagaLanunx savaRshabadxdiMda tegedukoLaLxbeVku.
\end{artha}

\vishaya{`mAMsawdanaM' eMba shabAdxthaR {\rm --}}

\begin{shl}
taNuDxlAnAmxMsasaMmishArxnapxkf tAvx mAMswdanaM viduH | \\
ukASx seVcanashakotxV gwH sa Eva QuSaBoV mahAnf \hfill|| 77 || 
\end{shl}

\begin{artha}
mAMsa mishirxtavAda akikxyanunx beVyisidalilx A ananxvanunx 
mAMsawdanaveMdu heVLuvaru. `aukeSxVNa' eMbuvalilx ukASx eMdare 
viVyaRvanunx seVcane mADalu shakatxvAda (gaMDu goVvu) vaqSaBa. adeV 
savxlapx doDaDxdAgidadxre QuSaBaveMdu `ASaRBeVNa' eMbuvalilx 
heVLalapxTiTxde.
\end{artha}

\vishaya{aukeSxVNa eMbudAgi mAMsakekx visheVSaNavAgiruvudariMda goVmAMsavu ililx BoVjayxveMdu athaRvAdiVtu. adu parxsidadhxvalalxvaSeTx? eMdare {\rm --}}

\begin{shl}
parxsidadhxyXsaMBavAtatxvXdayx huyxkatxM mAMswdanaM parxti | \\
mAMsaM kaqSaNxmaqgacACxgaviSayaM tavxdayx kuvaRteV \hfill|| 78 || 
\end{shl}

\begin{artha}
loVkaparxsididhxyilalxvAdadxriMda mAMsawdana eMbuvalilx mAMsaveMdare 
kaqSaNxmaqga, athavA ADu ivugaLa mAMsaveMdeV garxhisabeVku. adanenx 
Iga mADuvaru.
\end{artha}

\vishaya{hAgAdare I mAMsakAkxgi kaqSaNxmaqga ADu ivugaLanunx 
vadhisabeVkAguvudu. idu yukatxve? eMdare {\rm --}}

\begin{shl}
yatenxVnoVpAjiRtaM tacecxVtikxrXVtAvx vA mAMsamAhareVtf | \\
hiMsAyAH parxtiSidadhxtAvxtapxshUnahxnAyxnanx tu savxyamf \hfill|| 79 || 
\end{shl}

\begin{artha}
parxyatanxdiMda saMpAdisidare Agabahudu. athavA koMDukoMDAdarU 
mAMsavanunx tarabeVku. hiMseyeV niSidadhxvAgiruvudariMda savxyaM 
pashugaLanunx kolalxbAradu.
\end{artha}

\vishaya{`atha ya iceCxVtf' eMdu aneVkasala maMtarxgaLalilx atha eMdu parxyoVgisuvudara udedxVshaveVnu?}

\begin{shl}
athashabodxV vikalApxthoVR yathoVkAtxnAM yathAruci | \\
kAmAyxnAM hayxnayxtaramf pakaSxmAshirxtayx BaNayxteV \hfill|| 80 || 
\end{shl}

\begin{artha}
atha shabadxvu vikalApxthaRdalilxde. hiMde heVLida kAmayxkamaRgaLalilx iSaTxbaMdaMte yAvudAdaroMdu pakaSxvanunx avalaMbisi 
BoVjana niyamavanunx heVLiruvudu.
\end{artha}

\vishaya{hiMdina dina virxVhigaLa avaGAta saMsAkxra, mArane dina 
iSaTxvAda ananxvanunx BuMjisuvudeMdu vayxvasethxyilalxveMdu 
boVdhisuvaru {\rm --}}

\begin{shl}
AraBoyxVdagxmaneV BAnoVH savaRM sAnxnAdayxsheVSataH | \\
nivaRtayxR saMsAkxramatha yatAnxtApxrXgayx udAhaqtaH \hfill|| 81 || 
\end{shl}

\begin{artha}
sUyoRVdayavAdoDane alilxMda AraMBisi sAnxnAdi sakalakamaRgaLanunx 
Acarisi hiMde heVLida avaGAta saMsAkxravanunx yatanxdiMda mADi 
(iSaTxvAda ananxvanunx BuMjisabeVku)
\end{artha}

\vishaya{17, 18 kaMDikegaLa oTuTx tAtapxyaR {\rm --}}

\begin{artha}
yAvanu nanage paMDitaLAgiyU pUNARyuSayxdiMda kUDiyU iruva magaLu 
huTaTxleMdu iciCxsuvano avanu eLaLxnanxvanunx mADi tupapxvanunx 
seVrisi BuMjisabeVku. athavA vidavxtf saBege hoVgalu samathaRnU 
susaMsakxqqtavAda athaRvatAtxda manoVharavAda mAtanADuvavanU 
parxsidadhxnU Agiruva veVdacatuSaTxyavanunx tiLidu pUNARyuvAda maganu 
tanage huTaTxleMdu bayasuvano avanu hiMde heVLida mAMsa misharxvAda 
ananxvanunx BuMjisabeVku. AvAga I daMpatigaLu iMtaha makakxLanunx 
paDeyalu samathaRrAguvaru.
\end{artha}

\vishaya{baq. a.6, bArx. 4, kaMDike 19}

\begin{shl}
athABipArxtareVva sAthxliVpAkAvaqtAjayxM ceVSiTxtAvx sAthxliVpAkasoyxVpaGAtaM juhoVtayxganxyeV sAvxhAnumatayeV sAvxhA deVvAya saviterxV satayxparxsavAya sAvxheVti.
\end{shl}

\vishaya{idaralilx `sAthxliVpAkAvaqtA' eMbudara athaR {\rm --}}

\begin{shl}
sAthxliVpAkeV kirxyA yA sA cA\s \s vaqditayxBidhiVyateV | \\
sAthxliVpAkavidhAneVna saMsakxqqtAyx\s \s jayxM tatheYva tu \hfill|| 82 || 
\end{shl}

\begin{artha}
sAthxliVpAkadalilx yAva kirxyeyu ideyo adu Avaqtf eMdu 
heVLalapxDuvudu. sAthxliVpAkada vidhiyaMte Ajayxvanunx saMsakxrisi 
hAgeye hoVma mADabeVku.
\end{artha}

\begin{shl}
upalakaSxNamaneyxVSAmAjayxsayx garxhaNaM BaveVtf | \\
Adishabadxsayx vA loVpAdAjAyxdimiti gamayxtAmf \hfill|| 83 || 
\end{shl}

\begin{artha}
AjayxveMdu heVLidudx beVre caru modalAdavugaLigU upalakaSxNa. (sUcaka) 
athavA Adishabadxkekx loVpaviruvudariMda AjAyxdi eMdeV tiLiyabeVku.
\end{artha}

\vishaya{sAthxliVpAkasayx itAyxdi maMtarxda athaR}

\begin{shl}
upahateyxVpahatAyxtha sAthxliVpAkasayx manatxrXtaH | \\
nitAyxsatxtArx\s \s hutiVhuRtAvx AvApasAthxna AdarAtf \hfill|| 84 || 
\end{shl}

\begin{artha}
AjAyxdidarxvayxsaMsAkxravanunx mADida naMtara `aganxyeV sAvxhA' 
itAyxdi sAthxliVpAkada maMtarxdiMda matetx matetx AvApa mADi AvApa 
mADida sAthxnadalilx nitayxvAda AhutigaLanunx AdaradiMda hoVmavanunx 
mADi (caru hoVmavanunx mADabeVku).
\end{artha}

\vishaya{kataRvayxvAda Ahuti saMKeyxyanunx heVLutAtxre {\rm --}}

\begin{shl}
parxdhAnAhutayasitxsorxV yAH suyxragAnxyXdipUviRkAH | \\
hutAvx sivxSaTxkaqdanatxM tatasxmApayayx yathoVditamf \hfill|| 85 || 
\end{shl}

\begin{shl}
kameVRdaM tata udadhxqqtayx caruM sAthxlAyxH samAhitaH | \\
sapiRSamxnatxmathAshinxVyAtAkxmitAthARnuroVdhataH \hfill|| 86 || 
\end{shl}

\begin{artha}
aginx muMtAda deVvategaLanunx modalu mADi koDuva yAva AhutigaLu iveyo, 
avu mUru. sivxSaTxkaqdodhxVmapayaRMta mADi AkamaRvanunx mADi 
adariMda sAthxliyiMda caruvanunx meVlakekx tegedukoMDu samAdhAna 
citatxvuLaLxvanAgi iSATxthaRvanunx anusarisi tupapxdiMda adanunx 
saMsakxrisikoMDu BuMjisabeVku.
\end{artha}

\vishaya{pArxshayx itAyxdi maMtArxthaR}

\begin{shl}
caruM pArxshayx savxyaM sheVSaM BAyARyeY saMparxyacaCxti | \\
uciCxSaTxmeVva BAyARyeY caruM BatAR parxyacaCxti \hfill|| 87 || 
\end{shl}

\begin{artha}
hoVma mADi uLida caruvanunx tAnu BuMjisi uciCxSaTxvAda caruvanunx 
patinxge patiyu koDabeVku.
\end{artha}

\vishaya{baq. a.6, bArx. 4, kaMDike 19}

\begin{shl}
hutovxVdadhxqqtayx pArxshAnxti pArxsheyxVtarasAyxH parxyacaCxti parxkASxlayx pANiV udapAtarxM pUrayitAvx teVneYnAM tirxraBuyxkaSxtuyxtitxSAThxtoV vishAvxvasoV\s nAyxmicaCx parxpUvAyxRM saM jAyAM patAyx saheVti || 19 ||
\end{shl}

\begin{shl}
pANiV parxkASxlayx yatenxVna sAmathAyxRdeVva gamayxteV | \\
sAmxtaRmAcamanaM shudedhxyXY pANiparxkASxlanoVkitxtaH \hfill|| 88 || 
\end{shl}

\begin{artha}
parxyatanxdiMda eraDu keYgaLanunx toLedukoMDu shudidhxgAgi 
sAmxtARcamanavanunx mADabeVkeMdu tiLiyuvudu, heVge? pANigaLanunx 
parxkASxLana mADabeVkeMdu heVLidadxriMdaleV athaRsAmathaRyxdiMdaleV
shudidhxgAgi Acamana mADabeVkeMbudu tiLiyuvudu.
\end{artha}

\vishaya{`udapAtarxM pUrayitAvx' itAyxdi maMtarxda athaR {\rm --}}

\begin{shl}
udapAtarxmathA\s \s dAya tadadiBxsitxrXH sutApatxyeV | \\
vakaSxyXmANeVna manetxrXVNa jAyAmaBuyxkaSxyeVnumxhu \hfill|| 89 || 
\end{shl}

\begin{artha}
jalapAterxyanunx anaMtara tegedukoMDu adaralilx niVriniMda heMDatige muMde heVLuva maMtarxdiMda saMtAna lABakAkxgi matetxmatetx seVcisabeVku.
\end{artha}

\vishaya{A maMtarx `utitxSAThxtoV vishAvxvasoV......' itAyxdiyAgide. adara athaR {\rm --}}

\begin{shl}
atoV\s samxdiVyadAreVBayx utAthxyAnayxta Avarxja | \\
vishAvxvasavxBidhAneVna ganadhxvoVR\s tarx parxboVdhayxteV \hfill|| 90 || 
\end{shl}

\begin{shl}
parxpUvAyxRmiti nAyaRtarx BaNayxteV taruNiV kila | \\
parxpUviVR piVvariVmanAyxM yAhi vishAvxvasoV durxtamf \hfill|| 91 || 
\end{shl}

\begin{shl}
parxpUvAyxRmiti liknAgxcacx taruNAyxM satapxtiH sadA | \\
yathoVditaM kamaR satAyxM kuyARtasxtupxtarxjanamxneV \hfill|| 92 || 
\end{shl}

\begin{shl}
ahaM tu sAvxmimAM jAyAM samupeYmiVti saMgatiH | \\
EvaM parxsAthxpayx ganadhxvaRmatheYnAmaBipadayxteV \hfill|| 93 || 
\end{shl}

\begin{artha}
vishAvxvasu eMba hesariniMda ililx gaMdhavaRnanunx saMboVdhiside. eleY 
gaMdhavaR, vishAvxvasuH namamx heMDatiyanunx biTuTx meVlakekx edudx 
beVre kaDege hoVgu.
\end{artha}

\begin{artha}
parxpUvARyxmf eMbuvalilx taruNiyAda nAriyu heVLalapxTiTxde. eleY 
vishAvxvasuve? patiyoDane kirxVDisuva parxpUviRVM = puSaTxLAda beVre 
sitxrXVya hatitxra shiVGarx hoVgu. hiMde heVLida kamaRvanunx I 
satiyalilx satupxtarxna utapxtitxgAgi satapxtiyu avashayx 
AcarisabeVku. parxpUvARyxmf eMbuva shabadxsAmathaRyxdiMda aMdare 
(tAruNayxvanunx sUcisuva shabadx baladiMda) taruNiyAda satiyalilx 
QutukAlavu baMdAgalelAlx AcarisabeVku.
\end{artha}

\begin{artha}
nAnAdaro I nananx heMDatiyanunx seVrabeVkeMdiruvenu. adakekx avakAsha 
mADikoDu eMdu gaMdhavaRnanunx horage horaDisi Ikeyanunx hoMduvanu.
\end{artha}

\begin{shl}
aBiVSaTxgaBARdhAnAya jAyAmAliknagxteV patiH | \\
amoV\s hamitimanotxrXVkAtxyX tAvAvAM deVvatAtamxkw \hfill|| 94 || 
\end{shl}

\begin{artha}
iSaTxvAda gaBARdhAnakAkxgi patiyu heMDatiyanunx AlaMgisuvanu, heVge 
eMdare `amoVhamasimxsAtarxyXmf' eMba maMtarxvanunx heVLutAtx nAvu 
ibabxrU deVvatA savxrUpavAgidedxVve. BAvisutAtx AlaMgisuvanu.
\end{artha}

\vishaya{baq. a.6, bArx. 4, kaMDike 20}

\begin{shl}
atheYnAmaBipadayxteV\s moV\s hamasimx sA tavxM sA tavxmasayxmoV\s haM sAmAhamasimx QukatxvXM dwyxrahaM paqthiviV tavxM tAveVhi saMraBAvaheY saha reVtoV dadhAvaheY puMseV putArxya vitatxya iti || 20 ||
\end{shl}

\vishaya{atheYnAmaBipadayxteV - amoV\s ha masimx eMba maMtarxda athaR}

\begin{artha}
I patinxyanunx aBimaMtirxsi kiSxVrAnanx modalAda ananxvanunx AyAya 
putarxnanunx kAmaneyaMte BuMjisi patiyu AlaMgisuvudu. patiyu seVruva 
kAladalilx nAnu patipArxNavAgidedxVne. A niVnu vAkAkxgididx, heVge? 
vAkukx pArxNAdhiVnavAgidadxriMda niVneV vAkukx, nAneV 
pArxNavAgidedxVne. alalxde nAnu sAmaveVdavAgidedxVne. niVneV 
QukAkxgididxVye, nAnu aMtarikaSx loVka, niVnu paqthiviVloVka. A 
nAvibabxrU saha I namamx udayxmavanunx AraMBisoVNa reVtasasxnunx nAvu 
seVrisoVNa. EtakAkxgi satfputarxna lABakAkxgi eMdu aBipArxya.
\end{artha}

\begin{shl}
kamAR\s \s raBAvaheY deVvi satusxtoVtapxtitxsidadhxyeV | \\
tavxM cAhaM ceYva saMBUya yoVnw reVtoV dadhAvaheY \hfill|| 95 || 
\end{shl}
	
\begin{artha}
eleY deVviye? nAvibabxrU I udayxmavanunx mADoVNa. satupxtarxna 
janamxvu sididhxsalu niVnU matutx nAnU seVri yoVniyalilx reVtasasxnunx 
dharisuva.
\end{artha}

\begin{shl}
reVtaHkeSxVpaPalaM cA\s \s ha puMseV putArxya labadhxyeV | \\
manotxrXVkatxyXnanatxraM tasAyx vijihiVthAmitiVrayeVtf \hfill|| 96 || 
\end{shl}

\begin{artha}
reVtasasxnunx seVcane mADida Palavanunx shurxtiye gaMDumagananunx 
hoMduvudakAkxgi eMdu heVLide. `puMseputArxya vitatxyeV' eMbalilx 
alalxde `amoV\s ha masimx' eMba maMtorxVcAcxraNeyAda naMtara 
BAyeRyanunx udedxVshisi `vicihiVthAMdAyxnA paqthiviV' eMbudanunx 
heVLabeVku.
\end{artha}

\vishaya{baq. a.6, bArx. 4, kaMDike 21}

\begin{shl}
athAsAyx UrU vihApayati vijihiVthAM dAyxvApaqthiviV iti tasAyxmathaRM niSAThxya muKeVna muKaM sanAdhxya tirxreVnAmanuloVmAmanumASiTxR viSuNxyoVRniM kalapxyatu tavxSATx rUpANi piMshatu || Asicnacxtu parxjApatidhARtA gaBaRM dadhAtu teV || gaBaRM dheVhi siniVvAli gaBaRM dheVhi paqthuSuTxkeV || gaBaRM teV ashivxnw deVvAvAdhatAtxM puSakxrasarxjw || 21 ||
\end{shl}

\vishaya{I maMtarxda vAyxKAyxna AraMBavAgide {\rm --}}

\begin{shl}
vihApayati manetxrXVNa UrU patAnxyXH parxyatanxtaH | \\
UvoVRrAmanatxrXNaM ceYtadivxjihiVthAmitiVkaSxyXtAmf \hfill|| 97 || 
\end{shl}

\begin{artha}
`vijihiVthAmf' itAyxdi maMtarxdiMda patinxya toDegaLanunx biDisuvudu. 
toDegaLanunx udedxVshisi AmaMtirxsideyeMdu `vijihiVthA' eMbuvalilx 
tiLiyabeVkAdadudx.
\end{artha}

\vishaya{`vihApayati' eMbudara vuyxtapxtitx, matutx athaR}

\begin{shl}
vijihiVteVridaM rUpaM Nayxnatxsayx gatikamaRNaH \hfill|| 98 | \\
manatxrXtaH pANinA\s theYnAM tirxVnAvxrAnanuloVmataH | \\
anumASaTxRyXtha tAM jAyAM manatxrXM viSuNxritiVrayanf \hfill|| 99 || 
\end{shl}

\begin{artha}
`vihApayati' eMbudu vi upasagaRviruva ja{hA}ti eMba Nijf parxtayxyAMta, 
gatayxthaRda dhAtuvina rUpavu. mUru Pala anuloVmavAgi (taleyiMda 
kAlinavarege) heMDatiyanunx keYyiMda maMtarxdiMda anumAjaRne 
mADabeVku. maMtarxveVneMdare? `viSuNxyoRVniM kalapxyatu' eMdu 
ucacxrisutAtx anumAjaRne mADabeVku.
\end{artha}

\vishaya{`viSuNxyoRVniM kalapxyatu' eMbalilx kalapxyatu eMbudara athaR 
{\rm --}}

\begin{shl}
samathaRnaM kalapxnAthaRsatxvXSATx\s vayavashasatxthA | \\
nivaRtaRyatu rUpANi shoVBanAni sutasayx meV \hfill|| 100 || 
\end{shl}

\begin{artha}
kalapxneyeMdare samathaRne shakitxyuMTu mADuvudu eMdathaR. 
tavxSaTxqqbarxhamxnu nananx magana rUpagaLu. parxtiyoMdu avayavagaLU 
aMdavAgiruvaMte saqSiTxsali.
\end{artha}

\vishaya{`gaBaRMdheVhi siniVvAliV' eMbuvalilx siniVvAliV padada 
athaRvanunx heVLuvaru {\rm --}}

\begin{shl}
dashARhadeVRvatA ceVha siniVvAliVti BaNayxteV | \\
paqthuSuTxkeVti swvoVkAtx paqthusutxtirasw yataH \hfill|| 101 || 
\end{shl}

\begin{artha}
amAvAseyxya ahasisxna deVvateyeV siniVvAli eMdu heVLalapxDuvudu. 
paqthuSuTxke eMbudU adeV deVvate. kAraNaveVneMdare? hecucx 
sutxtiyuLaLx deVvate idu, adariMda
\end{artha}

\begin{shl}
gaBaRM teV samayxgAdhatAtxmashivxnw puSakxrasarxjw | \\
sUyARcanadxrXmasAveVva vijecnxVyAvashivxnAviha \hfill|| 102 || 
\end{shl}

\begin{artha}
ninage kamalada hAravanunx hAkikoMDiruva ashivx deVvategaLu 
gaBaRvanunx cenAnxgi uMTumADali. ililx sUyaRcaMdarxreV ashivx 
deVvategaLeMdu tiLiyabeVku.
\end{artha}

\vishaya{parxsidadhxvAda ashivxniV deVvategaLu Eke AgabAradu? eMdare {\rm --}}

\begin{shl}
savxrashimxsarxgivxNw tw hi parxsidwdhx jagatoV yataH \hfill|| 103 | \\
\end{shl}

\begin{artha}
tananx kiraNamAleyuLaLxvarAgi A sUyaRcaMdarxreV jagatitxnalilx 
parxsidadhxveV Agiruvaru. (adariMda sUyaRcaMdarxreV puSakxra sarxjaw 
ashivxnaw eMdu heVLalapxDuvaru)
\end{artha}

\vishaya{baq. a.6, bArx. 4, kaMDike 22}

\begin{shl}
hiraNamxyiV araNiV yABAyxM nimaRnathxtAmashivxnw || taM teV gaBaRM havAmaheV dashameV mAsi sUtayeV || yathAginxgaBAR paqthiviV yathA dwyxrinedxrXVNa gaBiRNiV || vAyudiRshAM yathA gaBaR EvaM gaBaRM dadhAmi teV\s sAviti || 22 ||
\end{shl}
 
\vishaya{hiraNamxyiV araNiV eMbudara athaR {\rm --}}

\begin{shl}
hiraNayxM joyxVtiramaqtaM tanamxyAyxvaraNiV shuBeV | \\
niramanathxtAM yABAyxM tAvashivxnAvamaqtaM purA \hfill|| 104 || 
\end{shl}

\begin{artha}
hiraNayxM eMdare amaqta savxrUpavAda joyxVti. tanamxyavAda araNigaLu 
aMdavAgiratakakxvu. A eraDu araNigaLiMda ashivx deVvategaLu hiMde 
amaqtavanunx mathana mADiruvaru.
\end{artha}

\begin{shl}
ashivxnw yAdaqshaM gaBaRM parxyatAnxninxramanathxtAmf | \\
AdadhAvasatxthA rUpaM dashameV mAsi sUtayeV \hfill|| 105 || 
\end{shl}

\begin{artha}
ashivx deVvategaLu eMtaha gaBaRvanunx parxyatanxdiMda mathana 
mADiruvaru aMtaha rUpavuLaLx gaBaRvanenxV nAvu irisoVNa. EtakAkxgi? 
hatatxne tiMgaLalilx adu janisuvudakAkxgi.
\end{artha}

\vishaya{yathA itAyxdi maMtArxthaR {\rm --}}

\begin{shl}
yathA\s ginxgaBAR paqthiviV dwyxrinedxrXVNeVva BAnunA | \\
vAyudiRshAM yathA gaBoVR daqSaTxH shalxthanakamaRkaqtf \hfill|| 106 || 
\end{shl}

\begin{artha}
paqthiviyu heVge aginxyeV gaBaRvAgivuLaLxdodx, matutx aMtarikaSx 
loVkavu sUyaRneMba iMdarxniMda gaBaRvuLaLxdodx, dikukxgaLige vAyuveV
heVge gaBaRvAgidudx calana kamaRvanunx mADuvudo, hAgeyeV {\rm --}
\end{artha}

\vishaya{asaw eMbudakekx eraDathaRgaLu {\rm --}}

\begin{shl}
AtamxnAma samucAcxyaR tathA gaBaRM dadhAmi teV | \\
tasAyx vA nAma gaqhiNxVyAnamxnatxrXmucAcxrayanapxtiH \hfill|| 107 ||  
\end{shl}

\begin{artha}
tananx hesaranunx ucacxrisi ninage hAgeyeV gaBaRvanunx irisuvenu eMdu 
`gaBaRM dadhAmi teV' eMdu maMtarxvanunx ucacxrisabeVku, athavA A 
maMtarxvanunx patiyu ucacxrisutAtx heMDatiya hesaranAnxdarU 
garxhisabeVku.
\end{artha}

\vishaya{baq. a.6, bArx. 4, kaMDike 23}

\begin{shl}
soVSayxnitxVmadiBxraBuyxkaSxti ||yathA vAyuH puSakxriNiVM samiknagxyati savaRtaH ||EvA teV gaBaR Ejatu sahAveYtu jarAyuNA ||inadxrXsAyxyaM varxjaH kaqtaH sAgaRlaH saparisharxyaH ||taminadxrX nijaRhi gaBeVRNa sAvarAM saheVti || 23 ||
\end{shl}

\vishaya{I maMtarxda athaR {\rm --}}

\begin{shl}
manetxrXVNAtheYSa soVSayxnitxVmadiBxraBuyxkaSxyeVcaCxneYH | \\
vAyuH puSakxriNiVM yadavxtasxmacnajxyati cAlayeVtf \hfill|| 108 || 
\end{shl}

\begin{artha}
hatutx mAsagaLu kaLeda naMtara parxsavisuva BAyeRyanunx `yathA vAyuH' 
itAyxdi maMtarxdiMda melalxne niVriniMda porxVkiSxsabeVku. vAyuvu 
koLavanunx calisuvaMte patiyu calisuvaMte mADuvanu. (gaBaRkekx 
upahatiyilalxdeyiruvaMte ADisuvanu)
\end{artha}

\vishaya{ideV daqSATxMtadalilx beVkAda aMshavanunx I muMde tiLisutAtxre {\rm --}}

\begin{shl}
yathA puSakxriNiVM vAyushAcxlayananxpi savaRtaH | \\
na karoVti kaSxtiM tadavxdagxBaR Ejatu teV suKamf \hfill|| 109 || 
\end{shl}

\begin{artha}
heVge gALiyu koLavanunx elelxDe ADisutAtx idadxrU hAniyanunxMTu 
mADuvudilalxvo hAgeye BAyeRge hAniyanunxMTu mADuvudilalx. ninanx 
gaBaRvu suKavAgi calisali.
\end{artha}

\begin{shl}
gaBaRsAtxvXM suKayanenxVtu sahoVlebxVNa jarAyuNA | \\
varxjoV mAgoVR hi gaBaRsayx sAgaRloV\s yaM kaqtaH purA \hfill|| 110 || 
\end{shl}

\begin{artha}
ninanxnunx suKapaDisutAtx gaBaRciVladoMdige gaBaRvu (shishuvu) horage 
barali varxja eMdare gaBaRda mAgaR. adu hiMde saqSiTxkAladalilx agaRla 
= agaLiyiMda (paDeyiMda) kUDiyeV saqSiTxsalapxTiTxde.
\end{artha}

\vishaya{agaRla matutx avarA eMbudanunx vAyxKAyxnisuvudu {\rm --}}

\begin{shl}
jarAyuragaRlasatxM tavxminadxrX nijaRhi satavxraH | \\
gaBeVR viniHsaqteV pashAcxdAyx nigaRcaCxti yoVnitaH \hfill|| 111 || 
\end{shl}

\begin{shl}
mAMsapeVshiV samA tanivxV sA\s vareVti nigadayxteV | \\
pArxNashecxVnodxrXV\s tarx vijecnxVyaH sa Eva pArxthayxRteV tataH \hfill|| 112 || 
\end{shl}

\begin{artha}
gaBaRda ciVlaveV (jarAyu) taDeyAguvudu, eleY iMdarx? niVnu shiVGarxvAgi 
agaRlavanunx oDedu hAku. (dAri mADikoDu), gaBaRvu horage horaTa meVle 
anaMtara yoVniyiMda yAvudu horage baruvudo, A mAMsada mudedxyu 
gaBaRkekx samAnavAgidudx avarA eMdu heVLalapxDuvudu. (adanunx horage 
horaDisu) ililx iMdarx eMdare pArxNavAyu, adanenxV patiyu 
pArxthiRsuvudu.

(\textbf{soVSayxnitxVmitAyxdi kaMDikeya tAtapxyARthaR-} patiyu 
parxsavisuva heMDatiyanunx niVriniMda parxsavakAladalilx 
suKaparxsavavAgalu I muMde heVLida maMtarxdiMda porxVkiSxsabeVku 
(`yathA vAyuH' itAyxdi maMtarxdiMda) heVge vAyuvu koLavanunx 
nAshapaDisade ADisuvudo hAgeye ninanx gaBaRvu ninage toMdare mADade 
calisali. gaBaRda ciVladiMdoDagUDi gaBaRvu Icege barali, pArxNa eMba 
iMdarxna mAgaRvu saqSiTxkAladalelxV jarAyuveMbuva ciVladiMda 
sututxvareyalapxTiTxruvudu. eleY iMdarx? niVnu A taDeyanunx BeVdisi A 
mAgaRdiMda gaBaRdoMdige horage bA. gaBaRvu horage baMda meVle baruva 
mAMsa piMDavanunx horage horaDisu. hiVge patiyu pArxthiRsabeVku.)
\end{artha}

\vishaya{baq. a.6, bArx. 4, kaMDike 24}

\begin{shl}
jAteV\s ginxmupasamAdhAyAknakx AdhAya kaMseV paqSadAjayxM saninxVya paqSadAjayxsoyxVpaGAtaM juhoVtayxsimxnasxhasarxM puSAyxsameVdhamAnaH sevxV gaqheV || asoyxVpasanAdxyXM mA ceCxYtisxVtapxrXjayA ca pashuBishacx sAvxhA || mayi pArxNAMsatxvXyi manasA juhoVmi sAvxhA || yatakxmaRNAtayxriVricaM yadAvx nUyxnamihAkaramf || aginxSaTxtisxvXSaTxkaqdivxdAvxnisxvXSaTxM suhutaM karoVtu naH sAvxheVti || 24 ||
\end{shl}

\vishaya{I maMtarxda athaR {\rm --} (putarx jananavAda meVle jAtakamaRvanunx 
vidhisutAtxre)}

\begin{shl}
jAteV kumAreV\s tha patiraknakxmAroVpayx taM pitA | \\
aginxM huyxpasamAdhAya parxsidAdhxthaRmathAparamf \hfill|| 113 || 
\end{shl}

\begin{artha}
kumAranu huTiTxda meVle anaMtara patiyu A kumArananunx tananx toDeya 
meVle hatitxsikoMDu AvasathayxveMba aginxyanunx sAthxpisikoMDu hiMde 
vAyxKAyxnadiMda parxsidadhxvAda athaRvuLaLx `kaMseVpaqSadAjayxmf' eMdu 
heVLida darxvayxvanunx hoVma mADabeVku.
\end{artha}

\vishaya{`paqSadAjayxmf' eMdare {\rm --}}

\begin{shl}
GaqtaM dadhi ca saMmisharxM paqSadAjayxmitiVyaRteV | \\
EkiVkaqtAyxtha saMniVya paqSadAjayxsayx manatxrXtaH \hfill|| 114 || 
\end{shl}

\begin{shl}
upadhAtaM juhoVtiVti NamulAvx\s \s BiVkaSxNXyX iSayxteV | \\
upahatoyxVpahateyxVti paqSadAjAyxhutiVH kiSxpeVtf \hfill|| 115 || 
\end{shl}

\begin{artha}
tupapx matutx mosaru eraDanunx misharx mADidalilx adanunx 
paqSadAjayxveMdu heVLuvudu. adanunx oMdu mADi anaMtara kaMcina 
pAterxyalilx muMde irisikoMDu maMtarxdiMda `paqSadAjayxsoyxVpaGArxtamf 
juhoVti' eMdaMte punaHpunaH savxlapx savxlapx tegedukoMDu misharxvAda AjAyxhutigaLanunx hoVma mADabeVku. ililx `upaGAtamf' eMbuvalilx Namulf 
parxtayxyadiMda punaHpunaH eMdathaRvu laBisuvudu.
\end{artha}

\begin{shl}
puSAyxsaM sevxV gaqheV\s tArxhaM manuSAyxNAM parxkAmataH | \\
sahasarxsaMKayxyA shashavxdabxhoVnARma sahasarxgiVH \hfill|| 116 || 
\end{shl}

\begin{artha}
yatheVcaCxvAgi I nananx gaqhadalilx sahasarxmanuSayxranunx yAvAgalU 
poVSisuvaMte nAnu Aguvenu. sahasarx eMba saMKAyxvAcaka padadiMda 
bahutavxvanunx heVLuvudu. sahasarx padavu bahutavxvanenx boVdhisuvudu.
\end{artha}

\begin{shl}
EvaM tavxM vadhaRmAnoV\s tarx manivxVthAH putarx saMtatimf | \\
saMtatAvupasanAdhxM tAvxM pashAvxdeVmeVR kariSayxsi \hfill|| 117 || 
\end{shl}

\begin{artha}
I riVtiyAgi vaqdidhx hoMdutAtx niVnu putarxsaMtatiyanunx ariyabeVku. 
upasaMdhi eMdare saMtati. idaralilx niVnu pashu muMtAdavugaLa 
saMtatiyanunx uMTumADuvavanAgu.
\end{artha}

\vishaya{`mayi pArxNA \c satxyXyi' eMba maMtarxda vAyxKAyxnavu 
anavashayxveMdu heVLutAtxre {\rm --}}

\begin{shl}
neYva vayxpeVkaSxteV vAyxKAyxM sapxSATxthaRtevxVna heVtunA | \\
mayi pArxNAniti garxnathxH savxyameVvAvagamayxtAmf \hfill|| 118 || 
\end{shl}

\begin{artha}
sapxSATxthaRvAgideyeMba kAraNadiMda ililx vAyxKAyxnavanunx 
apeVkiSxsuvudilalx. `mayipArxNAnf' eMba garxMthavanunx tAnAgiyeV
tiLiyabahudu.

\textbf{24ne kaMDikeyalilxruva maMtarxgaLa athaR}\\
I nananx gaqhadalilx putarxrUpadiMda vaqdidhx hoMdutAtx nAnu sAvirAru 
manuSayxranunx poVSisuvenu. aneVka manuSayxranunx poVSisuvanaMte 
AgabeVku. I nananx putarxnige saMtatiyalilx parxje athavA pashugaLa 
saMpatutx viciCxnanxvAgadirali. taMdeyAda nananxlilx yAva pArxNagaLu 
irutatxveyo, avugaLanunx putarxnAda ninanxlilx manasA apiRsuvenu. 
nAnu kamaRdalilx atireVkavanonx athavA nUyxnateyanonx mADiruvudu 
idadxre adelalxvanunx tiLida sAkiSxBUtanAda yajecnxVshavxranu cenAnxgi 
yAga mADidaMteye aMdare adhikavAgi mADiradaMteyU nUyxnavAgiradeyU 
samavAgiruvaMte mADali.
\end{artha}

\vishaya{baq. a.6, bArx. 4, kaMDike 25}

\begin{shl}
athAsayx dakiSxNaM kaNaRmaBinidhAya vAgAvxgiti tirxratha dadhi madhu GaqtaM saninxVyAnanatxhiRteVna jAtarUpeVNa pArxshayati || BUsetxV dadhAmi BuvasetxV dadhAmi savxsetxV dadhAmi BUBuRvaHsavxH savaRM tavxyi dadhAmiVti || 25 ||
\end{shl}

\begin{shl}
athAsayx savxmuKeV kaNaRM dakiSxNaM parxNidhAya tu | \\
vAgAvxgiti hi tirxbUrxRyAtarxrXyiV vAgiti BaNayxteV \hfill|| 119 || 
\end{shl}

\begin{shl}
tarxyiV vAkAtxvXM shorxVtarxmAgeVRNa sherxVyaseV parxvishativxyamf | \\
yasetxV satxna iti girA shasayxteV\s tarx sarasavxtiV \hfill|| 120 || 
\end{shl}

\begin{shl}
udAraguNasaMpatitxH sutasAyxsitxvXti BaNayxteV | \\
shashayaH sashayoV jecnxVyaH shayashacx PalamucayxteV \hfill|| 121 || 
\end{shl}

\begin{artha}
anaMtara I kumArana balagiviyanunx tananx bAyiya hatitxra irisikoMDu 
vAkf vAkf vAkf eMdu mUrusala heVLabeVku. vAkf eMdare tarxyiV aMdare 
mUru veVdagaLeMdu heVLalapxDuvudu. I tarxyiVrUpavAda vAkukx kiviya 
mUlaka oLage maguvina sherxVyasisxgAgi parxveVshisali. `yasetx satxnaH' 
- eMba maMtarxdiMda ililx sarasavxtiVdeVviyu 
hogaLalapxDuvaLu\footnote[1]{uLida maMtArxthaR - anaMtara mosaru, 
jeVnutupapx, tupapx I mUranUnx seVrisi hiraNayxdiMda maguvige 
pArxshana mADisabeVku. adakekx I nAlukx maMtarxvanunx heVLabeVku - (1) 
BUH setxVdadAmi, (2) BuvasetxV dadhAmi, (3) savxsetxVdadhAmi, (4) 
BUBuRvaHsavxH savaRM tavxyidadhAmi eMdu koneyalilx heVLi pArxshana 
mADisabeVku. ililx 120ne vAtiRkadiMda Acege naDuve 
vAtiRkavididxrabeVku. 120kekx maMtArxthaRvu pUNaRvAguvudilalxveMbude I 
riVti nAvu Uhisalu kAraNa.} - (putarxnige satxnayxpAnada mUlaka) 
udAravAda guNasaMpatutx uMTAgaleMdu heVLide. maMtarxdalilx shashayaH 
eMdare sashaya eMdu tiLiyabeVku. shaya eMdare kamaRda PalaveMdu 
heVLalapxDuvudu.
\end{artha}

\vishaya{baq. a.6, bArx. 4, kaMDike 26, 27}

\begin{shl}
athAsayx nAma karoVti veVdoV\s siVti tadasayx tadugxhayxmeVva nAma Bavati || 26 ||
\end{shl}

\begin{shl}
atheYnaM mAterxV parxdAya satxnaM parxyacaCxti yasetxV satxnaH shashayoV yoV mayoVBUyoVR ratanxdhA vasuvidayxH sudatarxH ||yeVna vishAvx puSayxsi vAyARNi sarasavxti tamiha dhAtaveV kariti || 27 ||
\end{shl}

\begin{shl}
guhA shayoV vA shashayaH shurxteyxYva parxtipAditaH | \\
mayoVBUrananxBUtoV\s yaM savaRpArxNiBaqducayxteV \hfill|| 122 || 
\end{shl}

\begin{artha}
satxnavu meVlakekx baruva edeya BAgaveV guhA. adanenxV shaya eMdu 
shurxtiyu parxtipAdiside. mayoVBUH eMdare savaRpArxNigaLanunx 
poVSisuva ananxrUpavAda vasutxveMdu heVLalapxDuvudu.
\end{artha}

\vishaya{yoVratanxdhA itAyxdi maMtArxthaR}

\begin{shl}
ratanxsayx payasoV\s tayxthaRM ratAnxdhArashacx yaH satxnaH | \\
vasunoV dhanasayx labAdhx ca tasayx vaqSATxyXdiheVtunaH \hfill|| 123 || 
\end{shl}

\begin{shl}
BUrikalAyxNadAtaqtAvxtusxdatarx iti BaNayxteV | \\
yeVna puSayxsi vAyARNi savARNiVha sarasavxti \hfill|| 124 || 
\end{shl}

\begin{artha}
ratanxdhA = eMdare sherxVSaThxvAda hAlanunx utApxdane mADuvudu matutx hAlige AdhAravU deVviya satxnavu AdhAravAgiruvudo matutx vasutxveMba dhanavanunx vaqSiTx muMtAdavugaLiMda huTuTxvudanunx hoMdiyU iruvudo, alalxde hecucx maMgaLakaravAgiruvudariMda sudatarxveMdU heVLalapxDuvudo, matutx yAva satxnadiMda nAvu beVDuva deVvAdi pArxNigaLanunx elalxvanunx niVnu poVSisuveyo eleY sarasavxtiV deVviye? nananx putarxnige pAnakAkxgi nananx heMDatiya satxnadalilx hiMde heVLida guNavuLaLx samasatx iSATxthaRvanunx koDuva A ninanx satxnavanunx irisu.
\end{artha}

\vishaya{baq. a.6, bArx. 4, kaMDike 28}

\footnotetext[1]{satxnayxpAnavanunx mADisida meVle I meVlina `ilAsi' eMba maMtarxdiMda patiyu aBimaMtirxsabeVku - idara athaR - niVnu sutxtige pAtarxLAgiruve. meYtArxvaruNi vasiSaThxra patinx aruMdhatiyeV Agididx. viVrapuruSanAda nananxnunx nimitatxmADi niVnu putarxnanunx janisiruve. A niVnu bahu putarxvuLaLxvaLAgu. niVneV namamxnunx bahuputarxsaMpananxranAnxgi mADiruve. alalxde putarxnu taMdeyanUnx tAtananUnx miVrisidavanu. AshacxyaRvidu alalxde siriyiMdalU kiVtiRyiMdalU barxhamxvacaRsisxniMdalU paramAvadhiyanunx hoMdiruvanu. yAru I riVtiyAgi putarx saMpatitxyanunx paDeda bArxhamxNanige putarxnAgi huTuTxvano A taMdeyU saha tananx taMde tAta ivaranenxlAlx miVrisiruvanu eMdathaR.}
\begin{shl}
\footnotemark[1]athAsayx mAtaramaBimanatxrXyateV || ilAsi meYtArxvaruNiV viVreV viVramajiVjanatf || sA tavxM viVravatiV Bava yAsAmxnivxVravatoV\s karaditi || taM vA EtamAhuratipitA batABUratipitAmahoV batABUH paramAM bata kASAThxM pArxpaciCxrXyA yashasA barxhamxvacaRseVna ya EvaMvidoV bArxhamxNasayx putorxV jAyata iti || 28 ||
\end{shl}

\begin{shl}
ilA\s siVtayxtha manetxrXVNa sUnoVmARtaramAdarAtf | \\
aBimanatxrXyateV sAdhukamARvApetxyXY savxyaM patiH \hfill|| 127 || 
\end{shl}

\begin{artha}
patiyu tAnu ilAsi eMba maMtarxdiMda putarxna tAyiyanunx AdaradiMda 
aBimaMtirxsabeVku. EtakAkxgi eMdare utatxmavAda kamaR (parxshasatx 
putarxpAlaneyeMba) kAkxgi aBimaMtirxsabeVku.
\end{artha}

\begin{center}
ililxge baqhadAraNayxkoVpaniSadf BASayx vAtiRkadalilx AraneV 
adhAyxyadalilx nAlakxneV bArxhamxNavu pUNaRgoMDide.\\
|| shirxVdakiSxNAmUtaRyeV namaH ||\\
|| namaH shirxVshaMkaraguraveV namaH ||
\end{center}
